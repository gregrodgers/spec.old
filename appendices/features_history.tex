% This is features_history.tex (Appendix E) of the OpenMP specification.
% This is an included file. See the master file for more information.
%
% When editing this file:
%
%    1. To change formatting, appearance, or style, please edit openmp.sty.
%
%    2. Custom commands and macros are defined in openmp.sty.
%
%    3. Be kind to other editors -- keep a consistent style by copying-and-pasting to
%       create new content.
%
%    4. We use semantic markup, e.g. (see openmp.sty for a full list):
%         \code{}     % for bold monospace keywords, code, operators, etc.
%         \plc{}      % for italic placeholder names, grammar, etc.
%
%    5. There are environments that provide special formatting, e.g. language bars.
%       Please use them whereever appropriate.  Examples are:
%
%         \begin{fortranspecific}
%         This is text that appears enclosed in blue language bars for Fortran.
%         \end{fortranspecific}
%
%         \begin{note}
%         This is a note.  The "Note -- " header appears automatically.
%         \end{note}
%
%    6. Other recommendations:
%         Use the convenience macros defined in openmp.sty for the minor headers
%         such as Comments, Syntax, etc.
%
%         To keep items together on the same page, prefer the use of
%         \begin{samepage}.... Avoid \parbox for text blocks as it interrupts line numbering.
%         When possible, avoid \filbreak, \pagebreak, \newpage, \clearpage unless that's
%         what you mean. Use \needspace{} cautiously for troublesome paragraphs.
%
%         Avoid absolute lengths and measures in this file; use relative units when possible.
%         Vertical space can be relative to \baselineskip or ex units. Horizontal space
%         can be relative to \linewidth or em units.
%
%         Prefer \emph{} to italicize terminology, e.g.:
%             This is a \emph{definition}, not a placeholder.
%             This is a \plc{var-name}.%


\chapter{Features History}
\index{features history}
\index{history of features}
\label{chap:Features History}
This appendix summarizes the major changes between recent versions of the OpenMP
API since version 2.5.

\section{Deprecated Features}
\index{deprecated features}
\label{chap:Deprecated Features}

The following features have been deprecated:

\begin{itemize}
\item the \plc{nest-var} ICV;
\item the \code{OMP_NESTED} environment variable;
\item the \code{omp_set_nested} and \code{omp_get_nested} routines;
\item the C/C++ type \code{omp_lock_hint_t} and the corresponding Fortran kind \code{omp_hint_hint_kind};
\item and the lock hint constants \code{omp_lock_hint_none}, \code{omp_lock_hint_uncontended}, \code{omp_lock_hint_contended}, \code{omp_lock_hint_nonspeculative}, and \code{omp_lock_hint_speculative}.
\end{itemize}



\section{Version 4.5 to 5.0 Differences}
\label{sec:Version 4.5 to 5.0 Differences}
\begin{itemize}
\item Stubs for Runtime Library Routines(previously Appendix A) were moved to a separate document.
\item Interface Declarations (previously Appendix B) were moved to a separate document.

\item The memory model was extended to distinguish different types of flush
      operations according to specified flush properties (see
      \specref{subsec:The Flush Operation}) and to define a happens
      before order based on synchronizing flush operations
      (see \specref{subsec:happens-before}).

\item Various changes throughout the specification were made to provide
      initial support of C11, C++11, C++14 and Fortran 2008 (see
      \specref{sec:normative references}).

\item Support for several features of Fortran 2003 was added (see
      \specref{sec:normative references} for features that are still
      not supported).

\item The list items allowable in a \code{depend} clause on a task generating
      construct was extended, including for C/C++ allowing any \plc{lvalue}
      expression (see \specref{sec:Directive Format} and
      \specref{subsec:depend Clause}).

\item The \code{requires} directive (see \specref{sec:requires Directive}) was
      added to support applications that require implementation-specific
      features generally and shared memory across devices specifically.

\item The \plc{target-offload-var} internal control variable (see
      \specref{sec:Internal Control Variables}) and the
      \code{OMP_TARGET_OFFLOAD} environment variable (see
      \specref{sec:OMP_TARGET_OFFLOAD}) were added to support runtime
      control of the execution of device constructs.

\item The default value of the \plc{nest-var} ICV was changed from \plc{false}
      to implementation defined (see \specref{subsec:ICV Initialization}).
      The \plc{nest-var} ICV (see \specref{subsec:ICV Descriptions}), the
      \code{OMP_NESTED} environment variable (see \specref{sec:OMP_NESTED}),
      and the \code{omp_set_nested} and \code{omp_get_nested} routines
      were deprecated (see \specref{subsec:omp_set_nested} and
      \specref{subsec:omp_get_nested}).

\item Support for array shaping (see \specref{sec:Array Shaping}) and 
      for array sections with non-unit strides  in C and C++ (see 
      \specref{sec:Array Sections}) were added to facilitate specification 
      of discontiguous storage and the \code{target update} construct (see 
      \specref{subsec:target update Construct}) and the \code{depend} clause 
      (see \specref{subsec:depend Clause}) were extended to allow the use 
      of shape-operators (see \specref{sec:Array Shaping}); further, the 
      \code{target update} construct (see \specref{subsec:target update 
      Construct}) was modified to allow array sections that specify 
      discontiguous storage.

\item Iterators (see \specref{sec:iterators}) were added to express that an
      expression in a list may expand to multiple expressions.

\item The canonical loop form was defined for Fortran and, for all base
    languages, extended to permit non-rectangular loop nests (see
    \specref{sec:Canonical Loop Form}).

\item The \plc{relational-op} in the canonical loop form for C/C++ was
      extended to include != (see \specref{sec:Canonical Loop Form}).

\item The collapse of associated loops that are imperfectly nested loops
      was defined for the loop (see \specref{subsec:Loop Construct}),
      \code{simd} (see \specref{subsec:simd Construct}), \code{taskloop}
      (see \specref{subsec:taskloop Construct}) and \code{distribute} (see
      \specref{subsec:distribute simd Construct}) constructs.

\item SIMD constructs (see \specref{sec:SIMD Constructs}) were extended
      to allow the use of \code{atomic} constructs within them.

\item The \code{if} and \code{nontemporal} clauses were added to the 
      \code{simd} construct (see \specref{subsec:simd Construct}).

\item The \code{concurrent} construct was added to support compiler
      optimization of loops for which iterations may run in any order
      concurrently (see \specref{sec:concurrent Construct}).

\item To support task reductions, the \code{task} (see
      \specref{subsec:task Construct}) and \code{target} (see
      \specref{subsec:target Construct}) constructs were extended to
      accept the the \code{in_reduction} clause (see
      \specref{subsubsec:in_reduction clause}) and the \code{taskgroup}
      construct (see \specref{subsec:taskgroup Construct}) was extended
      to accept the \code{task_reduction} clause
      \specref{subsubsec:task_reduction clause}).

\item The \code{affinity} clause was added to the \code{task} construct
      (see \specref{subsec:task Construct}) to support hints that indicate 
      data affinity of explicit tasks.

\item To support taskloop reductions, the \code{taskloop} (see
      \specref{subsec:taskloop Construct}) and \code{taskloop simd} (see
      \specref{subsec:taskloop simd Construct}) constructs were extended
      to accept the \code{reduction} (see \specref{subsubsec:reduction clause})
      and \code{in_reduction} (see \specref{subsubsec:in_reduction clause})
      clauses.

\item To support mutually exclusive inout sets, a \code{mutexinoutset}
      \plc{dependence-type} was added to the \code{depend} clause (see
      \specref{subsec:Task Scheduling} and \specref{subsec:depend Clause}).

\item Predefined memory spaces (see \specref{subsec:Memory Spaces}), 
      predefined memory allocators and allocator traits (see 
      \specref{subsec:Memory Allocators}) and directives, clauses (see 
      \specref{sec:Memory Management Directives} and API routines (see 
      \specref{sec:Memory Management Routines}) to use them were added 
      to support different kinds of memories.

\item To reduce programmer effort implicit declare target directives for
      some functions (C, C++, Fortran) and subroutines (Fortran) were added
      (see \specref{subsec:target Construct} and
      \specref{subsec:declare target Directive}).

\item Support for nested \code{declare}~\code{target} directives was added
      (see \specref{subsec:declare target Directive}).

\item The \code{implements} clause was added to the
      \code{declare}~\code{target} directive to support the use of
      device-specific function implementations (see
      \specref{subsec:declare target Directive}).

\item The \code{declare}~\code{mapper} directive was added to support
      mapping of complicated data types (see
      \specref{subsec:declare mapper Directive}).

\item The \code{depend} clause was added to the \code{taskwait} construct
      (see \specref{subsec:taskwait Construct}).

\item To support acquire and release semantics with weak memory ordering, the
      \code{acq_rel}, \code{acquire}, and \code{release} clauses were added to
      the \code{atomic} construct (see \specref{subsec:atomic Construct}) and
      \code{flush} construct (see \specref{subsec:flush Construct}).

\item The \code{atomic} construct was extended with the \code{hint} clause
      (see \specref{subsec:atomic Construct}).

\item The \code{depend} clause (see \specref{subsec:depend Clause}) was
      extended to support iterators.

\item Lock hints were renamed to synchronization hints, and the
      old names were deprecated (see \specref{subsec:Synchronization Hints}).

\item To support conditional assignment to lastprivate variables, the
      \code{conditional} modifier was added to the \code{lastprivate}
      clause (see \specref{subsubsec:lastprivate clause}).

\item The description of the \code{map} clause was modified to clarify how
      structure members are mapped. (see \specref{subsec:map Clause}).

\item The capability to map pointer variables (C/C++) and assign the
      address of device memory that is mapped by an array section to them
      was added (see \specref{subsec:map Clause}).

\item The \code{defaultmap} clause (see \specref{subsubsec:defaultmap clause})
      was extended to allow selecting the data-mapping or data-sharing attributes for
      any of the scalar, aggregate, pointer or allocatable classes on a
      per-region basis. Additionally it accepts the \code{none} parameter to support the
      requirement that all variables referenced in the
      construct must be explicitly mapped or privatized.

\item The \code{omp_get_device_num} runtime routine
      (see \specref{subsec:omp_get_device_num}) was added to support
      determination of the device on which a thread is executing.

\item Runtime routines (see \specref{subsec:omp_set_affinity_format},
      \specref{subsec:omp_get_affinity_format},
      \specref{subsec:omp_display_affinity}, and
      \specref{subsec:omp_capture_affinity}) and environment variables
      (see \specref{sec:OMP_DISPLAY_AFFINITY} and
      \specref{sec:OMP_AFFINITY_FORMAT}) were added to provide OpenMP
      thread affinity information.

\item The \code{omp_get_device_num} runtime routine
      (see \specref{subsec:omp_get_device_num}) was added to support
      determination of the device on which a thread is executing.

\item Support for a first-party tool interface (see
      \specref{sec:ompt-overview}) was added.

\item Support for a third-party tool interface (see
      \specref{sec:ompd-overview}) was added.

\item Support for controlling offloading behavior with the
      \code{OMP_TARGET_OFFLOAD} environment variable was added
      (see  \specref{sec:OMP_TARGET_OFFLOAD}).

\end{itemize}


\section{Version 4.0 to 4.5 Differences}
\label{sec:Version 4.0 to 4.5 Differences}
\begin{itemize}
\item Support for several features of Fortran 2003 was added (see
      \specref{sec:normative references} for features that are still
      not supported).

\item A parameter was added to the \code{ordered} clause of the loop construct
      (see \specref{subsec:Loop Construct}) and clauses were added to the
      \code{ordered} construct (see \specref{subsec:ordered Construct}) to
      support doacross loop nests and use of the \code{simd} construct on
      loops with loop-carried backward dependences.

\item The \code{linear} clause was added to the loop construct
      (see \specref{subsec:Loop Construct}).

\item The \code{simdlen} clause was added to the \code{simd} construct
      (see \specref{subsec:simd Construct}) to support specification of
      the exact number of iterations desired per SIMD chunk.

\item The \code{priority} clause was added to the \code{task} construct
      (see \specref{subsec:task Construct}) to support hints that specify
      the relative execution priority of explicit tasks. The
      \code{omp_get_max_task_priority} routine was added to return
      the maximum supported priority value (see
      \specref{subsec:omp_get_max_task_priority}) and the
      \code{OMP_MAX_TASK_PRIORITY} environment variable was added to
      control the maximum priority value allowed (see
      \specref{sec:OMP_MAX_TASK_PRIORITY}).

\item Taskloop constructs (see \specref{subsec:taskloop Construct} and
      \specref{subsec:taskloop simd Construct}) were added to support
      nestable parallel loops that create OpenMP tasks.

\item To support interaction with native device implementations, the
      \code{use_device_ptr} clause was added to the \code{target data}
      construct (see \specref{subsec:target data Construct}) and the
      \code{is_device_ptr} clause was added to the \code{target} construct
      (see \specref{subsec:target Construct}).

\item The \code{nowait} and \code{depend} clauses were added to the
      \code{target} construct (see \specref{subsec:target Construct})
      to improve support for asynchronous execution of \code{target} regions.

\item The \code{private}, \code{firstprivate} and \code{defaultmap} clauses
      were added to the \code{target} construct (see \specref{subsec:target
      Construct}).

\item The \code{declare}~\code{target} directive was extended to allow
      mapping of global variables to be deferred to specific device
      executions and to allow an \plc{extended-list}
      to be specified in C/C++ (see \specref{subsec:declare target Directive}).

\item To support unstructured data mapping for devices, the
      \code{target enter data} (see \specref{subsec:target enter data
      Construct}) and \code{target exit data} (see \specref{subsec:target
      exit data Construct}) constructs were added and the \code{map} clause
      (see \specref{subsec:map Clause}) was updated.

\item To support a more complete set of device construct shortcuts, the
      \code{target}~\code{parallel}
      (see \specref{subsec:target parallel Construct}),
      target parallel loop
      (see \specref{subsec:Target Parallel Loop Construct}),
      target parallel loop SIMD
      (see \specref{subsec:Target Parallel Loop SIMD Construct}),
      and \code{target}~\code{simd}
      (see \specref{subsec:target simd Construct}),
      combined constructs were added.

\item The \code{if} clause was extended to take a
      \plc{directive-name-modifier} that allows it to apply
      to combined constructs (see \specref{sec:if Clause}).

\item The \code{hint} clause was addded to the \code{critical} construct
      (see \specref{subsec:critical Construct}).

\item The \code{source} and \code{sink} dependence types were added to the
      \code{depend} clause (see \specref{subsec:depend Clause}) to support
      doacross loop nests.

\item The implicit data-sharing attribute for scalar variables in
      \code{target} regions was changed to \code{firstprivate} (see
      \specref{subsubsec:Data-sharing Attribute Rules for Variables
      Referenced in a Construct}).

\item Use of some C++ reference types was allowed in some data sharing
      attribute clauses (see \specref{subsec:Data-Sharing Attribute Clauses}).

\item Semantics for reductions on C/C++ array sections were added and
      restrictions on the use of arrays and pointers in reductions were
      removed (see \specref{subsubsec:reduction clause}).

\item The \code{ref}, \code{val}, and \code{uval} modifiers were added to the
      \code{linear} clause (see \specref{subsubsec:linear clause}).

\item Support was added to the map clauses to handle structure elements
	(see \specref{subsec:map Clause}).

\item Query functions for OpenMP thread affinity were added (see
      \specref{subsec:omp_get_num_places} to \specref{subsec:omp_get_partition_place_nums}).

\item The lock API was extended with lock routines that support storing a hint
      with a lock to select a desired lock implementation for a lock's
      intended usage by the application code (see
      \specref{subsec:omp_init_lock_with_hint and omp_init_nest_lock_with_hint}).

\item Device memory routines were added to allow explicit allocation,
      deallocation, memory transfers and memory associations (see
      \specref{sec:Device Memory Routines}).

\item C/C++ Grammar (previously Appendix B) was moved to a separate document.
\end{itemize}



\section{Version 3.1 to 4.0 Differences}
\label{sec:Version 3.1 to 4.0 Differences}
\begin{itemize}
\item Various changes throughout the specification were made to provide initial support of
Fortran 2003 (see
\specref{sec:normative references}).

\item C/C++ array syntax was extended to support array sections (see
\specref{sec:Array Sections}).

\item The \code{proc_bind} clause (see
\specref{subsec:Controlling OpenMP Thread Affinity}),
the \code{OMP_PLACES}
environment variable (see
\specref{sec:OMP_PLACES}), and the \code{omp_get_proc_bind}
runtime routine (see
\specref{subsec:omp_get_proc_bind})
were added to support thread
affinity policies.

\item SIMD constructs were added to support SIMD parallelism (see
\specref{sec:SIMD Constructs}).

\item Device constructs (see
\specref{sec:Device Constructs}),
the \code{OMP_DEFAULT_DEVICE}
environment variable (see
\specref{sec:OMP_DEFAULT_DEVICE}), the
\code{omp_set_default_device}, \code{omp_get_default_device},
\code{omp_get_num_devices}, \code{omp_get_num_teams}, \code{omp_get_team_num}, and
\code{omp_is_initial_device} routines were added to support execution on devices.

\item Implementation defined task scheduling points for untied tasks were removed (see
\specref{subsec:Task Scheduling}).

\item The \code{depend} clause (see
\specref{subsec:depend Clause})
was added to support task dependences.

\item The \code{taskgroup} construct (see
\specref{subsec:taskgroup Construct}) was added to support
more flexible deep task synchronization.

\item The \code{reduction} clause (see
\specref{subsubsec:reduction clause}) was extended and the
\code{declare}~\code{reduction} construct (see
\specref{sec:declare reduction Directive}) was added to
support user defined reductions.

\item The \code{atomic} construct (see
\specref{subsec:atomic Construct}) was extended to support
atomic swap with the \code{capture} clause, to allow new atomic update and capture
forms, and to support sequentially consistent atomic operations with a new \code{seq_cst}
clause.

\item The \code{cancel} construct (see
\specref{subsec:cancel Construct}), the \code{cancellation}~\code{point} construct (see
\specref{subsec:cancellation point Construct}),
the \code{omp_get_cancellation}
runtime routine (see
\specref{subsec:omp_get_cancellation})
and the \code{OMP_CANCELLATION}
environment variable (see
\specref{sec:OMP_CANCELLATION}) were added to support the
concept of cancellation.

\item The \code{OMP_DISPLAY_ENV} environment variable (see
\specref{sec:OMP_DISPLAY_ENV}) was
added to display the value of ICVs associated with the OpenMP environment
variables.

\item Examples (previously Appendix A) were moved to a separate document.
\end{itemize}






\section{Version 3.0 to 3.1 Differences}
\label{sec:Version 3.0 to 3.1 Differences}
\begin{itemize}
\item The \code{final} and \code{mergeable} clauses (see
\specref{subsec:task Construct}) were added to
the \code{task} construct to support optimization of task data environments.

\item The \code{taskyield} construct (see
\specref{subsec:taskyield Construct}) was added to allow
user-defined task scheduling points.

\item The \code{atomic} construct (see
\specref{subsec:atomic Construct}) was extended to include
\code{read}, \code{write}, and \code{capture} forms, and an \code{update} clause was added to apply
the already existing form of the \code{atomic} construct.

\item Data environment restrictions were changed to allow \code{intent(in)} and
\code{const}-qualified types for the \code{firstprivate} clause (see
\specref{subsubsec:firstprivate clause}).

\item Data environment restrictions were changed to allow Fortran pointers in
\code{firstprivate} (see
\specref{subsubsec:firstprivate clause})
and \code{lastprivate} (see
\specref{subsubsec:lastprivate clause}).

\item New reduction operators \code{min} and \code{max} were added for C and C++

\item The nesting restrictions in
\specref{sec:Nesting of Regions} were clarified to disallow
closely-nested OpenMP regions within an \code{atomic} region. This allows an \code{atomic}
region to be consistently defined with other OpenMP regions so that they include all
code in the atomic construct.

\item The \code{omp_in_final} runtime library routine (see
\specref{subsec:omp_in_final}) was
added to support specialization of final task regions.

\item The \plc{nthreads-var} ICV has been modified to be a list of the number of threads to use
at each nested parallel region level. The value of this ICV is still set with the
\code{OMP_NUM_THREADS} environment variable (see
\specref{sec:OMP_NUM_THREADS}), but the
algorithm for determining the number of threads used in a parallel region has been
modified to handle a list (see
\specref{subsec:Determining the Number of Threads for a parallel Region}).

\item The \plc{bind-var} ICV has been added, which controls whether or not threads are bound
to processors (see
\specref{subsec:ICV Descriptions}).
The value of this ICV can be set with
the \code{OMP_PROC_BIND} environment variable (see
\specref{sec:OMP_PROC_BIND}).

\item Descriptions of examples (previously Appendix A) were expanded and
clarified.

\item Replaced incorrect use of \code{omp_integer_kind} in Fortran interfaces with
\code{selected_int_kind(8)}.
\end{itemize}







\section{Version 2.5 to 3.0 Differences}
\label{sec:Version 2.5 to 3.0 Differences}
The concept of tasks has been added to the OpenMP execution model (see
\specref{subsec:Tasking Terminology} and
\specref{sec:Execution Model}).

\begin{itemize}
\item The \code{task} construct (see
\specref{sec:Tasking Constructs})
has been added, which provides
a mechanism for creating tasks explicitly.

\item The \code{taskwait} construct (see
\specref{subsec:taskwait Construct}) has been added, which
causes a task to wait for all its child tasks to complete.

\item The OpenMP memory model now covers atomicity of memory accesses (see
\specref{subsec:Structure of the OpenMP Memory Model}).
The description of the behavior of \code{volatile} in terms of
\code{flush} was removed.

\item In Version 2.5, there was a single copy of the \plc{nest-var}, \plc{dyn-var}, \plc{nthreads-var} and
\plc{run-sched-var} internal control variables (ICVs) for the whole program. In Version
3.0, there is one copy of these ICVs per task (see
\specref{sec:Internal Control Variables}). As a result,
the \code{omp_set_num_threads}, \code{omp_set_nested} and \code{omp_set_dynamic}
runtime library routines now have specified effects when called from inside a
\code{parallel} region (see
\specref{subsec:omp_set_num_threads},
\specref{subsec:omp_set_dynamic} and
\specref{subsec:omp_set_nested}).

\item The definition of active \code{parallel} region has been changed: in Version 3.0 a
\code{parallel} region is active if it is executed by a team consisting of more than one
thread (see
\specref{subsec:OpenMP Language Terminology}).

\item The rules for determining the number of threads used in a \code{parallel} region have
been modified (see
\specref{subsec:Determining the Number of Threads for a parallel Region}).

\item In Version 3.0, the assignment of iterations to threads in a loop construct with a
\code{static} schedule kind is deterministic (see
\specref{subsec:Loop Construct}).

\item In Version 3.0, a loop construct may be associated with more than one perfectly
nested loop. The number of associated loops may be controlled by the \code{collapse}
clause (see
\specref{subsec:Loop Construct}).

\item Random access iterators, and variables of unsigned integer type, may now be used as
loop iterators in loops associated with a loop construct (see
\specref{subsec:Loop Construct}).

\item The schedule kind \code{auto} has been added, which gives the implementation the
freedom to choose any possible mapping of iterations in a loop construct to threads in
the team (see \specref{subsec:Loop Construct}).

\item Fortran assumed-size arrays now have predetermined data-sharing attributes (see
\specref{subsubsec:Data-sharing Attribute Rules for Variables Referenced in a Construct}).

\item In Fortran, \code{firstprivate} is now permitted as an argument to the \code{default}
clause (see
\specref{subsubsec:default clause}).

\item For list items in the \code{private} clause, implementations are no longer permitted to use
the storage of the original list item to hold the new list item on the master thread. If
no attempt is made to reference the original list item inside the \code{parallel} region, its
value is well defined on exit from the \code{parallel} region (see
\specref{subsubsec:private clause}).

\item In Version 3.0, Fortran allocatable arrays may appear in \code{private},
\code{firstprivate}, \code{lastprivate}, \code{reduction}, \code{copyin} and \code{copyprivate}
clauses. (see
\specref{subsec:threadprivate Directive},
\specref{subsubsec:private clause},
\specref{subsubsec:firstprivate clause},
\specref{subsubsec:lastprivate clause},
\specref{subsubsec:reduction clause},
\specref{subsubsec:copyin clause} and
\specref{subsubsec:copyprivate clause}).

\item In Version 3.0, static class members variables may appear in a \code{threadprivate}
directive (see
\specref{subsec:threadprivate Directive}).

\item Version 3.0 makes clear where, and with which arguments, constructors and
destructors of private and threadprivate class type variables are called (see
\specref{subsec:threadprivate Directive},
\specref{subsubsec:private clause},
\specref{subsubsec:firstprivate clause},
\specref{subsubsec:copyin clause} and
\specref{subsubsec:copyprivate clause}).

\item The runtime library routines \code{omp_set_schedule} and \code{omp_get_schedule}
have been added; these routines respectively set and retrieve the value of the
\plc{run-sched-var} ICV (see\\
\specref{subsec:omp_set_schedule} and
\specref{subsec:omp_get_schedule}).

\item The \plc{thread-limit-var} ICV has been added, which controls the maximum number of
threads participating in the OpenMP program. The value of this ICV can be set with
the \code{OMP_THREAD_LIMIT} environment variable and retrieved with the
\code{omp_get_thread_limit} runtime library routine (see
\specref{subsec:ICV Descriptions},
\specref{subsec:omp_get_thread_limit} and
\specref{sec:OMP_THREAD_LIMIT}).

\item The \plc{max-active-levels-var} ICV has been added, which controls the number of nested
active \code{parallel} regions. The value of this ICV can be set with the
\code{OMP_MAX_ACTIVE_LEVELS} environment variable and the
\code{omp_set_max_active_levels} runtime library routine, and it can be retrieved
with the \code{omp_get_max_active_levels} runtime library routine (see
\specref{subsec:ICV Descriptions},
\specref{subsec:omp_set_max_active_levels},
\specref{subsec:omp_get_max_active_levels} and
\specref{sec:OMP_MAX_ACTIVE_LEVELS}).

\item The \plc{stacksize-var} ICV has been added, which controls the stack size for threads that
the OpenMP implementation creates. The value of this ICV can be set with the
\code{OMP_STACKSIZE} environment variable (see
\specref{subsec:ICV Descriptions} and
\specref{sec:OMP_STACKSIZE}).

\item The \plc{wait-policy-var} ICV has been added, which controls the desired behavior of
waiting threads. The value of this ICV can be set with the \code{OMP_WAIT_POLICY}
environment variable (see
\specref{subsec:ICV Descriptions} and
\specref{sec:OMP_WAIT_POLICY}).

\item The \code{omp_get_level} runtime library routine has been added, which returns the
number of nested \code{parallel} regions enclosing the task that contains the call (see
\specref{subsec:omp_get_level}).

\item The \code{omp_get_ancestor_thread_num} runtime library routine has been added,
which returns, for a given nested level of the current thread, the thread number of the
ancestor (see
\specref{subsec:omp_get_ancestor_thread_num}).

\item The \code{omp_get_team_size} runtime library routine has been added, which returns,
for a given nested level of the current thread, the size of the thread team to which the
ancestor belongs (see
\specref{subsec:omp_get_team_size}).

\item The \code{omp_get_active_level} runtime library routine has been added, which
returns the number of nested, active \code{parallel} regions enclosing the task that
contains the call (see\linebreak \specref{subsec:omp_get_active_level}).

\item In Version 3.0, locks are owned by tasks, not by threads (see
\specref{sec:Lock Routines}).
\end{itemize}


% This is the end of appendix-E-FeaturesHistory.tex

