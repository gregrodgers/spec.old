% This is implementation_defined.tex (Appendix C) of the OpenMP specification.
% This is an included file. See the master file for more information.
%
% When editing this file:
%
%    1. To change formatting, appearance, or style, please edit openmp.sty.
%
%    2. Custom commands and macros are defined in openmp.sty.
%
%    3. Be kind to other editors -- keep a consistent style by copying-and-pasting to
%       create new content.
%
%    4. We use semantic markup, e.g. (see openmp.sty for a full list):
%         \code{}     % for bold monospace keywords, code, operators, etc.
%         \plc{}      % for italic placeholder names, grammar, etc.
%
%    5. There are environments that provide special formatting, e.g. language bars.
%       Please use them whereever appropriate.  Examples are:
%
%         \begin{fortranspecific}
%         This is text that appears enclosed in blue language bars for Fortran.
%         \end{fortranspecific}
%
%         \begin{note}
%         This is a note.  The "Note -- " header appears automatically.
%         \end{note}
%
%    6. Other recommendations:
%         Use the convenience macros defined in openmp.sty for the minor headers
%         such as Comments, Syntax, etc.
%
%         To keep items together on the same page, prefer the use of
%         \begin{samepage}.... Avoid \parbox for text blocks as it interrupts line numbering.
%         When possible, avoid \filbreak, \pagebreak, \newpage, \clearpage unless that's
%         what you mean. Use \needspace{} cautiously for troublesome paragraphs.
%
%         Avoid absolute lengths and measures in this file; use relative units when possible.
%         Vertical space can be relative to \baselineskip or ex units. Horizontal space
%         can be relative to \linewidth or em units.
%
%         Prefer \emph{} to italicize terminology, e.g.:
%             This is a \emph{definition}, not a placeholder.
%             This is a \plc{var-name}.
%

\chapter{OpenMP Implementation-Defined Behaviors}
\label{chap:OpenMP Implementation-Defined Behaviors}
\index{implementation}
This appendix summarizes the behaviors that are described as implementation 
defined in this API. Each behavior is cross-referenced back to its description 
in the main specification. An implementation is required to define and to document 
its behavior in these cases.

\begin{itemize}
\item \textbf{Processor}: a hardware unit that is implementation defined (see
      \specref{subsec:Threading Concepts}).
\item \textbf{Device}: an implementation defined logical execution engine (see
      \specref{subsec:Threading Concepts}).
\item \textbf{Device address}: reference to an address in a 
      \emph{device data environment} (see \specref{subsec:Data Terminology}).
\item \textbf{Memory model}: the minimum size at which a memory update may 
      also read and write back adjacent variables that are part of another 
      variable (as array or structure elements) is implementation defined but 
      is no larger than required by the base language (see 
      \specref{subsec:Structure of the OpenMP Memory Model}).
\item \code{requires} \textbf{directive}: support of requirements is 
      implementation defined. All implementation-defined requirements should 
      begin with \code{ext_} (see \specref{sec:requires Directive}).
\item \textbf{Requires directive}: Support for any feature specified by a
      requirement clause on a \code{requires} directive is implementation
      defined (see \specref{sec:requires Directive}).
\index{internal control variables}
\item \textbf{Internal control variables}: the initial values of
      \plc{dyn-var}, \plc{nthreads-var}, \plc{run-sched-var},
      \plc{def-sched-var}, \plc{bind-var}, \plc{stacksize-var},
      \plc{wait-policy-var}, \plc{thread-limit-var},
      \plc{max-active-levels-var}, \plc{place-partition-var}, 
      \plc{affinity-format-var}, \plc{default-device-var} and 
      \plc{def-allocator-var} are implementation defined.  The method for
      initializing a target device's internal control variable is
      implementation defined (see \specref{subsec:ICV Initialization}).
\item \textbf{OpenMP context}: the accepted \plc{isa-name} values for
      the \plc{isa} trait, the accepted \plc{arch-name} values for the
      \plc{arch} trait, and the accepted \plc{extension-name} values for
      the \plc{extension} trait are implementation defined (see
      \specref{subsec:OpenMP Context}).
\item \code{declare}~\code{variant} \textbf{directive}: whether, for some specific
      OpenMP context, the prototype of the variant should differ from that of
      the base function, and if so how it should differ, is implementation defined
      (see \specref{subsec:declare variant Directive}).
\index{dynamic thread adjustment}
\item \textbf{Dynamic adjustment of threads}: providing the ability to 
      adjust the number of threads dynamically is implementation defined. 
      Implementations are allowed to deliver fewer threads (but at least one) 
      than indicated in Algorithm 2.1 even if dynamic adjustment is disabled (see 
      \specref{subsec:Determining the Number of Threads for a parallel Region}).
\item \textbf{Thread affinity}: For the \code{close} thread affinity policy, 
      if \emph{T > P} and \emph{P} does not divide \emph{T} evenly, the exact
      number of threads in a particular place is implementation defined.
      For the \code{spread} thread affinity, if \emph{T > P} and \emph{P} does not
      divide \emph{T} evenly, the exact number of threads in a particular
      subpartition is implementation defined.  The determination of
      whether the affinity request can be fulfilled is implementation
      defined.  If not, the mapping of threads in the team 
      to places is implementation defined (see
      \specref{subsec:Controlling OpenMP Thread Affinity}).
\item \code{teams} \textbf{construct}: the number of teams that are created 
      is implementation defined but less than or equal to the value of the 
      \code{num_teams} clause if specified. The maximum number of threads 
      that participate in the contention group that each team initiates is
      implementation defined but less than or equal to the value of the 
      \code{thread_limit} clause if specified.  The assignment of the 
      initial threads to places and the values of the \plc{place-partition-var} 
      and \plc{default-device-var} ICVs for each initial thread are
      implementation defined (see \specref{sec:teams Construct}).
\item \code{sections} \textbf{construct}: the method of scheduling the 
      structured blocks among threads in the team is implementation defined 
      (see \specref{subsec:sections Construct}).
\item \code{single} \textbf{construct}: the method of choosing a thread to 
      execute the structured block is implementation defined (see 
      \specref{subsec:single Construct})
\item \textbf{Worksharing-Loop directive}: the integer type (or kind, for 
      Fortran) used to compute the iteration count of a collapsed loop is 
      implementation defined. The effect of the \code{schedule(runtime)} 
      clause when the \plc{run-sched-var} ICV is set to \code{auto} is
      implementation defined. The value of \plc{simd_width} for the 
      \code{simd} schedule modifier is implementation defined (see 
      \specref{subsec:Worksharing-Loop Construct}).
\item \code{simd} \textbf{construct}: the integer type (or kind, for
      Fortran) used to compute the iteration count for the collapsed loop
      is implementation defined. The number of iterations that are executed 
      concurrently at any given time is implementation defined. If the 
      \plc{alignment} parameter is not specified in the \code{aligned} clause, 
      the default alignments for the SIMD instructions are implementation 
      defined (see \specref{subsubsec:simd Construct}).
\item \code{declare}~\code{simd} \textbf{directive}: if the parameter of the
      \code{simdlen} clause is not a constant positive integer expression,
      the number of concurrent arguments for the function is
      implementation defined. If the \plc{alignment} parameter of the
      \code{aligned} clause is not specified, the default alignments for
      SIMD instructions are implementation defined (see
      \specref{subsubsec:declare simd Directive}).
\item \code{distribute} \textbf{construct}: the integer type (or kind, for
      Fortran) used to compute the iteration count for the collapsed loop is
      implementation defined.  If no \code{dist_schedule} clause is specified 
      then the schedule for the \code{distribute} construct is implementation 
      defined (see \specref{subsec:distribute Construct}).
\item \code{taskloop} \textbf{construct}: The number of loop
      iterations assigned to a task created from a \code{taskloop}
      construct is implementation defined, unless the \code{grainsize} or
      \code{num_tasks} clause is specified. The integer type (or kind,
      for Fortran) used to compute the iteration count for the collapsed
      loop is implementation defined (see \specref{subsec:taskloop Construct}).

\begin{cppspecific}
\item \code{taskloop} \textbf{construct}: For \code{firstprivate} variables 
      of class type, the number of invocations of copy constructors to perform 
      the initialization is implementation defined 
      (see \specref{subsec:taskloop Construct}).
\end{cppspecific}

\item \textbf{Memory spaces}: The actual storage resource that each memory 
      space defined in \tabref{tab:Predefined Memory Spaces} represents is 
      implementation defined.
\item \textbf{Memory allocators}: The minimum partitioning size for partitioning 
      of allocated memory over the storage resources is implementation defined 
      (see \specref{subsec:Memory Allocators}). The default value for the 
      \code{pool_size} allocator trait is implementation defined (see 
      \tabref{tab:Allocator traits}). The associated memory space for each of 
      the predefined \code{omp_cgroup_mem_alloc}, \code{omp_pteam_mem_alloc} 
      and \code{omp_thread_mem_alloc} allocators is implementation defined 
      (see \tabref{tab:Predefined Allocators}).
\item \code{is_device_ptr} \textbf{clause}: Support for pointers created outside 
      of the OpenMP device data management routines is implementation defined 
      (see \specref{subsec:target Construct}).
\item \code{target} \textbf{construct}: the effect of invoking a virtual member
      function of an object on a device other than the device on which the object 
      was constructed is implementation defined (see 
      \specref{subsec:target Construct}).
\item \code{atomic} \textbf{construct}: a compliant implementation may enforce 
      exclusive access \index{atomic construct@{\code{atomic} construct}}
      between \code{atomic} regions that update different storage locations. 
      The circumstances under which this occurs are implementation defined. 
      If the storage location designated by \emph{x} is not size-aligned (that 
      is, if the byte alignment of \emph{x} is not a multiple of the size of 
      \emph{x}), then the behavior of the atomic region is implementation 
      defined (see \specref{subsec:atomic Construct}).

\begin{fortranspecific}
\item \textbf{Data-sharing attributes}: The data-sharing attributes of dummy 
      arguments without the \code{VALUE} attribute are implementation-defined 
      if the associated actual argument is shared, except for the conditions 
      specified (see 
      \specref{subsubsec:Variables Referenced in a Region but not in a Construct}).
\item \code{threadprivate} \textbf{directive}: if the conditions for values of 
      data in the threadprivate objects of threads (other than an initial thread) 
      to persist between two consecutive active parallel regions do not all hold, 
      the allocation status of an allocatable variable in the second region is 
      implementation defined (see \specref{subsec:threadprivate Directive}).
\item \textbf{Runtime library definitions}: it is implementation defined whether 
      the include file \code{omp_lib.h} or the module \code{omp_lib} (or both) 
      is provided. It is implementation defined whether any of the OpenMP runtime 
      library routines that take an argument are extended with a generic interface 
      so arguments of different \code{KIND} type can be accommodated 
      (see \specref{sec:runtime library definitions}).
\end{fortranspecific}

\item \code{omp_set_num_threads} \textbf{routine}: if the argument is not a 
      positive integer the behavior is implementation defined 
      (see \specref{subsec:omp_set_num_threads}).
\item \code{omp_set_schedule} \textbf{routine}: for implementation specific 
      schedule kinds, the values and associated meanings of the second argument 
      are implementation defined (see \specref{subsec:omp_set_schedule}).
\item \code{omp_get_supported_active_levels} \textbf{routine}: the number
      of active levels of parallelism supported by the implementation is
      implementation defined, but must be greater than 0 (see
      \specref{subsec:omp_get_supported_active_levels}).
\item \code{omp_set_max_active_levels} \textbf{routine}: when called from 
      within any explicit \code{parallel} region the binding thread set 
      (and binding region, if required) for the \code{omp_set_max_active_levels} 
      region is implementation defined and the behavior is implementation defined. 
      If the argument is not a non-negative integer then the behavior is 
      implementation defined (see \specref{subsec:omp_set_max_active_levels}).
\item \code{omp_get_max_active_levels} \textbf{routine}: when called from within 
      any explicit \code{parallel} region the binding thread set (and binding 
      region, if required) for the \code{omp_get_max_active_levels} region is 
       implementation defined (see \specref{subsec:omp_get_max_active_levels}).
\item \code{omp_get_place_proc_ids} \textbf{routine}: the meaning of the
      non-negative numerical identifiers returned by the 
      \code{omp_get_place_proc_ids} routine is implementation defined. The
      order of the numerical identifiers returned in the array \plc{ids} is
      implementation defined (see \specref{subsec:omp_get_place_proc_ids}).

\item \code{omp_set_affinity_format} \textbf{routine}: when called from within 
      any explicit \code{parallel} region, the binding thread set (and binding 
      region, if required) for the \code{omp_set_affinity_format} region is 
      implementation defined and the behavior is implementation defined. If 
      the argument does not conform to the specified format then the result 
      is implementation defined (see \specref{subsec:omp_set_affinity_format}).
\item \code{omp_get_affinity_format} \textbf{routine}: when called from within 
      any explicit \code{parallel} region the binding thread set (and binding 
      region, if required) for the \code{omp_get_affinity_format} region is 
      implementation defined (see \specref{subsec:omp_get_affinity_format}).
\item \code{omp_display_affinity} \textbf{routine}:  if the argument does not
      conform to the specified format then the result is implementation defined 
      (see \specref{subsec:omp_display_affinity}).
\item \code{omp_capture_affinity} \textbf{routine}:  if the \plc{format} argument 
      does not conform to the specified format then the result is implementation 
      defined (see \specref{subsec:omp_capture_affinity}).
\item \code{omp_get_initial_device} \textbf{routine}: the value of the device 
      number of the host device is implementation defined (see 
      \specref{subsec:omp_get_initial_device}).
\item \code{omp_target_memcpy_rect} \textbf{routine}: the maximum number of 
      dimensions supported is implementation defined, but must be at least 
      three (see \specref{subsec:omp_target_memcpy_rect}).
\item \code{ompt_callback_sync_region_wait}, \code{ompt_callback_mutex_released},
      \code{ompt_callback_dependences}, \code{ompt_callback_task_dependence},
      \code{ompt_callback_work}, \code{ompt_callback_master},
      \code{ompt_callback_target_map}, \code{ompt_callback_sync_region},
      \code{ompt_callback_lock_init}, \code{ompt_callback_lock_destroy},
      \code{ompt_callback_mutex_acquire}, \code{ompt_callback_mutex_acquired},
      \code{ompt_callback_nest_lock}, \code{ompt_callback_flush},
      \code{ompt_callback_cancel} and \code{ompt_callback_dispatch}
      \textbf{tool callbacks}: if a tool attempts to register a callback 
      with the string name using the runtime entry point \code{ompt_set_callback}, 
      it is implementation defined whether the registered callback may never or 
      sometimes invoke this callback for the associated events (see 
      \tabref{table:valid_rc})
\item \textbf{Device tracing}: Whether a target device supports tracing or 
      not is implementation defined; if a target device does not support tracing, 
      a \code{NULL} may be supplied for the \plc{lookup} function to a tool's 
      device initializer (see \specref{sec:tracing-device-activity}).
\item \code{ompt_set_trace_ompt} and \code{ompt_buffer_get_record_ompt} 
      \textbf{runtime entry points}: it is implementation defined whether 
      a device-specific tracing interface will define this runtime entry 
      point, indicating that it can collect traces in OMPT format. The kinds 
      of trace records available for a device is implementation defined 
      (see \specref{sec:tracing-device-activity}).
\item \code{ompt_callback_target_data_op_t} \textbf{callback type}: it is 
      implementation defined whether in some operations \plc{src_addr} or 
      \plc{dest_addr} might point to an intermediate buffer 
      (see \specref{sec:ompt_callback_target_data_op_t}).
\item \code{ompt_set_callback_t} \textbf{entry point type}: the subset
      of the associated event in which the callback is invoked is
      implementation defined (see \specref{sec:ompt_set_callback_t}).
\item \code{ompt_get_place_proc_ids_t} \textbf{entry point type}: the meaning 
      of the numerical identifiers returned is implementation defined.  The 
      order of \plc{ids} returned in the array is implementation defined (see 
      \specref{sec:ompt_get_place_proc_ids_t}).
\item \code{ompt_get_partition_place_nums_t} \textbf{entry point type}:
      the order of the identifiers returned in the array \plc{place_nums}
      is implementation defined (see \specref{sec:ompt_get_partition_place_nums_t}).
\item \code{ompt_get_proc_id_t} \textbf{entry point type}: the meaning of the 
      numerical identifier returned is implementation defined 
      (see \specref{sec:ompt_get_proc_id_t}).
\item \code{ompd_callback_print_string_fn_t} \textbf{callback function}:
      the value of \plc{catergory} is implementation defined (see
      \specref{subsubsubsec:ompd_callback_print_string_fn_t}).
\item \code{ompd_parallel_handle_compare} \textbf{operation}: the means by 
      which parallel region handles are ordered is implementation defined 
      (see \specref{subsubsubsec:ompd_parallel_handle_compare}).
\item \code{ompd_task_handle_compare} \textbf{operation}: the means by which 
      task handles are ordered is implementation defined (see
      \specref{subsubsubsec:ompd_task_handle_compare}).
\item \textbf{OMPT thread states}: The set of OMPT thread states supported 
      is implementation defined (see \specref{sec:thread-states}).
\item \code{OMP_SCHEDULE} \textbf{environment variable}: if the value does not
      conform to the specified format then the result is implementation defined 
      (see \specref{sec:OMP_SCHEDULE}).
\item \code{OMP_NUM_THREADS} \textbf{environment variable}: if any value of the 
      list specified leads to a number of threads that is greater than the 
      implementation can support, or if any value is not a positive integer,
      then the result is implementation defined (see \specref{sec:OMP_NUM_THREADS}).
\item \code{OMP_DYNAMIC} \textbf{environment variable}: if the value is neither
      \code{true} nor \code{false} the behavior is implementation defined (see
      \specref{sec:OMP_DYNAMIC}).
\item \code{OMP_PROC_BIND} \textbf{environment variable}: if the value is not 
      \code{true}, \code{false}, or a comma separated list of \code{master}, 
      \code{close}, or \code{spread}, the behavior is implementation defined. 
      The behavior is also implementation defined if an initial thread cannot 
      be bound to the first place in the OpenMP place list (see 
      \specref{sec:OMP_PROC_BIND}).
\item \code{OMP_PLACES} \textbf{environment variable}: the meaning of the numbers 
      specified in the environment variable and how the numbering is done are 
      implementation defined. The precise definitions of the abstract names are 
      implementation defined. An implementation may add implementation-defined 
      abstract names as appropriate for the target platform. When creating a 
      place list of n elements by appending the number \emph{n} to an abstract 
      name, the determination of which resources to include in the place list 
      is implementation defined. When requesting more resources than available,
      the length of the place list is also implementation defined. The behavior 
      of the program is implementation defined when the execution environment 
      cannot map a numerical value (either explicitly defined or implicitly 
      derived from an interval) within the \code{OMP_PLACES} list to a processor 
      on the target platform, or if it maps to an unavailable processor. The 
      behavior is also implementation defined when the \code{OMP_PLACES} 
      environment variable is defined using an abstract name 
      (see \specref{sec:OMP_PLACES}).
\item \code{OMP_STACKSIZE} \textbf{environment variable}: if the value does 
      not conform to the specified format or the implementation cannot provide 
      a stack of the specified size then the behavior is implementation defined 
      (see \specref{sec:OMP_STACKSIZE}).
\item \code{OMP_WAIT_POLICY} \textbf{environment variable}: the details of the 
      \code{ACTIVE} and \code{PASSIVE} behaviors are implementation defined 
      (see \specref{sec:OMP_WAIT_POLICY}).
\item \code{OMP_MAX_ACTIVE_LEVELS} \textbf{environment variable}: if the value 
      is not a non-negative integer or is greater than the number of parallel 
      levels an implementation can support then the behavior is implementation 
      defined (see \specref{sec:OMP_MAX_ACTIVE_LEVELS}).
\item \code{OMP_NESTED} \textbf{environment variable}: if the value is neither
      \code{true} nor \code{false} the behavior is implementation defined (see
      \specref{sec:OMP_NESTED}).
\item \textbf{Conflicting } \code{OMP_NESTED} \textbf{and} 
      \code{OMP_MAX_ACTIVE_LEVELS} \textbf{environment variables}: if both 
      environment variables are set, the value of \code{OMP_NESTED} is
      \code{false}, and the value of \code{OMP_MAX_ACTIVE_LEVELS} is greater
      than 1, the behavior is implementation defined (see \specref{sec:OMP_NESTED}).
\item \code{OMP_THREAD_LIMIT} \textbf{environment variable}: if the requested 
      value is greater than the number of threads an implementation can support, 
      or if the value is not a positive integer, the behavior of the program is 
      implementation defined (see \specref{sec:OMP_THREAD_LIMIT}).
\item \code{OMP_DISPLAY_AFFINITY} \textbf{environment variable}: for all values 
      of the environment variables other than \code{TRUE} or \code{FALSE}, the 
      display action is implementation defined (see 
      \specref{sec:OMP_DISPLAY_AFFINITY}).
\item \code{OMP_AFFINITY_FORMAT} \textbf{environment variable}: if the value 
      does not conform to the specified format then the result is implementation 
      defined (see \specref{sec:OMP_AFFINITY_FORMAT}).
\item \code{OMP_TARGET_OFFLOAD} \textbf{environment variable}: the support of 
      \code{disabled} is implementation defined (see 
      \specref{sec:OMP_TARGET_OFFLOAD}).
\item \code{OMP_DEBUG} \textbf{environment variable}: if the value is neither
      \code{disabled} nor \code{enabled} the behavior is implementation defined 
      (see \specref{sec:OMP_DEBUG}).
\end{itemize}

