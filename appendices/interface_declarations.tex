% This is interface_declarations.tex (Appendix B) of the OpenMP specification.
% This is an included file. See the master file for more information.
%
% When editing this file:
%
%    1. To change formatting, appearance, or style, please edit openmp.sty.
%
%    2. Custom commands and macros are defined in openmp.sty.
%
%    3. Be kind to other editors -- keep a consistent style by copying-and-pasting to
%       create new content.
%
%    4. We use semantic markup, e.g. (see openmp.sty for a full list):
%         \code{}     % for bold monospace keywords, code, operators, etc.
%         \plc{}      % for italic placeholder names, grammar, etc.
%
%    5. There are environments that provide special formatting, e.g. language bars.
%       Please use them whereever appropriate.  Examples are:
%
%         \begin{fortranspecific}
%         This is text that appears enclosed in blue language bars for Fortran.
%         \end{fortranspecific}
%
%         \begin{note}
%         This is a note.  The "Note -- " header appears automatically.
%         \end{note}
%
%    6. Other recommendations:
%         Use the convenience macros defined in openmp.sty for the minor headers
%         such as Comments, Syntax, etc.
%
%         To keep items together on the same page, prefer the use of
%         \begin{samepage}.... Avoid \parbox for text blocks as it interrupts line numbering.
%         When possible, avoid \filbreak, \pagebreak, \newpage, \clearpage unless that's
%         what you mean. Use \needspace{} cautiously for troublesome paragraphs.
%
%         Avoid absolute lengths and measures in this file; use relative units when possible.
%         Vertical space can be relative to \baselineskip or ex units. Horizontal space
%         can be relative to \linewidth or em units.
%
%         Prefer \emph{} to italicize terminology, e.g.:
%             This is a \emph{definition}, not a placeholder.
%             This is a \plc{var-name}.
%



\chapter{Interface Declarations}
\index{interface declarations}
\index{header files}
\index{include files}
\label{chap:Interface Declarations}
This section gives examples of the C/C++ header file, the Fortran \code{include} file and
Fortran \code{module} that shall be provided by implementations as specified in Chapter 3. It
also includes an example of a Fortran 90 generic interface for a library routine. This is a
non-normative section; implementation files may differ.




%\pagebreak
\section{Example of the \hcode{omp.h} Header File}
\label{sec:Example of the omp.h Header File}
\cstub{}{omp.h}

%\pagebreak

{\hyphenpenalty=10000\section{Example of an Interface Declaration \hcode{include} File}}
\label{sec:Example of an Interface Declaration include File}
\emph{omp\_lib\_kinds.h:}
\fstub{}{omp_lib_kinds.h}

\emph{omp\_lib.h:}
\fstub{}{omp_lib.h}

%\pagebreak
\section{Example of a Fortran Interface Declaration \hcode{module}}
\label{sec:Example of a Fortran Interface Declaration module}
\ffreestub{}{omp_lib.f90}

%\pagebreak
\section{Example of a Generic Interface for a Library Routine}
\label{sec:Example of a Generic Interface for a Library Routine}
Any of the OpenMP runtime library routines that take an argument may be extended
with a generic interface so arguments of different \code{KIND} type can be accommodated.

The \code{OMP_SET_NUM_THREADS} interface could be specified in the \code{omp_lib} module
as follows:

\begin{ompfSubroutine}
interface omp_set_num_threads

  subroutine omp_set_num_threads_4 (num_threads)
    use omp_lib_kinds
    integer(4), intent(in) :: num_threads
  end subroutine omp_set_num_threads_4

  subroutine omp_set_num_threads_8 (num_threads)
    use omp_lib_kinds
    integer(8), intent(in) :: num_threads
  end subroutine omp_set_num_threads_8

end interface omp_set_num_threads
\end{ompfSubroutine}

% This is the end of appendix-C-InterfaceDeclarations.tex

