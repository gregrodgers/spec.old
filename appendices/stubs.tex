% This is stubs.tex (Appendix A) of the OpenMP specification.
% This is an included file. See the master file for more information.
%
% When editing this file:
%
%    1. To change formatting, appearance, or style, please edit openmp.sty.
%
%    2. Custom commands and macros are defined in openmp.sty.
%
%    3. Be kind to other editors -- keep a consistent style by copying-and-pasting to
%       create new content.
%
%    4. We use semantic markup, e.g. (see openmp.sty for a full list):
%         \code{}     % for bold monospace keywords, code, operators, etc.
%         \plc{}      % for italic placeholder names, grammar, etc.
%
%    5. There are environments that provide special formatting, e.g. language bars.
%       Please use them whereever appropriate.  Examples are:
%
%         \begin{fortranspecific}
%         This is text that appears enclosed in blue language bars for Fortran.
%         \end{fortranspecific}
%
%         \begin{note}
%         This is a note.  The "Note -- " header appears automatically.
%         \end{note}
%
%    6. Other recommendations:
%         Use the convenience macros defined in openmp.sty for the minor headers
%         such as Comments, Syntax, etc.
%
%         To keep items together on the same page, prefer the use of
%         \begin{samepage}.... Avoid \parbox for text blocks as it interrupts line numbering.
%         When possible, avoid \filbreak, \pagebreak, \newpage, \clearpage unless that's
%         what you mean. Use \needspace{} cautiously for troublesome paragraphs.
%
%         Avoid absolute lengths and measures in this file; use relative units when possible.
%         Vertical space can be relative to \baselineskip or ex units. Horizontal space
%         can be relative to \linewidth or em units.
%
%         Prefer \emph{} to italicize terminology, e.g.:
%             This is a \emph{definition}, not a placeholder.
%             This is a \plc{var-name}.
%


\chapter{Stubs for Runtime Library Routines}
\label{chap:Stubs for Runtime Library Routines}
\label{chap:Appendix A}
This section provides stubs for the runtime library routines defined in the OpenMP API.
The stubs are provided to enable portability to platforms that do not support the
OpenMP API. On these platforms, OpenMP programs must be linked with a library
containing these stub routines. The stub routines assume that the directives in the
OpenMP program are ignored. As such, they emulate serial semantics
executing on the host.

Note that the lock variable that appears in the lock routines must be accessed
exclusively through these routines. It should not be initialized or otherwise modified in
the user program.

In an actual implementation the lock variable might be used to hold the address of an
allocated memory block, but here it is used to hold an integer value. Users should not
make assumptions about mechanisms used by OpenMP implementations to implement
locks based on the scheme used by the stub procedures.

\begin{fortranspecific}

In order to be able to compile the Fortran stubs file, the include file
\code{omp_lib.h} was split into two files: \code{omp_lib_kinds.h} and \code{omp_lib.h} and the
\code{omp_lib_kinds.h} file included where needed. There is no requirement for the
implementation to provide separate files.

\end{fortranspecific}


\filbreak
% This is stubs_ccpp.tex (Appendix A) of the OpenMP specification.
% This is an included file. See the master file for more information.
%
% When editing this file:
%
%    1. To change formatting, appearance, or style, please edit openmp.sty.
%
%    2. Custom commands and macros are defined in openmp.sty.
%
%    3. Be kind to other editors -- keep a consistent style by copying-and-pasting to
%       create new content.
%
%    4. We use semantic markup, e.g. (see openmp.sty for a full list):
%         \code{}     % for bold monospace keywords, code, operators, etc.
%         \plc{}      % for italic placeholder names, grammar, etc.
%
%    5. There are environments that provide special formatting, e.g. language bars.
%       Please use them whereever appropriate.  Examples are:
%
%         \begin{fortranspecific}
%         This is text that appears enclosed in blue language bars for Fortran.
%         \end{fortranspecific}
%
%         \begin{note}
%         This is a note.  The "Note -- " header appears automatically.
%         \end{note}
%
%    6. Other recommendations:
%         Use the convenience macros defined in openmp.sty for the minor headers
%         such as Comments, Syntax, etc.
%
%         To keep items together on the same page, prefer the use of 
%         \begin{samepage}.... Avoid \parbox for text blocks as it interrupts line numbering.
%         When possible, avoid \filbreak, \pagebreak, \newpage, \clearpage unless that's
%         what you mean. Use \needspace{} cautiously for troublesome paragraphs.
%
%         Avoid absolute lengths and measures in this file; use relative units when possible.
%         Vertical space can be relative to \baselineskip or ex units. Horizontal space
%         can be relative to \linewidth or em units.
%
%         Prefer \emph{} to italicize terminology, e.g.:
%             This is a \emph{definition}, not a placeholder.
%             This is a \plc{var-name}.
%


\section{C/C++ Stub Routines}
\index{C/C++ stub routines}
\index{stub routines}
\label{sec:C/C++ Stub Routines}
{\small \begin{ompcFunction}
#include <stdio.h>
#include <stdlib.h>
#include "omp.h"

void omp_set_num_threads(int num_threads)
{
}

int omp_get_num_threads(void)
{
  return 1;
}

int omp_get_max_threads(void)
{
  return 1;
}

int omp_get_thread_num(void)
{
  return 0;
}

int omp_get_num_procs(void)
{
  return 1;
}

int omp_in_parallel(void)
{
  return 0;
}

void omp_set_dynamic(int dynamic_threads)
{
}

int omp_get_dynamic(void)
{
  return 0;
}

int omp_get_cancellation(void)
{
  return 0;
}

void omp_set_nested(int nested)
{
}

int omp_get_nested(void)
{
  return 0;
}

void omp_set_schedule(omp_sched_t kind, int chunk_size)
{
}

void omp_get_schedule(omp_sched_t *kind, int *chunk_size)
{
  *kind = omp_sched_static;
  *chunk_size = 0;
}

int omp_get_thread_limit(void)
{
  return 1;
}

void omp_set_max_active_levels(int max_active_levels)
{
}

int omp_get_max_active_levels(void)
{
  return 0;
}

int omp_get_level(void)
{
  return 0;
}

int omp_get_ancestor_thread_num(int level)
{
  if (level == 0)
  {
    return 0;
  }
  else
  {
    return -1;
  }
}

int omp_get_team_size(int level)
{
  if (level == 0)
  {
    return 1;
  }
  else
  {
    return -1;
  }
}

int omp_get_active_level(void)
{
  return 0;
}

int omp_in_final(void)
{
  return 1;
}

omp_proc_bind_t omp_get_proc_bind(void)
{
  return omp_proc_bind_false;
}

int omp_get_num_places(void)
{
  return 0;
}

int omp_get_place_num_procs(int place_num)
{
  return 0;
}

void omp_get_place_proc_ids(int place_num, int *ids)
{
}

int omp_get_place_num(void)
{
  return -1;
}

int omp_get_partition_num_places(void)
{
  return 0;
}

void omp_get_partition_place_nums(int *place_nums)
{
}

void omp_set_affinity_format(char const *format)
{
}

size_t omp_get_affinity_format(char* buffer, size_t size)
{
  return 0;
}

void omp_display_affinity(char const *format)
{
}

size_t omp_capture_affinity(char *buffer, size_t size, char const *format)
{
  return 0;
}

void omp_set_default_device(int device_num)
{
}

int omp_get_default_device(void)
{
  return 0;
}

int omp_get_num_devices(void)
{
  return 0;
}

int omp_get_device_num(void)
{
  return -10;
}

int omp_get_num_teams(void)
{
  return 1;
}

int omp_get_team_num(void)
{
  return 0;
}

int omp_is_initial_device(void)
{
  return 1;
}

int omp_get_initial_device(void)
{
  return -10;
}

int omp_get_max_task_priority(void)
{
  return 0;
}

struct __omp_lock
{
  int lock;
};

enum { UNLOCKED = -1, INIT, LOCKED };

void omp_init_lock(omp_lock_t *arg)
{
  struct __omp_lock *lock = (struct __omp_lock *)arg;
  lock->lock = UNLOCKED;
}

void omp_init_lock_with_hint(omp_lock_t *arg, omp_lock_hint_t hint)
{
  omp_init_lock(arg);
}

void omp_destroy_lock(omp_lock_t *arg)
{
  struct __omp_lock *lock = (struct __omp_lock *)arg;
  lock->lock = INIT;
}

void omp_set_lock(omp_lock_t *arg)
{
  struct __omp_lock *lock = (struct __omp_lock *)arg;
  if (lock->lock == UNLOCKED)
  {
    lock->lock = LOCKED;
  }
  else if (lock->lock == LOCKED)
  {
    fprintf(stderr, "error: deadlock in using lock variable\n");
    exit(1);
  }
  
  else
  {
    fprintf(stderr, "error: lock not initialized\n");
    exit(1);
  }
}

void omp_unset_lock(omp_lock_t *arg)
{
  struct __omp_lock *lock = (struct __omp_lock *)arg;
  if (lock->lock == LOCKED)
  {
    lock->lock = UNLOCKED;
  }
  else if (lock->lock == UNLOCKED)
  {
    fprintf(stderr, "error: lock not set\n");
    exit(1);
  }
  else
  {
    fprintf(stderr, "error: lock not initialized\n");
    exit(1);
  }
}

int omp_test_lock(omp_lock_t *arg)
{
  struct __omp_lock *lock = (struct __omp_lock *)arg;
  if (lock->lock == UNLOCKED)
  {
    lock->lock = LOCKED;
    return 1;
  }
  else if (lock->lock == LOCKED)
  {
    return 0;
  }
  else
  {
    fprintf(stderr, "error: lock not initialized\n");
    exit(1);
  }
}

struct __omp_nest_lock
{
  short owner;
  short count;
};

enum { NOOWNER = -1, MASTER = 0 };

void omp_init_nest_lock(omp_nest_lock_t *arg)
{
  struct __omp_nest_lock *nlock=(struct __omp_nest_lock *)arg;
  nlock->owner = NOOWNER;
  nlock->count = 0;
}

void omp_init_nest_lock_with_hint(
  omp_nest_lock_t *arg,
  omp_lock_hint_t hint
)
{
  omp_init_nest_lock(arg);
}

void omp_destroy_nest_lock(omp_nest_lock_t *arg)
{
  struct __omp_nest_lock *nlock=(struct __omp_nest_lock *)arg;
  nlock->owner = NOOWNER;
  nlock->count = UNLOCKED;
}

void omp_set_nest_lock(omp_nest_lock_t *arg)
{
  struct __omp_nest_lock *nlock=(struct __omp_nest_lock *)arg;
  if (nlock->owner == MASTER && nlock->count >= 1)
  {
    nlock->count++;
  }
  else if (nlock->owner == NOOWNER && nlock->count == 0)
  {
    nlock->owner = MASTER;
    nlock->count = 1;
  }
  else
  {
    fprintf(stderr, "error: lock corrupted or not initialized\n");
    exit(1);
  }
}

void omp_unset_nest_lock(omp_nest_lock_t *arg)
{
  struct __omp_nest_lock *nlock=(struct __omp_nest_lock *)arg;
  if (nlock->owner == MASTER && nlock->count >= 1)
  {
    nlock->count--;
    if (nlock->count == 0)
    {
      nlock->owner = NOOWNER;
    }
  }
  else if (nlock->owner == NOOWNER && nlock->count == 0)
  {
    fprintf(stderr, "error: lock not set\n");
    exit(1);
  }
  else
  {
    fprintf(stderr, "error: lock corrupted or not initialized\n");
    exit(1);
  }
}

int omp_test_nest_lock(omp_nest_lock_t *arg)
{
  struct __omp_nest_lock *nlock=(struct __omp_nest_lock *)arg;
  omp_set_nest_lock(arg);
  return nlock->count;
}

double omp_get_wtime(void)
{
  /* This function does not provide a working
   * wallclock timer. Replace it with a version
   * customized for the target machine.
   */
  return 0.0;
}

double omp_get_wtick(void)
{
  /* This function does not provide a working
   * clock tick function. Replace it with
   * a version customized for the target machine.
   */
  return 365. * 86400.;
}

void * omp_target_alloc(size_t size, int device_num)
{
  if (device_num != -10)
  return NULL;
  return malloc(size)
}

void omp_target_free(void *device_ptr, int device_num)
{
  free(device_ptr);
}

int omp_target_is_present(void *ptr, int device_num)
{
  return 1;
}

int omp_target_memcpy(
  void *dst,
  void *src,
  size_t length,
  size_t dst_offset,
  size_t src_offset,
  int dst_device,
  int src_device
)
{
  // only the default device is valid in a stub
  if (
    dst_device != -10 || src_device != -10
    || ! dst || ! src )
  return EINVAL;
  memcpy((char *)dst + dst_offset,
    (char *)src + src_offset,
    length);
  return 0;
}

int omp_target_memcpy_rect(
  void *dst,
  void *src,
  size_t element_size,
  int num_dims,
  const size_t *volume,
  const size_t *dst_offsets,
  const size_t *src_offsets,
  const size_t *dst_dimensions,
  const size_t *src_dimensions,
  int dst_device_num,
  int src_device_num
)
{
  int ret=0;
  // Both null, return number of dimensions supported,
  // this stub supports an arbitrary number
  if (dst == NULL && src == NULL) return INT_MAX;
  
  if (
    !volume || !dst_offsets || !src_offsets
    || !dst_dimensions || !src_dimensions
    || num_dims < 1 ) {
    ret = EINVAL;
    goto done;
  }
  if (num_dims == 1) {
    ret = omp_target_memcpy(
      dst,
      src,
      element_size * volume[0],
      dst_offsets[0] * element_size,
      src_offsets[0] * element_size,
      dst_device_num,
      src_device_num
    );
    if(ret) goto done;
  } else {
    size_t dst_slice_size = element_size;
    size_t src_slice_size = element_size;
    for (int i=1; i < num_dims; i++) {
      dst_slice_size *= dst_dimensions[i];
      src_slice_size *= src_dimensions[i];
    }
    size_t dst_off = dst_offsets[0] * dst_slice_size;
    size_t src_off = src_offsets[0] * src_slice_size;
    for (size_t i=0; i < volume[0]; i++) {
      ret = omp_target_memcpy_rect(
        (char *)dst + dst_off + dst_slice_size*i,
        (char *)src + src_off + src_slice_size*i,
        element_size,
        num_dims - 1,
        volume + 1,
        dst_offsets + 1,
        src_offsets + 1,
        dst_dimensions + 1,
        src_dimensions + 1,
        dst_device_num,
        src_device_num);
      if (ret) goto done;
    }
  }
  done:
  return ret;
}

int omp_target_associate_ptr(
  void *host_ptr,
  void *device_ptr,
  size_t size,
  size_t device_offset,
  int device_num
)
{
  // No association is possible because all host pointers
  // are considered present
  return EINVAL;
}

int omp_target_disassociate_ptr(void *ptr, int device_num)
{
  return EINVAL;
}


int omp_control_tool(int command, int modifier, void *arg)
{
  return omp_control_tool_notool;
}

static omp_allocator_t * omp_allocator = OMP_NULL_ALLOCATOR;

omp_allocator_t * omp_default_mem_alloc;
omp_allocator_t * omp_large_cap_mem_alloc;
omp_allocator_t * omp_const_mem_alloc;
omp_allocator_t * omp_high_bw_mem_alloc;
omp_allocator_t * omp_low_lat_mem_alloc;
omp_allocator_t * omp_cgroup_mem_alloc;
omp_allocator_t * omp_pteam_mem_alloc;
omp_allocator_t * omp_thread_mem_alloc;

void omp_set_default allocator(omp_allocator_t *allocator)
{
  omp_allocator = allocator;
}

omp_allocator_t * omp_get_default_allocator (void)
{
  return omp_allocator;
}

#ifdef __cplusplus
void * omp_alloc (size_t size, omp_allocator_t *allocator = OMP_NULL_ALLOCATOR)
#else
void * omp_alloc (size_t size, omp_allocator_t *allocator)
#endif
{
  return malloc(size);
}

#ifdef __cplusplus
void omp_free (void * ptr, omp_allocator_t *allocator = OMP_NULL_ALLOCATOR)
#else
void omp_free (void * ptr, omp_allocator_t *allocator)
#endif
{
  free(ptr);
}

\end{ompcFunction}} % end \small block



%\pagebreak
% This is stubs_fortran.tex (Appendix A) of the OpenMP specification.
% This is an included file. See the master file for more information.
%
% When editing this file:
%
%    1. To change formatting, appearance, or style, please edit openmp.sty.
%
%    2. Custom commands and macros are defined in openmp.sty.
%
%    3. Be kind to other editors -- keep a consistent style by copying-and-pasting to
%       create new content.
%
%    4. We use semantic markup, e.g. (see openmp.sty for a full list):
%         \code{}     % for bold monospace keywords, code, operators, etc.
%         \plc{}      % for italic placeholder names, grammar, etc.
%
%    5. There are environments that provide special formatting, e.g. language bars.
%       Please use them whereever appropriate.  Examples are:
%
%         \begin{fortranspecific}
%         This is text that appears enclosed in blue language bars for Fortran.
%         \end{fortranspecific}
%
%         \begin{note}
%         This is a note.  The "Note -- " header appears automatically.
%         \end{note}
%
%    6. Other recommendations:
%         Use the convenience macros defined in openmp.sty for the minor headers
%         such as Comments, Syntax, etc.
%
%         To keep items together on the same page, prefer the use of
%         \begin{samepage}.... Avoid \parbox for text blocks as it interrupts line numbering.
%         When possible, avoid \filbreak, \pagebreak, \newpage, \clearpage unless that's
%         what you mean. Use \needspace{} cautiously for troublesome paragraphs.
%
%         Avoid absolute lengths and measures in this file; use relative units when possible.
%         Vertical space can be relative to \baselineskip or ex units. Horizontal space
%         can be relative to \linewidth or em units.
%
%         Prefer \emph{} to italicize terminology, e.g.:
%             This is a \emph{definition}, not a placeholder.
%             This is a \plc{var-name}.
%


\section{Fortran Stub Routines}
\label{sec:Fortran Stub Routines}
{\small \begin{ompfFunction}
subroutine omp_set_num_threads(num_threads)
  integer num_threads
  return
end subroutine

integer function omp_get_num_threads()
  omp_get_num_threads = 1
  return
end function

integer function omp_get_max_threads()
  omp_get_max_threads = 1
  return
end function

integer function omp_get_thread_num()
  omp_get_thread_num = 0
  return
end function

integer function omp_get_num_procs()
  omp_get_num_procs = 1
  return
end function

logical function omp_in_parallel()
  omp_in_parallel = .false.
  return
end function

subroutine omp_set_dynamic(dynamic_threads)
  logical dynamic_threads
  return
end subroutine

logical function omp_get_dynamic()
  omp_get_dynamic = .false.
  return
end function

logical function omp_get_cancellation()
  omp_get_cancellation = .false.
  return
end function

subroutine omp_set_nested(nested)
  logical nested
  return
end subroutine

logical function omp_get_nested()
  omp_get_nested = .false.
  return
end function

subroutine omp_set_schedule(kind, chunk_size)
  include 'omp_lib_kinds.h'
  integer (kind=omp_sched_kind) kind
  integer chunk_size
  return
end subroutine

subroutine omp_get_schedule(kind, chunk_size)
  include 'omp_lib_kinds.h'
  integer (kind=omp_sched_kind) kind
  integer chunk_size
  kind = omp_sched_static
  chunk_size = 0
  return
end subroutine

integer function omp_get_thread_limit()
  omp_get_thread_limit = 1
  return
end function

subroutine omp_set_max_active_levels(max_level)
  integer max_level
end subroutine

integer function omp_get_max_active_levels()
  omp_get_max_active_levels = 0
  return
end function

integer function omp_get_level()
  omp_get_level = 0
  return
end function

integer function omp_get_ancestor_thread_num(level)
  integer level
  if ( level .eq. 0 ) then
     omp_get_ancestor_thread_num = 0
  else
     omp_get_ancestor_thread_num = -1
  end if
  return
end function

integer function omp_get_team_size(level)
  integer level
  if ( level .eq. 0 ) then
     omp_get_team_size = 1
  else
     omp_get_team_size = -1
  end if
  return
end function

integer function omp_get_active_level()
  omp_get_active_level = 0
  return
end function

logical function omp_in_final()
  omp_in_final = .true.
  return
end function

function omp_get_proc_bind()
  include 'omp_lib_kinds.h'
  integer (kind=omp_proc_bind_kind) omp_get_proc_bind
  omp_get_proc_bind = omp_proc_bind_false
end function

integer function omp_get_num_places()
  return 0
end function

integer function omp_get_place_num_procs(place_num)
  integer place_num
  return 0
end function

subroutine omp_get_place_proc_ids(place_num, ids)
  integer place_num
  integer ids(*)
  return
end subroutine

integer function omp_get_place_num()
  return -1
end function

integer function omp_get_partition_num_places()
  return 0
end function

subroutine omp_get_partition_place_nums(place_nums)
  integer place_nums(*)
  return
end subroutine


subroutine omp_set_affinity_format(\plc{format})
   character(len=*),intent(in)::format
   return
end subroutine

integer function omp_get_affinity_format(buffer)
   character(len=*),intent(out)::buffer
   return 0
end function

subroutine omp_display_affinity(format)
   character(len=*),intent(in)::format
   return
end subroutine

integer function omp_capture_affinity(buffer,format)
   character(len=*),intent(out)::buffer
   character(len=*),intent(in)::format
   return 0
end function

subroutine omp_set_default_device(device_num)
  integer device_num
  return
end subroutine

integer function omp_get_default_device()
  omp_get_default_device = 0
  return
end function

integer function omp_get_num_devices()
  omp_get_num_devices = 0
  return
end function

integer function omp_get_device_num()
  omp_get_device_num = -10
  return
end function

integer function omp_get_num_teams()
  omp_get_num_teams = 1
  return
end function

integer function omp_get_team_num()
  omp_get_team_num = 0
  return
end function

logical function omp_is_initial_device()
  omp_is_initial_device = .true.
  return
end function

integer function omp_get_initial_device()
  omp_get_initial_device = -10
  return
end function

integer function omp_get_max_task_priority()
  omp_get_max_task_priority = 0
  return
end function

subroutine omp_init_lock(lock)
  ! lock is 0 if the simple lock is not initialized
  !        -1 if the simple lock is initialized but not set
  !         1 if the simple lock is set
  include 'omp_lib_kinds.h'
  integer(kind=omp_lock_kind) lock

  lock = -1
  return
end subroutine

subroutine omp_init_lock_with_hint(lock, hint)
  include 'omp_lib_kinds.h'
  integer(kind=omp_lock_kind) lock
  integer(kind=omp_lock_hint_kind) hint

  call omp_init_lock(lock)
  return
end subroutine

subroutine omp_destroy_lock(lock)
  include 'omp_lib_kinds.h'
  integer(kind=omp_lock_kind) lock

  lock = 0
  return
end subroutine

subroutine omp_set_lock(lock)
  include 'omp_lib_kinds.h'
  integer(kind=omp_lock_kind) lock

  if (lock .eq. -1) then
    lock = 1
  elseif (lock .eq. 1) then
    print *, 'error: deadlock in using lock variable'
    stop
  else
    print *, 'error: lock not initialized'
    stop
  endif
  return
end subroutine

subroutine omp_unset_lock(lock)
  include 'omp_lib_kinds.h'
  integer(kind=omp_lock_kind) lock

  if (lock .eq. 1) then
    lock = -1
  elseif (lock .eq. -1) then
    print *, 'error: lock not set'
    stop
  else
    print *, 'error: lock not initialized'
    stop
  endif
  return
end subroutine

logical function omp_test_lock(lock)
  include 'omp_lib_kinds.h'
  integer(kind=omp_lock_kind) lock

  if (lock .eq. -1) then
    lock = 1
    omp_test_lock = .true.
  elseif (lock .eq. 1) then
    omp_test_lock = .false.
  else
    print *, 'error: lock not initialized'
    stop
  endif

  return
end function

subroutine omp_init_nest_lock(nlock)
  ! nlock is
  ! 0 if the nestable lock is not initialized
  ! -1 if the nestable lock is initialized but not set
  ! 1 if the nestable lock is set
  ! no use count is maintained
  include 'omp_lib_kinds.h'
  integer(kind=omp_nest_lock_kind) nlock

  nlock = -1

  return
end subroutine

subroutine omp_init_nest_lock_with_hint(nlock, hint)
  include 'omp_lib_kinds.h'
  integer(kind=omp_nest_lock_kind) nlock
  integer(kind=omp_lock_hint_kind) hint

  call omp_init_nest_lock(nlock)
  return
end subroutine

subroutine omp_destroy_nest_lock(nlock)
  include 'omp_lib_kinds.h'
  integer(kind=omp_nest_lock_kind) nlock

  nlock = 0

  return
end subroutine

subroutine omp_set_nest_lock(nlock)
  include 'omp_lib_kinds.h'
  integer(kind=omp_nest_lock_kind) nlock

  if (nlock .eq. -1) then
    nlock = 1
  elseif (nlock .eq. 0) then
    print *, 'error: nested lock not initialized'
    stop
  else
    print *, 'error: deadlock using nested lock variable'
    stop
  endif

  return
end subroutine

subroutine omp_unset_nest_lock(nlock)
  include 'omp_lib_kinds.h'
  integer(kind=omp_nest_lock_kind) nlock

  if (nlock .eq. 1) then
    nlock = -1
  elseif (nlock .eq. 0) then
    print *, 'error: nested lock not initialized'
    stop
  else
    print *, 'error: nested lock not set'
    stop
  endif

  return
end subroutine

integer function omp_test_nest_lock(nlock)
  include 'omp_lib_kinds.h'
  integer(kind=omp_nest_lock_kind) nlock

  if (nlock .eq. -1) then
    nlock = 1
    omp_test_nest_lock = 1
  elseif (nlock .eq. 1) then
    omp_test_nest_lock = 0
  else
    print *, 'error: nested lock not initialized'
    stop
  endif

  return
end function

double precision function omp_get_wtime()
  ! this function does not provide a working
  ! wall clock timer. replace it with a version
  ! customized for the target machine.

  omp_get_wtime = 0.0d0

  return
end function

double precision function omp_get_wtick()
  ! this function does not provide a working
  ! clock tick function. replace it with
  ! a version customized for the target machine.
  double precision one_year
  parameter (one_year=365.d0*86400.d0)

  omp_get_wtick = one_year

  return
end function

int function omp_control_tool(command, modifier)
  include 'omp_lib_kinds.h'
  integer (kind=omp_control_tool_kind) command
  integer (kind=omp_control_tool_kind) modifier

  return omp_control_tool_notool
end function

subroutine omp_set_default_allocator(allocator)
  include 'omp_lib_kinds.h'
  integer (kind=omp_allocator_kind) allocator
  return
end subroutine

function omp_get_default_allocator
  include 'omp_lib_kinds.h'
  integer (kind=omp_allocator_kind) omp_get_default_allocator
  omp_get_default_allocator = omp_null_allocator
end function

\end{ompfFunction}} % end \small block

