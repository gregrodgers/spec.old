% This is appendix-A-stubs.tex of the OpenMP specification.
% This is an included file. See the master file for more information.
%
% When editing this file:
%
%    1. To change formatting, appearance, or style, please edit openmp.sty.
%
%    2. Custom commands and macros are defined in openmp.sty.
%
%    3. Be kind to other editors -- keep a consistent style by copying-and-pasting to
%       create new content.
%
%    4. We use semantic markup, e.g. (see openmp.sty for a full list):
%         \code{}     % for bold monospace keywords, code, operators, etc.
%         \plc{}      % for italic placeholder names, grammar, etc.
%
%    5. Other recommendations:
%         Use the convenience macros defined in openmp.sty for the minor headers
%         such as Comments, Syntax, etc.
%
%         To keep items together on the same page, prefer the use of 
%         \begin{samepage}.... Avoid \parbox for text blocks as it interrupts line numbering.
%         When possible, avoid \filbreak, \pagebreak, \newpage, \clearpage unless that's
%         what you mean. Use \needspace{} cautiously for troublesome paragraphs.
%
%         Avoid absolute lengths and measures in this file; use relative units when possible.
%         Vertical space can be relative to \baselineskip or ex units. Horizontal space
%         can be relative to \linewidth or em units.
%
%         Prefer \emph{} to italicize terminology, e.g.:
%             This is a \emph{definition}, not a placeholder.
%             This is a \plc{var-name}.
%


\chapter{Stubs for Runtime Library Routines}
\label{chap:Stubs for Runtime Library Routines}
\label{chap:Appendix A}
This section provides stubs for the runtime library routines defined in the OpenMP API. 
The stubs are provided to enable portability to platforms that do not support the 
OpenMP API. On these platforms, OpenMP programs must be linked with a library 
containing these stub routines. The stub routines assume that the directives in the 
OpenMP program are ignored. As such, they emulate serial semantics.

Note that the lock variable that appears in the lock routines must be accessed 
exclusively through these routines. It should not be initialized or otherwise modified in 
the user program. 

In an actual implementation the lock variable might be used to hold the address of an 
allocated memory block, but here it is used to hold an integer value. Users should not 
make assumptions about mechanisms used by OpenMP implementations to implement 
locks based on the scheme used by the stub procedures.

\fortranspecificstart

\noteheader – In order to be able to compile the Fortran stubs file, the include file 
\code{omp\_lib.h} was split into two files: \code{omp\_lib\_kinds.h} and \code{omp\_lib.h} and the 
\code{omp\_lib\_kinds.h} file included where needed. There is no requirement for the 
implementation to provide separate files.

\fortranspecificend







\filbreak
\section{C/C++ Stub Routines}
\index{C/C++ stub routines}
\index{stub routines}
\label{sec:C/C++ Stub Routines}
{\small \begin{codepar}
\#include <stdio.h>
\#include <stdlib.h>
\#include "omp.h"

void omp\_set\_num\_threads(int num\_threads)
\{
\}

int omp\_get\_num\_threads(void)
\{
    return 1;
\}

int omp\_get\_max\_threads(void)
\{
    return 1;
\}

int omp\_get\_thread\_num(void)
\{
    return 0;
\}

int omp\_get\_num\_procs(void)
\{
    return 1;
\}

int omp\_in\_parallel(void)
\{
    return 0;
\}

void omp\_set\_dynamic(int dynamic\_threads)
\{
\}

int omp\_get\_dynamic(void)
\{
    return 0;
\}

int omp\_get\_cancellation(void)
\{
    return 0;
\}

void omp\_set\_nested(int nested)
\{
\}

int omp\_get\_nested(void)
\{
    return 0;
\}

void omp\_set\_schedule(omp\_sched\_t kind, int chunk\_size)
\{
\}

void omp\_get\_schedule(omp\_sched\_t *kind, int *chunk\_size)
\{
    *kind = omp\_sched\_static;
    *chunk\_size = 0;
\}

int omp\_get\_thread\_limit(void)
\{
    return 1;
\}

void omp\_set\_max\_active\_levels(int max\_active\_levels)
\{
\}

int omp\_get\_max\_active\_levels(void)
\{
    return 0;
\}

int omp\_get\_level(void)
\{
    return 0;
\}

int omp\_get\_ancestor\_thread\_num(int level)
\{
    if (level == 0)
    \{
        return 0;
    \}
    else
    \{
        return -1;
    \}
\}

int omp\_get\_team\_size(int level)
\{
    if (level == 0)
    \{
        return 1;
    \}
    else
    \{
        return -1;
    \}
\}

int omp\_get\_active\_level(void)
\{
    return 0;
\}

int omp\_in\_final(void)
\{
    return 1;
\}

omp\_proc\_bind\_t omp\_get\_proc\_bind(void) 
\{
    return omp\_proc\_bind\_false;
\}

void omp\_set\_default\_device(int device\_num)
\{
\}

int omp\_get\_default\_device(void)
\{
    return 0;
\}

int omp\_get\_num\_devices(void)
\{
    return 0;
\}

int omp\_get\_num\_teams(void)
\{
    return 1;
\}

int omp\_get\_team\_num(void)
\{
    return 0;
\}

int omp\_is\_initial\_device(void)
\{
    return 1;
\}

int omp\_get\_max\_task\_priority(void)
\{
    return 0;
\}

struct \_\_omp\_lock
\{
    int lock;
\};

enum \{ UNLOCKED = -1, INIT, LOCKED \};

void omp\_init\_lock(omp\_lock\_t *arg)
\{
    struct \_\_omp\_lock *lock = (struct \_\_omp\_lock *)arg;
    lock->lock = UNLOCKED;
\}

void omp\_destroy\_lock(omp\_lock\_t *arg)
\{
    struct \_\_omp\_lock *lock = (struct \_\_omp\_lock *)arg;
    lock->lock = INIT;
\}

void omp\_set\_lock(omp\_lock\_t *arg)
\{
    struct \_\_omp\_lock *lock = (struct \_\_omp\_lock *)arg;
    if (lock->lock == UNLOCKED)
    \{
        lock->lock = LOCKED;
    \}
    else if (lock->lock == LOCKED)
    \{
        fprintf(stderr, "error: deadlock in using lock variable{\textbackslash}n");
        exit(1);
    \}

    else
    \{
        fprintf(stderr, "error: lock not initialized{\textbackslash}n");
        exit(1);
    \}
\}

void omp\_unset\_lock(omp\_lock\_t *arg)
\{
struct \_\_omp\_lock *lock = (struct \_\_omp\_lock *)arg;
    if (lock->lock == LOCKED)
    \{
        lock->lock = UNLOCKED;
    \}
    else if (lock->lock == UNLOCKED)
    \{
        fprintf(stderr, "error: lock not set{\textbackslash}n");
        exit(1);
    \}
    else
    \{
        fprintf(stderr, "error: lock not initialized{\textbackslash}n");
        exit(1);
    \}
\}

int omp\_test\_lock(omp\_lock\_t *arg)
\{
struct \_\_omp\_lock *lock = (struct \_\_omp\_lock *)arg;
    if (lock->lock == UNLOCKED)
    \{
        lock->lock = LOCKED;
        return 1;
    \}
    else if (lock->lock == LOCKED)
    \{
        return 0;
    \}
    else
    \{
        fprintf(stderr, "error: lock not initialized{\textbackslash} n");
        exit(1);
    \}
\}

struct \_\_omp\_nest\_lock
\{
    short owner;
    short count;
\};

enum \{ NOOWNER = -1, MASTER = 0 \};

void omp\_init\_nest\_lock(omp\_nest\_lock\_t *arg)
\{
    struct \_\_omp\_nest\_lock *nlock=(struct \_\_omp\_nest\_lock *)arg;
    nlock->owner = NOOWNER;
    nlock->count = 0;
\}

void omp\_destroy\_nest\_lock(omp\_nest\_lock\_t *arg)
\{
    struct \_\_omp\_nest\_lock *nlock=(struct \_\_omp\_nest\_lock *)arg;
    nlock->owner = NOOWNER;
    nlock->count = UNLOCKED;
\}

void omp\_set\_nest\_lock(omp\_nest\_lock\_t *arg)
\{
    struct \_\_omp\_nest\_lock *nlock=(struct \_\_omp\_nest\_lock *)arg;
    if (nlock->owner == MASTER \&\& nlock->count >= 1)
    \{
        nlock->count++;
    \}
    else if (nlock->owner == NOOWNER \&\& nlock->count == 0)
    \{
        nlock->owner = MASTER;
        nlock->count = 1;
    \}
    else
    \{
        fprintf(stderr, "error: lock corrupted or not initialized{\textbackslash}n");
        exit(1);
    \}
\}

void omp\_unset\_nest\_lock(omp\_nest\_lock\_t *arg)
\{
    struct \_\_omp\_nest\_lock *nlock=(struct \_\_omp\_nest\_lock *)arg;
    if (nlock->owner == MASTER \&\& nlock->count >= 1)
    \{
        nlock->count--;
        if (nlock->count == 0)
        \{
            nlock->owner = NOOWNER;
        \}
    \}
    else if (nlock->owner == NOOWNER \&\& nlock->count == 0)
    \{
        fprintf(stderr, "error: lock not set{\textbackslash}n");
        exit(1);
    \}
    else
    \{
        fprintf(stderr, "error: lock corrupted or not initialized{\textbackslash}n");
        exit(1);
    \}
\}

int omp\_test\_nest\_lock(omp\_nest\_lock\_t *arg)
\{
    struct \_\_omp\_nest\_lock *nlock=(struct \_\_omp\_nest\_lock *)arg;
    omp\_set\_nest\_lock(arg);
    return nlock->count;
\}

double omp\_get\_wtime(void)
\{
/* This function does not provide a working
 * wallclock timer. Replace it with a version
 * customized for the target machine.
 */
    return 0.0;
\}

double omp\_get\_wtick(void)
\{
/* This function does not provide a working
 * clock tick function. Replace it with
 * a version customized for the target machine.
 */
    return 365. * 86400.;
\}

\end{codepar}} % end \small block








\pagebreak
\section{Fortran Stub Routines}
\label{sec:Fortran Stub Routines}
{\small \begin{codepar}
subroutine omp\_set\_num\_threads(num\_threads)
  integer num\_threads
  return
end subroutine

integer function omp\_get\_num\_threads()
  omp\_get\_num\_threads = 1
  return
end function

integer function omp\_get\_max\_threads()
  omp\_get\_max\_threads = 1
  return
end function

integer function omp\_get\_thread\_num()
  omp\_get\_thread\_num = 0
  return
end function

integer function omp\_get\_num\_procs()
  omp\_get\_num\_procs = 1
  return
end function

logical function omp\_in\_parallel()
  omp\_in\_parallel = .false.
  return
end function

subroutine omp\_set\_dynamic(dynamic\_threads)
  logical dynamic\_threads
  return
end subroutine

logical function omp\_get\_dynamic()
  omp\_get\_dynamic = .false.
  return
end function

logical function omp\_get\_cancellation()
  omp\_get\_cancellation = .false.
  return
end function

subroutine omp\_set\_nested(nested)
  logical nested
  return
end subroutine

logical function omp\_get\_nested()
  omp\_get\_nested = .false.
  return
end function

subroutine omp\_set\_schedule(kind, chunk\_size)
  include 'omp\_lib\_kinds.h'
  integer (kind=omp\_sched\_kind) kind
  integer chunk\_size
  return
end subroutine

subroutine omp\_get\_schedule(kind, chunk\_size)
  include 'omp\_lib\_kinds.h'
  integer (kind=omp\_sched\_kind) kind
  integer chunk\_size
  kind = omp\_sched\_static
  chunk\_size = 0
  return
end subroutine

integer function omp\_get\_thread\_limit()
  omp\_get\_thread\_limit = 1
  return
end function

subroutine omp\_set\_max\_active\_levels( level )
  integer level
  end subroutine
  integer function omp\_get\_max\_active\_levels()
  omp\_get\_max\_active\_levels = 0
  return
end function

integer function omp\_get\_level()
  omp\_get\_level = 0
  return
end function

integer function omp\_get\_ancestor\_thread\_num( level )
  integer level
  if ( level .eq. 0 ) then
     omp\_get\_ancestor\_thread\_num = 0
  else
     omp\_get\_ancestor\_thread\_num = -1
  end if
  return
end function

integer function omp\_get\_team\_size( level )
  integer level
  if ( level .eq. 0 ) then
     omp\_get\_team\_size = 1
  else
     omp\_get\_team\_size = -1
  end if
  return
end function

integer function omp\_get\_active\_level()
  omp\_get\_active\_level = 0
  return
end function

logical function omp\_in\_final()
  omp\_in\_final = .true.
  return
end function

function omp\_get\_proc\_bind()
  include 'omp\_lib\_kinds.h'
  integer (kind=omp\_proc\_bind\_kind) omp\_get\_proc\_bind
  omp\_get\_proc\_bind = omp\_proc\_bind\_false
end function omp\_get\_proc\_bind

subroutine omp\_set\_default\_device(device\_num)
  integer device\_num
  return
end subroutine

integer function omp\_get\_default\_device()
  omp\_get\_default\_device = 0
  return
end function

integer function omp\_get\_num\_devices()
  omp\_get\_num\_devices = 0
  return
end function

integer function omp\_get\_num\_teams()
  omp\_get\_num\_teams = 1
  return
end function

integer function omp\_get\_team\_num()
  omp\_get\_team\_num = 0
  return
end function

logical function omp\_is\_initial\_device()
  omp\_is\_initial\_device = .true.
  return
end function

integer function omp\_get\_max\_task\_priority()
  omp\_get\_max\_task\_priority = 0
  return
end function

subroutine omp\_init\_lock(lock)
  ! lock is 0 if the simple lock is not initialized
  !        -1 if the simple lock is initialized but not set
  !         1 if the simple lock is set
  include 'omp\_lib\_kinds.h'
  integer(kind=omp\_lock\_kind) lock

  lock = -1
  return
end subroutine

subroutine omp\_destroy\_lock(lock)
  include 'omp\_lib\_kinds.h'
  integer(kind=omp\_lock\_kind) lock

  lock = 0
  return
end subroutine

subroutine omp\_set\_lock(lock)
  include 'omp\_lib\_kinds.h'
  integer(kind=omp\_lock\_kind) lock

  if (lock .eq. -1) then
    lock = 1
  elseif (lock .eq. 1) then
    print *, 'error: deadlock in using lock variable'
    stop
  else
    print *, 'error: lock not initialized'
    stop
  endif
  return
end subroutine

subroutine omp\_unset\_lock(lock)
  include 'omp\_lib\_kinds.h'
  integer(kind=omp\_lock\_kind) lock

  if (lock .eq. 1) then
    lock = -1
  elseif (lock .eq. -1) then
    print *, 'error: lock not set'
    stop
  else
    print *, 'error: lock not initialized'
    stop
  endif
  return
end subroutine

logical function omp\_test\_lock(lock)
  include 'omp\_lib\_kinds.h'
  integer(kind=omp\_lock\_kind) lock

  if (lock .eq. -1) then
    lock = 1
    omp\_test\_lock = .true.
  elseif (lock .eq. 1) then
    omp\_test\_lock = .false.
  else
    print *, 'error: lock not initialized'
    stop
  endif

  return
end function

subroutine omp\_init\_nest\_lock(nlock)
  ! nlock is 
  ! 0 if the nestable lock is not initialized
  ! -1 if the nestable lock is initialized but not set
  ! 1 if the nestable lock is set
  ! no use count is maintained
  include 'omp\_lib\_kinds.h'
  integer(kind=omp\_nest\_lock\_kind) nlock

  nlock = -1

  return
end subroutine

subroutine omp\_destroy\_nest\_lock(nlock)
  include 'omp\_lib\_kinds.h'
  integer(kind=omp\_nest\_lock\_kind) nlock

  nlock = 0

  return
end subroutine

subroutine omp\_set\_nest\_lock(nlock)
  include 'omp\_lib\_kinds.h'
  integer(kind=omp\_nest\_lock\_kind) nlock

  if (nlock .eq. -1) then
    nlock = 1
  elseif (nlock .eq. 0) then
    print *, 'error: nested lock not initialized'
    stop
  else
    print *, 'error: deadlock using nested lock variable'
    stop
  endif

  return
end subroutine

subroutine omp\_unset\_nest\_lock(nlock)
  include 'omp\_lib\_kinds.h'
  integer(kind=omp\_nest\_lock\_kind) nlock

  if (nlock .eq. 1) then
    nlock = -1
  elseif (nlock .eq. 0) then
    print *, 'error: nested lock not initialized'
    stop
  else
    print *, 'error: nested lock not set'
    stop
  endif

  return
end subroutine

integer function omp\_test\_nest\_lock(nlock)
  include 'omp\_lib\_kinds.h'
  integer(kind=omp\_nest\_lock\_kind) nlock

  if (nlock .eq. -1) then
    nlock = 1
    omp\_test\_nest\_lock = 1
  elseif (nlock .eq. 1) then
    omp\_test\_nest\_lock = 0
  else
    print *, 'error: nested lock not initialized'
    stop
  endif

  return
end function

double precision function omp\_get\_wtime()
  ! this function does not provide a working
  ! wall clock timer. replace it with a version
  ! customized for the target machine.

  omp\_get\_wtime = 0.0d0

  return
end function

double precision function omp\_get\_wtick()
  ! this function does not provide a working
  ! clock tick function. replace it with
  ! a version customized for the target machine.
  double precision one\_year
  parameter (one\_year=365.d0*86400.d0)

  omp\_get\_wtick = one\_year

  return
end function
\end{codepar}} % end \small block

% This is the end of appendix-A-stubs.tex

