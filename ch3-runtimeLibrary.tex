% This is ch3-runtimeLibrary.tex of the OpenMP specification.
% This is an included file. See the master file for more information.
%
% When editing this file:
%
%    1. To change formatting, appearance, or style, please edit openmp.sty.
%
%    2. Custom commands and macros are defined in openmp.sty.
%
%    3. Be kind to other editors -- keep a consistent style by copying-and-pasting to
%       create new content.
%
%    4. We use semantic markup, e.g. (see openmp.sty for a full list):
%         \code{}     % for bold monospace keywords, code, operators, etc.
%         \plc{}      % for italic placeholder names, grammar, etc.
%
%    5. Other recommendations:
%         Use the convenience macros defined in openmp.sty for the minor headers
%         such as Comments, Syntax, etc.
%
%         To keep items together on the same page, prefer the use of 
%         \begin{samepage}.... Avoid \parbox for text blocks as it interrupts line numbering.
%         When possible, avoid \filbreak, \pagebreak, \newpage, \clearpage unless that's
%         what you mean. Use \needspace{} cautiously for troublesome paragraphs.
%
%         Avoid absolute lengths and measures in this file; use relative units when possible.
%         Vertical space can be relative to \baselineskip or ex units. Horizontal space
%         can be relative to \linewidth or em units.
%
%         Prefer \emph{} to italicize terminology, e.g.:
%             This is a \emph{definition}, not a placeholder.
%             This is a \plc{var-name}.
%

\chapter{Runtime Library Routines}
\index{runtime library routines}
\label{chap:Runtime Library Routines}
This chapter describes the OpenMP API runtime library routines and is divided into the 
following sections:

\begin{itemize}
\item Runtime library definitions 
(\specref{sec:runtime library definitions}).

\item Execution environment routines that can be used to control and to query the parallel 
execution environment 
(\specref{sec:Execution Environment Routines}).

\item Lock routines that can be used to synchronize access to data 
(\specref{sec:Lock Routines}). 

\item Portable timer routines 
(\specref{sec:Timing Routines}).
\end{itemize}

Throughout this chapter, \plc{true} and \plc{false} are used as generic terms to simplify the 
description of the routines. 

\begin{samepage}
\ccppspecificstart
\plc{true} means a nonzero integer value and \plc{false} means an integer value of zero. 
\ccppspecificend
\end{samepage}
\bigskip

\begin{samepage}
\fortranspecificstart
\plc{true} means a logical value of \code{.TRUE.} and \plc{false} means a logical value of \code{.FALSE.}.
\fortranspecificend
\end{samepage}
\bigskip

\begin{samepage}
\vspace{1\baselineskip}
\fortranspecificstart
\vspace{-1\baselineskip}
\restrictions

The following restriction applies to all OpenMP runtime library routines:

\begin{itemize}
\item OpenMP runtime library routines may not be called from \code{PURE} or \code{ELEMENTAL} 
procedures. 
\end{itemize}
\fortranspecificend
\end{samepage}











\section{Runtime Library Definitions}
\index{runtime library definitions}
\index{header files}
\index{include files}
\label{sec:runtime library definitions}
For each base language, a compliant implementation must supply a set of definitions for 
the OpenMP API runtime library routines and the special data types of their parameters. 
The set of definitions must contain a declaration for each OpenMP API runtime library 
routine and a declaration for the \emph{simple lock}, \emph{nestable lock}, \emph{schedule}, and \emph{thread affinity
policy} data types. In addition, each set of definitions may specify other implementation 
specific values.

\ccppspecificstart
The library routines are external functions with ``C'' linkage.

Prototypes for the C/C++ runtime library routines described in this chapter shall be 
provided in a header file named \code{omp.h}. This file defines the following: 

\begin{itemize}
\item The prototypes of all the routines in the chapter. 

\item The type \code{omp\_lock\_t}. 

\item The type \code{omp\_nest\_lock\_t}.

\item The type \code{omp\_lock\_hint\_t}. 

\item The type \code{omp\_sched\_t}.

\item The type \code{omp\_proc\_bind\_t}.
\end{itemize}

% was: C.1 on page 288 
See Section \specref{sec:Example of the omp.h Header File} for an example of this file.
\ccppspecificend

\fortranspecificstart
The OpenMP Fortran API runtime library routines are external procedures. The return 
values of these routines are of default kind, unless otherwise specified.

Interface declarations for the OpenMP Fortran runtime library routines described in this 
chapter shall be provided in the form of a Fortran \code{include} file named \code{omp\_lib.h} or 
a Fortran~90 \code{module} named \code{omp\_lib}. It is implementation defined whether the 
\code{include} file or the \code{module} file (or both) is provided.

These files define the following:

\begin{itemize}
\item The interfaces of all of the routines in this chapter.

\item The \code{integer} \code{parameter} \code{omp\_lock\_kind}.

\item The \code{integer} \code{parameter} \code{omp\_nest\_lock\_kind}.

\item The \code{integer} \code{parameter} \code{omp\_lock\_hint\_kind}.

\item The \code{integer} \code{parameter} \code{omp\_sched\_kind}.

\item The \code{integer} \code{parameter} \code{omp\_proc\_bind\_kind}.

\item The \code{integer} \code{parameter} \code{openmp\_version} with a value \plc{yyyymm} where \plc{yyyy} 
and \plc{mm} are the year and month designations of the version of the OpenMP Fortran 
API that the implementation supports. This value matches that of the C preprocessor 
macro \code{\_OPENMP}, when a macro preprocessor is supported (see 
\specref{sec:Conditional Compilation}).
\end{itemize}

See \specref{sec:Example of an Interface Declaration include File} 
and \specref{sec:Example of a Fortran Interface Declaration module} 
for examples of these files.

It is implementation defined whether any of the OpenMP runtime library routines that 
take an argument are extended with a generic interface so arguments of different \code{KIND} 
type can be accommodated. See Appendix~\ref{sec:Example of a Generic Interface for a Library Routine}
for an example of such an extension. 
\fortranspecificend









\section{Execution Environment Routines}
\index{execution environment routines}
\label{sec:Execution Environment Routines}
This section describes routines that affect and monitor threads, processors, and the 
parallel environment. 






\subsection{\code{omp\_set\_num\_threads}}
\index{omp\_set\_num\_threads@{\code{omp\_set\_num\_threads}}}
\label{subsec:omp_set_num_threads}
\summary
The \code{omp\_set\_num\_threads} routine affects the number of threads to be used for 
subsequent parallel regions that do not specify a \code{num\_threads} clause, by setting the 
value of the first element of the \plc{nthreads-var} ICV of the current task.

\format
\ccppspecificstart
\begin{boxedcode}
void omp\_set\_num\_threads(int \plc{num\_threads});
\end{boxedcode}
\ccppspecificend

\fortranspecificstart
\begin{boxedcode}
subroutine omp\_set\_num\_threads(\plc{num\_threads})
integer \plc{num\_threads}
\end{boxedcode}
\fortranspecificend

\constraints
The value of the argument passed to this routine must evaluate to a positive integer, or 
else the behavior of this routine is implementation defined.

\binding
The binding task set for an \code{omp\_set\_num\_threads} region is the generating task.

\effect
The effect of this routine is to set the value of the first element of the \plc{nthreads-var} ICV 
of the current task to the value specified in the argument. 

See \specref{subsec:Determining the Number of Threads for a parallel Region} 
for the rules governing the number of threads used to 
execute a \code{parallel} region. 

\crossreferences
\begin{itemize}
\item \plc{nthreads-var} ICV, see 
\specref{sec:Internal Control Variables}.

\item \code{OMP\_NUM\_THREADS} environment variable, see 
\specref{sec:OMP_NUM_THREADS}.

\item \code{omp\_get\_max\_threads} routine, see 
\specref{subsec:omp_get_max_threads}.

\item \code{parallel} construct, see 
\specref{sec:parallel Construct}.

\item \code{num\_threads} clause, see 
\specref{sec:parallel Construct}.
\end{itemize}









\subsection{\code{omp\_get\_num\_threads}}
\index{omp\_get\_num\_threads@{\code{omp\_get\_num\_threads}}}
\label{subsec:omp_get_num_threads}
\summary
The \code{omp\_get\_num\_threads} routine returns the number of threads in the current 
team.
\format
\ccppspecificstart
\begin{boxedcode}
int omp\_get\_num\_threads(void); 
\end{boxedcode}
\ccppspecificend

\pagebreak
\fortranspecificstart
\begin{boxedcode}
integer function omp\_get\_num\_threads()
\end{boxedcode}
\fortranspecificend

\binding
The binding region for an \code{omp\_get\_num\_threads} region is the innermost enclosing 
\code{parallel} region.

\effect
The \code{omp\_get\_num\_threads} routine returns the number of threads in the team 
executing the \code{parallel} region to which the routine region binds. If called from the 
sequential part of a program, this routine returns 1. 

See \specref{subsec:Determining the Number of Threads for a parallel Region} 
for the rules governing the number of threads used to 
execute a \code{parallel} region. 

\crossreferences
\begin{itemize}
\item \code{parallel} construct, see 
\specref{sec:parallel Construct}.

\item \code{omp\_set\_num\_threads} routine, see 
\specref{subsec:omp_set_num_threads}.

\item \code{OMP\_NUM\_THREADS} environment variable, see 
\specref{sec:OMP_NUM_THREADS}.
\end{itemize}










\subsection{\code{omp\_get\_max\_threads}}
\index{omp\_get\_max\_threads@{\code{omp\_get\_max\_threads}}}
\label{subsec:omp_get_max_threads}
\summary
The \code{omp\_get\_max\_threads} routine returns an upper bound on the number of 
threads that could be used to form a new team if a \code{parallel} construct without a 
\code{num\_threads} clause were encountered after execution returns from this routine.

\pagebreak
\format
\ccppspecificstart
\begin{boxedcode}
int omp\_get\_max\_threads(void);
\end{boxedcode}
\ccppspecificend

\fortranspecificstart
\begin{boxedcode}
integer function omp\_get\_max\_threads()
\end{boxedcode}
\fortranspecificend

\binding
The binding task set for an \code{omp\_get\_max\_threads} region is the generating task. 

\effect
The value returned by \code{omp\_get\_max\_threads} is the value of the first element of 
the \plc{nthreads-var} ICV of the current task. This value is also an upper bound on the 
number of threads that could be used to form a new team if a parallel region without a 
\code{num\_threads} clause were encountered after execution returns from this routine.

See \specref{subsec:Determining the Number of Threads for a parallel Region} 
for the rules governing the number of threads used to 
execute a\linebreak \code{parallel} region. 

\notestart
\noteheader – The return value of the \code{omp\_get\_max\_threads} routine can be used to 
dynamically allocate sufficient storage for all threads in the team formed at the 
subsequent active \code{parallel} region.
\noteend

\crossreferences
\begin{itemize}
\item \plc{nthreads-var} ICV, see 
\specref{sec:Internal Control Variables}.

\item \code{parallel} construct, see 
\specref{sec:parallel Construct}.

\item \code{num\_threads} clause, see 
\specref{sec:parallel Construct}.

\item \code{omp\_set\_num\_threads} routine, see 
\specref{subsec:omp_set_num_threads}.

\item \code{OMP\_NUM\_THREADS} environment variable, see 
\specref{sec:OMP_NUM_THREADS}.
\end{itemize}










\subsection{\code{omp\_get\_thread\_num}}
\index{omp\_get\_thread\_num@{\code{omp\_get\_thread\_num}}}
\label{subsec:omp_get_thread_num}
\summary
The \code{omp\_get\_thread\_num} routine returns the thread number, within the current 
team, of the calling thread.

\format
\ccppspecificstart
\begin{boxedcode}
int omp\_get\_thread\_num(void); 
\end{boxedcode}
\ccppspecificend

\fortranspecificstart
\begin{boxedcode}
integer function omp\_get\_thread\_num() 
\end{boxedcode}
\fortranspecificend

\binding
The binding thread set for an \code{omp\_get\_thread\_num} region is the current team. The 
binding region for an \code{omp\_get\_thread\_num} region is the innermost enclosing 
\code{parallel} region. 

\effect
The \code{omp\_get\_thread\_num} routine returns the thread number of the calling thread, 
within the team executing the \code{parallel} region to which the routine region binds. The 
thread number is an integer between 0 and one less than the value returned by 
\code{omp\_get\_num\_threads}, inclusive. The thread number of the master thread of the 
team is 0. The routine returns 0 if it is called from the sequential part of a program.

\notestart
\noteheader – The thread number may change during the execution of an untied task. The 
value returned by \code{omp\_get\_thread\_num} is not generally useful during the execution 
of such a task region.
\noteend

\crossreferences
\begin{itemize}
\item \code{omp\_get\_num\_threads} routine, see 
\specref{subsec:omp_get_num_threads}.
\end{itemize}








\subsection{\code{omp\_get\_num\_procs}}
\index{omp\_get\_num\_procs@{\code{omp\_get\_num\_procs}}}
\label{subsec:omp_get_num_procs}
\summary
The \code{omp\_get\_num\_procs} routine returns the number of processors available to the 
device.

\format
\ccppspecificstart
\begin{boxedcode}
int omp\_get\_num\_procs(void);
\end{boxedcode}
\ccppspecificend

\fortranspecificstart
\begin{boxedcode}
integer function omp\_get\_num\_procs()
\end{boxedcode}
\fortranspecificend

\binding
The binding thread set for an \code{omp\_get\_num\_procs} region is all threads on a device. 
The effect of executing this routine is not related to any specific region corresponding to 
any construct or API routine.

\effect
The \code{omp\_get\_num\_procs} routine returns the number of processors that are available 
to the device at the time the routine is called. Note that this value may change between 
the time that it is determined by the \code{omp\_get\_num\_procs} routine and the time that it 
is read in the calling context due to system actions outside the control of the OpenMP 
implementation.









\subsection{\code{omp\_in\_parallel}}
\index{omp\_in\_parallel@{\code{omp\_in\_parallel}}}
\label{subsec:omp_in_parallel}
\summary
The \code{omp\_in\_parallel} routine returns \plc{true} if the \plc{active-levels-var} ICV is greater 
than zero; otherwise, it returns \plc{false}.

\pagebreak
\format
\ccppspecificstart
\begin{boxedcode}
int omp\_in\_parallel(void);
\end{boxedcode}
\ccppspecificend

\fortranspecificstart
\begin{boxedcode}
logical function omp\_in\_parallel()
\end{boxedcode}
\fortranspecificend

\binding
The binding task set for an \code{omp\_in\_parallel} region is the generating task.

\effect
The effect of the \code{omp\_in\_parallel} routine is to return \plc{true} if the current task is 
enclosed by an active \code{parallel} region, and the \code{parallel} region is enclosed by the 
outermost initial task region on the device; otherwise it returns \plc{false}.

\crossreferences
\begin{itemize}
\item \plc{active-levels-var}, see 
\specref{sec:Internal Control Variables}.

\item \code{omp\_get\_active\_level} routine, see 
\specref{subsec:omp_get_active_level}.
\end{itemize}








\bigskip
\subsection{\code{omp\_set\_dynamic}}
\index{omp\_set\_dynamic@{\code{omp\_set\_dynamic}}}
\label{subsec:omp_set_dynamic}
\summary
The \code{omp\_set\_dynamic} routine enables or disables dynamic adjustment of the 
number of threads available for the execution of subsequent \code{parallel} regions by 
setting the value of the \plc{dyn-var} ICV


\pagebreak
\format
\ccppspecificstart
\begin{boxedcode}
void omp\_set\_dynamic(int \plc{dynamic\_threads});
\end{boxedcode}
\ccppspecificend
\bigskip

\begin{samepage}
\fortranspecificstart
\begin{boxedcode}
subroutine omp\_set\_dynamic(\plc{dynamic\_threads})
logical \plc{dynamic\_threads
}\end{boxedcode}
\fortranspecificend
\end{samepage}

\binding
The binding task set for an \code{omp\_set\_dynamic} region is the generating task. 

\effect
For implementations that support dynamic adjustment of the number of threads, if the 
argument to \code{omp\_set\_dynamic} evaluates to \plc{true}, dynamic adjustment is enabled for 
the current task; otherwise, dynamic adjustment is disabled for the current task. For 
implementations that do not support dynamic adjustment of the number of threads this 
routine has no effect: the value of \plc{dyn-var} remains \plc{false}.

See \specref{subsec:Determining the Number of Threads for a parallel Region} 
for the rules governing the number of threads used to 
execute a\linebreak \code{parallel} region. 

\crossreferences
\begin{itemize}
\item \plc{dyn-var} ICV, see 
\specref{sec:Internal Control Variables}.

\item \code{omp\_get\_num\_threads} routine, see 
\specref{subsec:omp_get_num_threads}.

\item \code{omp\_get\_dynamic} routine, see 
\specref{subsec:omp_get_dynamic}.

\item \code{OMP\_DYNAMIC} environment variable, see 
\specref{sec:OMP_DYNAMIC}.
\end{itemize}








\subsection{\code{omp\_get\_dynamic}}
\index{omp\_get\_dynamic@{\code{omp\_get\_dynamic}}}
\label{subsec:omp_get_dynamic}
\summary
The \code{omp\_get\_dynamic} routine returns the value of the \plc{dyn-var} ICV, which 
determines whether dynamic adjustment of the number of threads is enabled or disabled.

\format
\ccppspecificstart
\begin{boxedcode}
int omp\_get\_dynamic(void);
\end{boxedcode}
\ccppspecificend

\fortranspecificstart
\begin{boxedcode}
logical function omp\_get\_dynamic()
\end{boxedcode}
\fortranspecificend

\binding
The binding task set for an \code{omp\_get\_dynamic} region is the generating task. 

\effect
This routine returns \plc{true} if dynamic adjustment of the number of threads is enabled for 
the current task; it returns \plc{false}, otherwise. If an implementation does not support 
dynamic adjustment of the number of threads, then this routine always returns \plc{false}.

See \specref{subsec:Determining the Number of Threads for a parallel Region} 
for the rules governing the number of threads used to 
execute a \code{parallel} region. 

\crossreferences
\begin{itemize}
\item \plc{dyn-var} ICV, see 
\specref{sec:Internal Control Variables}.

\item \code{omp\_set\_dynamic} routine, see 
\specref{subsec:omp_set_dynamic}.

\item \code{OMP\_DYNAMIC} environment variable, see 
\specref{sec:OMP_DYNAMIC}.
\end{itemize}









\subsection{\code{omp\_get\_cancellation}}
\index{omp\_get\_cancellation@{\code{omp\_get\_cancellation}}}
\label{subsec:omp_get_cancellation}
\summary
The \code{omp\_get\_cancellation} routine returns the value of the \plc{cancel-var} ICV, which 
controls the behavior of the \code{cancel} construct and cancellation points.

\format
\ccppspecificstart
\begin{boxedcode}
int omp\_get\_cancellation(void);
\end{boxedcode}
\ccppspecificend

\fortranspecificstart
\begin{boxedcode}
logical function omp\_get\_cancellation()
\end{boxedcode}
\fortranspecificend

\binding
The binding task set for an \code{omp\_get\_cancellation} region is the whole program.

\effect
This routine returns \plc{true} if cancellation is activated. It returns \plc{false} otherwise.

\crossreferences
\begin{itemize}
\item \plc{cancel-var} ICV, see 
\specref{subsec:ICV Descriptions}.

\item \code{OMP\_CANCELLATION} environment variable, see 
\specref{sec:OMP_CANCELLATION}
\end{itemize}










\subsection{\code{omp\_set\_nested}}
\index{omp\_set\_nested@{\code{omp\_set\_nested}}}
\label{subsec:omp_set_nested}
\summary
The \code{omp\_set\_nested} routine enables or disables nested parallelism, by setting the 
\plc{nest-var} ICV. 

\format
\ccppspecificstart
\begin{boxedcode}
void omp\_set\_nested(int \plc{nested});
\end{boxedcode}
\ccppspecificend

\fortranspecificstart
\begin{boxedcode}
subroutine omp\_set\_nested(\plc{nested})
logical \plc{nested}
\end{boxedcode}
\fortranspecificend

\binding
The binding task set for an \code{omp\_set\_nested} region is the generating task. 

\effect
For implementations that support nested parallelism, if the argument to 
\code{omp\_set\_nested} evaluates to \plc{true}, nested parallelism is enabled for the current task; 
otherwise, nested parallelism is disabled for the current task. For implementations that 
do not support nested parallelism, this routine has no effect: the value of \plc{nest-var} 
remains \plc{false}.

See \specref{subsec:Determining the Number of Threads for a parallel Region} 
for the rules governing the number of threads used to 
execute a \code{parallel} region. 

\crossreferences
\begin{itemize}
\item \plc{nest-var} ICV, see 
\specref{sec:Internal Control Variables}.

\item \code{omp\_set\_max\_active\_levels} routine, see 
\specref{subsec:omp_set_max_active_levels}.

\item \code{omp\_get\_max\_active\_levels} routine, see 
\specref{subsec:omp_get_max_active_levels}.

\item \code{omp\_get\_nested} routine, see 
\specref{subsec:omp_get_nested}.

\item \code{OMP\_NESTED} environment variable, see 
\specref{sec:OMP_NESTED}.
\end{itemize}








\subsection{\code{omp\_get\_nested}}
\index{omp\_get\_nested@{\code{omp\_get\_nested}}}
\label{subsec:omp_get_nested}
\summary
The \code{omp\_get\_nested} routine returns the value of the \plc{nest-var} ICV, which 
determines if nested parallelism is enabled or disabled.

\ccppspecificstart
\begin{boxedcode}
int omp\_get\_nested(void);
\end{boxedcode}
\ccppspecificend

\fortranspecificstart
\begin{boxedcode}
logical function omp\_get\_nested()
\end{boxedcode}
\fortranspecificend

\binding
The binding task set for an \code{omp\_get\_nested} region is the generating task. 

\effect
This routine returns \plc{true} if nested parallelism is enabled for the current task; it returns 
\plc{false}, otherwise. If an implementation does not support nested parallelism, this routine 
always returns \plc{false}.

See \specref{subsec:Determining the Number of Threads for a parallel Region} 
for the rules governing the number of threads used to 
execute a\linebreak \code{parallel} region. 

\crossreferences
\begin{itemize}
\item \plc{nest-var} ICV, see 
\specref{sec:Internal Control Variables}.

\item \code{omp\_set\_nested} routine, see 
\specref{subsec:omp_set_nested}.

\item \code{OMP\_NESTED} environment variable, see 
\specref{sec:OMP_NESTED}.
\end{itemize}









\subsection{\code{omp\_set\_schedule}}
\index{omp\_set\_schedule@{\code{omp\_set\_schedule}}}
\label{subsec:omp_set_schedule}
\summary
The \code{omp\_set\_schedule} routine affects the schedule that is applied when \code{runtime} 
is used as schedule kind, by setting the value of the \plc{run-sched-var} ICV. 

\format
\ccppspecificstart
\begin{boxedcode}
void omp\_set\_schedule(omp\_sched\_t \plc{kind}, int \plc{modifier});
\end{boxedcode}
\ccppspecificend

\fortranspecificstart
\begin{boxedcode}
subroutine omp\_set\_schedule(\plc{kind}, \plc{modifier}) 
integer (kind=omp\_sched\_kind) \plc{kind}
integer \plc{modifier}
\end{boxedcode}
\fortranspecificend

\constraints
The first argument passed to this routine can be one of the valid OpenMP schedule kinds 
(except for \code{runtime}) or any implementation specific schedule. The C/C++ header file 
(\code{omp.h}) and the Fortran include file (\code{omp\_lib.h}) and/or Fortran~90 module file 
(\code{omp\_lib}) define the valid constants. The valid constants must include the following, 
which can be extended with implementation specific values:

\pagebreak
\ccppspecificstart
\begin{boxedcode}
typedef enum omp\_sched\_t {
    omp\_sched\_static = 1,
    omp\_sched\_dynamic = 2,
    omp\_sched\_guided = 3,
    omp\_sched\_auto = 4
} omp\_sched\_t;
\end{boxedcode}
\ccppspecificend

\begin{samepage}
\fortranspecificstart
\begin{boxedcode}
integer(kind=omp\_sched\_kind), parameter :: omp\_sched\_static = 1
integer(kind=omp\_sched\_kind), parameter :: omp\_sched\_dynamic = 2
integer(kind=omp\_sched\_kind), parameter :: omp\_sched\_guided = 3
integer(kind=omp\_sched\_kind), parameter :: omp\_sched\_auto = 4
\end{boxedcode}
\fortranspecificend
\end{samepage}

\binding
The binding task set for an \code{omp\_set\_schedule} region is the generating task. 

\effect
The effect of this routine is to set the value of the \plc{run-sched-var} ICV of the current task 
to the values specified in the two arguments. The schedule is set to the schedule type 
specified by the first argument \plc{kind}. It can be any of the standard schedule types or 
any other implementation specific one. For the schedule types \code{static}, \code{dynamic}, and 
\code{guided} the \plc{chunk\_size} is set to the value of the second argument, or to the default 
\plc{chunk\_size} if the value of the second argument is less than 1; for the schedule type 
\code{auto} the second argument has no meaning; for implementation specific schedule types, 
the values and associated meanings of the second argument are implementation defined.

\crossreferences
\begin{itemize}
\item \plc{run-sched-var} ICV, see 
\specref{sec:Internal Control Variables}.

\item \code{omp\_get\_schedule} routine, see 
\specref{subsec:omp_get_schedule}.

\item \code{OMP\_SCHEDULE} environment variable, see 
\specref{sec:OMP_SCHEDULE}.

\item Determining the schedule of a worksharing loop, see 
\specref{subsubsec:Determining the Schedule of a Worksharing Loop}.
\end{itemize}









\subsection{\code{omp\_get\_schedule}}
\index{omp\_get\_schedule@{\code{omp\_get\_schedule}}}
\label{subsec:omp_get_schedule}
\summary
The \code{omp\_get\_schedule} routine returns the schedule that is applied when the 
runtime schedule is used. 
\format
\ccppspecificstart
\begin{boxedcode}
void omp\_get\_schedule(omp\_sched\_t * \plc{kind}, int * \plc{modifier}); 
\end{boxedcode}
\ccppspecificend

\fortranspecificstart
\begin{boxedcode}
subroutine omp\_get\_schedule(\plc{kind}, \plc{modifier}) 
integer (kind=omp\_sched\_kind) \plc{kind} 
integer \plc{modifier}
\end{boxedcode}
\fortranspecificend

\binding
The binding task set for an \code{omp\_get\_schedule} region is the generating task. 

\effect
This routine returns the \plc{run-sched-var} ICV in the task to which the routine binds. The 
first argument \plc{kind} returns the schedule to be used. It can be any of the standard 
schedule types as defined in 
\specref{subsec:omp_set_schedule}, 
or any implementation specific 
schedule type. The second argument is interpreted as in the \code{omp\_set\_schedule} call, 
defined in 
\specref{subsec:omp_set_schedule}.

\crossreferences
\begin{itemize}
\item \plc{run-sched-var} ICV, see 
\specref{sec:Internal Control Variables}.

\item \code{omp\_set\_schedule} routine, see 
\specref{subsec:omp_set_schedule}.

\item \code{OMP\_SCHEDULE} environment variable, see 
\specref{sec:OMP_SCHEDULE}.

\item Determining the schedule of a worksharing loop, see 
\specref{subsubsec:Determining the Schedule of a Worksharing Loop}.
\end{itemize}









\subsection{\code{omp\_get\_thread\_limit}}
\index{omp\_get\_thread\_limit@{\code{omp\_get\_thread\_limit}}}
\label{subsec:omp_get_thread_limit}
\summary
The \code{omp\_get\_thread\_limit} routine returns the maximum number of OpenMP 
threads available to participate in the current contention group. 

\format
\ccppspecificstart
\begin{boxedcode}
int omp\_get\_thread\_limit(void);
\end{boxedcode}
\ccppspecificend

\fortranspecificstart
\begin{boxedcode}
integer function omp\_get\_thread\_limit()
\end{boxedcode}
\fortranspecificend

\binding
The binding thread set for an \code{omp\_get\_thread\_limit} region is all threads on the 
device. The effect of executing this routine is not related to any specific region 
corresponding to any construct or API routine. 

\effect
The \code{omp\_get\_thread\_limit} routine returns the value of the \plc{thread-limit-var} ICV.

\crossreferences
\begin{itemize}
\item \plc{thread-limit-var} ICV, see 
\specref{sec:Internal Control Variables}.

\item \code{OMP\_THREAD\_LIMIT} environment variable, see 
\specref{sec:OMP_THREAD_LIMIT}.
\end{itemize}









\subsection{\code{omp\_set\_max\_active\_levels}}
\index{omp\_set\_max\_active\_levels@{\code{omp\_set\_max\_active\_levels}}}
\label{subsec:omp_set_max_active_levels}
\summary
The \code{omp\_set\_max\_active\_levels} routine limits the number of nested active 
parallel regions on the device, by setting the \plc{max-active-levels-var} ICV

\format
\ccppspecificstart
\begin{boxedcode}
void omp\_set\_max\_active\_levels(int \plc{max\_levels});
\end{boxedcode}
\ccppspecificend

\fortranspecificstart
\begin{boxedcode}
subroutine omp\_set\_max\_active\_levels(\plc{max\_levels})
integer \plc{max\_levels}
\end{boxedcode}
\fortranspecificend

\constraints
The value of the argument passed to this routine must evaluate to a non-negative integer, 
otherwise the behavior of this routine is implementation defined.

\binding
When called from a sequential part of the program, the binding thread set for an 
\code{omp\_set\_max\_active\_levels} region is the encountering thread. When called 
from within any explicit parallel region, the binding thread set (and binding region, if 
required) for the \code{omp\_set\_max\_active\_levels} region is implementation defined. 

\effect
The effect of this routine is to set the value of the \plc{max-active-levels-var} ICV to the value 
specified in the argument. 

If the number of parallel levels requested exceeds the number of levels of parallelism 
supported by the implementation, the value of the \plc{max-active-levels-var} ICV will be set 
to the number of parallel levels supported by the implementation.

This routine has the described effect only when called from a sequential part of the 
program. When called from within an explicit \code{parallel} region, the effect of this 
routine is implementation defined.

\crossreferences
\begin{itemize}
\item \plc{max-active-levels-var} ICV, see 
\specref{sec:Internal Control Variables}.

\item \code{omp\_get\_max\_active\_levels} routine, see 
\specref{subsec:omp_get_max_active_levels}.

\item \code{OMP\_MAX\_ACTIVE\_LEVELS} environment variable, see 
\specref{sec:OMP_MAX_ACTIVE_LEVELS}.
\end{itemize}










\subsection{\code{omp\_get\_max\_active\_levels}}
\index{omp\_get\_max\_active\_levels@{\code{omp\_get\_max\_active\_levels}}}
\label{subsec:omp_get_max_active_levels}
\summary
The \code{omp\_get\_max\_active\_levels} routine returns the value of the 
\plc{max-active-levels-var} ICV, which determines the maximum number of nested active parallel regions 
on the device. 

\format
\ccppspecificstart
\begin{boxedcode}
int omp\_get\_max\_active\_levels(void);
\end{boxedcode}
\ccppspecificend

\fortranspecificstart
\begin{boxedcode}
integer function omp\_get\_max\_active\_levels()
\end{boxedcode}
\fortranspecificend

\binding
When called from a sequential part of the program, the binding thread set for an 
\code{omp\_get\_max\_active\_levels} region is the encountering thread. When called 
from within any explicit parallel region, the binding thread set (and binding region, if 
required) for the \code{omp\_get\_max\_active\_levels} region is implementation defined. 

\effect
The \code{omp\_get\_max\_active\_levels} routine returns the value of the \plc{max-active-levels-var} ICV,
which determines the maximum number of nested active parallel regions 
on the device. 

\crossreferences
\begin{itemize}
\item \plc{max-active-levels-var} ICV, see 
\specref{sec:Internal Control Variables}.

\item \code{omp\_set\_max\_active\_levels} routine, see 
\specref{subsec:omp_set_max_active_levels}.

\item \code{OMP\_MAX\_ACTIVE\_LEVELS} environment variable, see 
\specref{sec:OMP_MAX_ACTIVE_LEVELS}.
\end{itemize}








\subsection{\code{omp\_get\_level}}
\index{omp\_get\_level@{\code{omp\_get\_level}}}
\label{subsec:omp_get_level}
\summary
The \code{omp\_get\_level} routine returns the value of the \plc{levels-var} ICV. 

\format
\ccppspecificstart
\begin{boxedcode}
int omp\_get\_level(void);
\end{boxedcode}
\ccppspecificend

\fortranspecificstart
\begin{boxedcode}
integer function omp\_get\_level()
\end{boxedcode}
\fortranspecificend

\binding
The binding task set for an \code{omp\_get\_level} region is the generating task. 

\effect
The effect of the \code{omp\_get\_level} routine is to return the number of nested 
\code{parallel} regions (whether active or inactive) enclosing the current task such that all 
of the \code{parallel} regions are enclosed by the outermost initial task region on the 
current device.

\crossreferences
\begin{itemize}
\item \plc{levels-var} ICV, see 
\specref{sec:Internal Control Variables}.

\item \code{omp\_get\_active\_level} routine, see 
\specref{subsec:omp_get_active_level}.

\item \code{OMP\_MAX\_ACTIVE\_LEVELS} environment variable, see 
\specref{sec:OMP_MAX_ACTIVE_LEVELS}.
\end{itemize}










\subsection{\code{omp\_get\_ancestor\_thread\_num}}
\index{omp\_get\_ancestor\_thread\_num@{\code{omp\_get\_ancestor\_thread\_num}}}
\label{subsec:omp_get_ancestor_thread_num}
\summary
The \code{omp\_get\_ancestor\_thread\_num} routine returns, for a given nested level of 
the current thread, the thread number of the ancestor of the current thread.

\parbox{\linewidth}{%
\format
\ccppspecificstart
\begin{boxedcode}
int omp\_get\_ancestor\_thread\_num(int \plc{level});
\end{boxedcode}
\ccppspecificend}

\fortranspecificstart
\begin{boxedcode}
integer function omp\_get\_ancestor\_thread\_num(\plc{level})
integer \plc{level}
\end{boxedcode}
\fortranspecificend

\binding
The binding thread set for an \code{omp\_get\_ancestor\_thread\_num} region is the 
encountering thread. The binding region for an \code{omp\_get\_ancestor\_thread\_num} 
region is the innermost enclosing \code{parallel} region. 

\effect
The \code{omp\_get\_ancestor\_thread\_num} routine returns the thread number of the 
ancestor at a given nest level of the current thread or the thread number of the current 
thread. If the requested nest level is outside the range of 0 and the nest level of the 
current thread, as returned by the \code{omp\_get\_level} routine, the routine returns -1.

\notestart
\noteheader – When the \code{omp\_get\_ancestor\_thread\_num} routine is called with a value 
of \code{level}=0, the routine always returns 0. If \code{level}=\code{omp\_get\_level()}, the routine 
has the same effect as the \code{omp\_get\_thread\_num} routine. 
\noteend

\crossreferences
\begin{itemize}
\item \code{omp\_get\_level} routine, see 
\specref{subsec:omp_get_level}.

\item \code{omp\_get\_thread\_num} routine, see 
\specref{subsec:omp_get_thread_num}.

\item \code{omp\_get\_team\_size} routine, see 
\specref{subsec:omp_get_team_size}.
\end{itemize}










\subsection{\code{omp\_get\_team\_size}}
\index{omp\_get\_team\_size@{\code{omp\_get\_team\_size}}}
\label{subsec:omp_get_team_size}
\summary
The \code{omp\_get\_team\_size} routine returns, for a given nested level of the current 
thread, the size of the thread team to which the ancestor or the current thread belongs. 

\format
\ccppspecificstart
\begin{boxedcode}
int omp\_get\_team\_size(int \plc{level});
\end{boxedcode}
\ccppspecificend

\fortranspecificstart
\begin{boxedcode}
integer function omp\_get\_team\_size(\plc{level})
integer \plc{level}
\end{boxedcode}
\fortranspecificend

\binding
The binding thread set for an \code{omp\_get\_team\_size} region is the encountering 
thread. The binding region for an \code{omp\_get\_team\_size} region is the innermost 
enclosing \code{parallel} region.

\effect
The \code{omp\_get\_team\_size} routine returns the size of the thread team to which the 
ancestor or the current thread belongs. If the requested nested level is outside the range 
of 0 and the nested level of the current thread, as returned by the \code{omp\_get\_level} 
routine, the routine returns -1. Inactive parallel regions are regarded like active parallel 
regions executed with one thread. 

\notestart
\noteheader – When the \code{omp\_get\_team\_size} routine is called with a value of \code{level}=0, 
the routine always returns 1. If \code{level}=\code{omp\_get\_level()}, the routine has the same 
effect as the\linebreak \code{omp\_get\_num\_threads} routine. 
\noteend

\crossreferences
\begin{itemize}
\item \code{omp\_get\_num\_threads} routine, see 
\specref{subsec:omp_get_num_threads}.

\item \code{omp\_get\_level} routine, see 
\specref{subsec:omp_get_level}.

\item \code{omp\_get\_ancestor\_thread\_num} routine, see 
\specref{subsec:omp_get_ancestor_thread_num}.
\end{itemize}









\subsection{\code{omp\_get\_active\_level}}
\index{omp\_get\_active\_level@{\code{omp\_get\_active\_level}}}
\label{subsec:omp_get_active_level}
\summary
The \code{omp\_get\_active\_level} routine returns the value of the \plc{active-level-vars} ICV..

\format
\ccppspecificstart
\begin{boxedcode}
int \code{omp\_get\_active\_level}(void);
\end{boxedcode}
\ccppspecificend

\pagebreak
\fortranspecificstart
\begin{boxedcode}
integer function omp\_get\_active\_level()
\end{boxedcode}
\fortranspecificend

\binding
The binding task set for the an \code{omp\_get\_active\_level} region is the generating 
task. 

\effect
The effect of the \code{omp\_get\_active\_level} routine is to return the number of nested, 
active \code{parallel} regions enclosing the current task such that all of the \code{parallel} 
regions are enclosed by the outermost initial task region on the current device. 

\crossreferences
\begin{itemize}
\item \plc{active-levels-var} ICV, see 
\specref{sec:Internal Control Variables}.

\item \code{omp\_get\_level} routine, see 
\specref{subsec:omp_get_level}. 
\end{itemize}











\subsection{\code{omp\_in\_final}}
\index{omp\_in\_final@{\code{omp\_in\_final}}}
\label{subsec:omp_in_final}
\summary
The \code{omp\_in\_final} routine returns \plc{true} if the routine is executed in a final task 
region; otherwise, it returns \plc{false}.

\format
\ccppspecificstart
\begin{boxedcode}
int omp\_in\_final(void);
\end{boxedcode}
\ccppspecificend

\fortranspecificstart
\begin{boxedcode}
logical function omp\_in\_final()
\end{boxedcode}
\fortranspecificend

\binding
The binding task set for an \code{omp\_in\_final} region is the generating task.

\effect
\code{omp\_in\_final} returns \plc{true} if the enclosing task region is final. Otherwise, it returns 
\plc{false}.










\subsection{\code{omp\_get\_proc\_bind}}
\index{omp\_get\_proc\_bind@{\code{omp\_get\_proc\_bind}}}
\label{subsec:omp_get_proc_bind}
\summary
The \code{omp\_get\_proc\_bind} routine returns the thread affinity policy to be used for the 
subsequent nested \code{parallel} regions that do not specify a \code{proc\_bind} clause.

\format
\ccppspecificstart
\begin{boxedcode}
omp\_proc\_bind\_t omp\_get\_proc\_bind(void);
\end{boxedcode}
\ccppspecificend

\fortranspecificstart
\begin{boxedcode}
integer (kind=omp\_proc\_bind\_kind) function omp\_get\_proc\_bind()
\end{boxedcode}
\fortranspecificend

\constraints
The value returned by this routine must be one of the valid affinity policy kinds. The C/
C++ header file (\code{omp.h}) and the Fortran include file (\code{omp\_lib.h}) and/or Fortran~90 
module file (\code{omp\_lib}) define the valid constants. The valid constants must include the 
following:

\ccppspecificstart
\begin{codepar}
typedef enum omp\_proc\_bind\_t \{
  omp\_proc\_bind\_false = 0,
  omp\_proc\_bind\_true = 1,
  omp\_proc\_bind\_master = 2,
  omp\_proc\_bind\_close = 3,
  omp\_proc\_bind\_spread = 4
\} omp\_proc\_bind\_t;
\end{codepar}
\ccppspecificend

\fortranspecificstart
\begin{codepar}
integer (kind=omp\_proc\_bind\_kind), &
                parameter :: omp\_proc\_bind\_false = 0
integer (kind=omp\_proc\_bind\_kind), &
                parameter :: omp\_proc\_bind\_true = 1
integer (kind=omp\_proc\_bind\_kind), &
                parameter :: omp\_proc\_bind\_master = 2
integer (kind=omp\_proc\_bind\_kind), &
                parameter :: omp\_proc\_bind\_close = 3
integer (kind=omp\_proc\_bind\_kind), &
                parameter :: omp\_proc\_bind\_spread = 4
\end{codepar}
\fortranspecificend

\binding
The binding task set for an \code{omp\_get\_proc\_bind} region is the generating task

\effect
The effect of this routine is to return the value of the first element of the \plc{bind-var} ICV 
of the current task. See \specref{subsec:Controlling OpenMP Thread Affinity} 
for the rules governing the thread affinity policy.

\crossreferences
\begin{itemize}
\item \plc{bind-var} ICV, see 
\specref{sec:Internal Control Variables}.

\item \code{OMP\_PROC\_BIND} environment variable, see 
\specref{sec:OMP_PROC_BIND}.

\item Controlling OpenMP thread affinity, see 
\specref{subsec:Controlling OpenMP Thread Affinity}. 
\end{itemize}










\subsection{\code{omp\_set\_default\_device}}
\index{omp\_set\_default\_device@{\code{omp\_set\_default\_device}}}
\label{subsec:omp_set_default_device}

\summary

The \code{omp\_set\_default\_device} routine controls the default target device by 
assigning the value of the \plc{default-device-var} ICV.

\format
\ccppspecificstart
\begin{boxedcode}
void omp\_set\_default\_device(int \plc{device\_num});
\end{boxedcode}
\ccppspecificend

\fortranspecificstart
\begin{boxedcode}
subroutine omp\_set\_default\_device(\plc{device\_num})
integer \plc{device\_num}
\end{boxedcode}
\fortranspecificend

\binding
The binding task set for an \code{omp\_set\_default\_device} region is the generating 
task.

\effect
The effect of this routine is to set the value of the \plc{default-device-var} ICV of the current 
task to the value specified in the argument. When called from within a \code{target} region 
the effect of this routine is unspecified.

\crossreferences
\begin{itemize}
\item \plc{default-device-var}, see 
\specref{sec:Internal Control Variables}.

\item \code{omp\_get\_default\_device}, see 
\specref{subsec:omp_get_default_device}.

\item \code{OMP\_DEFAULT\_DEVICE} environment variable, see 
\specref{sec:OMP_DEFAULT_DEVICE}
\end{itemize}










\subsection{\code{omp\_get\_default\_device}}
\index{omp\_get\_default\_device@{\code{omp\_get\_default\_device}}}
\label{subsec:omp_get_default_device}
\summary
The \code{omp\_get\_default\_device} routine returns the default target device.

\format
\ccppspecificstart
\begin{boxedcode}
int omp\_get\_default\_device(void);
\end{boxedcode}
\ccppspecificend

\fortranspecificstart
\begin{boxedcode}
integer function omp\_get\_default\_device()
\end{boxedcode}
\fortranspecificend

\binding
The binding task set for an \code{omp\_get\_default\_device} region is the generating 
task. 

\effect
The \code{omp\_get\_default\_device} routine returns the value of the \plc{default-device-var} 
ICV of the current task. When called from within a \code{target} region the effect of this 
routine is unspecified.

\crossreferences
\begin{itemize}
\item \plc{default-device-var}, see 
\specref{sec:Internal Control Variables}.

\item \code{omp\_set\_default\_device}, see 
\specref{subsec:omp_set_default_device}.

\item \code{OMP\_DEFAULT\_DEVICE} environment variable, see 
\specref{sec:OMP_DEFAULT_DEVICE}. 
\end{itemize}











\subsection{\code{omp\_get\_num\_devices}}
\index{omp\_get\_num\_devices@{\code{omp\_get\_num\_devices}}}
\label{subsec:omp_get_num_devices}
\summary
The \code{omp\_get\_num\_devices} routine returns the number of target devices.

\format
\ccppspecificstart
\begin{boxedcode}
int omp\_get\_num\_devices(void);
\end{boxedcode}
\ccppspecificend

\fortranspecificstart
\begin{boxedcode}
integer function omp\_get\_num\_devices()
\end{boxedcode}
\fortranspecificend

\binding
The binding task set for an \code{omp\_get\_num\_devices} region is the generating task.

\effect
The \code{omp\_get\_num\_devices} routine returns the number of available target devices. 
When called from within a \code{target} region the effect of this routine is unspecified.

\crossreferences
None.










\subsection{\code{omp\_get\_num\_teams}}
\index{omp\@{\code{omp\_get\_num\_teams}}}
\label{subsec:omp_get_num_teams}
\summary
The \code{omp\_get\_num\_teams} routine returns the number of teams in the current \code{teams} 
region.

\pagebreak
\format
\ccppspecificstart
\begin{boxedcode}
int omp\_get\_num\_teams(void);
\end{boxedcode}
\ccppspecificend

\fortranspecificstart
\begin{boxedcode}
integer function omp\_get\_num\_teams()
\end{boxedcode}
\fortranspecificend

\binding
The binding task set for an \code{omp\_get\_num\_teams} region is the generating task

\effect
The effect of this routine is to return the number of teams in the current \code{teams} region. 
The routine returns 1 if it is called from outside of a \code{teams} region.

\crossreferences
\begin{itemize}
\item \code{teams} construct, see 
\specref{subsec:teams Construct}. 
\end{itemize}










\pagebreak
\subsection{\code{omp\_get\_team\_num}}
\index{omp\_get\_team\_num@{\code{omp\_get\_team\_num}}}
\label{subsec:omp_get_team_num}
\summary
The \code{omp\_get\_team\_num} routine returns the team number of the calling thread.

\format
\ccppspecificstart
\begin{boxedcode}
int omp\_get\_team\_num(void);
\end{boxedcode}
\ccppspecificend

\fortranspecificstart
\begin{boxedcode}
integer function omp\_get\_team\_num()
\end{boxedcode}
\fortranspecificend

\binding
The binding task set for an \code{omp\_get\_team\_num} region is the generating task.

\effect
The \code{omp\_get\_team\_num} routine returns the team number of the calling thread. The 
team number is an integer between 0 and one less than the value returned by 
\code{omp\_get\_num\_teams}, inclusive. The routine returns 0 if it is called outside of a 
\code{teams} region.

\crossreferences
\begin{itemize}
\item \code{teams} construct, see 
\specref{subsec:teams Construct}.

\item \code{omp\_get\_num\_teams} routine, see 
\specref{subsec:omp_get_num_teams}. 
\end{itemize}









\subsection{\code{omp\_is\_initial\_device}}
\index{omp\_is\_initial\_device@{\code{omp\_is\_initial\_device}}}
\label{subsec:omp_is_initial_device}
\summary
The \code{omp\_is\_initial\_device} routine returns \plc{true} if the current task is executing 
on the host device; otherwise, it returns \plc{false}.

\parbox{\linewidth}{%
\format
\ccppspecificstart
\begin{boxedcode}
int omp\_is\_initial\_device(void);
\end{boxedcode}
\ccppspecificend}

\fortranspecificstart
\begin{boxedcode}
logical function omp\_is\_initial\_device()
\end{boxedcode}
\fortranspecificend

\binding
The binding task set for an \code{omp\_is\_initial\_device} region is the generating task.

\effect
The effect of this routine is to return \plc{true} if the current task is executing on the host 
device; otherwise, it returns \plc{false}.

\crossreferences
\begin{itemize}
\item \code{target} construct, see 
\specref{subsec:target Construct}
\end{itemize}










\newpage
\section{Lock Routines}
\index{lock routines}
\label{sec:Lock Routines}
The OpenMP runtime library includes a set of general-purpose lock routines that can be 
used for synchronization. These general-purpose lock routines operate on OpenMP locks 
that are represented by OpenMP lock variables. OpenMP lock variables must be 
accessed only through the routines described in this section; programs that otherwise 
access OpenMP lock variables are non-conforming.

An OpenMP lock can be in one of the following states: \emph{uninitialized}, \emph{unlocked}, or 
\emph{locked}. If a lock is in the unlocked state, a task can \emph{set} the lock, which changes its state 
to \emph{locked}. The task that sets the lock is then said to \emph{own} the lock. A task that owns a 
lock can \emph{unset} that lock, returning it to the \emph{unlocked} state. A program in which a task 
unsets a lock that is owned by another task is non-conforming.

Two types of locks are supported: \emph{simple locks} and \emph{nestable locks}. A \emph{nestable lock} can 
be set multiple times by the same task before being unset; a \emph{simple lock} cannot be set if 
it is already owned by the task trying to set it. \emph{Simple lock} variables are associated with 
\emph{simple locks} and can only be passed to \emph{simple lock} routines. \emph{Nestable lock} variables are 
associated with \emph{nestable locks} and can only be passed to \emph{nestable lock} routines.

Each type of lock can also have a \emph{lock hint} that contains information about the intended usage 
of the lock by the application code.  The effect of the lock hint is implementation defined, 
that is, the OpenMP implementation can use this hint to select a proper 
usage-specific lock if supported by the implementation.  
Lock hints do not change the mutual exclusion semantics of locks, but allow the 
user to suggest the use of a specific implementation to the runtime.
A conforming implementation can safely ignore the lock hint and instead use locks without hints.

Constraints on the state and ownership of the lock accessed by each of the lock routines 
are described with the routine. If these constraints are not met, the behavior of the 
routine is unspecified. 

The OpenMP lock routines access a lock variable in such a way that they always read 
and update the most current value of the lock variable. It is not necessary for an 
OpenMP program to include explicit \code{flush} directives to ensure that the lock variable’s 
value is consistent among different tasks. 

\binding
The binding thread set for all lock routine regions is all threads in the contention group. 
As a consequence, for each OpenMP lock, the lock routine effects relate to all tasks that 
call the routines, without regard to which teams the threads in the contention group 
executing the tasks belong.

\littleheader{Simple Lock Routines}
\index{Simple Lock Routines}
\ccppspecificstart
The type \code{omp\_lock\_t} is a data type capable of representing a simple lock. For the 
following routines, a simple lock variable must be of \code{omp\_lock\_t} type. All simple 
lock routines require an argument that is a pointer to a variable of type \code{omp\_lock\_t}.
\ccppspecificend

\fortranspecificstart
For the following routines, a simple lock variable must be an integer variable of 
\code{kind=omp\_lock\_kind}.
\fortranspecificend

The simple lock routines are as follows:

\begin{itemize}
\item The \code{omp\_init\_lock} routine initializes a simple lock.

\item The \code{omp\_init\_lock\_with\_hint} routine initializes a simple lock and attaches a hint to it.

\item The \code{omp\_destroy\_lock} routine uninitializes a simple lock.

\item The \code{omp\_set\_lock} routine waits until a simple lock is available, and then sets it.

\item The \code{omp\_unset\_lock} routine unsets a simple lock.

\item The \code{omp\_test\_lock} routine tests a simple lock, and sets it if it is available.
\end{itemize}

\littleheader{Nestable Lock Routines}
\ccppspecificstart
The type \code{omp\_nest\_lock\_t} is a data type capable of representing a nestable lock. 
For the following routines, a nested lock variable must be of \code{omp\_nest\_lock\_t} type. 
All nestable lock routines require an argument that is a pointer to a variable of type 
\code{omp\_nest\_lock\_t}.
\ccppspecificend

\fortranspecificstart
For the following routines, a nested lock variable must be an integer variable of 
\code{kind=omp\_nest\_lock\_kind}.
\fortranspecificend

The nestable lock routines are as follows:

\begin{itemize}
\item The \code{omp\_init\_nest\_lock} routine initializes a nestable lock.

\item The \code{omp\_init\_nest\_lock\_with\_hint} routine initializes a nestable lock and attaches a hint to it.

\item The \code{omp\_destroy\_nest\_lock} routine uninitializes a nestable lock.

\item The \code{omp\_set\_nest\_lock} routine waits until a nestable lock is available, and then 
sets it.

\item The \code{omp\_unset\_nest\_lock} routine unsets a nestable lock.

\item The \code{omp\_test\_nest\_lock} routine tests a nestable lock, and sets it if it is 
available
\end{itemize}

\restrictions
OpenMP lock routines have the following restrictions:

\begin{itemize}
\item The use of the same OpenMP lock in different contention groups results in 
unspecified behavior. 
\end{itemize}










\subsection{\code{omp\_init\_lock} and \code{omp\_init\_nest\_lock}}
\label{subsec:omp_init_lock and omp_init_nest_lock}
\index{omp\_init\_lock@{\code{omp\_init\_lock}}}
\index{omp\_init\_nest\_lock@{\code{omp\_init\_nest\_lock}}}
\summary
These routines initialize an OpenMP lock without a hint.

\format
\ccppspecificstart
\begin{boxedcode}
void omp\_init\_lock(omp\_lock\_t *\plc{lock});
void omp\_init\_nest\_lock(omp\_nest\_lock\_t *\plc{lock});
\end{boxedcode}
\ccppspecificend

\fortranspecificstart
\begin{boxedcode}
subroutine omp\_init\_lock(\plc{svar})
integer (kind=omp\_lock\_kind) \plc{svar}

subroutine omp\_init\_nest\_lock(\plc{nvar})
integer (kind=omp\_nest\_lock\_kind) \plc{nvar}
\end{boxedcode}
\fortranspecificend

\constraints
A program that accesses a lock that is not in the uninitialized state through either routine 
is non-conforming.

\effect
The effect of these routines is to initialize the lock to the unlocked state; that is, no task 
owns the lock. In addition, the nesting count for a nestable lock is set to zero.











\subsection[\code{omp\_init\_lock\_with\_hint} and \code{omp\_init\_nest\_lock\_with\_hint}]{\code{omp\_init\_lock\_with\_hint} and \\\code{omp\_init\_nest\_lock\_with\_hint}}
\label{subsec:omp_init_lock_with_hint and omp_init_nest_lock_with_hint}
\index{omp\_init\_lock@{\code{omp\_init\_lock}}}
\index{omp\_init\_nest\_lock@{\code{omp\_init\_nest\_lock}}}
\summary
These routines initialize an OpenMP lock with a hint.  
The effect of the hint is implementation-defined. The OpenMP implementation
can ignore the hint without changing program semantics.


\format
\ccppspecificstart
\begin{boxedcode}
void omp\_init\_lock\_with\_hint(omp\_lock\_t *\plc{lock}, 
                             omp\_lock\_hint\_t \plc{hint});
void omp\_init\_nest\_lock\_with\_hint(omp\_nest\_lock\_t *\plc{lock}, 
                                  omp\_lock\_hint\_t \plc{hint});
\end{boxedcode}
\ccppspecificend

\fortranspecificstart
\begin{boxedcode}
subroutine omp\_init\_lock\_with\_hint(\plc{svar}, \plc{hint})
integer (kind=omp\_lock\_kind) \plc{svar}
integer (kind=omp\_lock\_hint\_kind) \plc{hint}

subroutine omp\_init\_nest\_lock\_with\_hint(\plc{nvar}, \plc{hint})
integer (kind=omp\_nest\_lock\_kind) \plc{nvar}
integer (kind=omp\_lock\_hint\_kind) \plc{hint}
\end{boxedcode}
\fortranspecificend

\constraints
A program that accesses a lock that is not in the uninitialized state through either routine 
is non-conforming.

The second argument passed to this routine (\plc{hint}) can be one of the valid OpenMP lock hints below or any implementation-defined hint. 
The C/C++ header file (\code{omp.h}) and the Fortran include file (\code{omp\_lib.h}) and/or Fortran~90 module file (\code{omp\_lib}) define the valid lock hint constants.  
The valid constants must include the following, which can be extended with implementation-defined values:

\ccppspecificstart
\begin{codepar}
typedef enum omp\_lock\_hint\_t \{
  omp\_lock\_hint\_none = 0,
  omp\_lock\_hint\_uncontended = 1,
  omp\_lock\_hint\_contended = 2,
  omp\_lock\_hint\_nonspeculative = 4,
  omp\_lock\_hint\_speculative = 8,
\} omp\_lock\_hint\_t;
\end{codepar}
\ccppspecificend

\fortranspecificstart
\begin{codepar}
integer (kind=omp\_lock\_hint\_kind), &
        parameter :: omp\_lock\_hint\_none = 0
integer (kind=omp\_lock\_hint\_kind), &
        parameter :: omp\_lock\_hint\_uncontended = 1
integer (kind=omp\_lock\_hint\_kind), &
        parameter :: omp\_lock\_hint\_contended = 2
integer (kind=omp\_lock\_hint\_kind), &
        parameter :: omp\_lock\_hint\_nonspeculative = 4 
integer (kind=omp\_lock\_hint\_kind), &
        parameter :: omp\_lock\_hint\_speculative = 8
\end{codepar}
\fortranspecificend

The hints can be combined by using the \code{+} or \code{|} operator in C/C++ or the \code{+} operator in Fortran.  
The effect of the combined hint is implementation defined and can be ignored by the implementation.
Combining \code{omp\_lock\_hint\_none} with any other hint is equivalent to specifying the other hint.
The following restrictions apply to combined hints; violating these make the program a non-conforming program:
\begin{itemize}
%\item \code{omp\_lock\_hint\_none} cannot be combined with any other hint,
\item the hints \code{omp\_lock\_hint\_uncontended} and \code{omp\_lock\_hint\_contended} cannot be combined,
\item the hints \code{omp\_lock\_hint\_nonspeculative} and \code{omp\_lock\_hint\_speculative} cannot be combined.
\end{itemize}


\notestart
\noteheader – Future OpenMP specifications may add additional hints to the 
\code{omp\_lock\_hint\_t} type and the \code{omp\_lock\_hint\_kind} kind.  
Implementers are advised to add implementation-defined hints starting from 
the most significant bit of the \code{omp\_lock\_hint\_t} type and
\code{omp\_lock\_hint\_kind} kind and to include the name of the 
implementation in the name of the added hint to avoid name conflicts 
with other OpenMP implementations.
\noteend


\effect
The effect of these routines is to initialize the lock to the unlocked state and, optionally, to choose a specific lock implementation based on the hint. 
After initialization no task owns the lock. In addition, the nesting count for a nestable lock is set to zero.




\subsection[\code{omp\_destroy\_lock} and \code{omp\_destroy\_nest\_lock}]{\code{omp\_destroy\_lock} and\\ \code{omp\_destroy\_nest\_lock}}
\label{subsec:omp_destroy_lock and omp_destroy_nest_lock}
\index{omp\_destroy\_lock@{\code{omp\_destroy\_lock}}}
\index{omp\_destroy\_nest\_lock@{\code{omp\_destroy\_nest\_lock}}}
\summary
These routines ensure that the OpenMP lock is uninitialized

\format
\ccppspecificstart
\begin{boxedcode}
void omp\_destroy\_lock(omp\_lock\_t *\plc{lock});
void omp\_destroy\_nest\_lock(omp\_nest\_lock\_t *\plc{lock});
\end{boxedcode}
\ccppspecificend

\fortranspecificstart
\begin{boxedcode}
subroutine omp\_destroy\_lock(\plc{svar})
integer (kind=omp\_lock\_kind) \plc{svar}

subroutine omp\_destroy\_nest\_lock(\plc{nvar})
integer (kind=omp\_nest\_lock\_kind) \plc{nvar}
\end{boxedcode}
\fortranspecificend

\constraints
A program that accesses a lock that is not in the unlocked state through either routine is 
non-conforming.

\effect
The effect of these routines is to change the state of the lock to uninitialized.










\subsection{\code{omp\_set\_lock} and \code{omp\_set\_nest\_lock}}
\label{subsec:omp_set_lock and omp_set_nest_lock}
\index{omp\_set\_lock@{\code{omp\_set\_lock}}}
\index{omp\_set\_nest\_lock@{\code{omp\_set\_nest\_lock}}}
\summary
These routines provide a means of setting an OpenMP lock. The calling task region is 
suspended until the lock is set or the task region speculatively continues to execute.

\format
\ccppspecificstart
\begin{boxedcode}
void omp\_set\_lock(omp\_lock\_t *\plc{lock});
void omp\_set\_nest\_lock(omp\_nest\_lock\_t *\plc{lock});
\end{boxedcode}
\ccppspecificend

\fortranspecificstart
\begin{boxedcode}
subroutine omp\_set\_lock(\plc{svar})
integer (kind=omp\_lock\_kind) \plc{svar}

subroutine omp\_set\_nest\_lock(\plc{nvar})
integer (kind=omp\_nest\_lock\_kind) \plc{nvar}
\end{boxedcode}
\fortranspecificend

\constraints
A program that accesses a lock that is in the uninitialized state through either routine is 
non-conforming. A simple lock accessed by \code{omp\_set\_lock} that is in the locked state 
must not be owned by the task that contains the call or deadlock will result.

\effect
Each of these routines has an effect equivalent to suspension of the task
executing the routine until the specified lock is available. However, the
task region may continue to execute speculatively, provided that the 
implementation ensures that execution is as if the lock were owned
exclusively by one thread. 

A simple lock is available if it is unlocked. Ownership of the lock is 
granted to the task executing the routine.

A nestable lock is available if it is unlocked or if it is already owned by 
the task executing the routine. The task executing the routine is granted, 
or retains, ownership of the lock, and the nesting count for the lock is 
incremented.




\subsection{\code{omp\_unset\_lock} and \code{omp\_unset\_nest\_lock}}
\label{subsec:omp_unset_lock and omp_unset_nest_lock}
\index{omp\_unset\_lock@{\code{omp\_unset\_lock}}}
\index{omp\_unset\_nest\_lock@{\code{omp\_unset\_nest\_lock}}}
\summary
These routines provide the means of unsetting an OpenMP lock.

\format
\ccppspecificstart
\begin{boxedcode}
void omp\_unset\_lock(omp\_lock\_t *\plc{lock});
void omp\_unset\_nest\_lock(omp\_nest\_lock\_t *\plc{lock});
\end{boxedcode}
\ccppspecificend

\fortranspecificstart
\begin{boxedcode}
subroutine omp\_unset\_lock(\plc{svar})
integer (kind=omp\_lock\_kind) \plc{svar}

subroutine omp\_unset\_nest\_lock(\plc{nvar})
integer (kind=omp\_nest\_lock\_kind) \plc{nvar}
\end{boxedcode}
\fortranspecificend

\constraints
A program that accesses a lock that is not in the locked state or that is not owned by the 
task that contains the call through either routine is non-conforming.

\effect
For a simple lock, the \code{omp\_unset\_lock} routine causes the lock to become unlocked.

For a nestable lock, the \code{omp\_unset\_nest\_lock} routine decrements the nesting 
count, and causes the lock to become unlocked if the resulting nesting count is zero.

For either routine, if the lock becomes unlocked, and if one or more task 
regions were effectively suspended because the lock was unavailable, the 
effect is that one task is chosen and given ownership of the lock. 









\subsection{\code{omp\_test\_lock} and \code{omp\_test\_nest\_lock}}
\label{subsec:omp_test_lock and omp_test_nest_lock}
\index{omp\_test\_lock@{\code{omp\_test\_lock}}}
\index{omp\_test\_nest\_lock@{\code{omp\_test\_nest\_lock}}}
\summary
These routines attempt to set an OpenMP lock but do not suspend execution of the task 
executing the routine.

\format
\ccppspecificstart
\begin{boxedcode}
int omp\_test\_lock(omp\_lock\_t *\plc{lock});
int omp\_test\_nest\_lock(omp\_nest\_lock\_t *\plc{lock});
\end{boxedcode}
\ccppspecificend

\fortranspecificstart
\begin{boxedcode}
logical function omp\_test\_lock(\plc{svar})
integer (kind=omp\_lock\_kind) \plc{svar}
integer function omp\_test\_nest\_lock(\plc{nvar})
integer (kind=omp\_nest\_lock\_kind) \plc{nvar}
\end{boxedcode}
\fortranspecificend

\constraints
A program that accesses a lock that is in the uninitialized state through either routine is 
non-conforming. The behavior is unspecified if a simple lock accessed by 
\code{omp\_test\_lock} is in the locked state and is owned by the task that contains the call.

\effect 
These routines attempt to set a lock in the same manner as \code{omp\_set\_lock} and 
\code{omp\_set\_nest\_lock}, except that they do not suspend execution of the task 
executing the routine.

For a simple lock, the \code{omp\_test\_lock} routine returns \plc{true} if the lock is successfully 
set; otherwise, it returns \plc{false}.

For a nestable lock, the \code{omp\_test\_nest\_lock} routine returns the new nesting count 
if the lock is successfully set; otherwise, it returns zero.







\section{Timing Routines}
\index{timing routines}
\index{timer}
\index{wall clock timer}
\label{sec:Timing Routines}
This section describes routines that support a portable wall clock timer.









\subsection{\code{omp\_get\_wtime}}
\index{omp\_get\_wtime@{\code{omp\_get\_wtime}}}
\label{subsec:omp_get_wtime}
\summary
The \code{omp\_get\_wtime} routine returns elapsed wall clock time in seconds.

\format
\ccppspecificstart
\begin{boxedcode}
double omp\_get\_wtime(void);
\end{boxedcode}
\ccppspecificend

\fortranspecificstart
\begin{boxedcode}
double precision function omp\_get\_wtime()
\end{boxedcode}
\fortranspecificend

\binding
The binding thread set for an \code{omp\_get\_wtime} region is the encountering thread. The 
routine’s return value is not guaranteed to be consistent across any set of threads.

\effect
The \code{omp\_get\_wtime} routine returns a value equal to the elapsed wall clock time in 
seconds since some ``time in the past''. The actual ``time in the past'' is arbitrary, but it is 
guaranteed not to change during the execution of the application program. The time 
returned is a ``per-thread time'', so it is not required to be globally consistent across all 
the threads participating in an application.

\notestart
\noteheader – It is anticipated that the routine will be used to measure elapsed times as shown 
in the following example:

\ccppspecificstart
\begin{boxedcode}
double start;
double end;
start = omp\_get\_wtime();
\plc{... work to be timed ...}
end = omp\_get\_wtime();
printf("Work took \%f seconds{\textbackslash}n", end - start);
\end{boxedcode}
\ccppspecificend

\fortranspecificstart
\begin{boxedcode}
DOUBLE PRECISION START, END
START = omp\_get\_wtime()
\plc{... work to be timed ...}
END = omp\_get\_wtime()
PRINT *, "Work took", END - START, "seconds"
\end{boxedcode}
\fortranspecificend
\noteend










\pagebreak
\subsection{\code{omp\_get\_wtick}}
\index{omp\_get\_wtick@{\code{omp\_get\_wtick}}}
\label{subsec:omp_get_wtick}
\summary
The \code{omp\_get\_wtick} routine returns the precision of the timer used by 
\code{omp\_get\_wtime}.

\format
\ccppspecificstart
\begin{boxedcode}
double omp\_get\_wtick(void);
\end{boxedcode}
\ccppspecificend

\fortranspecificstart
\begin{boxedcode}
double precision function omp\_get\_wtick()
\end{boxedcode}
\fortranspecificend

\binding
The binding thread set for an \code{omp\_get\_wtick} region is the encountering thread. The 
routine’s return value is not guaranteed to be consistent across any set of threads.

\effect
The \code{omp\_get\_wtick} routine returns a value equal to the number of seconds between 
successive clock ticks of the timer used by \code{omp\_get\_wtime}.


% This is the end of chapter 3 - Runtime Library

