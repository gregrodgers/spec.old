% This is ch3-runtimeLibrary.tex of the OpenMP specification.
% This is an included file. See the master file for more information.
%
% When editing this file:
%
%    1. To change formatting, appearance, or style, please edit openmp.sty.
%
%    2. Custom commands and macros are defined in openmp.sty.
%
%    3. Be kind to other editors -- keep a consistent style by copying-and-pasting to
%       create new content.
%
%    4. We use semantic markup, e.g. (see openmp.sty for a full list):
%         \code{}     % for bold monospace keywords, code, operators, etc.
%         \plc{}      % for italic placeholder names, grammar, etc.
%
%    5. Other recommendations:
%         Use the convenience macros defined in openmp.sty for the minor headers
%         such as Comments, Syntax, etc.
%
%         To keep items together on the same page, prefer the use of 
%         \begin{samepage}.... Avoid \parbox for text blocks as it interrupts line numbering.
%         When possible, avoid \filbreak, \pagebreak, \newpage, \clearpage unless that's
%         what you mean. Use \needspace{} cautiously for troublesome paragraphs.
%
%         Avoid absolute lengths and measures in this file; use relative units when possible.
%         Vertical space can be relative to \baselineskip or ex units. Horizontal space
%         can be relative to \linewidth or em units.
%
%         Prefer \emph{} to italicize terminology, e.g.:
%             This is a \emph{definition}, not a placeholder.
%             This is a \plc{var-name}.
%

\chapter{Runtime Library Routines}
\index{runtime library routines}
\label{chap:Runtime Library Routines}
This chapter describes the OpenMP API runtime library routines and queryable runtime states, and is divided into the 
following sections:

\begin{itemize}
\item Runtime library definitions 
(\specref{sec:runtime library definitions}).

\item Execution environment routines that can be used to control and to query the parallel 
execution environment 
(\specref{sec:Execution Environment Routines}).

\item Lock routines that can be used to synchronize access to data 
(\specref{sec:Lock Routines}). 

\item Portable timer routines 
(\specref{sec:Timing Routines}).

\item Device memory routines that can be used to allocate memory and 
to manage pointers on target devices (\specref{sec:Device Memory Routines}).

%\item Runtime states (\specref{sec:runtimeStates}).

% \item OMPT Query Functions (\specref{sec:omptFunctions}).

\item Execution routines to control the application monitoring
(\specref{sec:tool_tcontrol})
\end{itemize}

Throughout this chapter, \plc{true} and \plc{false} are used as generic terms to simplify the 
description of the routines. 

\begin{samepage}
\ccppspecificstart
\plc{true} means a nonzero integer value and \plc{false} means an integer value of zero. 
\ccppspecificend
\end{samepage}
\bigskip

\begin{samepage}
\fortranspecificstart
\plc{true} means a logical value of \code{.TRUE.} and \plc{false} means a logical value of \code{.FALSE.}.
\fortranspecificend
\end{samepage}
\bigskip

\begin{samepage}
\vspace{1\baselineskip}
\fortranspecificstart
\vspace{-1\baselineskip}
\restrictions

The following restriction applies to all OpenMP runtime library routines:

\begin{itemize}
\item OpenMP runtime library routines may not be called from \code{PURE} or \code{ELEMENTAL} 
procedures. 
\end{itemize}
\fortranspecificend
\end{samepage}











\section{Runtime Library Definitions}
\index{runtime library definitions}
\index{header files}
\index{include files}
\label{sec:runtime library definitions}
For each base language, a compliant implementation must supply a set of definitions for 
the OpenMP API runtime library routines and the special data types of their parameters. 
The set of definitions must contain a declaration for each OpenMP API runtime library 
routine and a declaration for the \emph{simple lock}, \emph{nestable lock}, \emph{schedule}, and \emph{thread affinity
policy} data types. In addition, each set of definitions may specify other implementation 
specific values.

\ccppspecificstart
The library routines are external functions with ``C'' linkage.

Prototypes for the C/C++ runtime library routines described in this chapter shall be 
provided in a header file named \code{omp.h}. This file defines the following: 

\begin{itemize}
\item The prototypes of all the routines in the chapter. 

\item The type \code{omp\_lock\_t}. 

\item The type \code{omp\_nest\_lock\_t}.

\item The type \code{omp\_lock\_hint\_t}. 

\item The type \code{omp\_sched\_t}.

\item The type \code{omp\_proc\_bind\_t}.

\item The type \code{omp\_tcontrol\_t}.

\end{itemize}

% was: C.1 on page 288 
See Section \specref{sec:Example of the omp.h Header File} for an example of this file.
\ccppspecificend

\fortranspecificstart
The OpenMP Fortran API runtime library routines are external procedures. The return 
values of these routines are of default kind, unless otherwise specified.

Interface declarations for the OpenMP Fortran runtime library routines described in this 
chapter shall be provided in the form of a Fortran \code{include} file named \code{omp\_lib.h} or 
a Fortran~90 \code{module} named \code{omp\_lib}. It is implementation defined whether the 
\code{include} file or the \code{module} file (or both) is provided.

These files define the following:

\begin{itemize}
\item The interfaces of all of the routines in this chapter.

\item The \code{integer} \code{parameter} \code{omp\_lock\_kind}.

\item The \code{integer} \code{parameter} \code{omp\_nest\_lock\_kind}.

\item The \code{integer} \code{parameter} \code{omp\_lock\_hint\_kind}.

\item The \code{integer} \code{parameter} \code{omp\_sched\_kind}.

\item The \code{integer} \code{parameter} \code{omp\_proc\_bind\_kind}.

\item The \code{integer} \code{parameter} \code{openmp\_version} with a value \plc{yyyymm} where \plc{yyyy} 
and \plc{mm} are the year and month designations of the version of the OpenMP Fortran 
API that the implementation supports. This value matches that of the C preprocessor 
macro \code{\_OPENMP}, when a macro preprocessor is supported (see 
\specref{sec:Conditional Compilation}).
\end{itemize}

See \specref{sec:Example of an Interface Declaration include File} 
and \specref{sec:Example of a Fortran Interface Declaration module} 
for examples of these files.

It is implementation defined whether any of the OpenMP runtime library routines that 
take an argument are extended with a generic interface so arguments of different \code{KIND} 
type can be accommodated. See Appendix~\ref{sec:Example of a Generic Interface for a Library Routine}
for an example of such an extension. 
\fortranspecificend









\section{Execution Environment Routines}
\index{execution environment routines}
\label{sec:Execution Environment Routines}
This section describes routines that affect and monitor threads, processors, and the 
parallel environment. 






\subsection{\code{omp\_set\_num\_threads}}
\index{omp\_set\_num\_threads@{\code{omp\_set\_num\_threads}}}
\label{subsec:omp_set_num_threads}
\summary
The \code{omp\_set\_num\_threads} routine affects the number of threads to be used for 
subsequent parallel regions that do not specify a \code{num\_threads} clause, by setting the 
value of the first element of the \plc{nthreads-var} ICV of the current task.

\format
\ccppspecificstart
\begin{boxedcode}
void omp\_set\_num\_threads(int \plc{num\_threads});
\end{boxedcode}
\ccppspecificend

\fortranspecificstart
\begin{boxedcode}
subroutine omp\_set\_num\_threads(\plc{num\_threads})
integer \plc{num\_threads}
\end{boxedcode}
\fortranspecificend

\constraints
The value of the argument passed to this routine must evaluate to a positive integer, or 
else the behavior of this routine is implementation defined.

\binding
The binding task set for an \code{omp\_set\_num\_threads} region is the generating task.

\effect
The effect of this routine is to set the value of the first element of the \plc{nthreads-var} ICV 
of the current task to the value specified in the argument. 

\crossreferences
\begin{itemize}
\item \plc{nthreads-var} ICV, see 
\specref{sec:Internal Control Variables}.

\item \code{parallel} construct and \code{num\_threads} clause, see 
\specref{sec:parallel Construct}.

\item Determining the number of threads for a \code{parallel} region, see
\specref{subsec:Determining the Number of Threads for a parallel Region}. 

\item \code{omp\_get\_max\_threads} routine, see 
\specref{subsec:omp_get_max_threads}.

\item \code{OMP\_NUM\_THREADS} environment variable, see 
\specref{sec:OMP_NUM_THREADS}.
\end{itemize}









\subsection{\code{omp\_get\_num\_threads}}
\index{omp\_get\_num\_threads@{\code{omp\_get\_num\_threads}}}
\label{subsec:omp_get_num_threads}
\summary
The \code{omp\_get\_num\_threads} routine returns the number of threads in the current 
team.
\format
\ccppspecificstart
\begin{boxedcode}
int omp\_get\_num\_threads(void); 
\end{boxedcode}
\ccppspecificend

\pagebreak
\fortranspecificstart
\begin{boxedcode}
integer function omp\_get\_num\_threads()
\end{boxedcode}
\fortranspecificend

\binding
The binding region for an \code{omp\_get\_num\_threads} region is the innermost enclosing 
\code{parallel} region.

\effect
The \code{omp\_get\_num\_threads} routine returns the number of threads in the team 
executing the \code{parallel} region to which the routine region binds. If called from the 
sequential part of a program, this routine returns 1. 

\crossreferences
\begin{itemize}
\item \code{parallel} construct, see 
\specref{sec:parallel Construct}.

\item Determining the number of threads for a \code{parallel} region, see
\specref{subsec:Determining the Number of Threads for a parallel Region}. 

\item \code{omp\_set\_num\_threads} routine, see 
\specref{subsec:omp_set_num_threads}.

\item \code{OMP\_NUM\_THREADS} environment variable, see 
\specref{sec:OMP_NUM_THREADS}.
\end{itemize}










\subsection{\code{omp\_get\_max\_threads}}
\index{omp\_get\_max\_threads@{\code{omp\_get\_max\_threads}}}
\label{subsec:omp_get_max_threads}
\summary
The \code{omp\_get\_max\_threads} routine returns an upper bound on the number of 
threads that could be used to form a new team if a \code{parallel} construct without a 
\code{num\_threads} clause were encountered after execution returns from this routine.

\pagebreak
\format
\ccppspecificstart
\begin{boxedcode}
int omp\_get\_max\_threads(void);
\end{boxedcode}
\ccppspecificend

\fortranspecificstart
\begin{boxedcode}
integer function omp\_get\_max\_threads()
\end{boxedcode}
\fortranspecificend

\binding
The binding task set for an \code{omp\_get\_max\_threads} region is the generating task. 

\effect
The value returned by \code{omp\_get\_max\_threads} is the value of the first element of 
the \plc{nthreads-var} ICV of the current task. This value is also an upper bound on the 
number of threads that could be used to form a new team if a parallel region without a 
\code{num\_threads} clause were encountered after execution returns from this routine.

\notestart
\noteheader – The return value of the \code{omp\_get\_max\_threads} routine can be used to 
dynamically allocate sufficient storage for all threads in the team formed at the 
subsequent active \code{parallel} region.
\noteend

\crossreferences
\begin{itemize}
\item \plc{nthreads-var} ICV, see 
\specref{sec:Internal Control Variables}.

\item \code{parallel} construct, see 
\specref{sec:parallel Construct}.

\item \code{num\_threads} clause, see 
\specref{sec:parallel Construct}.

\item Determining the number of threads for a \code{parallel} region, see
\specref{subsec:Determining the Number of Threads for a parallel Region}. 

\item \code{omp\_set\_num\_threads} routine, see 
\specref{subsec:omp_set_num_threads}.

\item \code{OMP\_NUM\_THREADS} environment variable, see 
\specref{sec:OMP_NUM_THREADS}.
\end{itemize}










\subsection{\code{omp\_get\_thread\_num}}
\index{omp\_get\_thread\_num@{\code{omp\_get\_thread\_num}}}
\label{subsec:omp_get_thread_num}
\summary
The \code{omp\_get\_thread\_num} routine returns the thread number, within the current 
team, of the calling thread.

\format
\ccppspecificstart
\begin{boxedcode}
int omp\_get\_thread\_num(void); 
\end{boxedcode}
\ccppspecificend

\fortranspecificstart
\begin{boxedcode}
integer function omp\_get\_thread\_num() 
\end{boxedcode}
\fortranspecificend

\binding
The binding thread set for an \code{omp\_get\_thread\_num} region is the current team. The 
binding region for an \code{omp\_get\_thread\_num} region is the innermost enclosing 
\code{parallel} region. 

\effect
The \code{omp\_get\_thread\_num} routine returns the thread number of the calling thread, 
within the team executing the \code{parallel} region to which the routine region binds. The 
thread number is an integer between 0 and one less than the value returned by 
\code{omp\_get\_num\_threads}, inclusive. The thread number of the master thread of the 
team is 0. The routine returns 0 if it is called from the sequential part of a program.

\notestart
\noteheader – The thread number may change during the execution of an untied task. The 
value returned by \code{omp\_get\_thread\_num} is not generally useful during the execution 
of such a task region.
\noteend

\crossreferences
\begin{itemize}
\item \code{omp\_get\_num\_threads} routine, see 
\specref{subsec:omp_get_num_threads}.
\end{itemize}








\subsection{\code{omp\_get\_num\_procs}}
\index{omp\_get\_num\_procs@{\code{omp\_get\_num\_procs}}}
\label{subsec:omp_get_num_procs}
\summary
The \code{omp\_get\_num\_procs} routine returns the number of processors available to the 
device.

\format
\ccppspecificstart
\begin{boxedcode}
int omp\_get\_num\_procs(void);
\end{boxedcode}
\ccppspecificend

\fortranspecificstart
\begin{boxedcode}
integer function omp\_get\_num\_procs()
\end{boxedcode}
\fortranspecificend

\binding
The binding thread set for an \code{omp\_get\_num\_procs} region is all threads on a device. 
The effect of executing this routine is not related to any specific region corresponding to 
any construct or API routine.

\effect
The \code{omp\_get\_num\_procs} routine returns the number of processors that are available 
to the device at the time the routine is called. This value may change between 
the time that it is determined by the \code{omp\_get\_num\_procs} routine and the time that it 
is read in the calling context due to system actions outside the control of the OpenMP 
implementation.

\crossreferences
None.







\subsection{\code{omp\_in\_parallel}}
\index{omp\_in\_parallel@{\code{omp\_in\_parallel}}}
\label{subsec:omp_in_parallel}
\summary
The \code{omp\_in\_parallel} routine returns \plc{true} if the \plc{active-levels-var} ICV is greater 
than zero; otherwise, it returns \plc{false}.

\pagebreak
\format
\ccppspecificstart
\begin{boxedcode}
int omp\_in\_parallel(void);
\end{boxedcode}
\ccppspecificend

\fortranspecificstart
\begin{boxedcode}
logical function omp\_in\_parallel()
\end{boxedcode}
\fortranspecificend

\binding
The binding task set for an \code{omp\_in\_parallel} region is the generating task.

\effect
The effect of the \code{omp\_in\_parallel} routine is to return \plc{true} if the current task is 
enclosed by an active \code{parallel} region, and the \code{parallel} region is enclosed by the 
outermost initial task region on the device; otherwise it returns \plc{false}.

\crossreferences
\begin{itemize}
\item \plc{active-levels-var}, see 
\specref{sec:Internal Control Variables}.

\item \code{parallel} construct, see 
\specref{sec:parallel Construct}.

\item \code{omp\_get\_active\_level} routine, see 
\specref{subsec:omp_get_active_level}.
\end{itemize}








\bigskip
\subsection{\code{omp\_set\_dynamic}}
\index{omp\_set\_dynamic@{\code{omp\_set\_dynamic}}}
\label{subsec:omp_set_dynamic}
\summary
The \code{omp\_set\_dynamic} routine enables or disables dynamic adjustment of the 
number of threads available for the execution of subsequent \code{parallel} regions by 
setting the value of the \plc{dyn-var} ICV.


\pagebreak
\format
\ccppspecificstart
\begin{boxedcode}
void omp\_set\_dynamic(int \plc{dynamic\_threads});
\end{boxedcode}
\ccppspecificend
\bigskip

\begin{samepage}
\fortranspecificstart
\begin{boxedcode}
subroutine omp\_set\_dynamic(\plc{dynamic\_threads})
logical \plc{dynamic\_threads}
\end{boxedcode}
\fortranspecificend
\end{samepage}

\binding
The binding task set for an \code{omp\_set\_dynamic} region is the generating task. 

\effect
For implementations that support dynamic adjustment of the number of threads, if the 
argument to \code{omp\_set\_dynamic} evaluates to \plc{true}, dynamic adjustment is enabled for 
the current task; otherwise, dynamic adjustment is disabled for the current task. For 
implementations that do not support dynamic adjustment of the number of threads this 
routine has no effect: the value of \plc{dyn-var} remains \plc{false}.

\crossreferences
\begin{itemize}
\item \plc{dyn-var} ICV, see 
\specref{sec:Internal Control Variables}.

\item Determining the number of threads for a \code{parallel} region, see
\specref{subsec:Determining the Number of Threads for a parallel Region}. 

\item \code{omp\_get\_num\_threads} routine, see 
\specref{subsec:omp_get_num_threads}.

\item \code{omp\_get\_dynamic} routine, see 
\specref{subsec:omp_get_dynamic}.

\item \code{OMP\_DYNAMIC} environment variable, see 
\specref{sec:OMP_DYNAMIC}.
\end{itemize}








\subsection{\code{omp\_get\_dynamic}}
\index{omp\_get\_dynamic@{\code{omp\_get\_dynamic}}}
\label{subsec:omp_get_dynamic}
\summary
The \code{omp\_get\_dynamic} routine returns the value of the \plc{dyn-var} ICV, which 
determines whether dynamic adjustment of the number of threads is enabled or disabled.

\format
\ccppspecificstart
\begin{boxedcode}
int omp\_get\_dynamic(void);
\end{boxedcode}
\ccppspecificend

\fortranspecificstart
\begin{boxedcode}
logical function omp\_get\_dynamic()
\end{boxedcode}
\fortranspecificend

\binding
The binding task set for an \code{omp\_get\_dynamic} region is the generating task. 

\effect
This routine returns \plc{true} if dynamic adjustment of the number of threads is enabled for 
the current task; it returns \plc{false}, otherwise. If an implementation does not support 
dynamic adjustment of the number of threads, then this routine always returns \plc{false}.

\crossreferences
\begin{itemize}
\item \plc{dyn-var} ICV, see 
\specref{sec:Internal Control Variables}.

\item Determining the number of threads for a \code{parallel} region, see
\specref{subsec:Determining the Number of Threads for a parallel Region}. 

\item \code{omp\_set\_dynamic} routine, see 
\specref{subsec:omp_set_dynamic}.

\item \code{OMP\_DYNAMIC} environment variable, see 
\specref{sec:OMP_DYNAMIC}.
\end{itemize}









\subsection{\code{omp\_get\_cancellation}}
\index{omp\_get\_cancellation@{\code{omp\_get\_cancellation}}}
\label{subsec:omp_get_cancellation}
\summary
The \code{omp\_get\_cancellation} routine returns the value of the \plc{cancel-var} ICV, which determines if cancellation is enabled or disabled.

\format
\ccppspecificstart
\begin{boxedcode}
int omp\_get\_cancellation(void);
\end{boxedcode}
\ccppspecificend

\fortranspecificstart
\begin{boxedcode}
logical function omp\_get\_cancellation()
\end{boxedcode}
\fortranspecificend

\binding
The binding task set for an \code{omp\_get\_cancellation} region is the whole program.

\effect
This routine returns \plc{true} if cancellation is enabled. It returns \plc{false} otherwise.

\crossreferences
\begin{itemize}
\item \plc{cancel-var} ICV, see 
\specref{subsec:ICV Descriptions}.

\item \code{cancel} construct, see \specref{subsec:cancel Construct}

\item \code{OMP\_CANCELLATION} environment variable, see 
\specref{sec:OMP_CANCELLATION}
\end{itemize}









\subsection{\code{omp\_set\_nested}}
\index{omp\_set\_nested@{\code{omp\_set\_nested}}}
\label{subsec:omp_set_nested}
\summary
The \code{omp\_set\_nested} routine enables or disables nested parallelism, by setting the 
\plc{nest-var} ICV. 

\format
\ccppspecificstart
\begin{boxedcode}
void omp\_set\_nested(int \plc{nested});
\end{boxedcode}
\ccppspecificend

\fortranspecificstart
\begin{boxedcode}
subroutine omp\_set\_nested(\plc{nested})
logical \plc{nested}
\end{boxedcode}
\fortranspecificend

\binding
The binding task set for an \code{omp\_set\_nested} region is the generating task. 

\effect
For implementations that support nested parallelism, if the argument to 
\code{omp\_set\_nested} evaluates to \plc{true}, nested parallelism is enabled for the current task; 
otherwise, nested parallelism is disabled for the current task. For implementations that 
do not support nested parallelism, this routine has no effect: the value of \plc{nest-var} 
remains \plc{false}.

\crossreferences
\begin{itemize}
\item \plc{nest-var} ICV, see 
\specref{sec:Internal Control Variables}.

\item Determining the number of threads for a \code{parallel} region, see
\specref{subsec:Determining the Number of Threads for a parallel Region}. 

\item \code{omp\_set\_max\_active\_levels} routine, see 
\specref{subsec:omp_set_max_active_levels}.

\item \code{omp\_get\_max\_active\_levels} routine, see 
\specref{subsec:omp_get_max_active_levels}.

\item \code{omp\_get\_nested} routine, see 
\specref{subsec:omp_get_nested}.

\item \code{OMP\_NESTED} environment variable, see 
\specref{sec:OMP_NESTED}.
\end{itemize}








\subsection{\code{omp\_get\_nested}}
\index{omp\_get\_nested@{\code{omp\_get\_nested}}}
\label{subsec:omp_get_nested}
\summary
The \code{omp\_get\_nested} routine returns the value of the \plc{nest-var} ICV, which 
determines if nested parallelism is enabled or disabled.

\format
\ccppspecificstart
\begin{boxedcode}
int omp\_get\_nested(void);
\end{boxedcode}
\ccppspecificend

\fortranspecificstart
\begin{boxedcode}
logical function omp\_get\_nested()
\end{boxedcode}
\fortranspecificend

\binding
The binding task set for an \code{omp\_get\_nested} region is the generating task. 

\effect
This routine returns \plc{true} if nested parallelism is enabled for the current task; it returns 
\plc{false}, otherwise. If an implementation does not support nested parallelism, this routine 
always returns \plc{false}.

\crossreferences
\begin{itemize}
\item \plc{nest-var} ICV, see 
\specref{sec:Internal Control Variables}.

\item Determining the number of threads for a \code{parallel} region, see
\specref{subsec:Determining the Number of Threads for a parallel Region}. 

\item \code{omp\_set\_nested} routine, see 
\specref{subsec:omp_set_nested}.

\item \code{OMP\_NESTED} environment variable, see 
\specref{sec:OMP_NESTED}.
\end{itemize}









\subsection{\code{omp\_set\_schedule}}
\index{omp\_set\_schedule@{\code{omp\_set\_schedule}}}
\label{subsec:omp_set_schedule}
\summary
The \code{omp\_set\_schedule} routine affects the schedule that is applied when \code{runtime} 
is used as schedule kind, by setting the value of the \plc{run-sched-var} ICV. 

\format
\ccppspecificstart
\begin{boxedcode}
void omp\_set\_schedule(omp\_sched\_t \plc{kind}, int \plc{chunk\_size});
\end{boxedcode}
\ccppspecificend

\fortranspecificstart
\begin{boxedcode}
subroutine omp\_set\_schedule(\plc{kind}, \plc{chunk\_size}) 
integer (kind=omp\_sched\_kind) \plc{kind}
integer \plc{chunk\_size}
\end{boxedcode}
\fortranspecificend

\constraints
The first argument passed to this routine can be one of the valid OpenMP schedule kinds 
(except for \code{runtime}) or any implementation specific schedule. The C/C++ header file 
(\code{omp.h}) and the Fortran include file (\code{omp\_lib.h}) and/or Fortran~90 module file 
(\code{omp\_lib}) define the valid constants. The valid constants must include the following, 
which can be extended with implementation specific values:

\pagebreak
\ccppspecificstart
\begin{boxedcode}
typedef enum omp\_sched\_t \{
    omp\_sched\_static = 1,
    omp\_sched\_dynamic = 2,
    omp\_sched\_guided = 3,
    omp\_sched\_auto = 4
\} omp\_sched\_t;
\end{boxedcode}
\ccppspecificend

\begin{samepage}
\fortranspecificstart
\begin{boxedcode}
integer(kind=omp\_sched\_kind), parameter :: omp\_sched\_static = 1
integer(kind=omp\_sched\_kind), parameter :: omp\_sched\_dynamic = 2
integer(kind=omp\_sched\_kind), parameter :: omp\_sched\_guided = 3
integer(kind=omp\_sched\_kind), parameter :: omp\_sched\_auto = 4
\end{boxedcode}
\fortranspecificend
\end{samepage}

\binding
The binding task set for an \code{omp\_set\_schedule} region is the generating task. 

\effect
The effect of this routine is to set the value of the \plc{run-sched-var} ICV of the current task 
to the values specified in the two arguments. The schedule is set to the schedule type 
specified by the first argument \plc{kind}. It can be any of the standard schedule types or 
any other implementation specific one. For the schedule types \code{static}, \code{dynamic}, and 
\code{guided} the \plc{chunk\_size} is set to the value of the second argument, or to the default 
\plc{chunk\_size} if the value of the second argument is less than 1; for the schedule type 
\code{auto} the second argument has no meaning; for implementation specific schedule types, 
the values and associated meanings of the second argument are implementation defined.

\crossreferences
\begin{itemize}
\item \plc{run-sched-var} ICV, see 
\specref{sec:Internal Control Variables}.

\item Determining the schedule of a worksharing loop, see 
\specref{subsubsec:Determining the Schedule of a Worksharing Loop}.

\item \code{omp\_get\_schedule} routine, see 
\specref{subsec:omp_get_schedule}.

\item \code{OMP\_SCHEDULE} environment variable, see 
\specref{sec:OMP_SCHEDULE}.
\end{itemize}









\subsection{\code{omp\_get\_schedule}}
\index{omp\_get\_schedule@{\code{omp\_get\_schedule}}}
\label{subsec:omp_get_schedule}
\summary
The \code{omp\_get\_schedule} routine returns the schedule that is applied when the 
runtime schedule is used. 
\format
\ccppspecificstart
\begin{boxedcode}
void omp\_get\_schedule(omp\_sched\_t * \plc{kind}, int * \plc{chunk\_size}); 
\end{boxedcode}
\ccppspecificend

\fortranspecificstart
\begin{boxedcode}
subroutine omp\_get\_schedule(\plc{kind}, \plc{chunk\_size}) 
integer (kind=omp\_sched\_kind) \plc{kind} 
integer \plc{chunk\_size}
\end{boxedcode}
\fortranspecificend

\binding
The binding task set for an \code{omp\_get\_schedule} region is the generating task. 

\effect
This routine returns the \plc{run-sched-var} ICV in the task to which the routine binds. The 
first argument \plc{kind} returns the schedule to be used. It can be any of the standard 
schedule types as defined in 
\specref{subsec:omp_set_schedule}, 
or any implementation specific 
schedule type. The second argument is interpreted as in the \code{omp\_set\_schedule} call, 
defined in 
\specref{subsec:omp_set_schedule}.

\crossreferences
\begin{itemize}
\item \plc{run-sched-var} ICV, see 
\specref{sec:Internal Control Variables}.

\item Determining the schedule of a worksharing loop, see 
\specref{subsubsec:Determining the Schedule of a Worksharing Loop}.

\item \code{omp\_set\_schedule} routine, see 
\specref{subsec:omp_set_schedule}.

\item \code{OMP\_SCHEDULE} environment variable, see 
\specref{sec:OMP_SCHEDULE}.
\end{itemize}









\subsection{\code{omp\_get\_thread\_limit}}
\index{omp\_get\_thread\_limit@{\code{omp\_get\_thread\_limit}}}
\label{subsec:omp_get_thread_limit}
\summary
The \code{omp\_get\_thread\_limit} routine returns the maximum number of OpenMP 
threads available to participate in the current contention group. 

\format
\ccppspecificstart
\begin{boxedcode}
int omp\_get\_thread\_limit(void);
\end{boxedcode}
\ccppspecificend

\fortranspecificstart
\begin{boxedcode}
integer function omp\_get\_thread\_limit()
\end{boxedcode}
\fortranspecificend

\binding
The binding thread set for an \code{omp\_get\_thread\_limit} region is all threads on the 
device. The effect of executing this routine is not related to any specific region 
corresponding to any construct or API routine. 

\effect
The \code{omp\_get\_thread\_limit} routine returns the value of the \plc{thread-limit-var} ICV.

\crossreferences
\begin{itemize}
\item \plc{thread-limit-var} ICV, see 
\specref{sec:Internal Control Variables}.

\item \code{OMP\_THREAD\_LIMIT} environment variable, see 
\specref{sec:OMP_THREAD_LIMIT}.
\end{itemize}









\subsection{\code{omp\_set\_max\_active\_levels}}
\index{omp\_set\_max\_active\_levels@{\code{omp\_set\_max\_active\_levels}}}
\label{subsec:omp_set_max_active_levels}
\summary
The \code{omp\_set\_max\_active\_levels} routine limits the number of nested active 
parallel regions on the device, by setting the \plc{max-active-levels-var} ICV

\format
\ccppspecificstart
\begin{boxedcode}
void omp\_set\_max\_active\_levels(int \plc{max\_levels});
\end{boxedcode}
\ccppspecificend

\fortranspecificstart
\begin{boxedcode}
subroutine omp\_set\_max\_active\_levels(\plc{max\_levels})
integer \plc{max\_levels}
\end{boxedcode}
\fortranspecificend

\constraints
The value of the argument passed to this routine must evaluate to a non-negative integer, 
otherwise the behavior of this routine is implementation defined.

\binding
When called from a sequential part of the program, the binding thread set for an 
\code{omp\_set\_max\_active\_levels} region is the encountering thread. When called 
from within any explicit parallel region, the binding thread set (and binding region, if 
required) for the \code{omp\_set\_max\_active\_levels} region is implementation defined. 

\effect
The effect of this routine is to set the value of the \plc{max-active-levels-var} ICV to the value 
specified in the argument. 

If the number of parallel levels requested exceeds the number of levels of parallelism 
supported by the implementation, the value of the \plc{max-active-levels-var} ICV will be set 
to the number of parallel levels supported by the implementation.

This routine has the described effect only when called from a sequential part of the 
program. When called from within an explicit \code{parallel} region, the effect of this 
routine is implementation defined.

\crossreferences
\begin{itemize}
\item \plc{max-active-levels-var} ICV, see 
\specref{sec:Internal Control Variables}.

\item \code{omp\_get\_max\_active\_levels} routine, see 
\specref{subsec:omp_get_max_active_levels}.

\item \code{OMP\_MAX\_ACTIVE\_LEVELS} environment variable, see 
\specref{sec:OMP_MAX_ACTIVE_LEVELS}.
\end{itemize}










\subsection{\code{omp\_get\_max\_active\_levels}}
\index{omp\_get\_max\_active\_levels@{\code{omp\_get\_max\_active\_levels}}}
\label{subsec:omp_get_max_active_levels}
\summary
The \code{omp\_get\_max\_active\_levels} routine returns the value of the 
\plc{max-active-levels-var} ICV, which determines the maximum number of nested active parallel regions 
on the device. 

\format
\ccppspecificstart
\begin{boxedcode}
int omp\_get\_max\_active\_levels(void);
\end{boxedcode}
\ccppspecificend

\fortranspecificstart
\begin{boxedcode}
integer function omp\_get\_max\_active\_levels()
\end{boxedcode}
\fortranspecificend

\binding
When called from a sequential part of the program, the binding thread set for an 
\code{omp\_get\_max\_active\_levels} region is the encountering thread. When called 
from within any explicit parallel region, the binding thread set (and binding region, if 
required) for the \code{omp\_get\_max\_active\_levels} region is implementation defined. 

\effect
The \code{omp\_get\_max\_active\_levels} routine returns the value of the \plc{max-active-levels-var} ICV,
which determines the maximum number of nested active parallel regions 
on the device. 

\crossreferences
\begin{itemize}
\item \plc{max-active-levels-var} ICV, see 
\specref{sec:Internal Control Variables}.

\item \code{omp\_set\_max\_active\_levels} routine, see 
\specref{subsec:omp_set_max_active_levels}.

\item \code{OMP\_MAX\_ACTIVE\_LEVELS} environment variable, see 
\specref{sec:OMP_MAX_ACTIVE_LEVELS}.
\end{itemize}








\subsection{\code{omp\_get\_level}}
\index{omp\_get\_level@{\code{omp\_get\_level}}}
\label{subsec:omp_get_level}
\summary
The \code{omp\_get\_level} routine returns the value of the \plc{levels-var} ICV. 

\format
\ccppspecificstart
\begin{boxedcode}
int omp\_get\_level(void);
\end{boxedcode}
\ccppspecificend

\fortranspecificstart
\begin{boxedcode}
integer function omp\_get\_level()
\end{boxedcode}
\fortranspecificend

\binding
The binding task set for an \code{omp\_get\_level} region is the generating task. 

\effect
The effect of the \code{omp\_get\_level} routine is to return the number of nested 
\code{parallel} regions (whether active or inactive) enclosing the current task such that all 
of the \code{parallel} regions are enclosed by the outermost initial task region on the 
current device.

\crossreferences
\begin{itemize}
\item \plc{levels-var} ICV, see 
\specref{sec:Internal Control Variables}.

\item \code{omp\_get\_active\_level} routine, see 
\specref{subsec:omp_get_active_level}.

\item \code{OMP\_MAX\_ACTIVE\_LEVELS} environment variable, see 
\specref{sec:OMP_MAX_ACTIVE_LEVELS}.
\end{itemize}










\subsection{\code{omp\_get\_ancestor\_thread\_num}}
\index{omp\_get\_ancestor\_thread\_num@{\code{omp\_get\_ancestor\_thread\_num}}}
\label{subsec:omp_get_ancestor_thread_num}
\summary
The \code{omp\_get\_ancestor\_thread\_num} routine returns, for a given nested level of 
the current thread, the thread number of the ancestor of the current thread.

\begin{samepage}
\format
\ccppspecificstart
\begin{boxedcode}
int omp\_get\_ancestor\_thread\_num(int \plc{level});
\end{boxedcode}
\ccppspecificend
\end{samepage}

\fortranspecificstart
\begin{boxedcode}
integer function omp\_get\_ancestor\_thread\_num(\plc{level})
integer \plc{level}
\end{boxedcode}
\fortranspecificend

\binding
The binding thread set for an \code{omp\_get\_ancestor\_thread\_num} region is the 
encountering thread. The binding region for an \code{omp\_get\_ancestor\_thread\_num} 
region is the innermost enclosing \code{parallel} region. 

\effect
The \code{omp\_get\_ancestor\_thread\_num} routine returns the thread number of the 
ancestor at a given nest level of the current thread or the thread number of the current 
thread. If the requested nest level is outside the range of 0 and the nest level of the 
current thread, as returned by the \code{omp\_get\_level} routine, the routine returns -1.

\notestart
\noteheader – When the \code{omp\_get\_ancestor\_thread\_num} routine is called with a value 
of \code{level}=0, the routine always returns 0. If \code{level}=\code{omp\_get\_level()}, the routine 
has the same effect as the \code{omp\_get\_thread\_num} routine. 
\noteend

\crossreferences
\begin{itemize}
\item \code{omp\_get\_thread\_num} routine, see 
\specref{subsec:omp_get_thread_num}.

\item \code{omp\_get\_level} routine, see 
\specref{subsec:omp_get_level}.

\item \code{omp\_get\_team\_size} routine, see 
\specref{subsec:omp_get_team_size}.
\end{itemize}










\subsection{\code{omp\_get\_team\_size}}
\index{omp\_get\_team\_size@{\code{omp\_get\_team\_size}}}
\label{subsec:omp_get_team_size}
\summary
The \code{omp\_get\_team\_size} routine returns, for a given nested level of the current 
thread, the size of the thread team to which the ancestor or the current thread belongs. 

\format
\ccppspecificstart
\begin{boxedcode}
int omp\_get\_team\_size(int \plc{level});
\end{boxedcode}
\ccppspecificend

\fortranspecificstart
\begin{boxedcode}
integer function omp\_get\_team\_size(\plc{level})
integer \plc{level}
\end{boxedcode}
\fortranspecificend

\binding
The binding thread set for an \code{omp\_get\_team\_size} region is the encountering 
thread. The binding region for an \code{omp\_get\_team\_size} region is the innermost 
enclosing \code{parallel} region.

\effect
The \code{omp\_get\_team\_size} routine returns the size of the thread team to which the 
ancestor or the current thread belongs. If the requested nested level is outside the range 
of 0 and the nested level of the current thread, as returned by the \code{omp\_get\_level} 
routine, the routine returns -1. Inactive parallel regions are regarded like active parallel 
regions executed with one thread. 

\notestart
\noteheader – When the \code{omp\_get\_team\_size} routine is called with a value of \code{level}=0, 
the routine always returns 1. If \code{level}=\code{omp\_get\_level()}, the routine has the same 
effect as the\linebreak \code{omp\_get\_num\_threads} routine. 
\noteend

\crossreferences
\begin{itemize}
\item \code{omp\_get\_num\_threads} routine, see 
\specref{subsec:omp_get_num_threads}.

\item \code{omp\_get\_level} routine, see 
\specref{subsec:omp_get_level}.

\item \code{omp\_get\_ancestor\_thread\_num} routine, see 
\specref{subsec:omp_get_ancestor_thread_num}.
\end{itemize}









\subsection{\code{omp\_get\_active\_level}}
\index{omp\_get\_active\_level@{\code{omp\_get\_active\_level}}}
\label{subsec:omp_get_active_level}
\summary
The \code{omp\_get\_active\_level} routine returns the value of the \plc{active-level-vars} ICV..

\format
\ccppspecificstart
\begin{boxedcode}
int \code{omp\_get\_active\_level}(void);
\end{boxedcode}
\ccppspecificend

\pagebreak
\fortranspecificstart
\begin{boxedcode}
integer function omp\_get\_active\_level()
\end{boxedcode}
\fortranspecificend

\binding
The binding task set for the an \code{omp\_get\_active\_level} region is the generating 
task. 

\effect
The effect of the \code{omp\_get\_active\_level} routine is to return the number of nested, 
active \code{parallel} regions enclosing the current task such that all of the \code{parallel} 
regions are enclosed by the outermost initial task region on the current device. 

\crossreferences
\begin{itemize}
\item \plc{active-levels-var} ICV, see 
\specref{sec:Internal Control Variables}.

\item \code{omp\_get\_level} routine, see 
\specref{subsec:omp_get_level}. 
\end{itemize}











\subsection{\code{omp\_in\_final}}
\index{omp\_in\_final@{\code{omp\_in\_final}}}
\label{subsec:omp_in_final}
\summary
The \code{omp\_in\_final} routine returns \plc{true} if the routine is executed in a final task 
region; otherwise, it returns \plc{false}.

\format
\ccppspecificstart
\begin{boxedcode}
int omp\_in\_final(void);
\end{boxedcode}
\ccppspecificend

\fortranspecificstart
\begin{boxedcode}
logical function omp\_in\_final()
\end{boxedcode}
\fortranspecificend

\binding
The binding task set for an \code{omp\_in\_final} region is the generating task.

\effect
\code{omp\_in\_final} returns \plc{true} if the enclosing task region is final. Otherwise, it returns 
\plc{false}.

\crossreferences
\begin{itemize}
\item \code{task} construct, see 
\specref{subsec:task Construct}. 
\end{itemize}









\subsection{\code{omp\_get\_proc\_bind}}
\index{omp\_get\_proc\_bind@{\code{omp\_get\_proc\_bind}}}
\label{subsec:omp_get_proc_bind}
\summary
The \code{omp\_get\_proc\_bind} routine returns the thread affinity policy to be used for the 
subsequent nested \code{parallel} regions that do not specify a \code{proc\_bind} clause.

\format
\ccppspecificstart
\begin{boxedcode}
omp\_proc\_bind\_t omp\_get\_proc\_bind(void);
\end{boxedcode}
\ccppspecificend

\fortranspecificstart
\begin{boxedcode}
integer (kind=omp\_proc\_bind\_kind) function omp\_get\_proc\_bind()
\end{boxedcode}
\fortranspecificend

\constraints
The value returned by this routine must be one of the valid affinity policy kinds. The C/
C++ header file (\code{omp.h}) and the Fortran include file (\code{omp\_lib.h}) and/or Fortran~90 
module file (\code{omp\_lib}) define the valid constants. The valid constants must include the 
following:

\ccppspecificstart
\begin{codepar}
typedef enum omp\_proc\_bind\_t \{
  omp\_proc\_bind\_false = 0,
  omp\_proc\_bind\_true = 1,
  omp\_proc\_bind\_master = 2,
  omp\_proc\_bind\_close = 3,
  omp\_proc\_bind\_spread = 4
\} omp\_proc\_bind\_t;
\end{codepar}
\ccppspecificend

\fortranspecificstart
\begin{codepar}
integer (kind=omp\_proc\_bind\_kind), &
                parameter :: omp\_proc\_bind\_false = 0
integer (kind=omp\_proc\_bind\_kind), &
                parameter :: omp\_proc\_bind\_true = 1
integer (kind=omp\_proc\_bind\_kind), &
                parameter :: omp\_proc\_bind\_master = 2
integer (kind=omp\_proc\_bind\_kind), &
                parameter :: omp\_proc\_bind\_close = 3
integer (kind=omp\_proc\_bind\_kind), &
                parameter :: omp\_proc\_bind\_spread = 4
\end{codepar}
\fortranspecificend

\binding
The binding task set for an \code{omp\_get\_proc\_bind} region is the generating task

\effect
The effect of this routine is to return the value of the first element of the \plc{bind-var} ICV 
of the current task. See \specref{subsec:Controlling OpenMP Thread Affinity} 
for the rules governing the thread affinity policy.

\crossreferences
\begin{itemize}
\item \plc{bind-var} ICV, see 
\specref{sec:Internal Control Variables}.

\item Controlling OpenMP thread affinity, see 
\specref{subsec:Controlling OpenMP Thread Affinity}. 

\item \code{OMP\_PROC\_BIND} environment variable, see 
\specref{sec:OMP_PROC_BIND}.
\end{itemize}



%%%%%%%%%% begin 392

\subsection{\code{omp\_get\_num\_places}}
\index{omp\_get\_num\_places@{\code{omp\_get\_num\_places}}}
\label{subsec:omp_get_num_places}
\summary
The \code{omp\_get\_num\_places} routine returns the number of places 
available to the execution environment in the place list.

\format
\ccppspecificstart
\begin{boxedcode}
int omp\_get\_num\_places(void);
\end{boxedcode}
\ccppspecificend

\fortranspecificstart
\begin{boxedcode}
integer function omp\_get\_num\_places()
\end{boxedcode}
\fortranspecificend

\binding
The binding thread set for an \code{omp\_get\_num\_places}  region is all threads on a device. The effect of executing this routine is not related to any specific region corresponding to any construct or API routine.

\effect

The \code{omp\_get\_num\_places} routine returns the number of places in the place list. This value is equivalent to the number of places in the  \plc{place-partition-var} ICV in the execution environment of the initial task.

\crossreferences
\begin{itemize}
\item \plc{place-partition-var} ICV, see 
\specref{sec:Internal Control Variables}.

\item \code{OMP\_PLACES} environment variable, see 
\specref{sec:OMP_PLACES}.
\end{itemize}





\subsection{\code{omp\_get\_place\_num\_procs}}
\index{omp\_get\_place\_num\_procs@{\code{omp\_get\_place\_num\_procs}}}
\label{subsec:omp_get_place_num_procs}

\summary
The \code{omp\_get\_place\_num\_procs}  routine returns the number of processors available to the execution environment in the specified place.

\format
\ccppspecificstart
\begin{boxedcode}
int omp\_get\_place\_num\_procs(int \plc{place\_num});
\end{boxedcode}
\ccppspecificend

\fortranspecificstart
\begin{boxedcode}
integer function omp\_get\_place\_num\_procs(\plc{place\_num})
integer \plc{place\_num}
\end{boxedcode}
\fortranspecificend

\binding
The binding thread set for an \code{omp\_get\_place\_num\_procs}  region is all threads on a device. The effect of executing this routine is not related to any specific region corresponding to any construct or API routine.

\effect
The \code{omp\_get\_place\_num\_procs} routine returns the number of 
processors associated with the place numbered \plc{place\_num}. The 
routine returns zero when \plc{place\_num} is negative, or is equal 
to or larger than the value returned by \code{omp\_get\_num\_places()}. 

\crossreferences
\begin{itemize}
\item \code{OMP\_PLACES} environment variable, see 
\specref{sec:OMP_PLACES}.
\end{itemize}




\subsection{\code{omp\_get\_place\_proc\_ids}}
\index{omp\_get\_place\_proc\_ids@{\code{omp\_get\_place\_proc\_ids}}}
\label{subsec:omp_get_place_proc_ids}

\summary
The \code{omp\_get\_place\_proc\_ids} routine returns the numerical identifiers of the processors available to the execution environment in the specified place.

\format
\ccppspecificstart
\begin{boxedcode}
void omp\_get\_place\_proc\_ids(int \plc{place\_num}, int *\plc{ids});
\end{boxedcode}
\ccppspecificend

\fortranspecificstart
\begin{boxedcode}
subroutine omp\_get\_place\_proc\_ids(\plc{place_num}, \plc{ids})
integer \plc{place\_num}
integer \plc{ids}(*)
\end{boxedcode}
\fortranspecificend

\binding
The binding thread set for an \code{omp\_get\_place\_proc\_ids} region is all 
threads on a device. The effect of executing this routine is not related to 
any specific region corresponding to any construct or API routine.

\effect
The \code{omp\_get\_place\_proc\_ids} routine returns the numerical 
identifiers of each processor associated with the place numbered 
\plc{place\_num}. The numerical identifiers are non-negative, and 
their meaning is implementation defined.  The numerical identifiers 
are returned in the array \plc{ids} and their order in the array is 
implementation defined. The array must be sufficiently large to contain 
\code{omp\_get\_place\_num\_procs(}\plc{place\_num}\code{)} integers; 
otherwise, the behavior is unspecified.  The routine has no effect when 
\plc{place\_num} has a negative value, or a value equal or larger 
than \code{omp\_get\_num\_places()}.

\crossreferences
\begin{itemize}
\item \code{omp\_get\_place\_num\_procs} routine, see 
\specref{subsec:omp_get_place_num_procs}.

\item \code{omp\_get\_num\_places} routine, see 
\specref{subsec:omp_get_num_places}.

\item \code{OMP\_PLACES} environment variable, see 
\specref{sec:OMP_PLACES}.
\end{itemize}




\subsection{\code{omp\_get\_place\_num}}
\index{omp\_get\_place\_num@{\code{omp\_get\_place\_num}}}
\label{subsec:omp_get_place_num}

\summary
The \code{omp\_get\_place\_num} routine returns the place number of the place to which the encountering thread is bound.

\format
\ccppspecificstart
\begin{boxedcode}
int omp\_get\_place\_num(void);
\end{boxedcode}
\ccppspecificend

\fortranspecificstart
\begin{boxedcode}
integer function omp\_get\_place\_num()
\end{boxedcode}
\fortranspecificend

\binding
The binding thread set for an \code{omp\_get\_place\_num} region is the encountering thread.

\effect
When the encountering thread is bound to a place, the 
\code{omp\_get\_place\_num} routine returns the place number associated 
with the thread. The returned value is between 0 and one less than the 
value returned by \code{omp\_get\_num\_places()}, inclusive. When the 
encountering thread is not bound to a place, the routine returns -1.

\crossreferences
\begin{itemize}
\item Controlling OpenMP thread affinity, see 
\specref{subsec:Controlling OpenMP Thread Affinity}. 

\item \code{omp\_get\_num\_places} routine, see 
\specref{subsec:omp_get_num_places}.

\item \code{OMP\_PLACES} environment variable, see 
\specref{sec:OMP_PLACES}.
\end{itemize}





\subsection{\code{omp\_get\_partition\_num\_places}}
\index{omp\_get\_partition\_num\_places@{\code{omp\_get\_partition\_num\_places}}}
\label{subsec:omp_get_partition_num_places}

\summary
The \code{omp\_get\_partition\_num\_places} routine returns the number of places in the place partition of the innermost implicit task.

\format
\ccppspecificstart
\begin{boxedcode}
int omp\_get\_partition\_num\_places(void);
\end{boxedcode}
\ccppspecificend

\fortranspecificstart
\begin{boxedcode}
integer function omp\_get\_partition\_num\_places()
\end{boxedcode}
\fortranspecificend

\binding
The binding task set for an  \code{omp\_get\_partition\_num\_places} region is the encountering implicit task.

\effect
The \code{omp\_get\_partition\_num\_places} routine returns the number of places in the \plc{place-partition-var} ICV.

\crossreferences
\begin{itemize}
\item \plc{place-partition-var} ICV, see 
\specref{sec:Internal Control Variables}.

\item Controlling OpenMP thread affinity, see 
\specref{subsec:Controlling OpenMP Thread Affinity}. 

\item \code{OMP\_PLACES} environment variable, see 
\specref{sec:OMP_PLACES}.
\end{itemize}





\subsection{\code{omp\_get\_partition\_place\_nums}}
\index{omp\_get\_partition\_place\_nums@{\code{omp\_get\_partition\_place\_nums}}}
\label{subsec:omp_get_partition_place_nums}

\summary
The \code{omp\_get\_partition\_place\_nums} routine returns the list of place numbers corresponding to the places in the \plc{place-partition-var} ICV of the innermost implicit task.

\format
\ccppspecificstart
\begin{boxedcode}
void omp\_get\_partition\_place\_nums(int *\plc{place\_nums});
\end{boxedcode}
\ccppspecificend

\fortranspecificstart
\begin{boxedcode}
subroutine omp\_get\_partition\_place\_nums(\plc{place\_nums})
integer \plc{place\_nums}(*)
\end{boxedcode}
\fortranspecificend

\binding
The binding task set for an \code{omp\_get\_partition\_place\_nums} region is the encountering implicit task.

\effect
The \code{omp\_get\_partition\_place\_nums} routine returns the list of 
place numbers corresponding to the places in the \plc{place-partition-var} 
ICV of the innermost implicit task. The array must be sufficiently large 
to contain \code{omp\_get\_partition\_num\_places()} integers; otherwise, 
the behavior is unspecified.

\crossreferences
\begin{itemize}
\item \plc{place-partition-var} ICV, see 
\specref{sec:Internal Control Variables}.

\item Controlling OpenMP thread affinity, see 
\specref{subsec:Controlling OpenMP Thread Affinity}. 

\item \code{omp\_get\_partition\_num\_places} routine, see 
\specref{subsec:omp_get_partition_num_places}.

\item \code{OMP\_PLACES} environment variable, see 
\specref{sec:OMP_PLACES}.
\end{itemize}



%%%%%%%%%%


\subsection{\code{omp\_set\_default\_device}}
\index{omp\_set\_default\_device@{\code{omp\_set\_default\_device}}}
\label{subsec:omp_set_default_device}

\summary

The \code{omp\_set\_default\_device} routine controls the default target device by 
assigning the value of the \plc{default-device-var} ICV.

\format
\ccppspecificstart
\begin{boxedcode}
void omp\_set\_default\_device(int \plc{device\_num});
\end{boxedcode}
\ccppspecificend

\fortranspecificstart
\begin{boxedcode}
subroutine omp\_set\_default\_device(\plc{device\_num})
integer \plc{device\_num}
\end{boxedcode}
\fortranspecificend

\binding
The binding task set for an \code{omp\_set\_default\_device} region is the generating 
task.

\effect
The effect of this routine is to set the value of the \plc{default-device-var} ICV of the current 
task to the value specified in the argument. When called from within a \code{target} region 
the effect of this routine is unspecified.

\crossreferences
\begin{itemize}
\item \plc{default-device-var}, see 
\specref{sec:Internal Control Variables}.

\item \code{omp\_get\_default\_device}, see 
\specref{subsec:omp_get_default_device}.

\item \code{OMP\_DEFAULT\_DEVICE} environment variable, see 
\specref{sec:OMP_DEFAULT_DEVICE}
\end{itemize}










\subsection{\code{omp\_get\_default\_device}}
\index{omp\_get\_default\_device@{\code{omp\_get\_default\_device}}}
\label{subsec:omp_get_default_device}
\summary
The \code{omp\_get\_default\_device} routine returns the default target device.

\format
\ccppspecificstart
\begin{boxedcode}
int omp\_get\_default\_device(void);
\end{boxedcode}
\ccppspecificend

\fortranspecificstart
\begin{boxedcode}
integer function omp\_get\_default\_device()
\end{boxedcode}
\fortranspecificend

\binding
The binding task set for an \code{omp\_get\_default\_device} region is the generating 
task. 

\effect
The \code{omp\_get\_default\_device} routine returns the value of the \plc{default-device-var} 
ICV of the current task. When called from within a \code{target} region the effect of this 
routine is unspecified.

\crossreferences
\begin{itemize}
\item \plc{default-device-var}, see 
\specref{sec:Internal Control Variables}.

\item \code{omp\_set\_default\_device}, see 
\specref{subsec:omp_set_default_device}.

\item \code{OMP\_DEFAULT\_DEVICE} environment variable, see 
\specref{sec:OMP_DEFAULT_DEVICE}. 
\end{itemize}











\subsection{\code{omp\_get\_num\_devices}}
\index{omp\_get\_num\_devices@{\code{omp\_get\_num\_devices}}}
\label{subsec:omp_get_num_devices}
\summary
The \code{omp\_get\_num\_devices} routine returns the number of target devices.

\format
\ccppspecificstart
\begin{boxedcode}
int omp\_get\_num\_devices(void);
\end{boxedcode}
\ccppspecificend

\fortranspecificstart
\begin{boxedcode}
integer function omp\_get\_num\_devices()
\end{boxedcode}
\fortranspecificend

\binding
The binding task set for an \code{omp\_get\_num\_devices} region is the generating task.

\effect
The \code{omp\_get\_num\_devices} routine returns the number of available target devices. 
When called from within a \code{target} region the effect of this routine is unspecified.

\crossreferences
None.










\subsection{\code{omp\_get\_num\_teams}}
\index{omp\@{\code{omp\_get\_num\_teams}}}
\label{subsec:omp_get_num_teams}
\summary
The \code{omp\_get\_num\_teams} routine returns the number of teams in the current \code{teams} 
region.

\pagebreak
\format
\ccppspecificstart
\begin{boxedcode}
int omp\_get\_num\_teams(void);
\end{boxedcode}
\ccppspecificend

\fortranspecificstart
\begin{boxedcode}
integer function omp\_get\_num\_teams()
\end{boxedcode}
\fortranspecificend

\binding
The binding task set for an \code{omp\_get\_num\_teams} region is the generating task

\effect
The effect of this routine is to return the number of teams in the current \code{teams} region. 
The routine returns 1 if it is called from outside of a \code{teams} region.

\crossreferences
\begin{itemize}
\item \code{teams} construct, see 
\specref{subsec:teams Construct}. 
\end{itemize}










\pagebreak
\subsection{\code{omp\_get\_team\_num}}
\index{omp\_get\_team\_num@{\code{omp\_get\_team\_num}}}
\label{subsec:omp_get_team_num}
\summary
The \code{omp\_get\_team\_num} routine returns the team number of the calling thread.

\format
\ccppspecificstart
\begin{boxedcode}
int omp\_get\_team\_num(void);
\end{boxedcode}
\ccppspecificend

\fortranspecificstart
\begin{boxedcode}
integer function omp\_get\_team\_num()
\end{boxedcode}
\fortranspecificend

\binding
The binding task set for an \code{omp\_get\_team\_num} region is the generating task.

\effect
The \code{omp\_get\_team\_num} routine returns the team number of the calling thread. The 
team number is an integer between 0 and one less than the value returned by 
\code{omp\_get\_num\_teams()}, inclusive. The routine returns 0 if it is called outside of a 
\code{teams} region.

\crossreferences
\begin{itemize}
\item \code{teams} construct, see 
\specref{subsec:teams Construct}.

\item \code{omp\_get\_num\_teams} routine, see 
\specref{subsec:omp_get_num_teams}. 
\end{itemize}









\subsection{\code{omp\_is\_initial\_device}}
\index{omp\_is\_initial\_device@{\code{omp\_is\_initial\_device}}}
\label{subsec:omp_is_initial_device}
\summary
The \code{omp\_is\_initial\_device} routine returns \plc{true} if the current task is executing 
on the host device; otherwise, it returns \plc{false}.

\begin{samepage}
\format
\ccppspecificstart
\begin{boxedcode}
int omp\_is\_initial\_device(void);
\end{boxedcode}
\ccppspecificend
\end{samepage}

\fortranspecificstart
\begin{boxedcode}
logical function omp\_is\_initial\_device()
\end{boxedcode}
\fortranspecificend

\binding
The binding task set for an \code{omp\_is\_initial\_device} region is the generating task.

\effect
The effect of this routine is to return \plc{true} if the current task is executing on the host 
device; otherwise, it returns \plc{false}.

\crossreferences
\begin{itemize}
\item \code{target} construct, see 
\specref{subsec:target Construct}
\end{itemize}





\subsection{\code{omp\_get\_initial\_device}}
\index{omp\_get\_initial\_device@{\code{omp\_get\_initial\_device}}}
\label{subsec:omp_get_initial_device}
\summary
The \code{omp\_get\_initial\_device} routine returns a device number representing
the host device.

\pagebreak

\begin{samepage}
\format
\ccppspecificstart
\begin{boxedcode}
int omp\_get\_initial\_device(void);
\end{boxedcode}
\ccppspecificend
\end{samepage}

\fortranspecificstart
\begin{boxedcode}
integer function omp\_get\_initial\_device()
\end{boxedcode}
\fortranspecificend

\binding
The binding task set for an \code{omp\_get\_initial\_device} region is the generating task.

\effect
The effect of this routine is to return the device number of the host device.
The value of the device number is implementation defined. If it is between 0 
and one less than \code{omp\_get\_num\_devices()} then it is valid for use 
with all device constructs and routines; if it is outside that range, then 
it is only valid for use with the device memory routines and not in the 
\code{device} clause. When called from within a \code{target} region 
the effect of this routine is unspecified.

\crossreferences
\begin{itemize}
\item \code{target} construct, see 
\specref{subsec:target Construct}

\item Device memory routines, see \specref{sec:Device Memory Routines}.
\end{itemize}




\subsection{\code{omp\_get\_max\_task\_priority}}
\index{omp\_get\_max\_task\_priority@{\code{omp\_get\_max\_task\_priority}}}
\label{subsec:omp_get_max_task_priority}
\summary

The \code{omp\_get\_max\_task\_priority} routine returns the maximum value that can be
specified in the \code{priority} clause.

\pagebreak

\begin{samepage}
\format
\ccppspecificstart
\begin{boxedcode}
int omp\_get\_max\_task\_priority(void);
\end{boxedcode}
\ccppspecificend
\end{samepage}

\fortranspecificstart
\begin{boxedcode}
integer function omp\_get\_max\_task\_priority()
\end{boxedcode}
\fortranspecificend

\binding

The binding thread set for an \code{omp\_get\_max\_task\_priority} region is all threads
on the device. The effect of executing this routine is not related to any specific region
corresponding to any construct or API routine.

\effect

The \code{omp\_get\_max\_task\_priority} routine returns the value of the \plc{max-task-priority-var}
ICV, which determines the maximum value that can be specified in the \code{priority} clause.

\crossreferences

\begin{itemize}
\item \plc{max-task-priority-var}, see 
\specref{sec:Internal Control Variables}.

\item \code{task} construct, see 
\specref{subsec:task Construct}. 
\end{itemize}










%% \newpage
\section{Lock Routines}
\index{lock routines}
\label{sec:Lock Routines}

\vspace{-8pt}

The OpenMP runtime library includes a set of general-purpose lock routines that can be 
used for synchronization. These general-purpose lock routines operate on OpenMP locks 
that are represented by OpenMP lock variables. OpenMP lock variables must be 
accessed only through the routines described in this section; programs that otherwise 
access OpenMP lock variables are non-conforming.

An OpenMP lock can be in one of the following states: \emph{uninitialized}, \emph{unlocked}, or 
\emph{locked}. If a lock is in the \emph{unlocked} state, a task can \emph{set} the lock, which changes its state 
to \emph{locked}. The task that sets the lock is then said to \emph{own} the lock. A task that owns a 
lock can \emph{unset} that lock, returning it to the \emph{unlocked} state. A program in which a task 
unsets a lock that is owned by another task is non-conforming.

Two types of locks are supported: \emph{simple locks} and \emph{nestable locks}. A \emph{nestable lock} can 
be set multiple times by the same task before being unset; a \emph{simple lock} cannot be set if 
it is already owned by the task trying to set it. \emph{Simple lock} variables are associated with 
\emph{simple locks} and can only be passed to \emph{simple lock} routines. \emph{Nestable lock} variables are 
associated with \emph{nestable locks} and can only be passed to \emph{nestable lock} routines.

Each type of lock can also have a \emph{lock hint} that contains information about the intended usage 
of the lock by the application code.  The effect of the lock hint is
implementation defined.  An OpenMP implementation can use this hint to select a
usage-specific lock, but lock hints do not change the mutual exclusion semantics of locks. 
A conforming implementation can safely ignore the lock hint.

Constraints on the state and ownership of the lock accessed by each of the lock routines 
are described with the routine. If these constraints are not met, the behavior of the 
routine is unspecified. 

The OpenMP lock routines access a lock variable such that they always read 
and update the most current value of the lock variable. It is not necessary for an 
OpenMP program to include explicit \code{flush} directives to ensure that the lock variable's 
value is consistent among different tasks. 

\vspace{-8pt}

\binding
The binding thread set for all lock routine regions is all threads in the contention group. 
As a consequence, for each OpenMP lock, the lock routine effects relate to all tasks that 
call the routines, without regard to which teams the threads in the contention group 
executing the tasks belong.


\vspace{-8pt}

\littleheader{Simple Lock Routines}
\index{Simple Lock Routines}
\ccppspecificstart
The type \code{omp\_lock\_t} represents a simple lock. For the 
following routines, a simple lock variable must be of \code{omp\_lock\_t} type. All simple 
lock routines require an argument that is a pointer to a variable of type \code{omp\_lock\_t}.
\ccppspecificend

\fortranspecificstart
For the following routines, a simple lock variable must be an integer variable of 
\code{kind=omp\_lock\_kind}.
\fortranspecificend

The simple lock routines are as follows:

\begin{itemize}
\item The \code{omp\_init\_lock} routine initializes a simple lock.

\item The \code{omp\_init\_lock\_with\_hint} routine initializes a simple lock and attaches a hint to it.

\item The \code{omp\_destroy\_lock} routine uninitializes a simple lock.

\item The \code{omp\_set\_lock} routine waits until a simple lock is available, and then sets it.

\item The \code{omp\_unset\_lock} routine unsets a simple lock.

\item The \code{omp\_test\_lock} routine tests a simple lock, and sets it if it is available.
\end{itemize}

\littleheader{Nestable Lock Routines}
\ccppspecificstart
The type \code{omp\_nest\_lock\_t} represents a nestable lock. 
For the following routines, a nestable lock variable must be of \code{omp\_nest\_lock\_t} type. 
All nestable lock routines require an argument that is a pointer to a variable of type 
\code{omp\_nest\_lock\_t}.
\ccppspecificend

\fortranspecificstart
For the following routines, a nestable lock variable must be an integer variable of 
\code{kind=omp\_nest\_lock\_kind}.
\fortranspecificend

The nestable lock routines are as follows:

\begin{itemize}
\item The \code{omp\_init\_nest\_lock} routine initializes a nestable lock.

\item The \code{omp\_init\_nest\_lock\_with\_hint} routine initializes a nestable lock and attaches a hint to it.

\item The \code{omp\_destroy\_nest\_lock} routine uninitializes a nestable lock.

\item The \code{omp\_set\_nest\_lock} routine waits until a nestable lock is available, and then 
sets it.

\item The \code{omp\_unset\_nest\_lock} routine unsets a nestable lock.

\item The \code{omp\_test\_nest\_lock} routine tests a nestable lock, and sets it if it is 
available
\end{itemize}

\restrictions
OpenMP lock routines have the following restrictions:

\begin{itemize}
\item The use of the same OpenMP lock in different contention groups results in 
unspecified behavior. 
\end{itemize}










\subsection{\code{omp\_init\_lock} and \code{omp\_init\_nest\_lock}}
\label{subsec:omp_init_lock and omp_init_nest_lock}
\index{omp\_init\_lock@{\code{omp\_init\_lock}}}
\index{omp\_init\_nest\_lock@{\code{omp\_init\_nest\_lock}}}
\summary
These routines initialize an OpenMP lock without a hint.

\format
\ccppspecificstart
\begin{boxedcode}
void omp\_init\_lock(omp\_lock\_t *\plc{lock});
void omp\_init\_nest\_lock(omp\_nest\_lock\_t *\plc{lock});
\end{boxedcode}
\ccppspecificend

\fortranspecificstart
\begin{boxedcode}
subroutine omp\_init\_lock(\plc{svar})
integer (kind=omp\_lock\_kind) \plc{svar}

subroutine omp\_init\_nest\_lock(\plc{nvar})
integer (kind=omp\_nest\_lock\_kind) \plc{nvar}
\end{boxedcode}
\fortranspecificend

\constraints
A program that accesses a lock that is not in the uninitialized state through either routine 
is non-conforming.

\effect
The effect of these routines is to initialize the lock to the unlocked state; that is, no task 
owns the lock. In addition, the nesting count for a nestable lock is set to zero.











\subsection[\code{omp\_init\_lock\_with\_hint} and \\\code{omp\_init\_nest\_lock\_with\_hint}]{\code{omp\_init\_lock\_with\_hint} and \\\code{omp\_init\_nest\_lock\_with\_hint}}
\label{subsec:omp_init_lock_with_hint and omp_init_nest_lock_with_hint}
\index{omp\_init\_lock@{\code{omp\_init\_lock}}}
\index{omp\_init\_nest\_lock@{\code{omp\_init\_nest\_lock}}}
\summary
These routines initialize an OpenMP lock with a hint.  
The effect of the hint is implementation-defined. The OpenMP implementation
can ignore the hint without changing program semantics.


\format
\ccppspecificstart
\begin{boxedcode}
void omp\_init\_lock\_with\_hint(omp\_lock\_t *\plc{lock}, 
                             omp\_lock\_hint\_t \plc{hint});
void omp\_init\_nest\_lock\_with\_hint(omp\_nest\_lock\_t *\plc{lock}, 
                                  omp\_lock\_hint\_t \plc{hint});
\end{boxedcode}
\ccppspecificend

\fortranspecificstart
\begin{boxedcode}
subroutine omp\_init\_lock\_with\_hint(\plc{svar}, \plc{hint})
integer (kind=omp\_lock\_kind) \plc{svar}
integer (kind=omp\_lock\_hint\_kind) \plc{hint}

subroutine omp\_init\_nest\_lock\_with\_hint(\plc{nvar}, \plc{hint})
integer (kind=omp\_nest\_lock\_kind) \plc{nvar}
integer (kind=omp\_lock\_hint\_kind) \plc{hint}
\end{boxedcode}
\fortranspecificend

\constraints
A program that accesses a lock that is not in the uninitialized state through either routine 
is non-conforming.

The second argument passed to this routine (\plc{hint}) can be one of the valid OpenMP lock hints below or any implementation-defined hint. 
The C/C++ header file (\code{omp.h}) and the Fortran include file (\code{omp\_lib.h}) and/or Fortran~90 module file (\code{omp\_lib}) define the valid lock hint constants.  
The valid constants must include the following, which can be extended with implementation-defined values:

\ccppspecificstart
\begin{codepar}
typedef enum omp\_lock\_hint\_t \{
  omp\_lock\_hint\_none = 0,
  omp\_lock\_hint\_uncontended = 1,
  omp\_lock\_hint\_contended = 2,
  omp\_lock\_hint\_nonspeculative = 4,
  omp\_lock\_hint\_speculative = 8
\} omp\_lock\_hint\_t;
\end{codepar}
\ccppspecificend

\fortranspecificstart
\begin{codepar}
integer (kind=omp\_lock\_hint\_kind), &
        parameter :: omp\_lock\_hint\_none = 0
integer (kind=omp\_lock\_hint\_kind), &
        parameter :: omp\_lock\_hint\_uncontended = 1
integer (kind=omp\_lock\_hint\_kind), &
        parameter :: omp\_lock\_hint\_contended = 2
integer (kind=omp\_lock\_hint\_kind), &
        parameter :: omp\_lock\_hint\_nonspeculative = 4 
integer (kind=omp\_lock\_hint\_kind), &
        parameter :: omp\_lock\_hint\_speculative = 8
\end{codepar}
\fortranspecificend

The hints can be combined by using the \code{+} or \code{|} operators in C/C++ or the \code{+} operator in Fortran.  
The effect of the combined hint is implementation defined and can be ignored by the implementation.
Combining \code{omp\_lock\_hint\_none} with any other hint is equivalent to specifying the other hint.
The following restrictions apply to combined hints; violating these 
restrictions results in unspecified behavior:
\begin{itemize}
%\item \code{omp\_lock\_hint\_none} cannot be combined with any other hint,
\item the hints \code{omp\_lock\_hint\_uncontended} and \code{omp\_lock\_hint\_contended} cannot be combined,
\item the hints \code{omp\_lock\_hint\_nonspeculative} and \code{omp\_lock\_hint\_speculative} cannot be combined.
\end{itemize}


\notestart
\noteheader – Future OpenMP specifications may add additional hints to the 
\code{omp\_lock\_hint\_t} type and the \code{omp\_lock\_hint\_kind} kind.  
Implementers are advised to add implementation-defined hints starting from 
the most significant bit of the \code{omp\_lock\_hint\_t} type and
\code{omp\_lock\_hint\_kind} kind and to include the name of the 
implementation in the name of the added hint to avoid name conflicts 
with other OpenMP implementations.
\noteend


\effect
The effect of these routines is to initialize the lock to the unlocked state and, optionally, to choose a specific lock implementation based on the hint. 
After initialization no task owns the lock. In addition, the nesting count for a nestable lock is set to zero.




\subsection[\code{omp\_destroy\_lock} and \code{omp\_destroy\_nest\_lock}]{\code{omp\_destroy\_lock} and\\ \code{omp\_destroy\_nest\_lock}}
\label{subsec:omp_destroy_lock and omp_destroy_nest_lock}
\index{omp\_destroy\_lock@{\code{omp\_destroy\_lock}}}
\index{omp\_destroy\_nest\_lock@{\code{omp\_destroy\_nest\_lock}}}
\summary
These routines ensure that the OpenMP lock is uninitialized.

\format
\ccppspecificstart
\begin{boxedcode}
void omp\_destroy\_lock(omp\_lock\_t *\plc{lock});
void omp\_destroy\_nest\_lock(omp\_nest\_lock\_t *\plc{lock});
\end{boxedcode}
\ccppspecificend

\fortranspecificstart
\begin{boxedcode}
subroutine omp\_destroy\_lock(\plc{svar})
integer (kind=omp\_lock\_kind) \plc{svar}

subroutine omp\_destroy\_nest\_lock(\plc{nvar})
integer (kind=omp\_nest\_lock\_kind) \plc{nvar}
\end{boxedcode}
\fortranspecificend

\constraints
A program that accesses a lock that is not in the unlocked state through either routine is 
non-conforming.

\effect
The effect of these routines is to change the state of the lock to uninitialized.










\subsection{\code{omp\_set\_lock} and \code{omp\_set\_nest\_lock}}
\label{subsec:omp_set_lock and omp_set_nest_lock}
\index{omp\_set\_lock@{\code{omp\_set\_lock}}}
\index{omp\_set\_nest\_lock@{\code{omp\_set\_nest\_lock}}}
\summary
These routines provide a means of setting an OpenMP lock. The calling
task region behaves as if it was suspended until the lock can be set
by this task.

\format
\ccppspecificstart
\begin{boxedcode}
void omp\_set\_lock(omp\_lock\_t *\plc{lock});
void omp\_set\_nest\_lock(omp\_nest\_lock\_t *\plc{lock});
\end{boxedcode}
\ccppspecificend

\fortranspecificstart
\begin{boxedcode}
subroutine omp\_set\_lock(\plc{svar})
integer (kind=omp\_lock\_kind) \plc{svar}

subroutine omp\_set\_nest\_lock(\plc{nvar})
integer (kind=omp\_nest\_lock\_kind) \plc{nvar}
\end{boxedcode}
\fortranspecificend

\constraints
A program that accesses a lock that is in the uninitialized state through either routine is 
non-conforming. A simple lock accessed by \code{omp\_set\_lock} that is in the locked state 
must not be owned by the task that contains the call or deadlock will result.

\effect
Each of these routines has an effect equivalent to suspension of the task
executing the routine until the specified lock is available. 

% However, the
% task region may continue to execute speculatively, provided that the 
% implementation ensures that execution is as if the lock were owned
% exclusively by one thread. 

% A note because it's only explanatory. The as-ifness already allows
% this via the ``equivalent to''
\notestart \noteheader – The semantics of these routines is specified
\emph{as if} they serialize execution of the region guarded by the
lock. However, implementations may implement them in other ways
provided that the isolation properties are respected so that the
actual execution delivers a result that could arise from some
serialization. 
\noteend

A simple lock is available if it is unlocked. Ownership of the lock is 
granted to the task executing the routine.

A nestable lock is available if it is unlocked or if it is already owned by 
the task executing the routine. The task executing the routine is granted, 
or retains, ownership of the lock, and the nesting count for the lock is 
incremented.

\events

The \plc{lock-acquire} or \plc{nested-lock-acquire} event occurs in the thread 
executing a \code{omp\_set\_lock} or \code{omp\_set\_nest\_lock} function
before the associated lock is requested.

The \plc{lock-acquired} or \plc{nested-lock-acquired} event occurs in the thread 
executing a \code{omp\_set\_lock} or \code{omp\_set\_nest\_lock} function
after acquiring the associated lock, but before returning from the function call.

\tools

A thread invokes a registered \code{ompt\_callback\_mutex\_acquire}
callback for each occurrence of a \plc{lock-acquire} or \plc{nested-lock-acquire} event 
in that thread.  This callback has the type signature \code{ompt\_callback\_mutex\_acquire\_t}.
A thread invokes a registered \code{ompt\_callback\_mutex\_acquired}
callback for each occurrence of a \plc{lock-acquired} or \plc{nested-lock-acquired} event 
in that thread.  This callback has the type signature \code{ompt\_callback\_mutex\_t}.
The callbacks occur in the task encountering
the lock function.  The callbacks receive \code{ompt\_mutex\_lock} or 
 \code{ompt\_mutex\_nest\_lock} as their \plc{kind} argument as appropriate.




\subsection{\code{omp\_unset\_lock} and \code{omp\_unset\_nest\_lock}}
\label{subsec:omp_unset_lock and omp_unset_nest_lock}
\index{omp\_unset\_lock@{\code{omp\_unset\_lock}}}
\index{omp\_unset\_nest\_lock@{\code{omp\_unset\_nest\_lock}}}
\summary
These routines provide the means of unsetting an OpenMP lock.

\format
\ccppspecificstart
\begin{boxedcode}
void omp\_unset\_lock(omp\_lock\_t *\plc{lock});
void omp\_unset\_nest\_lock(omp\_nest\_lock\_t *\plc{lock});
\end{boxedcode}
\ccppspecificend

\fortranspecificstart
\begin{boxedcode}
subroutine omp\_unset\_lock(\plc{svar})
integer (kind=omp\_lock\_kind) \plc{svar}

subroutine omp\_unset\_nest\_lock(\plc{nvar})
integer (kind=omp\_nest\_lock\_kind) \plc{nvar}
\end{boxedcode}
\fortranspecificend

\constraints
A program that accesses a lock that is not in the locked state or that is 
not owned by the task that contains the call through either routine is 
non-conforming.

% Jim Cownie: I don't believe we need anything more about speculation
% here. It has to behave ``as-if'' it held the lock, and released it
% here, and that's what we say happens.

\effect
For a simple lock, the \code{omp\_unset\_lock} routine causes the lock to become unlocked.

For a nestable lock, the \code{omp\_unset\_nest\_lock} routine decrements the nesting 
count, and causes the lock to become unlocked if the resulting nesting count is zero.

For either routine, if the lock becomes unlocked, and if one or more task 
regions were effectively suspended because the lock was unavailable, the 
effect is that one task is chosen and given ownership of the lock. 

\events

The \plc{lock-release} or \plc{nested-lock-release} event occurs in the thread 
executing a \code{omp\_unset\_lock} or \code{omp\_unset\_nest\_lock} function
after releasing the associated lock, but before returning from the function call.

\tools

A thread invokes a registered \code{ompt\_callback\_mutex\_release}
callback for each occurrence of a \plc{lock-release} or \plc{nested-lock-release} event 
in that thread.  This callback has the type signature \code{ompt\_callback\_mutex\_t}.
The callbacks occur in the task encountering
the lock function.  The callbacks receive \code{ompt\_mutex\_lock} or 
 \code{ompt\_mutex\_nest\_lock} as their \plc{kind} argument as appropriate.









\subsection{\code{omp\_test\_lock} and \code{omp\_test\_nest\_lock}}
\label{subsec:omp_test_lock and omp_test_nest_lock}
\index{omp\_test\_lock@{\code{omp\_test\_lock}}}
\index{omp\_test\_nest\_lock@{\code{omp\_test\_nest\_lock}}}
\summary
These routines attempt to set an OpenMP lock but do not suspend execution of the task 
executing the routine.

\format
\ccppspecificstart
\begin{boxedcode}
int omp\_test\_lock(omp\_lock\_t *\plc{lock});
int omp\_test\_nest\_lock(omp\_nest\_lock\_t *\plc{lock});
\end{boxedcode}
\ccppspecificend

\fortranspecificstart
\begin{boxedcode}
logical function omp\_test\_lock(\plc{svar})
integer (kind=omp\_lock\_kind) \plc{svar}
integer function omp\_test\_nest\_lock(\plc{nvar})
integer (kind=omp\_nest\_lock\_kind) \plc{nvar}
\end{boxedcode}
\fortranspecificend

\constraints
A program that accesses a lock that is in the uninitialized state through either routine is 
non-conforming. The behavior is unspecified if a simple lock accessed by 
\code{omp\_test\_lock} is in the locked state and is owned by the task that contains the call.

\effect 
These routines attempt to set a lock in the same manner as \code{omp\_set\_lock} and 
\code{omp\_set\_nest\_lock}, except that they do not suspend execution of the task 
executing the routine.

For a simple lock, the \code{omp\_test\_lock} routine returns \plc{true} if the lock is successfully 
set; otherwise, it returns \plc{false}.

For a nestable lock, the \code{omp\_test\_nest\_lock} routine returns the new nesting count 
if the lock is successfully set; otherwise, it returns zero.





\vspace{-24pt}


\section{Timing Routines}
\index{timing routines}
\index{timer}
\index{wall clock timer}
\label{sec:Timing Routines}
This section describes routines that support a portable wall clock timer.

\vspace{-12pt}









\subsection{\code{omp\_get\_wtime}}
\index{omp\_get\_wtime@{\code{omp\_get\_wtime}}}
\label{subsec:omp_get_wtime}
\summary
The \code{omp\_get\_wtime} routine returns elapsed wall clock time in seconds.

\format
\ccppspecificstart
\begin{boxedcode}
double omp\_get\_wtime(void);
\end{boxedcode}
\ccppspecificend

\fortranspecificstart
\begin{boxedcode}
double precision function omp\_get\_wtime()
\end{boxedcode}
\fortranspecificend

\binding
The binding thread set for an \code{omp\_get\_wtime} region is the encountering thread. The 
routine’s return value is not guaranteed to be consistent across any set of threads.

\effect
The \code{omp\_get\_wtime} routine returns a value equal to the elapsed wall clock time in 
seconds since some ``time in the past''. The actual ``time in the past'' is arbitrary, but it is 
guaranteed not to change during the execution of the application program. The time 
returned is a ``per-thread time'', so it is not required to be globally consistent across all 
threads participating in an application.

\notestart
\noteheader – It is anticipated that the routine will be used to measure elapsed times as shown 
in the following example:

\ccppspecificstart
\begin{boxedcode}
double start;
double end;
start = omp\_get\_wtime();
\plc{... work to be timed ...}
end = omp\_get\_wtime();
printf("Work took \%f seconds{\textbackslash}n", end - start);
\end{boxedcode}
\ccppspecificend

\fortranspecificstart
\begin{boxedcode}
DOUBLE PRECISION START, END
START = omp\_get\_wtime()
\plc{... work to be timed ...}
END = omp\_get\_wtime()
PRINT *, "Work took", END - START, "seconds"
\end{boxedcode}
\fortranspecificend
\noteend










\pagebreak
\subsection{\code{omp\_get\_wtick}}
\index{omp\_get\_wtick@{\code{omp\_get\_wtick}}}
\label{subsec:omp_get_wtick}
\summary
The \code{omp\_get\_wtick} routine returns the precision of the timer used by 
\code{omp\_get\_wtime}.

\format
\ccppspecificstart
\begin{boxedcode}
double omp\_get\_wtick(void);
\end{boxedcode}
\ccppspecificend

\fortranspecificstart
\begin{boxedcode}
double precision function omp\_get\_wtick()
\end{boxedcode}
\fortranspecificend

\binding
The binding thread set for an \code{omp\_get\_wtick} region is the encountering thread. The 
routine’s return value is not guaranteed to be consistent across any set of threads.

\effect
The \code{omp\_get\_wtick} routine returns a value equal to the number of seconds between 
successive clock ticks of the timer used by \code{omp\_get\_wtime}.







\pagebreak
\ccppspecificstart
\vspace{-3\baselineskip}
\section{Device Memory Routines}
\index{device memory routines}
\index{target memory routines}
\label{sec:Device Memory Routines}
This section describes routines that support allocation of memory and 
management of pointers in the data environments of target devices.


\subsection{\code{omp\_target\_alloc}}
\index{omp\_target\_alloc@{\code{omp\_target\_alloc}}}
\label{subsec:omp_target_alloc}
\summary
The \code{omp\_target\_alloc} routine allocates memory in a device data
environment.

\format
\begin{boxedcode}
void* omp\_target\_alloc(size\_t \plc{size}, int \plc{device\_num});
\end{boxedcode}

\effect

The \code{omp\_target\_alloc} routine returns the device address of a storage
location of \plc{size} bytes. The storage location is dynamically allocated in
the device data environment of the device specified by \plc{device\_num}, which
must be greater than or equal to zero and less than the result of
\code{omp\_get\_num\_devices()} or the result of a call to
\code{omp\_get\_initial\_device()}. When called from within a \code{target} region 
the effect of this routine is unspecified.

The \code{omp\_target\_alloc} routine returns \code{NULL} if it cannot dynamically
allocate the memory in the device data environment. 

The device address returned by \code{omp\_target\_alloc} can be used in an 
\code{is\_device\_ptr} clause, \specref{subsec:target Construct}.

Pointer arithmetic is not supported on the device address returned by
\code{omp\_target\_alloc}.

Freeing the storage returned by \code{omp\_target\_alloc} with any routine
other than \code{omp\_target\_free} results in unspecified behavior.

\events
The \plc{target-data-allocation} event occurs when a thread allocates data on a target device.

\tools

A thread invokes a registered \code{ompt\_callback\_target\_data\_op}
callback for each occurrence of a \plc{target-data-allocation} event in that thread. 
The callback occurs in the context of the target task.  The callback has type signature
\code{ompt\_callback\_target\_data\_op\_t}. 


\crossreferences
\begin{itemize}
\item \code{target} construct, see 
\specref{subsec:target Construct}

\item \code{omp\_get\_num\_devices} routine, see 
\specref{subsec:omp_get_num_devices}

\item \code{omp\_get\_initial\_device} routine, see 
\specref{subsec:omp_get_initial_device}

\item \code{omp\_target\_free} routine, see 
\specref{subsec:omp_target_free}

\item \code{ompt\_callback\_target\_data\_op\_t}, see 
\specref{sec:ompt_callback_target_data_op_t}.

\end{itemize}



\pagebreak

\linewitharrows{-1}{dashed}{C/C++ (cont.)}{8em}
\subsection{\code{omp\_target\_free}}
\index{omp\_target\_free@{\code{omp\_target\_free}}}
\label{subsec:omp_target_free}
\summary
The \code{omp\_target\_free} routine frees the device memory allocated by the \code{omp\_target\_alloc} routine.

\format
\begin{boxedcode}
void omp\_target\_free(void * \plc{device\_ptr}, int \plc{device\_num});
\end{boxedcode}

\constraints

A program that calls \code{omp\_target\_free} with a non-\code{NULL} pointer 
that does not have a value returned from \code{omp\_target\_alloc} is
non-conforming.  The \plc{device\_num} must be greater than or equal to 
zero and less than the result of \code{omp\_get\_num\_devices()} or the
 result of a call to \code{omp\_get\_initial\_device()}.

\effect

The \code{omp\_target\_free} routine frees the memory in the device data
environment associated with \plc{device\_ptr}.  If \plc{device\_ptr} is
\code{NULL}, the operation is ignored.

Synchronization must be inserted to ensure that all accesses to 
\plc{device\_ptr} are completed before the call to \code{omp\_target\_free}.

When called from within a \code{target} region the effect of this routine
is unspecified.

\events
The \plc{target-data-free} event occurs when a thread frees data on a target device.

\tools

A thread invokes a registered \code{ompt\_callback\_target\_data\_op}
callback for each occurrence of a \plc{target-data-free} event in that thread. 
The callback occurs in the context of the target task.  The callback has type signature
\code{ompt\_callback\_target\_data\_op\_t}. 


\crossreferences
\begin{itemize}
\item \code{target} construct, see 
\specref{subsec:target Construct}

\item \code{omp\_get\_num\_devices} routine, see 
\specref{subsec:omp_get_num_devices}

\item \code{omp\_get\_initial\_device} routine, see 
\specref{subsec:omp_get_initial_device}

\item \code{omp\_target\_alloc} routine, see 
\specref{subsec:omp_target_alloc}

\item \code{ompt\_callback\_target\_data\_op\_t}, see 
\specref{sec:ompt_callback_target_data_op_t}.

\end{itemize}



\pagebreak
\linewitharrows{-1}{dashed}{C/C++ (cont.)}{8em}
\subsection{\code{omp\_target\_is\_present}}
\index{omp\_target\_is\_present@{\code{omp\_target\_is\_present}}}
\label{subsec:omp_target_is_present}
\summary

The \code{omp\_target\_is\_present} routine tests whether a host pointer
has corresponding storage on a given device.

\format
\begin{boxedcode}
int omp\_target\_is\_present(void * \plc{ptr}, int \plc{device\_num});
\end{boxedcode}

\constraints

The value of \plc{ptr} must be a valid host pointer or \code{NULL}.
The \plc{device\_num}
must be greater than or equal to zero and less than the result of
\code{omp\_get\_num\_devices()} or the result of a call to
\code{omp\_get\_initial\_device()}.

\effect

This routine returns \plc{true} if the specified pointer
would be found present on device \plc{device\_num} by a \code{map}
clause; otherwise, it returns \plc{false}.

When called from within a \code{target} region 
the effect of this routine is unspecified.

\crossreferences
\begin{itemize}
\item \code{target} construct, see \specref{subsec:target Construct}

\item \code{map} clause, see \specref{subsec:map Clause}.

\item \code{omp\_get\_num\_devices} routine, see 
\specref{subsec:omp_get_num_devices}

\item \code{omp\_get\_initial\_device} routine, see 
\specref{subsec:omp_get_initial_device}
\end{itemize}


\pagebreak
\linewitharrows{-1}{dashed}{C/C++ (cont.)}{8em}
\subsection{\code{omp\_target\_memcpy}}
\index{omp\_target\_memcpy@{\code{omp\_target\_memcpy}}}
\label{subsec:omp_target_memcpy}
\summary

The \code{omp\_target\_memcpy} routine copies memory between any combination
of host and device pointers.

\format
\begin{boxedcode}
int omp\_target\_memcpy(void * \plc{dst}, void * \plc{src}, size\_t \plc{length},
                        size\_t \plc{dst\_offset}, size\_t \plc{src\_offset},
                        int \plc{dst\_device\_num}, int \plc{src\_device\_num});
\end{boxedcode}

\constraints
Each device must
be compatible with the device pointer specified on the same side of the copy.
The \plc{dst\_device\_num} and \plc{src\_device\_num}
must be greater than or equal to zero and less than the result of
\code{omp\_get\_num\_devices()} or equal to the result of a call to
\code{omp\_get\_initial\_device()}.

\effect


\plc{length} bytes of memory at offset \plc{src\_offset} from  \plc{src}
in the device data environment of device \plc{src\_device\_num} are
copied to \plc{dst} starting at offset \plc{dst\_offset} in the device data
environment of device \plc{dst\_device\_num}.  
The return value is zero on success and non-zero on failure.  The host device
and host device data environment can be referenced with the device number
returned by \code{omp\_get\_initial\_device}. This routine contains a task
scheduling point.

When called from within a \code{target} region 
the effect of this routine is unspecified.

\events
The \plc{target-data-transfer} event occurs when a thread transfer data on a target device.

\tools

A thread invokes a registered \code{ompt\_callback\_target\_data\_op}
callback for each occurrence of a \plc{target-data-transfer} event in that thread. 
The callback occurs in the context of the target task.  The callback has type signature
\code{ompt\_callback\_target\_data\_op\_t}. 


\crossreferences
\begin{itemize}
\item \code{target} construct, see \specref{subsec:target Construct}

\item \code{omp\_get\_initial\_device} routine, see 
\specref{subsec:omp_get_initial_device}

\item \code{omp\_target\_alloc} routine, see \specref{subsec:omp_target_alloc}


\item \code{ompt\_callback\_target\_data\_op\_t}, see 
\specref{sec:ompt_callback_target_data_op_t}.
\end{itemize}


\pagebreak
\linewitharrows{-1}{dashed}{C/C++ (cont.)}{8em}
\vspace{-1\baselineskip}
\subsection{\code{omp\_target\_memcpy\_rect}}
\index{omp\_target\_memcpy\_rect@{\code{omp\_target\_memcpy\_rect}}}
\label{subsec:omp_target_memcpy_rect}
\summary

The \code{omp\_target\_memcpy\_rect} routine copies a rectangular subvolume from
a multi-dimensional array to another multi-dimensional array. The copies can
use any combination of host and device pointers.

\format
\begin{samepage}
\begin{boxedcode}
int omp\_target\_memcpy\_rect(
               void * \plc{dst}, void * \plc{src},
               size\_t \plc{element\_size},
               int \plc{num\_dims},
               const size\_t* \plc{volume},
               const size\_t* \plc{dst\_offsets},
               const size\_t* \plc{src\_offsets},
               const size\_t* \plc{dst\_dimensions},
               const size\_t* \plc{src\_dimensions},
               int \plc{dst\_device\_num}, int \plc{src\_device\_num});
\end{boxedcode}
\end{samepage}

\constraints
The length of the offset and dimension arrays must be at least the
value of \plc{num\_dims}.
The \code{dst\_device\_num} and \code{src\_device\_num}
must be greater than or equal to zero and less than the result of
\code{omp\_get\_num\_devices()} or equal to the result of a call to
\code{omp\_get\_initial\_device()}.

The value of \plc{num\_dims} must be between 1 and the implementation-defined
limit, which must be at least three.


\pagebreak
\linewitharrows{-1}{dashed}{C/C++ (cont.)}{8em}
\effect

This routine copies a rectangular subvolume of \plc{src},
in the device data environment of device \plc{src\_device\_num},
to \plc{dst}, in the device data environment of device \plc{dst\_device\_num}.
The volume is specified in terms of the size of an element, 
number of dimensions, and constant arrays of length \plc{num\_dims}.  The 
maximum number of dimensions supported is at least three, support for higher 
dimensionality is implementation defined. The volume array specifies the 
length, in number of elements, to copy in each dimension from \plc{src} 
to \plc{dst}. The \plc{dst\_offsets} (\plc{src\_offsets}) parameter specifies
number of elements from the origin of \plc{dst} (\plc{src}) in elements.  
The \plc{dst\_dimensions} (\plc{src\_dimensions}) parameter specifies the 
length of each dimension of \plc{dst} (\plc{src})

The routine returns zero if successful. If both \plc{dst} and \plc{src} are
\code{NULL} pointers, the routine returns the number of dimensions supported
by the implementation for the specified device numbers. The host device and
host device data environment can be referenced with the device number returned
by \code{omp\_get\_initial\_device}.  Otherwise, it returns a non-zero value. The
routine contains a task scheduling point.

When called from within a \code{target} region 
the effect of this routine is unspecified.

%% As an example, this routine could be used to copy a $3\times3\times3$ cube 
%% from the center of a $9\times9\times9$ cube in \code{src} to a \code{dst} 
%% matrix of size $3\times3\times3$ on the host as follows:
%%\begin{samepage}
%%\begin{boxedcode}
%%int src[9][9][9], dst[3][3][3];
%%size_t volume[]    = \{3, 3, 3\},
%%       dst_offset[]= \{0, 0, 0\},
%%       src_offset[]= \{3, 3, 3\},
%%       dst_dims[]  = \{3, 3, 3\},
%%       src_dims[]  = \{9, 9, 9\};
%%omp_target_memcpy_rect(
%%          dst, src,
%%          sizeof(int), //elem_size
%%          3, //num_dims
%%          volume,
%%          dst_offset,
%%          src_offset,
%%          dst_dims,
%%          src_dims,
%%          0, 0);
%%\end{boxedcode}
%%\end{samepage}

\events
The \plc{target-data-transfer} event occurs when a thread transfer data on a target device.

\tools

A thread invokes a registered \code{ompt\_callback\_target\_data\_op}
callback for each occurrence of a \plc{target-data-transfer} event in that thread. 
The callback occurs in the context of the target task.  The callback has type signature
\code{ompt\_callback\_target\_data\_op\_t}. 


\crossreferences
\begin{itemize}
\item \code{target} construct, see \specref{subsec:target Construct}

\item \code{omp\_get\_initial\_device} routine, see 
\specref{subsec:omp_get_initial_device}

\item \code{omp\_target\_alloc} routine, see \specref{subsec:omp_target_alloc}


\item \code{ompt\_callback\_target\_data\_op\_t}, see 
\specref{sec:ompt_callback_target_data_op_t}.

\end{itemize}

\subsection{\code{omp\_target\_associate\_ptr}}
\index{omp\_target\_associate\_ptr@{\code{omp\_target\_associate\_ptr}}}
\label{subsec:omp_target_associate_ptr}
\summary

The \code{omp\_target\_associate\_ptr} routine maps a device pointer, which may
be returned from \code{omp\_target\_alloc} or implementation-defined runtime 
routines, to a host pointer.

\format
\begin{boxedcode}
int omp\_target\_associate\_ptr(void * \plc{host\_ptr}, void * \plc{device\_ptr},
                             size\_t \plc{size}, size\_t \plc{device\_offset},
                             int \plc{device\_num});
\end{boxedcode}

\pagebreak
\linewitharrows{-1}{dashed}{C/C++ (cont.)}{8em}
\constraints

The value of \plc{device\_ptr} value must be a valid pointer to device 
memory for the device denoted by the value of \plc{device\_num}.
The \plc{device\_num} argument
must be greater than or equal to zero and less than the result of
\code{omp\_get\_num\_devices()} or equal to the result of a call to
\code{omp\_get\_initial\_device()}.

\effect

The \code{omp\_target\_associate\_ptr} routine associates a device pointer
in the device data environment of device \plc{device\_num}
with a host pointer such that when the host pointer appears in a subsequent
\code{map} clause, the associated device pointer is used as the target for
data motion associated with that host pointer.  The \plc{device\_offset}
parameter specifies what offset into \plc{device\_ptr} will be used as the
base address for the device side of the mapping.  The reference count of the
resulting mapping will be infinite.  After being successfully associated, the
buffer pointed to by the device pointer is invalidated and accessing data
directly through the device pointer results in unspecified behavior.  The
pointer can be retrieved for other uses by disassociating it.
When called from within a \code{target} region 
the effect of this routine is unspecified.

The routine returns zero if successful. Otherwise it returns a non-zero value.

Only one device buffer can be associated with a given host pointer value and
device number pair. Attempting to associate a second buffer will return
non-zero. Associating the same pair of pointers on the same device with the
same offset has no effect and returns zero.  Associating pointers that share
underlying storage will result in unspecified behavior. The
\code{omp\_target\_is\_present} function can be used to test whether a given
host pointer has a corresponding variable in the device data environment.

\events
The \plc{target-data-associate} event occurs when a thread associate data on a target device.

\tools

A thread invokes a registered \code{ompt\_callback\_target\_data\_op}
callback for each occurrence of a \plc{target-data-associate} event in that thread. 
The callback occurs in the context of the target task.  The callback has type signature
\code{ompt\_callback\_target\_data\_op\_t}. 

\crossreferences
\begin{itemize}
\item \code{target} construct, see \specref{subsec:target Construct}

\item \code{map} clause, see \specref{subsec:map Clause}.

\item \code{omp\_target\_alloc} routine, see \specref{subsec:omp_target_alloc}

\item \code{omp\_target\_disassociate\_ptr} routine, see 
\specref{subsec:omp_target_associate_ptr}

\item \code{ompt\_callback\_target\_data\_op\_t}, see 
\specref{sec:ompt_callback_target_data_op_t}.

\end{itemize}

\pagebreak
\linewitharrows{-1}{dashed}{C/C++ (cont.)}{8em}
\subsection{\code{omp\_target\_disassociate\_ptr}}
\index{omp\_target\_disassociate\_ptr@{\code{omp\_target\_disassociate\_ptr}}}
\label{subsec:omp_target_disassociate_ptr}
\summary

The \code{omp\_target\_disassociate\_ptr} removes the associated pointer for a
given device from a host pointer.

\format
\begin{boxedcode}
int omp\_target\_disassociate\_ptr(void * \plc{ptr}, int \plc{device\_num});
\end{boxedcode}

\constraints


The \plc{device\_num}
must be greater than or equal to zero and less than the result of
\code{omp\_get\_num\_devices()} or equal to the result of a call to
\code{omp\_get\_initial\_device()}.

\effect

The \code{omp\_target\_disassociate\_ptr} removes the associated device data
on device \plc{device\_num} from the presence table for host pointer
\plc{ptr}. A call to this routine on a pointer that is not 
\code{NULL} and does not have associated data on the given device results
in unspecified behavior.  The reference count of the mapping is reduced to 
zero, regardless of its current value.

When called from within a \code{target} region 
the effect of this routine is unspecified.

After a call to \code{omp\_target\_disassociate\_ptr}, the contents of the device
buffer are invalidated.

\events
The \plc{target-data-diassociate} event occurs when a thread diassociate data on a target device.

\tools

A thread invokes a registered \code{ompt\_callback\_target\_data\_op}
callback for each occurrence of a \plc{target-data-diassociate} event in that thread. 
The callback occurs in the context of the target task.  The callback has type signature
\code{ompt\_callback\_target\_data\_op\_t}. 

\crossreferences
\begin{itemize}
\item \code{target} construct, see 
\specref{subsec:target Construct}
\item \code{omp\_target\_associate\_ptr} routine, see 
\specref{subsec:omp_target_associate_ptr}
\item \code{ompt\_callback\_target\_data\_op\_t}, see 
\specref{sec:ompt_callback_target_data_op_t}.

\end{itemize}

\ccppspecificend

\section{Tool Control Routines}
\index{tool control}
\label{sec:tool_tcontrol}

\summary
The \code{omp\_tcontrol} routine enables a program to 
pass commands to an active tool.

\format

\vbox{
\ccppspecificstart
\begin{boxedcode}
int omp\_tcontrol(
  long long int \plc{command}, 
  long long int \plc{modifier},
  void *arg);
\end{boxedcode}
\ccppspecificend
}

\fortranspecificstart
\begin{boxedcode}
subroutine omp\_tcontrol(\plc{command}, \plc{modifier}, \plc{error\_code})
integer (kind=omp\_tcontrol\_kind) \plc{command}
integer (kind=omp\_tcontrol\_kind) \plc{modifier}
integer (kind=omp\_tcontrol\_kind) \plc{error\_code}
\end{boxedcode}
\fortranspecificend

\descr

An OpenMP program may use \code{omp\_tcontrol} to pass commands to 
a tool. 
Using \code{omp\_tcontrol}, an application can request that
a tool start or restart data collection when a code
region of interest is encountered, pause data collection when 
leaving the region of interest, 
flush any data that it has collected so far, and
end data collection. 
Additionally, \code{omp\_tcontrol} can be used to pass 
tool-specific commands to a particular tool.


\vbox{
\ccppspecificstart
\begin{boxedcode}
  #define OMP_TRESULT_NOTOOL     (-2)
  #define OMP_TRESULT_NOCALLBACK (-1)
  #define OMP_TRESULT_SUCCESS      0
  #define OMP_TRESULT_IGNORED      1
\end{boxedcode}
\ccppspecificend
}

\vbox{
\fortranspecificstart
\begin{boxedcode}
integer(kind=omp\_tcontrol\_kind), &
        parameter :: OMP\_TRESULT\_NOTOOL =     -2 
integer(kind=omp\_tcontrol\_kind), &
        parameter :: OMP\_TRESULT\_NOCALLBACK = -1
integer(kind=omp\_tcontrol\_kind), &
        parameter :: OMP\_TRESULT\_SUCCESS =     0
integer(kind=omp\_tcontrol\_kind), &
        parameter :: OMP\_TRESULT\_IGNORED =     1
\end{boxedcode}
\fortranspecificend
}

If no tool is active, the OpenMP implementation will return
\code{OMP\_TRESULT\_NOTOOL}. If a tool is active, but it has not
registered a callback for the \plc{tool-control} event, the OpenMP
implementation will return \code{OMP\_TRESULT\_NOCALLBACK}.  
An OpenMP implementation may return other implementation-defined negative
values; an application may assume that any negative return value
indicates that a tool has not received the command.  A return value of
\code{OMP\_TRESULT\_SUCCESS} indicates that the tool has performed the
specified command.  A return value of \code{OMP\_TRESULT\_IGNORED}
indicates that the tool has ignored the specified command.
A tool may return other positive values that are tool-defined. 
When \code{omp\_tcontrol} is called from C/C++, the tool return value will be
returned from \code{omp\_tcontrol}. 
When \code{omp\_tcontrol} is called from Fortran, the tool return value will be
returned in \plc{error\_code}. 

\constraints

The following enumeration type defines four standard commands.
Table~\ref{table:std-tool-cmds} describes the 
actions that these commands request from a tool.

\vbox{
\ccppspecificstart
\begin{boxedcode}
typedef enum omp_tcontrol_t \{
  omp_tcontrol_start = 1,
  omp_tcontrol_pause = 2,
  omp_tcontrol_flush = 3,
  omp_tcontrol_end = 4
\} omp_tcontrol_t;
\end{boxedcode}
\ccppspecificend
}

\vbox{
\fortranspecificstart
\begin{boxedcode}
integer(kind=omp\_tcontrol\_kind), &
        parameter :: omp\_tcontrol\_start = 1
integer(kind=omp\_tcontrol\_kind), & 
        parameter :: omp\_tcontrol\_pause = 2
integer(kind=omp\_tcontrol\_kind), & 
        parameter :: omp\_tcontrol\_flush = 3
integer(kind=omp\_tcontrol\_kind), &
        parameter :: omp\_tcontrol\_end = 4
\end{boxedcode}
\fortranspecificend
}

Tool-specific values for \plc{command} must be $\geq$ 64.
Tools must ignore \plc{command} values that they are not
explicitly designed to handle.
Other values accepted by a tool for \plc{command}, 
and any values for \plc{modifier} and \plc{arg} are tool-defined.

\begin{table}
\begin{center}
\begin{tabular}{|l|p{3.5in}|}\hline
Command & Action\\\hline\hline
\code{omp\_tcontrol\_start} & Start or 
restart monitoring if it is off. If monitoring is already on, this 
command is idempotent. If monitoring has already been turned off 
permanently, this command will have no effect.\\\hline
\code{omp\_tcontrol\_pause} & Temporarily
turn monitoring off. If monitoring is already off, it is idempotent.\\\hline
\code{omp\_tcontrol\_flush} & Flush any data buffered by a tool.
This command may be applied whether monitoring is on or off.\\\hline
\code{omp\_tcontrol\_end} & Turn monitoring off permanently; 
the tool finalizes itself and flushes all output.\\\hline
\end{tabular}
\vskip 1ex
\caption{Standard tool control commands.}
\label{table:std-tool-cmds}
\end{center}
\end{table}

\events

The \plc{tool-control} event occurs in the thread encountering a call
to \code{omp\_tcontrol} at a point inside its associated OpenMP region.

\tools

An OpenMP implementation invokes a registered
\code{ompt\_callback\_tcontrol} callback for each occurrence of a
\plc{tool-control} event.  The callback executes in the context of the
call that occurs in the user program.  This callback has type
signature \code{ompt\_callback\_tcontrol\_t}.The callback may return
any non-negative value, which will be returned to the application by
the OpenMP implementation as the return value of the
\code{omp\_tcontrol} call that triggered the callback.


Arguments passed to the callback are those passed by the user to
\code{omp\_tcontrol}. If the call is made in Fortran, the tool will be
passed a \code{NULL} as the third argument to the callback. If any of the four standard commands is
presented to a tool, the tool will ignore the \plc{modifier} and \plc{arg} values.



\crossreferences
\begin{itemize}
\item Tool Interface, see
\specchapterref{chap:ToolsSupport}
\item \code{ompt\_callback\_tcontrol\_t}, see \specref{sec:ompt_callback_tcontrol_t}
\end{itemize}


% This is the end of chapter 3 - Runtime Library

