% This is ch4-toolsSupport.tex of the OpenMP specification.
% This is an included file. See the master file for more information.
%
% When editing this file:
%
%    1. To change formatting, appearance, or style, please edit openmp.sty.
%
%    2. Custom commands and macros are defined in openmp.sty.
%
%    3. Be kind to other editors -- keep a consistent style by copying-and-pasting to
%       create new content.
%
%    4. We use semantic markup, e.g. (see openmp.sty for a full list):
%         \code{}     % for bold monospace keywords, code, operators, etc.
%         \plc{}      % for italic placeholder names, grammar, etc.
%
%    5. Other recommendations:
%         Use the convenience macros defined in openmp.sty for the minor headers
%         such as Comments, Syntax, etc.
%
%         To keep items together on the same page, prefer the use of 
%         \begin{samepage}.... Avoid \parbox for text blocks as it interrupts line numbering.
%         When possible, avoid \filbreak, \pagebreak, \newpage, \clearpage unless that's
%         what you mean. Use \needspace{} cautiously for troublesome paragraphs.
%
%         Avoid absolute lengths and measures in this file; use relative units when possible.
%         Vertical space can be relative to \baselineskip or ex units. Horizontal space
%         can be relative to \linewidth or em units.
%
%         Prefer \emph{} to italicize terminology, e.g.:
%             This is a \emph{definition}, not a placeholder.
%             This is a \plc{var-name}.
%


\chapter{Tools Support}
\index{Tools Support}
\label{chap:ToolsSupport}

This chapter describes the OpenMP API specific to interact with third-party tools.
The chapter covers how an OpenMP runtime uses the value of the \plc{omp-tool-var} ICV to decide whether or not an OpenMP runtime will try to register a tool prior to runtime initialization.
Follows an explanation on how a registered tool initializes itself.

\section{Tool Registration}
\index{tool registration}
\label{sec:ToolsSupport_Registration}

The \code{OMP\_TOOL} environment variable sets the \plc{omp-tool-var} ICV, which controls whether or not an OpenMP runtime will try to register a tool.
The value assigned to \code{OMP\_TOOL} is case insensitive and may have leading and trailing white space.
The value of this environment variable must be \code{enabled} or \code{disabled}.
If \code{OMP\_TOOL} is set to any value other than \code{enabled} or \code{disabled}, an OpenMP runtime will print a fatal error to the standard error file descriptor indicating that an illegal value had been supplied for \code{OMP\_TOOL} and the program's execution will terminate.
If \code{OMP\_TOOL} is not defined, the default value for \plc{omp-tool-var} is \code{enabled}.

\begin{table}
\begin{center}
\begin{tabular}{|c|p{4.5in}|}
\hline
 {\em omp-tool-var} value & action \\\hline
enabled & the OpenMP runtime will call \code{ompt\_tool} before initializing itself.\\\hline
disabled & the OpenMP runtime will not call \code{ompt\_tool}, regardless of whether a tool is present or not.\\\hline
\end{tabular}
\end{center}
\caption{OpenMP runtime responses to settings of the {\em omp-tool-var} ICV.}
\label{table:ToolsSupport_env-var}
\end{table}

\sloppy
Table~\ref{table:ToolsSupport_env-var} describes the action that an OpenMP runtime will take in response to possible values of \plc{omp-tool-var}.
If the value of \plc{omp-tool-var} is \code{enabled}, the runtime will attempt to register a tool by calling the function \code{ompt\_tool} before performing runtime initialization.
The signature for \code{ompt\_tool} is shown below:
\begin{quote}
\begin{verbatim}
extern "C" {
  ompt_initialize_fn_t ompt_tool(void);
};
\end{verbatim}
\end{quote}
If a tool provides an implementation of \code{ompt\_tool} in the application's address space, it may return \code{NULL} indicating that the tool declines to register itself with the runtime; otherwise, the tool may register itself with the runtime by returning a non-\code{NULL} pointer to a function with type signature \code{ompt\_initialize\_fn\_t}.
The type signature for \code{ompt\_initialize\_fn\_t} is described in Section~\ref{sec:ToolsSupport_init}. 
Since only one tool-provided definition of \code{ompt\_tool} will be seen by an OpenMP runtime, only one tool may register itself. 
If a tool-supplied implementation of \code{ompt\_tool} returns a non-\code{NULL} initializer, the OpenMP runtime will maintain state information for each OpenMP thread and will perform OMPT event callbacks registered during tool initialization. 

After a process fork, if OpenMP is re-initialized in the child process,
the OpenMP runtime in the child process will call \code{ompt\_tool} under the same conditions as it would for any process. 

\section{Tool Initialization}
\label{sec:ToolsSupport_init}

When an OpenMP runtime receives a non-\code{NULL} pointer to a tool initializer function with signature \code{ompt\_initialize\_fn\_t} as the return value from a call to a tool-provided implementation of \code{ompt\_tool}, the runtime will call the tool initializer immediately after the runtime fully initializes itself.
The initializer must be called before beginning execution of any OpenMP construct or completing any execution environment routine invocation.
The signature for the tool initializer callback is shown below:

\begin{quote}
\begin{verbatim}
typedef void (*ompt_initialize_fn_t) (
  ompt_function_lookup_t lookup,
  const char *runtime_version, 
  unsigned int ompt_version
);
\end{verbatim}
\end{quote}

The second argument to \code{ompt\_initialize} is a version string that unambiguously identifies an OpenMP runtime implementation.
This argument is useful to tool developers trying to debug a statically-linked executable that contains both a tool implementation and an OpenMP runtime implementation.
Knowing exactly what version of an OpenMP runtime is present in a binary may be helpful when diagnosing a problem, e.g., identifying an old runtime system that may be incompatible with a newer tool.

The third argument \code{ompt\_version} indicates the version of the OMPT interface supported by a runtime system.
The version of OMPT described by this document is 2.

The two principal duties of a tool initializer are looking up pointers to all OMPT API functions that the tool uses and registering tool callbacks.
These two operations are described below.

\subsection{Looking up functions in the OMPT API.}
The first argument to \code{ompt\_initialize} is \code{lookup}---a callback that a tool must use to interrogate the runtime system to obtain pointers to all OMPT interface functions.
The type signature for \code{lookup} is:

\begin{quote}
\begin{verbatim}
typedef ompt_interface_fn_t (*ompt_function_lookup_t) (
  const char *interface_function_name
);
\end{verbatim}
\end{quote}

\noindent
The \code{lookup} callback is necessary because, when the OpenMP runtime is dynamically loaded by a shared library, the OMPT interface functions provided by the library may not be visible to a preloaded tool.
Within a tool, one uses \code{lookup} to obtain function pointers to each function in the OMPT API.
All functions in the OMPT API are marked with \code{OMPT\_API}.
These functions should not be global symbols in an OpenMP runtime implementation to avoid tempting tool developers to call them directly. 

Below, we show how to use the \code{lookup} function to obtain a pointer to the OMPT API function \code{ompt\_get\_thread\_id}:

\begin{quote}
\begin{verbatim}
ompt_interface_fn_t ompt_get_thread_id_fn =
                        lookup("ompt_get_thread_id");
\end{verbatim}
\end{quote}
Other functions in the OMPT API may be looked up analogously.
If a named function is not available in an OpenMP runtime's implementation of OMPT, \code{lookup} will return \code{NULL}.

\subsection{Registering Callbacks.} 
Tools register callbacks to receive notification of various events that occur as an OpenMP program executes by using the OMPT API function \code{ompt\_set\_callback}.
The signature for this function is shown below
{
\begin{quote}
\begin{verbatim}
OMPT_API int ompt_set_callback(
  ompt_event_t event, 
  ompt_callback_t callback
);
\end{verbatim}
\end{quote}
}

\begin{table}
\centering
\begin{tabular}{|l|l|}
\hline
return code & meaning \\\hline
0 & callback registration error (e.g., callbacks cannot be registered at this time).\\\hline
1 & event may occur; no callback is possible\\\hline
2 & event will never occur in runtime\\\hline
3 & event may occur; callback invoked when convenient\\\hline
4 & event may occur; callback always invoked when event occurs\\\hline
\end{tabular}
\caption{Meaning of return codes for {\tt ompt\_set\_callback}.}
\label{table:ToolsSupport_set_rc}
\end{table}

\noindent
The  function \code{ompt\_set\_callback} may only be called within the implementation of \code{ompt\_initialize}.
The possible return codes for \code{ompt\_set\_callback} and their meaning is shown in Table~\ref{table:ToolsSupport_set_rc}.
Registration of supported callbacks may fail if this function is called outside  \code{ompt\_initialize}.
The \code{ompt\_callback\_t} type for a callback does not reflect the actual signature of the callback; OMPT uses this generic type to avoid the need to declare a separate registration function for each actual callback type.

The OMPT API function \code{ompt\_get\_callback} may be called at any time to determine whether a callback has been registered or not. 

\begin{quote}
\begin{verbatim}
OMPT_API int ompt_get_callback(
  ompt_event_t event, 
  ompt_callback_t *callback
);
\end{verbatim}
\end{quote}

\noindent
If a callback has been registered, \code{ompt\_get\_callback} will return 1 and set \code{callback} to the address of the callback function; otherwise \code{ompt\_get\_callback} will return 0.

% This is the end of ch5-toolsSupport.tex
