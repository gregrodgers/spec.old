% This is ch4-toolsSupport.tex of the OpenMP specification.
% This is an included file. See the master file for more information.
%
% When editing this file:
%
%    1. To change formatting, appearance, or style, please edit openmp.sty.
%
%    2. Custom commands and macros are defined in openmp.sty.
%
%    3. Be kind to other editors -- keep a consistent style by copying-and-pasting to
%       create new content.
%
%    4. We use semantic markup, e.g. (see openmp.sty for a full list):
%         \code{}     % for bold monospace keywords, code, operators, etc.
%         \plc{}      % for italic placeholder names, grammar, etc.
%
%    5. Other recommendations:
%         Use the convenience macros defined in openmp.sty for the minor headers
%         such as Comments, Syntax, etc.
%
%         To keep items together on the same page, prefer the use of 
%         \begin{samepage}.... Avoid \parbox for text blocks as it interrupts line numbering.
%         When possible, avoid \filbreak, \pagebreak, \newpage, \clearpage unless that's
%         what you mean. Use \needspace{} cautiously for troublesome paragraphs.
%
%         Avoid absolute lengths and measures in this file; use relative units when possible.
%         Vertical space can be relative to \baselineskip or ex units. Horizontal space
%         can be relative to \linewidth or em units.
%
%         Prefer \emph{} to italicize terminology, e.g.:
%             This is a \emph{definition}, not a placeholder.
%             This is a \plc{var-name}.
%


\chapter{Tools Support}
\index{Tools Support}
\label{chap:ToolsSupport}

This chapter describes the OpenMP API specific to interact with third-party tools.
The chapter covers how an OpenMP runtime uses the value of the \plc{omp-tool-var} ICV to decide whether or not an OpenMP runtime will try to register a tool prior to runtime initialization.
Follows an explanation on how a registered tool initializes itself.

\section{Event Callback Signatures}
\index{event callback signatures}
\label{sec:ToolsSupport_callback_signatures}

This section describes the signatures of tool callback functions that an OMPT 
tool might register and that are called during runtime of an OpenMP program.

\subsection{\code{ompt\_parallel\_begin\_callback\_t}}
\index{ompt\_parallel\_begin\_callback\_t@{\code{ompt\_parallel\_begin\_callback\_t}}}
\label{subsec:ompt_parallel_begin_callback_t}
\format
\begin{boxedcode}
typedef void (*ompt_parallel_begin_callback_t) (
               ompt_data_t * \plc{parent_task_data},
               const ompt_frame_t * \plc{parent_frame},  
               ompt_data_t * \plc{parallel_data},
               uint32_t \plc{requested_team_size}, 
               uint32_t \plc{actual_team_size}, 
               ompt_invoker_t \plc{invoker}, 
               const void * \plc{codeptr_ra} 
);
\end{boxedcode}

\descr
The callbacks with type signature \code{ompt\_parallel\_begin\_callback\_t}, 
include a parameter \code{requested\_team\_size} 
that indicates the number of threads requested by the user, and a parameter
\code{actual\_team\_size} that indicates the number of threads in the team.
The \code{invoker} argument explains whether the execution of the parallel
region code is inlined into the application code or started by the runtime.

\crossreferences
\begin{itemize}
\item \code{ompt\_data\_t} type signature, see 
\specref{subsec:ompt_data_t}.
\item \code{ompt\_frame\_t} type signature, see 
\specref{subsec:ompt_frame_t}.
\item \code{ompt\_invoker\_t} type signature, see 
\specref{subsec:ompt_invoker_t}.
\end{itemize}



\subsection{\code{ompt\_parallel\_end\_callback\_t}}
\index{ompt\_parallel\_end\_callback\_t@{\code{ompt\_parallel\_end\_callback\_t}}}
\label{subsec:ompt_parallel_end_callback_t}
\format
\begin{boxedcode}
typedef void (*ompt_parallel_end_callback_t) (
               ompt_data_t * \plc{parallel_data},
               ompt_data_t * \plc{task_data},
               ompt_invoker_t \plc{invoker},
               const void * \plc{codeptr_ra}
);
\end{boxedcode}

\descr
The callbacks with type signature \code{ompt\_parallel\_begin\_callback\_t}, 
include a parameter \code{invoker} which explains whether the execution of the parallel
region code is inlined into the application code or started by the runtime.

%\effect
% ompt events have no effect

\crossreferences
\begin{itemize}
\item \code{ompt\_data\_t} type signature, see 
\specref{subsec:ompt_data_t}.
\item \code{ompt\_invoker\_t} type signature, see 
\specref{subsec:ompt_invoker_t}.
\end{itemize}





% This is the end of ch5-toolsSupport.tex
