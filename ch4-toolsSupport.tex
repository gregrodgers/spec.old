% This is ch4-toolsSupport.tex of the OpenMP specification.
% This is an included file. See the master file for more information.
%
% When editing this file:
%
%    1. To change formatting, appearance, or style, please edit openmp.sty.
%
%    2. Custom commands and macros are defined in openmp.sty.
%
%    3. Be kind to other editors -- keep a consistent style by copying-and-pasting to
%       create new content.
%
%    4. We use semantic markup, e.g. (see openmp.sty for a full list):
%         \code{}     % for bold monospace keywords, code, operators, etc.
%         \plc{}      % for italic placeholder names, grammar, etc.
%
%    5. Other recommendations:
%         Use the convenience macros defined in openmp.sty for the minor headers
%         such as Comments, Syntax, etc.
%
%         To keep items together on the same page, prefer the use of
%         \begin{samepage}.... Avoid \parbox for text blocks as it interrupts line numbering.
%         When possible, avoid \filbreak, \pagebreak, \newpage, \clearpage unless that's
%         what you mean. Use \needspace{} cautiously for troublesome paragraphs.
%
%         Avoid absolute lengths and measures in this file; use relative units when possible.
%         Vertical space can be relative to \baselineskip or ex units. Horizontal space
%         can be relative to \linewidth or em units.
%
%         Prefer \emph{} to italicize terminology, e.g.:
%             This is a \emph{definition}, not a placeholder.
%             This is a \plc{var-name}.
%


\chapter{Tool Interface}
\index{Tools Support}
\label{chap:ToolsSupport}

This chapter describes the OMPT interface provided by the OpenMP API
to support third-party monitoring and performance analysis tools.
The chapter first explains if and how an implementation of
the OpenMP API will attempt to register a tool prior to initializing the OpenMP runtime and
how a registered tool will use the OMPT interface to initialize tool state maintained by
an implementation of the OpenMP API.
It concludes with a description of data types
used to identify entities managed by an OpenMP runtime and to interpret the
call stack of an OpenMP thread.

\section{Tool Registration}
\index{tool registration}
\label{sec:ToolsSupport_Registration}

The \code{OMP\_TOOL} environment variable sets the \plc{omp-tool-var} ICV, which controls whether or not an
implementation of the OpenMP API will attempt to register a tool.
The value of \code{OMP\_TOOL} is case insensitive and may have leading and trailing white space.
The only legal values of \code{OMP\_TOOL} are \code{enabled} or \code{disabled}.
If the value of \plc{omp-tool-var} is \code{enabled},
the OpenMP API implementation will attempt to register a tool by determining if
an implementation of the function \code{ompt\_tool} exists in a program's address space and, if so,
calling \code{ompt\_tool} before initializing the OpenMP runtime.
If the value of \plc{omp-tool-var} is \code{disabled}, the OpenMP API implementation will not
call \code{ompt\_tool}, regardless of whether or not an implementation of \code{ompt\_tool} exists in the
address space.
If \code{OMP\_TOOL} is set to any other value, an implementation of the OpenMP API
will print a fatal error to the standard error file descriptor indicating that an illegal value had been supplied for \code{OMP\_TOOL} and the program's execution will terminate.
If \code{OMP\_TOOL} is not defined, the default value for \plc{omp-tool-var} is \code{enabled}.

\begin{comment}
\begin{table}
\begin{center}
\begin{tabular}{|c|p{4in}|}
\hline
 {\em omp-tool-var} value & action \\\hline
enabled & the OpenMP runtime will call \code{ompt\_tool} before initializing itself.\\\hline
disabled & the OpenMP runtime will not call \code{ompt\_tool}, regardless of whether a tool is present or not.\\\hline
\end{tabular}
\end{center}
\caption{OpenMP runtime responses to settings of the {\em omp-tool-var} ICV.}
\label{table:ToolsSupport_env-var}
\end{table}

\sloppy
Table~\ref{table:ToolsSupport_env-var} describes the action that an OpenMP runtime will take in response to possible values of \plc{omp-tool-var}.
If the value of \plc{omp-tool-var} is \code{enabled}, the runtime will attempt to register a tool by calling the function \code{ompt\_tool} before performing runtime initialization.
\end{comment}

The signature for \code{ompt\_tool} is shown below:
\begin{boxedcode}
extern "C" {
  ompt\_initialize\_fn\_t \plc{ompt\_tool}(void);
};
\end{boxedcode}
If a tool provides an implementation of \code{ompt\_tool} in the application's address space, it may return \code{NULL} indicating that the tool declines to register itself with the runtime; otherwise, the tool may register itself with the runtime by returning a non-\code{NULL} pointer to a function with type signature \code{ompt\_initialize\_fn\_t}.

The type signature for \code{ompt\_initialize\_fn\_t} is described in Section~\ref{sec:ToolsSupport_init}.
Since only one tool-provided definition of \code{ompt\_tool} will be seen by an OpenMP runtime, only one tool may register itself.
If a tool-supplied implementation of \code{ompt\_tool} returns a non-\code{NULL} initializer,
an implementation of the OpenMP API will maintain state information for each OpenMP thread and will perform
any OMPT event callbacks registered during tool initialization.

After a process fork, if OpenMP is re-initialized in the child process,
an implementation of the OpenMP API in the child process will call \code{ompt\_tool}
under the same conditions as it would for any process.

\section{Tool Initialization}
\label{sec:ToolsSupport_init}

When an OpenMP runtime receives a non-\code{NULL} pointer to a tool initializer function with signature \code{ompt\_initialize\_fn\_t} as the return value from a call to a tool-provided implementation of \code{ompt\_tool}, the runtime will call the tool initializer immediately after the runtime fully initializes itself.
The initializer must be called before beginning execution of any OpenMP construct or completing any execution environment routine invocation.
The signature for the tool initializer callback is shown below:

\begin{boxedcode}
typedef void (*\plc{ompt\_initialize\_fn\_t}) (
  ompt\_function\_lookup\_t \plc{lookup},
  const char *\plc{runtime\_version},
  unsigned int \plc{ompt\_version}
);
\end{boxedcode}

The second argument to a tool initializer is a version string that unambiguously identifies an OpenMP runtime implementation.
This argument is useful to tool developers trying to debug a statically-linked executable that contains both a tool implementation and an OpenMP runtime implementation.
Knowing exactly what version of an OpenMP runtime is present in a binary may be helpful when diagnosing a problem, e.g., identifying an old runtime system that may be incompatible with a newer tool.

The third argument \code{ompt\_version} indicates the version of the OMPT interface supported by a runtime system.
The version of OMPT described by this document is 2.

The two principal duties of a tool initializer are looking up pointers to all OMPT interface functions that the tool uses and registering tool callbacks.
These two operations are described below.

\subsection{Looking up OMPT Interface Functions}
The first argument to a tool initializer is \code{lookup}---a callback that a tool must use to interrogate the runtime system to obtain pointers to all OMPT interface functions.
The type signature for \code{lookup} is:

\begin{boxedcode}
typedef ompt\_interface\_fn\_t (*\plc{ompt\_function\_lookup\_t}) (
  const char *\plc{interface\_function\_name}
);
\end{boxedcode}

The \code{lookup} callback is necessary because, when the OpenMP runtime is dynamically loaded by a shared library, the OMPT interface functions provided by the library may not be visible to a preloaded tool.
Within a tool, one uses \code{lookup} to obtain function pointers to each function in the OMPT interface.
All functions in the OMPT interface are marked with \code{OMPT\_API}.
These functions should not be global symbols in an OpenMP runtime implementation to avoid tempting tool developers to call them directly.

The following example shows how to use the \code{lookup} function to obtain a pointer to the OMPT interface function \code{ompt\_get\_thread\_id}:

\begin{boxedcode}
ompt\_interface\_fn\_t \plc{ompt\_get\_thread\_id\_fn} =
                        \plc{lookup}("ompt\_get\_thread\_id");
\end{boxedcode}
Other functions in the OMPT API may be looked up analogously.
If a named function is not available in an OpenMP runtime's implementation of OMPT, \code{lookup} will return \code{NULL}.

\subsection{Registering Callbacks}
\index{event callback registration}
\label{sec:ToolsSupport_callback_registration}

Tools register callbacks to receive notification of various events that occur as
an OpenMP program executes by using the OMPT API function \code{ompt\_set\_callback}.
The signature for this function is
\begin{boxedcode}
OMPT\_API int ompt\_set\_callback(
  ompt\_event\_t \plc{event},
  ompt\_callback\_t \plc{callback}
);
\end{boxedcode}


\begin{table}
\centering
\begin{tabular}{ll}
\hline
return code & meaning \\
\hline
0 & callback registration error (e.g., callbacks cannot be registered at this time).\\
1 & event may occur; no callback is possible\\
2 & event will never occur in runtime\\
3 & event may occur; callback invoked when convenient\\
4 & event may occur; callback always invoked when event occurs\\
\hline
\end{tabular}
\caption{Meaning of return codes for \code{ompt\_set\_callback}.}
\label{table:ToolsSupport_set_rc}
\end{table}

The  function \code{ompt\_set\_callback}  may only be called within the implementation of a tool's initializer.
The possible return codes for \code{ompt\_set\_callback} and their meaning is shown in Table~\ref{table:ToolsSupport_set_rc}.
Registration of supported callbacks may fail if this function is
called outside  \code{ompt\_initialize\_fn\_t}.
The \code{ompt\_callback\_t} type for a callback does not reflect the actual signature of the callback; OMPT uses  this generic type
 to avoid the need to declare a separate registration function for each actual callback type.
The valid return codes for each event are listed in Table~\ref{table:valid_rc}.
Some events always need to invoke the registered callback.
For other events, its implementation defined whether they invoke the registered
callback never, sometimes, or always.
The return code makes this implementation defined behaviour machine readable to
the connected tool.

The  OMPT API function \code{ompt\_get\_callback} may be called at any time to determine whether a callback has been registered or not.

\begin{boxedcode}
OMPT\_API int ompt\_get\_callback(
  ompt\_event\_t \plc{event},
  ompt\_callback\_t *\plc{callback}
);
\end{boxedcode}

If a callback has been registered,  \code{ompt\_get\_callback}  will return 1
and set \code{callback} to the address of the callback function;
otherwise \code{ompt\_get\_callback} will return 0.

\begin{table}
\renewcommand{\arraystretch}{1.2}
\begin{tabular}{lp{3em}p{3em}p{3em}p{3em}p{3em}}
                                & \rot{callback registration error}
                                & \rot{event may occur; no callback is possible}
                                & \rot{event will never occur in runtime}
                                & \rot{event may occur; callback invoked when convenient}
                                & \rot{event may occur; callback always invoked when event occurs}\\
                                \midrule
ompt\_event\_thread\_begin          & * &   &   &   & * \\
ompt\_event\_thread\_end            & * &   &   &   & * \\
ompt\_event\_parallel\_begin        & * &   &   &   & * \\
ompt\_event\_parallel\_end          & * &   &   &   & * \\
ompt\_event\_task\_create           & * &   &   &   & * \\
ompt\_event\_task\_schedule         & * &   &   &   & * \\
ompt\_event\_implicit\_task         & * &   &   &   & * \\
ompt\_event\_target                 & * &   &   &   & * \\
ompt\_event\_target\_data           & * &   &   &   & * \\
ompt\_event\_target\_submit         & * &   &   &   & * \\
ompt\_event\_control                & * &   &   &   & * \\
ompt\_event\_runtime\_shutdown      & * &   &   &   & * \\
ompt\_event\_idle                   & * & * & * & * & * \\
ompt\_event\_sync\_region\_wait     & * & * & * & * & * \\
ompt\_event\_mutex\_release         & * & * & * & * & * \\
ompt\_event\_task\_dependences      & * & * & * & * & * \\
ompt\_event\_task\_dependence\_pair & * & * & * & * & * \\
ompt\_event\_worksharing            & * & * & * & * & * \\
ompt\_event\_master                 & * & * & * & * & * \\
ompt\_event\_target\_data\_map      & * & * & * & * & * \\
ompt\_event\_sync\_region           & * & * & * & * & * \\
ompt\_event\_init\_lock             & * & * & * & * & * \\
ompt\_event\_destroy\_lock          & * & * & * & * & * \\
ompt\_event\_mutex\_acquire         & * & * & * & * & * \\
ompt\_event\_mutex\_acquired        & * & * & * & * & * \\
ompt\_event\_nested\_lock           & * & * & * & * & * \\
ompt\_event\_flush                  & * & * & * & * & * \\
\bottomrule
\end{tabular}
\caption{Valid return codes of \code{ompt\_set\_callback} for each input event argument.}
\label{table:valid_rc}
\end{table}

\section{OMPT Data Types}
\label{sec:data}

\littleheader{Thread Identifier}
\index{Thread Identifier}

Each OpenMP thread  has an associated identifier of type \code{ompt\_thread\_id\_t}.
\begin{comment}
\begin{boxedcode}
typedef uint64\_t ompt\_thread\_id\_t;
\end{boxedcode}
\end{comment}
  A thread's identifier is assigned when the thread is created.
  Identifiers assigned to threads on each device are unique from the time an OpenMP runtime is initialized until it is shut down.
  A thread identifier can be retrieved
  on demand by invoking the  \code{ompt\_get\_thread\_id}
  function (described in Section~\ref{sec:thread-inquiry}).
  Tools should not assume that \code{ompt\_thread\_id\_t} values are small or densely allocated.
  The value \code{ompt\_thread\_id\_none} is reserved to indicate an invalid thread id.

\littleheader{Parallel Region Identifiers}
\index{Parallel Region Identifiers}
Each OpenMP parallel region has an associated identifier of type
\code{ompt\_parallel\_id\_t}.
\begin{comment}
\begin{boxedcode}
typedef uint64\_t ompt\_parallel\_id\_t;
\end{boxedcode}
\end{comment}
  A parallel region's identifier is assigned
  when the region is created.  Identifiers assigned to parallel regions on each device are unique from the time an
  OpenMP runtime is initialized until it is shut down.
  A parallel region identifer can be retrieved for an enclosing parallel region
  on demand using the function \code{ompt\_get\_parallel\_info}  (described in Section~\ref{sec:parallel-inquiry}).
  Tools should not assume that \code{ompt\_parallel\_id\_t} values are small or densely allocated.
  The value \code{ompt\_parallel\_id\_none} is reserved to indicate an invalid parallel id.


\littleheader{Task Region Identifiers}
\index{Task Region Identifiers}
Each OpenMP task has an associated identifier of type
\code{ompt\_task\_id\_t}. Task identifiers are assigned to
initial, implicit, explicit, and target tasks.
\begin{comment}
\begin{boxedcode}
typedef uint64\_t ompt\_task\_id\_t;
\end{boxedcode}
\end{comment}
  A task's identifier is assigned
  when the task is created.
  Identifiers assigned to tasks on each device are unique from the time an
  OpenMP runtime is initialized until it is shut down.
  A task's identifier can be retrieved
  on demand by invoking the \code{ompt\_get\_task\_info}  function (described in Section~\ref{sec:task-region}).
  Tools should not assume that \code{ompt\_task\_id\_t} values are small or densely allocated.
  The value \code{ompt\_task\_id\_none} is reserved to indicate an invalid task identifier.

\littleheader{Target Region and Operation Identifiers}
\index{Target Region and Operation Identifiers}
Each OpenMP target region and target operation has an associated identifier of type \code{ompt\_target\_id\_t}.
A unique target identifier is assigned on the host each time an instance of a target construct is encountered.
Each operation within a target region, e.g., transferring data to/from a device or launching a kernel launch
on a device, is also assigned a unique target identifier.
Identifiers assigned to target regions or operations
are unique from the time an OpenMP runtime is initialized until it is shut down.
The current target region and operation identifiers can be retrieved by invoking the \code{ompt\_get\_target\_info} function (described in Section~\ref{sec:target-region}).
Tools should not assume that \code{ompt\_target\_id\_t} values are small or densely allocated.
The value \code{ompt\_target\_id\_none} is reserved to indicate an invalid target identifier.
The value \code{ompt\_target\_id\_none} will be returned for (a) the target region identifier if \code{ompt\_get\_target\_info} is invoked outside a target region and (b) the target operation identifier if \code{ompt\_get\_target\_info} is invoked while no target operation is in progress.

\littleheader{Wait Identifiers}
\index{Wait Identifiers}

Each thread instance maintains a {\em wait identifier} of type \code{ompt\_wait\_id\_t}.
When a task executing on a thread is waiting for something, the thread's wait identifer indicates what the thread is awaiting.
A wait identifier may represent a critical section {\em name}, a lock,  a program variable accessed in an atomic region, or a synchronization object internal to an OpenMP runtime implementation.
\begin{comment}
\begin{boxedcode}
typedef uint64\_t ompt\_wait\_id\_t;
\end{boxedcode}
\end{comment}
A thread's wait identifier can be retrieved on demand by invoking the \code{ompt\_get\_state} function (described in Section~\ref{sec:thread-inquiry}).
Tools should not assume that \code{ompt\_wait\_id\_t} values are small or densely allocated.
When a thread is not in a wait state, a thread's wait identifier has an undefined value.
%%? Does that mean that the value is undefined and cannot sensibley be read, or that it has a specific value which we have
%%? defined somewhere, whose name is (something like) ompt_wait_id_undefined  ?
%%johnmc says: a wait_id typically is set to the address of a lock on which you are spinning. If you aren't spinning on a lock, this value is undefined.
%%             we could zero it out, but that would cost more.

\littleheader{Frames}
\index{Frames}

When executing an OpenMP program, at times procedure frames from the OpenMP runtime appear on the call stack between procedure frames for user code.
To enable a tool to determine whether each procedure frame on the call stack represents
user code or an OpenMP runtime routine,
an implementation of the OpenMP API must maintains an instance of an \code{ompt\_frame\_t}
for each (possibly degenerate\footnote{
A task is considered degenerate if a call to the OpenMP runtime to create a parallel
region or task does not create a new task.
A degenerate task may arise when a parallel construct is encountered
in a parallel region and nested parallelism is not enabled or when an orphaned directive that would create a task is encountered outside a parallel region.
A degenerate task region may add runtime frames to the call stack before
invoking an outlined function for the degenerate task and thus require an \code{ompt\_frame\_t}
data structure.}) task that is active on a thread's stack.
To simplify discussion that follows,
we omit the qualifier ``possibly degenerate'' each time we use the term {\em task}.

  Each initial, implicit, explicit, or target task maintains an \code{ompt\_frame\_t} data structure
  that contains a pair of pointers.

\vbox{
\ccppspecificstart
\begin{boxedcode}
typedef struct ompt_frame_s \{\\
  void * \plc{exit_frame};  /* runtime frame that calls user code */
  void * \plc{enter_frame}; /* user frame that calls the runtime  */
\} ompt_frame_t;
\end{boxedcode}
\ccppspecificend
}

An \code{ompt\_frame\_t}'s  lifetime begins when a task  is
created and ends when the task is destroyed.  Tools should not assume that a frame structure remains at a constant location in memory
throughout a task's lifetime.
A frame object is passed to some callbacks; it can also be retrieved
asynchronously
by invoking the \code{ompt\_get\_task\_info}  function (described in Section~\ref{sec:task-region}) in a signal handler.
A task's frame object contains two fields: \code{exit\_frame} and \code{enter\_frame}.

The \code{exit\_frame} field of a task's frame object
is set just before the task starts executing the structred block of the task region.
The field contains the canonical frame address of the procedure that invoked the outlined function.
In cases where the an outlined procedure is invoked directly from a user code frame,
\code{exit\_frame} will contain the canonical frame address of a procedure containing  user code
that belongs to the enclosing task.
The value of \code{exit\_frame} is \code{NULL} in a task's frame object until just
before the task calls an outlined function to begin execution. The value of
\code{exit\_frame} is set to \code{NULL} when the task returns to the runtime,
whether finished or deferred.

The \code{enter\_frame} field of a task's frame object is set each time the task re-enters the
runtime to create a new implicit, explicit, or target task. When a task invokes an entry point in the
OpenMP runtime to create a task,
the \code{enter\_frame} field of the task's frame object will be set to
the canonical frame address of the user function that invoked the runtime.
The value of \code{enter\_frame} is set when a task enters the OpenMP runtime
and cleared before the OpenMP runtime returns control to the task.

\begin{table}
\begin{center}
\begin{tabular}{p{1in}p{2in}p{2in}}
\hline
\code{exit\_frame} / \code{enter\_frame} 	& \code{enter\_frame} is \code{NULL}										& \code{enter\_frame} is defined \\
\hline
\code{exit\_frame} is \code{NULL} & case 1)  initial task in user code\newline case 2) task that is created but not yet scheduled or already finished & initial task entered the runtime to schedule an implicit, explicit, or target task \\
\code{exit\_frame} is defined 	& non-initial task in (or soon to be in) user code							& non-initial task entered the runtime and scheduled an implicit, explicit, or target task\\
\hline
\end{tabular}
\vspace{1ex}
\end{center}
\caption{Meaning of various values for \code{exit\_frame} and \code{enter\_frame}.}
\label{tab:frame}
\end{table}

Table~\ref{tab:frame} describes the meaning of this structure with various values.
In the presence of nested parallelism, a tool may observe a sequence of \code{ompt\_frame\_t} records for a thread. Appendix~\ref{app:frame} discusses  an example that illustrates the use of \code{ompt\_frame\_t} records with nested parallelism.

\littleheader{Advice to tool implementers:} A monitoring tool using
      asynchronous sampling can observe values of
      \code{exit\_frame} and \code{enter\_frame} at inconvenient times.
      Tools must be prepared to observe and handle frame exit and reenter values that have not yet been set or reset as the program enters or leaves the runtime.


\section{Events without Directives}
\subsection{\code{ompt\_event\_thread\_begin}}
The OpenMP runtime invokes this callback in the context of an initial thread just after it initializes the runtime, or in the context of a new thread created by the runtime just after the thread initializes itself. In either case, this callback must be the first callback for a thread
and must occur before the thread executes any OpenMP tasks. This callback has type signature \code{ompt\_thread\_begin\_callback\_t}.
The callback argument \code{thread\_type} indicates the type of the thread: initial, worker, or other.

\subsection{\code{ompt\_event\_thread\_end}}
The OpenMP runtime invokes this callback
after an OpenMP thread completes all of
its tasks but before the thread is destroyed. The callback
executes in the context of the OpenMP thread. This callback must be the last callback event for any worker thread; it is optional for other types of threads.
This callback has type signature \code{ompt\_thread\_end\_callback\_t}.

\subsection{\code{ompt\_event\_target\_submit}}

The OpenMP runtime invokes this callback prior to submitting a kernel for execution on a target device.
This callback has type signature \code{ompt\_target\_submit\_callback\_t}.
The callback argument \code{target\_id} indicates the instance of the target construct associated with this operation.
The callback argument \code{host\_op\_id} provides a unique host-side identifier that represents the activity on the device.
The callback arguments \code{requested\_num\_teams}  \code{granted\_num\_teams} indicate, respectively, the number of teams requested by the user and granted by the runtime.

\subsection{\code{ompt\_event\_control}}
If the user program calls \code{ompt\_control}, the
OpenMP runtime invokes this callback.
The callback executes in the context that the call occurs in the user program.
This callback has type signature \code{ompt\_control\_callback\_t}.
Arguments passed to the callback are those passed by the user to \code{ompt\_control}.

\subsection{\code{ompt\_event\_runtime\_shutdown}}
The OpenMP runtime invokes this callback before it shuts down the
 runtime system.  This callback enables a tool to clean up its
 state and record or report information gathered. A runtime may later restart and reinitialize the tool by
calling the tool initializer
function (described in Section~\ref{sec:init}) again.
 This callback has type signature \code{ompt\_callback\_t}.

\subsection{\code{ompt\_event\_idle}}
  The OpenMP runtime invokes this callback with \code{endpoint=}\code{ompt\_scope\_begin} when a thread waits for work outside a parallel region.
  The OpenMP runtime invokes this callback with \code{endpoint=}\code{ompt\_scope\_end} before the thread  begins to execute an implicit task for
   a parallel region or terminates. The callback executes in the environment of the waiting thread.
  This callback has type signature \code{ompt\_idle\_callback\_t}.

\subsection{\code{ompt\_event\_sync\_region\_wait}}

  The OpenMP runtime invokes this callback with \code{endpoint=}\code{ompt\_scope\_begin} when a task starts waiting in a barrier region, taskwait region, or taskgroup region.
   The OpenMP runtime invokes this callback with the \code{endpoint=}\code{ompt\_scope\_end} when the task stops waiting in the region.
  This callback has type signature \code{ompt\_scoped\_sync\_region\_callback\_t}.
   The argument \code{kind} indicates the kind of region causing the wait.
   One region may generate multiple pairs of begin/end callbacks if another task is scheduled on the thread while the task awaiting completion of the region is stalled.
   The callback argument \code{codeptr\_ra} may be NULL.
   This callback executes in the context of the task that encountered the barrier, taskwait, or taskgroup construct.


\section{Event Callback Signatures}
\index{event callback signatures}
\label{sec:ToolsSupport_callback_signatures}

This section describes the signatures of tool callback functions that an OMPT
tool might register and that are called during runtime of an OpenMP program.

\subsection{\code{ompt\_thread\_begin\_callback\_t}}
\index{ompt\_thread\_begin\_callback\_t@{\code{ompt\_thread\_begin\_callback\_t}}}
\label{subsec:ompt_thread_begin_callback_t}
\format
\begin{boxedcode}
typedef void (*ompt\_thread\_begin\_callback\_t) (
  ompt\_thread\_type\_t \plc{thread\_type},
  ompt\_data\_t *\plc{thread\_data}
);
\end{boxedcode}
\descr
The callback with type signature \code{ompt\_thread\_begin\_callback\_t},
includes a parameter \plc{thread\_type}
that indicates the type of the new thread: initial, worker, or other.
The binding of \plc{thread\_data} is the new thread.

\crossreferences
\begin{itemize}
\item \code{ompt\_data\_t} type signature, see
\specref{subsec:ompt_data_t}.
\item \code{ompt\_thread\_type\_t} type signature, see
\specref{subsec:ompt_thread_type_t}.
\end{itemize}



\subsection{\code{ompt\_thread\_end\_callback\_t}}
\index{ompt\_thread\_end\_callback\_t@{\code{ompt\_thread\_end\_callback\_t}}}
\label{subsec:ompt_thread_end_callback_t}
\format
\begin{boxedcode}
typedef void (*ompt\_thread\_end\_callback\_t) (
  ompt\_data\_t *\plc{thread\_data}
);
\end{boxedcode}
\descr
The binding of \plc{thread\_data} is the finished thread.

\crossreferences
\begin{itemize}
\item \code{ompt\_data\_t} type signature, see
\specref{subsec:ompt_data_t}.
\end{itemize}



\subsection{\code{ompt\_idle\_callback\_t}}
\index{ompt\_idle\_callback\_t@{\code{ompt\_idle\_callback\_t}}}
\label{subsec:ompt_idle_callback_t}
\format
\begin{boxedcode}
typedef void (*ompt\_idle\_callback\_t) (
  ompt\_scope\_endpoint\_t \plc{endpoint}
);
\end{boxedcode}
\descr
The callback with type signature \code{ompt\_idle\_callback\_t},
includes a parameter \plc{endpoint}
that indicates whether the callback signals the begin or end of the scope.

\crossreferences
\begin{itemize}
\item \code{ompt\_scope\_endpoint\_t} type signature, see
\specref{subsec:ompt_scope_endpoint_t}.
\end{itemize}



\subsection{\code{ompt\_parallel\_begin\_callback\_t}}
\index{ompt\_parallel\_begin\_callback\_t@{\code{ompt\_parallel\_begin\_callback\_t}}}
\label{subsec:ompt_parallel_begin_callback_t}
\format
\begin{boxedcode}
typedef void (*ompt\_parallel\_begin\_callback\_t) (
               ompt\_data\_t * \plc{parent\_task\_data},
               const ompt\_frame\_t * \plc{parent\_frame},
               ompt\_data\_t * \plc{parallel\_data},
               uint32\_t \plc{requested\_team\_size},
               uint32\_t \plc{actual\_team\_size},
               ompt\_invoker\_t \plc{invoker},
               const void * \plc{codeptr\_ra}
);
\end{boxedcode}

\descr
The callbacks with type signature \code{ompt\_parallel\_begin\_callback\_t},
include a parameter \plc{requested\_team\_size}
that indicates the number of threads requested by the user, and a parameter
\plc{actual\_team\_size} that indicates the number of threads in the team.
The \plc{invoker} argument explains whether the execution of the parallel
region code is inlined into the application code or started by the runtime.
The binding of \plc{parent\_task\_data} is the enclosing task, the binding of
\plc{parallel\_data} is the new team.


\crossreferences
\begin{itemize}
\item \code{ompt\_data\_t} type signature, see
\specref{subsec:ompt_data_t}.
\item \code{ompt\_frame\_t} type signature, see
\specref{subsec:ompt_frame_t}.
\item \code{ompt\_invoker\_t} type signature, see
\specref{subsec:ompt_invoker_t}.
\end{itemize}



\subsection{\code{ompt\_parallel\_end\_callback\_t}}
\index{ompt\_parallel\_end\_callback\_t@{\code{ompt\_parallel\_end\_callback\_t}}}
\label{subsec:ompt_parallel_end_callback_t}
\format
\begin{boxedcode}
typedef void (*ompt\_parallel\_end\_callback\_t) (
               ompt\_data\_t * \plc{parallel\_data},
               ompt\_data\_t * \plc{task\_data},
               ompt\_invoker\_t \plc{invoker},
               const void * \plc{codeptr\_ra}
);
\end{boxedcode}

\descr
The callbacks with type signature \code{ompt\_parallel\_begin\_callback\_t}
include a parameter \plc{invoker} which explains whether the execution of the parallel
region code is inlined into the application code or started by the runtime.
The binding of \plc{parallel\_data} is the finishing team.

%\effect
% ompt events have no effect

\crossreferences
\begin{itemize}
\item \code{ompt\_data\_t} type signature, see
\specref{subsec:ompt_data_t}.
\item \code{ompt\_invoker\_t} type signature, see
\specref{subsec:ompt_invoker_t}.
\end{itemize}

\subsection{\code{ompt\_scoped\_master\_callback\_t}}
\index{ompt\_scoped\_master\_callback\_t@{\code{ompt\_scoped\_master\_callback\_t}}}
\label{subsec:ompt_scoped_master_callback_t}
\format
\begin{boxedcode}
typedef void (*\plc{ompt\_scoped\_master\_callback\_t}) (
               ompt\_scope\_endpoint\_t \plc{endpoint},
               ompt\_data\_t *\plc{parallel\_data},
               ompt\_data\_t *\plc{task\_data},
               const void *\plc{codeptr\_ra}
);
\end{boxedcode}

\descr
The callbacks with type signature \code{ompt\_scoped\_master\_callback\_t}
include a parameter \plc{codeptr\_ra} which contains the return address of the
call to the OpenMP runtime routine and a parameter \plc{endpoint}
that indicates whether the callback signals the begin or end of the scope.
The binding of \plc{task\_data} is the innermost enclosing implicit task,
the binding of \plc{parallel\_data} is the innermost enclosing parallel region.

%\effect
% ompt events have no effect

\crossreferences
\begin{itemize}
\item \code{ompt\_scope\_endpoint\_t} type signature, see
\specref{subsec:ompt_scope_endpoint_t}.
\item \code{ompt\_data\_t} type signature, see
\specref{subsec:ompt_data_t}.
\end{itemize}





\subsection{\code{ompt\_task\_create\_callback\_t}}
\index{ompt\_task\_create\_callback\_t@{\code{ompt\_task\_create\_callback\_t}}}
\label{subsec:ompt_task_create_callback_t}
\format
\begin{boxedcode}
typedef void (*ompt\_task\_create\_callback\_t) (
  ompt\_data\_t *\plc{parent\_task\_data},
  const ompt\_frame\_t *\plc{parent\_frame},
  ompt\_data\_t *\plc{new\_task\_data},
  ompt\_task\_type\_t \plc{type},
  \_Bool \plc{has\_dependences},
  const void *\plc{codeptr\_ra}
);
\end{boxedcode}
\descr
The callbacks with type signature \code{ompt\_task\_create\_callback\_t},
include a parameter \plc{parent\_frame}
that indicates the frame information of the encountering task, and a parameter
\plc{type} that indicates the kind of the task: initial, explicit or target.
The binding of \plc{parent\_task\_data} is the encountering task,
the binding of \plc{new\_task\_data} is the created task.

\crossreferences
\begin{itemize}
\item \code{ompt\_data\_t} type signature, see
\specref{subsec:ompt_data_t}.
\item \code{ompt\_frame\_t} type signature, see
\specref{subsec:ompt_frame_t}.
\item \code{ompt\_task\_type\_t} type signature, see
\specref{subsec:ompt_task_type_t}.
\end{itemize}



\subsection{\code{ompt\_task\_dependences\_callback\_t}}
\index{ompt\_task\_dependences\_callback\_t@{\code{ompt\_task\_dependences\_callback\_t}}}
\label{subsec:ompt_task_dependences_callback_t}
\format
\begin{boxedcode}
typedef void (*ompt\_task\_dependences\_callback\_t) (
  ompt\_data\_t *\plc{task\_data},
  const ompt\_task\_dependence\_t *\plc{deps},
  int \plc{ndeps}
);
\end{boxedcode}
\descr
The callback with type signature \code{ompt\_task\_dependences\_callback\_t},
includes a parameter \plc{deps} that lists all dependences of a new task and
a parameter \plc{ndeps} that specifies the length of the list.
The memory ownage of \plc{deps} remains at the caller, the tool cannot rely on
the data after the callback returns.
The binding of \plc{task\_data} is the created task.

\crossreferences
\begin{itemize}
\item \code{ompt\_data\_t} type signature, see
\specref{subsec:ompt_data_t}.
\item \code{ompt\_task\_dependence\_t} type signature, see
\specref{subsec:ompt_task_dependence_t}.
\end{itemize}



\subsection{\code{ompt\_task\_dependence\_callback\_t}}
\index{ompt\_task\_dependence\_callback\_t@{\code{ompt\_task\_dependence\_callback\_t}}}
\label{subsec:ompt_task_dependence_callback_t}
\format
\begin{boxedcode}
typedef void (*ompt\_task\_dependence\_callback\_t) (
  ompt\_data\_t *\plc{src\_task\_data},
  ompt\_data\_t *\plc{sink\_task\_data}
);
\end{boxedcode}
\descr
The binding of \plc{src\_task\_data} is a running task with an out dependency.
The binding of \plc{sink\_task\_data} is a task with a non-satisfied in dependency.

\crossreferences
\begin{itemize}
\item \code{ompt\_data\_t} type signature, see
\specref{subsec:ompt_data_t}.
\end{itemize}



\subsection{\code{ompt\_task\_schedule\_callback\_t}}
\index{ompt\_task\_schedule\_callback\_t@{\code{ompt\_task\_schedule\_callback\_t}}}
\label{subsec:ompt_task_schedule_callback_t}
\format
\begin{boxedcode}
typedef void (*ompt\_task\_schedule\_callback\_t) (
  ompt\_data\_t *\plc{prior\_task\_data},
  \_Bool \plc{prior\_completed},
  ompt\_data\_t *\plc{next\_task\_data}
);
\end{boxedcode}
\descr
The callback with type signature \code{ompt\_task\_schedule\_callback\_t},
include a parameter \plc{prior\_completed}
that indicates whether the task that reached the task scheduling point finished
execution its task region.
The binding of \plc{prior\_task\_data} is the task that reached the task scheduling point.
The binding of \plc{next\_task\_data} is the task to be scheduled.

\crossreferences
\begin{itemize}
\item \code{ompt\_data\_t} type signature, see
\specref{subsec:ompt_data_t}.
\end{itemize}



\subsection{\code{ompt\_scoped\_implicit\_callback\_t}}
\index{ompt\_scoped\_implicit\_callback\_t@{\code{ompt\_scoped\_implicit\_callback\_t}}}
\label{subsec:ompt_scoped_implicit_callback_t}
\format
\begin{boxedcode}
typedef void (*ompt\_scoped\_implicit\_callback\_t) (
  ompt\_scope\_endpoint\_t \plc{endpoint},
  ompt\_data\_t *\plc{parallel\_data},
  ompt\_data\_t *\plc{task\_data},
  uint32\_t \plc{thread\_num}
);
\end{boxedcode}
\descr
The callback with type signature \code{ompt\_scoped\_implicit\_callback\_t},
includes a parameter \plc{endpoint} that indicates whether the callback signals
the begin or end of the scope and a parameter \plc{thread\_num} that indicates
the thread number of the calling thread, within the team executing the parallel
region to which the implicit region binds.
The binding of \plc{parallel\_data} is the innermost enclosing parallel region.
The binding of \plc{task\_data} is the implicit task executing the parallel
region's structured block.


\crossreferences
\begin{itemize}
\item \code{ompt\_scope\_endpoint\_t} type signature, see
\specref{subsec:ompt_scope_endpoint_t}.
\item \code{ompt\_data\_t} type signature, see
\specref{subsec:ompt_data_t}.
\end{itemize}



\subsection{\code{ompt\_scoped\_sync\_region\_callback\_t}}
\index{ompt\_scoped\_sync\_region\_callback\_t@{\code{ompt\_scoped\_sync\_region\_callback\_t}}}
\label{subsec:ompt_scoped_sync_region_callback_t}
\format
\begin{boxedcode}
typedef void (*ompt\_scoped\_sync\_region\_callback\_t) (
  ompt\_sync\_region\_kind\_t \plc{kind},
  ompt\_scope\_endpoint\_t \plc{endpoint},
  ompt\_data\_t *\plc{parallel\_data},
  ompt\_data\_t *\plc{task\_data},
  const void *\plc{codeptr\_ra}
);
\end{boxedcode}
\descr ToDo
The callbacks with type signature \code{ompt\_scoped\_sync\_region\_callback\_t},
include a parameter \plc{requested\_team\_size}
that indicates the number of threads requested by the user, and a parameter
\plc{actual\_team\_size} that indicates the number of threads in the team.
The \plc{invoker} argument explains whether the execution of the parallel
region code is inlined into the application code or started by the runtime.

include a parameter \plc{codeptr\_ra} which contains the return address of the
call to the OpenMP runtime routine and a parameter \plc{endpoint}
that indicates whether the callback signals the begin or end of the scope.

\crossreferences
\begin{itemize}
\item \code{ompt\_sync\_region\_kind\_t} type signature, see
\specref{subsec:ompt_sync_region_kind_t}.
\item \code{ompt\_scope\_endpoint\_t} type signature, see
\specref{subsec:ompt_scope_endpoint_t}.
\item \code{ompt\_data\_t} type signature, see
\specref{subsec:ompt_data_t}.
\end{itemize}



\subsection{\code{ompt\_lock\_init\_callback\_t}}
\index{ompt\_lock\_init\_callback\_t@{\code{ompt\_lock\_init\_callback\_t}}}
\label{subsec:ompt_lock_init_callback_t}
\format
\begin{boxedcode}
typedef void (*ompt\_lock\_init\_callback\_t) (
  \_Bool \plc{is\_nest\_lock},
  ompt\_wait\_id\_t \plc{wait\_id},
  uint32\_t \plc{hint},
  uint32\_t \plc{kind},
  const void *\plc{codeptr\_ra}
);
\end{boxedcode}
\descr ToDo
The callbacks with type signature \code{ompt\_lock\_init\_callback\_t},
include a parameter \plc{requested\_team\_size}
that indicates the number of threads requested by the user, and a parameter
\plc{actual\_team\_size} that indicates the number of threads in the team.
The \plc{invoker} argument explains whether the execution of the parallel
region code is inlined into the application code or started by the runtime.

\crossreferences
\begin{itemize}
\item \code{ompt\_wait\_id\_t} type signature, see
\specref{subsec:ompt_wait_id_t}.
\end{itemize}



\subsection{\code{ompt\_lock\_destroy\_callback\_t}}
\index{ompt\_lock\_destroy\_callback\_t@{\code{ompt\_lock\_destroy\_callback\_t}}}
\label{subsec:ompt_lock_destroy_callback_t}
\format
\begin{boxedcode}
typedef void (*ompt\_lock\_destroy\_callback\_t) (
  \_Bool \plc{is\_nest\_lock},
  ompt\_wait\_id\_t \plc{wait\_id},
  const void *\plc{codeptr\_ra}
);
\end{boxedcode}
\descr ToDo
The callbacks with type signature \code{ompt\_lock\_destroy\_callback\_t},
include a parameter \plc{requested\_team\_size}
that indicates the number of threads requested by the user, and a parameter
\plc{actual\_team\_size} that indicates the number of threads in the team.
The \plc{invoker} argument explains whether the execution of the parallel
region code is inlined into the application code or started by the runtime.

\crossreferences
\begin{itemize}
\item \code{ompt\_wait\_id\_t} type signature, see
\specref{subsec:ompt_wait_id_t}.
\end{itemize}



\subsection{\code{ompt\_mutex\_acquire\_callback\_t}}
\index{ompt\_mutex\_acquire\_callback\_t@{\code{ompt\_mutex\_acquire\_callback\_t}}}
\label{subsec:ompt_mutex_acquire_callback_t}
\format
\begin{boxedcode}
typedef void (*ompt\_mutex\_acquire\_callback\_t) (
  ompt\_mutex\_kind\_t \plc{kind},
  uint32\_t \plc{hint},
  uint32\_t \plc{impl},
  ompt\_wait\_id\_t \plc{wait\_id},
  const void *\plc{codeptr\_ra}
);
\end{boxedcode}
\descr ToDo
The callbacks with type signature \code{ompt\_mutex\_acquire\_callback\_t},
include a parameter \plc{requested\_team\_size}
that indicates the number of threads requested by the user, and a parameter
\plc{actual\_team\_size} that indicates the number of threads in the team.
The \plc{invoker} argument explains whether the execution of the parallel
region code is inlined into the application code or started by the runtime.

\crossreferences
\begin{itemize}
\item \code{ompt\_mutex\_kind\_t} type signature, see
\specref{subsec:ompt_mutex_kind_t}.
\item \code{ompt\_wait\_id\_t} type signature, see
\specref{subsec:ompt_wait_id_t}.
\end{itemize}



\subsection{\code{ompt\_mutex\_callback\_t}}
\index{ompt\_mutex\_callback\_t@{\code{ompt\_mutex\_callback\_t}}}
\label{subsec:ompt_mutex_callback_t}
\format
\begin{boxedcode}
typedef void (*ompt\_mutex\_callback\_t) (
  ompt\_mutex\_kind\_t \plc{kind},
  ompt\_wait\_id\_t \plc{wait\_id},
  const void *\plc{codeptr\_ra}
);
\end{boxedcode}
\descr ToDo
The callbacks with type signature \code{ompt\_mutex\_callback\_t},
include a parameter \plc{requested\_team\_size}
that indicates the number of threads requested by the user, and a parameter
\plc{actual\_team\_size} that indicates the number of threads in the team.
The \plc{invoker} argument explains whether the execution of the parallel
region code is inlined into the application code or started by the runtime.

\crossreferences
\begin{itemize}
\item \code{ompt\_mutex\_kind\_t} type signature, see
\specref{subsec:ompt_mutex_kind_t}.
\item \code{ompt\_wait\_id\_t} type signature, see
\specref{subsec:ompt_wait_id_t}.
\end{itemize}



\subsection{\code{ompt\_scoped\_nested\_lock\_callback\_t}}
\index{ompt\_scoped\_nested\_lock\_callback\_t@{\code{ompt\_scoped\_nested\_lock\_callback\_t}}}
\label{subsec:ompt_scoped_nested_lock_callback_t}
\format
\begin{boxedcode}
typedef void (*ompt\_scoped\_nested\_lock\_callback\_t) (
  ompt\_scope\_endpoint\_t \plc{endpoint},
  ompt\_wait\_id\_t \plc{wait\_id},
  const void *\plc{codeptr\_ra}
);
\end{boxedcode}
\descr ToDo
The callbacks with type signature \code{ompt\_scoped\_nested\_lock\_callback\_t},
include a parameter \plc{requested\_team\_size}
that indicates the number of threads requested by the user, and a parameter
\plc{actual\_team\_size} that indicates the number of threads in the team.
The \plc{invoker} argument explains whether the execution of the parallel
region code is inlined into the application code or started by the runtime.

include a parameter \plc{codeptr\_ra} which contains the return address of the
call to the OpenMP runtime routine and a parameter \plc{endpoint}
that indicates whether the callback signals the begin or end of the scope.

\crossreferences
\begin{itemize}
\item \code{ompt\_scope\_endpoint\_t} type signature, see
\specref{subsec:ompt_scope_endpoint_t}.
\item \code{ompt\_wait\_id\_t} type signature, see
\specref{subsec:ompt_wait_id_t}.
\end{itemize}



\subsection{\code{ompt\_scoped\_worksharing\_callback\_t}}
\index{ompt\_scoped\_worksharing\_callback\_t@{\code{ompt\_scoped\_worksharing\_callback\_t}}}
\label{subsec:ompt_scoped_worksharing_callback_t}
\format
\begin{boxedcode}
typedef void (*ompt\_scoped\_worksharing\_callback\_t) (
  ompt\_worksharing\_type\_t \plc{wstype},
  ompt\_scope\_endpoint\_t \plc{endpoint},
  ompt\_data\_t *\plc{parallel\_data},
  ompt\_data\_t *\plc{task\_data},
  const void *\plc{codeptr\_ra}
);
\end{boxedcode}
\descr ToDo
The callbacks with type signature \code{ompt\_scoped\_worksharing\_callback\_t},
include a parameter \plc{requested\_team\_size}
that indicates the number of threads requested by the user, and a parameter
\plc{actual\_team\_size} that indicates the number of threads in the team.
The \plc{invoker} argument explains whether the execution of the parallel
region code is inlined into the application code or started by the runtime.

include a parameter \plc{codeptr\_ra} which contains the return address of the
call to the OpenMP runtime routine and a parameter \plc{endpoint}
that indicates whether the callback signals the begin or end of the scope.

\crossreferences
\begin{itemize}
\item \code{ompt\_worksharing\_type\_t} type signature, see
\specref{subsec:ompt_worksharing_type_t}.
\item \code{ompt\_scope\_endpoint\_t} type signature, see
\specref{subsec:ompt_scope_endpoint_t}.
\item \code{ompt\_data\_t} type signature, see
\specref{subsec:ompt_data_t}.
\end{itemize}



\subsection{\code{ompt\_flush\_callback\_t}}
\index{ompt\_flush\_callback\_t@{\code{ompt\_flush\_callback\_t}}}
\label{subsec:ompt_flush_callback_t}
\format
\begin{boxedcode}
typedef void (*ompt\_flush\_callback\_t) (
  ompt\_data\_t *\plc{thread\_data},
  const void *\plc{codeptr\_ra}
);
\end{boxedcode}
\descr ToDo
The callbacks with type signature \code{ompt\_flush\_callback\_t},
include a parameter \plc{requested\_team\_size}
that indicates the number of threads requested by the user, and a parameter
\plc{actual\_team\_size} that indicates the number of threads in the team.
The \plc{invoker} argument explains whether the execution of the parallel
region code is inlined into the application code or started by the runtime.

\crossreferences
\begin{itemize}
\item \code{ompt\_data\_t} type signature, see
\specref{subsec:ompt_data_t}.
\end{itemize}



\subsection{\code{ompt\_scoped\_target\_callback\_t}}
\index{ompt\_scoped\_target\_callback\_t@{\code{ompt\_scoped\_target\_callback\_t}}}
\label{subsec:ompt_scoped_target_callback_t}
\format
\begin{boxedcode}
typedef void (*ompt\_scoped\_target\_callback\_t) (
  int32\_t \plc{device\_id},
  ompt\_target\_type\_t \plc{kind},
  ompt\_data\_t *\plc{task\_data},
  ompt\_scope\_endpoint\_t \plc{endpoint},
  ompt\_id\_t \plc{target\_id},
  const void *\plc{codeptr\_ra}
);
\end{boxedcode}
\descr ToDo
The callbacks with type signature \code{ompt\_scoped\_target\_callback\_t},
include a parameter \plc{requested\_team\_size}
that indicates the number of threads requested by the user, and a parameter
\plc{actual\_team\_size} that indicates the number of threads in the team.
The \plc{invoker} argument explains whether the execution of the parallel
region code is inlined into the application code or started by the runtime.

include a parameter \plc{codeptr\_ra} which contains the return address of the
call to the OpenMP runtime routine and a parameter \plc{endpoint}
that indicates whether the callback signals the begin or end of the scope.

\crossreferences
\begin{itemize}
\item \code{ompt\_target\_type\_t} type signature, see
\specref{subsec:ompt_target_type_t}.
\item \code{ompt\_data\_t} type signature, see
\specref{subsec:ompt_data_t}.
\item \code{ompt\_scope\_endpoint\_t} type signature, see
\specref{subsec:ompt_scope_endpoint_t}.
\item \code{ompt\_id\_t} type signature, see
\specref{subsec:ompt_id_t}.
\end{itemize}



\subsection{\code{ompt\_target\_data\_callback\_t}}
\index{ompt\_target\_data\_callback\_t@{\code{ompt\_target\_data\_callback\_t}}}
\label{subsec:ompt_target_data_callback_t}
\format
\begin{boxedcode}
typedef void (*ompt\_target\_data\_callback\_t) (
  ompt\_id\_t \plc{target\_id},
  ompt\_id\_t \plc{host\_op\_id},
  ompt\_target\_data\_op\_t \plc{optype},
  void *\plc{host\_addr},
  void *\plc{device\_addr},
  size\_t \plc{bytes}
);
\end{boxedcode}
\descr ToDo
The callbacks with type signature \code{ompt\_target\_data\_callback\_t},
include a parameter \plc{requested\_team\_size}
that indicates the number of threads requested by the user, and a parameter
\plc{actual\_team\_size} that indicates the number of threads in the team.
The \plc{invoker} argument explains whether the execution of the parallel
region code is inlined into the application code or started by the runtime.

\crossreferences
\begin{itemize}
\item \code{ompt\_id\_t} type signature, see
\specref{subsec:ompt_id_t}.
\item \code{ompt\_target\_data\_op\_t} type signature, see
\specref{subsec:ompt_target_data_op_t}.
\end{itemize}



\subsection{\code{ompt\_target\_data\_map\_callback\_t}}
\index{ompt\_target\_data\_map\_callback\_t@{\code{ompt\_target\_data\_map\_callback\_t}}}
\label{subsec:ompt_target_data_map_callback_t}
\format
\begin{boxedcode}
typedef void (*ompt\_target\_data\_map\_callback\_t) (
  ompt\_id\_t \plc{target\_id},
  uint32\_t \plc{nitems},
  void **\plc{host\_addr},
  void **\plc{device\_addr},
  size\_t *\plc{bytes},
  uint32\_t *\plc{mapping\_flags}
);
\end{boxedcode}
\descr ToDo
The callbacks with type signature \code{ompt\_target\_data\_map\_callback\_t},
include a parameter \plc{requested\_team\_size}
that indicates the number of threads requested by the user, and a parameter
\plc{actual\_team\_size} that indicates the number of threads in the team.
The \plc{invoker} argument explains whether the execution of the parallel
region code is inlined into the application code or started by the runtime.

\crossreferences
\begin{itemize}
\item \code{ompt\_id\_t} type signature, see
\specref{subsec:ompt_id_t}.
\end{itemize}



\subsection{\code{ompt\_target\_submit\_callback\_t}}
\index{ompt\_target\_submit\_callback\_t@{\code{ompt\_target\_submit\_callback\_t}}}
\label{subsec:ompt_target_submit_callback_t}
\format
\begin{boxedcode}
typedef void (*ompt\_target\_submit\_callback\_t) (
  ompt\_id\_t \plc{target\_id},
  ompt\_id\_t \plc{host\_op\_id},
  uint32\_t \plc{requested\_num\_teams},
  uint32\_t \plc{granted\_num\_teams}
);
\end{boxedcode}
\descr ToDo
The callbacks with type signature \code{ompt\_target\_submit\_callback\_t},
include a parameter \plc{requested\_team\_size}
that indicates the number of threads requested by the user, and a parameter
\plc{actual\_team\_size} that indicates the number of threads in the team.
The \plc{invoker} argument explains whether the execution of the parallel
region code is inlined into the application code or started by the runtime.

\crossreferences
\begin{itemize}
\item \code{ompt\_id\_t} type signature, see
\specref{subsec:ompt_id_t}.
\end{itemize}


\begin{targetRecord}
\subsection{\code{ompt\_target\_buffer\_request\_callback\_t}}
\index{ompt\_target\_buffer\_request\_callback\_t@{\code{ompt\_target\_buffer\_request\_callback\_t}}}
\label{subsec:ompt_target_buffer_request_callback_t}
\format
\begin{boxedcode}
typedef void (*ompt\_target\_buffer\_request\_callback\_t) (
  int32\_t \plc{device\_id},
  ompt\_target\_buffer\_t** \plc{buffer},
  size\_t *\plc{bytes}
);
\end{boxedcode}
\descr ToDo
The callbacks with type signature \code{ompt\_target\_buffer\_request\_callback\_t},
include a parameter \plc{requested\_team\_size}
that indicates the number of threads requested by the user, and a parameter
\plc{actual\_team\_size} that indicates the number of threads in the team.
The \plc{invoker} argument explains whether the execution of the parallel
region code is inlined into the application code or started by the runtime.

\crossreferences
\begin{itemize}
\item \code{ompt\_target\_buffer\_t} type signature, see
\specref{subsec:ompt_target_buffer_t}.
\end{itemize}



\subsection{\code{ompt\_target\_buffer\_complete\_callback\_t}}
\index{ompt\_target\_buffer\_complete\_callback\_t@{\code{ompt\_target\_buffer\_complete\_callback\_t}}}
\label{subsec:ompt_target_buffer_complete_callback_t}
\format
\begin{boxedcode}
typedef void (*ompt\_target\_buffer\_complete\_callback\_t) (
  int32\_t \plc{device\_id},
  const ompt\_target\_buffer\_t *\plc{buf},
  size\_t \plc{bytes},
  ompt\_target\_buffer\_cursor\_t \plc{begin},
  \_Bool \plc{buffer\_owned}
);
\end{boxedcode}
\descr ToDo
The callbacks with type signature \code{ompt\_target\_buffer\_complete\_callback\_t},
include a parameter \plc{requested\_team\_size}
that indicates the number of threads requested by the user, and a parameter
\plc{actual\_team\_size} that indicates the number of threads in the team.
The \plc{invoker} argument explains whether the execution of the parallel
region code is inlined into the application code or started by the runtime.

\crossreferences
\begin{itemize}
\item \code{ompt\_target\_buffer\_t} type signature, see
\specref{subsec:ompt_target_buffer_t}.
\item \code{ompt\_target\_buffer\_cursor\_t} type signature, see
\specref{subsec:ompt_target_buffer_cursor_t}.
\end{itemize}



\subsection{\code{ompt\_get\_target\_info\_inquiry\_t}}
\index{ompt\_get\_target\_info\_inquiry\_t@{\code{ompt\_get\_target\_info\_inquiry\_t}}}
\label{subsec:ompt_get_target_info_inquiry_t}
\format
\begin{boxedcode}
typedef void (*ompt\_get\_target\_info\_inquiry\_t) (
  ompt\_id\_t *\plc{target\_id},
  ompt\_id\_t *\plc{host\_op\_id}
);
\end{boxedcode}
\descr ToDo
The callbacks with type signature \code{ompt\_get\_target\_info\_inquiry\_t},
include a parameter \plc{requested\_team\_size}
that indicates the number of threads requested by the user, and a parameter
\plc{actual\_team\_size} that indicates the number of threads in the team.
The \plc{invoker} argument explains whether the execution of the parallel
region code is inlined into the application code or started by the runtime.

\crossreferences
\begin{itemize}
\item \code{ompt\_id\_t} type signature, see
\specref{subsec:ompt_id_t}.
\end{itemize}
\end{targetRecord}


\subsection{\code{ompt\_control\_callback\_t}}
\index{ompt\_control\_callback\_t@{\code{ompt\_control\_callback\_t}}}
\label{subsec:ompt_control_callback_t}
\format
\begin{boxedcode}
typedef void (*ompt\_control\_callback\_t) (
  uint64\_t \plc{command},
  uint64\_t \plc{modifier},
  const void *\plc{codeptr\_ra}
);
\end{boxedcode}
\descr ToDo
The callbacks with type signature \code{ompt\_control\_callback\_t},
include a parameter \plc{requested\_team\_size}
that indicates the number of threads requested by the user, and a parameter
\plc{actual\_team\_size} that indicates the number of threads in the team.
The \plc{invoker} argument explains whether the execution of the parallel
region code is inlined into the application code or started by the runtime.











% This is the end of ch5-toolsSupport.tex
