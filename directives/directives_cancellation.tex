% This is an included file. See the master file for more information.
%
% When editing this file:
%
%    1. To change formatting, appearance, or style, please edit openmp.sty.
%
%    2. Custom commands and macros are defined in openmp.sty.
%
%    3. Be kind to other editors -- keep a consistent style by copying-and-pasting to
%       create new content.
%
%    4. We use semantic markup, e.g. (see openmp.sty for a full list):
%         \code{}     % for bold monospace keywords, code, operators, etc.
%         \plc{}      % for italic placeholder names, grammar, etc.
%
%    5. There are environments that provide special formatting, e.g. language bars.
%       Please use them whereever appropriate.  Examples are:
%
%         \begin{fortranspecific}
%         This is text that appears enclosed in blue language bars for Fortran.
%         \end{fortranspecific}
%
%         \begin{note}
%         This is a note.  The "Note -- " header appears automatically.
%         \end{note}
%
%    6. Other recommendations:
%         Use the convenience macros defined in openmp.sty for the minor headers
%         such as Comments, Syntax, etc.
%
%         To keep items together on the same page, prefer the use of
%         \begin{samepage}.... Avoid \parbox for text blocks as it interrupts line numbering.
%         When possible, avoid \filbreak, \pagebreak, \newpage, \clearpage unless that's
%         what you mean. Use \needspace{} cautiously for troublesome paragraphs.
%
%         Avoid absolute lengths and measures in this file; use relative units when possible.
%         Vertical space can be relative to \baselineskip or ex units. Horizontal space
%         can be relative to \linewidth or em units.
%
%         Prefer \emph{} to italicize terminology, e.g.:
%             This is a \emph{definition}, not a placeholder.
%             This is a \plc{var-name}.
%


\section{Cancellation Constructs}
\label{sec:Cancellation Constructs}
\index{cancellation constructs}
\index{constructs!cancellation constructs}
\subsection{\hcode{cancel} Construct}
\index{cancel@{\code{cancel}}}
\index{constructs!cancel@{\code{cancel}}}
\index{cancellation constructs!cancel@{\code{cancel}}}
\label{subsec:cancel Construct}
\summary
The \code{cancel} construct activates cancellation of the innermost enclosing 
region of the type specified. The \code{cancel} construct is a stand-alone directive.

\syntax
\begin{ccppspecific}
The syntax of the \code{cancel} construct is as follows:

\begin{ompcPragma}
#pragma omp cancel \plc{construct-type-clause [ [},\plc{] if-clause] new-line}
\end{ompcPragma}

\begin{samepage}
where \plc{construct-type-clause} is one of the following:

\begin{indentedcodelist}
parallel
sections
for
taskgroup
\end{indentedcodelist}
\end{samepage}

and \plc{if-clause} is
\begin{indentedcodelist}
if (\plc{[} cancel :\plc{] scalar-expression})
\end{indentedcodelist}
\end{ccppspecific}

\begin{fortranspecific}
The syntax of the \code{cancel} construct is as follows:

\begin{ompfPragma}
!$omp cancel \plc{construct-type-clause [ [},\plc{] if-clause]}
\end{ompfPragma}


\begin{samepage}
where \plc{construct-type-clause} is one of the following:
\begin{indentedcodelist}
parallel
sections
do
taskgroup
\end{indentedcodelist}
\end{samepage}

and \plc{if-clause} is
\begin{indentedcodelist}
if (\plc{[} cancel :\plc{] scalar-logical-expression})
\end{indentedcodelist}
\end{fortranspecific}


\binding
The binding thread set of the \code{cancel} region is the current team.
The binding region of the \code{cancel} region is the innermost enclosing 
region of the type corresponding to the \plc{construct-type-clause}
specified in the directive (that is, the innermost \code{parallel}, 
\code{sections}, worksharing-loop, or \code{taskgroup} region).

\descr
The \code{cancel} construct activates cancellation of the binding region 
only if the \plc{cancel-var} ICV is \plc{true}, in which case the \code{cancel} 
construct causes the encountering task to continue execution at the end of the 
binding region if \plc{construct-type-clause} is \code{parallel}, \code{for}, 
\code{do}, or \code{sections}. If the \plc{cancel-var} ICV is \plc{true} and 
\plc{construct-type-clause} is \code{taskgroup}, the encountering task 
continues execution at the end of the current task region. If the 
\plc{cancel-var} ICV is \plc{false}, the \code{cancel} construct is ignored.

Threads check for active cancellation only at cancellation points that are
implied at the following locations:

\begin{itemize}
\item \code{cancel} regions;
\item \code{cancellation}~\code{point} regions;
\item \code{barrier} regions;
\item implicit barriers regions.
\end{itemize}

When a thread reaches one of the above cancellation points and if the 
\plc{cancel-var} ICV is \plc{true},  then: 

\begin{itemize}
\item If the thread is at a \code{cancel} or \code{cancellation point} region
      and \plc{construct-type-clause} is \code{parallel}, \code{for}, \code{do}, 
      or \code{sections}, the thread continues execution at the end of the canceled
      region if cancellation has been activated for the innermost enclosing region
      of the type specified.
\item If the thread is at a \code{cancel} or \code{cancellation point} region
      and \plc{construct-type-clause} is \code{taskgroup}, the encountering task
      checks for active cancellation of all of the \plc{taskgroup sets} to which
      the encountering task belongs, and continues execution at the end of the
      current task region if cancellation has been activated for any of 
      the \plc{taskgroup sets}.
\item If the encountering task is at a barrier region, the encountering task
      checks for active cancellation of the innermost enclosing \code{parallel}
      region. If cancellation has been activated, then the encountering task
      continues execution at the end of the canceled region.
\end{itemize}

\begin{note}
If one thread activates cancellation and another thread encounters a cancellation
point, the order of execution between the two threads is non-deterministic.
Whether the thread that encounters a cancellation point detects the activated 
cancellation depends on the underlying hardware and operating system.
\end{note}

When cancellation of tasks is activated through a \code{cancel} construct with
the \code{taskgroup} \plc{construct-type-clause}, the tasks that belong to the
\plc{taskgroup set} of the innermost enclosing \code{taskgroup} region
will be canceled. The task that encountered that construct continues execution
at the end of its task region, which implies completion of that
task. Any task that belongs to the innermost enclosing
\code{taskgroup} and has already begun execution must run to
completion or until a cancellation point is reached. Upon reaching a
cancellation point and if cancellation is active, the task continues
execution at the end of its task region, which implies the
task's completion. Any task that belongs to the innermost enclosing
\code{taskgroup} and that has not begun execution may be discarded,
which implies its completion.

When cancellation is active for a \code{parallel}, \code{sections}, or
worksharing-loop region, each thread of the binding thread set resumes 
execution at the end of the canceled region if a cancellation point is 
encountered. If the canceled region is a \code{parallel} region, any
tasks that have been created by a \code{task} or a \code{taskloop} construct and their descendent 
tasks are canceled according to the above \code{taskgroup} cancellation 
semantics. If the canceled region is a \code{sections}, or worksharing-loop 
region, no task cancellation occurs.

\begin{cppspecific}
The usual C++ rules for object destruction are followed when cancellation is performed.
\end{cppspecific}

\begin{fortranspecific}
All private objects or subobjects with \code{ALLOCATABLE} attribute that 
are allocated inside the canceled construct are deallocated.
\end{fortranspecific}

If the canceled construct contains a \code{reduction}, \code{task_reduction} 
or \code{lastprivate} clause, the final value of the list items that 
appeared in those clauses are undefined.

When an \code{if} clause is present on a \code{cancel} construct and the 
\code{if} expression evaluates to \plc{false}, the \code{cancel} construct 
does not activate cancellation. The cancellation point associated with the 
\code{cancel} construct is always encountered regardless of the value of
the \code{if} expression.

\begin{note}
The programmer is responsible for releasing locks and
other synchronization data structures that might cause a deadlock when
a \code{cancel} construct is encountered and blocked threads cannot be
canceled. The programmer is also responsible for ensuring proper
synchronizations to avoid deadlocks that might arise from cancellation
of OpenMP regions that contain OpenMP synchronization constructs.
\end{note}

\events

If a task encounters a \code{cancel} construct that will
  activate cancellation then a \plc{cancel} event occurs.

A \plc{discarded-task} event occurs for any discarded tasks.

\tools

A thread dispatches a registered \code{ompt_callback_cancel} callback 
for each occurrence of a \plc{cancel} event in the context of the encountering 
task. This callback has type signature \code{ompt_callback_cancel_t}; 
\code{(}\plc{flags}~\code{&}~\code{ompt_cancel_activated)} always evaluates 
to \plc{true} in the dispatched callback; 
\code{(}\plc{flags}~\code{&}~\code{ompt_cancel_parallel)} evaluates to 
\plc{true} in the dispatched callback if \plc{construct-type-clause} is
\code{parallel}; 
\code{(}\plc{flags}~\code{&}~\code{ompt_cancel_sections)} evaluates to 
\plc{true} in the dispatched callback if \plc{construct-type-clause} is
\code{sections}; 
\code{(}\plc{flags}~\code{&}~\code{ompt_cancel_loop)} evaluates to 
\plc{true} in the dispatched callback if \plc{construct-type-clause} is
\code{for} or \code{do}; and 
\code{(}\plc{flags}~\code{&}~\code{ompt_cancel_taskgroup)} evaluates to 
\plc{true} in the dispatched callback if \plc{construct-type-clause} is
\code{taskgroup}. 

A thread dispatches a registered \code{ompt_callback_cancel} callback with
the \plc{ompt_data_t} associated with the discarded task as its \plc{task_data} 
argument and \code{ompt_cancel_discarded_task} as its \plc{flags} argument
for each occurrence of a \plc{discarded-task} event. The callback occurs in 
the context of the task that discards the task and has type signature 
\code{ompt_callback_cancel_t}.

\restrictions
The restrictions to the \code{cancel} construct are as follows:

\begin{itemize}
\item The behavior for concurrent cancellation of a region and a region nested 
      within it is unspecified.
\item If \plc{construct-type-clause} is \code{taskgroup}, the \code{cancel}
      construct must be closely nested inside a \code{task} or a \code{taskloop} construct and the
      \code{cancel} region must be closely nested inside a \code{taskgroup} 
      region. If \plc{construct-type-clause} is \code{sections}, the 
      \code{cancel} construct must be closely nested inside a \code{sections} 
      or \code{section} construct. Otherwise, the \code{cancel} construct must 
      be closely nested inside an OpenMP construct that matches the type 
      specified in \plc{construct-type-clause} of the \code{cancel} construct.
\item A worksharing construct that is canceled must not have a \code{nowait} clause.
\item A worksharing-loop construct that is canceled must not have an 
      \code{ordered} clause.
\item During execution of a construct that may be subject to cancellation, a
      thread must not encounter an orphaned cancellation point. That is, a
      cancellation point must only be encountered within that construct and must
      not be encountered elsewhere in its region.
\end{itemize}

\crossreferences
\begin{itemize}
\item \plc{cancel-var} ICV, see
\specref{subsec:ICV Descriptions}.

\item \code{if} clause, see \specref{sec:if Clause}.

\item \code{cancellation}~\code{point} construct, see
\specref{subsec:cancellation point Construct}.

\item \code{omp_get_cancellation} routine, see
\specref{subsec:omp_get_cancellation}.

\item \code{omp_cancel_flag_t} enumeration type, see \specref{sec:ompt_cancel_flag_t}.

\item \code{ompt_callback_cancel_t}, see \specref{sec:ompt_callback_cancel_t}.
\end{itemize}



\subsection{\hcode{cancellation}~\hcode{point} Construct}
\index{cancellation point@{\code{cancellation}~\code{point}}}
\index{constructs!cancellation point@{\code{cancellation}~\code{point}}}
\label{subsec:cancellation point Construct}
\index{cancellation constructs!cancellation point@{\code{cancellation}~\code{point}}}
\summary
The \code{cancellation}~\code{point} construct introduces a user-defined 
cancellation point at which implicit or explicit tasks check if cancellation 
of the innermost enclosing region of the type specified has been activated. 
The \code{cancellation}~\code{point} construct is a stand-alone directive.

\syntax
\begin{ccppspecific}
The syntax of the \code{cancellation}~\code{point} construct is as follows:

\begin{ompcPragma}
#pragma omp cancellation point \plc{construct-type-clause new-line}
\end{ompcPragma}

where \plc{construct-type-clause} is one of the following:

\begin{indentedcodelist}
parallel
sections
for
taskgroup
\end{indentedcodelist}
\end{ccppspecific}

\begin{fortranspecific}
The syntax of the \code{cancellation}~\code{point} construct is as follows:

\begin{ompfPragma}
!$omp cancellation point \plc{construct-type-clause}
\end{ompfPragma}

where \plc{construct-type-clause} is one of the following:

\begin{indentedcodelist}
parallel
sections
do
taskgroup
\end{indentedcodelist}
\end{fortranspecific}

\binding
The binding thread set of the \code{cancellation point} construct is the current 
team. The binding region of the \code{cancellation point} region is the innermost 
enclosing region of the type corresponding to the \plc{construct-type-clause}
specified in the directive (that is, the innermost \code{parallel}, \code{sections}, 
worksharing-loop, or \code{taskgroup} region).

\descr
This directive introduces a user-defined cancellation point at which an implicit or
explicit task must check if cancellation of the innermost enclosing region of the type
specified in the clause has been requested. This construct does not implement any
synchronization between threads or tasks.

When an implicit or explicit task reaches a user-defined cancellation point and if
the \plc{cancel-var} ICV is \plc{true}, then:

\begin{itemize}
\item If the \plc{construct-type-clause} of the encountered \code{cancellation point} 
      construct is \code{parallel}, \code{for}, \code{do}, or \code{sections},
      the thread continues execution at the end of the canceled region if
      cancellation has been activated for the innermost enclosing region of
      the type specified.
\item If the \plc{construct-type-clause} of the encountered
      \code{cancellation point} construct is \code{taskgroup}, the encountering
      task checks for active cancellation of all  \plc{taskgroup sets} to which the
      encountering task belongs and continues execution at the end of the current
      task region if cancellation has been activated for any of them.
\end{itemize}

\events

The \plc{cancellation} event occurs if a task encounters a
cancellation point and detected the activation of cancellation.

\tools

A thread dispatches a registered \code{ompt_callback_cancel} callback 
for each occurrence of a \plc{cancel} event in the context of the encountering 
task. This callback has type signature \code{ompt_callback_cancel_t}; 
\code{(}\plc{flags}~\code{&}~\code{ompt_cancel_detected)} always evaluates 
to \plc{true} in the dispatched callback; 
\code{(}\plc{flags}~\code{&}~\code{ompt_cancel_parallel)} evaluates to 
\plc{true} in the dispatched callback if \plc{construct-type-clause} of 
the encountered \code{cancellation point} construct is \code{parallel}; 
\code{(}\plc{flags}~\code{&}~\code{ompt_cancel_sections)} evaluates to 
\plc{true} in the dispatched callback if \plc{construct-type-clause} of 
the encountered \code{cancellation point} construct is \code{sections}; 
\code{(}\plc{flags}~\code{&}~\code{ompt_cancel_loop)} evaluates to 
\plc{true} in the dispatched callback if \plc{construct-type-clause} of 
the encountered \code{cancellation point} construct is \code{for} or 
\code{do}; and
\code{(}\plc{flags}~\code{&}~\code{ompt_cancel_taskgroup)} evaluates to 
\plc{true} in the dispatched callback if \plc{construct-type-clause} of 
the encountered \code{cancellation point} construct is \code{taskgroup}.

\restrictions
\begin{itemize}
\item A \code{cancellation}~\code{point} construct for which 
      \plc{construct-type-clause} is \code{taskgroup} must be closely nested
      inside a \code{task} or \code{taskloop} construct, and the \code{cancellation}~\code{point} region 
      must be closely nested inside a \code{taskgroup} region. 
\item A \code{cancellation}~\code{point} construct for which 
      \plc{construct-type-clause} is \code{sections} must be closely nested inside 
      a \code{sections} or \code{section} construct.
\item A \code{cancellation}~\code{point} construct for which 
      \plc{construct-type-clause} is neither \code{sections} nor \code{taskgroup}
      must be closely nested inside an OpenMP construct that matches the type 
      specified in \plc{construct-type-clause}.
\end{itemize}

\begin{samepage}
\crossreferences
\begin{itemize}
\item \plc{cancel-var} ICV, see
\specref{subsec:ICV Descriptions}.

\item \code{cancel} construct, see
\specref{subsec:cancel Construct}.

\item \code{omp_get_cancellation} routine, see
\specref{subsec:omp_get_cancellation}.

\item \code{ompt_callback_cancel_t}, see \specref{sec:ompt_callback_cancel_t}.

\end{itemize}
\end{samepage}
