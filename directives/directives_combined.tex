% This is an included file. See the master file for more information.
%
% When editing this file:
%
%    1. To change formatting, appearance, or style, please edit openmp.sty.
%
%    2. Custom commands and macros are defined in openmp.sty.
%
%    3. Be kind to other editors -- keep a consistent style by copying-and-pasting to
%       create new content.
%
%    4. We use semantic markup, e.g. (see openmp.sty for a full list):
%         \code{}     % for bold monospace keywords, code, operators, etc.
%         \plc{}      % for italic placeholder names, grammar, etc.
%
%    5. There are environments that provide special formatting, e.g. language bars.
%       Please use them whereever appropriate.  Examples are:
%
%         \begin{fortranspecific}
%         This is text that appears enclosed in blue language bars for Fortran.
%         \end{fortranspecific}
%
%         \begin{note}
%         This is a note.  The "Note -- " header appears automatically.
%         \end{note}
%
%    6. Other recommendations:
%         Use the convenience macros defined in openmp.sty for the minor headers
%         such as Comments, Syntax, etc.
%
%         To keep items together on the same page, prefer the use of
%         \begin{samepage}.... Avoid \parbox for text blocks as it interrupts line numbering.
%         When possible, avoid \filbreak, \pagebreak, \newpage, \clearpage unless that's
%         what you mean. Use \needspace{} cautiously for troublesome paragraphs.
%
%         Avoid absolute lengths and measures in this file; use relative units when possible.
%         Vertical space can be relative to \baselineskip or ex units. Horizontal space
%         can be relative to \linewidth or em units.
%
%         Prefer \emph{} to italicize terminology, e.g.:
%             This is a \emph{definition}, not a placeholder.
%             This is a \plc{var-name}.
%


\section{Combined Constructs}
\label{sec:Combined Constructs}
\index{combined constructs}
\index{constructs!combined constructs}
Combined constructs are shortcuts for specifying one construct immediately nested
inside another construct. The semantics of the combined constructs are identical 
to that of explicitly specifying the first construct containing one instance of 
the second construct and no other statements.

For combined constructs, tool callbacks are invoked as if the constructs were
explicitly nested.



\subsection{Parallel Worksharing-Loop Construct}
\label{subsec:Parallel Worksharing-Loop Construct}
\index{parallel worksharing-loop construct}
\index{constructs!parallel worksharing-loop construct}
\index{constructs!parallel for@{\code{parallel}~\code{for} \emph{C/C++}}}
\index{constructs!parallel do@{\code{parallel}~\code{do} \emph{Fortran}}}
\index{combined constructs!parallel worksharing-loop construct}
\index{worksharing!parallel}
\summary
The parallel worksharing-loop construct is a shortcut for specifying a 
\code{parallel} construct containing one worksharing-loop construct with 
one or more associated loops and no other statements.

\syntax
\begin{ccppspecific}
The syntax of the parallel worksharing-loop construct is as follows:

\begin{ompcPragma}
#pragma omp parallel for \plc{[clause[ [},\plc{] clause] ... ] new-line}
   \plc{for-loops}
\end{ompcPragma}

where \plc{clause} can be any of the clauses accepted by the \code{parallel} 
or \code{for} directives, except the \code{nowait} clause, with identical 
meanings and restrictions.
\end{ccppspecific}

\begin{fortranspecific}
The syntax of the parallel worksharing-loop construct is as follows:

\begin{ompfPragma}
!$omp parallel do \plc{[clause[ [},\plc{] clause] ... ]}
   \plc{do-loops}
\plc{[}!$omp end parallel do\plc{]}
\end{ompfPragma}

where \plc{clause} can be any of the clauses accepted by the \code{parallel} 
or \code{do} directives, with identical meanings and restrictions.

If an \code{end}~\code{parallel}~\code{do} directive is not specified, an 
\code{end}~\code{parallel}~\code{do} directive is assumed at the end of the 
\plc{do-loops}. \code{nowait} may not be specified on an
\code{end}~\code{parallel}~\code{do} directive.
\end{fortranspecific}

\descr
The semantics are identical to explicitly specifying a \code{parallel} directive 
immediately followed by a worksharing-loop directive.

\restrictions
\begin{itemize}
\item The restrictions for the \code{parallel} construct and the
      worksharing-loop construct apply.
\end{itemize}

\crossreferences
\begin{itemize}
\item \code{parallel} construct, see
\specref{sec:parallel Construct}.

\item Worksharing-loop construct, see
\specref{subsec:Worksharing-Loop Construct}.

\item Data attribute clauses, see
\specref{subsec:Data-Sharing Attribute Clauses}.
\end{itemize}



\subsection{\hcode{parallel}~\hcode{loop} Construct}
\label{subsec:parallel loop Construct}
\index{parallel loop@{\code{parallel}~\code{loop}}}
\index{constructs!parallel loop@{\code{parallel}~\code{loop}}}
\index{combined constructs!parallel loop@{\code{parallel}~\code{loop}}}
\summary
The \code{parallel}~\code{loop} construct is a shortcut for specifying a 
\code{parallel} construct containing one \code{loop} construct with one or 
more associated loops and no other statements.

\syntax
\begin{ccppspecific}
The syntax of the \code{parallel}~\code{loop} construct is as follows:

\begin{ompcPragma}
#pragma omp parallel loop \plc{[clause[ [},\plc{] clause] ... ] new-line}
   \plc{for-loops}
\end{ompcPragma}

where \plc{clause} can be any of the clauses accepted by the \code{parallel} or
\code{loop} directives, with identical meanings and restrictions.
\end{ccppspecific}

\begin{fortranspecific}
The syntax of the \code{parallel} \code{loop} construct is as follows:

\begin{ompfPragma}
!$omp parallel loop \plc{[clause[ [},\plc{] clause] ... ]}
   \plc{do-loops}
\plc{[}!$omp end parallel loop\plc{]}
\end{ompfPragma}

where \plc{clause} can be any of the clauses accepted by the \code{parallel} or
  \code{loop} directives, with identical meanings and restrictions.

If an \code{end}~\code{parallel}~\code{loop} directive is not specified, an
\code{end}~\code{parallel}~\code{loop} directive is assumed at the end of the
\plc{do-loops}. \code{nowait} may not be specified on an
\code{end}~\code{parallel}~\code{loop} directive.
\end{fortranspecific}

\descr
The semantics are identical to explicitly specifying a \code{parallel} directive 
immediately followed by a \code{loop} directive. 

\restrictions
\begin{itemize}
\item The restrictions for the \code{parallel} construct and the
  \code{loop} construct apply.
\end{itemize}

\crossreferences
\begin{itemize}
\item \code{parallel} construct, see
\specref{sec:parallel Construct}.

\item \code{loop} construct, see
\specref{subsec:loop Construct}.

\item Data attribute clauses, see
\specref{subsec:Data-Sharing Attribute Clauses}.
\end{itemize}




\subsection{\hcode{parallel}~\hcode{sections} Construct}
\label{subsec:parallel sections Construct}
\index{parallel sections@{\code{parallel}~\code{sections}}}
\index{constructs!parallel sections@{\code{parallel}~\code{sections}}}
\index{combined constructs!parallel sections@{\code{parallel}~\code{sections}}}
\summary
The \code{parallel}~\code{sections} construct is a shortcut for specifying a 
\code{parallel} construct containing one \code{sections} construct and no other 
statements.

\syntax
\begin{ccppspecific}
The syntax of the \code{parallel}~\code{sections} construct is as follows:

\begin{ompcPragma}
#pragma omp parallel sections \plc{[clause[ [},\plc{] clause] ... ] new-line}
    {
    \plc{[}#pragma omp section \plc{new-line]}
        \plc{structured-block}
    \plc{[}#pragma omp section \plc{new-line}
        \plc{structured-block]}
    \plc{...}
    }
\end{ompcPragma}

where \plc{clause} can be any of the clauses accepted by the \code{parallel} 
or \code{sections} directives, except the \code{nowait} clause, with identical 
meanings and restrictions.
\end{ccppspecific}

\begin{fortranspecific}
The syntax of the \code{parallel}~\code{sections} construct is as follows:

\begin{ompfPragma}
!$omp parallel sections \plc{[clause[ [},\plc{] clause] ... ]}
    \plc{[}!$omp section\plc{]}
        \plc{structured-block}
    \plc{[}!$omp section
        \plc{structured-block]}
    \plc{...}
!$omp end parallel sections
\end{ompfPragma}

where \plc{clause} can be any of the clauses accepted by the \code{parallel} 
or \code{sections} directives, with identical meanings and restrictions.

The last section ends at the \code{end}~\code{parallel}~\code{sections} directive. 
\code{nowait} cannot be specified on an \code{end}~\code{parallel}~\code{sections} 
directive.
\end{fortranspecific}

\descr
\begin{ccppspecific}
The semantics are identical to explicitly specifying a \code{parallel} directive 
immediately followed by a \code{sections} directive.
\end{ccppspecific}

\begin{fortranspecific}
The semantics are identical to explicitly specifying a \code{parallel} directive 
immediately followed by a \code{sections} directive, and an \code{end}~\code{sections}
directive immediately followed by an \code{end}~\code{parallel} directive.
\end{fortranspecific}

\restrictions
The restrictions for the \code{parallel} construct and the \code{sections} 
construct apply.

\crossreferences
\begin{itemize}
\item \code{parallel} construct, see
\specref{sec:parallel Construct}.

\item \code{sections} construct, see
\specref{subsec:sections Construct}.

\item Data attribute clauses, see
\specref{subsec:Data-Sharing Attribute Clauses}.
\end{itemize}



\begin{fortranspecific}
\subsection{\hcode{parallel}~\hcode{workshare} Construct}
\label{subsec:parallel workshare Construct}
\index{parallel workshare@{\code{parallel}~\code{workshare}}}
\index{constructs!parallel workshare@{\code{parallel}~\code{workshare}}}
\index{combined constructs!parallel workshare@{\code{parallel}~\code{workshare}}}
\summary
The \code{parallel}~\code{workshare} construct is a shortcut for specifying a 
\code{parallel} construct containing one \code{workshare} construct and no 
other statements.

\syntax
The syntax of the \code{parallel}~\code{workshare} construct is as follows:

\begin{ompfPragma}
!$omp parallel workshare \plc{[clause[ [},\plc{] clause] ... ]}
   \plc{structured-block }
!$omp end parallel workshare
\end{ompfPragma}

where \plc{clause} can be any of the clauses accepted by the \code{parallel} 
directive, with identical meanings and restrictions. \code{nowait} may not be 
specified on an \code{end}~\code{parallel}~\code{workshare} directive.

\descr
The semantics are identical to explicitly specifying a \code{parallel} directive 
immediately followed by a \code{workshare} directive, and an 
\code{end}~\code{workshare} directive immediately followed by an 
\code{end}~\code{parallel} directive.

\restrictions
The restrictions for the \code{parallel} construct and the \code{workshare} 
construct apply.

\crossreferences
\begin{itemize}
\item \code{parallel} construct, see
\specref{sec:parallel Construct}.

\item \code{workshare} construct, see
\specref{subsec:workshare Construct}.

\item Data attribute clauses, see
\specref{subsec:Data-Sharing Attribute Clauses}.
\end{itemize}
\end{fortranspecific}



\subsection{Parallel Worksharing-Loop SIMD Construct}
\label{subsec:Parallel Worksharing-Loop SIMD Construct}
\index{parallel worksharing-loop SIMD construct}
\index{constructs!parallel worksharing-loop SIMD construct}
\index{combined constructs!parallel worksharing-loop SIMD construct}
\summary
The parallel worksharing-loop SIMD construct is a shortcut for specifying a 
\code{parallel} construct containing one worksharing-loop SIMD construct and 
no other statements.

\begin{samepage}
\syntax
\begin{ccppspecific}
The syntax of the parallel worksharing-loop SIMD construct is as follows:

\begin{ompcPragma}
#pragma omp parallel for simd \plc{[clause[ [},\plc{] clause] ... ] new-line}
    \plc{for-loops}
\end{ompcPragma}

where \plc{clause} can be any of the clauses accepted by the \code{parallel}
or \code{for}~\code{simd} directives, except the \code{nowait} clause, with
identical meanings and restrictions.
\end{ccppspecific}
\end{samepage}

\begin{fortranspecific}
\begin{samepage}
The syntax of the parallel worksharing-loop SIMD construct is as follows:

\begin{ompfPragma}
!$omp parallel do simd \plc{[clause[ [},\plc{] clause] ... ]}
    \plc{do-loops}
\plc{[}!$omp end parallel do simd\plc{]}
\end{ompfPragma}
\end{samepage}

where \plc{clause} can be any of the clauses accepted by the \code{parallel}
or \code{do}~\code{simd} directives, with identical meanings and restrictions.

\begin{samepage}
If an \code{end}~\code{parallel}~\code{do}~\code{simd} directive is not specified, an
\code{end}~\code{parallel}~\code{do}~\code{simd} directive is assumed at the end of the
\plc{do-loops}. \code{nowait} may not be specified on
an \code{end}~\code{parallel} \code{do}~\code{simd} directive.
\end{samepage}
\end{fortranspecific}

\descr
The semantics of the parallel worksharing-loop SIMD construct are identical 
to explicitly specifying a \code{parallel} directive immediately followed by 
a worksharing-loop SIMD directive.

\restrictions
The restrictions for the \code{parallel} construct and the worksharing-loop 
SIMD construct apply.

\crossreferences
\begin{itemize}
\item \code{parallel} construct, see
\specref{sec:parallel Construct}.

\item Worksharing-loop SIMD construct, see
\specref{subsubsec:Worksharing-Loop SIMD Construct}.

\item Data attribute clauses, see
\specref{subsec:Data-Sharing Attribute Clauses}.
\end{itemize}



\subsection{\hcode{teams}~\hcode{distribute} Construct}
\label{subsec:teams distribute Construct}
\index{teams distribute@{\code{teams}~\code{distribute}}}
\index{constructs!teams distribute@{\code{teams}~\code{distribute}}}
\index{combined constructs!teams distribute@{\code{teams}~\code{distribute}}}
\summary
The \code{teams}~\code{distribute} construct is a shortcut for specifying 
a \code{teams} construct containing a \code{distribute} construct and no 
other statements.

\syntax
\begin{ccppspecific}
The syntax of the \code{teams}~\code{distribute} construct is as follows:

\begin{ompcPragma}
#pragma omp teams distribute \plc{[clause[ [},\plc{] clause] ... ] new-line}
    \plc{for-loops}
\end{ompcPragma}

where \plc{clause} can be any of the clauses accepted by the \code{teams} 
or \code{distribute} directives with identical meanings and restrictions.
\end{ccppspecific}

\begin{fortranspecific}
The syntax of the \code{teams}~\code{distribute} construct is as follows:

\begin{ompfPragma}
!$omp teams distribute \plc{[clause[ [},\plc{] clause] ... ]}
    \plc{do-loops}
\plc{[}!$omp end teams distribute\plc{]}
\end{ompfPragma}

where \plc{clause} can be any of the clauses accepted by the \code{teams} 
or \code{distribute} directives with identical meanings and restrictions.

If an \code{end}~\code{teams}~\code{distribute} directive is not specified, an
\code{end}~\code{teams}~\code{distribute} directive is assumed at the end of 
the \plc{do-loops}.
\end{fortranspecific}

\descr
The semantics are identical to explicitly specifying a \code{teams} directive 
immediately followed by a \code{distribute} directive.

\restrictions
The restrictions for the \code{teams} and \code{distribute} constructs apply.

\crossreferences
\begin{itemize}
\item \code{teams} construct, see
\specref{sec:teams Construct}.

\item \code{distribute} construct, see
\specref{subsec:distribute Construct}.

\item Data attribute clauses, see
\specref{subsec:Data-Sharing Attribute Clauses}.
\end{itemize}



\subsection{\hcode{teams}~\hcode{distribute}~\hcode{simd} Construct}
\label{subsec:teams distribute simd Construct}
\index{teams distribute simd@{\code{teams}~\code{distribute}~\code{simd}}}
\index{constructs!teams distribute simd@{\code{teams}~\code{distribute}~\code{simd}}}
\index{combined constructs!teams distribute simd@{\code{teams}~\code{distribute}~\code{simd}}}
\summary
The \code{teams}~\code{distribute}~\code{simd} construct is a shortcut for 
specifying a \code{teams} construct containing a \code{distribute}~\code{simd} 
construct and no other statements.

\syntax
\begin{ccppspecific}
The syntax of the \code{teams}~\code{distribute}~\code{simd} construct is as follows:

\begin{ompcPragma}
#pragma omp teams distribute simd \plc{[clause[ [},\plc{] clause] ... ] new-line}
    \plc{for-loops}
\end{ompcPragma}

where \plc{clause} can be any of the clauses accepted by the \code{teams} or 
\code{distribute}~\code{simd} directives with identical meanings and restrictions.
\end{ccppspecific}

\begin{fortranspecific}
The syntax of the \code{teams}~\code{distribute}~\code{simd} construct is as follows:

\begin{ompfPragma}
!$omp teams distribute simd \plc{[clause[ [},\plc{] clause] ... ]}
    \plc{do-loops}
\plc{[}!$omp end teams distribute simd\plc{]}
\end{ompfPragma}

where \plc{clause} can be any of the clauses accepted by the \code{teams} or 
\code{distribute}~\code{simd} directives with identical meanings and restrictions.

If an \code{end}~\code{teams}~\code{distribute}~\code{simd} directive is
not specified, an \code{end}~\code{teams} \code{distribute}~\code{simd}
directive is assumed at the end of the \plc{do-loops}.
\end{fortranspecific}

\descr
The semantics are identical to explicitly specifying a \code{teams} directive 
immediately followed by a \code{distribute}~\code{simd} directive.

\restrictions
The restrictions for the \code{teams} and \code{distribute}~\code{simd} 
constructs apply.

\crossreferences
\begin{itemize}
\item \code{teams} construct, see
\specref{sec:teams Construct}.

\item \code{distribute}~\code{simd} construct, see
\specref{subsec:distribute simd Construct}.

\item Data attribute clauses, see
\specref{subsec:Data-Sharing Attribute Clauses}.
\end{itemize}



\subsection{Teams Distribute Parallel Worksharing-Loop Construct}
\label{subsec:Teams Distribute Parallel Worksharing-Loop Construct}
\index{teams distribute parallel worksharing-loop construct}
\index{constructs!teams distribute parallel worksharing-loop construct}
\index{combined constructs!teams distribute parallel worksharing-loop construct}
\summary
The teams distribute parallel worksharing-loop construct is a shortcut 
for specifying a \code{teams} construct containing a distribute parallel 
worksharing-loop construct and no other statements.

\syntax
\begin{ccppspecific}
The syntax of the teams distribute parallel worksharing-loop construct is as follows:

\begin{ompcPragma}
#pragma omp teams distribute parallel for \
            \plc{[clause[ [},\plc{] clause] ...  ] new-line}
    \plc{for-loops}
\end{ompcPragma}

where \plc{clause} can be any of the clauses accepted by the \code{teams} or
\code{distribute}~\code{parallel}~\code{for} directives with identical meanings 
and restrictions.
\end{ccppspecific}

\begin{fortranspecific}
The syntax of the teams distribute parallel worksharing-loop construct is as follows:

\begin{ompfPragma}
!$omp teams distribute parallel do \plc{[clause[ [},\plc{] clause] ... ]}
   \plc{do-loops}
\plc{[} !$omp end teams distribute parallel do \plc{]}
\end{ompfPragma}

where \plc{clause} can be any of the clauses accepted by the \code{teams} or
\code{distribute}~\code{parallel}~\code{do} directives with identical meanings 
and restrictions.

If an \code{end}~\code{teams}~\code{distribute}~\code{parallel}~\code{do} directive 
is not specified, an \code{end}~\code{teams} \code{distribute}
\code{parallel}~\code{do} directive is assumed at the end of the \plc{do-loops}.
\end{fortranspecific}

\descr
The semantics are identical to explicitly specifying a \code{teams} directive 
immediately followed by a distribute parallel worksharing-loop directive.

\restrictions
The restrictions for the \code{teams} and distribute parallel worksharing-loop 
constructs apply.

\crossreferences
\begin{itemize}
\item \code{teams} construct, see
\specref{sec:teams Construct}.

\item Distribute parallel worksharing-loop construct, see
\specref{subsec:Distribute Parallel Worksharing-Loop Construct}.

\item Data attribute clauses, see
\specref{subsec:Data-Sharing Attribute Clauses}.
\end{itemize}



\subsection{Teams Distribute Parallel Worksharing-Loop SIMD Construct}
\label{subsec:Teams Distribute Parallel Worksharing-Loop SIMD Construct}
\index{teams distribute parallel worksharing-loop SIMD construct}
\index{constructs!teams distribute parallel worksharing-loop SIMD construct}
\index{combined constructs!teams distribute parallel worksharing-loop SIMD construct}
\summary
The teams distribute parallel worksharing-loop SIMD construct is a shortcut 
for specifying a \code{teams} construct containing a distribute parallel 
worksharing-loop SIMD construct and no other statements.

\syntax
\begin{ccppspecific}
The syntax of the teams distribute parallel worksharing-loop SIMD construct 
is as follows:

\begin{ompcPragma}
#pragma omp teams distribute parallel for simd \
            \plc{[clause[ [},\plc{] clause] ... ] new-line}
    \plc{for-loops}
\end{ompcPragma}

where \plc{clause} can be any of the clauses accepted by the \code{teams} or
\code{distribute}~\code{parallel} \code{for}~\code{simd} directives with 
identical meanings and restrictions.
\end{ccppspecific}

\begin{fortranspecific}
The syntax of the teams distribute parallel worksharing-loop SIMD construct 
is as follows:

\begin{ompfPragma}
!$omp teams distribute parallel do simd \plc{[clause[ [},\plc{] clause] ... ]}
    \plc{do-loops}
\plc{[}!$omp end teams distribute parallel do simd\plc{]}
\end{ompfPragma}

where \plc{clause} can be any of the clauses accepted by the \code{teams} or
\code{distribute} \code{parallel} \code{do}~\code{simd} directives with 
identical meanings and restrictions.

If an \code{end}~\code{teams} \code{distribute} \code{parallel} \code{do}~\code{simd} 
directive is not specified, an \code{end}~\code{teams} \code{distribute} 
\code{parallel} \code{do}~\code{simd} directive is assumed at the end of
the \plc{do-loops}.
\end{fortranspecific}

\descr
The semantics are identical to explicitly specifying a \code{teams} directive 
immediately followed by a distribute parallel worksharing-loop SIMD directive. 

\restrictions
The restrictions for the \code{teams} and distribute parallel worksharing-loop
SIMD constructs apply.

\crossreferences
\begin{itemize}
\item \code{teams} construct, see
\specref{sec:teams Construct}.

\item Distribute parallel worksharing-loop SIMD construct, see
\specref{subsec:Distribute Parallel Worksharing-Loop SIMD Construct}.

\item Data attribute clauses, see
\specref{subsec:Data-Sharing Attribute Clauses}.
\end{itemize}



\subsection{\hcode{teams}~\hcode{loop} Construct}
\label{subsec:teams loop Construct}
\index{teams loop@{\code{teams}~\code{loop}}}
\index{constructs!teams loop@{\code{teams}~\code{loop}}}
\index{combined constructs!teams loop@{\code{teams}~\code{loop}}}
\summary
The \code{teams}~\code{loop} construct is a shortcut for specifying a \code{teams} 
construct containing a \code{loop} construct and no other statements.

\syntax
\begin{ccppspecific}
The syntax of the \code{teams}~\code{loop} construct is as follows:

\begin{ompcPragma}
#pragma omp teams loop \plc{[clause[ [},\plc{] clause] ... ] new-line}
    \plc{for-loops}
\end{ompcPragma}

where \plc{clause} can be any of the clauses accepted by the \code{teams} or
\code{loop} directives with identical meanings and restrictions.
\end{ccppspecific}

\begin{fortranspecific}
The syntax of the \code{teams}~\code{loop} construct is as follows:

\begin{ompfPragma}
!$omp teams loop \plc{[clause[ [},\plc{] clause] ... ]}
    \plc{do-loops}
\plc{[}!$omp end teams loop\plc{]}
\end{ompfPragma}

where \plc{clause} can be any of the clauses accepted by the \code{teams} or
\code{loop} directives with identical meanings and restrictions.

If an \code{end}~\code{teams}~\code{loop} directive is not specified, an
\code{end}~\code{teams}~\code{loop} directive is assumed at the end of the 
\plc{do-loops}.
\end{fortranspecific}

\descr
The semantics are identical to explicitly specifying a \code{teams} directive 
immediately followed by a \code{loop} directive.

\restrictions
The restrictions for the \code{teams} and \code{loop} constructs apply.

\crossreferences
\begin{itemize}
\item \code{teams} construct, see
\specref{sec:teams Construct}.

\item \code{loop} construct, see
\specref{subsec:loop Construct}.

\item Data attribute clauses, see
\specref{subsec:Data-Sharing Attribute Clauses}.
\end{itemize}



\subsection{\hcode{target}~\hcode{parallel} Construct}
\label{subsec:target parallel Construct}
\index{target parallel@{\code{target}~\code{parallel}}}
\index{constructs!target parallel@{\code{target}~\code{parallel}}}
\index{combined constructs!target parallel@{\code{target}~\code{parallel}}}
\summary
The \code{target} \code{parallel} construct is a shortcut for specifying a 
\code{target} construct containing a \code{parallel} construct and no other statements.

\syntax
\begin{ccppspecific}
The syntax of the \code{target} \code{parallel} construct is as follows:

\begin{ompcPragma}
#pragma omp target parallel \plc{[clause[ [},\plc{] clause] ... ] new-line}
    \plc{structured-block}
\end{ompcPragma}

where \plc{clause} can be any of the clauses accepted by the \code{target} or
\code{parallel} directives, except for \code{copyin}, with identical meanings 
and restrictions.
\end{ccppspecific}

\begin{samepage}
\begin{fortranspecific}
The syntax of the \code{target} \code{parallel} construct is as follows:

\begin{ompfPragma}
!$omp target parallel \plc{[clause[ [},\plc{] clause] ... ]}
    \plc{structured-block}
!$omp end target parallel
\end{ompfPragma}

where \plc{clause} can be any of the clauses accepted by the \code{target} or
\code{parallel} directives, except for \code{copyin}, with identical meanings 
and restrictions.
\end{fortranspecific}
\end{samepage}

\descr
The semantics are identical to explicitly specifying a \code{target} directive
immediately followed by a \code{parallel} directive.

\restrictions

The restrictions for the \code{target} and \code{parallel} constructs apply 
except for the following explicit modifications:

\begin{itemize}
\item If any \code{if} clause on the directive includes a
      \plc{directive-name-modifier} then all \code{if} clauses
      on the directive must include a \plc{directive-name-modifier}.
\item At most one \code{if} clause without a
      \plc{directive-name-modifier} can appear on the directive.
\item At most one \code{if} clause with the \code{parallel}
      \plc{directive-name-modifier} can appear on the directive.
\item At most one \code{if} clause with the \code{target}
      \plc{directive-name-modifier} can appear on the directive.
\end{itemize}

\crossreferences
\begin{itemize}
\item \code{parallel} construct, see
\specref{sec:parallel Construct}.

\item \code{target} construct, see
\specref{subsec:target Construct}.

\item \code{if} Clause, see \specref{sec:if Clause}.

\item Data attribute clauses, see
\specref{subsec:Data-Sharing Attribute Clauses}.
\end{itemize}



\subsection{Target Parallel Worksharing-Loop Construct}
\label{subsec:Target Parallel Worksharing-Loop Construct}
\index{target parallel worksharing-loop construct}
\index{constructs!target parallel worksharing-loop}
\index{constructs!target parallel for@{\code{target}~\code{parallel}~\code{for}}}
\index{constructs!target parallel do@{\code{target}~\code{parallel}~\code{do}}}
\index{combined constructs!target parallel worksharing-loop}
\summary
The target parallel worksharing-loop construct is a shortcut for specifying 
a \code{target} construct containing a parallel worksharing-loop construct 
and no other statements.

\syntax
\begin{ccppspecific}
The syntax of the target parallel worksharing-loop construct is as follows:

\begin{ompcPragma}
#pragma omp target parallel for \plc{[clause[ [},\plc{] clause] ... ] new-line}
    \plc{for-loops}
\end{ompcPragma}

where \plc{clause} can be any of the clauses accepted by the \code{target} or
\code{parallel}~\code{for} directives, except for \code{copyin}, with identical 
meanings and restrictions.
\end{ccppspecific}

\needspace{6\baselineskip}
\begin{fortranspecific}
The syntax of the target parallel worksharing-loop construct is as follows:

\begin{ompfPragma}
!$omp target parallel do \plc{[clause[ [},\plc{] clause] ... ]}
    \plc{do-loops}
\plc{[}!$omp end target parallel do\plc{]}
\end{ompfPragma}

where \plc{clause} can be any of the clauses accepted by the \code{target} or
\code{parallel}~\code{do} directives, except for \code{copyin}, with identical 
meanings and restrictions.

If an \code{end}~\code{target}~\code{parallel}~\code{do} directive is not 
specified, an \code{end}~\code{target}~\code{parallel}~\code{do} directive 
is assumed at the end of the \plc{do-loops}.
\end{fortranspecific}

\descr
The semantics are identical to explicitly specifying a \code{target} directive
immediately followed by a parallel worksharing-loop directive.


\restrictions
The restrictions for the \code{target} and parallel worksharing-loop constructs 
apply except for the following explicit modifications:

\begin{itemize}
\item If any \code{if} clause on the directive includes a
      \plc{directive-name-modifier} then all \code{if} clauses
      on the directive must include a \plc{directive-name-modifier}.
\item At most one \code{if} clause without a
      \plc{directive-name-modifier} can appear on the directive.
\item At most one \code{if} clause with the \code{parallel}
      \plc{directive-name-modifier} can appear on the directive.
\item At most one \code{if} clause with the \code{target}
      \plc{directive-name-modifier} can appear on the directive.
\end{itemize}

\crossreferences
\begin{itemize}
\item \code{target} construct, see
\specref{subsec:target Construct}.

\item Parallel Worksharing-Loop construct, see
\specref{subsec:Parallel Worksharing-Loop Construct}.

\item \code{if} Clause, see \specref{sec:if Clause}.

\item Data attribute clauses, see
\specref{subsec:Data-Sharing Attribute Clauses}.
\end{itemize}



\subsection{Target Parallel Worksharing-Loop SIMD Construct}
\label{subsec:Target Parallel Worksharing-Loop SIMD Construct}
\index{target parallel worksharing-loop SIMD construct}
\index{constructs!target parallel worksharing-loop SIMD}
\index{constructs!target parallel for simd@{\code{target}~\code{parallel}~\code{for}~\code{simd}}}
\index{constructs!target parallel do simd@{\code{target}~\code{parallel}~\code{do}~\code{simd}}}
\index{combined constructs!target parallel worksharing-loop SIMD}
\summary
The target parallel worksharing-loop SIMD construct is a shortcut for specifying 
a \code{target} construct containing a parallel worksharing-loop SIMD construct 
and no other statements.

\syntax
\begin{ccppspecific}
The syntax of the target parallel worksharing-loop SIMD construct is as follows:

\begin{ompcPragma}
#pragma omp target parallel for simd \
            \plc{[clause[[},\plc{] clause] ... ] new-line}
    \plc{for-loops}
\end{ompcPragma}

where \plc{clause} can be any of the clauses accepted by the \code{target} or
\code{parallel}~\code{for}~\code{simd} directives, except for \code{copyin}, 
with identical meanings and restrictions.
\end{ccppspecific}

\needspace{6\baselineskip}
\begin{fortranspecific}
The syntax of the target parallel worksharing-loop SIMD construct is as follows:

\begin{ompfPragma}
!$omp target parallel do simd \plc{[clause[ [},\plc{] clause] ... ]}
    \plc{do-loops}
\plc{[}!$omp end target parallel do simd\plc{]}
\end{ompfPragma}

where \plc{clause} can be any of the clauses accepted by the \code{target} or
\code{parallel}~\code{do}~\code{simd} directives, except for \code{copyin}, 
with identical meanings and restrictions.

If an \code{end}~\code{target}~\code{parallel}~\code{do}~\code{simd} directive 
is not specified, an \code{end}~\code{target} \code{parallel} \code{do}~\code{simd} 
directive is assumed at the end of the \plc{do-loops}.
\end{fortranspecific}

\descr
The semantics are identical to explicitly specifying a \code{target} directive
immediately followed by a parallel worksharing-loop SIMD directive.


\restrictions
The restrictions for the \code{target} and parallel worksharing-loop SIMD 
constructs apply except for the following explicit modifications:

\begin{itemize}
\item If any \code{if} clause on the directive includes a
      \plc{directive-name-modifier} then all \code{if} clauses
      on the directive must include a \plc{directive-name-modifier}.
\item At most one \code{if} clause without a
      \plc{directive-name-modifier} can appear on the directive.
\item At most one \code{if} clause with the \code{parallel}
      \plc{directive-name-modifier} can appear on the directive.
\item At most one \code{if} clause with the \code{target}
      \plc{directive-name-modifier} can appear on the directive.
\end{itemize}

\crossreferences
\begin{itemize}
\item \code{target} construct, see
\specref{subsec:target Construct}.

\item Parallel worksharing-loop SIMD construct, see
\specref{subsec:Parallel Worksharing-Loop SIMD Construct}.

\item \code{if} Clause, see \specref{sec:if Clause}.

\item Data attribute clauses, see
\specref{subsec:Data-Sharing Attribute Clauses}.
\end{itemize}



\subsection{\hcode{target} \hcode{parallel} \hcode{loop} Construct}
\label{subsec:target parallel loop Construct}
\index{target parallel loop@{\code{target}~\code{parallel}~\code{loop}}}
\index{constructs!target parallel loop@{\code{target}~\code{parallel}~\code{loop}}}
\index{combined constructs!target parallel loop@{\code{target}~\code{parallel}~\code{loop}}}
\summary
The \code{target}~\code{parallel}~\code{loop} construct is a shortcut for 
specifying a \code{target} construct containing a \code{parallel}~\code{loop} 
construct and no other statements.

\syntax
\begin{ccppspecific}
The syntax of the \code{target}~\code{parallel}~\code{loop} construct is as follows:

\begin{ompcPragma}
#pragma omp target parallel loop \plc{[clause[ [},\plc{] clause] ... ] new-line}
    \plc{for-loops}
\end{ompcPragma}

where \plc{clause} can be any of the clauses accepted by the \code{target} 
or \code{parallel}~\code{loop} directives with identical meanings and restrictions.
\end{ccppspecific}

\begin{fortranspecific}
The syntax of the \code{target}~\code{parallel}~\code{loop} construct is as follows:

\begin{ompfPragma}
!$omp target parallel loop \plc{[clause[ [},\plc{] clause] ... ]}
    \plc{do-loops}
\plc{[}!$omp end target parallel loop\plc{]}
\end{ompfPragma}

where \plc{clause} can be any of the clauses accepted by the \code{teams} 
or \code{parallel}~\code{loop} directives with identical meanings and restrictions.

If an \code{end}~\code{target} \code{parallel}~\code{loop} directive is not 
specified, an \code{end}~\code{target} \code{parallel}~\code{loop} directive 
is assumed at the end of the \plc{do-loops}. \code{nowait} may not be specified 
on an \code{end}~\code{target}~\code{parallel}~\code{loop} directive.
\end{fortranspecific}

\descr
The semantics are identical to explicitly specifying a \code{target} directive 
immediately followed by a \code{parallel}~\code{loop} directive.

\restrictions
The restrictions for the \code{target} and \code{parallel}~\code{loop} constructs 
apply.

\crossreferences
\begin{itemize}
\item \code{target} construct, see \specref{subsec:target Construct}.

\item \code{parallel}~\code{loop} construct, see
\specref{subsec:parallel loop Construct}.

\item Data attribute clauses, see \specref{subsec:Data-Sharing Attribute Clauses}.
\end{itemize}



\subsection{\hcode{target}~\hcode{simd} Construct}
\label{subsec:target simd Construct}
\index{target simd@{\code{target}~\code{simd}}}
\index{constructs!target simd@{\code{target}~\code{simd}}}
\index{combined constructs!target simd@{\code{target}~\code{simd}}}

\summary
The \code{target} \code{simd} construct is a shortcut for specifying a \code{target}
construct containing a \code{simd} construct and no other statements.

\syntax
\begin{ccppspecific}
The syntax of the \code{target} \code{simd} construct is as follows:

\begin{ompcPragma}
#pragma omp target simd \plc{[clause[ [},\plc{] clause] ... ] new-line}
    \plc{for-loops}
\end{ompcPragma}

where \plc{clause} can be any of the clauses accepted by the \code{target} or
\code{simd} directives with identical meanings and restrictions.

\end{ccppspecific}

\needspace{6\baselineskip}
\begin{fortranspecific}
The syntax of the \code{target} \code{simd} construct is as follows:

\begin{ompfPragma}
!$omp target simd \plc{[clause[ [},\plc{] clause] ... ]}
    \plc{do-loops}
\plc{[}!$omp end target simd\plc{]}
\end{ompfPragma}

where \plc{clause} can be any of the clauses accepted by the \code{target} or
\code{simd} directives with identical meanings and restrictions.

If an \code{end}~\code{target}~\code{simd} directive is not specified, an
\code{end}~\code{target}~\code{simd} directive is assumed at the end of
the \plc{do-loops}.
\end{fortranspecific}

\descr
The semantics are identical to explicitly specifying a \code{target} directive
immediately followed by a \code{simd} directive.

\restrictions

The restrictions for the \code{target} and \code{simd} constructs apply.

\crossreferences
\begin{itemize}
\item \code{simd} construct, see
\specref{subsubsec:simd Construct}.

\item \code{target} construct, see
\specref{subsec:target Construct}.

\item Data attribute clauses, see
\specref{subsec:Data-Sharing Attribute Clauses}.
\end{itemize}



\subsection{\hcode{target}~\hcode{teams} Construct}
\label{subsec:target teams Construct}
\index{target teams@{\code{target}~\code{teams}}}
\index{constructs!target teams@{\code{target}~\code{teams}}}
\index{combined constructs!target teams@{\code{target}~\code{teams}}}
\summary
The \code{target}~\code{teams} construct is a shortcut for specifying 
a \code{target} construct containing a \code{teams} construct and no other statements.

\syntax
\begin{ccppspecific}
The syntax of the \code{target}~\code{teams} construct is as follows:

\begin{ompcPragma}
#pragma omp target teams \plc{[clause[ [},\plc{] clause] ... ] new-line}
   \plc{structured-block}
\end{ompcPragma}

where \plc{clause} can be any of the clauses accepted by the \code{target} 
or \code{teams} directives with identical meanings and restrictions.
\end{ccppspecific}

\begin{fortranspecific}
The syntax of the \code{target}~\code{teams} construct is as follows:

\begin{ompfPragma}
!$omp target teams \plc{[clause[ [},\plc{] clause] ... ]}
    \plc{structured-block}
!$omp end target teams
\end{ompfPragma}

where \plc{clause} can be any of the clauses accepted by the \code{target} 
or \code{teams} directives with identical meanings and restrictions.
\end{fortranspecific}

\descr

The semantics are identical to explicitly specifying a \code{target} directive
immediately followed by a \code{teams} directive.

\restrictions
The restrictions for the \code{target} and \code{teams} constructs apply.

\crossreferences
\begin{itemize}
\item \code{teams} construct, see \specref{sec:teams Construct}.

\item \code{target} construct, see \specref{subsec:target Construct}.

\item Data attribute clauses, see
\specref{subsec:Data-Sharing Attribute Clauses}.
\end{itemize}



\subsection{\hcode{target}~\hcode{teams}~\hcode{distribute} Construct}
\label{subsec:target teams distribute construct}
\index{target teams distribute@{\code{target}~\code{teams}~\code{distribute}}}
\index{constructs!target teams distribute@{\code{target}~\code{teams}~\code{distribute}}}
\index{combined constructs!target teams distribute@{\code{target}~\code{teams}~\code{distribute}}}
\summary
The \code{target}~\code{teams}~\code{distribute} construct is a shortcut for 
specifying a \code{target} construct containing a \code{teams}~\code{distribute} 
construct and no other statements.

\syntax
\begin{ccppspecific}
The syntax of the \code{target}~\code{teams}~\code{distribute} construct is as follows:

\begin{ompcPragma}
#pragma omp target teams distribute \plc{[clause[ [},\plc{] clause] ... ] new-line}
   \plc{for-loops}
\end{ompcPragma}

where \plc{clause} can be any of the clauses accepted by the \code{target} or 
\code{teams}~\code{distribute} directives with identical meanings and restrictions.
\end{ccppspecific}

\begin{fortranspecific}
The syntax of the \code{target}~\code{teams}~\code{distribute} construct is as follows:

\begin{ompfPragma}
!$omp target teams distribute \plc{[clause[ [},\plc{] clause] ... ]}
    \plc{do-loops}
\plc{[}!$omp end target teams distribute\plc{]}
\end{ompfPragma}

where \plc{clause} can be any of the clauses accepted by the \code{target} 
or \code{teams}~\code{distribute} directives with identical meanings and restrictions.

If an \code{end}~\code{target}~\code{teams}~\code{distribute} directive is not 
specified, an \code{end}~\code{target} \code{teams} \code{distribute} directive 
is assumed at the end of the \plc{do-loops}.
\end{fortranspecific}

\descr
The semantics are identical to explicitly specifying a \code{target} directive 
immediately followed by a \code{teams}~\code{distribute} directive.

\restrictions
The restrictions for the \code{target} and \code{teams}~\code{distribute} constructs.

\crossreferences
\begin{itemize}
\item \code{target} construct, see
\specref{subsec:target data Construct}.

\item \code{teams}~\code{distribute} construct, see
\specref{subsec:teams distribute Construct}.

\item Data attribute clauses, see
\specref{subsec:Data-Sharing Attribute Clauses}.
\end{itemize}



\subsection{\hcode{target}~\hcode{teams}~\hcode{distribute}~\hcode{simd} Construct}
\label{subsec:target teams distribute simd construct}
\index{target teams distribute simd@{\code{target}~\code{teams}~\code{distribute}~\code{simd}}}
\index{constructs!target teams distribute simd@{\code{target}~\code{teams}~\code{distribute}~\code{simd}}}
\index{combined constructs!target teams distribute simd@{\code{target}~\code{teams}~\code{distribute}~\code{simd}}}
\summary
The \code{target}~\code{teams}~\code{distribute}~\code{simd} construct is a 
shortcut for specifying a \code{target} construct containing a 
\code{teams}~\code{distribute}~\code{simd} construct and no other statements.

\syntax
\begin{ccppspecific}
The syntax of the \code{target}~\code{teams}~\code{distribute}~\code{simd} 
construct is as follows:

\begin{ompcPragma}
#pragma omp target teams distribute simd \
            \plc{[clause[ [},\plc{] clause] ...  ] new-line}
   \plc{for-loops}
\end{ompcPragma}

where \plc{clause} can be any of the clauses accepted by the \code{target} or
\code{teams}~\code{distribute}~\code{simd} directives with identical meanings 
and restrictions.
\end{ccppspecific}

\begin{fortranspecific}
The syntax of the \code{target}~\code{teams}~\code{distribute}~\code{simd} 
construct is as follows:

\begin{ompfPragma}
!$omp target teams distribute simd \plc{[clause[ [},\plc{] clause] ... ]}
    \plc{do-loops}
\plc{[}!$omp end target teams distribute simd\plc{]}
\end{ompfPragma}

where \plc{clause} can be any of the clauses accepted by the \code{target} or
\code{teams}~\code{distribute}~\code{simd} directives with identical meanings 
and restrictions.

If an \code{end}~\code{target}~\code{teams}~\code{distribute}~\code{simd} 
directive is not specified, an \code{end}~\code{target} \code{teams} 
\code{distribute}~\code{simd} directive is assumed at the end of the \plc{do-loops}.
\end{fortranspecific}

\descr
The semantics are identical to explicitly specifying a \code{target} directive 
immediately followed by a \code{teams}~\code{distribute}~\code{simd} directive.

\restrictions
The restrictions for the \code{target} and \code{teams}~\code{distribute}~\code{simd} 
constructs apply.

\crossreferences
\begin{itemize}
\item \code{target} construct, see
\specref{subsec:target data Construct}.

\item \code{teams}~\code{distribute}~\code{simd} construct, see
\specref{subsec:teams distribute simd Construct}.

\item Data attribute clauses, see
\specref{subsec:Data-Sharing Attribute Clauses}.
\end{itemize}



\subsection{Teams Distribute Parallel Worksharing-Loop Construct}
\label{subsec:Teams Distribute Parallel Worksharing-Loop Construct}
\index{teams distribute parallel worksharing-loop construct}
\index{constructs!teams distribute parallel worksharing-loop construct}
\index{combined constructs!teams distribute parallel worksharing-loop construct}
\summary
The teams distribute parallel worksharing-loop construct is a shortcut for 
specifying a \code{teams} construct containing a distribute parallel worksharing-loop 
construct and no other statements.

\syntax
\begin{ccppspecific}
The syntax of the teams distribute parallel worksharing-loop construct is as follows:

\begin{ompcPragma}[fontsize=\small]
#pragma omp teams distribute parallel for \
            \plc{[clause[ [},\plc{] clause] ...  ] new-line}
    \plc{for-loops}
\end{ompcPragma}

where \plc{clause} can be any of the clauses accepted by the \code{teams} or
\code{distribute}~\code{parallel}~\code{for} directives with identical meanings 
and restrictions.
\end{ccppspecific}

\begin{fortranspecific}
The syntax of the teams distribute parallel worksharing-loop construct is as follows:

\begin{ompfPragma}
!$omp teams distribute parallel do \plc{[clause[ [},\plc{] clause] ... ]}
   \plc{do-loops}
\plc{[} !$omp end teams distribute parallel do \plc{]}
\end{ompfPragma}

where \plc{clause} can be any of the clauses accepted by the \code{teams} or
\code{distribute}~\code{parallel}~\code{do} directives with identical meanings 
and restrictions.

If an \code{end}~\code{teams}~\code{distribute}~\code{parallel}~\code{do} 
directive is not specified, an \code{end}~\code{teams} \code{distribute} 
\code{parallel}~\code{do} directive is assumed at the end of the \plc{do-loops}.
\end{fortranspecific}

\descr
The semantics are identical to explicitly specifying a \code{teams} directive 
immediately followed by a distribute parallel worksharing-loop directive.

\restrictions
The restrictions for the \code{teams} and distribute parallel worksharing-loop 
constructs apply.

\crossreferences
\begin{itemize}
\item \code{teams} construct, see
\specref{sec:teams Construct}.

\item Distribute parallel worksharing-loop construct, see
\specref{subsec:Distribute Parallel Worksharing-Loop Construct}.

\item Data attribute clauses, see
\specref{subsec:Data-Sharing Attribute Clauses}.
\end{itemize}



\subsection{\hcode{target}~\code{teams}~\code{loop} Construct}
\label{subsec:target teams loop}
\index{target teams loop@{\code{target}~\code{teams}~\code{loop}}}
\index{constructs!target teams loop construct@{\code{target}~\code{teams}~\code{loop}}}
\index{combined constructs!target teams loop construct}
\summary
The \code{target}~\code{teams}~\code{loop} construct is a shortcut for specifying a 
\code{target} construct containing a \code{teams}~\code{loop} construct and no other 
statements.

\syntax
\begin{ccppspecific}
The syntax of the \code{target}~\code{teams}~\code{loop} construct is as follows:

\begin{ompcPragma}
#pragma omp target teams loop \plc{[clause[ [},\plc{] clause] ... ] new-line}
    \plc{for-loops}
\end{ompcPragma}

where \plc{clause} can be any of the clauses accepted by the \code{target} or
\code{teams}~\code{loop} directives with identical meanings and restrictions.
\end{ccppspecific}

\needspace{6\baselineskip}
\begin{fortranspecific}
The syntax of the \code{target}~\code{teams}~\code{loop} construct is as follows:

\begin{ompfPragma}
!$omp target teams loop \plc{[clause[ [},\plc{] clause] ... ]}
    \plc{do-loops}
\plc{[}!$omp end target teams loop\plc{]}
\end{ompfPragma}

where \plc{clause} can be any of the clauses accepted by the \code{target} or
\code{teams}~\code{loop} directives with identical meanings and restrictions.

If an \code{end}~\code{target}~\code{teams}~\code{loop} directive is not 
specified, an \code{end}~\code{target}~\code{teams}~\code{loop} directive 
is assumed at the end of the \plc{do-loops}.
\end{fortranspecific}

\descr
The semantics are identical to explicitly specifying a \code{target}
directive immediately followed by a \code{teams}~\code{loop} directive.

\restrictions
The restrictions for the \code{target} and \code{teams}~\code{loop} constructs.

\crossreferences
\begin{itemize}
\item \code{target} construct, see \specref{subsec:target Construct}.

\item Teams loop construct, see \specref{subsec:teams loop Construct}.

\item Data attribute clauses, see \specref{subsec:Data-Sharing Attribute Clauses}.
\end{itemize}



\subsection{Target Teams Distribute Parallel Worksharing-Loop Construct}
\label{subsec:Target Teams Distribute Parallel Worksharing-Loop Construct}
\index{target teams distribute parallel worksharing-loop construct}
\index{constructs!target teams distribute parallel worksharing-loop construct}
\index{combined constructs!target teams distribute parallel worksharing-loop construct}
\summary
The target teams distribute parallel worksharing-loop construct is a shortcut 
for specifying a \code{target} construct containing a teams distribute parallel 
worksharing-loop construct and no other statements.

\syntax
\begin{ccppspecific}
The syntax of the target teams distribute parallel worksharing-loop construct is 
as follows:

\begin{ompcPragma}[fontsize=\small]
#pragma omp target teams distribute parallel for \
            \plc{[clause[ [},\plc{] clause] ... ] new-line}
    \plc{for-loops}
\end{ompcPragma}

where \plc{clause} can be any of the clauses accepted by the \code{target} or
\code{teams}~\code{distribute} \code{parallel}~\code{for} directives with identical
meanings and restrictions.
\end{ccppspecific}

\needspace{6\baselineskip}
\begin{fortranspecific}
The syntax of the target teams distribute parallel worksharing-loop construct 
is as follows:

\begin{ompfPragma}
!$omp target teams distribute parallel do \plc{[clause[ [},\plc{] clause] ... ]}
    \plc{do-loops}
\plc{[}!$omp end target teams distribute parallel do\plc{]}
\end{ompfPragma}

where \plc{clause} can be any of the clauses accepted by the \code{target} or
\code{teams}~\code{distribute} \code{parallel}~\code{do} directives with
identical meanings and restrictions.

If an \code{end}~\code{target} \code{teams} \code{distribute} 
\code{parallel}~\code{do} directive is not specified, an \code{end}~\code{target}
\code{teams} \code{distribute} \code{parallel}~\code{do} directive is assumed at 
the end of the \plc{do-loops}.
\end{fortranspecific}

\descr
The semantics are identical to explicitly specifying a \code{target}
directive immediately followed by a teams distribute parallel worksharing-loop 
directive.

\restrictions
The restrictions for the \code{target} and teams distribute parallel
worksharing-loop constructs apply except for the following explicit modifications:

\begin{itemize}
\item If any \code{if} clause on the directive includes a
      \plc{directive-name-modifier} then all \code{if} clauses
      on the directive must include a \plc{directive-name-modifier}.
\item At most one \code{if} clause without a
      \plc{directive-name-modifier} can appear on the directive.
\item At most one \code{if} clause with the \code{parallel}
      \plc{directive-name-modifier} can appear on the directive.
\item At most one \code{if} clause with the \code{target}
      \plc{directive-name-modifier} can appear on the directive.
\end{itemize}

\crossreferences
\begin{itemize}
\item \code{target} construct, see \specref{subsec:target Construct}.

\item Teams distribute parallel worksharing-loop construct, see
\specref{subsec:Teams Distribute Parallel Worksharing-Loop Construct}.

\item \code{if} Clause, see \specref{sec:if Clause}.

\item Data attribute clauses, see
\specref{subsec:Data-Sharing Attribute Clauses}.
\end{itemize}










\subsection{Teams Distribute Parallel Worksharing-Loop SIMD Construct}
\label{subsec:Teams Distribute Parallel Worksharing-Loop SIMD Construct}
\index{teams distribute parallel worksharing-loop SIMD construct}
\index{constructs!teams distribute parallel worksharing-loop SIMD construct}
\index{combined constructs!teams distribute parallel worksharing-loop SIMD construct}
\summary
The teams distribute parallel worksharing-loop SIMD construct is a shortcut for specifying a \code{teams}
construct containing a distribute parallel worksharing-loop SIMD construct and no other statements.


\syntax
\begin{ccppspecific}
The syntax of the teams distribute parallel worksharing-loop construct is as follows:

\begin{ompcPragma}[fontsize=\small]
#pragma omp teams distribute parallel for simd \
            \plc{[clause[ [},\plc{] clause] ... ] new-line}
    \plc{for-loops}
\end{ompcPragma}

where \plc{clause} can be any of the clauses accepted by the \code{teams} or
\code{distribute}~\code{parallel}~\code{for}~\code{simd}
directives with identical meanings and restrictions.
\end{ccppspecific}

\begin{fortranspecific}
The syntax of the teams distribute parallel worksharing-loop construct is as follows:

\begin{ompfPragma}
!$omp teams distribute parallel do simd \plc{[clause[ [},\plc{] clause] ... ]}
    \plc{do-loops}
\plc{[}!$omp end teams distribute parallel do simd\plc{]}
\end{ompfPragma}

where \plc{clause} can be any of the clauses accepted by the \code{teams} or
\code{distribute}~\code{parallel}~\code{do}~\code{simd}
directives with identical meanings and restrictions.

If an \code{end}~\code{teams}~\code{distribute}~\code{parallel}~\code{do}~\code{simd} directive is not specified, an
\code{end}~\code{teams}~\code{distribute}~\code{parallel}~\code{do}~\code{simd} directive is assumed at the end of
the \plc{do-loops}.
\end{fortranspecific}

\descr
The semantics are identical to explicitly specifying a \code{teams} directive immediately
followed by a distribute parallel worksharing-loop SIMD directive. 

\restrictions
The restrictions for the \code{teams} and distribute parallel worksharing-loop
SIMD constructs apply.

\crossreferences
\begin{itemize}
\item \code{teams} construct, see
\specref{sec:teams Construct}.

\item Distribute parallel worksharing-loop SIMD construct, see
\specref{subsec:Distribute Parallel Worksharing-Loop SIMD Construct}.

\item Data attribute clauses, see
\specref{subsec:Data-Sharing Attribute Clauses}.
\end{itemize}



\subsection{Target Teams Distribute Parallel Worksharing-Loop SIMD Construct}
\label{subsec:Target Teams Distribute Parallel Loop SIMD Construct}
\index{target teams distribute parallel worksharing-loop SIMD construct}
\index{constructs!target teams distribute parallel worksharing-loop SIMD construct}
\index{combined constructs!target teams distribute parallel worksharing-loop SIMD construct}
\summary
The target teams distribute parallel worksharing-loop SIMD construct is a shortcut 
for specifying a \code{target} construct containing a teams distribute parallel 
worksharing-loop SIMD construct and no other statements.

\syntax
\begin{ccppspecific}
The syntax of the target teams distribute parallel worksharing-loop SIMD construct 
is as follows:

\begin{ompcPragma}
#pragma omp target teams distribute parallel for simd \
            \plc{[clause[ [},\plc{] clause] ... ] new-line}
    \plc{for-loops}
\end{ompcPragma}

where \plc{clause} can be any of the clauses accepted by the \code{target} or
\code{teams}~\code{distribute} \code{parallel} \code{for}~\code{simd}
directives with identical meanings and restrictions.
\end{ccppspecific}

\begin{fortranspecific}
The syntax of the target teams distribute parallel worksharing-loop SIMD construct 
is as follows:

\begin{ompfPragma}
!$omp target teams distribute parallel do simd \plc{[clause[ [},\plc{] clause] ... ]}
    \plc{do-loops}
\plc{[}!$omp end target teams distribute parallel do simd\plc{]}
\end{ompfPragma}

where \plc{clause} can be any of the clauses accepted by the
\code{target} or \code{teams}~\code{distribute} \code{parallel} \code{do}~\code{simd}
directives with identical meanings and restrictions.

If an \code{end}~\code{target} \code{teams} \code{distribute} \code{parallel}
\code{do}~\code{simd} directive is not specified, an \code{end}~\code{target}
\code{teams} \code{distribute} \code{parallel} \code{do}~\code{simd}
directive is assumed at the end of the \plc{do-loops}.
\end{fortranspecific}

\descr
The semantics are identical to explicitly specifying a \code{target}
directive immediately followed by a teams distribute parallel worksharing-loop
SIMD directive. 


\restrictions
The restrictions for the \code{target} and teams distribute parallel
worksharing-loop SIMD constructs apply except for the following explicit
 modifications:

\begin{itemize}
\item If any \code{if} clause on the directive includes a
      \plc{directive-name-modifier} then all \code{if} clauses
      on the directive must include a \plc{directive-name-modifier}.
\item At most one \code{if} clause without a
      \plc{directive-name-modifier} can appear on the directive.
\item At most one \code{if} clause with the \code{parallel}
      \plc{directive-name-modifier} can appear on the directive.
\item At most one \code{if} clause with the \code{target}
      \plc{directive-name-modifier} can appear on the directive.
\end{itemize}

\crossreferences
\begin{itemize}
\item \code{target} construct, see \specref{subsec:target Construct}.

\item Teams distribute parallel worksharing-loop SIMD construct, see
\specref{subsec:Teams Distribute Parallel Worksharing-Loop SIMD Construct}.

\item \code{if} Clause, see \specref{sec:if Clause}.

\item Data attribute clauses, see
\specref{subsec:Data-Sharing Attribute Clauses}.
\end{itemize}



\section{Clauses on Combined and Composite Constructs}
\label{sec:Clauses on Combined and Composite Constructs}
This section specifies the handling of clauses on combined or composite constructs 
and the handling of implicit clauses from variables with predetermined data sharing 
if they are not predetermined only on a particular construct. Some clauses are 
permitted only on a single construct of the constructs that constitute the 
combined or composite construct, in which case the effect is as if the 
clause is applied to that specific construct. As detailed in this section, other 
clauses have the effect as if they are applied to one or more constituent constructs.

The \code{collapse} clause is applied once to the combined or composite construct.

The effect of the \code{private} clause is as if it is applied only to the innermost 
constituent construct.

The effect of the \code{firstprivate} clause is as if it is applied to one or 
more constructs as follows:

\begin{itemize}
\item To the \code{distribute} construct if it is among the constituent constructs;
\item To the \code{teams} construct if it is among the constituent constructs and 
      the \code{distribute} construct is not;
\item To the worksharing-loop construct if it is among the constituent constructs;
\item To the \code{parallel} construct if it is among the constituent
      constructs and the worksharing-loop construct is not;
\item To the outermost constituent construct if not already applied to it by the 
      above rules and the outermost constituent construct is not a \code{teams} 
      construct, a \code{parallel} construct, or a \code{target} construct; and
\item To the \code{target} construct if it is among the constituent
      constructs and the same list item does not appear in a \code{lastprivate} 
      or \code{map} clause.
\end{itemize}

If the \code{parallel} construct is among the constituent constructs and the
effect is not as if the \code{firstprivate} clause is applied to it by the
above rules, then the effect is as if the \code{shared} clause with the same
list item is applied to the \code{parallel} construct. If the \code{teams} 
construct is among the constituent constructs and the effect is not as if the 
\code{firstprivate} clause is applied to it by the above rules, then the effect 
is as if the \code{shared} clause with the same list item is applied to the 
\code{teams} construct.

The effect of the \code{lastprivate} clause is as if it is applied to one or 
more constructs as follows: 

\begin{itemize}
\item To the worksharing-loop construct if it is among the constituent constructs;
\item To the \code{distribute} construct if it is among the constituent constructs; and
\item To the innermost constituent construct that permits it unless it is a
      worksharing-loop or \code{distribute} construct.
\end{itemize}

If the \code{parallel} construct is among the constituent constructs and the
list item is not also specified in the \code{firstprivate} clause, then the 
effect of the \code{lastprivate} clause is as if the \code{shared} clause 
with the same list item is applied to the \code{parallel} construct. If the 
\code{teams} construct is among the constituent constructs and the list item 
is not also specified in the \code{firstprivate} clause, then the effect of the
\code{lastprivate} clause is as if the \code{shared} clause with the same list 
item is applied to the \code{teams} construct. If the \code{target} construct 
is among the constituent constructs and the list item is not specified in a 
\code{map} clause, the effect of the \code{lastprivate} clause is as if the same 
list item appears in a \code{map} clause with a \plc{map-type} of \code{tofrom}.

The effect of the \code{shared}, \code{default}, \code{order}, or \code{allocate} 
clause is as if it is applied to all constituent constructs that permit the clause.

The effect of the \code{reduction} clause is as if it is applied to all 
constructs that permit the clause, except for the following constructs:

\begin{itemize}
\item The \code{parallel} construct, when combined with the
      worksharing-loop, \code{loop}, or \code{sections} construct; and
\item The \code{teams} construct,  when combined with the \code{loop} construct.
\end{itemize}

For the \code{parallel} and \code{teams} constructs above, the effect of the
\code{reduction} clause instead is as if each list item or, for any list item 
that is an array item, its corresponding base array or base pointer appears 
in a \code{shared} clause for the construct. If the \code{task} 
\plc{reduction-modifier} is specified, the effect is as if it only modifies 
the behavior of the \code{reduction} clause on the innermost construct that 
constitutes the combined construct and that accepts the modifier (see 
\specref{subsubsec:reduction clause}). If the \code{inscan} \plc{reduction-modifier} 
is specified, the effect is as if it modifies the behavior of the \code{reduction} 
clause on all constructs  of the combined construct to which the clause is applied 
and that accept the modifier. If a construct to which the \code{inscan} 
\plc{reduction-modifier} is applied is combined with the \code{target} construct, 
the effect is as if the same list item also appears in a \code{map} clause with a 
\plc{map-type} of \code{tofrom}.

The \code{in_reduction} clause is permitted on a single construct among those that
constitute the combined or composite construct and the effect is as if the clause 
is applied to that construct, but if that construct is a \code{target} construct, 
the effect is also as if the same list item appears in a \code{map} clause with a 
\plc{map-type} of \code{tofrom} and a \plc{map-type-modifier} of \code{always}.

The effect of the \code{if} clause is described in \specref{sec:if Clause}.
 
The effect of the \code{linear} clause is as if it is applied to the innermost
constituent construct. Additionally, if the list item is not the iteration 
variable of a \code{simd} or worksharing-loop SIMD construct, the effect on 
the outer constituent constructs is as if the list item was specified in 
\code{firstprivate} and \code{lastprivate} clauses on the combined or composite 
construct, with the rules specified above applied. If a list item of the 
\code{linear} clause is the iteration variable of a \code{simd} or worksharing-loop 
SIMD construct and it is not declared in the construct, the effect on the outer 
constituent constructs is as if the list item was specified in a \code{lastprivate} 
clause on the combined or composite construct with the rules specified above applied.

The effect of the \code{nowait} clause is as if it is applied to the outermost 
constituent construct that permits it.

If the clauses have expressions on them, such as for various clauses where the 
argument of the clause is an expression, or \plc{lower-bound}, \plc{length}, or 
\plc{stride} expressions inside array sections (or \plc{subscript} and \plc{stride} 
expressions in \plc{subscript-triplet} for Fortran), or \plc{linear-step} or
\plc{alignment} expressions, the expressions are evaluated immediately before 
the construct to which the clause has been split or duplicated per the above 
rules (therefore inside of the outer constituent constructs). However, the 
expressions inside the \code{num_teams} and \code{thread_limit} clauses are 
always evaluated before the outermost constituent construct.

The restriction that a list item may not appear in more than one data
sharing clause with the exception of specifying a variable in both
\code{firstprivate} and \code{lastprivate} clauses applies after the clauses
are split or duplicated per the above rules.
