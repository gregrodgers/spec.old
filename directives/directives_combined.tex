% This is an included file. See the master file for more information.
%
% When editing this file:
%
%    1. To change formatting, appearance, or style, please edit openmp.sty.
%
%    2. Custom commands and macros are defined in openmp.sty.
%
%    3. Be kind to other editors -- keep a consistent style by copying-and-pasting to
%       create new content.
%
%    4. We use semantic markup, e.g. (see openmp.sty for a full list):
%         \code{}     % for bold monospace keywords, code, operators, etc.
%         \plc{}      % for italic placeholder names, grammar, etc.
%
%    5. There are environments that provide special formatting, e.g. language bars.
%       Please use them whereever appropriate.  Examples are:
%
%         \begin{fortranspecific}
%         This is text that appears enclosed in blue language bars for Fortran.
%         \end{fortranspecific}
%
%         \begin{note}
%         This is a note.  The "Note -- " header appears automatically.
%         \end{note}
%
%    6. Other recommendations:
%         Use the convenience macros defined in openmp.sty for the minor headers
%         such as Comments, Syntax, etc.
%
%         To keep items together on the same page, prefer the use of
%         \begin{samepage}.... Avoid \parbox for text blocks as it interrupts line numbering.
%         When possible, avoid \filbreak, \pagebreak, \newpage, \clearpage unless that's
%         what you mean. Use \needspace{} cautiously for troublesome paragraphs.
%
%         Avoid absolute lengths and measures in this file; use relative units when possible.
%         Vertical space can be relative to \baselineskip or ex units. Horizontal space
%         can be relative to \linewidth or em units.
%
%         Prefer \emph{} to italicize terminology, e.g.:
%             This is a \emph{definition}, not a placeholder.
%             This is a \plc{var-name}.
%


\section{Combined Constructs}
\label{sec:Combined Constructs}
\index{combined constructs}
\index{constructs!combined constructs}
Combined constructs are shortcuts for specifying one construct immediately nested
inside another construct. The semantics of the combined constructs are identical to that
of explicitly specifying the first construct containing one instance of the second
construct and no other statements.

Some combined constructs have clauses that are permitted on both 
constructs that were combined. For any particular combined construct, a clause
may explicitly be specified to apply to both constructs or to one specific
construct of those that constitute the combined construct.  Otherwise, the
constructs to which the clause applies is unspecified, except that it will
apply to at least one of them.

If a list item appears in a clause that is permitted by both constructs that
are combined by a combined target construct, the clause will apply to both
constructs for the list item unless the list item also appears in a \code{map}
clause on the construct. If the list item appears in a \code{map} clause, any
other clause in which it appears is not applied to the \code{target} construct
for the list item. Instead, the clause is applied to the other construct that
constitutes the combined target construct.

For combined constructs, tool callbacks shall be invoked as if the constructs were
explicitly nested.






\subsection{Parallel Loop Construct}
\label{subsec:Parallel Loop Construct}
\index{parallel loop construct}
\index{constructs!parallel loop construct}
\index{constructs!parallel for@{\code{parallel}~\code{for} \emph{C/C++}}}
\index{constructs!parallel do@{\code{parallel}~\code{do} \emph{Fortran}}}
\index{combined constructs!parallel loop construct}
\index{worksharing!parallel}
\summary
The parallel loop construct is a shortcut for specifying a \code{parallel} construct
containing one loop constuct with one or more associated loops and no other statements.

%\newpage %% HACK

\syntax
\begin{ccppspecific}
The syntax of the parallel loop construct is as follows:

\begin{ompcPragma}
#pragma omp parallel for \plc{[clause[ [},\plc{] clause] ... ] new-line}
   \plc{for-loops}
\end{ompcPragma}

where \plc{clause} can be any of the clauses accepted by the \code{parallel} or \code{for} directives,
except the \code{nowait} clause, with identical meanings and restrictions.
\end{ccppspecific}

\begin{fortranspecific}
The syntax of the parallel loop construct is as follows:

\begin{ompfPragma}
!$omp parallel do \plc{[clause[ [},\plc{] clause] ... ]}
   \plc{do-loops}
\plc{[}!$omp end parallel do\plc{]}
\end{ompfPragma}

where \plc{clause} can be any of the clauses accepted by the \code{parallel} or \code{do} directives,
with identical meanings and restrictions.

If an \code{end}~\code{parallel}~\code{do} directive is not specified, an \code{end}~\code{parallel}~\code{do} directive is
assumed at the end of the \plc{do-loops}. \code{nowait} may not be specified on an
\code{end}~\code{parallel}~\code{do} directive.
\end{fortranspecific}

\descr
The semantics are identical to explicitly specifying a \code{parallel} directive immediately
followed by a loop directive.

\restrictions
\begin{itemize}
\item The restrictions for the \code{parallel} construct and the loop construct apply.
\end{itemize}

\crossreferences
\begin{itemize}
\item \code{parallel} construct, see
\specref{sec:parallel Construct}.

\item loop SIMD construct, see
\specref{subsec:Loop SIMD Construct}.

\item Data attribute clauses, see
\specref{subsec:Data-Sharing Attribute Clauses}.
\end{itemize}







\subsection{\hcode{parallel}~\hcode{sections} Construct}
\index{parallel sections@{\code{parallel}~\code{sections}}}
\index{constructs!parallel sections@{\code{parallel}~\code{sections}}}
\index{combined constructs!parallel sections@{\code{parallel}~\code{sections}}}
\label{subsec:parallel sections Construct}
\summary
The \code{parallel}~\code{sections} construct is a shortcut for specifying a \code{parallel}
construct containing one \code{sections} construct and no other statements.

\syntax
\begin{ccppspecific}
The syntax of the \code{parallel}~\code{sections} construct is as follows:

\begin{ompcPragma}
#pragma omp parallel sections \plc{[clause[ [},\plc{] clause] ... ] new-line}
    {
    \plc{[}#pragma omp section \plc{new-line]}
        \plc{structured-block}
    \plc{[}#pragma omp section \plc{new-line}
        \plc{structured-block]}
    \plc{...}
    }
\end{ompcPragma}

where \plc{clause} can be any of the clauses accepted by the \code{parallel} or \code{sections}
directives, except the \code{nowait} clause, with identical meanings and restrictions.
\end{ccppspecific}

\begin{fortranspecific}
The syntax of the \code{parallel}~\code{sections} construct is as follows:

\begin{ompfPragma}
!$omp parallel sections \plc{[clause[ [},\plc{] clause] ... ]}
    \plc{[}!$omp section\plc{]}
        \plc{structured-block}
    \plc{[}!$omp section
        \plc{structured-block]}
    \plc{...}
!$omp end parallel sections
\end{ompfPragma}

where \plc{clause} can be any of the clauses accepted by the \code{parallel} or \code{sections}
directives, with identical meanings and restrictions.

The last section ends at the \code{end}~\code{parallel}~\code{sections} directive. \code{nowait} cannot be
specified on an \code{end}~\code{parallel}~\code{sections} directive.
\end{fortranspecific}

\descr
\begin{ccppspecific}
The semantics are identical to explicitly specifying a \code{parallel} directive immediately
followed by a \code{sections} directive.
\end{ccppspecific}

\begin{fortranspecific}
The semantics are identical to explicitly specifying a \code{parallel} directive immediately
followed by a \code{sections} directive, and an \code{end}~\code{sections} directive immediately
followed by an \code{end}~\code{parallel} directive.
\end{fortranspecific}

\restrictions
The restrictions for the \code{parallel} construct and the \code{sections} construct apply.

\crossreferences
\begin{itemize}
\item \code{parallel} construct, see
\specref{sec:parallel Construct}.

\item \code{sections} construct, see
\specref{subsec:sections Construct}.

\item Data attribute clauses, see
\specref{subsec:Data-Sharing Attribute Clauses}.
\end{itemize}









% Here we need to force the blue floater down lower and force the subsection
% header higher to reduce the space between the blue floater and the header,
% as per Richard:
\vspace{3\baselineskip}
\begin{fortranspecific}
\vspace{-1\baselineskip}
\subsection{\hcode{parallel}~\hcode{workshare} Construct}
\index{parallel workshare@{\code{parallel}~\code{workshare}}}
\index{constructs!parallel workshare@{\code{parallel}~\code{workshare}}}
\index{combined constructs!parallel workshare@{\code{parallel}~\code{workshare}}}
\label{subsec:parallel workshare Construct}
\summary
The \code{parallel}~\code{workshare} construct is a shortcut for specifying a \code{parallel}
construct containing one \code{workshare} construct and no other statements.

\syntax
The syntax of the \code{parallel}~\code{workshare} construct is as follows:

\begin{ompfPragma}
!$omp parallel workshare \plc{[clause[ [},\plc{] clause] ... ]}
   \plc{structured-block }
!$omp end parallel workshare
\end{ompfPragma}

where \plc{clause} can be any of the clauses accepted by the \code{parallel} directive, with
identical meanings and restrictions. \code{nowait} may not be specified on an
\code{end}~\code{parallel}~\code{workshare} directive.

\descr
The semantics are identical to explicitly specifying a \code{parallel} directive immediately
followed by a \code{workshare} directive, and an \code{end}~\code{workshare} directive immediately
followed by an \code{end}~\code{parallel} directive.

\restrictions
The restrictions for the \code{parallel} construct and the \code{workshare} construct apply.

\crossreferences
\begin{itemize}
\item \code{parallel} construct, see
\specref{sec:parallel Construct}.

\item \code{workshare} construct, see
\specref{subsec:workshare Construct}.

\item Data attribute clauses, see
\specref{subsec:Data-Sharing Attribute Clauses}.
\end{itemize}
\end{fortranspecific}










\subsection{Parallel Loop SIMD Construct}
\label{subsec:Parallel Loop SIMD Construct}
\index{parallel loop SIMD construct}
\index{constructs!parallel loop SIMD construct}
\index{combined constructs!parallel loop SIMD construct}
\summary
The parallel loop SIMD construct is a shortcut for specifying a \code{parallel} construct
containing one loop SIMD construct and no other statement.

\begin{samepage}
\syntax
\begin{ccppspecific}
The syntax of the parallel loop SIMD construct is as follows:

\begin{ompcPragma}
#pragma omp parallel for simd \plc{[clause[ [},\plc{] clause] ... ] new-line}
    \plc{for-loops}
\end{ompcPragma}

where \plc{clause} can be any of the clauses accepted by the \code{parallel}
or \code{for}~\code{simd} directives, except the \code{nowait} clause, with
identical meanings and restrictions.
\end{ccppspecific}
\end{samepage}

\begin{fortranspecific}
\begin{samepage}
The syntax of the parallel loop SIMD construct is as follows:

\begin{ompfPragma}
!$omp parallel do simd \plc{[clause[ [},\plc{] clause] ... ]}
    \plc{do-loops}
\plc{[}!$omp end parallel do simd\plc{]}
\end{ompfPragma}
\end{samepage}

where \plc{clause} can be any of the clauses accepted by the \code{parallel}
or \code{do}~\code{simd} directives, with identical meanings and restrictions.

\begin{samepage}
If an \code{end}~\code{parallel}~\code{do}~\code{simd} directive is not specified, an
\code{end}~\code{parallel}~\code{do}~\code{simd} directive is assumed at the end of the
\plc{do-loops}. \code{nowait} may not be specified on
an \code{end}~\code{parallel}~\code{do}~\code{simd} directive.
\end{samepage}
\end{fortranspecific}

\descr
The semantics of the parallel loop SIMD construct are identical to explicitly specifying
a \code{parallel} directive immediately followed by a loop SIMD directive. The effect of
any clause that applies to both constructs is as if it were applied to the loop SIMD
construct and not to the \code{parallel} construct.

\restrictions
The restrictions for the \code{parallel} construct and the loop SIMD construct apply.

\crossreferences
\begin{itemize}
\item \code{parallel} construct, see
\specref{sec:parallel Construct}.

\item loop SIMD construct, see
\specref{subsec:Loop SIMD Construct}.

\item Data attribute clauses, see
\specref{subsec:Data-Sharing Attribute Clauses}.
\end{itemize}









\subsection{\hcode{target}~\hcode{parallel} Construct}
\label{subsec:target parallel Construct}
\index{target parallel@{\code{target}~\code{parallel}}}
\index{constructs!target parallel@{\code{target}~\code{parallel}}}
\index{combined constructs!target parallel@{\code{target}~\code{parallel}}}
\summary
The \code{target} \code{parallel} construct is a shortcut for specifying a \code{target}
construct containing a \code{parallel} construct and no other statements.

\syntax
\begin{ccppspecific}
The syntax of the \code{target} \code{parallel} construct is as follows:

\begin{ompcPragma}
#pragma omp target parallel \plc{[clause[ [},\plc{] clause] ... ] new-line}
    \plc{structured-block}
\end{ompcPragma}

where \plc{clause} can be any of the clauses accepted by the \code{target} or
\code{parallel} directives, except for \code{copyin}, with identical meanings and restrictions.
\end{ccppspecific}

\begin{samepage}
\begin{fortranspecific}
The syntax of the \code{target} \code{parallel} construct is as follows:

\begin{ompfPragma}
!$omp target parallel \plc{[clause[ [},\plc{] clause] ... ]}
    \plc{structured-block}
!$omp end target parallel
\end{ompfPragma}

where \plc{clause} can be any of the clauses accepted by the \code{target} or
\code{parallel} directives, except for \code{copyin}, with identical meanings and restrictions.
\end{fortranspecific}
\end{samepage}

\descr
The semantics are identical to explicitly specifying a \code{target} directive
immediately followed by a \code{parallel} directive.

\restrictions

The restrictions for the \code{target} and \code{parallel} constructs apply except for the following explicit modifications:

\begin{itemize}
\item If any \code{if} clause on the directive includes a
      \plc{directive-name-modifier} then all \code{if} clauses
      on the directive must include a \plc{directive-name-modifier}.

\item At most one \code{if} clause without a
      \plc{directive-name-modifier} can appear on the directive.

\item At most one \code{if} clause with the \code{parallel}
      \plc{directive-name-modifier} can appear on the directive.


\item At most one \code{if} clause with the \code{target}
      \plc{directive-name-modifier} can appear on the directive.

\item If an \code{allocator} clause specifies an \plc{allocator} it can only be a predefined allocator variable.
\end{itemize}

\crossreferences
\begin{itemize}
\item \code{parallel} construct, see
\specref{sec:parallel Construct}.

\item \code{target} construct, see
\specref{subsec:target Construct}.

\item \code{if} Clause, see \specref{sec:if Clause}.

\item Data attribute clauses, see
\specref{subsec:Data-Sharing Attribute Clauses}.


%% \item Multi-if Clause, see \specref{subsec:Multi-if Clause}.
\end{itemize}









%Similar to Distribute Parallel Loop Construct

\subsection{Target Parallel Loop Construct}
\label{subsec:Target Parallel Loop Construct}
\index{target parallel loop construct}
\index{constructs!target parallel loop}
\index{constructs!target parallel for@{\code{target}~\code{parallel}~\code{for}}}
\index{constructs!target parallel do@{\code{target}~\code{parallel}~\code{do}}}
\index{combined constructs!target parallel loop}
\summary
The target parallel loop construct is a shortcut for specifying a \code{target}
construct containing a parallel loop construct and no other statements.

\syntax
\begin{ccppspecific}
The syntax of the target parallel loop construct is as follows:

\begin{ompcPragma}
#pragma omp target parallel for \plc{[clause[ [},\plc{] clause] ... ] new-line}
    \plc{for-loops}
\end{ompcPragma}

where \plc{clause} can be any of the clauses accepted by the \code{target} or
\code{parallel}~\code{for} directives, except for \code{copyin}, with identical meanings and restrictions.
\end{ccppspecific}

\needspace{6\baselineskip}
\begin{fortranspecific}
The syntax of the target parallel loop construct is as follows:

\begin{ompfPragma}
!$omp target parallel do \plc{[clause[ [},\plc{] clause] ... ]}
    \plc{do-loops}
\plc{[}!$omp end target parallel do\plc{]}
\end{ompfPragma}

where \plc{clause} can be any of the clauses accepted by the \code{target} or
\code{parallel}~\code{do} directives, except for \code{copyin}, with identical meanings and restrictions.

If an \code{end}~\code{target}~\code{parallel}~\code{do} directive is not specified, an
\code{end}~\code{target}~\code{parallel}~\code{do} directive is assumed at the end of
the \plc{do-loops}.
\end{fortranspecific}

\descr
The semantics are identical to explicitly specifying a \code{target} directive
immediately followed by a parallel loop directive.


\restrictions
The restrictions for the \code{target} and parallel loop constructs apply except for the following explicit modifications:

\begin{itemize}
\item If any \code{if} clause on the directive includes a
      \plc{directive-name-modifier} then all \code{if} clauses
      on the directive must include a \plc{directive-name-modifier}.

\item At most one \code{if} clause without a
      \plc{directive-name-modifier} can appear on the directive.

\item At most one \code{if} clause with the \code{parallel}
      \plc{directive-name-modifier} can appear on the directive.


\item At most one \code{if} clause with the \code{target}
      \plc{directive-name-modifier} can appear on the directive.
\end{itemize}

\crossreferences
\begin{itemize}
\item \code{target} construct, see
\specref{subsec:target Construct}.

\item Parallel loop construct, see
\specref{subsec:Parallel Loop Construct}.

\item \code{if} Clause, see \specref{sec:if Clause}.

\item Data attribute clauses, see
\specref{subsec:Data-Sharing Attribute Clauses}.

%% \item Multi-if Clause, see \specref{subsec:Multi-if Clause}.
\end{itemize}









% Similar to Distribute Parallel Loop SIMD Construct

\subsection{Target Parallel Loop SIMD Construct}
\label{subsec:Target Parallel Loop SIMD Construct}
\index{target parallel loop SIMD construct}
\index{constructs!target parallel loop SIMD}
\index{constructs!target parallel for simd@{\code{target}~\code{parallel}~\code{for}~\code{simd}}}
\index{constructs!target parallel do simd@{\code{target}~\code{parallel}~\code{do}~\code{simd}}}
\index{combined constructs!target parallel loop SIMD}
\summary
The target parallel loop SIMD construct is a shortcut for specifying a \code{target}
construct containing a parallel loop SIMD construct and no other statements.

\syntax
\begin{ccppspecific}
The syntax of the target parallel loop SIMD construct is as follows:

\begin{ompcPragma}
#pragma omp target parallel for simd \plc{[clause[
[},\plc{] clause] ... ] new-line}
    \plc{for-loops}
\end{ompcPragma}

where \plc{clause} can be any of the clauses accepted by the \code{target} or
\code{parallel}~\code{for}~\code{simd} directives, except for \code{copyin}, with identical meanings and restrictions.
\end{ccppspecific}

\needspace{6\baselineskip}
\begin{fortranspecific}
The syntax of the target parallel loop SIMD construct is as follows:

\begin{ompfPragma}
!$omp target parallel do simd \plc{[clause[ [},\plc{] clause] ... ]}
    \plc{do-loops}
\plc{[}!$omp end target parallel do simd\plc{]}
\end{ompfPragma}

where \plc{clause} can be any of the clauses accepted by the \code{target} or
\code{parallel}~\code{do}~\code{simd} directives, except for \code{copyin}, with identical meanings and restrictions.

If an \code{end}~\code{target}~\code{parallel}~\code{do}~\code{simd} directive is not specified, an
\code{end}~\code{target}~\code{parallel}~\code{do}~\code{simd} directive is assumed at the end of
the \plc{do-loops}.
\end{fortranspecific}

\descr
The semantics are identical to explicitly specifying a \code{target} directive
immediately followed by a parallel loop SIMD directive.


\restrictions
The restrictions for the \code{target} and parallel loop SIMD constructs apply except for the following explicit modifications:

\begin{itemize}
\item If any \code{if} clause on the directive includes a
      \plc{directive-name-modifier} then all \code{if} clauses
      on the directive must include a \plc{directive-name-modifier}.

\item At most one \code{if} clause without a
      \plc{directive-name-modifier} can appear on the directive.

\item At most one \code{if} clause with the \code{parallel}
      \plc{directive-name-modifier} can appear on the directive.


\item At most one \code{if} clause with the \code{target}
      \plc{directive-name-modifier} can appear on the directive.
\end{itemize}

\crossreferences
\begin{itemize}
\item \code{target} construct, see
\specref{subsec:target Construct}.

\item Parallel loop SIMD construct, see
\specref{subsec:Parallel Loop SIMD Construct}.

\item \code{if} Clause, see \specref{sec:if Clause}.

\item Data attribute clauses, see
\specref{subsec:Data-Sharing Attribute Clauses}.

%% \item Multi-if Clause, see \specref{subsec:Multi-if Clause}.
\end{itemize}










\subsection{\hcode{target}~\hcode{simd} Construct}
\index{target simd@{\code{target}~\code{simd}}}
\index{constructs!target simd@{\code{target}~\code{simd}}}
\index{combined constructs!target simd@{\code{target}~\code{simd}}}
\label{subsec:target simd Construct}

\summary
The \code{target} \code{simd} construct is a shortcut for specifying a \code{target}
construct containing a \code{simd} construct and no other statements.

\syntax
\begin{ccppspecific}
The syntax of the \code{target} \code{simd} construct is as follows:

\begin{ompcPragma}
#pragma omp target simd \plc{[clause[ [},\plc{] clause] ... ] new-line}
    \plc{for-loops}
\end{ompcPragma}

where \plc{clause} can be any of the clauses accepted by the \code{target} or
\code{simd} directives with identical meanings and restrictions.

\end{ccppspecific}

\needspace{6\baselineskip}
\begin{fortranspecific}
The syntax of the \code{target} \code{simd} construct is as follows:

\begin{ompfPragma}
!$omp target simd \plc{[clause[ [},\plc{] clause] ... ]}
    \plc{do-loops}
\plc{[}!$omp end target simd\plc{]}
\end{ompfPragma}

where \plc{clause} can be any of the clauses accepted by the \code{target} or
\code{simd} directives with identical meanings and restrictions.

If an \code{end}~\code{target}~\code{simd} directive is not specified, an
\code{end}~\code{target}~\code{simd} directive is assumed at the end of
the \plc{do-loops}.
\end{fortranspecific}

\descr
The semantics are identical to explicitly specifying a \code{target} directive
immediately followed by a \code{simd} directive.

\restrictions

The restrictions for the \code{target} and \code{simd} constructs apply.

\crossreferences
\begin{itemize}
\item \code{simd} construct, see
\specref{subsec:simd Construct}.

\item \code{target} construct, see
\specref{subsec:target Construct}.

\item Data attribute clauses, see
\specref{subsec:Data-Sharing Attribute Clauses}.
\end{itemize}









\subsection{\hcode{target}~\hcode{teams} Construct}
\label{subsec:target teams Construct}
\index{target teams@{\code{target}~\code{teams}}}
\index{constructs!target teams@{\code{target}~\code{teams}}}
\index{combined constructs!target teams@{\code{target}~\code{teams}}}
\summary
The \code{target}~\code{teams} construct is a shortcut for specifying a \code{target} construct
containing a \code{teams} construct and no other statements.

\syntax
\begin{ccppspecific}
The syntax of the \code{target}~\code{teams} construct is as follows:

\begin{ompcPragma}
#pragma omp target teams \plc{[clause[ [},\plc{] clause] ... ] new-line}
   \plc{structured-block}
\end{ompcPragma}

where \plc{clause} can be any of the clauses accepted by the \code{target} or \code{teams} directives
with identical meanings and restrictions.
\end{ccppspecific}

\begin{fortranspecific}
The syntax of the \code{target}~\code{teams} construct is as follows:

\begin{ompfPragma}
!$omp target teams \plc{[clause[ [},\plc{] clause] ... ]}
    \plc{structured-block}
!$omp end target teams
\end{ompfPragma}

where \plc{clause} can be any of the clauses accepted by the \code{target} or \code{teams} directives
with identical meanings and restrictions.
\end{fortranspecific}

\descr

The semantics are identical to explicitly specifying a \code{target} directive
immediately followed by a \code{teams} directive.

\restrictions
The restrictions for the \code{target} and \code{teams} constructs apply except for the following explicit modifications:

\begin{itemize}

\item If an \code{allocator} clause specifies an \plc{allocator} it can only be a predefined allocator variable.

\end{itemize}

\crossreferences
\begin{itemize}
\item \code{target} construct, see
\specref{subsec:target Construct}.

\item \code{teams} construct, see
\specref{subsec:teams Construct}.

\item Data attribute clauses, see
\specref{subsec:Data-Sharing Attribute Clauses}.
\end{itemize}









\subsection{\hcode{teams}~\hcode{distribute} Construct}
\index{teams distribute@{\code{teams}~\code{distribute}}}
\index{constructs!teams distribute@{\code{teams}~\code{distribute}}}
\index{combined constructs!teams distribute@{\code{teams}~\code{distribute}}}
\label{subsec:teams distribute Construct}
\summary
The \code{teams}~\code{distribute} construct is a shortcut for specifying a \code{teams} construct
containing a \code{distribute} construct and no other statements.

\syntax
\begin{ccppspecific}
The syntax of the \code{teams}~\code{distribute} construct is as follows:

\begin{ompcPragma}
#pragma omp teams distribute \plc{[clause[ [},\plc{] clause] ... ] new-line}
    \plc{for-loops}
\end{ompcPragma}

where \plc{clause} can be any of the clauses accepted by the \code{teams} or \code{distribute}
directives with identical meanings and restrictions.
\end{ccppspecific}

\begin{fortranspecific}
The syntax of the \code{teams}~\code{distribute} construct is as follows:

\begin{ompfPragma}
!$omp teams distribute \plc{[clause[ [},\plc{] clause] ... ]}
    \plc{do-loops}
\plc{[}!$omp end teams distribute\plc{]}
\end{ompfPragma}

where \plc{clause} can be any of the clauses accepted by the \code{teams} or \code{distribute}
directives with identical meanings and restrictions.

If an \code{end}~\code{teams}~\code{distribute} directive is not specified, an
\code{end}~\code{teams}~\code{distribute} directive is assumed at the end of the \plc{do-loops}.
\end{fortranspecific}

\descr
The semantics are identical to explicitly specifying a \code{teams} directive immediately
followed by a \code{distribute} directive. The effect of any clause that applies to both constructs is as if it were applied to both constructs separately.


\restrictions
The restrictions for the \code{teams} and \code{distribute} constructs apply.

\crossreferences
\begin{itemize}
\item \code{teams} construct, see
\specref{subsec:teams Construct}.

\item \code{distribute} construct, see
\specref{subsec:distribute Construct}.

\item Data attribute clauses, see
\specref{subsec:Data-Sharing Attribute Clauses}.
\end{itemize}












\subsection{\hcode{teams}~\hcode{distribute}~\hcode{simd} Construct}
\index{teams distribute simd@{\code{teams}~\code{distribute}~\code{simd}}}
\index{constructs!teams distribute simd@{\code{teams}~\code{distribute}~\code{simd}}}
\index{combined constructs!teams distribute simd@{\code{teams}~\code{distribute}~\code{simd}}}
\label{subsec:teams distribute simd Construct}
\summary
The \code{teams}~\code{distribute}~\code{simd} construct is a shortcut for specifying a \code{teams} construct
containing a \code{distribute}~\code{simd} construct and no other statements.

\syntax
\begin{ccppspecific}
The syntax of the \code{teams}~\code{distribute}~\code{simd} construct is as follows:

\begin{ompcPragma}
#pragma omp teams distribute simd \plc{[clause[ [},\plc{] clause] ... ] new-line}
    \plc{for-loops}
\end{ompcPragma}

where \plc{clause} can be any of the clauses accepted by the \code{teams} or \code{distribute}~\code{simd}
directives with identical meanings and restrictions.
\end{ccppspecific}

\begin{fortranspecific}
The syntax of the \code{teams}~\code{distribute}~\code{simd} construct is as follows:

\begin{ompfPragma}
!$omp teams distribute simd \plc{[clause[ [},\plc{] clause] ... ]}
    \plc{do-loops}
\plc{[}!$omp end teams distribute simd\plc{]}
\end{ompfPragma}

where \plc{clause} can be any of the clauses accepted by the \code{teams} or \code{distribute}~\code{simd}
directives with identical meanings and restrictions.

If an \code{end}~\code{teams}~\code{distribute}~\code{simd} directive is
not specified, an \code{end}~\code{teams}~\code{distribute}~\code{simd}
directive is assumed at the end of the \plc{do-loops}.
\end{fortranspecific}

\descr
The semantics are identical to explicitly specifying a \code{teams} directive immediately
followed by a \code{distribute}~\code{simd} directive. The effect of any clause that applies to both constructs is as if it were applied to both constructs separately.


\restrictions
The restrictions for the \code{teams} and \code{distribute}~\code{simd} constructs apply.

\crossreferences
\begin{itemize}
\item \code{teams} construct, see
\specref{subsec:teams Construct}.

\item \code{distribute}~\code{simd} construct, see
\specref{subsec:distribute simd Construct}.

\item Data attribute clauses, see
\specref{subsec:Data-Sharing Attribute Clauses}.
\end{itemize}











\subsection{\hcode{target}~\hcode{teams}~\hcode{distribute} Construct}
\index{target teams distribute@{\code{target}~\code{teams}~\code{distribute}}}
\index{constructs!target teams distribute@{\code{target}~\code{teams}~\code{distribute}}}
\index{combined constructs!target teams distribute@{\code{target}~\code{teams}~\code{distribute}}}
\label{subsec:target teams distribute construct}
\summary
The \code{target}~\code{teams}~\code{distribute} construct is a shortcut for specifying a \code{target} construct
containing a \code{teams}~\code{distribute} construct and no other statements.

\syntax
\begin{ccppspecific}
The syntax of the \code{target}~\code{teams}~\code{distribute} construct is as follows:

\begin{ompcPragma}
#pragma omp target teams distribute \plc{[clause[ [},\plc{] clause] ... ] new-line}
   \plc{for-loops}
\end{ompcPragma}

where \plc{clause} can be any of the clauses accepted by the \code{target} or \code{teams}~\code{distribute} directives
with identical meanings and restrictions.
\end{ccppspecific}

\begin{fortranspecific}
The syntax of the \code{target}~\code{teams}~\code{distribute} construct is as follows:

\begin{ompfPragma}
!$omp target teams distribute \plc{[clause[ [},\plc{] clause] ... ]}
    \plc{do-loops}
\plc{[}!$omp end target teams distribute\plc{]}
\end{ompfPragma}

where \plc{clause} can be any of the clauses accepted by the \code{target} or \code{teams}~\code{distribute} directives
with identical meanings and restrictions.

If an \code{end}~\code{target}~\code{teams}~\code{distribute} directive is not specified, an
\code{end}~\code{target}~\code{teams}~\code{distribute} directive is assumed at the end of the \plc{do-loops}.
\end{fortranspecific}

\descr
The semantics are identical to explicitly specifying a \code{target} directive immediately
followed by a \code{teams}~\code{distribute} directive.

\restrictions
The restrictions for the \code{target} and \code{teams}~\code{distribute} constructs apply except for the following explicit modifications:

\begin{itemize}

\item If an \code{allocator} clause specifies an \plc{allocator} it can only be a predefined allocator variable.

\end{itemize}

\crossreferences
\begin{itemize}
\item \code{target} construct, see
\specref{subsec:target data Construct}.

\item \code{teams}~\code{distribute} construct, see
\specref{subsec:teams distribute Construct}.

\item Data attribute clauses, see
\specref{subsec:Data-Sharing Attribute Clauses}.
\end{itemize}










\subsection{\hcode{target}~\hcode{teams}~\hcode{distribute}~\hcode{simd} Construct}
\index{target teams distribute simd@{\code{target}~\code{teams}~\code{distribute}~\code{simd}}}
\index{constructs!target teams distribute simd@{\code{target}~\code{teams}~\code{distribute}~\code{simd}}}
\index{combined constructs!target teams distribute simd@{\code{target}~\code{teams}~\code{distribute}~\code{simd}}}
\label{subsec:target teams distribute simd construct}
\summary
The \code{target}~\code{teams}~\code{distribute}~\code{simd} construct is a shortcut for specifying a \code{target} construct
containing a \code{teams}~\code{distribute}~\code{simd} construct and no other statements.

\syntax
\begin{ccppspecific}
The syntax of the \code{target}~\code{teams}~\code{distribute}~\code{simd} construct is as follows:

\begin{ompcPragma}
#pragma omp target teams distribute simd \plc{\}
            \plc{[clause[ [},\plc{] clause] ...  ] new-line}
   \plc{for-loops}
\end{ompcPragma}

where \plc{clause} can be any of the clauses accepted by the \code{target} or
\code{teams}~\code{distribute}~\code{simd} directives with identical meanings and restrictions.
\end{ccppspecific}

\begin{fortranspecific}
The syntax of the \code{target}~\code{teams}~\code{distribute}~\code{simd} construct is as follows:

\begin{ompfPragma}
!$omp target teams distribute simd \plc{[clause[ [},\plc{] clause] ... ]}
    \plc{do-loops}
\plc{[}!$omp end target teams distribute simd\plc{]}
\end{ompfPragma}

where \plc{clause} can be any of the clauses accepted by the \code{target} or
\code{teams}~\code{distribute}~\code{simd} directives with identical meanings and restrictions.

If an \code{end}~\code{target}~\code{teams}~\code{distribute}~\code{simd} directive is not specified, an
\code{end}~\code{target}~\code{teams}~\code{distribute}~\code{simd} directive is assumed at the end of the \plc{do-loops}.
\end{fortranspecific}

\descr
The semantics are identical to explicitly specifying a \code{target} directive immediately
followed by a \code{teams}~\code{distribute}~\code{simd} directive.

\restrictions
The restrictions for the \code{target} and \code{teams}~\code{distribute}~\code{simd} constructs apply.

\crossreferences
\begin{itemize}
\item \code{target} construct, see
\specref{subsec:target data Construct}.

\item \code{teams}~\code{distribute}~\code{simd} construct, see
\specref{subsec:teams distribute simd Construct}.

\item Data attribute clauses, see
\specref{subsec:Data-Sharing Attribute Clauses}.
\end{itemize}











\subsection{Teams Distribute Parallel Loop Construct}
\label{subsec:Teams Distribute Parallel Loop Construct}
\index{teams distribute parallel loop construct}
\index{constructs!teams distribute parallel loop construct}
\index{combined constructs!teams distribute parallel loop construct}
\summary
The teams distribute parallel loop construct is a shortcut for specifying a \code{teams}
construct containing a distribute parallel loop construct and no other statements.

\syntax
\begin{ccppspecific}
The syntax of the teams distribute parallel loop construct is as follows:

\begin{ompcPragma}[fontsize=\small]
#pragma omp teams distribute parallel for \plc{\}
            \plc{[clause[ [},\plc{] clause] ...  ] new-line}
    \plc{for-loops}
\end{ompcPragma}

where \plc{clause} can be any of the clauses accepted by the \code{teams} or
\code{distribute}~\code{parallel}~\code{for} directives with identical meanings and restrictions.
\end{ccppspecific}

\begin{fortranspecific}
The syntax of the teams distribute parallel loop construct is as follows:

\begin{ompfPragma}
!$omp teams distribute parallel do \plc{[clause[ [},\plc{] clause] ... ]}
   \plc{do-loops}
\plc{[} !$omp end teams distribute parallel do \plc{]}
\end{ompfPragma}

where \plc{clause} can be any of the clauses accepted by the \code{teams} or
\code{distribute}~\code{parallel}~\code{do} directives with identical meanings and restrictions.

If an \code{end}~\code{teams}~\code{distribute}~\code{parallel}~\code{do} directive is not specified, an
\code{end}~\code{teams}~\code{distribute}~\code{parallel}~\code{do} directive is assumed at the end of the \plc{do-loops}.
\end{fortranspecific}

\descr
The semantics are identical to explicitly specifying a \code{teams} directive immediately
followed by a distribute parallel loop directive. The effect of any clause that applies to
both constructs is as if it were applied to both constructs separately.

\restrictions
The restrictions for the \code{teams} and distribute parallel loop constructs apply.

\crossreferences
\begin{itemize}
\item \code{teams} construct, see
\specref{subsec:teams Construct}.

\item Distribute parallel loop construct, see
\specref{subsec:Distribute Parallel Loop Construct}.

\item Data attribute clauses, see
\specref{subsec:Data-Sharing Attribute Clauses}.
\end{itemize}









\subsection{Target Teams Distribute Parallel Loop Construct}
\label{subsec:Target Teams Distribute Parallel Loop Construct}
\index{target teams distribute parallel loop construct}
\index{constructs!target teams distribute parallel loop construct}
\index{combined constructs!target teams distribute parallel loop construct}
\summary
The target teams distribute parallel loop construct is a shortcut for specifying a \code{target}
construct containing a teams distribute parallel loop construct and no other statements.

\syntax
\begin{ccppspecific}
The syntax of the target teams distribute parallel loop construct is as follows:

\begin{ompcPragma}[fontsize=\small]
#pragma omp target teams distribute parallel for \plc{\}
            \plc{[clause[ [},\plc{] clause] ... ] new-line}
    \plc{for-loops}
\end{ompcPragma}

where \plc{clause} can be any of the clauses accepted by the \code{target} or
\code{teams}~\code{distribute}~\code{parallel}~\code{for} directives with identical
meanings and restrictions.
\end{ccppspecific}

\needspace{6\baselineskip}
\begin{fortranspecific}
The syntax of the target teams distribute parallel loop construct is as follows:

\begin{ompfPragma}
!$omp target teams distribute parallel do \plc{[clause[ [},\plc{] clause] ... ]}
    \plc{do-loops}
\plc{[}!$omp end target teams distribute parallel do\plc{]}
\end{ompfPragma}

where \plc{clause} can be any of the clauses accepted by the \code{target} or
\code{teams}~\code{distribute}~\code{parallel}~\code{do} directives with
identical meanings and restrictions.

If an \code{end}~\code{target}~\code{teams}~\code{distribute}~\code{parallel}~\code{do} directive is not specified, an
\code{end}~\code{target}~\code{teams}~\code{distribute}~\code{parallel}~\code{do}
directive is assumed at the end of the \plc{do-loops}.
\end{fortranspecific}

\descr
The semantics are identical to explicitly specifying a \code{target}
directive immediately followed by a teams distribute parallel loop directive.


\restrictions
The restrictions for the \code{target} and teams distribute parallel
loop constructs apply except for the following explicit modifications:

\begin{itemize}
\item If any \code{if} clause on the directive includes a
      \plc{directive-name-modifier} then all \code{if} clauses
      on the directive must include a \plc{directive-name-modifier}.

\item At most one \code{if} clause without a
      \plc{directive-name-modifier} can appear on the directive.

\item At most one \code{if} clause with the \code{parallel}
      \plc{directive-name-modifier} can appear on the directive.


\item At most one \code{if} clause with the \code{target}
      \plc{directive-name-modifier} can appear on the directive.
\end{itemize}

\crossreferences
\begin{itemize}
\item \code{target} construct, see \specref{subsec:target Construct}.

\item Teams distribute parallel loop construct, see
      \specref{subsec:Teams Distribute Parallel Loop Construct}.

\item \code{if} Clause, see \specref{sec:if Clause}.

\item Data attribute clauses, see
      \specref{subsec:Data-Sharing Attribute Clauses}.
\end{itemize}










\subsection{Teams Distribute Parallel Loop SIMD Construct}
\label{subsec:Teams Distribute Parallel Loop SIMD Construct}
\index{teams distribute parallel loop SIMD construct}
\index{constructs!teams distribute parallel loop SIMD construct}
\index{combined constructs!teams distribute parallel loop SIMD construct}
\summary
The teams distribute parallel loop SIMD construct is a shortcut for specifying a \code{teams}
construct containing a distribute parallel loop SIMD construct and no other statements.

%\newpage %% HACK
\syntax
\begin{ccppspecific}
The syntax of the teams distribute parallel loop construct is as follows:

\begin{ompcPragma}[fontsize=\small]
#pragma omp teams distribute parallel for simd \plc{\}
            \plc{[clause[ [},\plc{] clause] ... ] new-line}
    \plc{for-loops}
\end{ompcPragma}

where \plc{clause} can be any of the clauses accepted by the \code{teams} or
\code{distribute}~\code{parallel}~\code{for}~\code{simd}
directives with identical meanings and restrictions.
\end{ccppspecific}

\begin{fortranspecific}
The syntax of the teams distribute parallel loop construct is as follows:

\begin{ompfPragma}
!$omp teams distribute parallel do simd \plc{[clause[ [},\plc{] clause] ... ]}
    \plc{do-loops}
\plc{[}!$omp end teams distribute parallel do simd\plc{]}
\end{ompfPragma}

where \plc{clause} can be any of the clauses accepted by the \code{teams} or
\code{distribute}~\code{parallel}~\code{do}~\code{simd}
directives with identical meanings and restrictions.

If an \code{end}~\code{teams}~\code{distribute}~\code{parallel}~\code{do}~\code{simd} directive is not specified, an
\code{end}~\code{teams}~\code{distribute}~\code{parallel}~\code{do}~\code{simd} directive is assumed at the end of
the \plc{do-loops}.
\end{fortranspecific}

\descr
The semantics are identical to explicitly specifying a \code{teams} directive immediately
followed by a distribute parallel loop SIMD directive. The effect of any clause that
applies to both constructs is as if it were applied to both constructs separately.

\restrictions
The restrictions for the \code{teams} and distribute parallel loop
SIMD constructs apply.

\crossreferences
\begin{itemize}
\item \code{teams} construct, see
\specref{subsec:teams Construct}.

\item Distribute parallel loop SIMD construct, see
\specref{subsec:Distribute Parallel Loop SIMD Construct}.

\item Data attribute clauses, see
\specref{subsec:Data-Sharing Attribute Clauses}.
\end{itemize}










\subsection{Target Teams Distribute Parallel Loop SIMD Construct}
\label{subsec:Target Teams Distribute Parallel Loop SIMD Construct}
\index{target teams distribute parallel loop SIMD construct}
\index{constructs!target teams distribute parallel loop SIMD construct}
\index{combined constructs!target teams distribute parallel loop SIMD construct}
\summary
The target teams distribute parallel loop SIMD construct is a shortcut for specifying a \code{target}
construct containing a teams distribute parallel loop SIMD construct and no other statements.

\syntax
\begin{ccppspecific}
The syntax of the target teams distribute parallel loop SIMD construct is as follows:

\begin{ompcPragma}
#pragma omp target teams distribute parallel for simd \plc{\}
            \plc{[clause[ [},\plc{] clause] ... ] new-line}
    \plc{for-loops}
\end{ompcPragma}

where \plc{clause} can be any of the clauses accepted by the \code{target} or
\code{teams}~\code{distribute}~\code{parallel}~\code{for}~\code{simd}
directives with identical meanings and restrictions.
\end{ccppspecific}

\begin{fortranspecific}
The syntax of the target teams distribute parallel loop SIMD construct is as follows:

\begin{ompfPragma}
!$omp target teams distribute parallel do simd \plc{[clause[ [},\plc{] clause] ... ]}
    \plc{do-loops}
\plc{[}!$omp end target teams distribute parallel do simd\plc{]}
\end{ompfPragma}

where \plc{clause} can be any of the clauses accepted by the
\code{target} or \code{teams}~\code{distribute}~\code{parallel}~\code{do}~\code{simd}
directives with identical meanings and restrictions.

If an \code{end}~\code{target}~\code{teams}~\code{distribute}~\code{parallel}~\code{do}~\code{simd}
directive is not specified, an
\code{end}~\code{target}~\code{teams}~\code{distribute}~\code{parallel}~\code{do}~\code{simd}
directive is assumed at the end of the \plc{do-loops}.
\end{fortranspecific}

\descr
The semantics are identical to explicitly specifying a \code{target}
directive immediately followed by a teams distribute parallel loop
SIMD directive.

\restrictions
The restrictions for the \code{target} and teams distribute parallel
loop SIMD constructs apply except for the following explicit modifications:

\begin{itemize}
\item If any \code{if} clause on the directive includes a
      \plc{directive-name-modifier} then all \code{if} clauses
      on the directive must include a \plc{directive-name-modifier}.

\item At most one \code{if} clause without a
      \plc{directive-name-modifier} can appear on the directive.

\item At most one \code{if} clause with the \code{parallel}
      \plc{directive-name-modifier} can appear on the directive.

\item At most one \code{if} clause with the \code{target}
      \plc{directive-name-modifier} can appear on the directive.
\end{itemize}

\crossreferences
\begin{itemize}
\item \code{target} construct, see \specref{subsec:target Construct}.

\item Teams distribute parallel loop SIMD construct, see
      \specref{subsec:Teams Distribute Parallel Loop SIMD Construct}.

\item \code{if} Clause, see \specref{sec:if Clause}.

\item Data attribute clauses, see
      \specref{subsec:Data-Sharing Attribute Clauses}.
\end{itemize}
