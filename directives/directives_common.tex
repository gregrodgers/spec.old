% This is an included file. See the master file for more information.
%
% When editing this file:
%
%    1. To change formatting, appearance, or style, please edit openmp.sty.
%
%    2. Custom commands and macros are defined in openmp.sty.
%
%    3. Be kind to other editors -- keep a consistent style by copying-and-pasting to
%       create new content.
%
%    4. We use semantic markup, e.g. (see openmp.sty for a full list):
%         \code{}     % for bold monospace keywords, code, operators, etc.
%         \plc{}      % for italic placeholder names, grammar, etc.
%
%    5. There are environments that provide special formatting, e.g. language bars.
%       Please use them whereever appropriate.  Examples are:
%
%         \begin{fortranspecific}
%         This is text that appears enclosed in blue language bars for Fortran.
%         \end{fortranspecific}
%
%         \begin{note}
%         This is a note.  The "Note -- " header appears automatically.
%         \end{note}
%
%    6. Other recommendations:
%         Use the convenience macros defined in openmp.sty for the minor headers
%         such as Comments, Syntax, etc.
%
%         To keep items together on the same page, prefer the use of
%         \begin{samepage}.... Avoid \parbox for text blocks as it interrupts line numbering.
%         When possible, avoid \filbreak, \pagebreak, \newpage, \clearpage unless that's
%         what you mean. Use \needspace{} cautiously for troublesome paragraphs.
%
%         Avoid absolute lengths and measures in this file; use relative units when possible.
%         Vertical space can be relative to \baselineskip or ex units. Horizontal space
%         can be relative to \linewidth or em units.
%
%         Prefer \emph{} to italicize terminology, e.g.:
%             This is a \emph{definition}, not a placeholder.
%             This is a \plc{var-name}.
%



\section{Directive Format}
\label{sec:Directive Format}
\index{directive format}

\begin{ccppspecific}
OpenMP directives for C/C++ are specified with \pcode{\#pragma} directives.
The syntax of an OpenMP directive is as follows:

\begin{ompcPragma}
#pragma omp\plc{ directive-name [clause[ [},\plc{] clause] ... ] new-line}
\end{ompcPragma}

Each directive  starts with \pcode{\#pragma}~\code{omp}. The remainder of 
the directive follows the conventions of the C and C++ standards for compiler 
directives. In particular, white space can be used before and after the \pcode{\#}, 
and sometimes white space must be used to separate the words in a directive. 
Preprocessing tokens following \pcode{\#pragma}~\code{omp}
are subject to macro replacement.

Some OpenMP directives may be composed of consecutive \pcode{\#pragma}
directives if specified in their syntax.

Directives are case-sensitive.

Each of the expressions used in the OpenMP syntax inside of the clauses
must be a valid \plc{assignment-expression} of the base language unless
otherwise specified.
\end{ccppspecific}

\begin{cppspecific}
Directives may not appear in \code{constexpr} functions or in constant expressions.
Variadic parameter packs cannot be expanded into a directive or its clauses
except as part of an expression argument to be evaluated by the base language,
such as into a function call inside an \code{if} clause.
\end{cppspecific}

\begin{fortranspecific}
OpenMP directives for Fortran are specified as follows:

\begin{ompfPragma}
\plc{sentinel directive-name [clause[ [},\plc{] clause]...]}
\end{ompfPragma}

All OpenMP compiler directives must begin with a directive \emph{sentinel}. 
The format of a sentinel differs between fixed form and free form source files, 
as described in \specref{subsec:Fixed Source Form Directives} and 
\specref{subsec:Free Source Form Directives}.

Directives are case insensitive. Directives cannot be embedded within continued
statements, and statements cannot be embedded within directives.

Each of the expressions used in the OpenMP syntax inside of the clauses must 
be a valid \plc{expression} of the base language unless otherwise specified.

In order to simplify the presentation, free form is used for the syntax of OpenMP
directives for Fortran in the remainder of this document, except as noted.
\end{fortranspecific}

Only one \emph{directive-name} can be specified per directive (note that this 
includes combined directives, see \specref{sec:Combined Constructs}). The order 
in which clauses appear on directives is not significant. Clauses on directives 
may be repeated as needed, subject to the restrictions listed in the description 
of each clause.

Some clauses accept a \emph{list}, an \plc{extended-list}, or a \plc{locator-list}.  
A \plc{list} consists of a comma-separated collection of one or more \plc{list items}. 
An \plc{extended-list} consists of a comma-separated collection of one or more
\plc{extended list items}. A \plc{locator-list} consists of a comma-separated
collection of one or more \plc{locator list items}.

\begin{ccppspecific}
A \plc{list item} is a variable or an array section. An \plc{extended list item} is
a \plc{list item} or a function name.  A \plc{locator list item} is any lvalue
expression, including variables, or an array section.
\end{ccppspecific}

\begin{fortranspecific}
A \plc{list item} is a variable, array section or common block name
(enclosed in slashes). An \plc{extended list item} is a \plc{list item}
or a procedure name. A \plc{locator list item} is a \plc{list item}.

When a named common block appears in a \plc{list}, it has the same
meaning as if every explicit member of the common block appeared in
the list.  An explicit member of a common block is a variable that is
named in a \code{COMMON} statement that specifies the common block
name and is declared in the same scoping unit in which the clause
appears.

Although variables in common blocks can be accessed by use association
or host association, common block names cannot.  As a result, a common
block name specified in a data-sharing attribute, a data copying or
a data-mapping attribute clause must be declared to be a common block in
the same scoping unit in which the clause appears.

If a list item that appears in a directive or clause is an optional
dummy argument that is not present, the directive or clause for that
list item is ignored.

If the variable referenced inside a construct is an optional dummy
argument that is not present, any explicitly determined, implicitly
determined, or predetermined data-sharing and data-mapping attribute
rules for that variable are ignored.  Otherwise, if the variable is an
optional dummy argument that is present, it is present inside the
construct.
\end{fortranspecific}

For all base languages, a \plc{list item}, an \plc{extended list item}, or
a \plc{locator list item} is subject to the restrictions specified in 
\specref{subsec:Array Sections} and in each of the sections describing clauses 
and directives for which the \plc{list}, the \plc{extended-list}, or the
\plc{locator-list} appears.

Some executable directives include a structured block. A structured block:

\begin{itemize}
\item may contain infinite loops where the point of exit is never reached; 
\item may halt due to an IEEE exception;

\begin{ccppspecific}
\item may contain calls to \code{exit()}, \code{_Exit()}, \code{quick_exit()}, 
      \code{abort()} or functions with a \code{_Noreturn} specifier (in C) or 
      a \code{noreturn} attribute (in C/C++); 
\item may be an expression statement, iteration statement, selection statement,
      or try block, provided that the corresponding compound statement obtained 
      by enclosing it in \tcode{\{} and \tcode{\}} would be a structured block; and
\end{ccppspecific}

\begin{fortranspecific}
\item may contain \code{STOP} statements.
\end{fortranspecific}
\end{itemize}

\restrictions
Restrictions to structured blocks are as follows:

\begin{itemize}
\item Entry to a structured block must not be the result of a branch.
\item The point of exit cannot be a branch out of the structured block.

\begin{ccppspecific}
\item The point of entry to a structured block must not be a call to \code{setjmp()}.
\item \code{longjmp()} and \code{throw()} must not violate the entry/exit criteria.
\end{ccppspecific}
\end{itemize}



\begin{fortranspecific}

\subsection{Fixed Source Form Directives}
\label{subsec:Fixed Source Form Directives}
\index{fixed source form directives}

The following sentinels are recognized in fixed form source files:

\begin{ompfPragma}
!$omp \textnormal{|} c$omp \textnormal{|} *$omp
\end{ompfPragma} %$ close off misinterpreted dollar sign/math symbol

Sentinels must start in column 1 and appear as a single word with no intervening
characters. Fortran fixed form line length, white space, continuation, and column 
rules apply to the directive line. Initial directive lines must have a space or 
a zero in column 6, and continuation directive lines must have a character other 
than a space or a zero in column 6.

Comments may appear on the same line as a directive. The exclamation point initiates a
comment when it appears after column 6. The comment extends to the end of the source
line and is ignored. If the first non-blank character after the directive sentinel of 
an initial or continuation directive line is an exclamation point, the line is ignored.

\begin{note}
In the following example, the three formats for specifying the directive are
equivalent (the first line represents the position of the first 9 columns):

\begin{ompfPragma}
c23456789
!$omp parallel do shared(a,b,c)

c$omp parallel do
c$omp+shared(a,b,c)

c$omp paralleldoshared(a,b,c)
\end{ompfPragma}
\end{note}



\subsection{Free Source Form Directives}
\label{subsec:Free Source Form Directives}
\index{free source form directives}

The following sentinel is recognized in free form source files:

\begin{ompfPragma}
!$omp
\end{ompfPragma} %$ close off misinterpreted dollar sign/math symbol

The sentinel can appear in any column as long as it is preceded only by white space. 
It must appear as a single word with no intervening
white space. Fortran free form line length, white space, and continuation rules 
apply to the directive line. Initial directive lines must have a space after the 
sentinel. Continued directive lines must have an ampersand~(\code{&}) as the last 
non-blank character on the line, prior to any comment placed inside the directive. 
Continuation directive lines can have an ampersand after the directive sentinel 
with optional white space before and after the ampersand.

Comments may appear on the same line as a directive. The exclamation point~(\code{!})
initiates a comment. The comment extends to the end of the source line and is ignored.
If the first non-blank character after the directive sentinel is an exclamation point,
the line is ignored.

One or more blanks or horizontal tabs are optional to separate adjacent
keywords in \plc{directive-names} unless otherwise specified.

\begin{note}
In the following example the three formats for specifying the directive are
equivalent (the first line represents the position of the first 9 columns):

\begin{ompfPragma}
!23456789
       !$omp parallel do &
                 !$omp shared(a,b,c)

       !$omp parallel &
      !$omp&do shared(a,b,c)

!$omp paralleldo shared(a,b,c)
\end{ompfPragma} %$ Close off misinterpreted dollar sign/math symbol
\end{note}
\bigskip
\end{fortranspecific}



\subsection{Stand-Alone Directives}
\label{subsec:Stand-Alone Directives}
\index{stand-alone directives}

\summary
Stand-alone directives are executable directives that have no associated user code.

\descr
Stand-alone directives do not have any associated executable user code. Instead, 
they represent executable statements that typically do not have succinct equivalent 
statements in the base language. There are some restrictions on the placement of a 
stand-alone directive within a program. A stand-alone directive may be placed only 
at a point where a base language executable statement is allowed.

\restrictions
\begin{ccppspecific}
\begin{itemize}
\item A stand-alone directive may not be used in place of the statement following
      an \code{if}, \code{while}, \code{do}, \code{switch}, or \code{label}.
\end{itemize}
\end{ccppspecific}

\begin{fortranspecific}
\begin{itemize}
\item A stand-alone directive may not be used as the action statement in an 
      \code{if} statement or as the executable statement following a label 
      if the label is referenced in the program.
\end{itemize}
\end{fortranspecific}



\begin{ccppspecific}

\subsection{Array Shaping}
\label{subsec:Array Shaping}
\index{array shaping}

If an expression has a type of pointer to \plc{T}, then a shape-operator can be
used to specify the extent of that pointer. In other words, the
shape-operator is used to reinterpret, as an n-dimensional array, the region of
memory to which that expression points.

Formally, the syntax of the shape-operator is as follows:
\begin{indentedcodelist}
\plc{ shaped-expression } := ([\plc{s}@\textsubscript{\plc{1}}@\plc{}][\plc{s}@\textsubscript{\plc{2}}@]...[\plc{s}@\textsubscript{\plc{n}}@])\plc{cast-expression}
\end{indentedcodelist}

The result of applying the shape-operator to an expression is an lvalue
expression with an n-dimensional array type with dimensions
\plc{s}\textsubscript{\plc{1}} $\times$ \plc{s}\textsubscript{\plc{2}} \ldots
$\times$ \plc{s}\textsubscript{\plc{n}} and element type \plc{T}.

The precedence of the shape-operator is the same as a type cast.

Each $\plc{s}_\plc{i}$ is an integral type expression that must 
evaluate to a positive integer.

\restrictions
Restrictions to the shape-operator are as follows:

\begin{itemize}
\item The type \plc{T} must be a complete type.
\item The shape-operator can appear only in clauses where it is explicitly allowed.
\item The result of a shape-operator must be a named array of a list item.
\item The type of the expression upon which a shape-operator is applied must be 
      a pointer type.
\begin{cppspecific}
\item If the type \plc{T} is a reference to a type \plc{T'}, then the type will 
      be considered to be \plc{T'} for all purposes of the designated array.
\end{cppspecific}

\end{itemize}
\end{ccppspecific}



\subsection{Array Sections}
\label{subsec:Array Sections}
\index{array sections}

An array section designates a subset of the elements in an array.

\begin{ccppspecific}

To specify an array section in an OpenMP construct, array subscript expressions are
extended with the following syntax:

\begin{indentedcodelist}
[\plc{ lower-bound }:\plc{ length }:\plc{ stride}] \textnormal{or}
[\plc{ lower-bound }:\plc{ length }:\plc{ }] \textnormal{or}
[\plc{ lower-bound }:\plc{ length }] \textnormal{or}
[\plc{ lower-bound }:\plc{ }:\plc{ stride}] \textnormal{or}
[\plc{ lower-bound }:\plc{ }:\plc{ }] \textnormal{or}
[\plc{ lower-bound }:\plc{ }] \textnormal{or}
[ :\plc{ length }:\plc{ stride}] \textnormal{or}
[ :\plc{ length }:\plc{ }] \textnormal{or}
[ :\plc{ length }] \textnormal{or}
[\plc{ }:\plc{ }:\plc{ stride}]
[\plc{ }:\plc{ }:\plc{ }]
[\plc{ }:\plc{ }]
\end{indentedcodelist}

The array section must be a subset of the original array.

Array sections are allowed on multidimensional arrays. Base language array subscript
expressions can be used to specify length-one dimensions of multidimensional array
sections.

Each of the \plc{lower-bound}, \plc{length}, and \plc{stride} expressions
if specified must be an integral type \plc{expression} of the base language.
When evaluated they represent a set of integer values as follows:

\{ \plc{lower-bound}, \plc{lower-bound} + \plc{stride}, \plc{lower-bound} + 2 * \plc{stride},... , \plc{lower-bound} + ((\plc{length} - 1) * \plc{stride}) \}

The \plc{length} must evaluate to a non-negative integer.

The \plc{stride} must evaluate to a positive integer.

When the size of the array dimension is not known, the \plc{length} must
be specified explicitly.

When the \plc{stride} is absent it defaults to 1.

When the \plc{length} is absent it defaults to
$\blceil(\plc{size} - \plc{lower-bound})/\plc{stride}\brceil$, where \plc{size} is
the size of the array dimension.

When the \plc{lower-bound} is absent it defaults to 0.

The precedence of a subscript operator that uses the array section syntax is
the same as the precedence of a subscript operator that does not use the
array section syntax.

\begin{note}
The following are examples of array sections:

\begin{indentedcodelist}
a[0:6]
a[0:6:1]
a[1:10]
a[1:]
a[:10:2]
b[10][:][:]
b[10][:][:0]
c[42][0:6][:]
c[42][0:6:2][:]
c[1:10][42][0:6]
S.c[:100]
p->y[:10]
this->a[:N]
(p+10)[:N]
\end{indentedcodelist}

Assume \code{a} is declared to be a 1-dimensional array with dimension size
11.  The first two examples are equivalent, and the third and fourth
examples are equivalent. The fifth example specifies a stride of 2 and
therefore is not contiguous.

Assume \code{b} is declared to be a pointer to a 2-dimensional array with
dimension sizes 10 and 10. The sixth example refers to all elements of the
2-dimensional array given by \code{b[10]}. The seventh
example is a zero-length array section.

Assume \code{c} is declared to be a 3-dimensional array with dimension sizes
50, 50, and 50.  The eighth example is contiguous, while the ninth and
tenth examples are not contiguous.

The final four examples show array sections that are formed from
more general base expressions.

The following are examples that are non-conforming array sections:

\begin{indentedcodelist}
s[:10].x
p[:10]->y
*(xp[:10])
\end{indentedcodelist}

For all three examples, a base language operator is applied in an undefined
manner to an array section. The only operator that may be applied to an array
section is a subscript operator for which the array section appears as the
postfix expression.
\end{note}
\medskip
\end{ccppspecific}

\begin{fortranspecific}
Fortran has built-in support for array sections although some
restrictions apply to their use, as enumerated in the following section.
\end{fortranspecific}

\restrictions
Restrictions to array sections are as follows:

\begin{itemize}
\item An array section can appear only in clauses where it is explicitly allowed.
\item A \plc{stride} expression may not be specified unless otherwise stated.

\begin{ccppspecific}
\item An element of an array section with a non-zero size must have a complete type.
\item The base expression of an array section must have an array or pointer type.
\item If a consecutive sequence of array subscript expressions appears in an
      array section, and the first subscript expression in the sequence uses the
      extended array section syntax defined in this section, then only the last
      subscript expression in the sequence may select array elements that have
      a pointer type.
\end{ccppspecific}

\begin{cppspecific}
\item If the type of the base expression of an array section is a reference to
      a type \plc{T}, then the type will be considered to be \plc{T} for all 
      purposes of the array section.
\item An array section cannot be used in an overloaded \code{[]} operator.
\end{cppspecific}

\begin{fortranspecific}
\item If a stride expression is specified, it must be positive.
\item The upper bound for the last dimension of an assumed-size dummy
      array must be specified.
\item If a list item is an array section with vector subscripts, the
      first array element must be the lowest in the array element order of
      the array section.
\item If a list item is an array section, the last \plc{part-ref} of the list
      item must have a section subscript list.
\end{fortranspecific}
\end{itemize}



\subsection{Iterators}
\index{iterators}
\label{subsec:iterators}

Iterators are identifiers that expand to multiple values in the clause on which 
they appear.

The syntax of the \code{iterator} modifier is as follows:
\begin{ompSyntax}
iterator(\plc{iterators-definition})
\end{ompSyntax}

where \plc{iterators-definition} is one of the following:
\begin{indentedcodelist}
\plc{iterator-specifier [}, \plc{iterators-definition ]}
\end{indentedcodelist}

where \plc{iterator-specifier} is one of the following:
\begin{indentedcodelist}
\plc{[ iterator-type ] } \plc{identifier} = \plc{range-specification}
\end{indentedcodelist}

where:
\begin{itemize}
\item \plc{identifier} is a base language identifier.

\begin{ccppspecific}
\item \plc{iterator-type} is a type name.
\end{ccppspecific}

\begin{fortranspecific}
\item \plc{iterator-type} is a type specifier.
\end{fortranspecific}
\item \plc{range-specification} is of the form 
\plc{begin}\code{:}\plc{end[}\code{:}\plc{step]}, where \plc{begin} and 
\plc{end} are expressions for which their types can be converted to 
\plc{iterator-type} and \plc{step} is an integral expression.
\end{itemize}

\begin{ccppspecific}
In an \plc{iterator-specifier}, if the \plc{iterator-type} is not specified then the type of that iterator is of \code{int} type.
\end{ccppspecific}

\begin{fortranspecific}
In an \plc{iterator-specifier}, if the \plc{iterator-type} is not specified then the type of that iterator is default integer.
\end{fortranspecific}

In a \plc{range-specification}, if the \plc{step} is not specified its value is
implicitly defined to be 1.

An iterator only exists in the context of the clause in which it appears. An
iterator also hides all accessible symbols with the same name in the context of
the clause.

The use of a variable in an expression that appears in the
\plc{range-specification} causes an implicit reference to the variable in all
enclosing constructs.

\begin{ccppspecific}
The values of the iterator are the set of values $i_{0}$,~\ldots,~$i_{N-1}$ where:
\begin{itemize}
\item $i_{0}$~$=$~$(\plc{iterator-type})$~$begin$, 
\item $i_{j}$~$=$~$(\plc{iterator-type})$~$(i_{j-1}$~$+$~$step)$, and
\item  if $step$~$>$~$0$,
\begin{itemize}
\item $i_{0}$~$<$~$(\plc{iterator-type})$~$end$,
\item $i_{N-1}$~$<$~$(\plc{iterator-type})$~$end$, and 
\item $(\plc{iterator-type})$~$(i_{N-1}$~$+$~$step)$~$\geq$~$(\plc{iterator-type})$~$end$;
\end{itemize}
\item if $step$~$<$~$0$,
\begin{itemize}
\item $i_{0}$~$>$~$(\plc{iterator-type})$~$end$,
\item $i_{N-1}$~$>$~$(\plc{iterator-type})$~$end$, and 
\item $(\plc{iterator-type})$~$(i_{N-1}$~$+$~$step)$~$\leq$~$(\plc{iterator-type})$~$end$.
\end{itemize}
\end{itemize}
\end{ccppspecific}

\begin{fortranspecific}
The values of the iterator are the set of values $i_{1}$,~\ldots,~$i_{N}$ where:
\begin{itemize}
\item $i_{1}$~$=$~$begin$,  
\item $i_{j}$~$=$~$i_{j-1}$~$+$~$step$, and
\item if $step$~$>$~$0$,
\begin{itemize}
\item $i_{1}$~$\leq$~$end$,
\item $i_{N}$~$\leq$~$end$, and 
\item $i_{N}$~$+$~$step$~$>$~$end$;
\end{itemize}
\item if $step$~$<$~$0$,
\begin{itemize}
\item $i_{1}$~$\geq$~$end$,
\item $i_{N}$~$\geq$~$end$, and 
\item $i_{N}$~$+$~$step$~$<$~$end$.
\end{itemize}
\end{itemize}
\end{fortranspecific}

The set of values will be empty if no possible value complies with the 
conditions above.

For those clauses that contain expressions that contain iterator identifiers, the
effect is as if the list item is instantiated within the clause for each value
of the iterator in the set defined above, substituting each occurrence of the
iterator identifier in the expression with the iterator value. If the set of
values of the iterator is empty then the effect is as if the clause was not
specified.

The behavior is unspecified if $i_{j}$~$+$~$step$ cannot be represented in 
\plc{iterator-type} in any of the $i_{j}$~$+$~$step$ computations for any 
$0$~$\leq$~$j$~$<$~$N$ in C/C++ or $0$~$<$~$j$~$\leq$~$N$ in Fortran.

\restrictions

\begin{itemize}
\item An expression that contains an iterator identifier can only appear in 
      clauses that explicitly allow expressions that contain iterators.
\item The \plc{iterator-type} must not declare a new type.
\begin{ccppspecific}
\item The \plc{iterator-type} must be an integral or pointer type.
\item The \plc{iterator-type} must not be \code{const} qualified.
\end{ccppspecific}
\begin{fortranspecific}
\item The \plc{iterator-type} must be an integer type.
\end{fortranspecific}
\item If the \plc{step} expression of a \plc{range-specification} equals zero, 
      the behavior is unspecified.
\item Each iterator identifier can only be defined once in an 
      \plc{iterators-definition}.
\item Iterators cannot appear in the \plc{range-specification}.
\end{itemize}



\section{Conditional Compilation}
\label{sec:Conditional Compilation}
\index{conditional compilation}
\index{_OPENMP@{\code{_OPENMP} macro}}

In implementations that support a preprocessor, the \code{_OPENMP} macro name is 
defined to have the decimal value \plc{yyyymm} where \plc{yyyy} and \plc{mm} are 
the year and month designations of the version of the OpenMP API that the 
implementation supports.

If a \pcode{\#define} or a \pcode{\#undef} preprocessing directive in user
code defines or undefines the \code{_OPENMP} macro name, the behavior is
unspecified.



\begin{fortranspecific}
The OpenMP API requires Fortran lines to be compiled conditionally, as described in
the following sections.



\subsection{Fixed Source Form Conditional Compilation Sentinels}
\label{subsec:Fixed Source Form Conditional Compilation Sentinels}
\index{fixed source form conditional compilation sentinels}
\index{compilation sentinels}

The following conditional compilation sentinels are recognized in fixed form source
files:

\begin{ompfPragma}
!$ \textnormal{|} *$ \textnormal{|} c$
\end{ompfPragma} %$ close off misinterpreted dollar sign/math symbol

To enable conditional compilation, a line with a conditional compilation sentinel must
satisfy the following criteria:

\begin{itemize}
\item The sentinel must start in column 1 and appear as a single word with no 
      intervening white space;

\item After the sentinel is replaced with two spaces, initial lines must have a 
      space or zero in column 6 and only white space and numbers in columns 1 
      through 5;

\item After the sentinel is replaced with two spaces, continuation lines must have 
      a character other than a space or zero in column 6 and only white space in 
      columns 1 through 5.
\end{itemize}

If these criteria are met, the sentinel is replaced by two spaces. If these criteria 
are not met, the line is left unchanged.

\begin{note}
In the following example, the two forms for specifying conditional compilation
in fixed source form are equivalent (the first line represents the position of 
the first 9 columns):

\begin{ompfPragma}
c23456789
!$ 10 iam = omp_get_thread_num() +
!$   &          index

#ifdef _OPENMP
   10 iam = omp_get_thread_num() +
     &            index
#endif
\end{ompfPragma}
\end{note}



\subsection{Free Source Form Conditional Compilation Sentinel}
\label{subsec:Free Source Form Conditional Compilation Sentinel}
\index{free source form conditional compilation sentinel}
\index{compilation sentinels}
The following conditional compilation sentinel is recognized in free form source files:

\begin{ompfPragma}
!$
\end{ompfPragma} %$ close off misinterpreted dollar sign/math symbol

To enable conditional compilation, a line with a conditional compilation sentinel must
satisfy the following criteria:

\begin{itemize}
\item The sentinel can appear in any column but must be preceded only by white space;
\item The sentinel must appear as a single word with no intervening white space;
\item Initial lines must have a space after the sentinel;
\item Continued lines must have an ampersand as the last non-blank character on 
      the line, prior to any comment appearing on the conditionally compiled line. 
\end{itemize}

Continuation lines can have an ampersand after the sentinel, with optional white 
space before and after the ampersand. If these criteria are met, the sentinel is 
replaced by two spaces. If these criteria are not met, the line is left unchanged.

\begin{note}
In the following example, the two forms for specifying conditional compilation
in free source form are equivalent (the first line represents the position of 
the first 9 columns):

\begin{ompfPragma}
c23456789
 !$ iam = omp_get_thread_num() +     &
 !$&    index

#ifdef _OPENMP
    iam = omp_get_thread_num() +     &
        index
#endif
\end{ompfPragma}
\end{note}
\bigskip
\end{fortranspecific}



\section{Variant Directives}
\label{sec:Variant Directives}
\index{variant directives}
\index{directives!variant directives}

\subsection{OpenMP Context}
\label{subsec:OpenMP Context}

At any point in a program, an OpenMP context exists that defines traits
that describe the active OpenMP constructs, the execution devices, and
functionality supported by the implementation. The traits are grouped into
trait sets. The following trait sets exist: \plc{construct}, \plc{device} and
\plc{implementation}.

The \plc{construct} set is composed of the directive names, each being a
trait, of all enclosing constructs at that point in the program up
to a \code{target} construct. Combined and composite constructs are added
to the set as distinct constructs in the same nesting order specified by
the original construct. The set is ordered by nesting level in ascending order.
Specifically, the ordering of the set of constructs is
$c_{1}$,~\ldots,~$c_{N}$, where $c_{1}$ is the construct at the 
outermost nesting level and $c_{N}$ is the construct at the innermost
nesting level. In addition, if the point in the program is not enclosed by
a \code{target} construct, the following rules are applied in order:

\begin{enumerate}
\item For functions with a \code{declare}~\code{simd} directive, the \plc{simd} trait
      is added to the beginning of the set as $c_{1}$ for any generated SIMD
      versions so the total size of the set is increased by 1.
\item For functions that are determined to be function variants by a
      \code{declare}~\code{variant} directive, the selectors 
      $c_{1}$,~\ldots,~$c_{M}$ of the \code{construct} selector set are added
      in the same order to the beginning of the set as $c_{1}$,~\ldots,~$c_{M}$
      so the total size of the set is increased by $M$.
\item For functions within a \code{declare}~\code{target} block, the \plc{target}
      trait is added to the beginning of the set as $c_{1}$ for any versions of 
      the function that are generated for \code{target} regions so the total 
      size of the set is increased by 1.
\end{enumerate}

The \plc{simd} trait can be further defined with properties that match the
clauses accepted by the \code{declare}~\code{simd} directive with the same
name and semantics. The \plc{simd} trait must define at least the
\plc{simdlen} property and one of  the \plc{inbranch} or \plc{notinbranch} properties.

The \plc{device} set includes traits that define the characteristics of the 
device being targeted by the compiler at that point in the program. At least 
the following traits must be defined:

\begin{itemize}
\item The \plc{kind(kind-name-list)} trait specifies the general kind of the 
      device. The following \plc{kind-name} values are defined:

\begin{itemize}
\item \plc{host}, which specifies that the device is the host device;
\item \plc{nohost}, which specifies that the devices is not the host device; and
\item the values defined in the ``OpenMP Context Definitions'' document,
      which is available at \url{http://www.openmp.org/}.
\end{itemize}

\item The \plc{isa(isa-name-list)} trait specifies the Instruction Set 
      Architectures supported by the device. The accepted \plc{isa-name} 
      values are implementation defined.
\item The \plc{arch(arch-name-list)} trait specifies the architectures 
     supported by the device. The accepted \plc{arch-name} values are 
      implementation defined.
\end{itemize}

The \plc{implementation} set includes traits that describe the functionality 
supported by the OpenMP implementation at that point in the program. At least 
the following traits can be defined:

\begin{itemize}
\item The \plc{vendor(vendor-name-list)} trait, which specifies the vendor
      identifiers of the implementation. OpenMP defined values for \plc{vendor-name}
      are defined in the ``OpenMP Context Definitions'' document, which is
      available at \url{http://www.openmp.org/}.
\item The \plc{extension(extension-name-list)} trait, which specifies vendor
      specific extensions to the OpenMP specification. The accepted
      \plc{extension-name} values are implementation defined.
\item A trait with a name that is identical to the name of any clause that can be
      supplied to the \code{requires} directive.
\end{itemize}

Implementations can define further traits in the \plc{device} and \plc{implementation}
sets. All implementation defined traits must follow the following syntax:

\begin{ompSyntax}
\plc{identifier[}(\plc{context-element[}, \plc{context-element[}, \plc{...]]})\plc{]}

\plc{context-element}:
  \plc{identifier[}(\plc{context-element[}, \plc{context-element[}, \plc{...]]})\plc{]}
  or
  \plc{context-value}

\plc{context-value}:
  \plc{constant string}
  or
  \plc{constant integer expression}
\end{ompSyntax}

where \plc{identifier} is a base language identifier.

\subsection{Context Selectors}
\label{subsec:Context Selectors}

Context selectors are used to define the properties of an OpenMP context that
a directive or clause can match. OpenMP defines different sets of selectors, 
each containing different selectors.

The syntax to define a \plc{context-selector-specification} is the following:

\begin{ompSyntax}
\plc{trait-set-selector[},\plc{trait-set-selector[},\plc{...]]}

\plc{trait-set-selector}:
   \plc{trait-set-selector-name}={\plc{trait-selector[}, \plc{trait-selector[}, \plc{...]]}}

\plc{trait-selector}:
   \plc{trait-selector-name[}(\plc{[trait-score}: \plc{]} \plc{trait-property[}, \plc{trait-property[}, \plc{...]]})\plc{]}

\plc{trait-score}:
    score(\plc{score-expression})
\end{ompSyntax}

The \code{construct} selector set defines the \plc{construct} traits that should
be active in the OpenMP context. The following selectors can be defined in the
\code{construct} set: \code{target}; \code{teams}; \code{parallel}; \code{for}
(in C/C++); \code{do} (in Fortran); and \code{simd}. The properties of each
selector are the same properties that are defined for the corresponding trait.
The \code{construct} selector is an ordered list $c_{1}$,~\ldots,~$c_{N}$.

The \code{device} and \code{implementation} selector sets define the traits that
should be active in the corresponding trait set of the OpenMP context. The
same traits defined in the corresponding traits sets can be used as selectors
with the same properties. The \code{kind} selector of the \code{device}
selector set can also be set to the value \code{any}, which is as if no
\code{kind} selector was specified.

The \code{user} selector set defines the \code{condition} selector that provides 
additional user-defined conditions.

\begin{cspecific}
The \code{condition(}\plc{boolean-expr}\code{)} selector defines a \plc{constant 
expression} that must evaluate to true for the selector to be true.
\end{cspecific}

\begin{cppspecific}
The \code{condition(}\plc{boolean-expr}\code{)} selector defines a \code{constexpr} 
expression that must evaluate to true for the selector to be true.
\end{cppspecific}

\begin{fortranspecific}
The \code{condition(}\plc{logical-expr}\code{)} selector defines a \plc{constant 
expression} that must evaluate to true for the selector to be true.
\end{fortranspecific}

A \plc{score-expression} must be an constant integer expression.

Implementations can allow further selectors to be specified. Implementations can 
ignore specified selectors that are not those described in this section.

\restrictions
\begin{itemize}
\item Each \plc{trait-set-selector-name} can only be specified once.
\item Each \plc{trait-selector-name} can only be specified once.
\item A \plc{trait-score} cannot be specified in traits from the \code{construct} 
      or \code{device} \plc{trait-selector-sets}.
\end{itemize}



\subsection{Matching and Scoring Context Selectors}
\label{subsec:Matching and Scoring Context Selectors}

A given context selector is compatible with a given OpenMP context if the
following conditions are satisfied:

\begin{itemize}
\item All selectors in the \code{user} set of the context selector are true;
\item All selectors in the \code{construct}, \code{device}, and \code{implementation} 
      sets of the context selector appear in the corresponding trait set of the 
      OpenMP context;
\item For each selector in the context selector, its properties are a subset of 
      the properties of the corresponding trait of the OpenMP context; and
\item Selectors in the \code{construct} set of the context selector appear 
      in the same relative order as their corresponding traits in the 
      \plc{construct} trait set of the OpenMP context.
\end{itemize}

Some properties of the \code{simd} selector have special rules to match the 
properties of the \plc{simd} trait:

\begin{itemize}
\item The \code{simdlen(}\plc{N}\code{)} property of the selector matches the
      \plc{simdlen(M)} trait of the OpenMP context if $M \% N$ equals zero; and
\item The \code{aligned(}\plc{list:N}\code{)} property of the selector matches 
      the \plc{aligned(list:M)} trait of the OpenMP context if $N \% M$ equals zero.
\end{itemize}

Among compatible context selectors, a score is computed using the following algorithm:

\begin{enumerate}
\item Each trait that appears in the \plc{construct} trait set in the OpenMP context 
      is given the value $2^{p-1}$ where $p$ is the position of the construct trait,
      $c_{p}$, in the set;
\item The \code{kind}, \code{arch}, and \code{isa} selectors are given the
      values $2^{l}$, $2^{l+1}$ and $2^{l+2}$, respectively, where $l$ is 
      the number of traits in the \plc{construct} set;
\item Traits for which a \plc{trait-score} is specified are given the value specified by the
      \plc{trait-score} \plc{score-expression}; 
\item The values given to any additional selectors allowed by the implementation 
      are implemented defined;
\item Other selectors are given a value of zero; and
\item A context selector that is a strict subset of another context selector
      has a score of zero. For other context selectors, the final score is the
      sum of the values of all specified selectors plus $1$. If the traits
      that correspond to the \code{construct} selectors appear multiple times 
      in the OpenMP context, the highest valued subset of traits that contains 
      all selectors in the same order are used.
\end{enumerate}



\subsection{Metadirectives}
\index{metadirective}
\index{directives!metadirective@{\code{metadirective}}}
\label{subsec:Metadirective Meta-Directive}
\summary
A metadirective is a directive that can specify multiple directive variants
of which one may be conditionally selected to replace the metadirective based
on the enclosing OpenMP context.

\syntax
\begin{ccppspecific}
The syntax of a metadirective takes one of the
following forms:
\begin{ompcPragma}
#pragma omp metadirective \plc{[clause[ [},\plc{] clause] ... ] new-line}
\end{ompcPragma}
or
\begin{ompcPragma}
#pragma omp begin metadirective \plc{[clause[ [},\plc{] clause] ... ] new-line}
    \plc{stmt(s)}
#pragma omp end metadirective
\end{ompcPragma}


\begin{samepage}
where \plc{clause} is one of the following:
\begin{indentedcodelist}
when(\plc{context-selector-specification}: \plc{[directive-variant]})
default(\plc{directive-variant})
\end{indentedcodelist}
\end{samepage}

\end{ccppspecific}

\begin{fortranspecific}
The syntax of a metadirective takes one of the following forms:

\begin{ompfPragma}
!$omp metadirective \plc{[clause[ [},\plc{] clause] ... ]}
\end{ompfPragma} %$ close off misinterpreted dollar sign/math symbol

or

\begin{ompfPragma}
!$omp begin metadirective \plc{[clause[ [},\plc{] clause] ... ]}
    \plc{stmt(s)}
!$omp end metadirective
\end{ompfPragma}

\begin{samepage}
where \plc{clause} is one of the following:

\begin{indentedcodelist}
when(\plc{context-selector-specification}: \plc{[directive-variant]})
default(\plc{directive-variant})
\end{indentedcodelist}
\end{samepage}

\end{fortranspecific}

In the \code{when} clause, \plc{context-selector-specification} specifies a context
selector (see Section~\ref{subsec:Context Selectors}).


In the \code{when} and \code{default} clauses, \plc{directive-variant}
has the following form and specifies a directive variant that specifies an OpenMP
directive with clauses that apply to it.

\begin{indentedcodelist}
\plc{ directive-name [clause[ [},\plc{] clause] ... ]}
\end{indentedcodelist}

\descr
A metadirective is a directive that behaves as if it is either ignored or
replaced by the directive variant specified in one of the \code{when} or
\code{default} clauses that appears on the metadirective.

The OpenMP context for a given metadirective is defined according to
Section~\ref{subsec:OpenMP Context}.  For each \code{when} clause that appears
on a metadirective, the specified directive variant, if present, is a candidate
to replace the metadirective if the corresponding context selector is compatible
with the OpenMP context according to the matching rules defined in
Section~\ref{subsec:Matching and Scoring Context Selectors}.  If only one
compatible context selector specified by a \code{when} clause has the highest
score and it specifies a directive variant, the directive variant will replace
the metadirective. If more than one \code{when} clause specifies a compatible
context selector that has the highest computed score and at least one specifies
a directive variant, the first directive variant specified in the lexical order
of those \code{when} clauses will replace the metadirective.

If no context selector from any \code{when} clause is compatible with the
OpenMP context and a \code{default} clause is present, the directive variant
specified in the \code{default} clause will replace the metadirective.

If a directive variant is not selected to replace a metadirective according
to the above rules, the metadirective has no effect on the execution of the
program.

The \code{begin}~\code{metadirective} directive behaves identically to the
\code{metadirective} directive, except that the directive syntax for the
specified directive variants must accept a paired \code{end}~\plc{directive}.
For any directive variant that is selected to replace the
\code{begin}~\code{metadirective} directive, the
\code{end}~\code{metadirective} directive will be implicitly replaced by its
paired \code{end}~\plc{directive} to demarcate the statements that are
affected by or are associated with the directive variant. If no directive
variant is selected to replace the \code{begin}~\code{metadirective}
directive, its paired \code{end}~\code{metadirective} directive is ignored.

\restrictions
Restrictions to metadirectives are as follows:

\begin{itemize}
\item The directive variant appearing in a \code{when} or \code{default}
      clause must not specify a \code{metadirective},
      \code{begin}~\code{metadirective}, or \code{end}~\code{metadirective}
      directive.
\item The context selector that appears in a \code{when} clause must not
      specify any properties for the \code{simd} selector.
\item Any replacement that occurs for a metadirective must not result in a
      non-conforming OpenMP program.
\item Any directive variant that is specified by a \code{when} or \code{default}
      clause on a \code{begin}~\code{metadirective}
      directive must be an OpenMP directive that has a paired 
      \code{end}~\plc{directive}, and the \code{begin}~\code{metadirective} 
      directive must have a paired \code{end}~\code{metadirective} directive.
\item The \code{default} clause may appear at most once on a metadirective.
\end{itemize}



\subsection{\hcode{declare}~\hcode{variant} Directive}
\index{declare variant@{\code{declare variant}}}
\index{directives!declare variant@{\code{declare variant}}}
\label{subsec:declare variant Directive}
\summary
The \code{declare}~\code{variant} directive declares a specialized variant
of a base function and specifies the context in which that specialized variant
is used. The \code{declare}~\code{variant} directive is a declarative
directive.

\syntax
\begin{ccppspecific}
\begin{samepage}
The syntax of the \code{declare}~\code{variant} directive is as follows:

\begin{ompcPragma}
#pragma omp declare variant(\plc{variant-func-id}) \plc{clause new-line}
\plc{[}#pragma omp declare variant(\plc{variant-func-id}) \plc{clause new-line]}
\plc{[ ... ]}
   \plc{function definition or declaration}
\end{ompcPragma}
\end{samepage}

\begin{samepage}
where \plc{clause} is one of the following{}:

\begin{indentedcodelist}
match(\plc{context-selector-specification})
\end{indentedcodelist}
\end{samepage}

and where \plc{variant-func-id} is the name of a function variant that is
either a base language identifier or, for C++, a \plc{template-id}.
\end{ccppspecific}

\begin{fortranspecific}
The syntax of the \code{declare variant} directive is as follows:

\begin{ompfPragma}
!$omp declare variant(\plc{[base-proc-name}:\plc{]variant-proc-name}) \plc{clause}
\end{ompfPragma} %$ close off misinterpreted dollar sign/math symbol

where \plc{clause} is one of the following{}:

\begin{indentedcodelist}
match(\plc{context-selector-specification})
\end{indentedcodelist}

and where \plc{variant-proc-name} is the name of a function variant that is
a base language identifier.
\end{fortranspecific}

\descr

The \code{declare}~\code{variant} directive declares the
\textit{base function} to have the specified function variant.
The context selector in the \code{match} clause is associated
with the variant.

The OpenMP context for a call to a given base function is defined according 
to Section~\ref{subsec:OpenMP Context}. If the context selector that is 
associated with a declared function variant is compatible with the OpenMP 
context of a call to a base function according to the matching rules defined in
Section~\ref{subsec:Matching and Scoring Context Selectors} then a call to
the variant is a candidate to replace the base function call. For any call
to the base function for which candidate variants exist, the variant with 
the highest score is selected from all compatible variants. If multiple 
variants have the highest score, the selected variant is implementation 
defined. If a compatible variant exists, the call to the base function is 
replaced with a call to the selected variant. If no compatible variants 
exist then the call to the base function is not changed.

Different \code{declare}~\code{variant} directives may be specified for
different declarations of the same base function.

Any differences that the specific OpenMP context requires in the prototype 
of the variant from the base function prototype are implementation defined.

\begin{cppspecific}
The function variant is determined by base language standard name lookup
rules ([basic.lookup]) of \plc{variant-func-id} with arguments that correspond
to the argument types in the base function declaration.

The \plc{variant-func-id} and any expressions inside of the \code{match}
clause are interpreted as if they appeared at the scope of the trailing return
type of the base function.
\end{cppspecific}

\restrictions
Restrictions to the \code{declare variant} directive are as follows:

\begin{itemize}
\item Calling functions that a \code{declare}~\code{variant} directive determined
      to be a function variant directly in an OpenMP context that is different 
      from the one that the \code{construct} selector set of the context selector 
      specifies is non-conforming.
\item If a function is determined to be a function variant through more than one
      \code{declare}~\code{variant} directive then the \code{construct}
      selector set of their context selectors must be the same.

\begin{ccppspecific}
\item If the function has any declarations, then the \code{declare}~\code{variant} 
      directives for any declarations that have one must be equivalent. If the 
      function definition has a \code{declare}~\code{variant}, it must also be 
      equivalent. Otherwise, the result is unspecified.
\end{ccppspecific}

\begin{cppspecific}
\item The \code{declare}~\code{variant} directive cannot be specified
      for a virtual function.
\item The type of the function variant must be compatible with
      the type of the base function after the implementation-defined
      transformation for its OpenMP context.
\end{cppspecific}

\begin{fortranspecific}
\item \plc{base-proc-name} must not be a generic name, procedure pointer, 
      or entry name.
\item If \plc{base-proc-name} is omitted then the \code{declare}~\code{variant} 
      directive must appear in the specification part of a subroutine subprogram 
      or a function subprogram.
\item Any \code{declare}~\code{variant} directive must appear in the specification 
      part of a subroutine, subprogram, function subprogram, or interface body to 
      which it applies.
\item If a \code{declare}~\code{variant} directive is specified in an interface 
      block for a procedure then it must match a \code{declare}~\code{variant} 
      directive in the definition of the procedure.
\item If a procedure is declared via a procedure declaration statement then the 
      procedure \plc{base-proc-name} should appear in the same specification.
\item If a \code{declare}~\code{variant} directive is specified for a procedure 
      name with an explicit interface and a \code{declare}~\code{variant}
      directive is also specified for the definition of the procedure, the two
      \code{declare}~\code{variant} directives must match. Otherwise the result 
      is unspecified.
\end{fortranspecific}
\end{itemize}

\crossreferences
\begin{itemize}
\item OpenMP Context Specification, see \specref{subsec:OpenMP Context}.

\item Context Selectors, see \specref{subsec:Context Selectors}.
\end{itemize}



\section{\hcode{requires} Directive}
\label{sec:requires Directive}
\index{requires@{\code{requires}}}
\index{directives!requires@{\code{requires}}}

\summary The \code{requires} directive specifies the features that an implementation
must provide in order for the code to compile and to execute correctly.
The \code{requires} directive is a declarative directive.

\syntax
\begin{ccppspecific}
  The syntax of the \code{requires} directive is as follows:

\begin{ompcPragma}
  #pragma omp requires \plc{clause[ [ [},\plc{] clause] ... ] new-line}

\end{ompcPragma}

\end{ccppspecific}

\begin{fortranspecific}
  The syntax of the \code{requires} directive is as follows:

\begin{ompfPragma}
!$omp requires \plc{clause[ [ [},\plc{] clause] ... ]}
\end{ompfPragma} %$ close off misinterpreted dollar sign/math symbol

\end{fortranspecific}

Where \plc{clause} is either one of the requirement clauses listed below or a
clause of the form {\scode{ext_}\plc{implementation-defined-requirement}} for an
implementation defined requirement clause.

\begin{indentedcodelist}
reverse_offload
unified_address
unified_shared_memory
atomic_default_mem_order(seq_cst \textnormal{|} acq_rel \textnormal{|} relaxed)
dynamic_allocators
\end{indentedcodelist}

\descr

The \code{requires} directive specifies features that an implementation must
support for correct execution. The behavior that a requirement clause specifies
may override the normal behavior specified elsewhere in this document. Whether 
an implementation supports the feature that a given requirement clause specifies 
is implementation defined.

The \code{requires} directive specifies requirements for the execution of all
code in the current compilation unit.

\begin{note}
Use of this directive makes your code less portable. Users should be aware that not all
devices or implementations support all requirements.
\end{note}

When the \code{reverse_offload} clause appears on a \code{requires} directive, the
implementation guarantees that a \code{target} region, for which the \code{target}
construct specifies a \code{device} clause in which the \code{ancestor} modifier appears,
can execute on the parent device of an enclosing \code{target} region.

When the \code{unified_address} clause appears on a \code{requires}
directive, the implementation guarantees that all devices accessible through
OpenMP API routines and directives use a unified address space. In this
address space, a pointer will always refer to the same location in memory
from all devices accessible through OpenMP.  The pointers returned by
\code{omp_target_alloc} and accessed through \code{use_device_ptr} are
guaranteed to be pointer values that can support pointer arithmetic while
still being native device pointers. The \code{is_device_ptr} clause is not
necessary for device pointers to be translated in \code{target} regions, and
pointers found not present are not set to null but keep their original value.
Memory local to a specific execution context may be exempt from this requirement,
following the restrictions of locality to a given execution context, thread, or
contention group.  Target devices may still have discrete memories and
dereferencing a device pointer on the host device or host pointer on a target
device remains unspecified behavior.

The \code{unified_shared_memory} clause implies the \code{unified_address}
requirement, inheriting all of its behaviors.  Additionally, memory in the
device data environment of any device visible to OpenMP, including but not
limited to the host, is considered part of the device data environment of all
devices accessible through OpenMP except as noted below.  Every device address
allocated through OpenMP device memory routines is a valid host pointer. Memory
local to an execution context as defined in \code{unified_address} above may remain
part of distinct device data environments as long as the execution context is
local to the device containing that environment.

The \code{unified_shared_memory} clause makes the \code{map} clause optional
on \code{target} constructs and the \code{declare}~\code{target}
directive optional for static lifetime variables accessed inside
\code{declare}~\code{target} functions.  Scalar variables are still
firstprivate by default when referenced inside \code{target} constructs.  Values stored into
memory by one device may not be visible to another device until those two
devices synchronize with each other or both devices synchronize with the host.

The \code{atomic_default_mem_order} clause specifies the default memory ordering
behavior for \code{atomic} constructs that must be provided by an
implementation. If the default memory ordering is specified as \code{seq_cst}, all
\code{atomic} constructs on which \plc{memory-order-clause} is not specified
behave as if the \code{seq_cst} clause appears. If the default memory
ordering is specified as \code{relaxed}, all \code{atomic} constructs on which
\plc{memory-order-clause} is not specified behave as if the \code{relaxed}
clause appears.

If the default memory ordering is specified as \code{acq_rel}, \code{atomic}
constructs on which \plc{memory-order-clause} is not specified behave as if
the \code{release} clause appears if the atomic write or atomic update
operation is specified, as if the \code{acquire} clause appears if the atomic
read operation is specified, and as if the \code{acq_rel} clause appears if
the atomic captured update operation is specified.

The \code{dynamic_allocators} clause removes certain restrictions on the use
of memory allocators in \code{target} regions. It makes the
\code{uses_allocators} clause optional on \code{target} constructs for the
purpose of using allocators in the corresponding \code{target} regions. It
allows calls to the \code{omp_init_allocator} and \code{omp_destroy_allocator}
API routines in \code{target} regions. Finally, it allows default allocators
to be used by \code{allocate} directives, \code{allocate} clauses, and
\code{omp_alloc} API routines in \code{target} regions.

Implementers are allowed to include additional implementation defined
requirement clauses.  All implementation defined requirements should begin with
\code{ext_}.  Requirement names that do not start with \code{ext_} are
reserved.

\restrictions
The restrictions for the \code{requires} directive are as follows:

\begin{itemize}
\item Each of the clauses can appear at most once on the directive.
\item At most one \code{requires} directive with \code{atomic_default_mem_order} 
      clause can appear in a single compilation unit.
\item A \code{requires} directive with a \code{unified_address},
      \code{unified_shared_memory}, or \code{reverse_offload} clause must 
      appear lexically before any device constructs or device routines.
\item A \code{requires} directive with any of the following clauses must appear 
      in all \emph{compilation units} of a program that contain device
      constructs or device routines or in none of them:

\begin{itemize}
\item \code{reverse_offload}
\item \code{unified_address}
\item \code{unified_shared_memory}
\end{itemize}

\item The \code{requires} directive with \code{atomic_default_mem_order}
      clause may not appear lexically after any \code{atomic} construct on which
      \plc{memory-order-clause} is not specified.

\begin{cspecific}
\item The \code{requires} directive may only appear at file scope.
\end{cspecific}

\begin{cppspecific}
\item The \code{requires} directive may only appear at file or namespace scope.
\end{cppspecific}
\end{itemize}



\section{Internal Control Variables}
\label{sec:Internal Control Variables}
\index{internal control variables (ICVs)}
\index{ICVs (internal control variables)}

An OpenMP implementation must act as if there are internal control variables (ICVs)
that control the behavior of an OpenMP program. These ICVs store information such as
the number of threads to use for future \code{parallel} regions, the schedule to use 
for worksharing loops and whether nested parallelism is enabled or not. The ICVs are 
given values at various times (described below) during the execution of the program. 
They are initialized by the implementation itself and may be given values through 
OpenMP environment variables and through calls to OpenMP API routines. The program 
can retrieve the values of these ICVs only through OpenMP API routines.

For purposes of exposition, this document refers to the ICVs by certain names, but an
implementation is not required to use these names or to offer any way to access the
variables other than through the ways shown in
\specref{subsec:ICV Initialization}.



\subsection{ICV Descriptions}
\label{subsec:ICV Descriptions}
The following ICVs store values that affect the operation of \code{parallel} regions.

\begin{itemize}
\item \plc{dyn-var} - controls whether dynamic adjustment of the number of threads 
      is enabled for encountered \code{parallel} regions. There is one copy of this 
      ICV per data environment.
\item \plc{nthreads-var} - controls the number of threads requested for encountered 
      \code{parallel} regions. There is one copy of this ICV per data environment.
\item \plc{thread-limit-var} - controls the maximum number of threads participating 
      in the contention group. There is one copy of this ICV per data environment.
\item \plc{max-active-levels-var} - controls the maximum number of nested active 
      \code{parallel} regions. There is one copy of this ICV per device.
\item \plc{place-partition-var} - controls the place partition available to the 
      execution environment for encountered \code{parallel} regions. There is one 
      copy of this ICV per implicit task.
\item \plc{active-levels-var} - the number of nested active \code{parallel} regions 
      that enclose the current task such that all of the \code{parallel} regions are 
      enclosed by the outermost initial task region on the current device. There is 
      one copy of this ICV per data environment.
\item \plc{levels-var} - the number of nested parallel regions that enclose the 
      current task such that all of the \code{parallel} regions are enclosed by the 
      outermost initial task region on the current device. There is one copy of this 
      ICV per data environment.
\item \plc{bind-var} - controls the binding of OpenMP threads to places. When binding 
      is requested, the variable indicates that the execution environment is advised 
      not to move threads between places. The variable can also provide default thread 
      affinity policies. There is one copy of this ICV per data environment.
\end{itemize}

The following ICVs store values that affect the operation of worksharing-loop regions.

\begin{itemize}
\item \plc{run-sched-var} - controls the schedule that is used for 
      worksharing-loop regions when the \code{runtime} schedule kind is
      specified. There is one copy of this ICV per data environment.
\item \plc{def-sched-var} - controls the implementation defined default scheduling 
      of worksharing-loop regions. There is one copy of this ICV per device.
\end{itemize}

The following ICVs store values that affect program execution.

\begin{itemize}
\item \plc{stacksize-var} - controls the stack size for threads that the OpenMP 
      implementation creates. There is one copy of this ICV per device.
\item \plc{wait-policy-var} - controls the desired behavior of waiting threads. 
      There is one copy of this ICV per device.
\item \plc{display-affinity-var} - controls whether to display thread affinity. 
      There is one copy of this ICV for the whole program.
\item \plc{affinity-format-var} - controls the thread affinity format when displaying 
      thread affinity. There is one copy of this ICV per device.
\item \plc{cancel-var} - controls the desired behavior of the \code{cancel} construct 
      and cancellation points. There is one copy of this ICV for the whole program.
\item \plc{default-device-var} - controls the default target device. There is one copy 
      of this ICV per data environment.
\item \plc{target-offload-var} - controls the offloading behavior. There is one copy 
      of this ICV for the whole program.
\item \plc{max-task-priority-var} - controls the maximum priority value that can be 
      specified in the \code{priority} clause of the \code{task} construct. There is 
      one copy of this ICV for the whole program.
\end{itemize}

The following ICVs store values that affect the operation of the OMPT tool interface.

\begin{itemize}
\item \plc{tool-var} - controls whether an OpenMP implementation will try to register 
      a tool. There is one copy of this ICV for the whole program.
\item \plc{tool-libraries-var} - specifies a list of absolute paths to tool libraries 
      for OpenMP devices. There is one copy of this ICV for the whole program.
\end{itemize}

The following ICVs store values that affect the operation of the OMPD tool interface.

\begin{itemize}
\item \plc{debug-var} - controls whether an OpenMP implementation will collect
      information that an OMPD library can access to satisfy requests from
      a tool. There is one copy of this ICV for the whole program.
\end{itemize}

The following ICVs store values that affect default memory allocation.

\begin{itemize}
\item \plc{def-allocator-var} - controls the memory allocator to be used by 
      memory allocation routines, directives and clauses when a memory allocator 
      is not specified by the user. There is one copy of this ICV per implicit task.
\end{itemize}



\subsection{ICV Initialization}
\label{subsec:ICV Initialization}
\index{modifying ICVs}
Table~\ref{tab:ICV Initial Values} shows the ICVs, associated
environment variables, and initial values.

\nolinenumbers
\renewcommand{\arraystretch}{1.5}
\tablefirsthead{%
\hline
\textsf{\textbf{ICV}} & \textsf{\textbf{Environment Variable}} & \textsf{\textbf{Initial value}}\\
\hline\\[-3ex]
}
\tablehead{%
\multicolumn{2}{l}{\small\slshape table continued from previous page}\\
\hline
\textsf{\textbf{ICV}} & \textsf{\textbf{Environment Variable}} & \textsf{\textbf{Initial value}}\\
\hline\\[-3ex]
}
\tabletail{%
\hline\\[-4ex]
\multicolumn{2}{l}{\small\slshape table continued on next page}\\
}
\tablelasttail{\hline}
\tablecaption{ICV Initial Values\label{tab:ICV Initial Values}}
\begin{supertabular}{p{1.4in} p{1.8in} p{1.5in}}
{\splc{dyn-var}}               & {\scode{OMP_DYNAMIC}}           & See description below  \\
{\splc{nthreads-var}}          & {\scode{OMP_NUM_THREADS}}       & Implementation defined \\
{\splc{run-sched-var}}         & {\scode{OMP_SCHEDULE}}          & Implementation defined \\
{\splc{def-sched-var}}         & (none)                          & Implementation defined \\
{\splc{bind-var}}              & {\scode{OMP_PROC_BIND}}         & Implementation defined \\
{\splc{stacksize-var}}         & {\scode{OMP_STACKSIZE}}         & Implementation defined \\
{\splc{wait-policy-var}}       & {\scode{OMP_WAIT_POLICY}}       & Implementation defined \\
{\splc{thread-limit-var}}      & {\scode{OMP_THREAD_LIMIT}}      & Implementation defined \\
{\splc{max-active-levels-var}} & {\scode{OMP_MAX_ACTIVE_LEVELS}}, {\scode{OMP_NESTED}} & See description below\\
{\splc{active-levels-var}}     & (none)                          & {\splc{zero}}          \\
{\splc{levels-var}}            & (none)                          & {\splc{zero}}          \\
{\splc{place-partition-var}}   & {\scode{OMP_PLACES}}            & Implementation defined \\
{\splc{cancel-var}}            & {\scode{OMP_CANCELLATION}}      & {\splc{false}}         \\
{\splc{display-affinity-var}}  & {\scode{OMP_DISPLAY_AFFINITY}}  & {\splc{false}}         \\
{\splc{affinity-format-var}}   & {\scode{OMP_AFFINITY_FORMAT}}   & Implementation defined \\
{\splc{default-device-var}}    & {\scode{OMP_DEFAULT_DEVICE}}    & Implementation defined \\
{\splc{target-offload-var}}    & {\scode{OMP_TARGET_OFFLOAD}}    & {\scode{DEFAULT}}      \\
{\splc{max-task-priority-var}} & {\scode{OMP_MAX_TASK_PRIORITY}} & {\splc{zero}}          \\
{\splc{tool-var}}              & {\scode{OMP_TOOL}}              & {\splc{enabled}}       \\
{\splc{tool-libraries-var}}    & {\scode{OMP_TOOL_LIBRARIES}}    & {\splc{empty string}}  \\
{\splc{debug-var}}             & {\scode{OMP_DEBUG}}             & {\splc{disabled}}      \\
{\splc{def-allocator-var}}     & {\scode{OMP_ALLOCATOR}}         & Implementation defined \\
\end{supertabular}

\linenumbers

\descr

\begin{itemize}
\item Each device has its own ICVs.
\item The initial value of \plc{dyn-var} is implementation defined if the 
      implementation supports dynamic adjustment of the number of threads; 
      otherwise, the initial value is \plc{false}.
\item The value of the \plc{nthreads-var} ICV is a list.
\item The value of the \plc{bind-var} ICV is a list.
\item The initial value of \plc{max-active-levels-var} is the number of 
      active levels of parallelism that the implementation supports if
      \code{OMP_NUM_THREADS} or \code{OMP_PROC_BIND} is set to a comma-separated
      list of more than one value. Otherwise, the initial value of
      \plc{max-active-levels-var} is implementation defined.
\end{itemize}

The host and target device ICVs are initialized before any OpenMP API construct or
OpenMP API routine executes. After the initial values are assigned, the values of any
OpenMP environment variables that were set by the user are read and the associated
ICVs for the host device are modified accordingly. The method for initializing a target
device's ICVs is implementation defined.

\crossreferences
\begin{itemize}
\item \code{OMP_SCHEDULE} environment variable, see \specref{sec:OMP_SCHEDULE}.

\item \code{OMP_NUM_THREADS} environment variable, see \specref{sec:OMP_NUM_THREADS}.

\item \code{OMP_DYNAMIC} environment variable, see \specref{sec:OMP_DYNAMIC}.

\item \code{OMP_PROC_BIND} environment variable, see \specref{sec:OMP_PROC_BIND}.

\item \code{OMP_PLACES} environment variable, see \specref{sec:OMP_PLACES}.

\item \code{OMP_STACKSIZE} environment variable, see \specref{sec:OMP_STACKSIZE}.

\item \code{OMP_WAIT_POLICY} environment variable, see \specref{sec:OMP_WAIT_POLICY}.

\item \code{OMP_MAX_ACTIVE_LEVELS} environment variable, see \specref{sec:OMP_MAX_ACTIVE_LEVELS}.

\item \code{OMP_NESTED} environment variable, see \specref{sec:OMP_NESTED}.

\item \code{OMP_THREAD_LIMIT} environment variable, see \specref{sec:OMP_THREAD_LIMIT}.

\item \code{OMP_CANCELLATION} environment variable, see \specref{sec:OMP_CANCELLATION}.

\item \code{OMP_DISPLAY_AFFINITY} environment variable, see \specref{sec:OMP_DISPLAY_AFFINITY}.

\item \code{OMP_AFFINITY_FORMAT} environment variable, see \specref{sec:OMP_AFFINITY_FORMAT}.

\item \code{OMP_DEFAULT_DEVICE} environment variable, see \specref{sec:OMP_DEFAULT_DEVICE}.

\item \code{OMP_MAX_TASK_PRIORITY} environment variable, see \specref{sec:OMP_MAX_TASK_PRIORITY}.

\item \code{OMP_TARGET_OFFLOAD} environment variable, see \specref{sec:OMP_TARGET_OFFLOAD}.

\item \code{OMP_TOOL} environment variable, see \specref{sec:OMP_TOOL}.

\item \code{OMP_TOOL_LIBRARIES} environment variable, see \specref{sec:OMP_TOOL_LIBRARIES}.

\item \code{OMP_DEBUG} environment variable, see \specref{sec:OMP_DEBUG}.

\item \code{OMP_ALLOCATOR} environment variable, see \specref{sec:OMP_ALLOCATOR}.
\end{itemize}



\subsection{Modifying and Retrieving ICV Values}
\label{subsec:Modifying and Retrieving ICV Values}
\index{modifying and retrieving ICV values}
Table~\ref{tab:Ways to Modify and to Retrieve ICV Values} shows the method 
for modifying and retrieving the values of ICVs through OpenMP API routines.


{\small%
\nolinenumbers
\renewcommand{\arraystretch}{1.5}
\tablefirsthead{%
\hline
\textsf{\textbf{ICV}} & \textsf{\textbf{Ways to Modify Value}} & \textsf{\textbf{Ways to Retrieve Value}}\\
\hline\\[-3ex]
}
\tablehead{%
\multicolumn{2}{l}{\small\slshape table continued from previous page}\\
\hline
\textsf{\textbf{ICV}} & \textsf{\textbf{Ways to Modify Value}} & \textsf{\textbf{Ways to Retrieve Value}}\\
\hline\\[-3ex]
}
\tabletail{%
\hline\\[-4ex]
\multicolumn{2}{l}{\small\slshape table continued on next page}\\
}
\tablelasttail{\hline}
\tablecaption{Ways to Modify and to Retrieve ICV Values\label{tab:Ways to Modify and to Retrieve ICV Values}}
\begin{supertabular}{ p{1.2in} p{2.0in} p{1.5in}}

{\splc{dyn-var}}               & {\scode{omp_set_dynamic()}}           & {\scode{omp_get_dynamic()}}           \\

{\splc{nthreads-var}}          & {\scode{omp_set_num_threads()}}       & {\scode{omp_get_max_threads()}}       \\

{\splc{run-sched-var}}         & {\scode{omp_set_schedule()}}          & {\scode{omp_get_schedule()}}          \\

{\splc{def-sched-var}}         & (none)                                & (none)                                \\

{\splc{bind-var}}              & (none)                                & {\scode{omp_get_proc_bind()}}         \\

{\splc{stacksize-var}}         & (none)                                & (none)                                \\

{\splc{wait-policy-var}}       & (none)                                & (none)                                \\

{\splc{thread-limit-var}}      & {\scode{thread_limit}} clause         & {\scode{omp_get_thread_limit()}}      \\

{\splc{max-active-levels-var}} & {\scode{omp_set_max_active_levels()}}, {\scode{omp_set_nested()}} & {\scode{omp_get_max_active_levels()}}\\

{\splc{active-levels-var}}     & (none)                                & {\scode{omp_get_active_level()}}      \\

{\splc{levels-var}}            & (none)                                & {\scode{omp_get_level()}}             \\

{\splc{place-partition-var}}   & (none)                                & See description below                 \\

{\splc{cancel-var}}            & (none)                                & {\scode{omp_get_cancellation()}}      \\

{\splc{display-affinity-var}}  & (none)                                & (none)                                \\

{\splc{affinity-format-var}}   & {\scode{omp_set_affinity_format()}}   & {\scode{omp_get_affinity_format()}}   \\

{\splc{default-device-var}}    & {\scode{omp_set_default_device()}}    & {\scode{omp_get_default_device()}}    \\

{\splc{target-offload-var}}    & (none)                                & (none)                                \\

{\splc{max-task-priority-var}} & (none)                                & {\scode{omp_get_max_task_priority()}} \\

{\splc{tool-var}}              & (none)                                & (none)                                \\

{\splc{tool-libraries-var}}    & (none)                                & (none)                                \\

{\splc{debug-var}}             & (none)                                & (none)                                \\

{\splc{def-allocator-var}}     & {\scode{omp_set_default_allocator()}} & {\scode{omp_get_default_allocator()}} \\

\end{supertabular}

\linenumbers} % end of \small block

\descr
\begin{itemize}
\item The value of the \plc{nthreads-var} ICV is a list. The runtime call
      \code{omp_set_num_threads} sets the value of the first element of 
      this list, and \code{omp_get_max_threads} retrieves the value of 
      the first element of this list.
\item The value of the \plc{bind-var} ICV is a list. The runtime call 
      \code{omp_get_proc_bind} retrieves the value of the first element of this list.
\item Detailed values in the \plc{place-partition-var} ICV are retrieved 
      using the runtime calls \code{omp_get_partition_num_places}, 
      \code{omp_get_partition_place_nums}, \code{omp_get_place_num_procs}, 
      and \code{omp_get_place_proc_ids}.
\end{itemize}

\crossreferences
\begin{itemize}
\item \code{thread_limit} clause of the \code{teams} construct, see \specref{sec:teams Construct}.

\item \code{omp_set_num_threads} routine, see \specref{subsec:omp_set_num_threads}.

\item \code{omp_get_max_threads} routine, see \specref{subsec:omp_get_max_threads}.

\item \code{omp_set_dynamic} routine, see \specref{subsec:omp_set_dynamic}.

\item \code{omp_get_dynamic} routine, see \specref{subsec:omp_get_dynamic}.

\item \code{omp_get_cancellation} routine, see \specref{subsec:omp_get_cancellation}.

\item \code{omp_set_nested} routine, see \specref{subsec:omp_set_nested}.

\item \code{omp_get_nested} routine, see \specref{subsec:omp_get_nested}.

\item \code{omp_set_schedule} routine, see \specref{subsec:omp_set_schedule}.

\item \code{omp_get_schedule} routine, see \specref{subsec:omp_get_schedule}.

\item \code{omp_get_thread_limit} routine, see \specref{subsec:omp_get_thread_limit}.

\item \code{omp_get_supported_active_levels}, see \specref{subsec:omp_get_supported_active_levels}.

\item \code{omp_set_max_active_levels} routine, see \specref{subsec:omp_set_max_active_levels}.

\item \code{omp_get_max_active_levels} routine, see \specref{subsec:omp_get_max_active_levels}.

\item \code{omp_get_level} routine, see \specref{subsec:omp_get_level}.

\item \code{omp_get_active_level} routine, see \specref{subsec:omp_get_active_level}.

\item \code{omp_get_proc_bind} routine, see \specref{subsec:omp_get_proc_bind}.

\item \code{omp_get_place_num_procs} routine, see \specref{subsec:omp_get_place_num_procs}.

\item \code{omp_get_place_proc_ids} routine, see \specref{subsec:omp_get_place_proc_ids}.

\item \code{omp_get_partition_num_places} routine, see \specref{subsec:omp_get_partition_num_places}.

\item \code{omp_get_partition_place_nums} routine, see \specref{subsec:omp_get_partition_place_nums}.

\item \code{omp_set_affinity_format} routine, see \specref{subsec:omp_set_affinity_format}.

\item \code{omp_get_affinity_format} routine, see \specref{subsec:omp_get_affinity_format}.

\item \code{omp_set_default_device} routine, see \specref{subsec:omp_set_default_device}.

\item \code{omp_get_default_device} routine, see \specref{subsec:omp_get_default_device}.

\item \code{omp_get_max_task_priority} routine, see \specref{subsec:omp_get_max_task_priority}.

\item \code{omp_set_default_allocator} routine, see \specref{subsec:omp_set_default_allocator}.

\item \code{omp_get_default_allocator} routine, see \specref{subsec:omp_get_default_allocator}.
\end{itemize}



\subsection{How ICVs are Scoped}
\label{subsec:How ICVs are Scoped}
Table~\ref{tab:Scopes of ICVs} shows the ICVs and their scope.

\nolinenumbers
\tablefirsthead{%
\hline
\textsf{\textbf{ICV}} & \textsf{\textbf{Scope}}\\
\hline \\[-3ex]
}
\tablehead{%
\multicolumn{2}{l}{\small\slshape table continued from previous page}\\
\hline
\textsf{\textbf{ICV}} & \textsf{\textbf{Scope}}\\
\hline \\[-3ex]
}
\tabletail{%
\hline\\[-4ex]
\multicolumn{2}{l}{\small\slshape table continued on next page}\\
}
\tablelasttail{\hline}
\tablecaption{Scopes of ICVs\label{tab:Scopes of ICVs}}
\begin{supertabular}{p{1.5in} p{2.5in}}
{\splc{dyn-var}}               & data environment\\
{\splc{nthreads-var}}          & data environment\\
{\splc{run-sched-var}}         & data environment\\
{\splc{def-sched-var}}         & device\\
{\splc{bind-var}}              & data environment\\
{\splc{stacksize-var}}         & device\\
{\splc{wait-policy-var}}       & device\\
{\splc{thread-limit-var}}      & data environment\\
{\splc{max-active-levels-var}} & device\\
{\splc{active-levels-var}}     & data environment\\
{\splc{levels-var}}            & data environment\\
{\splc{place-partition-var}}   & implicit task\\
{\splc{cancel-var}}            & global\\
{\splc{display-affinity-var}}  & global \\
{\splc{affinity-format-var}}   & device \\
{\splc{default-device-var}}    & data environment\\
{\splc{target-offload-var}}    & global\\
{\splc{max-task-priority-var}} & global\\
{\splc{tool-var}}              & global\\
{\splc{tool-libraries-var}}    & global\\
{\splc{debug-var}}             & global \\
{\splc{def-allocator-var}}     & implicit task\\
\end{supertabular}

\linenumbers

\descr
\begin{itemize}
\item There is one copy per device of each ICV with device scope.
\item Each data environment has its own copies of ICVs with data environment scope.
\item Each implicit task has its own copy of ICVs with implicit task scope.
\end{itemize}

Calls to OpenMP API routines retrieve or modify data environment scoped ICVs in the
data environment of their binding tasks.



\subsubsection{How the Per-Data Environment ICVs Work}
\label{subsubsec:How the Per-Data Environment ICVs Work}

When a \code{task} construct or \code{parallel} construct is encountered, the 
generated task(s) inherit the values of the data environment scoped ICVs from 
the generating task's ICV values.

When a \code{parallel} construct is encountered, the value of each ICV with
implicit task scope is inherited, unless otherwise specified, from the implicit
binding task of the generating task unless otherwise specified.

When a \code{task} construct is encountered, the generated task inherits the 
value of \plc{nthreads-var} from the generating task's \plc{nthreads-var} value. 
When a \code{parallel} construct is encountered, and the generating task's 
\plc{nthreads-var} list contains a single element, the generated task(s) inherit 
that list as the value of \plc{nthreads-var}. When a \code{parallel} construct is 
encountered, and the generating task's \plc{nthreads-var} list contains multiple 
elements, the generated task(s) inherit the value of \plc{nthreads-var} as the 
list obtained by deletion of the first element from the generating task's 
\plc{nthreads-var} value. The \plc{bind-var} ICV is handled in the same way 
as the \plc{nthreads-var} ICV.

When a \plc{target task} executes a \code{target} region, the generated initial
task uses the values of the data environment scoped ICVs from the device data
environment ICV values of the device that will execute the region.

If a \code{teams} construct with a \code{thread_limit} clause is encountered,
the \plc{thread-limit-var} ICV from the data environment of the initial task for
each team is instead set to a value that is less than or equal to the value
specified in the clause.

When encountering a worksharing-loop region for which the \code{runtime}
schedule kind is specified, all implicit task regions that constitute the 
binding parallel region must have the same value for \plc{run-sched-var} 
in their data environments. Otherwise, the behavior is unspecified.



\subsection{ICV Override Relationships}
\label{subsec:ICV Override Relationships}

Table~\ref{tab:ICV Override Relationships} shows the override relationships
among construct clauses and ICVs.

\nolinenumbers
\renewcommand{\arraystretch}{1.5}
\tablefirsthead{%
\hline
\textsf{\textbf{ICV}} & \textsf{\textbf{construct clause, if used}}\\
\hline\\[-3ex]
}
\tablehead{%
\multicolumn{2}{l}{\small\slshape table continued from previous page}\\
\hline
\textsf{\textbf{ICV}} & \textsf{\textbf{construct clause, if used}}\\
\hline\\[-3ex]
}
\tabletail{%
\hline\\[-4ex]
\multicolumn{2}{l}{\small\slshape table continued on next page}\\
}
\tablelasttail{\hline}
\tablecaption{ICV Override Relationships\label{tab:ICV Override Relationships}}
\begin{supertabular}{ p{1.3in} p{2.0in}}
{\splc{dyn-var}}               & (none)\\
{\splc{nthreads-var}}          & {\scode{num_threads}}\\
{\splc{run-sched-var}}         & {\scode{schedule}}\\
{\splc{def-sched-var}}         & {\scode{schedule}}\\
{\splc{bind-var}}              & {\scode{proc_bind}}\\
{\splc{stacksize-var}}         & (none)\\
{\splc{wait-policy-var}}       & (none)\\
{\splc{thread-limit-var}}      & (none)\\
{\splc{max-active-levels-var}} & (none)\\
{\splc{active-levels-var}}     & (none)\\
{\splc{levels-var}}            & (none)\\
{\splc{place-partition-var}}   & (none)\\
{\splc{cancel-var}}            & (none)\\
{\splc{display-affinity-var}}  & (none) \\
{\splc{affinity-format-var}}   & (none) \\
{\splc{default-device-var}}    & (none)\\
{\splc{target-offload-var}}    & (none)\\
{\splc{max-task-priority-var}} & (none)\\
{\splc{tool-var}}              & (none)\\
{\splc{tool-libraries-var}}    & (none)\\
{\splc{debug-var}}             & (none) \\
{\splc{def-allocator-var}}     & {\scode{allocator}}\\
\end{supertabular}

\linenumbers

\descr
\begin{itemize}
\item The \code{num_threads} clause overrides the value of the first element of the
      \plc{nthreads-var} ICV.
\item If a \code{schedule} clause specifies a modifier then that modifier overrides 
      any modifier that is specified in the \plc{run-sched-var} ICV.
\item If \plc{bind-var} is not set to \plc{false} then the \code{proc_bind} clause 
      overrides the value of the first element of the \plc{bind-var} ICV; otherwise, 
      the \code{proc_bind} clause has no effect.
\end{itemize}

\crossreferences
\begin{itemize}
\item \code{parallel} construct, see
\specref{sec:parallel Construct}.

\item \code{proc_bind} clause,
\specref{sec:parallel Construct}.

\item \code{num_threads} clause, see
\specref{subsec:Determining the Number of Threads for a parallel Region}.

\item Worksharing-Loop construct, see
\specref{subsec:Worksharing-Loop Construct}.

\item \code{schedule} clause, see
\specref{subsubsec:Determining the Schedule of a Worksharing-Loop}.
\end{itemize}
