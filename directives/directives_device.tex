% This is an included file. See the master file for more information.
%
% When editing this file:
%
%    1. To change formatting, appearance, or style, please edit openmp.sty.
%
%    2. Custom commands and macros are defined in openmp.sty.
%
%    3. Be kind to other editors -- keep a consistent style by copying-and-pasting to
%       create new content.
%
%    4. We use semantic markup, e.g. (see openmp.sty for a full list):
%         \code{}     % for bold monospace keywords, code, operators, etc.
%         \plc{}      % for italic placeholder names, grammar, etc.
%
%    5. There are environments that provide special formatting, e.g. language bars.
%       Please use them whereever appropriate.  Examples are:
%
%         \begin{fortranspecific}
%         This is text that appears enclosed in blue language bars for Fortran.
%         \end{fortranspecific}
%
%         \begin{note}
%         This is a note.  The "Note -- " header appears automatically.
%         \end{note}
%
%    6. Other recommendations:
%         Use the convenience macros defined in openmp.sty for the minor headers
%         such as Comments, Syntax, etc.
%
%         To keep items together on the same page, prefer the use of 
%         \begin{samepage}.... Avoid \parbox for text blocks as it interrupts line numbering.
%         When possible, avoid \filbreak, \pagebreak, \newpage, \clearpage unless that's
%         what you mean. Use \needspace{} cautiously for troublesome paragraphs.
%
%         Avoid absolute lengths and measures in this file; use relative units when possible.
%         Vertical space can be relative to \baselineskip or ex units. Horizontal space
%         can be relative to \linewidth or em units.
%
%         Prefer \emph{} to italicize terminology, e.g.:
%             This is a \emph{definition}, not a placeholder.
%             This is a \plc{var-name}.
%


\section{Device Constructs}
\label{sec:Device Constructs}
\index{device constructs}
\index{constructs!device constructs}
\index{device constructs!device constructs}

\subsection{Device Initialization}

\events

The \plc{device-initialize} event occurs in a thread that encounters
the first \code{target}, \code{target data}, or \code{target enter
data} construct associated with a particular target device 
after the thread initiates initialization of OpenMP on the device and the
device's OpenMP initialization, which may include device-side tool
initialization, completes. 

The \plc{device-load} event for a code block for a target device occurs in some thread 
before any thread executes code from that code block on that target device.

The \plc{device-unload} event for a target device occurs in some thread 
whenever a code block is unloaded from the device.

The \plc{device-finalize} event for a target device that has been initialized
occurs in some thread before an OpenMP implementation shuts down.

\tools 

A thread dispatches a registered \code{ompt\_callback\_device\_initialize}
callback for each occurrence of a \plc{device-initialize} event in
that thread.  This callback has type signature
\code{ompt\_callback\_device\_initialize\_t}.

A thread dispatches a registered \code{ompt\_callback\_device\_load}
callback for each occurrence of a \plc{device-load} event in
that thread.  This callback has type signature
\code{ompt\_callback\_device\_load\_t}.

A thread dispatches a registered \code{ompt\_callback\_device\_unload}
callback for each occurrence of a \plc{device-unload} event in
that thread.  This callback has type signature
\code{ompt\_callback\_device\_unload\_t}.

A thread dispatches a registered \code{ompt\_callback\_device\_finalize}
callback for each occurrence of a \plc{device-finalize} event in
that thread.  This callback has type signature
\code{ompt\_callback\_device\_finalize\_t}.

\restrictions
No thread may offload execution of an OpenMP construct to a device until a 
dispatched \code{ompt\_callback\_device\_initialize} callback completes.

No thread may offload execution of an OpenMP construct to a device after a 
dispatched \code{ompt\_callback\_device\_finalize} callback occurs.

\crossreferences
\begin{itemize}
\item \code{ompt\_callback\_device\_initialize\_t}, see
\specref{sec:ompt_callback_device_initialize_t}.
\item \code{ompt\_callback\_device\_load\_t}, see
\specref{sec:ompt_callback_device_load_t}.
\item \code{ompt\_callback\_device\_unload\_t}, see
\specref{sec:ompt_callback_device_unload_t}.
\item \code{ompt\_callback\_device\_finalize\_t}, see
\specref{sec:ompt_callback_device_finalize_t}.
\end{itemize}


\subsection{\code{target}~\code{data} Construct}
\index{target data@{\code{target}~\code{data}}}
\index{constructs!target data@{\code{target}~\code{data}}}
\label{subsec:target data Construct}
\summary
 Map variables to a device data environment for the extent of the region.
\syntax
\begin{ccppspecific}
The syntax of the \code{target}~\code{data} construct is as follows:

\begin{boxedcode}
\#pragma omp target data \plc{clause[ [ [},\plc{] clause] ... ] new-line}
    \plc{structured-block}
\end{boxedcode}

\needspace{10\baselineskip}
\begin{samepage}
where \plc{clause} is one of the following:

\begin{indentedcodelist}
if(\plc{[} target data :\plc{] scalar-expression})
device(\plc{integer-expression})
map(\plc{[[map-type-modifier[,]] map-type}:\plc{ ] list})
use\_device\_ptr(\plc{list})
\end{indentedcodelist}
\end{samepage}
\end{ccppspecific}
\medskip

\begin{fortranspecific}
The syntax of the \code{target}~\code{data} construct is as follows:

\begin{boxedcode}
!\$omp target data \plc{clause[ [ [},\plc{] clause] ... ]}
    \plc{structured-block}
!\$omp end target data
\end{boxedcode}

where \plc{clause} is one of the following:

\begin{indentedcodelist}
if(\plc{[} target data :\plc{] scalar-logical-expression})
device(\plc{scalar-integer-expression})
map(\plc{[[map-type-modifier[,]] map-type}:\plc{ ] list})
use\_device\_ptr(\plc{list})
\end{indentedcodelist}

The \code{end}~\code{target}~\code{data} directive denotes the end of the \code{target}~\code{data} construct.
\end{fortranspecific}

\binding
The binding task set for a \code{target}~\code{data} region is the generating task. The 
\code{target}~\code{data} region binds to the region of the generating task.

\descr
When a \code{target}~\code{data} construct is encountered, the encountering task executes the region. If there is no \code{device} clause, the default device is determined by the \plc{default-device-var} ICV. Variables are mapped for the extent of the region, according to any data-mapping attribute clauses, from the data environment of the encountering task to the device data environment. When an \code{if} clause is present and the \code{if} clause expression evaluates to \plc{false}, the device is the host.

List items that appear in a \code{use\_device\_ptr} clause are converted into
device pointers to the corresponding list items in the device data environment.
If a \code{use\_device\_ptr} clause and one or more \code{map} clauses are
present on the same construct, this conversion will occur as if performed after
all variables are mapped according to those \code{map} clauses.

\def\omptTargetRegionEvents#1#2#3
{
The \plc{#3{-begin}} event occurs when a thread enters a
\code{#1} region.

The \plc{#3{-end}} event occurs when a thread exits a
\code{#1} region.
}

\def\omptTargetRegionTools#1#2#3#4
{
A thread dispatches a registered \code{ompt\_callback\_target}
callback for each occurrence of a \plc{#3{-begin}} and
\plc{#3{-end}} event in that thread in the context of #4.
The callback has type signature
\code{ompt\_callback\_target\_t}. The callback receives
\code{ompt\_scope\_begin} or \code{ompt\_scope\_end} 
as its \plc{endpoint} argument, as appropriate, and 
\code{#2} as its \plc{kind} argument.
}

\def\omptTargetRegion#1#2#3#4
{
\events
\omptTargetRegionEvents{#1}{#2}{#3}

\tools
\omptTargetRegionTools{#1}{#2}{#3}{#4}
}
\omptTargetRegion{\code{target}~\code{data}}{ompt\_target\_enter\_data}{target-data}{the task encountering the construct}

\restrictions
\begin{itemize}
    \item A program must not depend on any ordering of the evaluations of the clauses of the
    \code{target}~\code{data} directive, or on any side effects of the evaluations of the clauses.

  \item At most one \code{device} clause can appear on the directive. The \code{device} expression
    must evaluate to a non-negative integer value less than the value
    of \code{omp\_get\_num\_devices()}.

  \item At most one \code{if} clause can appear on the directive.
  \item A \plc{map-type} in a \code{map} clause must be \code{to}, \code{from}, \code{tofrom} or \code{alloc}.
  \item At least one \code{map} or \code{use\_device\_ptr} clause must appear on the directive.


  \item A list item in a \code{use\_device\_ptr} clause must have a
    corresponding list item in the device data environment.

  \item A list item that specifies a given variable may not appear in more than
	  one \code{use\_device\_ptr} clause.

  \item References in the construct to a list item that appears in a
    \code{use\_device\_ptr} clause must be to the address of the list item.

\end{itemize}


\crossreferences
\begin{itemize}
\item \plc{default-device-var}, see 
\specref{sec:Internal Control Variables}. 

\item \code{if} Clause, see \specref{sec:if Clause}.

\item \code{map} clause, see 
\specref{subsec:map Clause}.

\item \code{omp\_get\_num\_devices} routine, see \specref{subsec:omp_get_num_devices}

\item \code{ompt\_callback\_target\_t}, see
\specref{sec:ompt_callback_target_t}.

\end{itemize}










\subsection{\code{target}~\code{enter}~\code{data} Construct}
\label{subsec:target enter data Construct}
\index{constructs!target enter data@{\code{target}~\code{enter}~\code{data}}}
\index{device data environments}
\summary
The \code{target}~\code{enter}~\code{data} directive specifies that variables are mapped to a device data environment. The \code{target}~\code{enter}~\code{data} directive is a stand-alone directive.
\syntax
\begin{ccppspecific}
The syntax of the \code{target}~\code{enter}~\code{data} construct is as follows:
\begin{boxedcode}
\#pragma omp target enter data \plc{[ clause[ [,] clause]...] new-line}
\end{boxedcode}
where \plc{clause} is one of the following:
\begin{indentedcodelist}
if(\plc{[} target enter data :\plc{] scalar-expression})
device(\plc{integer-expression})
map(\plc{[ [map-type-modifier[,]] map-type} : \plc{] list})
depend(\plc{dependence-type} : \plc{locator-list}[:\plc{iterators-definition}])
nowait
\end{indentedcodelist}
\end{ccppspecific}
\begin{fortranspecific}
The syntax of the \code{target}~\code{enter}~\code{data} is as follows:
\begin{boxedcode}
!\$omp target enter data \plc{[ clause[ [,] clause]...]}
\end{boxedcode}
where clause is one of the following:
\begin{indentedcodelist}
if(\plc{[} target enter data :\plc{] scalar-logical-expression})
device(\plc{scalar-integer-expression})
map(\plc{[ [map-type-modifier[,]] map-type} : \plc{] list})
depend(\plc{dependence-type} : \plc{locator-list}[:\plc{iterators-definition}])
nowait
\end{indentedcodelist}
\end{fortranspecific}

\binding
The binding task set for a \code{target}~\code{enter}~\code{data} region is
the generating task, which is the \plc{target task} generated by the
\code{target}~\code{enter}~\code{data} construct. The
\code{target}~\code{enter}~\code{data} region binds to the corresponding
\plc{target task} region.

\descr
When a \code{target}~\code{enter}~\code{data} construct is encountered, the list items are mapped to the device data environment according to the \code{map} clause semantics.

The \code{target}~\code{enter}~\code{data} construct is a task generating construct.  The generated task is a \plc{target task}.  The generated task region encloses the \code{target}~\code{enter}~\code{data} region.

All clauses are evaluated when the \code{target}~\code{enter}~\code{data} construct is encountered.  The data environment of the \plc{target task} is created according to the data-sharing attribute clauses on the \code{target}~\code{enter}~\code{data} construct, per-data environment ICVs, and any default data-sharing attribute rules that apply to the \code{target}~\code{enter}~\code{data} construct.  A variable that is mapped in the \code{target}~\code{enter}~\code{data} construct has a default data-sharing attribute of shared in the data environment of the \plc{target task}.

Assignment operations associated with mapping a variable (see \specref{subsec:map Clause}) occur when the \plc{target task} executes.

If the \code{nowait} clause is present, execution of the \plc{target task} may be deferred.  If the \code{nowait} clause is not present, the \plc{target task} is an included task.

If a \code{depend} clause is present, it is associated with the \plc{target task}.

If there is no \code{device} clause, the default device is determined by the \plc{default-device-var} ICV.

When an \code{if} clause is present and the \code{if} clause expression evaluates to \plc{false}, the device is the host. 


%\tools 
\def\omptTarget#1#2#3#4
{

\events
Events associated with a \plc{target task} are the same as for the \code{task} construct defined in \specref{subsec:task Construct}.

\omptTargetRegionEvents{#1}{#2}{#3}

\tools
Callbacks associated with events for \plc{target tasks} are the same as for the \code{task} construct defined in \specref{subsec:task Construct}.

\omptTargetRegionTools{#1}{#2}{#3}{#4}
}
\omptTarget{\code{target}~\code{enter}~\code{data}}{ompt\_target\_enter\_data}{target-enter-data}{the target task on the host}



\restrictions
\begin{itemize}
\item A program must not depend on any ordering of the evaluations of the clauses of the \code{target}~\code{enter}~\code{data} directive, or on any side effects of the evaluations of the clauses.
\item At least one \code{map} clause must appear on the directive.

\item At most one \code{device} clause can appear on the
  directive. The \code{device} expression must evaluate to a
  non-negative integer value less than the value of
  \code{omp\_get\_num\_devices()}.

\item At most one \code{if} clause can appear on the directive.
\item A \plc{map-type} must be specified in all \code{map} clauses and must be either \code{to} or \code{alloc}.
\item At most one \code{nowait} clause can appear on the directive.
\end{itemize}

\crossreferences
\begin{itemize}
\item \plc{default-device-var}, see \specref{subsec:ICV Descriptions}.

\item \code{task}, see \specref{subsec:task Construct}.

\item \code{task}~\code{scheduling}~\code{constraints}, 
      see \specref{subsec:Task Scheduling}. 

\item \code{target}~\code{data}, see \specref{subsec:target data Construct}.

\item \code{target}~\code{exit}~\code{data}, 
      see \specref{subsec:target exit data Construct}.

\item \code{if} Clause, see \specref{sec:if Clause}.

\item \code{map} clause, see \specref{subsec:map Clause}.

\item \code{omp\_get\_num\_devices} routine, see \specref{subsec:omp_get_num_devices}
  
\item \code{ompt\_callback\_target\_t}, see
\specref{sec:ompt_callback_target_t}.

\end{itemize}





\subsection{\code{target}~\code{exit}~\code{data} Construct}
\label{subsec:target exit data Construct}
\index{constructs!target exit data@{\code{target}~\code{exit}~\code{data}}}
\index{device data environments}
\summary
The \code{target}~\code{exit}~\code{data} directive specifies that list items are unmapped from a device data environment. The \code{target}~\code{exit}~\code{data} directive is a stand-alone directive.
\syntax
\begin{ccppspecific}
The syntax of the \code{target}~\code{exit}~\code{data} construct is as follows:
\begin{boxedcode}
\#pragma omp target exit data \plc{[ clause[ [,] clause]...] new-line}
\end{boxedcode}
where \plc{clause} is one of the following:
\begin{indentedcodelist}
if(\plc{[} target exit data :\plc{] scalar-expression})
device(\plc{integer-expression})
map(\plc{[ [map-type-modifier[,]] map-type} : \plc{] list})
depend(\plc{dependence-type} : \plc{locator-list}[:\plc{iterators-definition}])
nowait
\end{indentedcodelist}
\end{ccppspecific}
\begin{fortranspecific}
The syntax of the \code{target}~\code{exit}~\code{data} is as follows:
\begin{boxedcode}
!\$omp target exit data \plc{[ clause[ [,] clause]...]}
\end{boxedcode}
where clause is one of the following:
\begin{indentedcodelist}
if(\plc{[} target exit data :\plc{] scalar-logical-expression})
device(\plc{scalar-integer-expression})
map(\plc{[ [map-type-modifier[,]] map-type} : \plc{] list})
depend(\plc{dependence-type} : \plc{locator-list}[:\plc{iterators-definition}])
nowait
\end{indentedcodelist}
\end{fortranspecific}

\binding
The binding task set for a \code{target}~\code{exit}~\code{data} region is
the generating task, which is the \plc{target task} generated by the
\code{target}~\code{exit}~\code{data} construct. The
\code{target}~\code{exit}~\code{data} region binds to the corresponding
\plc{target task} region.

\descr
When a \code{target}~\code{exit}~\code{data} construct is encountered, the list items in the \code{map} clauses are unmapped from the device data environment according to the \code{map} clause semantics.

The \code{target}~\code{exit}~\code{data} construct is a task generating construct.  The generated task is a \plc{target task}.  The generated task region encloses the \code{target}~\code{exit}~\code{data} region.

All clauses are evaluated when the \code{target}~\code{exit}~\code{data} construct is encountered.  The data environment of the \plc{target task} is created according to the data-sharing attribute clauses on the \code{target}~\code{exit}~\code{data} construct, per-data environment ICVs, and any default data-sharing attribute rules that apply to the \code{target}~\code{exit}~\code{data} construct.  A variable that is mapped in the \code{target}~\code{exit}~\code{data} construct has a default data-sharing attribute of shared in the data environment of the \plc{target task}.

Assignment operations associated with mapping a variable (see \specref{subsec:map Clause}) occur when the \plc{target task} executes.

If the \code{nowait} clause is present, execution of the \plc{target task} may be deferred.  If the \code{nowait} clause is not present, the \plc{target task} is an included task.

If a \code{depend} clause is present, it is associated with the \plc{target task}.

If there is no \code{device} clause, the default device is determined by the \plc{default-device-var} ICV.

When an \code{if} clause is present and the \code{if} clause expression evaluates to \plc{false}, the device is the host. 
%\tools
\omptTarget{\code{target}~\code{exit}~\code{data}}{ompt\_target\_exit\_data}{target-exit}{the target task on the host}

\restrictions
\begin{itemize}
\item A program must not depend on any ordering of the evaluations of the clauses of the \code{target}~\code{exit}~\code{data} directive, or on any side effects of the evaluations of the clauses.
\item At least one \code{map} clause must appear on the directive.
\item At most one \code{device} clause can appear on the
  directive.  The \code{device} expression must evaluate to a
  non-negative integer value less than the value of
  \code{omp\_get\_num\_devices()}.
\item At most one \code{if} clause can appear on the directive.
\item A \plc{map-type} must be specified in all \code{map} clauses and must be either \code{from}, \code{release}, or \code{delete}.
\item At most one \code{nowait} clause can appear on the directive.
\end{itemize}

\crossreferences
\begin{itemize}
\item \plc{default-device-var}, see \specref{subsec:ICV Descriptions}.

\item \code{task}, see \specref{subsec:task Construct}.

\item \code{task}~\code{scheduling}~\code{constraints}, 
      see \specref{subsec:Task Scheduling}. 

\item \code{target}~\code{data}, see \specref{subsec:target data Construct}.

\item \code{target}~\code{enter}~\code{data}, 
      see \specref{subsec:target enter data Construct}.

\item \code{if} Clause, see \specref{sec:if Clause}.

\item \code{map} clause, see \specref{subsec:map Clause}.

\item \code{omp\_get\_num\_devices} routine, see \specref{subsec:omp_get_num_devices}

\item \code{ompt\_callback\_target\_t}, see
\specref{sec:ompt_callback_target_t}.

\end{itemize}





\pagebreak

\subsection{\code{target} Construct}
\index{target@{\code{target}}}
\index{constructs!target@{\code{target}}}
\index{device constructs!target@{\code{target}}}
\label{subsec:target Construct}
\summary
Map variables to a device data environment and execute the construct on that device.
\syntax
\begin{ccppspecific}
The syntax of the \code{target} construct is as follows:

\begin{boxedcode}
\#pragma omp target \plc{[clause[ [},\plc{] clause] ... ] new-line}
    \plc{structured-block}
\end{boxedcode}

where \plc{clause} is one of the following:

\begin{indentedcodelist}
if(\plc{[} target :\plc{] scalar-expression})
device(\plc{integer-expression})
private(\plc{list})
firstprivate(\plc{list})
in_reduction(\plc{reduction-identifier }:\plc{ list})
reduction(\plc{reduction-identifier }:\plc{ list})
map(\plc{[[map-type-modifier[,]] map-type}:\plc{ ] list})
is\_device\_ptr(\plc{list})
defaultmap(tofrom:scalar)
nowait
depend(\plc{dependence-type} : \plc{locator-list}[:\plc{iterators-definition}])
allocate(\plc{allocator:}\plc{list})
\end{indentedcodelist}
\end{ccppspecific}

\begin{samepage}
\smallskip
\begin{fortranspecific}
The syntax of the \code{target} construct is as follows:

\begin{boxedcode}
!\$omp target \plc{[clause[ [},\plc{] clause] ... ]}
    \plc{structured-block}
!\$omp end target
\end{boxedcode}

where \plc{clause} is one of the following:

\begin{indentedcodelist}
if(\plc{[} target :\plc{] scalar-logical-expression})
device(\plc{scalar-integer-expression})
private(\plc{list})
firstprivate(\plc{list})
in_reduction(\plc{reduction-identifier }:\plc{ list})
reduction(\plc{reduction-identifier }:\plc{ list})
map(\plc{[[map-type-modifier[,]] map-type}:\plc{ ] list})
is\_device\_ptr(\plc{list})
defaultmap(tofrom:scalar)
nowait
depend(\plc{dependence-type} : \plc{locator-list}[:\plc{iterators-definition}])
allocate(\plc{allocator:}\plc{list})
\end{indentedcodelist}

The \code{end}~\code{target} directive denotes the end of the \code{target} construct
\end{fortranspecific}
\end{samepage}

\binding
The binding task set for a \code{target} region is the
generating task, which is the \plc{target task} generated
by the \code{target} construct. The \code{target}
region binds to the corresponding \plc{target task} region.

\descr
The \code{target} construct provides a superset of the functionality
provided by the \code{target}~\code{data} directive, except for 
the \code{use\_device\_ptr} clause.

The functionality added to the \code{target} directive is the inclusion of an executable region to be executed by a device. That is, the \code{target} directive is an executable directive.

The \code{target} construct is a task generating construct.  The generated task is a \plc{target task}.  The generated task region encloses the \code{target} region.

All clauses are evaluated when the \code{target} construct is encountered.
The data environment of the \plc{target task} is created according to the
data-sharing attribute clauses on the \code{target} construct, per-data
environment ICVs, and any default data-sharing attribute rules that apply to
the \code{target} construct.  If a variable or part of a variable appears as a
list item in a \code{reduction} clause on the \code{target} construct and does
not appear as a list item in an \code{in\_reduction} clause on the construct,
the variable has a default data-sharing attribute of shared in the data
environment of the \plc{target task}. Likewise, if a variable or part of a
variable is mapped by the \code{target} construct, the variable has a default
data-sharing attribute of shared in the data environment of the \plc{target
task}.

If an \code{in\_reduction} clause is present on the \code{target} construct,
the behavior is as if a \code{reduction} clause with the same reduction
operator and list items is present on the construct.

Assignment operations associated with mapping a variable (see \specref{subsec:map Clause}) occur when the \plc{target task} executes.

If the \code{nowait} clause is present, execution of the \plc{target task} may be deferred.  If the \code{nowait} clause is not present, the \plc{target task} is an included task.

If a \code{depend} clause is present, it is associated with the \plc{target task}.

When an \code{if} clause is present and the \code{if} clause expression evaluates to \plc{false}, the \code{target} region is executed by the host device in the host data environment.

The \code{is\_device\_ptr} clause is used to indicate that a list item is a device
pointer already in the device data environment and that it should be used
directly.  Support for device pointers created outside of OpenMP, specifically
outside of the \code{omp\_target\_alloc} routine and the \code{use\_device\_ptr} clause,
is implementation defined.

If a function (C, C++, Fortran) or subroutine (Fortran) is referenced in a
\code{target} construct then that function or subroutine is treated as if its
name had appeared in a \code{to} clause on a \code{declare}~\code{target}
directive.

\begin{ccppspecific}

If an array section is a list item in a \code{map} clause and it has a named
pointer that is a scalar variable with a predetermined data-sharing attribute
of firstprivate (see \specref{subsubsec:Data-sharing Attribute Rules for
Variables Referenced in a Construct}) then on entry to the \code{target}
region:

\begin{itemize}
\item If the list item is not a zero-length array section, the corresponding 
private variable is initialized relative to the address of the storage
location of the corresponding array section in the device data environment
that is created by the \code{map} clause.

\item If the list item is a zero-length array section, the corresponding
private variable is initialized relative to the address of the 
corresponding storage location in the device data environment. If the
corresponding storage location is not present in the device data environment,
the corresponding private variable is initialized to NULL.
\end{itemize}

\end{ccppspecific}

%\tools

\events
\omptTargetRegionEvents{\code{target}}{ompt\_target}{target}

The \plc{target-submit} event occurs prior to creating an initial task on a target device for a target
region. 

% target task
\tools
\omptTargetRegionTools{\code{target}}{ompt\_target}{target}{target task on the host}

A thread dispatches a registered \code{ompt\_callback\_target\_submit}
callback for each occurrence of a \plc{target-submit} event
in that thread. The callback has type signature
\code{ompt\_callback\_target\_submit\_t}. 

\restrictions
\begin{itemize}
\item If a \code{target}, \code{target}~\code{update}, 
\code{target}~\code{data}, \code{target}~\code{enter}~\code{data}, or 
\code{target}~\code{exit}~\code{data} construct is encountered during
execution of a \code{target} region, the behavior is unspecified.

\item The result of an \code{omp\_set\_default\_device}, 
\code{omp\_get\_default\_device}, or \code{omp\_get\_num\_devices} 
routine called within a \code{target} region is unspecified.

\item The effect of an access to a \code{threadprivate} variable in a target region is 
unspecified.

\item If a list item in a \code{map} clause is a structure element, any other 
element of that structure that is referenced in the \code{target} construct 
must also appear as a list item in a \code{map} clause.

\item A variable referenced in a \code{target} region but not the \code{target} construct that is not 
declared in the \code{target} region must appear in a \code{declare}~\code{target} directive. 

\item At most one \code{defaultmap} clause can appear on the directive.

\item At most one \code{nowait} clause can appear on the directive.

\item A \plc{map-type} in a \code{map} clause must be \code{to}, \code{from}, \code{tofrom} or \code{alloc}.

\item A list item that appears in an \code{is\_device\_ptr} clause must be a valid device pointer in the device data environment.

\item At most one \code{device} clause can appear on the
  directive.  The \code{device} expression must evaluate to a
  non-negative integer value less than the value of
  \code{omp\_get\_num\_devices()}.

\begin{ccppspecific}
\item An attached pointer may not be modified in a \code{target} region.
\end{ccppspecific}

\begin{cspecific}
\item A list item that appears in an \code{is\_device\_ptr} clause must have a type of pointer or array.
\end{cspecific}

\begin{cppspecific}
\item A list item that appears in an \code{is\_device\_ptr} clause must have a type
  of pointer, array, reference to pointer or reference to array.
\item The effect of invoking a virtual member function of an object on a device other than the device on which the object was constructed is implementation defined. 
\item A throw executed inside a \code{target} region must cause execution to resume within the
same \code{target} region, and the same thread that threw the exception must catch it. 
\end{cppspecific}

\begin{fortranspecific}
\item A list item that appears in an \code{is\_device\_ptr} clause
  must be a dummy argument that does not have the \code{ALLOCATABLE},
  \code{POINTER} or \code{VALUE} attribute.
\item If a list item in a \code{map} clause is an array section, and the array section is derived from a variable with a \code{POINTER} or \code{ALLOCATABLE} attribute then the behavior is unspecified if the corresponding list item's variable is modified in the region.
\end{fortranspecific}
\end{itemize}

\crossreferences
\begin{itemize}
\item \plc{default-device-var}, see 
\specref{sec:Internal Control Variables}. 

\item \code{task} construct, see 
\specref{subsec:task Construct}.

\item \code{task} scheduling constraints, see 
\specref{subsec:Task Scheduling}

\item \code{target}~\code{data} construct, see 
\specref{subsec:target data Construct}.

\item \code{if} Clause, see \specref{sec:if Clause}.

\item \code{private} and \code{firstprivate} clauses, see 
\specref{subsec:Data-Sharing Attribute Clauses}.

\item Data-mapping Attribute Rules and Clauses, see 
\specref{subsec:Data-mapping Attribute Rules and Clauses}.

\item \code{omp\_get\_num\_devices} routine, see \specref{subsec:omp_get_num_devices}

\item \code{ompt\_callback\_target\_t}, see 
\specref{sec:ompt_callback_target_t}.

\item \code{ompt\_callback\_target\_submit\_t}, 
\specref{sec:ompt_callback_target_submit_t}.

\end{itemize}







\vspace{-24pt}

\subsection{\code{target}~\code{update} Construct}
\index{target update@{\code{target}~\code{update}}}
\index{constructs!target update@{\code{target}~\code{update}}}
\index{device constructs!target update@{\code{target}~\code{update}}}
\label{subsec:target update Construct}
\summary
The \code{target}~\code{update} directive makes the corresponding list items in the device data 
environment consistent with their original list items, according to the specified motion 
clauses. The \code{target}~\code{update} construct is a stand-alone directive.

\syntax
\begin{ccppspecific}
The syntax of the \code{target}~\code{update} construct is as follows:

\begin{boxedcode}
\#pragma omp target update \plc{clause[ [ [},\plc{] clause] ... ] new-line}
\end{boxedcode}
where \plc{clause} is either \plc{motion-clause} or one of the following:

\begin{indentedcodelist}
if(\plc{[} target update :\plc{] scalar-expression})
device(\plc{integer-expression})
nowait
depend(\plc{dependence-type} : \plc{locator-list}[:\plc{iterators-definition}])
\end{indentedcodelist}

and \plc{motion-clause} is one of the following:

\begin{indentedcodelist}
to([mapper(\plc{mapper-identifier}):]\plc{list})
from([mapper(\plc{mapper-identifier}):]\plc{list})
\end{indentedcodelist}
\end{ccppspecific}

\begin{fortranspecific}
The syntax of the \code{target}~\code{update} construct is as follows:

\begin{boxedcode}
!\$omp target update \plc{clause[ [ [},\plc{] clause] ... ]}
\end{boxedcode}

where \plc{clause} is either \plc{motion-clause} or one of the following:

\begin{indentedcodelist}
if(\plc{[}target update :\plc{] scalar-logical-expression})
device(\plc{scalar-integer-expression})
nowait
depend(\plc{dependence-type} : \plc{locator-list}[:\plc{iterators-definition}])
\end{indentedcodelist}

and \plc{motion-clause} is one of the following:

\begin{indentedcodelist}
to([mapper(\plc{mapper-identifier}):]\plc{list})
from([mapper(\plc{mapper-identifier}):]\plc{list})
\end{indentedcodelist}
\end{fortranspecific}

\begin{samepage}

\binding
The binding task set for a \code{target}~\code{update} region is the
generating task, which is the \plc{target task} generated
by the \code{target}~\code{update} construct. The \code{target}~\code{update}
region binds to the corresponding \plc{target task} region.

\descr
For each list item in a \code{to} or \code{from} clause there is a corresponding list item and an 
original list item. If the corresponding list item is not present in the device data environment then no assignment occurs to or from the original list item. Otherwise, each corresponding list item in the 
device data environment has an original list item in the current task's data
environment.  If a \code{mapper()} modifier appears in a \code{to} clause, each list item is
replaced with the list items that the given mapper specifies are to be mapped
with a \code{to} or \code{tofrom} map-type. If a \code{mapper()} modifier
appears in a \code{from} clause, each list item is replaced with the list items
that the given mapper specifies are to be mapped with a \code{from} or
\code{tofrom} map-type.

For each list item in a \code{from} or a \code{to} clause: \begin{itemize}

    \item For each part of the list item that is an attached pointer:  \begin{itemize}

        \item On exit from the region that part of the original list item will
        have the value it had on entry to the region; 

        \item On exit from the region that part of the corresponding list item will
        have the value it had on entry to the region; 

   \end{itemize}

    \item For each part of the list item that is not an attached pointer: \begin{itemize}

        \item If the clause is \code{from}, the value of that part of the
        corresponding list item is assigned to that part of the original list
       item; 

        \item If the clause is \code{to}, the value of that part of the
        original list item is assigned to that part of the corresponding list
        item. 

   \end{itemize}

   \item To avoid race conditions: \begin{itemize}

        \item Concurrent reads or updates of any part
        of the original list item must be synchronized with the update of the
        original list item that occurs as a result of the \code{from} clause;

        \item Concurrent reads or updates of any part
        of the corresponding list item must be synchronized with the update of the
        corresponding list item that occurs as a result of the \code{to} clause.

   \end{itemize}

\end{itemize}

\end{samepage}

The list items that appear in the \code{to} or \code{from} clauses may include array sections.

The \code{target}~\code{update} construct is a task generating construct.  The generated task is a \plc{target task}.  The generated task region encloses the \code{target}~\code{update} region.

All clauses are evaluated when the \code{target}~\code{update} construct is encountered.  The data environment of the \plc{target task} is created according to the data-sharing attribute clauses on the \code{target}~\code{update} construct, per-data environment ICVs, and any default data-sharing attribute rules that apply to the \code{target}~\code{update} construct.  A variable that is mapped in the \code{target}~\code{update} construct has a default data-sharing attribute of shared in the data environment of the \plc{target task}.

Assignment operations associated with mapping a variable (see \specref{subsec:map Clause}) occur when the \plc{target task} executes.

If the \code{nowait} clause is present, execution of the \plc{target task} may be deferred.  If the \code{nowait} clause is not present, the \plc{target task} is an included task.

If a \code{depend} clause is present, it is associated with the \plc{target task}.

The device is specified in the \code{device} clause. If there is no \code{device} clause, the device 
is determined by the \plc{default-device-var} ICV. When an \code{if} clause is present and the \code{if} 
clause expression evaluates to \plc{false} then no assignments occur.

%\tools
\omptTarget{\code{target}~\code{update}}{ompt\_target\_update}{target-update}{the target task on the host}

\restrictions
\begin{itemize}
\item A program must not depend on any ordering of the evaluations of the clauses of the 
\code{target}~\code{update} directive, or on any side effects of the evaluations of the clauses. 

\item At least one \plc{motion-clause} must be specified.

\item If a list item is an array section it must specify contiguous storage. 

\item A list item can only appear in a \code{to} or \code{from} clause, but not both.

\item A list item in a \code{to} or \code{from} clause must have a mappable type.

\item At most one \code{device} clause can appear on the directive. The \code{device} expression
      must evaluate to a non-negative integer value less than the value
      of \code{omp\_get\_num\_devices()}.

\item At most one \code{if} clause can appear on the directive. 

\item At most one \code{nowait} clause can appear on the directive.
\end{itemize}

\crossreferences
\begin{itemize}
\item \plc{default-device-var}, see 
\specref{sec:Internal Control Variables}.

\item Array sections, 
\specref{sec:Array Sections}

\item \code{task} construct, see 
\specref{subsec:task Construct}.

\item \code{task} scheduling constraints, see 
\specref{subsec:Task Scheduling}

\item \code{target}~\code{data}, see 
\specref{subsec:target data Construct}. 

\item \code{if} Clause, see \specref{sec:if Clause}.

\item \code{omp\_get\_num\_devices} routine, see \specref{subsec:omp_get_num_devices}

\item \code{ompt\_callback\_task\_create\_t}, see
\specref{sec:ompt_callback_task_create_t}.

\item \code{ompt\_callback\_target\_t}, see
\specref{sec:ompt_callback_target_t}.


\end{itemize}










\subsection{\code{declare}~\code{target} Directive}
\index{declare target@{\code{declare}~\code{target}}}
\index{directives!declare target@{\code{declare}~\code{target}}}
\index{constructs!declare target@{\code{declare}~\code{target}}}
\index{device constructs!declare target@{\code{declare}~\code{target}}}
\label{subsec:declare target Directive}
\summary
The \code{declare}~\code{target} directive specifies that variables, 
functions (C, C++ and Fortran), and subroutines (Fortran) are mapped 
to a device. The \code{declare}~\code{target} directive is a declarative 
directive.

\syntax
\begin{ccppspecific}
The syntax of the \code{declare}~\code{target} directive takes either of 
the following forms:

\begin{boxedcode}
\#pragma omp declare target \plc{new-line}
\plc{declaration-definition-seq}
\#pragma omp end declare target \plc{new-line}
\end{boxedcode}

or

\begin{boxedcode}
\#pragma omp declare target (\plc{extended-list}) \plc{new-line}
\end{boxedcode}

or

\begin{boxedcode}
\#pragma omp declare target \plc{clause[ [},\plc{] clause ... ] new-line}
\end{boxedcode}

where \plc{clause} is one of the following:

\begin{indentedcodelist}
to(\plc{extended-list})
link(\plc{list})
implements(\plc{function-name})
device_type(\code{host \textnormal{|} nohost \textnormal{|} any})
\end{indentedcodelist}
\end{ccppspecific}

\begin{samepage}

\begin{fortranspecific}
The syntax of the \code{declare}~\code{target} directive is as follows:

\begin{boxedcode}
!\$omp declare target (\plc{extended-list})
\end{boxedcode}

or

\begin{boxedcode}
!\$omp declare target \plc{[clause[ [},\plc{] clause] ... ]} 
\end{boxedcode}

where \plc{clause} is one of the following:

\begin{indentedcodelist}
to(\plc{extended-list})
link(\plc{list})
implements(\plc{subroutine-name})
device_type(\code{host \textnormal{|} nohost \textnormal{|} any})
\end{indentedcodelist}
\end{fortranspecific}

\descr

The \code{declare} \code{target} directive ensures that procedures
and global variables can be executed or accessed on a device.
Variables are mapped for all device executions, or for specific
device executions through a \code{link} clause.

If an \plc{extended-list} is present with no clause then the \code{to}
clause is assumed.

The \code{implements} clause specifies an alternate version of a procedure should be 
used.

The \code{device\_type} clause specifies if a version of the procedure should be made 
available on host, device or both. If \code{host} is specified only host version
of the procedure is made available.  If \code{nohost} is specified then only device
version of the procedure is made available.  If \code{any} is specified then both 
device and host version of the procedure is made available.

\begin{ccppspecific}
If a function is treated as if it appeared as a list item in a \code{to} clause
on a \code{declare}~\code{target} directive in the same translation unit in
which the definition of the function occurs then a device-specific version of
the function is created.

If a variable is treated as if it appeared as a list item in a \code{to} clause
on a \code{declare}~\code{target} directive in the same translation unit in
which the definition of the variable occurs then the original list item is
allocated a corresponding list item in the device data environment of all
devices.  

All calls in \code{target} constructs to the function in the \code{implements} clause 
are replaced by the function following the \code{declare} \code{target} constructs. 
\end{ccppspecific}

\begin{fortranspecific}
If a procedure that is host associated is treated as if it appeared as a
list item in a \code{to} clause on a \code{declare}~\code{target} directive
then a device-specific version of the procedure is created.

If a variable that is host associated is treated as if it appeared as a
list item in a \code{to} clause on a \code{declare}~\code{target} directive
then the original list item is allocated a corresponding list item in the
device data environment of all devices.

All calls in \code{target} constructs to the procedure in the \code{implements} 
clause are replaced by the procedure in which \code{declare} \code{target} construct appeared.
\end{fortranspecific}

\end{samepage}

If a variable is treated as if it appeared as a list item in a \code{to}
clause on a \code{declare}~\code{target} directive then the corresponding list
item in the device data environment of each device is initialized once, in the
manner specified by the program, but at an unspecified point in the program
prior to the first reference to that list item.  The list item is never removed
from those device data environments as if its reference count is initialized to
positive infinity.

The list items of a \code{link} clause are not mapped 
by the \code{declare}~\code{target} directive. Instead, their mapping
is deferred until they are mapped by \code{target}~\code{data} 
or \code{target} constructs. They are mapped only for such regions.

\pagebreak
\begin{ccppspecific}
If a function is referenced in a function that is treated as if it appeared as
a list item in a \code{to} clause on a \code{declare}~\code{target} directive
then the name of the referenced function is treated as if it had appeared in a
\code{to} clause on a \code{declare}~\code{target} directive.

If a variable with static storage duration or a function (except
\plc{lambda} for C++) is referenced in the
initializer expression list of a variable with static storage duration that is
treated as if it appeared as a list item in a \code{to} clause on a
\code{declare}~\code{target} directive then the name of the referenced variable
or function is treated as if it had appeared in a \code{to} clause on a
\code{declare}~\code{target} directive.

The form of the \code{declare}~\code{target} directive that has no clauses 
and requires a matching \code{end}~\code{declare}~\code{target} directive
defines an implicit \plc{extended-list} to an implicit \code{to}
clause. The implicit \plc{extended-list} consists of the variable names 
of any variable declarations at file or namespace scope that appear between 
the two directives and of the function names of any function declarations at 
file, namespace or class scope that appear between the two directives.

The \plc{declaration-definition-seq} defined by a 
\code{declare}~\code{target} directive and an 
\code{end}~\code{declare}~\code{target} directive may contain
\code{declare}~\code{target} directives without the \code{device\_type} clause. 
If a list item appears both in an implicit and explicit list, the explicit 
list takes precedence.
 
\end{ccppspecific}

\begin{fortranspecific}
If a procedure is referenced in a procedure that is treated as if it appeared
as a list item in a \code{to} clause on a \code{declare}~\code{target}
directive then the name of the procedure is treated as if it had appeared in a
\code{to} clause on a \code{declare}~\code{target} directive.

If a \code{declare}~\code{target} does not have any clauses then an implicit 
\plc{extended-list} to an implicit \code{to} clause of one item is formed from
the name of the enclosing subroutine subprogram, function subprogram or 
interface body to which it applies.

If a \code{declare}~\code{target} directive has an \code{implements} or \code{device\_type} clause
then any enclosed internal procedures cannot contain any \code{declare}~\code{target} directives.  
The enclosing \code{device\_type} clause implicitly applies to internal procedures.
\end{fortranspecific}


\restrictions
\begin{itemize}
\item A threadprivate variable cannot appear in a 
      \code{declare}~\code{target} directive.

\item A variable declared in a \code{declare}~\code{target} directive 
      must have a mappable type.

\item The same list item must not appear multiple times in clauses on the same directive.

\item The same list item must not explicitly appear in both a \code{to} clause on one
\code{declare}~\code{target} directive and a \code{link} clause on
another \code{declare}~\code{target} directive.

\item The \code{implements} clause and \code{device\_type} clause are the only valid combination
that can appear if either one of them is present.

\end{itemize}

\begin{cppspecific}
\begin{itemize}
\item The function names of overloaded functions or template functions
may only be specified within an implicit \plc{extended-list}.

\item If a \plc{lambda declaration and definition} appears between a 
\code{declare target} directive and the matching \code{end declare
target} directive, all the variables that are captured by the
\plc{lambda} expression must also be variables that are treated as
if they appear in a \code{to} clause.
\end{itemize}
\end{cppspecific}

\begin{fortranspecific}
\begin{itemize}
\item If a list item is a procedure name, it must not be a generic name, 
      procedure pointer or entry name.

\item Any \code{declare}~\code{target} directive with clauses must appear 
      in a specification part of a subroutine subprogram, function subprogram,
      program or module.

\item Any \code{declare}~\code{target} directive without clauses must appear 
      in a specification part of a subroutine subprogram, function subprogram 
      or interface body to which it applies.

\item If a \code{declare}~\code{target} directive is specified in an 
      interface block for a procedure, it must match a 
      \code{declare}~\code{target} directive in the definition of the 
      procedure.

\item If an external procedure is a type-bound procedure of a derived 
      type and a \code{declare}~\code{target} directive is specified in 
      the definition of the external procedure, such a directive must 
      appear in the interface block that is accessible to the derived 
      type definition.

\item If any procedure is declared via a procedure declaration statement 
      that is not in the type-bound procedure part of a derived-type 
      definition, any \code{declare}~\code{target} with the procedure 
      name must appear in the same specification part.

\item A variable that is part of another variable (as an array or structure 
      element) cannot appear in a \code{declare}~\code{target} directive.

\item The \code{declare}~\code{target} directive must appear in the 
      declaration section of a scoping unit in which the common block 
      or variable is declared. Although variables in common blocks can 
      be accessed by use association or host association, common block 
      \code{declare}~\code{target} directive must be declared to be a 
      common block in the same scoping unit in which the 
      \code{declare}~\code{target} directive appears.

\item If a \code{declare}~\code{target} directive specifying a common 
      block name appears in one program unit, then such a directive must 
      also appear in every other program unit that contains a \code{COMMON} 
      statement specifying the same name. It must appear after the last 
      such \code{COMMON} statement in the program unit.

\item If a list item is declared with the \code{BIND} attribute, the 
      corresponding C entities must also be specified in a 
      \code{declare}~\code{target} directive in the C program.

\item A blank common block cannot appear in a \code{declare}~\code{target} 
      directive.

\item A variable can only appear in a \code{declare}~\code{target} directive 
      in the scope in which it is declared. It must not be an element of a 
      common block or appear in an \code{EQUIVALENCE} statement.

\item A variable that appears in a \code{declare}~\code{target} directive 
      must be declared in the Fortran scope of a module or have the 
      \code{SAVE} attribute, either explicitly or implicitly. 
\end{itemize}
\end{fortranspecific}


\subsection{\code{declare}~\code{mapper} Directive}
\index{declare mapper@{\code{declare}~\code{mapper}}}
\index{directives!declare mapper@{\code{declare}~\code{mapper}}}
\index{constructs!declare mapper@{\code{declare}~\code{mapper}}}
\index{device constructs!declare mapper@{\code{declare}~\code{mapper}}}
\label{subsec:declare mapper Directive}

\summary

The \code{declare mapper} directive declares a user-defined mapper for a given
type, and may define a \plc{mapper-identifier} that can be used in a \code{map}
clause. The \code{declare mapper} directive is a declarative directive.

\syntax
\begin{ccppspecific}
The syntax of the \code{declare mapper} directive is as follows:

\begin{boxedcode}
\#pragma omp declare mapper(\plc{[mapper-identifier:]type var})
\plc{[clause[ [\code{,}] clause] ... ] new-line}
\end{boxedcode}

\end{ccppspecific}

\begin{fortranspecific}
The syntax of the \code{declare mapper} directive is as follows:

\begin{boxedcode}
!\$omp declare mapper(\plc{[mapper-identifier:] type \code{::} var})
\plc{[clause[ [\code{,}] clause] ... ]}
\end{boxedcode}

\end{fortranspecific}

where:

\begin{itemize}
  \item \plc{mapper-identifier} is a base-language identifier
  \item \plc{type} is a valid type in scope
  \item \plc{var} is a valid base-language identifier
  \item \plc{clause} is one of the following:
    \begin{itemize}
      \item \code{map(\plc{[[map-type-modifier[,]] map-type}:\plc{ ] list})}
      \item \code{use\_by\_default}
    \end{itemize}
  \item \plc{map-type} is one of the following:
    \begin{itemize}
      \item \code{alloc}
      \item \code{to}
      \item \code{from}
      \item \code{tofrom}
    \end{itemize}
  \item and \plc{map-type-modifier} is \code{always}
\end{itemize}

\descr


User-defined mappers can be defined using the \code{declare mapper} directive.
The type and, if specified, the \plc{mapper-identifier}, uniquely identify the
mapper for use in a \code{map} clause later in the program.  The visibility and
accessibility of this declaration are the same as those of a variable declared
at the same point in the program.

The \code{use\_by\_default} clause makes this mapper the default for all
variables of \plc{type}, so that all map clauses with this construct in scope
that map a list item of \plc{type} will use this mapper unless another is
explicitly specified.

The variable declared by \plc{var} is available for use in all \code{map}
clauses on the directive, and no part of the variable to be mapped is mapped by
default.

The default mapper for all types \plc{T}, designated by the pre-defined
mapper-identifier \code{default}, is as follows unless a user-defined mapper is
specified for that type.

\begin{boxedcode}

  declare mapper(\plc{T} v) map(tofrom: v) use\_by\_default

\end{boxedcode}

All \code{map} clauses on the dirctive are expanded into corresponding
\code{map} clauses wherever this mapper is invoked, either by matching type or
by being explicitly named in a \code{map} clause. A \code{map} clause with list item
\plc{var} maps var as though no mapper were specified.


\begin{cppspecific}
The \code{declare mapper} directive can also appear at points in the program at
which a static data member could be declared. In this case, the visibility and
accessibility of the declaration are the same as those of a static data member
declared at the same point in the program.

\end{cppspecific}

\restrictions

\begin{itemize}
\item If the \code{use\_by\_default} clause is not specified, then a
  mapper-identifier is required.
\item No instance of the mapper type can be mapped as part of the mapper,
  either directly or indirectly through another type, except the instance passed
  as the list item.
\end{itemize}

\subsection{\code{teams} Construct}
\index{teams@{\code{teams}}}
\index{constructs!teams@{\code{teams}}}
\index{device constructs!teams@{\code{teams}}}
\label{subsec:teams Construct}
\summary
The \code{teams} construct creates a league of initial teams and the initial thread in each 
team executes the region.

\syntax
\begin{ccppspecific}
The syntax of the \code{teams} construct is as follows:

\begin{boxedcode}
\#pragma omp teams \plc{[clause[ [},\plc{] clause] ... ] new-line}
    \plc{structured-block}
\end{boxedcode}

where \plc{clause} is one of the following:

\begin{indentedcodelist}
num\_teams(\plc{integer-expression})
thread\_limit(\plc{integer-expression})
default(shared \textnormal{|} none)
private(\plc{list})
firstprivate(\plc{list})
shared(\plc{list})
reduction(\plc{reduction-identifier }:\plc{ list})
allocate(\plc{[allocator: ]}\plc{list})
\end{indentedcodelist}
\end{ccppspecific}

\begin{fortranspecific}
The syntax of the \code{teams} construct is as follows:

\begin{boxedcode}
!\$omp teams \plc{[clause[ [},\plc{] clause] ... ]}
    \plc{structured-block}
!\$omp end teams
\end{boxedcode}

where \plc{clause} is one of the following:

\begin{indentedcodelist}
num\_teams(\plc{scalar-integer-expression})
thread\_limit(\plc{scalar-integer-expression})
default(shared \textnormal{|} firstprivate \textnormal{|} private \textnormal{|} none)
private(\plc{list})
firstprivate(\plc{list})
shared(\plc{list})
reduction(\plc{reduction-identifier }:\plc{ list})
allocate(\plc{[allocator: ]}\plc{list})
\end{indentedcodelist}

The \code{end}~\code{teams} directive denotes the end of the \code{teams} construct.
\end{fortranspecific}

\begin{samepage}

\binding
The binding thread set for a \code{teams} region is the encountering thread,
which is the initial thread of the \code{target} region.

\descr
When a thread encounters a \code{teams} construct, a league of teams is
created. Each team is an initial team, and the initial thread in each team
executes the \code{teams} region.

The number of teams created is implementation defined, but is less than or equal to the 
value specified in the \code{num\_teams} clause.
A thread may obtain the number of initial teams created by the construct by a call to the \code{omp\_get\_num\_teams} routine.

\end{samepage}

The maximum number of threads participating in the contention group that each team 
initiates is implementation defined, but is less than or equal to the value specified in the 
\code{thread\_limit} clause.

On a combined or composite construct that includes \code{target} and
\code{teams} constructs, the expressions in \code{num\_teams} and
\code{thread\_limit} clauses are evaluated on the host device on
entry to the \code{target} construct.

Once the teams are created, the number of initial teams remains constant for the duration of the 
\code{teams} region.

Within a \code{teams} region, initial team numbers uniquely identify each
initial team. Initial team numbers are 
consecutive whole numbers ranging from zero to one less than the number of
initial teams. A 
thread may obtain its own initial team number by a call to the \code{omp\_get\_team\_num} library 
routine.

After the teams have completed execution of the \code{teams} region, the encountering thread 
resumes execution of the enclosing \code{target} region.

There is no implicit barrier at the end of a \code{teams} construct.

\restrictions
Restrictions to the \code{teams} construct are as follows:

\begin{itemize}
\item A program that branches into or out of a \code{teams} region is non-conforming.

\item A program must not depend on any ordering of the evaluations of the clauses of the 
\code{teams} directive, or on any side effects of the evaluation of the clauses.

\item At most one \code{thread\_limit} clause can appear on the directive. The 
\code{thread\_limit} expression must evaluate to a positive integer value.

\item At most one \code{num\_teams} clause can appear on the directive. The \code{num\_teams}
expression must evaluate to a positive integer value.

\item If specified, a \code{teams} construct must be contained within a \code{target} construct. That \code{target} construct must contain no statements, declarations or directives outside of the \code{teams} construct.

\item \code{distribute}, \code{distribute simd}, distribute parallel loop,
distribute parallel loop SIMD, and \code{parallel} regions, including any
\code{parallel} regions arising from combined constructs, are the only OpenMP regions
that may be strictly nested inside the \code{teams} region.

\end{itemize}

\crossreferences
\begin{itemize}

\item Data-sharing attribute clauses, see 
\specref{subsec:Data-Sharing Attribute Clauses}.

\item \code{omp\_get\_num\_teams} routine, see 
\specref{subsec:omp_get_num_teams}.

\item \code{omp\_get\_team\_num} routine, see 
\specref{subsec:omp_get_team_num}.
\end{itemize}









\subsection{\code{distribute} Construct}
\index{distribute@{\code{distribute}}}
\index{constructs!distribute@{\code{distribute}}}
\index{device constructs!distribute@{\code{distribute}}}
\label{subsec:distribute Construct}
\summary
The \code{distribute} construct specifies that the iterations of one or more loops will be 
executed by the initial teams in the context of their implicit tasks. The iterations are 
distributed across the initial threads of all initial teams that execute the \code{teams} region to 
which the \code{distribute} region binds.

\syntax
\begin{ccppspecific}
The syntax of the \code{distribute} construct is as follows:

\begin{boxedcode}
\#pragma omp distribute \plc{[clause[ [},\plc{] clause] ... ] new-line}
   \plc{for-loops}
\end{boxedcode}

Where \plc{clause} is one of the following:

\begin{indentedcodelist}
private(\plc{list})
firstprivate(\plc{list})
lastprivate(\plc{list})
collapse(\plc{n})
dist\_schedule(\plc{kind[},\plc{ chunk\_size]})
allocate(\plc{[allocator: ]}\plc{list})
\end{indentedcodelist}

All associated \plc{for-loops} must have the canonical form described in 
\specref{sec:Canonical Loop Form}.
\end{ccppspecific}
\bigskip

\begin{fortranspecific}
The syntax of the \code{distribute} construct is as follows:

\begin{boxedcode}
!\$omp distribute \plc{[clause[ [},\plc{] clause] ... ]}
   \plc{do-loops}
\plc{[}!\$omp end distribute\plc{]}
\end{boxedcode}

Where \plc{clause} is one of the following:

\begin{indentedcodelist}
private(\plc{list})
firstprivate(\plc{list})
lastprivate(\plc{list})
collapse(\plc{n})
dist\_schedule(\plc{kind[},\plc{ chunk\_size]})
allocate(\plc{[allocator: ]}\plc{list})
\end{indentedcodelist}

If an \code{end}~\code{distribute} directive is not specified, an \code{end}~\code{distribute} directive 
is assumed at the end of the \plc{do-loops}.

Any associated \plc{do-loop} must be a \plc{do-construct} or an
\plc{inner-shared-do-construct} as defined by the Fortran standard. If an 
\code{end}~\code{distribute} directive follows a \plc{do-construct} in which several loop statements share a \code{DO} 
termination statement, then the directive can only be specified for the outermost of these 
\code{DO} statements.
\end{fortranspecific}

\begin{samepage}

\binding
The binding thread set for a \code{distribute} region is the set of initial
threads executing an enclosing \code{teams} region. A \code{distribute} region
binds to this \code{teams} region. 
%Only the threads executing the binding \code{teams} region participate in the execution of the loop iterations.

\descr
The \code{distribute} construct is associated with a loop nest consisting of one or more 
loops that follow the directive.

There is no implicit barrier at the end of a \code{distribute} construct.
To avoid data races the original
list items modified due to \code{lastprivate} or \code{linear} clauses
should not be accessed between the end of the \code{distribute} construct and the end
of the \code{teams} region to which the \code{distribute} binds.

\end{samepage}

The \code{collapse} clause may be used to specify how many loops are
associated with the \code{distribute} construct.  The parameter of the
\code{collapse} clause must be a constant positive integer expression.
If no \code{collapse} clause is present or its paraemter is 1, the
only loop that is associated with the \code{distribute} construct is
the one that immediately follows the \code{distribute} construct.  If
a \code{collapse} clause is specified with a parameter value greater
than 1 and more than one loop is associated with the \code{distribute}
construct, then the iteration of all associated loops are collapsed
into one larger iteration space.  The sequential execution of the
iterations in all associated loops determines the order of the
iterations in the collapsed iteration space.

A distribute loop has logical iterations numbered 0,1,...,N-1 where N
is the number of loop iterations, and the logical numbering denotes
the sequence in which the iterations would be executed if the set of
associated loop(s) were executed sequentially.  At the beginning of
each logical iteration, the loop iteration variable of each associated
loop has the value that it would have if the set of the associated
loop(s) were executed sequentially.

If more than one loop is associated with the \code{distribute}
construct then the number of times that any intervening code between
any two associated loops will be executed is unspecified but will be
at least once per iteration of the loop enclosing the intervening code
and at most once per iteration of the innermost loop associated with
the construct.

The iteration count for each associated loop is computed before entry to the outermost loop. If execution of any associated loop changes any of the values used to compute any of the iteration counts, then the behavior is unspecified. 

The integer type (or kind, for Fortran) used to compute the iteration count for the collapsed loop is implementation defined.

If \code{dist\_schedule} is specified, \plc{kind} must be \code{static}. If specified, iterations are 
divided into chunks of size \plc{chunk\_size}, chunks are assigned to the
initial teams of the league in 
a round-robin fashion in the order of the initial team number. When no \plc{chunk\_size} is specified, 
the iteration space is divided into chunks that are approximately equal in size, and at 
most one chunk is distributed to each initial team of the league. The size of the 
chunks is unspecified in this case.

When no \code{dist\_schedule} clause is specified, the schedule is implementation defined.

%\tools
\omptWorksharing{distribute}{ompt\_work\_distribute} 

\restrictions
Restrictions to the \code{distribute} construct are as follows:

\begin{itemize}
\item The \code{distribute} construct inherits the restrictions of the loop construct.

\item The region associated with the \code{distribute} construct must be
strictly nested inside a \code{teams} region.

\item A list item may appear in a \code{firstprivate} or \code{lastprivate} clause but not both.
\end{itemize}

\crossreferences
\begin{itemize}
\item loop construct, see 
\specref{subsec:Loop Construct}.

\item \code{teams} construct, see 
\specref{subsec:teams Construct}

\item \code{ompt\_work\_distribute}, see \specref{sec:ompt_work_type_t}.
\item \code{ompt\_callback\_work\_t}, see
\specref{sec:ompt_callback_work_t}.
\end{itemize}










\subsection{\code{distribute}~\code{simd} Construct}
\index{distribute simd@{\code{distribute}~\code{simd}}}
\index{constructs!distribute simd@{\code{distribute}~\code{simd}}}
\index{device constructs!distribute simd@{\code{distribute}~\code{simd}}}
\label{subsec:distribute simd Construct}
\summary
The \code{distribute}~\code{simd} construct specifies a loop that will be distributed across the 
master threads of the \code{teams} region and executed concurrently using SIMD instructions. The \code{distribute}~\code{simd} construct is a composite construct.

\syntax
The syntax of the \code{distribute}~\code{simd} construct is as follows:

\begin{ccppspecific}
\begin{boxedcode}
\#pragma omp distribute simd \plc{[clause[ [},\plc{] clause] ... ] newline}
   \plc{for-loops}
\end{boxedcode}

where \plc{clause} can be any of the clauses accepted by the \code{distribute} or \code{simd} 
directives with identical meanings and restrictions.
\end{ccppspecific}

\begin{fortranspecific}
\begin{boxedcode}
!\$omp distribute simd \plc{[clause[ [},\plc{] clause] ... ]}
   \plc{do-loops}
\plc{[}!\$omp end distribute simd\plc{]}
\end{boxedcode}

where \plc{clause} can be any of the clauses accepted by the \code{distribute} or \code{simd} 
directives with identical meanings and restrictions.

If an \code{end}~\code{distribute}~\code{simd} directive is not specified, an \code{end}~\code{distribute}~\code{simd}
directive is assumed at the end of the \plc{do-loops}.
\end{fortranspecific}

\descr
The \code{distribute}~\code{simd} construct will first distribute the iterations of the associated 
loop(s) according to the semantics of the \code{distribute} construct and any clauses that 
apply to the distribute construct. The resulting chunks of iterations will then be 
converted to a SIMD loop in a manner consistent with any clauses that apply to the 
\code{simd} construct. The effect of any clause that applies to both constructs is as if it were applied to both constructs separately except the \code{collapse} clause, which is applied once.

\events

This composite construct generates the same events as the \code{distribute} construct.

\tools

This composite construct dispatches the same callbacks as the \code{distribute} construct.

\restrictions
\begin{itemize}
\item The restrictions for the \code{distribute} and \code{simd} constructs apply.

\item A list item may not appear in a \code{linear} clause, unless it is the loop iteration variable.

\item The \code{conditional} modifier may not appear in a \code{lastprivate} clause.
\end{itemize}

\crossreferences
\begin{itemize}
\item \code{simd} construct, see 
\specref{subsec:simd Construct}.

\item \code{distribute} construct, see 
\specref{subsec:distribute Construct}.

\item Data attribute clauses, see 
\specref{subsec:Data-Sharing Attribute Clauses}.

\item Events and tool callbacks for the \code{distribute} construct, see
\specref{subsec:simd Construct}.

\end{itemize}









\subsection{Distribute Parallel Loop Construct}
\index{distribute parallel loop construct}
\index{constructs!distribute parallel loop}
\index{device constructs!distribute parallel loop}
\index{constructs!distribute parallel for@{\code{distribute}~\code{parallel}~\code{for}}}
\index{constructs!distribute parallel do@{\code{distribute}~\code{parallel}~\code{do}}}
\label{subsec:Distribute Parallel Loop Construct}
\summary
The distribute parallel loop construct specifies a loop that can be executed in parallel by 
multiple threads that are members of multiple teams. The distribute parallel loop construct is a composite construct.

\syntax
The syntax of the distribute parallel loop construct is as follows:

\begin{ccppspecific}
\begin{boxedcode}
\#pragma omp distribute parallel for \plc{[clause[ [},\plc{] clause] ... ] newline}
    \plc{for-loops}
\end{boxedcode}

where \plc{clause} can be any of the clauses accepted by the \code{distribute} or parallel loop 
directives with identical meanings and restrictions.
\end{ccppspecific}

\begin{fortranspecific}
\begin{boxedcode}
!\$omp distribute parallel do \plc{[clause[ [},\plc{] clause] ... ]}
    \plc{do-loops}
\plc{[}!\$omp end distribute parallel do\plc{]}
\end{boxedcode}

where \plc{clause} can be any of the clauses accepted by the \code{distribute} or parallel loop 
directives with identical meanings and restrictions.

If an \code{end}~\code{distribute}~\code{parallel}~\code{do} directive is not specified, an 
\code{end}~\code{distribute}~\code{parallel}~\code{do} directive is assumed at the end of the \plc{do-loops}.
\end{fortranspecific}

\descr
The distribute parallel loop construct will first distribute the iterations of the associated loop(s) into chunks according to the semantics of the \code{distribute} construct and any clauses that apply to the \code{distribute} construct. Each of these chunks will form a loop. Each resulting loop will then be distributed across the threads within the teams region to which the \code{distribute} construct binds in a manner consistent with any clauses that apply to the parallel loop construct. 
The effect of any clause that applies to both constructs is as if it were applied to both constructs separately except the \code{collapse} clause, which is applied once.

\events

This composite construct generates the same events as the \code{distribute} and parallel loop constructs.

\tools

This composite construct dispatches the same callbacks as the \code{distribute} and parallel loop constructs.

%% BRONIS: Ugly trick to avoid page starting with end c/C++ sentinel
\vspace{-6pt}

\restrictions
\begin{itemize}
\item The restrictions for the \code{distribute} and parallel loop constructs apply.
\item No \code{ordered} clause can be specified.
\item No \code{linear} clause can be specified.
\item The \code{conditional} modifier may not appear in a \code{lastprivate} clause.
\end{itemize}

%% BRONIS: Another ugly trick to avoid page starting with end c/C++ sentinel
\vspace{-6pt}

\crossreferences
\begin{itemize}
\item \code{distribute} construct, see 
\specref{subsec:distribute Construct}.

\item Parallel loop construct, see 
\specref{subsec:Parallel Loop Construct}.

\item Data attribute clauses, see 
\specref{subsec:Data-Sharing Attribute Clauses}.

\item Events and tool callbacks for \code{distribute} construct, see
\specref{subsec:distribute Construct}.

\item Events and tool callbacks for parallel loop construct, see
\specref{subsec:Parallel Loop Construct}.

\end{itemize}






%% BRONIS: Another ugly trick to avoid page starting with end c/C++ sentinel
\vspace{-24pt}

\subsection{Distribute Parallel Loop SIMD Construct}
\label{subsec:Distribute Parallel Loop SIMD Construct}
\index{distribute parallel loop SIMD construct}
\index{constructs!distribute parallel loop SIMD}
\index{constructs!distribute parallel for simd@{\code{distribute}~\code{parallel}~\code{for}~\code{simd}}}
\index{constructs!distribute parallel do simd@{\code{distribute}~\code{parallel}~\code{do}~\code{simd}}}
\index{device constructs!distribute parallel loop SIMD}
\summary
The distribute parallel loop SIMD construct specifies a loop that can be executed 
concurrently using SIMD instructions in parallel by multiple threads that are members 
of multiple teams. The distribute parallel loop SIMD construct is a composite construct.

\syntax
\begin{ccppspecific}
The syntax of the distribute parallel loop SIMD construct is as follows:

\begin{boxedcode}
\#pragma omp distribute parallel for simd \plc{[clause[ [},\plc{] clause] ... ] newline}
    \plc{for-loops}
\end{boxedcode}

where \plc{clause} can be any of the clauses accepted by the \code{distribute} or parallel loop 
SIMD directives with identical meanings and restrictions
\end{ccppspecific}

\begin{fortranspecific}
The syntax of the distribute parallel loop SIMD construct is as follows:

\begin{boxedcode}
!\$omp distribute parallel do simd \plc{[clause[ [},\plc{] clause] ... ]}
    \plc{do-loops}
\plc{[}!\$omp end distribute parallel do simd\plc{]}
\end{boxedcode}

where \plc{clause} can be any of the clauses accepted by the \code{distribute} or parallel loop 
SIMD directives with identical meanings and restrictions.

If an \code{end}~\code{distribute}~\code{parallel}~\code{do}~\code{simd} directive is not specified, an 
\code{end}~\code{distribute}~\code{parallel}~\code{do}~\code{simd} directive is assumed at the end of the \plc{do-loops}.
\end{fortranspecific}

\descr
The distribute parallel loop SIMD construct will first distribute the iterations of the 
associated loop(s) according to the semantics of the \code{distribute} construct and any 
clauses that apply to the \code{distribute} construct. The resulting loops will then be 
distributed across the threads contained within the \code{teams} region to which the
\code{distribute} construct binds in a manner consistent with any clauses that apply to the 
parallel loop construct. The resulting chunks of iterations will then be converted to a 
SIMD loop in a manner consistent with any clauses that apply to the \code{simd} construct. 
The effect of any clause that applies to both constructs is as if it were applied to both constructs separately except the \code{collapse} clause, which is applied once.

\events

This composite construct generates the same events as the \code{distribute} and parallel loop constructs.

\tools

This composite construct dispatches the same callbacks as the \code{distribute} and parallel loop constructs.

\restrictions
\begin{itemize}
\item The restrictions for the \code{distribute} and parallel loop SIMD constructs apply.
\item No \code{ordered} clause can be specified.
\item A list item may not appear in a \code{linear} clause, unless it is the loop iteration variable.
\item The \code{conditional} modifier may not appear in a \code{lastprivate} clause.
\end{itemize}

\crossreferences
\begin{itemize}
\item \code{distribute} construct, see 
\specref{subsec:distribute Construct}.

\item Parallel loop SIMD construct, see 
\specref{subsec:Parallel Loop SIMD Construct}.

\item Data attribute clauses, see \specref{subsec:Data-Sharing Attribute Clauses}.

\item Events and tool callbacks for \code{distribute} construct, see
\specref{subsec:distribute Construct}.

\item Events and tool callbacks for parallel loop construct, see
\specref{subsec:Parallel Loop Construct}.

\end{itemize}
