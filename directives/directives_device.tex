% This is an included file. See the master file for more information.
%
% When editing this file:
%
%    1. To change formatting, appearance, or style, please edit openmp.sty.
%
%    2. Custom commands and macros are defined in openmp.sty.
%
%    3. Be kind to other editors -- keep a consistent style by copying-and-pasting to
%       create new content.
%
%    4. We use semantic markup, e.g. (see openmp.sty for a full list):
%         \code{}     % for bold monospace keywords, code, operators, etc.
%         \plc{}      % for italic placeholder names, grammar, etc.
%
%    5. There are environments that provide special formatting, e.g. language bars.
%       Please use them whereever appropriate.  Examples are:
%
%         \begin{fortranspecific}
%         This is text that appears enclosed in blue language bars for Fortran.
%         \end{fortranspecific}
%
%         \begin{note}
%         This is a note.  The "Note -- " header appears automatically.
%         \end{note}
%
%    6. Other recommendations:
%         Use the convenience macros defined in openmp.sty for the minor headers
%         such as Comments, Syntax, etc.
%
%         To keep items together on the same page, prefer the use of
%         \begin{samepage}.... Avoid \parbox for text blocks as it interrupts line numbering.
%         When possible, avoid \filbreak, \pagebreak, \newpage, \clearpage unless that's
%         what you mean. Use \needspace{} cautiously for troublesome paragraphs.
%
%         Avoid absolute lengths and measures in this file; use relative units when possible.
%         Vertical space can be relative to \baselineskip or ex units. Horizontal space
%         can be relative to \linewidth or em units.
%
%         Prefer \emph{} to italicize terminology, e.g.:
%             This is a \emph{definition}, not a placeholder.
%             This is a \plc{var-name}.
%


\section{Device Directives}
\label{sec:Device Directives}
\index{device directives}
\index{constructs!device constructs}
\index{device constructs!device constructs}

\subsection{Device Initialization}

\events

The \plc{device-initialize} event occurs in a thread that encounters
the first \code{target}, \code{target data}, or \code{target enter data} construct associated with a particular target device
after the thread initiates initialization of OpenMP on the device and the
device's OpenMP initialization, which may include device-side tool
initialization, completes.

The \plc{device-load} event for a code block for a target device occurs in some thread
before any thread executes code from that code block on that target device.

The \plc{device-unload} event for a target device occurs in some thread
whenever a code block is unloaded from the device.

The \plc{device-finalize} event for a target device that has been initialized
occurs in some thread before an OpenMP implementation shuts down.

\tools

A thread dispatches a registered \code{ompt_callback_device_initialize}
callback for each occurrence of a \plc{device-initialize} event in
that thread.  This callback has type signature
\code{ompt_callback_device_initialize_t}.

A thread dispatches a registered \code{ompt_callback_device_load}
callback for each occurrence of a \plc{device-load} event in
that thread.  This callback has type signature
\code{ompt_callback_device_load_t}.

A thread dispatches a registered \code{ompt_callback_device_unload}
callback for each occurrence of a \plc{device-unload} event in
that thread.  This callback has type signature
\code{ompt_callback_device_unload_t}.

A thread dispatches a registered \code{ompt_callback_device_finalize}
callback for each occurrence of a \plc{device-finalize} event in
that thread.  This callback has type signature
\code{ompt_callback_device_finalize_t}.

\restrictions
No thread may offload execution of an OpenMP construct to a device until a
dispatched \code{ompt_callback_device_initialize} callback completes.

No thread may offload execution of an OpenMP construct to a device after a
dispatched \code{ompt_callback_device_finalize} callback occurs.

\crossreferences
\begin{itemize}
\item \code{ompt_callback_device_initialize_t}, see
\specref{sec:ompt_callback_device_initialize_t}.
\item \code{ompt_callback_device_load_t}, see
\specref{sec:ompt_callback_device_load_t}.
\item \code{ompt_callback_device_unload_t}, see
\specref{sec:ompt_callback_device_unload_t}.
\item \code{ompt_callback_device_finalize_t}, see
\specref{sec:ompt_callback_device_finalize_t}.
\end{itemize}


\subsection{\hcode{target}~\hcode{data} Construct}
\index{target data@{\code{target}~\code{data}}}
\index{constructs!target data@{\code{target}~\code{data}}}
\label{subsec:target data Construct}
\summary
 Map variables to a device data environment for the extent of the region.
\syntax
\begin{ccppspecific}
The syntax of the \code{target}~\code{data} construct is as follows:

\begin{ompcPragma}
#pragma omp target data \plc{clause[ [ [},\plc{] clause] ... ] new-line}
    \plc{structured-block}
\end{ompcPragma}

\needspace{10\baselineskip}
\begin{samepage}
where \plc{clause} is one of the following:

\begin{indentedcodelist}
if(\plc{[} target data :\plc{] scalar-expression})
device(\plc{integer-expression})
map(\plc{[[map-type-modifier[},\plc{] [map-type-modifier[},\plc{] ...] map-type}:\plc{ ] locator-list})
use_device_ptr(\plc{ptr-list})
use_device_addr(\plc{list})
\end{indentedcodelist}
\end{samepage}
\end{ccppspecific}
\medskip

\begin{fortranspecific}
The syntax of the \code{target}~\code{data} construct is as follows:

\begin{ompfPragma}
!$omp target data \plc{clause[ [ [},\plc{] clause] ... ]}
    \plc{structured-block}
!$omp end target data
\end{ompfPragma}

where \plc{clause} is one of the following:

\begin{indentedcodelist}
if(\plc{[} target data :\plc{] scalar-logical-expression})
device(\plc{scalar-integer-expression})
map(\plc{[[map-type-modifier[},\plc{] [map-type-modifier[},\plc{] ...] map-type}:\plc{ ] locator-list})
use_device_ptr(\plc{ptr-list})
use_device_addr(\plc{list})
\end{indentedcodelist}

The \code{end}~\code{target}~\code{data} directive denotes the end of the \code{target}~\code{data} construct.
\end{fortranspecific}

\binding
The binding task set for a \code{target}~\code{data} region is the generating task. The
\code{target}~\code{data} region binds to the region of the generating task.

\descr
When a \code{target}~\code{data} construct is encountered, the encountering task executes the region. If there is no \code{device} clause, the default device is determined by the \plc{default-device-var} ICV. Variables are mapped for the extent of the region, according to any data-mapping attribute clauses, from the data environment of the encountering task to the device data environment. When an \code{if} clause is present and the \code{if} clause expression evaluates to \plc{false}, the device is the host.

Pointers that appear in a \code{use_device_ptr} clause are privatized and the
device pointer to the corresponding list items in the device data environment
are assigned into the private versions.  

List items that appear in a \code{use_device_addr} clause have the address of
the corresponding object in the device data environment inside the construct.
For objects, any reference to the value of the object will be to the
corresponding object on the device, while references to the address will result
in a valid device address pointing to that object.  Array sections privatize the
base of the array section and assign the private copy to the address of the
corresponding array section in the device data environment.  

If one or more of the \code{use_device_ptr} or
\code{use_device_addr} clauses and one or more \code{map} clauses are present on the
same construct, the address conversions of \code{use_device_addr} and
\code{use_device_ptr} clauses will occur as if performed after all variables
are mapped according to those \code{map} clauses.

\def\omptTargetRegionEvents#1#2#3
{
The \plc{#3-begin} event occurs when a thread enters a
#1 region.

The \plc{#3-end} event occurs when a thread exits a
#1 region.
}

\def\omptTargetRegionTools#1#2#3#4
{
A thread dispatches a registered \code{ompt_callback_target}
callback for each occurrence of a \plc{#3-begin} and
\plc{#3-end} event in that thread in the context of #4.
The callback has type signature
\code{ompt_callback_target_t}. The callback receives
\code{ompt_scope_begin} or \code{ompt_scope_end}
as its \plc{endpoint} argument, as appropriate, and
\code{#2} as its \plc{kind} argument.
}

\def\omptTargetRegion#1#2#3#4
{
\events
\omptTargetRegionEvents{#1}{#2}{#3}

\tools
\omptTargetRegionTools{#1}{#2}{#3}{#4}
}
\omptTargetRegion{\code{target}~\code{data}}{ompt_target_enter_data}{target-data}{the task encountering the construct}

\restrictions
\begin{itemize}
    \item A program must not depend on any ordering of the evaluations of the clauses of the
    \code{target}~\code{data} directive, or on any side effects of the evaluations of the clauses.

  \item At most one \code{device} clause can appear on the directive. The \code{device} clause expression  
      must evaluate to a non-negative integer value less than the value
    of \code{omp_get_num_devices()}.

  \item At most one \code{if} clause can appear on the directive.
  \item A \plc{map-type} in a \code{map} clause must be \code{to}, \code{from}, \code{tofrom} or \code{alloc}.
  \item At least one \code{map}, \code{use_device_addr} or \code{use_device_ptr}
    clause must appear on the directive.


  \item A list item in a \code{use_device_ptr} clause must hold the address of
    an object that has a corresponding list item in the device data environment.
  \item A list item in a \code{use_device_addr} clause must have a
    corresponding list item in the device data environment.

  \item A list item that specifies a given variable may not appear in more than
	  one \code{use_device_ptr} clause.

  \item A reference to a list item in a \code{use_device_addr} clause must be to
    the address of the list item.

\end{itemize}


\crossreferences
\begin{itemize}
\item \plc{default-device-var}, see
\specref{sec:Internal Control Variables}.

\item \code{if} Clause, see \specref{sec:if Clause}.

\item \code{map} clause, see
\specref{subsec:map Clause}.

\item \code{omp_get_num_devices} routine, see \specref{subsec:omp_get_num_devices}.

\item \code{ompt_callback_target_t}, see
\specref{sec:ompt_callback_target_t}.

\end{itemize}










\subsection{\hcode{target}~\hcode{enter}~\hcode{data} Construct}
\label{subsec:target enter data Construct}
\index{constructs!target enter data@{\code{target}~\code{enter}~\code{data}}}
\index{device data environments}
\summary
The \code{target}~\code{enter}~\code{data} directive specifies that variables are mapped to a device data environment. The \code{target}~\code{enter}~\code{data} directive is a stand-alone directive.
\syntax
\begin{ccppspecific}
The syntax of the \code{target}~\code{enter}~\code{data} construct is as follows:
\begin{ompcPragma}
#pragma omp target enter data \plc{[ clause[ [},\plc{] clause]...] new-line}
\end{ompcPragma}
where \plc{clause} is one of the following:
\begin{indentedcodelist}
if(\plc{[} target enter data :\plc{] scalar-expression})
device(\plc{integer-expression})
map(\plc{[map-type-modifier[},\plc{] [map-type-modifier[},\plc{] ...] map-type}:\plc{ locator-list})
depend(\plc{[depend-modifier},\plc{] dependence-type }:\plc{ locator-list})
nowait
\end{indentedcodelist}
\end{ccppspecific}
\begin{fortranspecific}
The syntax of the \code{target}~\code{enter}~\code{data} is as follows:
\begin{ompfPragma}
!$omp target enter data \plc{[ clause[ [},\plc{] clause]...]}
\end{ompfPragma}
where clause is one of the following:
\begin{indentedcodelist}
if(\plc{[} target enter data :\plc{] scalar-logical-expression})
device(\plc{scalar-integer-expression})
map(\plc{[map-type-modifier[},\plc{] [map-type-modifier[},\plc{] ...] map-type}:\plc{ locator-list})
depend(\plc{[depend-modifier},\plc{] dependence-type }:\plc{ locator-list})
nowait
\end{indentedcodelist}
\end{fortranspecific}

\binding
The binding task set for a \code{target}~\code{enter}~\code{data} region is
the generating task, which is the \plc{target task} generated by the
\code{target}~\code{enter}~\code{data} construct. The
\code{target}~\code{enter}~\code{data} region binds to the corresponding
\plc{target task} region.

\descr
When a \code{target}~\code{enter}~\code{data} construct is encountered, the list items are mapped to the device data environment according to the \code{map} clause semantics.

The \code{target}~\code{enter}~\code{data} construct is a task generating construct.  The generated task is a \plc{target task}.  The generated task region encloses the \code{target}~\code{enter}~\code{data} region.

All clauses are evaluated when the \code{target}~\code{enter}~\code{data} construct is encountered.  The data environment of the \plc{target task} is created according to the data-sharing attribute clauses on the \code{target}~\code{enter}~\code{data} construct, per-data environment ICVs, and any default data-sharing attribute rules that apply to the \code{target}~\code{enter}~\code{data} construct.  A variable that is mapped in the \code{target}~\code{enter}~\code{data} construct has a default data-sharing attribute of shared in the data environment of the \plc{target task}.

Assignment operations associated with mapping a variable (see \specref{subsec:map Clause}) occur when the \plc{target task} executes.

If the \code{nowait} clause is present, execution of the \plc{target task} may be deferred.  If the \code{nowait} clause is not present, the \plc{target task} is an included task.

If a \code{depend} clause is present, it is associated with the \plc{target task}.

If there is no \code{device} clause, the default device is determined by the \plc{default-device-var} ICV.

When an \code{if} clause is present and the \code{if} clause expression evaluates to \plc{false}, the device is the host.


%\tools
\def\omptTarget#1#2#3#4
{

\events
Events associated with a \plc{target task} are the same as for the \code{task} construct defined in \specref{subsec:task Construct}.

\omptTargetRegionEvents{#1}{#2}{#3}

\tools
Callbacks associated with events for \plc{target tasks} are the same as for the \code{task} construct defined in \specref{subsec:task Construct}.

\omptTargetRegionTools{#1}{#2}{#3}{#4}
}
\omptTarget{\code{target}~\code{enter}~\code{data}}{ompt_target_enter_data}{target-enter-data}{the target task on the host}



\restrictions
\begin{itemize}
\item A program must not depend on any ordering of the evaluations of the clauses of the \code{target}~\code{enter}~\code{data} directive, or on any side effects of the evaluations of the clauses.
\item At least one \code{map} clause must appear on the directive.

\item At most one \code{device} clause can appear on the
  directive. The \code{device} clause expression must evaluate to a
  non-negative integer value less than the value of
  \code{omp_get_num_devices()}.

\item At most one \code{if} clause can appear on the directive.
\item A \plc{map-type} must be specified in all \code{map} clauses and must be either \code{to} or \code{alloc}.
\item At most one \code{nowait} clause can appear on the directive.
\end{itemize}

\crossreferences
\begin{itemize}
\item \plc{default-device-var}, see \specref{subsec:ICV Descriptions}.

\item \code{task}, see \specref{subsec:task Construct}.

\item \code{task}~\code{scheduling}~\code{constraints},
      see \specref{subsec:Task Scheduling}.

\item \code{target}~\code{data}, see \specref{subsec:target data Construct}.

\item \code{target}~\code{exit}~\code{data},
      see \specref{subsec:target exit data Construct}.

\item \code{if} Clause, see \specref{sec:if Clause}.

\item \code{map} clause, see \specref{subsec:map Clause}.

\item \code{omp_get_num_devices} routine, see \specref{subsec:omp_get_num_devices}.

\item \code{ompt_callback_target_t}, see
\specref{sec:ompt_callback_target_t}.

\end{itemize}





\subsection{\hcode{target}~\hcode{exit}~\hcode{data} Construct}
\label{subsec:target exit data Construct}
\index{constructs!target exit data@{\code{target}~\code{exit}~\code{data}}}
\index{device data environments}
\summary
The \code{target}~\code{exit}~\code{data} directive specifies that list items are unmapped from a device data environment. The \code{target}~\code{exit}~\code{data} directive is a stand-alone directive.
\syntax
\begin{ccppspecific}
The syntax of the \code{target}~\code{exit}~\code{data} construct is as follows:
\begin{ompcPragma}
#pragma omp target exit data \plc{[ clause[ [},\plc{] clause]...] new-line}
\end{ompcPragma}
where \plc{clause} is one of the following:
\begin{indentedcodelist}
if(\plc{[} target exit data :\plc{] scalar-expression})
device(\plc{integer-expression})
map(\plc{[map-type-modifier[},\plc{] [map-type-modifier[},\plc{] ...] map-type}:\plc{ locator-list})
depend(\plc{[depend-modifier},\plc{] dependence-type }:\plc{ locator-list})
nowait
\end{indentedcodelist}
\end{ccppspecific}
\begin{fortranspecific}
The syntax of the \code{target}~\code{exit}~\code{data} is as follows:
\begin{ompfPragma}
!$omp target exit data \plc{[ clause[ [},\plc{] clause]...]}
\end{ompfPragma}
where clause is one of the following:
\begin{indentedcodelist}
if(\plc{[} target exit data :\plc{] scalar-logical-expression})
device(\plc{scalar-integer-expression})
map(\plc{[map-type-modifier[},\plc{] [map-type-modifier[},\plc{] ...] map-type}:\plc{ locator-list})
depend(\plc{[depend-modifier},\plc{] dependence-type }:\plc{ locator-list})
nowait
\end{indentedcodelist}
\end{fortranspecific}

\binding
The binding task set for a \code{target}~\code{exit}~\code{data} region is
the generating task, which is the \plc{target task} generated by the
\code{target}~\code{exit}~\code{data} construct. The
\code{target}~\code{exit}~\code{data} region binds to the corresponding
\plc{target task} region.

\descr
When a \code{target}~\code{exit}~\code{data} construct is encountered, the list items in the \code{map} clauses are unmapped from the device data environment according to the \code{map} clause semantics.

The \code{target}~\code{exit}~\code{data} construct is a task generating construct.  The generated task is a \plc{target task}.  The generated task region encloses the \code{target}~\code{exit}~\code{data} region.

All clauses are evaluated when the \code{target}~\code{exit}~\code{data} construct is encountered.  The data environment of the \plc{target task} is created according to the data-sharing attribute clauses on the \code{target}~\code{exit}~\code{data} construct, per-data environment ICVs, and any default data-sharing attribute rules that apply to the \code{target}~\code{exit}~\code{data} construct.  A variable that is mapped in the \code{target}~\code{exit}~\code{data} construct has a default data-sharing attribute of shared in the data environment of the \plc{target task}.

Assignment operations associated with mapping a variable (see \specref{subsec:map Clause}) occur when the \plc{target task} executes.

If the \code{nowait} clause is present, execution of the \plc{target task} may be deferred.  If the \code{nowait} clause is not present, the \plc{target task} is an included task.

If a \code{depend} clause is present, it is associated with the \plc{target task}.

If there is no \code{device} clause, the default device is determined by the \plc{default-device-var} ICV.

When an \code{if} clause is present and the \code{if} clause expression evaluates to \plc{false}, the device is the host.
%\tools
\omptTarget{\code{target}~\code{exit}~\code{data}}{ompt_target_exit_data}{target-exit}{the target task on the host}

\restrictions
\begin{itemize}
\item A program must not depend on any ordering of the evaluations of the clauses of the \code{target}~\code{exit}~\code{data} directive, or on any side effects of the evaluations of the clauses.
\item At least one \code{map} clause must appear on the directive.
\item At most one \code{device} clause can appear on the
  directive.  The \code{device} clause expression must evaluate to a
  non-negative integer value less than the value of
  \code{omp_get_num_devices()}.
\item At most one \code{if} clause can appear on the directive.
\item A \plc{map-type} must be specified in all \code{map} clauses and must be either \code{from}, \code{release}, or \code{delete}.
\item At most one \code{nowait} clause can appear on the directive.
\end{itemize}

\crossreferences
\begin{itemize}
\item \plc{default-device-var}, see \specref{subsec:ICV Descriptions}.

\item \code{task}, see \specref{subsec:task Construct}.

\item \code{task}~\code{scheduling}~\code{constraints},
      see \specref{subsec:Task Scheduling}.

\item \code{target}~\code{data}, see \specref{subsec:target data Construct}.

\item \code{target}~\code{enter}~\code{data},
      see \specref{subsec:target enter data Construct}.

\item \code{if} Clause, see \specref{sec:if Clause}.

\item \code{map} clause, see \specref{subsec:map Clause}.

\item \code{omp_get_num_devices} routine, see \specref{subsec:omp_get_num_devices}.

\item \code{ompt_callback_target_t}, see
\specref{sec:ompt_callback_target_t}.

\end{itemize}





%\pagebreak

\subsection{\hcode{target} Construct}
\index{target@{\code{target}}}
\index{constructs!target@{\code{target}}}
\index{device constructs!target@{\code{target}}}
\label{subsec:target Construct}
\summary
Map variables to a device data environment and execute the construct on that device.
\syntax
\begin{ccppspecific}
The syntax of the \code{target} construct is as follows:

\begin{ompcPragma}
#pragma omp target \plc{[clause[ [},\plc{] clause] ... ] new-line}
    \plc{structured-block}
\end{ompcPragma}

where \plc{clause} is one of the following:

\begin{indentedcodelist}
if(\plc{[} target :\plc{] scalar-expression})
device(\plc{[ device-modifier} :\plc{] integer-expression})
private(\plc{list})
firstprivate(\plc{list})
in_reduction(\plc{reduction-identifier }:\plc{ list})
map(\plc{[[map-type-modifier[},\plc{] [map-type-modifier[},\plc{] ...] map-type}:\plc{ ] locator-list})
is_device_ptr(\plc{list})
defaultmap(\plc{implicit-behavior[:variable-category]})
nowait
depend(\plc{[depend-modifier},\plc{] dependence-type }:\plc{ locator-list})
allocate([\plc{[allocator }:\plc{] list})
uses_allocators(\plc{allocator[}(\plc{allocator-traits-array})\plc{]}
	       \plc{[},\plc{allocator[}(\plc{allocator-traits-array})\plc{] ...]})
\end{indentedcodelist}

where \plc{device-modifier} is one of the following:
\begin{indentedcodelist}
ancestor
device_num
\end{indentedcodelist}

where \plc{allocator} is an identifier of \code{const omp_allocator_t *} type.

where \plc{allocator-traits-array} is an identifier of \code{const omp_alloctrait_t *} type.
\end{ccppspecific}

\begin{samepage}
\smallskip
\begin{fortranspecific}
The syntax of the \code{target} construct is as follows:

\begin{ompfPragma}
!$omp target \plc{[clause[ [},\plc{] clause] ... ]}
    \plc{structured-block}
!$omp end target
\end{ompfPragma}

where \plc{clause} is one of the following:

\begin{indentedcodelist}
if(\plc{[} target :\plc{] scalar-logical-expression})
device(\plc{[ device-modifier} :\plc{] scalar-integer-expression})
private(\plc{list})
firstprivate(\plc{list})
in_reduction(\plc{reduction-identifier }:\plc{ list})
map(\plc{[[map-type-modifier[},\plc{] [map-type-modifier[},\plc{] ...] map-type}:\plc{ ] locator-list})
is_device_ptr(\plc{list})
defaultmap(\plc{implicit-behavior[:variable-category]})
nowait
depend(\plc{[depend-modifier},\plc{] dependence-type }:\plc{ locator-list})
allocate(\plc{[allocator}:\plc{]list})
uses_allocators(\plc{allocator[}(\plc{allocator-traits-array})\plc{]}
	       \plc{[},\plc{allocator[}(\plc{allocator-traits-array})\plc{] ...]})
\end{indentedcodelist}

where \plc{device-modifier} is one of the following:
\begin{indentedcodelist}
ancestor
device_num
\end{indentedcodelist}

where \plc{allocator} is an integer expression of \code{omp_allocator_kind} \plc{kind}.

where \plc{allocator-traits-array} is an array of \code{type(omp_alloctrait)} type.

The \code{end}~\code{target} directive denotes the end of the \code{target} construct
\end{fortranspecific}

\end{samepage}

\binding
The binding task set for a \code{target} region is the
generating task, which is the \plc{target task} generated
by the \code{target} construct. The \code{target}
region binds to the corresponding \plc{target task} region.

\descr
The \code{target} construct provides a superset of the functionality
provided by the \code{target}~\code{data} directive, except for
the \code{use_device_ptr} clause.

The functionality added to the \code{target} directive is the inclusion of an executable region to be executed by a device. That is, the \code{target} directive is an executable directive.

The \code{target} construct is a task generating construct.  The generated task is a \plc{target task}.  The generated task region encloses the \code{target} region.

All clauses are evaluated when the \code{target} construct is encountered.
The data environment of the \plc{target task} is created according to the
data-sharing attribute clauses on the \code{target} construct, per-data
environment ICVs, and any default data-sharing attribute rules that apply to
the \code{target} construct.  If a variable or part of a variable is mapped by
the \code{target} construct and does not appear as a list item in an
\code{in_reduction} clause on the construct, the variable has a default
data-sharing attribute of shared in the data environment of the \plc{target
task}.

Assignment operations associated with mapping a variable (see \specref{subsec:map Clause}) occur when the \plc{target task} executes.

If a \code{device} clause in which the \code{device_num} \plc{device-modifier} appears is present on the construct, the
\code{device} clause expression specifies the device number of the target device. If
\plc{device-modifier} does not appear in the clause, the behavior of the
clause is as if \plc{device-modifier} is \code{device_num}.

If a \code{device} clause in which the \code{ancestor} \plc{device-modifier} appears is present on the
\code{target} construct and the \code{device} clause expression evaluates to 1,
execution of the \code{target} region occurs on the parent device of the
enclosing \code{target} region.  If the \code{target} construct is not
encountered in a \code{target} region, the current device is treated as the
parent device.  The encountering thread waits for completion of the
\code{target} region on the parent device before resuming. For any list item
that appears in a \code{map} clause on the same construct, if the
corresponding list item exists in the device data environment of the parent
device, it is treated as if it has a reference count of positive infinity.

If the \code{nowait} clause is present, execution of the \plc{target task} may be deferred.  If the \code{nowait} clause is not present, the \plc{target task} is an included task.

If a \code{depend} clause is present, it is associated with the \plc{target task}.

When an \code{if} clause is present and the \code{if} clause expression evaluates to \plc{false}, the \code{target} region is executed by the host device in the host data environment.

The \code{is_device_ptr} clause is used to indicate that a list item is a device
pointer already in the device data environment and that it should be used
directly.  Support for device pointers created outside of OpenMP, specifically
outside of the \code{omp_target_alloc} routine and the \code{use_device_ptr} clause,
is implementation defined.

If a function (C, C++, Fortran) or subroutine (Fortran) is referenced in a
\code{target} construct then that function or subroutine is treated as if its
name had appeared in a \code{to} clause on a \code{declare}~\code{target}
directive.

Each memory \plc{allocator} specified in the \code{uses_allocators} clause will be made available in the \code{target} region. For each non-predefined allocator that is specified, a new allocator handle will be associated with an allocator that is created with the specified \plc{traits} as if by a call to \code{omp_init_allocator} at the beginning of the \code{target} region. Each non-predefined allocator will be destroyed as if by a call to \code{omp_destroy_allocator} at the end of the \code{target} region.

\begin{ccppspecific}

If a list item in a \code{map} clause has a base pointer and it is a
scalar variable with a predetermined data-sharing attribute of firstprivate
(see \specref{subsubsec:Data-sharing Attribute Rules for Variables Referenced
in a Construct}), then on entry to the \code{target} region:

\begin{itemize}
\item If the list item is not a zero-length array section, the corresponding
private variable is initialized such that the corresponding list item in the
device data environment can be accessed through the pointer in the
\code{target} region.

\item If the list item is a zero-length array section, the corresponding
private variable is initialized such that the corresponding storage location
of the array section can be referenced through the pointer in the \code{target}
region. If the corresponding storage location is not present in the device
data environment, the corresponding private variable is initialized to NULL.
\end{itemize}

\end{ccppspecific}

%\tools

\events
\omptTargetRegionEvents{\code{target}}{ompt_target}{target}

The \plc{target-submit} event occurs prior to creating an initial task on a target device for a target
region.

% target task
\tools
\omptTargetRegionTools{\code{target}}{ompt_target}{target}{target task on the host}

A thread dispatches a registered \code{ompt_callback_target_submit}
callback for each occurrence of a \plc{target-submit} event
in that thread. The callback has type signature
\code{ompt_callback_target_submit_t}.

\restrictions
\begin{itemize}
\item If a \code{target}, \code{target}~\code{update},
\code{target}~\code{data}, \code{target}~\code{enter}~\code{data}, or
\code{target}~\code{exit}~\code{data} construct is encountered during
execution of a \code{target} region, the behavior is unspecified.

\item The result of an \code{omp_set_default_device},
\code{omp_get_default_device}, or \code{omp_get_num_devices}
routine called within a \code{target} region is unspecified.

\item The effect of an access to a \code{threadprivate} variable in a target region is
unspecified.

\item If a list item in a \code{map} clause is a structure element, any other
element of that structure that is referenced in the \code{target} construct
must also appear as a list item in a \code{map} clause.

\item A variable referenced in a \code{target} region but not the \code{target} construct that is not
declared in the \code{target} region must appear in a \code{declare}~\code{target} directive.

\item At most one \code{defaultmap} clause for each category can appear on the directive.

\item At most one \code{nowait} clause can appear on the directive.

\item A \plc{map-type} in a \code{map} clause must be \code{to}, \code{from}, \code{tofrom} or \code{alloc}.

\item A list item that appears in an \code{is_device_ptr} clause must be a valid device pointer in the device data environment.

\item At most one \code{device} clause can appear on the directive.  The
    \code{device} clause expression 
must evaluate to a non-negative integer value less than the value of \code{omp_get_num_devices()}.

\item If a \code{device} clause in which the \code{ancestor} \plc{device-modifier} appears is present
    on the construct, then the following restrictions apply:
\begin{itemize}
   \item A \code{requires} directive with the \code{reverse_offload}
       clause must be specified.
   \item The \code{device} clause expression must evaluate to 1.
   \item Only the \code{device}, \code{firstprivate}, \code{private},
       \code{defaultmap}, and \code{map} clauses may appear on the construct.
   \item No OpenMP constructs or calls to OpenMP API runtime routines are allowed
       inside the corresponding \code{target} region.
\end{itemize}

\item Memory allocators that do not appear in a \code{uses_allocators} clause cannot appear as an allocator in an \code{allocate} clause or be used in the \code{target} region unless a \code{requires} directive with the \code{dynamic_allocators} clause is present in the same compilation unit.

\item Memory allocators that appear in a \code{uses_allocators} clause cannot appear in other data-sharing attribute clauses or data-mapping attribute clauses in the same construct.

\item Predefined allocators appearing in a \code{uses_allocators} clause cannot have \plc{traits} specified.

\item Non-predefined allocators appearing in a \code{uses_allocators} clause must have \plc{traits} specified.

\item Arrays containing allocators traits that appear in a \code{uses_allocators} clause must be constant arrays, have constant values and be defined in the same scope as the construct in which the clause appears. 

\item Any IEEE floating-point exception status flag, halting mode, or rounding mode set prior to a \code{target} region is unspecified in the region.

\item Any IEEE floating-point exception status flag, halting mode, or rounding mode set in a \code{target} region is unspecified upon exiting the region.

\begin{ccppspecific}
\item An attached pointer may not be modified in a \code{target} region.
\end{ccppspecific}

\begin{cspecific}
\item A list item that appears in an \code{is_device_ptr} clause must have a type of pointer or array.
\end{cspecific}

\begin{cppspecific}
\item A list item that appears in an \code{is_device_ptr} clause must have a type
  of pointer, array, reference to pointer or reference to array.
\item The effect of invoking a virtual member function of an object on a device other than the device on which the object was constructed is implementation defined.
\item A throw executed inside a \code{target} region must cause execution to resume within the
same \code{target} region, and the same thread that threw the exception must catch it.
\end{cppspecific}

\begin{fortranspecific}
\item A list item that appears in an \code{is_device_ptr} clause
  must be a dummy argument that does not have the \code{ALLOCATABLE},
  \code{POINTER} or \code{VALUE} attribute.
\item If a list item in a \code{map} clause is an array section, and the array section is derived from a variable with a \code{POINTER} or \code{ALLOCATABLE} attribute then the behavior is unspecified if the corresponding list item's variable is modified in the region.
\end{fortranspecific}
\end{itemize}

\crossreferences
\begin{itemize}
\item \plc{default-device-var}, see
\specref{sec:Internal Control Variables}.

\item \code{task} construct, see
\specref{subsec:task Construct}.

\item \code{task} scheduling constraints, see
\specref{subsec:Task Scheduling}

\item Memory allocators, see \specref{subsec:Memory Allocators}.

\item \code{target}~\code{data} construct, see
\specref{subsec:target data Construct}.

\item \code{if} Clause, see \specref{sec:if Clause}.

\item \code{private} and \code{firstprivate} clauses, see
\specref{subsec:Data-Sharing Attribute Clauses}.

\item Data-Mapping Attribute Rules and Clauses, see
\specref{subsec:Data-Mapping Attribute Rules, Clauses, and Directives}.

\item \code{omp_get_num_devices} routine, see \specref{subsec:omp_get_num_devices}.

\item \code{omp_set_default_allocator} routine, see \specref{subsec:omp_set_default_allocator}.

\item \code{omp_get_default_allocator} routine, see \specref{subsec:omp_get_default_allocator}.

\item \code{omp_alloctrait_t} and \code{omp_alloctrait} types, see \specref{subsec:Memory Management Types}.

\item \code{ompt_callback_target_t}, see
\specref{sec:ompt_callback_target_t}.

\item \code{ompt_callback_target_submit_t},
\specref{sec:ompt_callback_target_submit_t}.

\end{itemize}








\subsection{\hcode{target}~\hcode{update} Construct}
\index{target update@{\code{target}~\code{update}}}
\index{constructs!target update@{\code{target}~\code{update}}}
\index{device constructs!target update@{\code{target}~\code{update}}}
\label{subsec:target update Construct}
\summary
The \code{target}~\code{update} directive makes the corresponding list items in the device data
environment consistent with their original list items, according to the specified motion
clauses. The \code{target}~\code{update} construct is a stand-alone directive.


\syntax
\begin{ccppspecific}
The syntax of the \code{target}~\code{update} construct is as follows:

\begin{ompcPragma}
#pragma omp target update \plc{clause[ [ [},\plc{] clause] ... ] new-line}
\end{ompcPragma}
where \plc{clause} is either \plc{motion-clause} or one of the following:

\begin{indentedcodelist}
if(\plc{[} target update :\plc{] scalar-expression})
device(\plc{integer-expression})
nowait
depend(\plc{[depend-modifier},\plc{] dependence-type }:\plc{ locator-list})
\end{indentedcodelist}

and \plc{motion-clause} is one of the following:

\begin{indentedcodelist}
to([mapper(\plc{mapper-identifier}):]\plc{locator-list})
from([mapper(\plc{mapper-identifier}):]\plc{locator-list})
\end{indentedcodelist}
\end{ccppspecific}

\begin{fortranspecific}
The syntax of the \code{target}~\code{update} construct is as follows:

\begin{ompfPragma}
!$omp target update \plc{clause[ [ [},\plc{] clause] ... ]}
\end{ompfPragma}

where \plc{clause} is either \plc{motion-clause} or one of the following:

\begin{indentedcodelist}
if(\plc{[}target update :\plc{] scalar-logical-expression})
device(\plc{scalar-integer-expression})
nowait
depend(\plc{[depend-modifier},\plc{] dependence-type }:\plc{ locator-list})
\end{indentedcodelist}

and \plc{motion-clause} is one of the following:

\begin{indentedcodelist}
to([mapper(\plc{mapper-identifier}):]\plc{locator-list})
from([mapper(\plc{mapper-identifier}):]\plc{locator-list})
\end{indentedcodelist}
\end{fortranspecific}


\binding
The binding task set for a \code{target}~\code{update} region is the
generating task, which is the \plc{target task} generated
by the \code{target}~\code{update} construct. The \code{target}~\code{update}
region binds to the corresponding \plc{target task} region.

\descr
For each list item in a \code{to} or \code{from} clause there is a corresponding list item and an
original list item. If the corresponding list item is not present in the device data environment then no assignment occurs to or from the original list item. Otherwise, each corresponding list item in the
device data environment has an original list item in the current task's data
environment.  If a \code{mapper()} modifier appears in a \code{to} clause, each list item is
replaced with the list items that the given mapper specifies are to be mapped
with a \code{to} or \code{tofrom} map-type. If a \code{mapper()} modifier
appears in a \code{from} clause, each list item is replaced with the list items
that the given mapper specifies are to be mapped with a \code{from} or
\code{tofrom} map-type.

For each list item in a \code{from} or a \code{to} clause: \begin{itemize}

    \item For each part of the list item that is an attached pointer:  \begin{itemize}

        \item On exit from the region that part of the original list item will
        have the value it had on entry to the region;

        \item On exit from the region that part of the corresponding list item will
        have the value it had on entry to the region;

   \end{itemize}

    \item For each part of the list item that is not an attached pointer: \begin{itemize}

        \item If the clause is \code{from}, the value of that part of the
        corresponding list item is assigned to that part of the original list
       item;

        \item If the clause is \code{to}, the value of that part of the
        original list item is assigned to that part of the corresponding list
        item.

   \end{itemize}

   \item To avoid race conditions: \begin{itemize}

        \item Concurrent reads or updates of any part
        of the original list item must be synchronized with the update of the
        original list item that occurs as a result of the \code{from} clause;

        \item Concurrent reads or updates of any part
        of the corresponding list item must be synchronized with the update of the
        corresponding list item that occurs as a result of the \code{to} clause.

   \end{itemize}

\end{itemize}


\begin{ccppspecific}
The list items that appear in the \code{to} or \code{from} clauses may
use shape-operators.
\end{ccppspecific}

The list items that appear in the \code{to} or \code{from} clauses may
include array sections with \plc{stride} expressions.

The \code{target}~\code{update} construct is a task generating construct.  The generated task is a \plc{target task}.  The generated task region encloses the \code{target}~\code{update} region.

All clauses are evaluated when the \code{target}~\code{update} construct is encountered.  The data environment of the \plc{target task} is created according to the data-sharing attribute clauses on the \code{target}~\code{update} construct, per-data environment ICVs, and any default data-sharing attribute rules that apply to the \code{target}~\code{update} construct.  A variable that is mapped in the \code{target}~\code{update} construct has a default data-sharing attribute of shared in the data environment of the \plc{target task}.

Assignment operations associated with mapping a variable (see \specref{subsec:map Clause}) occur when the \plc{target task} executes.

If the \code{nowait} clause is present, execution of the \plc{target task} may be deferred.  If the \code{nowait} clause is not present, the \plc{target task} is an included task.

If a \code{depend} clause is present, it is associated with the \plc{target task}.

The device is specified in the \code{device} clause. If there is no \code{device} clause, the device
is determined by the \plc{default-device-var} ICV. When an \code{if} clause is present and the \code{if}
clause expression evaluates to \plc{false} then no assignments occur.

%\tools
\omptTarget{\code{target}~\code{update}}{ompt_target_update}{target-update}{the target task on the host}

\restrictions
\begin{itemize}
\item A program must not depend on any ordering of the evaluations of the clauses of the
\code{target}~\code{update} directive, or on any side effects of the evaluations of the clauses.

\item At least one \plc{motion-clause} must be specified.

\item A list item can only appear in a \code{to} or \code{from} clause, but not both.

\item A list item in a \code{to} or \code{from} clause must have a mappable type.

\item At most one \code{device} clause can appear on the directive. The
    \code{device} clause expression
      must evaluate to a non-negative integer value less than the value
      of \code{omp_get_num_devices()}.

\item At most one \code{if} clause can appear on the directive.

\item At most one \code{nowait} clause can appear on the directive.
\end{itemize}

\crossreferences
\begin{itemize}
\item \plc{default-device-var}, see
\specref{sec:Internal Control Variables}.

\item Array shaping,
\specref{sec:Array Shaping}

\item Array sections,
\specref{sec:Array Sections}

\item \code{task} construct, see
\specref{subsec:task Construct}.

\item \code{task} scheduling constraints, see
\specref{subsec:Task Scheduling}

\item \code{target}~\code{data}, see
\specref{subsec:target data Construct}.

\item \code{if} Clause, see \specref{sec:if Clause}.

\item \code{omp_get_num_devices} routine, see \specref{subsec:omp_get_num_devices}.

\item \code{ompt_callback_task_create_t}, see
\specref{sec:ompt_callback_task_create_t}.

\item \code{ompt_callback_target_t}, see
\specref{sec:ompt_callback_target_t}.


\end{itemize}










\subsection{\hcode{declare}~\hcode{target} Directive}
\index{declare target@{\code{declare}~\code{target}}}
\index{directives!declare target@{\code{declare}~\code{target}}}
\index{constructs!declare target@{\code{declare}~\code{target}}}
\index{device constructs!declare target@{\code{declare}~\code{target}}}
\label{subsec:declare target Directive}
\summary
The \code{declare}~\code{target} directive specifies that variables,
functions (C, C++ and Fortran), and subroutines (Fortran) are mapped
to a device. The \code{declare}~\code{target} directive is a declarative
directive.

\syntax
\begin{ccppspecific}
The syntax of the \code{declare}~\code{target} directive takes either of
the following forms:

\begin{ompcPragma}
#pragma omp declare target \plc{new-line}
\plc{declaration-definition-seq}
#pragma omp end declare target \plc{new-line}
\end{ompcPragma}

or

\begin{ompcPragma}
#pragma omp declare target (\plc{extended-list}) \plc{new-line}
\end{ompcPragma}

or

\begin{ompcPragma}
#pragma omp declare target \plc{clause[ [},\plc{] clause ... ] new-line}
\end{ompcPragma}

where \plc{clause} is one of the following:

\begin{indentedcodelist}
to(\plc{extended-list})
link(\plc{list})
implements(\plc{function-name})
device_type(host \textnormal{| nohost \textnormal{|} any})
\end{indentedcodelist}
\end{ccppspecific}

\begin{fortranspecific}
The syntax of the \code{declare}~\code{target} directive is as follows:

\begin{ompfPragma}
!$omp declare target (\plc{extended-list})
\end{ompfPragma}

or

\begin{ompfPragma}
!$omp declare target \plc{[clause[ [},\plc{] clause] ... ]}
\end{ompfPragma}

where \plc{clause} is one of the following:

\begin{indentedcodelist}
to(\plc{extended-list})
link(\plc{list})
implements(\plc{subroutine-name})
device_type(host \textnormal{| nohost \textnormal{|} any})
\end{indentedcodelist}
\end{fortranspecific}

\descr

The \code{declare} \code{target} directive ensures that procedures
and global variables can be executed or accessed on a device.
Variables are mapped for all device executions, or for specific
device executions through a \code{link} clause.

If an \plc{extended-list} is present with no clause then the \code{to}
clause is assumed.

The \code{implements} clause specifies that an alternate version of a procedure should be
used.

The \code{device_type} clause specifies if a version of the procedure should be made
available on host, device or both. If \code{host} is specified only host version
of the procedure is made available.  If \code{nohost} is specified then only device
version of the procedure is made available.  If \code{any} is specified then both
device and host version of the procedure is made available.

\begin{ccppspecific}
If a function is treated as if it appeared as a list item in a \code{to} clause
on a \code{declare}~\code{target} directive in the same translation unit in
which the definition of the function occurs then a device-specific version of
the function is created.

If a variable is treated as if it appeared as a list item in a \code{to} clause
on a \code{declare}~\code{target} directive in the same translation unit in
which the definition of the variable occurs then the original list item is
allocated a corresponding list item in the device data environment of all
devices.

All calls in \code{target} constructs to the function in the \code{implements} clause
are replaced by the function following the \code{declare} \code{target} constructs.
\end{ccppspecific}

\begin{fortranspecific}
If an internal procedure is treated as if it appeared as a
list item in a \code{to} clause on a \code{declare}~\code{target} directive
then a device-specific version of the procedure is created.

If a variable that is host associated is treated as if it appeared as a
list item in a \code{to} clause on a \code{declare}~\code{target} directive
then the original list item is allocated a corresponding list item in the
device data environment of all devices.



All calls in \code{target} constructs to the procedure in the \code{implements}
clause are replaced by the procedure in which \code{declare} \code{target} construct appeared.
\end{fortranspecific}

If a variable is treated as if it appeared as a list item in a \code{to}
clause on a \code{declare}~\code{target} directive then the corresponding list
item in the device data environment of each device is initialized once, in the
manner specified by the program, but at an unspecified point in the program
prior to the first reference to that list item.  The list item is never removed
from those device data environments as if its reference count is initialized to
positive infinity.

Including list items in a \code{link} clause supports compilation of
functions called in a \code{target} region that refer to the list
items.  They are not mapped by the \code{declare}~\code{target}
directive.  Instead, they are mapped according to the data mapping
rules described in \specref{subsec:Data-Mapping Attribute Rules, Clauses, and Directives}.

\begin{ccppspecific}
If a function is referenced in a function that is treated as if it appeared as
a list item in a \code{to} clause on a \code{declare}~\code{target} directive
then the name of the referenced function is treated as if it had appeared in a
\code{to} clause on a \code{declare}~\code{target} directive.

If a variable with static storage duration or a function (except
\plc{lambda} for C++) is referenced in the
initializer expression list of a variable with static storage duration that is
treated as if it appeared as a list item in a \code{to} clause on a
\code{declare}~\code{target} directive then the name of the referenced variable
or function is treated as if it had appeared in a \code{to} clause on a
\code{declare}~\code{target} directive.

The form of the \code{declare}~\code{target} directive that has no clauses
and requires a matching \code{end}~\code{declare}~\code{target} directive
defines an implicit \plc{extended-list} to an implicit \code{to}
clause. The implicit \plc{extended-list} consists of the variable names
of any variable declarations at file or namespace scope that appear between
the two directives and of the function names of any function declarations at
file, namespace or class scope that appear between the two directives.

The \plc{declaration-definition-seq} defined by a
\code{declare}~\code{target} directive and an
\code{end}~\code{declare}~\code{target} directive may contain
\code{declare}~\code{target} directives. If a \code{device_type} clause
is present on the contained \code{declare}~\code{target} directive, then its
argument determines which versions are made available.
If a list item appears both in an implicit and explicit list, the explicit
list determines which versions are made available.

\end{ccppspecific}

\begin{fortranspecific}
If a procedure is referenced in a procedure that is treated as if it appeared
as a list item in a \code{to} clause on a \code{declare}~\code{target}
directive then the name of the procedure is treated as if it had appeared in a
\code{to} clause on a \code{declare}~\code{target} directive.

If a \code{declare}~\code{target} does not have any clauses then an implicit
\plc{extended-list} to an implicit \code{to} clause of one item is formed from
the name of the enclosing subroutine subprogram, function subprogram or
interface body to which it applies.

If a \code{declare}~\code{target} directive has an \code{implements} or \code{device_type} clause
then any enclosed internal procedures cannot contain any \code{declare}~\code{target} directives.
The enclosing \code{device_type} clause implicitly applies to internal procedures.
\end{fortranspecific}


\restrictions
\begin{itemize}
\item A threadprivate variable cannot appear in a
      \code{declare}~\code{target} directive.

\item A variable declared in a \code{declare}~\code{target} directive
      must have a mappable type.

\item The same list item must not appear multiple times in clauses on the same directive.

\item The same list item must not explicitly appear in both a \code{to} clause on one
\code{declare}~\code{target} directive and a \code{link} clause on
another \code{declare}~\code{target} directive.

\item The \code{implements} clause can only appear with \code{device_type} clause.

\end{itemize}

\begin{cppspecific}
\begin{itemize}
\item The function names of overloaded functions or template functions
may only be specified within an implicit \plc{extended-list}.

\item If a \plc{lambda declaration and definition} appears between a
\code{declare target} directive and the matching \code{end declare target} directive, all the variables that are captured by the
\plc{lambda} expression must also be variables that are treated as
if they appear in a \code{to} clause.
\end{itemize}
\end{cppspecific}

\begin{fortranspecific}
\begin{itemize}
\item If a list item is a procedure name, it must not be a generic name,
      procedure pointer or entry name.

\item Any \code{declare}~\code{target} directive with clauses must appear
      in a specification part of a subroutine subprogram, function subprogram,
      program or module.

\item Any \code{declare}~\code{target} directive without clauses must appear
      in a specification part of a subroutine subprogram, function subprogram
      or interface body to which it applies.

\item If a \code{declare}~\code{target} directive is specified in an
      interface block for a procedure, it must match a
      \code{declare}~\code{target} directive in the definition of the
      procedure.

\item If an external procedure is a type-bound procedure of a derived
      type and a \code{declare}~\code{target} directive is specified in
      the definition of the external procedure, such a directive must
      appear in the interface block that is accessible to the derived
      type definition.

\item If any procedure is declared via a procedure declaration statement
      that is not in the type-bound procedure part of a derived-type
      definition, any \code{declare}~\code{target} with the procedure
      name must appear in the same specification part.

\item A variable that is part of another variable (as an array, structure
      element or type parameter inquiry) cannot appear in a
      \code{declare}~\code{target} directive.

\item The \code{declare}~\code{target} directive must appear in the
      declaration section of a scoping unit in which the common block
      or variable is declared. Although variables in common blocks can
      be accessed by use association or host association, common block
      names cannot. This means that a common block name specified in a
      \code{declare}~\code{target} directive must be declared to be a
      common block in the same scoping unit in which the
      \code{declare}~\code{target} directive appears.

\item If a \code{declare}~\code{target} directive specifying a common
      block name appears in one program unit, then such a directive must
      also appear in every other program unit that contains a \code{COMMON}
      statement specifying the same name. It must appear after the last
      such \code{COMMON} statement in the program unit.

\item If a list item is declared with the \code{BIND} attribute, the
      corresponding C entities must also be specified in a
      \code{declare}~\code{target} directive in the C program.

\item A blank common block cannot appear in a \code{declare}~\code{target}
      directive.

\item A variable can only appear in a \code{declare}~\code{target} directive
      in the scope in which it is declared. It must not be an element of a
      common block or appear in an \code{EQUIVALENCE} statement.

\item A variable that appears in a \code{declare}~\code{target} directive
      must be declared in the Fortran scope of a module or have the
      \code{SAVE} attribute, either explicitly or implicitly.
\end{itemize}
\end{fortranspecific}
