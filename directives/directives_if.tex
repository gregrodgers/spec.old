\section{\code{if} Clause}
\index{clauses!if Clause@{\code{if} Clause}}
\index{if Clause@{\code{if} Clause}}
\label{sec:if Clause}

\summary
The semantics of an \code{if} clause are described in the section on the
construct to which it applies. 
The \code{if} clause \plc{directive-name-modifier} names the associated
construct to which an expression applies, and is particularly useful for 
composite and combined constructs.

\syntax
\begin{ccppspecific}

The syntax of the \code{if} clause is as follows:

\begin{boxedcode}
if(\plc{[ directive-name-modifier }:\plc{] scalar-expression}) 
\end{boxedcode}
\end{ccppspecific}

\begin{fortranspecific}

The syntax of the \code{if} clause is as follows:

\begin{boxedcode}
if(\plc{[ directive-name-modifier }:\plc{] scalar-logical-expression}) 
\end{boxedcode}
\end{fortranspecific}

\descr
The effect of the \code{if} clause depends on the construct to 
which it is applied. For combined or composite constructs, the
\code{if} clause only applies to the semantics of the construct 
named in the \plc{directive-name-modifier} if one is specified. 
If no  \plc{directive-name-modifier} is specified for a combined 
or composite construct then the \code{if} clause applies to all 
constructs to which an \code{if} clause can apply.




