% This is an included file. See the master file for more information.
%
% When editing this file:
%
%    1. To change formatting, appearance, or style, please edit openmp.sty.
%
%    2. Custom commands and macros are defined in openmp.sty.
%
%    3. Be kind to other editors -- keep a consistent style by copying-and-pasting to
%       create new content.
%
%    4. We use semantic markup, e.g. (see openmp.sty for a full list):
%         \code{}     % for bold monospace keywords, code, operators, etc.
%         \plc{}      % for italic placeholder names, grammar, etc.
%
%    5. There are environments that provide special formatting, e.g. language bars.
%       Please use them whereever appropriate.  Examples are:
%
%         \begin{fortranspecific}
%         This is text that appears enclosed in blue language bars for Fortran.
%         \end{fortranspecific}
%
%         \begin{note}
%         This is a note.  The "Note -- " header appears automatically.
%         \end{note}
%
%    6. Other recommendations:
%         Use the convenience macros defined in openmp.sty for the minor headers
%         such as Comments, Syntax, etc.
%
%         To keep items together on the same page, prefer the use of 
%         \begin{samepage}.... Avoid \parbox for text blocks as it interrupts line numbering.
%         When possible, avoid \filbreak, \pagebreak, \newpage, \clearpage unless that's
%         what you mean. Use \needspace{} cautiously for troublesome paragraphs.
%
%         Avoid absolute lengths and measures in this file; use relative units when possible.
%         Vertical space can be relative to \baselineskip or ex units. Horizontal space
%         can be relative to \linewidth or em units.
%
%         Prefer \emph{} to italicize terminology, e.g.:
%             This is a \emph{definition}, not a placeholder.
%             This is a \plc{var-name}.
%


\section{\code{independent} Construct}
\index{independent@{\code{independent}}}
\index{constructs!independent@{\code{independent}}}
\label{sec:independent Construct}
\summary
An independent construct specifies that the iterations of one or more
associated loops are free of cross-iteration data dependencies and
synchronizations to protect against data races. An implementation may choose
to use this assertion to optimize and/or parallelize the iterations of the
associated loop or loops above and beyond what is specified using OpenMP
worksharing. The independent construct is an informational directive, which
asserts information to the compiler but provides no guaranteed semantics for
parallelizing the iterations of the associated loops.

\syntax
\begin{ccppspecific}
The syntax of the \code{independent} construct is as follows:
\begin{boxedcode}
\#pragma omp independent \plc{[clause[ [},\plc{] clause] ... ] new-line}
   \plc{structured-block}
\end{boxedcode}

where \plc{clause} is one of the following:

\begin{indentedcodelist}
private(\plc{list})
firstprivate(\plc{list})
reduction(\plc{reduction-identifier }:\plc{ list})
\end{indentedcodelist}
The \code{independent} directive places restrictions on the structure of all associated \plc{for-loops}. 
Specifically, all associated \plc{for-loops} must have \emph{canonical loop form} (see 
\specref{sec:Canonical Loop Form}).
\end{ccppspecific}

\begin{fortranspecific}
The syntax of the \code{independent} construct is as follows:

\begin{boxedcode}
!\$omp independent \plc{[clause[ [},\plc{] clause] ... ]}
   \plc{structured-block}
[!\$omp end independent]
\end{boxedcode}

\begin{samepage}
where \plc{clause} is one of the following:

\begin{indentedcodelist}
private(\plc{list})
firstprivate(\plc{list})
reduction(\plc{reduction-identifier }:\plc{ list})
\end{indentedcodelist}
\end{samepage}

If an \code{end}~\code{independent} directive is not specified, an
  \code{end}~\code{independent} directive is assumed at the end of the 
\plc{do-loops}.

Any associated \plc{do-loop} must be a \plc{do-construct} or an
\plc{inner-shared-do-construct} as defined by the Fortran standard. If
an \code{end}~\code{independent} directive follows a \plc{do-construct} in
which several loop statements share a \code{DO} termination statement,
then the directive can only be specified for the outermost of these
\code{DO} statements.

If any of the loop iteration variables would otherwise be shared, they are implicitly 
  made private on the \code{independent} construct.
\end{fortranspecific}

Loop induction variables for the associated loop(s) are firstprivate by
default.

If a private or firstprivate clause appears on the \code{independent} construct then
if an implementation chooses to parallelize the associated loops, the loop
iteration variables must be private to the team, thread, and/or simd lane that
executes the loop iteration.

If a reduction clause appears on the \code{independent} construct and an
implementation chooses to parallelize the associated loops it must perform a
parallel reduction with respect to the parallel execution units (teams,
threads, or SIMD lanes) used to execute the loop iterations.


\binding
The binding for a given \code{independent} region is determined by the implementation.
If the implementation distributes the associated loop iterations across teams
defined by an enclosing teams region, the binding is as described for the
\code{distribute} construct. If the implementation the workshares the associated loop
iterations across threads in the team of an enclosing \code{parallel} region, the
binding is as described for the loop construct. If the implementation
determines the the associated loop can be transformed into a SIMD loop, the
binding is as described for the SIMD construct. An implementation may choose
not to distribute, workshare, or SIMD vectorize the associated loops, in which
case the binding is inherited from the nearest enclosing distribute, loop, or
SIMD construct. An implementation may choose to combine distributing,
worksharing, and/or SIMD vectorizing, in which case the binding will be as
described for the most specific type of parallelism used.

\descr
The \code{independent} construct is associated with a loop nest consisting of one more
more loops that follow the directive. The construct asserts that all iterations
of the associated loop nest are independent with respect to each other; that is
to say they are guaranteed to be free of cross-iteration data dependencies and
data races. 

If more than one thread encounters an \code{independent} construct the compiler will
ensure that each iteration of the corresponding loops is executed exactly once.

An implementation may choose to treat an \code{independen} construct as a
\code{distribute},
Loop, and/or SIMD construct if it would be legal for a programmer to have
explicitly specified the selected construct. If the implementation chooses to
treat an \code{independent} construct as one of these constructs then the default data
sharing and binding will be as-if the programmer had specified the
implicitly-added construct.

The \code{levels} clause may be used to specify how many loops are associated with the
\code{independent} construct. The parameter of the \code{levels} clause must be
a constant positive integer expression. If a \code{levels} clause is present then for the next N
tightly-nested loops all iterations of the loops are independent of all other
iterations in the combined iteration space of the associated loops. 

If a \code{private} or \code{firstprivate} clause appears on the
\code{independent} construct and if
an implementation chooses to parallelize the associated loops, the variables
specified within the \code{private} or \code{firstprivate} clause must be privatized to the team, thread,
and/or simd lane that executes the loop iteration.

An \code{independent} construct may be nested within other OpenMP regions, but does
not increase the nesting level, as defined in [add crossref]. [Should we
restrict the nesting to only particular constructs for the first
implementation?]

\restrictions
Restrictions to the \code{independent} construct are as follows:

\begin{itemize}
  \item All loops associated with the independent construct must be perfectly
    nested; that is, there must be no intervening code nor any OpenMP directive
    between any two loops.
    
  \item An implementation may not generate additional thread teams or threads
    as the result of an independent construct.

  \item Only one levels clause can appear on an independent directive.
  
  \item The region generated by an independent construct may not contain a
    critical, single, master, or barrier construct, or any OpenMP lock
    routines.
  
  \item The region generated by an independent construct may not contain device
    directives.
    
  \item The region generated by an independent construct may not contain task
    generating or task synchronization constructs.
    
  \item An independent construct may not be orphaned.

\end{itemize}

\begin{ccppspecific}
\begin{itemize}
    
  \item The associated for-loops must be structured blocks.

  \item No statement can branch to any associated independent statement.

\end{itemize}

\end{ccppspecific}

\begin{fortranspecific}
\begin{itemize}
    
  \item The associated do-loops must be structured blocks.
    
  \item No statement in the associated loops other than the DO statements can cause
    a branch out of the loops.


\end{itemize}
\end{fortranspecific}

\crossreferences
\begin{itemize}
  \item The Loop construct, see \specref{sec:Loop Construct}.

  \item \code{distribute} constrict, see \specref{sec:distribute Construct}.

  \item SIMD constructs, see \specref{sec:SIMD Constructs}.


\end{itemize}
%% \pagebreak
