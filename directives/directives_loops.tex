% This is an included file. See the master file for more information.
%
% When editing this file:
%
%    1. To change formatting, appearance, or style, please edit openmp.sty.
%
%    2. Custom commands and macros are defined in openmp.sty.
%
%    3. Be kind to other editors -- keep a consistent style by copying-and-pasting to
%       create new content.
%
%    4. We use semantic markup, e.g. (see openmp.sty for a full list):
%         \code{}     % for bold monospace keywords, code, operators, etc.
%         \plc{}      % for italic placeholder names, grammar, etc.
%
%    5. There are environments that provide special formatting, e.g. language bars.
%       Please use them whereever appropriate.  Examples are:
%
%         \begin{fortranspecific}
%         This is text that appears enclosed in blue language bars for Fortran.
%         \end{fortranspecific}
%
%         \begin{note}
%         This is a note.  The "Note -- " header appears automatically.
%         \end{note}
%
%    6. Other recommendations:
%         Use the convenience macros defined in openmp.sty for the minor headers
%         such as Comments, Syntax, etc.
%
%         To keep items together on the same page, prefer the use of
%         \begin{samepage}.... Avoid \parbox for text blocks as it interrupts line numbering.
%         When possible, avoid \filbreak, \pagebreak, \newpage, \clearpage unless that's
%         what you mean. Use \needspace{} cautiously for troublesome paragraphs.
%
%         Avoid absolute lengths and measures in this file; use relative units when possible.
%         Vertical space can be relative to \baselineskip or ex units. Horizontal space
%         can be relative to \linewidth or em units.
%
%         Prefer \emph{} to italicize terminology, e.g.:
%             This is a \emph{definition}, not a placeholder.
%             This is a \plc{var-name}.
%

\section{Loop Constructs}
\label{sec:LoopConstructs}

\section{Canonical Loop Form}
\label{subsec:Canonical Loop Form}
\index{canonical loop form}
\begin{cppspecific}
A range-based for loop with random access iterator has a \emph{canonical loop form}.
\end{cppspecific}
\begin{ccppspecific}
A loop has \emph{canonical loop form} if it conforms to the following:

\medskip
\nolinenumbers
\renewcommand{\arraystretch}{1.0}
\tablefirsthead{%
    \hline\\[-2ex]
    \multicolumn{2}{l}{\hspace*{-5pt}%
        {\scode{for (}\splc{init-expr}\scode{; }\splc{test-expr}\scode{; }\splc{incr-expr}\scode{) }\splc{structured-block}}}\\[2pt]
    \hline\\[-2ex]
}
\tablehead{%
    \multicolumn{2}{l}{\small\slshape continued from previous page}\\
    \hline\\[-2ex]
}
\tabletail{%
    \hline\\[-2ex]
    \multicolumn{2}{l}{\small\slshape continued on next page}\\
}
\tablelasttail{\hline}
\begin{supertabular}{ p{0.8in} p{4.5in}}
    {\splc{init-expr}} & One of the following:\\
    & {\splc{var}} = {\splc{lb}}\\
    & {\splc{integer-type}} {\splc{var}} = {\splc{lb}}\\
    & {\splc{random-access-iterator-type}} {\splc{var}} = {\splc{lb}}\\
    & {\splc{pointer-type}} {\splc{var}} = {\splc{lb}}\\
    & \\
    {\splc{test-expr}} & One of the following:\\
    & {\splc{var}} {\splc{relational-op}} {\splc{b}}\\
    & {\splc{b}} {\splc{relational-op}} {\splc{var}}\\
    & \\
    {\splc{incr-expr}} & One of the following:\\
    & ++{\splc{var}}\\
    & {\splc{var}}++\\
    & {-} {-} {\splc{var}}\\
    & {\splc{var}} {-} {-}\\
    & {\splc{var}} += {\splc{incr}}\\
    & {\splc{var}} {-} = {\splc{incr}}\\
    & {\splc{var}} = {\splc{var}} + {\splc{incr}}\\
    & {\splc{var}} = {\splc{incr}} + {\splc{var}}\\
    & {\splc{var}} = {\splc{var}} - {\splc{incr}}\\
    & \\
    {\splc{var}} & One of the following:\\
    & \hspace{1.5em}A variable of a signed or unsigned integer type.\\
    & \hspace{1.5em}For C++, a variable of a random access iterator type.\\
    & \hspace{1.5em}For C, a variable of a pointer type.\\
    & If this variable would otherwise be shared, it is implicitly made private in the loop
    construct. This variable must not be modified during the execution of the {\splc{for-loop}}
    other than in {\splc{incr-expr}}. Unless the variable is specified {\scode{lastprivate}}
    or {\scode{linear}} on the loop construct, its value after the loop is unspecified.\\
    {\splc{relational-op}} & One of the following:\\
    & {\scode{<}}\\
    & {\scode{<=}}\\
    & {\scode{>}}\\
    & {\scode{>=}}\\
    & {\scode{!=}}\\
    & \\
    {\splc{lb}} and {\splc{b}} & Loop invariant expressions of a type compatible with the type of {\splc{var}}.\\
    & \\
    {\splc{incr}} & A loop invariant integer expression.\\
\end{supertabular}
\medskip

\linenumbers

The canonical form allows the iteration count of all associated loops to be computed
before executing the outermost loop. The computation is performed for each loop in an
integer type. This type is derived from the type of \plc{var} as follows:

\begin{itemize}
    \item If \plc{var} is of an integer type, then the type is the type of \plc{var}.

    \item For C++, if \plc{var} is of a random access iterator type, then the type is the type that
    would be used by \plc{std::distance} applied to variables of the type of \plc{var}.

    \item For C, if \plc{var} is of a pointer type, then the type is \code{ptrdiff_t}.
\end{itemize}

The behavior is unspecified if any intermediate result required to compute the iteration
count cannot be represented in the type determined above.

There is no implied synchronization during the evaluation of the \plc{lb}, \plc{b}, or \plc{incr}
expressions. It is unspecified whether, in what order, or how many times any side effects
within the \plc{lb}, \plc{b}, or \plc{incr} expressions occur.

\begin{note}
Random access iterators are required to support random access to elements in
constant time. Other iterators are precluded by the restrictions since they can take linear
time or offer limited functionality. It is therefore advisable to use tasks to parallelize
those cases.

% The word "Restrictions" seems out of place; was it meant to be a header outside of the Note?

%Restrictions
\end{note}

\restrictions
The following restrictions also apply:

\begin{itemize}
    \item If \plc{test-expr} is of the form \plc{var} \plc{relational-op}
    \plc{b} and \plc{relational-op} is < or <= then \plc{incr-expr} must cause \plc{var} to increase on each
    iteration of the loop. If \plc{test-expr} is of
    the form \plc{var} \plc{relational-op} \plc{b} and \plc{relational-op}
    is > or >= then \plc{incr-expr} must cause \plc{var} to decrease on each iteration of the loop.

    \item If \plc{test-expr} is of the form \plc{b} \plc{relational-op}
    \plc{var} and \plc{relational-op} is < or <= then
    \plc{incr-expr} must cause \plc{var} to decrease on each iteration of the loop. If \plc{test-expr} is of
    the form \plc{b} \plc{relational-op} \plc{var} and \plc{relational-op}
    is > or >= then \plc{incr-expr} must cause \plc{var} to increase on each iteration of the loop.

    \item If \plc{test-expr} is of the form \plc{b} != \plc{var} or
    \plc{var} != \plc{b} then \plc{incr-expr} must cause \plc{var}
    either to increase on each iteration of the loop or to decrease on
    each iteration of the loop.

    \item For C++, in the \code{simd} construct the only random access iterator types that are
    allowed for \plc{var} are pointer types.

    \item The \plc{b}, \plc{lb} and \plc{incr} expressions may not reference
    \plc{var} of any of the associated loops.

    \item If \plc{relational-op} is != and \plc{incr-expr} is of the
    form that has \plc{incr} then \plc{incr} must be a constant expression and
    evaluate to -1 or 1.
\end{itemize}
\end{ccppspecific}







\subsection{\hcode{distribute} Loop Constructs}
\label{sec:distribute Loop Constructs}

\subsubsection{\hcode{distribute} Construct}
\index{distribute@{\code{distribute}}}
\index{constructs!distribute@{\code{distribute}}}
\index{device constructs!distribute@{\code{distribute}}}
\label{subsec:distribute Construct}
\summary
The \code{distribute} construct specifies that the iterations of one or more loops will be
executed by the initial teams in the context of their implicit tasks. The iterations are
distributed across the initial threads of all initial teams that execute the \code{teams} region to
which the \code{distribute} region binds.

\syntax
\begin{ccppspecific}
The syntax of the \code{distribute} construct is as follows:

\begin{ompcPragma}
#pragma omp distribute \plc{[clause[ [},\plc{] clause] ... ] new-line}
   \plc{for-loops}
\end{ompcPragma}

Where \plc{clause} is one of the following:

\begin{indentedcodelist}
private(\plc{list})
firstprivate(\plc{list})
lastprivate(\plc{list})
collapse(\plc{n})
dist_schedule(\plc{kind[},\plc{ chunk_size]})
allocate(\plc{[allocator: ]}\plc{list})
\end{indentedcodelist}

All associated \plc{for-loops} must have the canonical form described in
\specref{subsec:Canonical Loop Form}.
\end{ccppspecific}
\bigskip

\begin{fortranspecific}
The syntax of the \code{distribute} construct is as follows:

\begin{ompfPragma}
!$omp distribute \plc{[clause[ [},\plc{] clause] ... ]}
   \plc{do-loops}
\plc{[}!$omp end distribute\plc{]}
\end{ompfPragma}

Where \plc{clause} is one of the following:

\begin{indentedcodelist}
private(\plc{list})
firstprivate(\plc{list})
lastprivate(\plc{list})
collapse(\plc{n})
dist_schedule(\plc{kind[},\plc{ chunk_size]})
allocate(\plc{[allocator: ]}\plc{list})
\end{indentedcodelist}

If an \code{end}~\code{distribute} directive is not specified, an \code{end}~\code{distribute} directive
is assumed at the end of the \plc{do-loops}.

Any associated \plc{do-loop} must be a \plc{do-construct} or an
\plc{inner-shared-do-construct} as defined by the Fortran standard. If an
\code{end}~\code{distribute} directive follows a \plc{do-construct} in which several loop statements share a \code{DO}
termination statement, then the directive can only be specified for the outermost of these
\code{DO} statements.
\end{fortranspecific}

\begin{samepage}

\binding
The binding thread set for a \code{distribute} region is the set of initial
threads executing an enclosing \code{teams} region. A \code{distribute} region
binds to this \code{teams} region.
%Only the threads executing the binding \code{teams} region participate in the execution of the loop iterations.

\descr
The \code{distribute} construct is associated with a loop nest consisting of one or more
loops that follow the directive.

There is no implicit barrier at the end of a \code{distribute} construct.
To avoid data races the original
list items modified due to \code{lastprivate} or \code{linear} clauses
should not be accessed between the end of the \code{distribute} construct and the end
of the \code{teams} region to which the \code{distribute} binds.

\end{samepage}

The \code{collapse} clause may be used to specify how many loops are
associated with the \code{distribute} construct.  The parameter of the
\code{collapse} clause must be a constant positive integer expression.
If no \code{collapse} clause is present or its paraemter is 1, the
only loop that is associated with the \code{distribute} construct is
the one that immediately follows the \code{distribute} construct.  If
a \code{collapse} clause is specified with a parameter value greater
than 1 and more than one loop is associated with the \code{distribute}
construct, then the iteration of all associated loops are collapsed
into one larger iteration space.  The sequential execution of the
iterations in all associated loops determines the order of the
iterations in the collapsed iteration space.

A distribute loop has logical iterations numbered 0,1,...,N-1 where N
is the number of loop iterations, and the logical numbering denotes
the sequence in which the iterations would be executed if the set of
associated loop(s) were executed sequentially.  At the beginning of
each logical iteration, the loop iteration variable of each associated
loop has the value that it would have if the set of the associated
loop(s) were executed sequentially.

If more than one loop is associated with the \code{distribute}
construct then the number of times that any intervening code between
any two associated loops will be executed is unspecified but will be
at least once per iteration of the loop enclosing the intervening code
and at most once per iteration of the innermost loop associated with
the construct.  If the iteration count of any loop that is associated with the
\code{distribute} construct and does not enclose the intervening code is zero
then the behavior is unspecified.


The iteration count for each associated loop is computed before entry to the outermost loop. If execution of any associated loop changes any of the values used to compute any of the iteration counts, then the behavior is unspecified.

The integer type (or kind, for Fortran) used to compute the iteration count for the collapsed loop is implementation defined.

If \code{dist_schedule} is specified, \plc{kind} must be \code{static}. If specified, iterations are
divided into chunks of size \plc{chunk_size}, chunks are assigned to the
initial teams of the league in
a round-robin fashion in the order of the initial team number. When no \plc{chunk_size} is specified,
the iteration space is divided into chunks that are approximately equal in size, and at
most one chunk is distributed to each initial team of the league. The size of the
chunks is unspecified in this case.

When no \code{dist_schedule} clause is specified, the schedule is implementation defined.

%\tools
\omptWorksharing{distribute}{ompt_work_distribute}

\restrictions
Restrictions to the \code{distribute} construct are as follows:

\begin{itemize}
\item The \code{distribute} construct inherits the restrictions of the loop construct.

\item Each \code{distribute} region must be encountered by the initial
threads of all initial teams in a league or by none at all.

\item The sequence of the \code{distribute} regions encountered must
be the same for every initial thread of every initial team in a league.

\item The region corresponding to the \code{distribute} construct must be
strictly nested inside a \code{teams} region.

\item A list item may appear in a \code{firstprivate} or \code{lastprivate} clause but not both.
\end{itemize}

\crossreferences
\begin{itemize}
\item loop construct, see
\specref{subsec:Loop Construct}.

\item \code{teams} construct, see
\specref{sec:teams Construct}

\item \code{ompt_work_distribute}, see \specref{sec:ompt_work_t}.
\item \code{ompt_callback_work_t}, see
\specref{sec:ompt_callback_work_t}.
\end{itemize}





\subsubsection{\hcode{distribute}~\hcode{simd} Construct}
\index{distribute simd@{\code{distribute}~\code{simd}}}
\index{constructs!distribute simd@{\code{distribute}~\code{simd}}}
\index{device constructs!distribute simd@{\code{distribute}~\code{simd}}}
\label{subsec:distribute simd Construct}
\summary
The \code{distribute}~\code{simd} construct specifies a loop that will be distributed across the
master threads of the \code{teams} region and executed concurrently using SIMD instructions. The \code{distribute}~\code{simd} construct is a composite construct.

\syntax
The syntax of the \code{distribute}~\code{simd} construct is as follows:

\begin{ccppspecific}
\begin{ompcPragma}
#pragma omp distribute simd \plc{[clause[ [},\plc{] clause] ... ] newline}
   \plc{for-loops}
\end{ompcPragma}

where \plc{clause} can be any of the clauses accepted by the \code{distribute} or \code{simd}
directives with identical meanings and restrictions.
\end{ccppspecific}

\begin{fortranspecific}
\begin{ompfPragma}
!$omp distribute simd \plc{[clause[ [},\plc{] clause] ... ]}
   \plc{do-loops}
\plc{[}!$omp end distribute simd\plc{]}
\end{ompfPragma}

where \plc{clause} can be any of the clauses accepted by the \code{distribute} or \code{simd}
directives with identical meanings and restrictions.

If an \code{end}~\code{distribute}~\code{simd} directive is not specified, an \code{end}~\code{distribute}~\code{simd}
directive is assumed at the end of the \plc{do-loops}.
\end{fortranspecific}

\descr
The \code{distribute}~\code{simd} construct will first distribute the iterations of the associated
loop(s) according to the semantics of the \code{distribute} construct and any clauses that
apply to the distribute construct. The resulting chunks of iterations will then be
converted to a SIMD loop in a manner consistent with any clauses that apply to the
\code{simd} construct. The effect of any clause that applies to both constructs is as if it were applied to both constructs separately except the \code{collapse} clause, which is applied once.

\events

This composite construct generates the same events as the \code{distribute} construct.

\tools

This composite construct dispatches the same callbacks as the \code{distribute} construct.

\restrictions
\begin{itemize}
\item The restrictions for the \code{distribute} and \code{simd} constructs apply.

\item A list item may not appear in a \code{linear} clause, unless it is the loop iteration variable.

\item The \code{conditional} modifier may not appear in a \code{lastprivate} clause.
\end{itemize}

\crossreferences
\begin{itemize}
\item \code{simd} construct, see
\specref{subsec:simd Construct}.

\item \code{distribute} construct, see
\specref{subsec:distribute Construct}.

\item Data attribute clauses, see
\specref{subsec:Data-Sharing Attribute Clauses}.

\item Events and tool callbacks for the \code{distribute} construct, see
\specref{subsec:simd Construct}.

\end{itemize}









\subsubsection{Distribute Parallel Loop Construct}
\index{distribute parallel loop construct}
\index{constructs!distribute parallel loop}
\index{device constructs!distribute parallel loop}
\index{constructs!distribute parallel for@{\code{distribute}~\code{parallel}~\code{for}}}
\index{constructs!distribute parallel do@{\code{distribute}~\code{parallel}~\code{do}}}
\label{subsec:Distribute Parallel Loop Construct}
\summary
The distribute parallel loop construct specifies a loop that can be executed in parallel by
multiple threads that are members of multiple teams. The distribute parallel loop construct is a composite construct.

\syntax
The syntax of the distribute parallel loop construct is as follows:

\begin{ccppspecific}
\begin{ompcPragma}
#pragma omp distribute parallel for \plc{[clause[ [},\plc{] clause] ... ] newline}
    \plc{for-loops}
\end{ompcPragma}

where \plc{clause} can be any of the clauses accepted by the \code{distribute} or parallel loop
directives with identical meanings and restrictions.
\end{ccppspecific}

\begin{fortranspecific}
\begin{ompfPragma}
!$omp distribute parallel do \plc{[clause[ [},\plc{] clause] ... ]}
    \plc{do-loops}
\plc{[}!$omp end distribute parallel do\plc{]}
\end{ompfPragma}

where \plc{clause} can be any of the clauses accepted by the \code{distribute} or parallel loop
directives with identical meanings and restrictions.

If an \code{end}~\code{distribute}~\code{parallel}~\code{do} directive is not specified, an
\code{end}~\code{distribute}~\code{parallel}~\code{do} directive is assumed at the end of the \plc{do-loops}.
\end{fortranspecific}

\descr
The distribute parallel loop construct will first distribute the iterations of the associated loop(s) into chunks according to the semantics of the \code{distribute} construct and any clauses that apply to the \code{distribute} construct. Each of these chunks will form a loop. Each resulting loop will then be distributed across the threads within the teams region to which the \code{distribute} construct binds in a manner consistent with any clauses that apply to the parallel loop construct.
The effect of any clause that applies to both constructs is as if it were applied to both constructs separately except the \code{collapse} clause, which is applied once.

\events

This composite construct generates the same events as the \code{distribute} and parallel loop constructs.

\tools

This composite construct dispatches the same callbacks as the \code{distribute} and parallel loop constructs.

%% BRONIS: Ugly trick to avoid page starting with end c/C++ sentinel
\vspace{-6pt}

\restrictions
\begin{itemize}
\item The restrictions for the \code{distribute} and parallel loop constructs apply.
\item No \code{ordered} clause can be specified.
\item No \code{linear} clause can be specified.
\item The \code{conditional} modifier may not appear in a \code{lastprivate} clause.
\end{itemize}

%% BRONIS: Another ugly trick to avoid page starting with end c/C++ sentinel
\vspace{-6pt}

\crossreferences
\begin{itemize}
\item \code{distribute} construct, see
\specref{subsec:distribute Construct}.

\item Parallel loop construct, see
\specref{subsec:Parallel Loop Construct}.

\item Data attribute clauses, see
\specref{subsec:Data-Sharing Attribute Clauses}.

\item Events and tool callbacks for \code{distribute} construct, see
\specref{subsec:distribute Construct}.

\item Events and tool callbacks for parallel loop construct, see
\specref{subsec:Parallel Loop Construct}.

\end{itemize}






\subsubsection{Distribute Parallel Loop SIMD Construct}
\label{subsec:Distribute Parallel Loop SIMD Construct}
\index{distribute parallel loop SIMD construct}
\index{constructs!distribute parallel loop SIMD}
\index{constructs!distribute parallel for simd@{\code{distribute}~\code{parallel}~\code{for}~\code{simd}}}
\index{constructs!distribute parallel do simd@{\code{distribute}~\code{parallel}~\code{do}~\code{simd}}}
\index{device constructs!distribute parallel loop SIMD}
\summary
The distribute parallel loop SIMD construct specifies a loop that can be executed
concurrently using SIMD instructions in parallel by multiple threads that are members
of multiple teams. The distribute parallel loop SIMD construct is a composite construct.

\syntax
\begin{ccppspecific}
The syntax of the distribute parallel loop SIMD construct is as follows:

\begin{ompcPragma}
#pragma omp distribute parallel for simd \plc{\}
            \plc{[clause[ [},\plc{] clause] ... ] newline}
    \plc{for-loops}
\end{ompcPragma}

where \plc{clause} can be any of the clauses accepted by the \code{distribute} or parallel loop
SIMD directives with identical meanings and restrictions
\end{ccppspecific}

\begin{fortranspecific}
The syntax of the distribute parallel loop SIMD construct is as follows:

\begin{ompfPragma}
!$omp distribute parallel do simd \plc{[clause[ [},\plc{] clause] ... ]}
    \plc{do-loops}
\plc{[}!$omp end distribute parallel do simd\plc{]}
\end{ompfPragma}

where \plc{clause} can be any of the clauses accepted by the \code{distribute} or parallel loop
SIMD directives with identical meanings and restrictions.

If an \code{end}~\code{distribute}~\code{parallel}~\code{do}~\code{simd} directive is not specified, an
\code{end}~\code{distribute}~\code{parallel}~\code{do}~\code{simd} directive is assumed at the end of the \plc{do-loops}.
\end{fortranspecific}

\descr
The distribute parallel loop SIMD construct will first distribute the iterations of the
associated loop(s) according to the semantics of the \code{distribute} construct and any
clauses that apply to the \code{distribute} construct. The resulting loops will then be
distributed across the threads contained within the \code{teams} region to which the
\code{distribute} construct binds in a manner consistent with any clauses that apply to the
parallel loop construct. The resulting chunks of iterations will then be converted to a
SIMD loop in a manner consistent with any clauses that apply to the \code{simd} construct.
The effect of any clause that applies to both constructs is as if it were applied to both constructs separately except the \code{collapse} clause, which is applied once.

\events

This composite construct generates the same events as the \code{distribute} and parallel loop constructs.

\tools

This composite construct dispatches the same callbacks as the \code{distribute} and parallel loop constructs.

\restrictions
\begin{itemize}
\item The restrictions for the \code{distribute} and parallel loop SIMD constructs apply.
\item No \code{ordered} clause can be specified.
\item A list item may not appear in a \code{linear} clause, unless it is the loop iteration variable.
\item The \code{conditional} modifier may not appear in a \code{lastprivate} clause.
\end{itemize}

\crossreferences
\begin{itemize}
\item \code{distribute} construct, see
\specref{subsec:distribute Construct}.

\item Parallel loop SIMD construct, see
\specref{subsec:Parallel Loop SIMD Construct}.

\item Data attribute clauses, see \specref{subsec:Data-Sharing Attribute Clauses}.

\item Events and tool callbacks for \code{distribute} construct, see
\specref{subsec:distribute Construct}.

\item Events and tool callbacks for parallel loop construct, see
\specref{subsec:Parallel Loop Construct}.

\end{itemize}


\subsection{concurrent Construct}
THE SECTION ``CONCURRENT CONSTRUCT'' SHOULD GO HERE.
