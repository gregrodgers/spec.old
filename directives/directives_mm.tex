\section{Memory Management Directives}
\label{sec:Memory Management Directives}
\index{memory management}
\index{directives!memory management directives}
\index{memory management directives!memory management directives}


\subsection{\code{allocate} Directive}
\index{allocate@{\code{allocate}}}
\index{directives!allocate@{\code{allocate}}}
\label{subsec:allocate Directive}
\summary

The \code{allocate} directive specifies how a set of variables are allocated. The \code{allocate} directive is a declarative directive if it is not associated with an allocation statement.

\syntax
\begin{ccppspecific}
The syntax of the \code{allocate} directive is as follows:

\begin{boxedcode}
\#pragma omp allocate(\plc{list}) \plc{[clause[ [ [},\plc{] clause] ... ]] new-line}
\end{boxedcode}

where \plc{clause} is one of the following:

\begin{indentedcodelist}
allocator(\plc{allocator})
\end{indentedcodelist}

where \plc{allocator} is a expression of the of \code{const omp\_allocator\_t *}.

\end{ccppspecific}
\medskip

\begin{fortranspecific}
The syntax of the \code{allocate} directive is as follows:

\begin{boxedcode}
!\$omp allocate(\plc{list}) \plc{[clause[ [ [},\plc{] clause] ... ]]}
\end{boxedcode}

or
\begin{boxedcode}
!\$omp allocate[(\plc{list})] \plc{clause[ [ [},\plc{] clause] ... ]}
[!\$omp allocate(\plc{list}) \plc{clause[ [ [},\plc{] clause] ... ]}]
[...]
   \plc{allocate statement}
\end{boxedcode}

where \plc{clause} is one of the following:

\begin{indentedcodelist}
allocator(\plc{allocator})
\end{indentedcodelist}

where \plc{allocator} is an integer expression of the \code{omp\_allocator\_kind} \plc{kind}.

\end{fortranspecific}

\descr

If the directive is not associated with a statement, the storage for each \plc{list item} that appears in the directive will be provided by an allocation through a memory allocator. If no clause is specified then the memory allocator specified by the \plc{def-allocator-var} ICV will be used. If the \code{allocator} clause is specified, the memory allocator specified in the clause will be used. The inability of a memory allocator to fulfil the allocation request for any list item results in undefined behavior.

The scope of this allocation is that of the list item in the base language. When the allocation reaches the end of the scope it will be deallocated using the same memory allocator used for the allocation. %If the execution leaves the scope in a manner not supported by the base language it is unspecified whether the deallocation happens or not.

\begin{fortranspecific}
If the directive is associated with an \code{allocate} statement, the allocation of the specified list items will be provided through an allocator. If no list item is specified then all variables allocated by the \code{allocate} statement will be provided by the allocator.
\end{fortranspecific}

\restrictions
\begin{itemize}
\item A variable that is part of another variable (as an array or structure element) cannot appear in an \code{allocate} directive.
\item The directive must appear in the same scope of the \plc{list item} declaration and before its first use.
\item At most one \code{allocator} clause can appear on the \code{allocate} directive.
\begin{ccppspecific}
\item If a list item has a static storage type, only predefined memory allocator variables can be used in the \code{allocator} clause.
\end{ccppspecific}
\begin{fortranspecific}
\item List items specified in the \code{allocate} directive must not have the \code{ALLOCATABLE} attribute unless the directive is associated with an \code{allocate} statement.
\item List items specified in an \code{allocate} directive that is associated with an \code{allocate} statement must be \code{ALLOCATABLE} variables or \code{POINTER} variables allocated by the \code{allocate} statement.
\item Multiple directives can only be associated with an \code{allocate} statement if list items are specified on each \code{allocate} directive.
\item If a list item has the \code{SAVE} attribute, or if the list item is a common block name, or if a list item is declared in the scope of a module, then only predefined memory allocator variables can be used in the \code{allocator} clause.
\end{fortranspecific}
\end{itemize}

\crossreferences
\begin{itemize}
\item Memory allocators, see \specref{sec:Memory Allocators}.
\item \code{omp\_allocator\_t} and \code{omp\_allocator\_kind}, see \specref{subsec:Memory Management Types}.
\item \plc{def-allocator-var} ICV, see \specref{subsec:ICV Descriptions}.
\end{itemize}

\subsection{\code{allocate} Clause}
\index{allocate@{\code{allocate}}}
\index{clauses!allocate@{\code{allocate}}}
\label{subsec:allocate Clause}
\summary
The \code{allocate} clause specifies the memory allocator to be used to obtain storage for private variables of a directive.

\syntax

The syntax of the \code{allocate} clause is as follows:

\begin{boxedcode}
allocate(\plc{[allocator}:\plc{] list})
\end{boxedcode}
\needspace{10\baselineskip}

\begin{ccppspecific}
where \plc{allocator} is an integer expression of the \code{const omp\_allocator\_t *} type.
\end{ccppspecific}
\begin{fortranspecific}
where \plc{allocator} is an integer expression of the \code{omp\_allocator\_kind} \plc{kind}.
\end{fortranspecific}

\descr

The storage for new list items that arise from \plc{list items} that appear in the directive will be provided by an allocation through a memory allocator. If an \plc{allocator} is specified in the clause this will be the memory allocator used for allocations. For all directives except for the \code{target} directive, if no \plc{allocator} is specified in the clause then the memory allocator specified by the \plc{def-allocator-var} ICV will be used. For the \code{target} directive, if no \plc{allocator} is specified in the clause then the memory allocator specified by the \plc{target-def-allocator-var} ICV will be used. The inability of a memory allocator to fulfill the allocation request for any list item results is implementation defined.

\restrictions
\begin{itemize}
\item For any list item that is specified in the \code{allocate} clause on a directive, a data-sharing attribute caluse that may create a private copy of that list item must be specified on the same directive.
\end{itemize}

\crossreferences
\begin{itemize}
\item Memory allocators, see \specref{sec:Memory Allocators}.
\item \code{omp\_allocator\_t} and \code{omp\_allocator\_kind}, see \specref{subsec:Memory Management Types}.
\item \plc{def-allocator-var} and \plc{target-def-allocator-var} ICVs, see \specref{subsec:ICV Descriptions}.
\end{itemize}
