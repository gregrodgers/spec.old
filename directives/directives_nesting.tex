% This is an included file. See the master file for more information.
%
% When editing this file:
%
%    1. To change formatting, appearance, or style, please edit openmp.sty.
%
%    2. Custom commands and macros are defined in openmp.sty.
%
%    3. Be kind to other editors -- keep a consistent style by copying-and-pasting to
%       create new content.
%
%    4. We use semantic markup, e.g. (see openmp.sty for a full list):
%         \code{}     % for bold monospace keywords, code, operators, etc.
%         \plc{}      % for italic placeholder names, grammar, etc.
%
%    5. There are environments that provide special formatting, e.g. language bars.
%       Please use them whereever appropriate.  Examples are:
%
%         \begin{fortranspecific}
%         This is text that appears enclosed in blue language bars for Fortran.
%         \end{fortranspecific}
%
%         \begin{note}
%         This is a note.  The "Note -- " header appears automatically.
%         \end{note}
%
%    6. Other recommendations:
%         Use the convenience macros defined in openmp.sty for the minor headers
%         such as Comments, Syntax, etc.
%
%         To keep items together on the same page, prefer the use of
%         \begin{samepage}.... Avoid \parbox for text blocks as it interrupts line numbering.
%         When possible, avoid \filbreak, \pagebreak, \newpage, \clearpage unless that's
%         what you mean. Use \needspace{} cautiously for troublesome paragraphs.
%
%         Avoid absolute lengths and measures in this file; use relative units when possible.
%         Vertical space can be relative to \baselineskip or ex units. Horizontal space
%         can be relative to \linewidth or em units.
%
%         Prefer \emph{} to italicize terminology, e.g.:
%             This is a \emph{definition}, not a placeholder.
%             This is a \plc{var-name}.
%


\section{Nesting of Regions}
\label{sec:Nesting of Regions}
\index{nesting of regions}
This section describes a set of restrictions on the nesting of regions. The restrictions on
nesting are as follows:

\begin{itemize}
\item A worksharing region may not be closely nested inside a worksharing, \code{task}, \code{taskloop},
\code{critical}, \code{ordered}, \code{atomic}, or \code{master} region.

\item A \code{barrier} region may not be closely nested inside a worksharing, \code{task}, \code{taskloop},
\code{critical}, \code{ordered}, \code{atomic}, or \code{master} region.

\item A \code{master} region may not be closely nested inside a worksharing,
\code{atomic}, \code{task}, or \code{taskloop} region.

\item An \code{ordered} region corresponding to an \code{ordered} construct without
any clause or with the \code{threads} or \code{depend} clause may not be closely
nested inside a \code{critical}, \code{ordered}, \code{atomic}, \code{task},
or \code{taskloop} region.

\item An \code{ordered} region corresponding to an \code{ordered} construct without
the \code{simd} clause specified must be closely nested inside a loop region.

\item An \code{ordered} region corresponding to an \code{ordered} construct with the
\code{simd} clause specified must be closely nested inside a \code{simd} or loop SIMD
region.

\item An \code{ordered} region corresponding to an \code{ordered} construct with
  both the \code{simd} and \code{threads} clauses must be closely nested inside
  a loop SIMD region or closely nested inside a loop and \code{simd} region.

\item A \code{critical} region may not be nested (closely or otherwise) inside a \code{critical}
region with the same name. This restriction is not sufficient to prevent
deadlock.

\item OpenMP constructs may not be encountered during execution of an
\code{atomic} region.

\item The only OpenMP constructs that can be encountered during execution of a
  \code{simd} (or loop SIMD) region are the \code{atomic} construct,
  \code{concurrent} construct and an \code{ordered} construct with the \code{simd} clause.

\item If a \code{target}, \code{target}~\code{update},
\code{target}~\code{data}, \code{target}~\code{enter}~\code{data}, or
\code{target}~\code{exit}~\code{data} construct is encountered during
execution of a \code{target} region, the behavior is unspecified.

\item If specified, a \code{teams} construct must be contained within a \code{target} construct. That
\code{target} construct must not contain any statements or directives outside of the \code{teams}
construct.

\item \code{distribute}, \code{distribute simd}, distribute parallel loop,
  distribute parallel loop SIMD, \code{concurrent}, and \code{parallel} regions, including any
\code{parallel} regions arising from combined constructs, are the only OpenMP regions
that may be strictly nested inside the \code{teams} region.

\item The region associated with the \code{distribute} construct must be
strictly nested inside a \code{teams} region.

\item If \plc{construct-type-clause} is \code{taskgroup}, the \code{cancel}
construct must be closely nested inside a \code{task} construct and the
\code{cancel} region must be closely nested inside a \code{taskgroup} region. If
\plc{construct-type-clause} is \code{sections}, the \code{cancel} construct
must be closely nested inside a \code{sections} or \code{section} construct.
Otherwise, the \code{cancel} construct must be closely
nested inside an OpenMP construct that matches the type specified in
\plc{construct-type-clause} of the \code{cancel} construct.

\item A \code{cancellation}~\code{point} construct for which
\plc{construct-type-clause} is \code{taskgroup} must be closely nested
inside a \code{task} construct, and the \code{cancellation}~\code{point}
region must be closely nested inside a \code{taskgroup} region. A
\code{cancellation}~\code{point} construct for which
\plc{construct-type-clause} is \code{sections} must be closely nested
inside a \code{sections} or \code{section} construct. Otherwise, a
\code{cancellation}~\code{point} construct must be closely nested inside
an OpenMP construct that matches the type specified in
\plc{construct-type-clause}.

\item A \code{concurrent} region must be closely nested within a \code{teams}
  construct, \code{distribute} construct, \code{parallel} construct, Loop
  construct, SIMD construct, \code{concurrent} construct, or regions resulting
  from the combined or composite constructs composed of these constructs.

\item Only the following directives may be nested within a \code{concurrent}
  construct: \code{parallel} construct, Loop construct, SIMD construct,
  \code{concurrent} construct, \code{single} construct, and any combined or
  compound construct composed of these constructs.
  %JLARKIN We'd like to go back and unrestrict atomic later.

\end{itemize}
