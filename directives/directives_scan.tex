% This is an included file. See the master file for more information.
%
% When editing this file:
%
%    1. To change formatting, appearance, or style, please edit openmp.sty.
%
%    2. Custom commands and macros are defined in openmp.sty.
%
%    3. Be kind to other editors -- keep a consistent style by copying-and-pasting to
%       create new content.
%
%    4. We use semantic markup, e.g. (see openmp.sty for a full list):
%         \code{}     % for bold monospace keywords, code, operators, etc.
%         \plc{}      % for italic placeholder names, grammar, etc.
%
%    5. There are environments that provide special formatting, e.g. language bars.
%       Please use them whereever appropriate.  Examples are:
%
%         \begin{fortranspecific}
%         This is text that appears enclosed in blue language bars for Fortran.
%         \end{fortranspecific}
%
%         \begin{note}
%         This is a note.  The "Note -- " header appears automatically.
%         \end{note}
%
%    6. Other recommendations:
%         Use the convenience macros defined in openmp.sty for the minor headers
%         such as Comments, Syntax, etc.
%
%         To keep items together on the same page, prefer the use of
%         \begin{samepage}.... Avoid \parbox for text blocks as it interrupts line numbering.
%         When possible, avoid \filbreak, \pagebreak, \newpage, \clearpage unless that's
%         what you mean. Use \needspace{} cautiously for troublesome paragraphs.
%
%         Avoid absolute lengths and measures in this file; use relative units when possible.
%         Vertical space can be relative to \baselineskip or ex units. Horizontal space
%         can be relative to \linewidth or em units.
%
%         Prefer \emph{} to italicize terminology, e.g.:
%             This is a \emph{definition}, not a placeholder.
%             This is a \plc{var-name}.
%

\section{\hcode{scan} Directive}
\index{directives!scan Directive@{\code{scan} DIrective}}
\index{scan Directive@{\code{scan} Directive}}
\label{sec:scan Directive}

\summary
The \code{scan} directive specifies that a scan computation is to be performed over the
values used on each iteration to update a list item.

%%%\begin{samepage}
\syntax
\begin{ccppspecific}
The syntax of the \code{scan} directive is as follows:

\begin{ompcPragma}
\plc{loop-directive}
\plc{for-loop-headers(s)}
{
   \plc{structured-block}
   #pragma omp scan \plc{clause} \plc{new-line}
   \plc{structured-block}
}
\end{ompcPragma}

where \plc{clause} is one of the following:
\begin{indentedcodelist}
inclusive(\plc{list})
exclusive(\plc{list})
\end{indentedcodelist}

and where \plc{loop-directive} is a \code{for}, \code{for}~\code{simd}, or
\code{simd} directive.

\end{ccppspecific}

\begin{fortranspecific}
The syntax of the \code{scan} directive is as follows:

\begin{ompfPragma}
\plc{loop-directive}
\plc{do-loop-header(s)}
   \plc{structured-block}
   !$omp scan \plc{clause}
   \plc{structured-block}
\plc{do-termination-stmts(s)}
\textsl{[}\plc{end-loop-directive}\textsl{]}
\end{ompfPragma}

where \plc{clause} is one of the following:
\begin{indentedcodelist}
inclusive(\plc{list})
exclusive(\plc{list})
\end{indentedcodelist}

and where \plc{loop-directive} (\plc{end-loop-directive}) is a \code{do}
(\code{end}~\code{do}), \code{do}~\code{simd}
(\code{end}~\code{do}~\code{simd}), or
\code{simd} (\code{end}~\code{simd}) directive.

\end{fortranspecific}
%%%\end{samepage}

\descr
The \code{scan} directive may appear in the body of a loop or loop nest
associated with an enclosing loop, loop SIMD, or \code{simd} construct, to
specify that one or more scan computations is to be performed by the loop.
The directive specifies that either an inclusive scan computation is to be performed
for each list item that appears in an \code{inclusive} clause on the
directive, or an an exclusive scan computation is to be performed for each list item that
appears in an \code{exclusive} clause on the directive.  For each list item
for which a scan computation is specified, statements that lexically precede
or follow the directive constitute one of two phases for a given logical
iteration of the loop -- an \textit{input phase} or a \textit{scan phase}. 

If the list item appears in an \code{inclusive} clause, all statements in the
structured block that lexically precede the directive constitute the input
phase and all statements in the structured block that lexically follow the
directive constitute the scan phase. If the list item appears in an
\code{exclusive} clause and the iteration is not the last iteration, all
statements in the structured block that lexically precede the directive
constitute the scan phase and all statements in the structured block that
lexically follow the directive constitute the input phase. If the list item
appears in an \code{exclusive} clause and the iteration is the last iteration,
there is no input phase for the iteration and all statements that lexically
precede or follow the directive constitute the scan phase for the iteration.
The input phase contains all computations that update the list item in the
iteration, and the scan phase ensures that any statement that reads the list
item will see the result of the scan computation for that iteration.

The result of a scan computation for a given iteration is calculated according
to the last \textit{generalized prefix sum}
(\texttt{PRESUM\textsubscript{last}}) applied over the sequence of values
given by the original value of the list item prior to the loop and all
preceding updates to the list
item in the logical iteration space of the loop. The operation
\texttt{PRESUM\textsubscript{last}}(
\plc{op}, \plc{a}\textsubscript{1}, \ldots, \plc{a}\textsubscript{N} ) is
defined for a given binary operator \plc{op} and a sequence of \plc{N} values
\mbox{\plc{a}\textsubscript{1}, \ldots, \plc{a}\textsubscript{N}} as follows:

\begin{itemize}

\item if $\plc{N} = 1$, \plc{a}\textsubscript{1}

\item if $\plc{N} > 1$, \plc{op}(
    \texttt{PRESUM\textsubscript{last}}(\plc{op}, \plc{a}\textsubscript{1},
    \ldots, \plc{a}\textsubscript{K}), \texttt{PRESUM\textsubscript{last}}(\plc{op}, \plc{a}\textsubscript{L}, \ldots, \plc{a}\textsubscript{N}) ), where $1 \leq \plc{K}+1 = \plc{L} \leq \plc{N}$.
\end{itemize}

If the operator \plc{op} is not a mathematically associative operation, the result of
the \texttt{PRESUM\textsubscript{last}} operation is nondeterministic.

At the beginning of the input phase of each iteration, the list item is
initialized with the initializer value of the \plc{reduction-identifier}
specified by the \code{reduction} clause on the innermost enclosing construct.
The \textit{update value} of a list item is, for a given iteration, the value
of the list item on completion of its input phase. 

Let \plc{orig-val} be the value of the original list item on entry to
enclosing loop, loop SIMD, or \code{simd} construct. Let \plc{combiner} be the
combiner for the \plc{reduction-identifier} specified by the \code{reduction}
clause on the construct. And let \plc{u}\textsubscript{I} be the update value
of a list item for iteration \plc{I}.  For list items appearing in an
\code{inclusive} clause on the \code{scan} directive, at the beginning of the
scan phase for iteration \plc{I} the list item is assigned the result of the
operation \texttt{PRESUM\textsubscript{last}}( \plc{combiner}, \plc{orig-val},
\plc{u}\textsubscript{0}, \ldots, \plc{u}\textsubscript{I}).  For list items
appearing in an \code{exclusive} clause on the \code{scan} directive, at the
beginning of the scan phase for iteration $\plc{I} = 0$  the list item is
assigned the value \plc{orig-val}, and at the beginning of the scan phase for
iteration $\plc{I} > 0$  the list item is assigned the result of the operation
\texttt{PRESUM\textsubscript{last}}( \plc{combiner}, \plc{orig-val},
\plc{u}\textsubscript{0}, \ldots, \plc{u}\textsubscript{I-1}).


\restrictions
Restrictions to the \code{scan} directive are as follows:

\begin{itemize}

\item Exactly one \code{scan} directive must appear in the loop body of an
    enclosing loop, loop SIMD, or \code{simd} construct on which a
    \code{reduction} clause with the \code{inscan} modifier is present.

\item A list item that appears in the \code{inclusive} or \code{exclusive}
    clause must appear in a \code{reduction} clause with the \code{inscan}
    modifier on the enclosing loop, loop SIMD, or \code{simd} construct.

\item Cross-iteration dependences across different logical iterations must not
    exist, except for dependences for the list items specified in an
    \code{inclusive} or \code{exclusive} clause.

\item Intra-iteration dependences from a statement in the structured block
    preceding a \code{scan} directive to a statement in the structured block
    following a \code{scan} directive must not exist, except for dependences
    for the list items specified in an \code{inclusive} or \code{exclusive}
    clause.

%\item A list item must not appear in more than one \code{inclusive} or
%    \code{exclusive} clause of \code{scan} directives inside the enclosing
%    loop, loop SIMD, or \code{simd} construct.

%\item At most one \code{scan} directive specified with the \code{exclusive}
%    clause is permitted within the enclosing loop, loop SIMD, or \code{simd}
%    construct.

%\item An \code{inclusive} clause must not appear on a \code{scan} directive
%    specified with the \code{exclusive} clause. 

\end{itemize}

\crossreferences
\begin{itemize}
\item loop construct, see
\specref{subsec:Loop Construct}.

\item \code{simd} construct, see
\specref{subsec:simd Construct}.

\item loop SIMD construct, see
\specref{subsec:Loop SIMD Construct}.

\item \code{reduction} clause, see \specref{subsubsec:reduction clause}.
\end{itemize}

\pagebreak
