% This is an included file. See the master file for more information.
%
% When editing this file:
%
%    1. To change formatting, appearance, or style, please edit openmp.sty.
%
%    2. Custom commands and macros are defined in openmp.sty.
%
%    3. Be kind to other editors -- keep a consistent style by copying-and-pasting to
%       create new content.
%
%    4. We use semantic markup, e.g. (see openmp.sty for a full list):
%         \code{}     % for bold monospace keywords, code, operators, etc.
%         \plc{}      % for italic placeholder names, grammar, etc.
%
%    5. There are environments that provide special formatting, e.g. language bars.
%       Please use them whereever appropriate.  Examples are:
%
%         \begin{fortranspecific}
%         This is text that appears enclosed in blue language bars for Fortran.
%         \end{fortranspecific}
%
%         \begin{note}
%         This is a note.  The "Note -- " header appears automatically.
%         \end{note}
%
%    6. Other recommendations:
%         Use the convenience macros defined in openmp.sty for the minor headers
%         such as Comments, Syntax, etc.
%
%         To keep items together on the same page, prefer the use of
%         \begin{samepage}.... Avoid \parbox for text blocks as it interrupts line numbering.
%         When possible, avoid \filbreak, \pagebreak, \newpage, \clearpage unless that's
%         what you mean. Use \needspace{} cautiously for troublesome paragraphs.
%
%         Avoid absolute lengths and measures in this file; use relative units when possible.
%         Vertical space can be relative to \baselineskip or ex units. Horizontal space
%         can be relative to \linewidth or em units.
%
%         Prefer \emph{} to italicize terminology, e.g.:
%             This is a \emph{definition}, not a placeholder.
%             This is a \plc{var-name}.
%

\section{\hcode{scan} Directive}
\index{directives!scan Directive@{\code{scan} DIrective}}
\index{scan Directive@{\code{scan} Directive}}
\label{sec:scan Directive}

\summary
The \code{scan} directive specifies list items that participate in the \plc{scan} 
computation while retaining the cross-iteration data dependencies imposed by the \plc{scan} list items among 
logical iterations of the enclosing loop (or parallel loop) construct, \code{simd} or loop SIMD construct. 


%%%\begin{samepage}
\syntax
\begin{ccppspecific}
The syntax of the \code{scan} directive is as follows:

\begin{ompcPragma}
#pragma omp scan \plc{clause[ [},\plc{] clause]} \plc{new-line}
\end{ompcPragma}

where \plc{clause} is one of the following:
\begin{indentedcodelist}
inclusive(\plc{list})
exclusive(\plc{list})
\end{indentedcodelist}

\end{ccppspecific}

\begin{fortranspecific}
The syntax of the \code{scan} directive is as follows:

\begin{ompfPragma}
!$omp scan \plc{clause[ [},\plc{] clause]} \plc{new-line}
\end{ompfPragma}

where \plc{clause} is one of the following:
\begin{indentedcodelist}
inclusive(\plc{list})
exclusive(\plc{list})
\end{indentedcodelist}

\end{fortranspecific}
%%%\end{samepage}

\binding
The binding thread set for a \code{scan} directive is the current team. A 
\code{scan} directive binds to the innermost enclosing loop (or parallel loop) 
region, \code{simd} or loop SIMD region. 

\descr
When a list item appears in the \code{scan} directive specified with \code{inclusive} 
or \code{exclusive} clause, it participates in the \plc{scan} 
computation specified by the \plc{reduction-identifier}, 
and this computation produces the result of each iteration for the enclosing loop (or parallel 
loop) construct, \code{simd} or loop SIMD construct specified with \code{reduction} 
clause and \code{inscan} modifier (see \ref{subsubsec:reduction clause}), while 
retaining the cross-iteration data dependencies imposed by the \plc{scan} list items among all logical iterations.

Formally, assignments through each logical iteration \plc{k} in the iteration space [\plc{0}, \plc{N-1}]  
to produce the result of the \plc{scan} for the list item 
\plc{X}\textsubscript{i}, for every \plc{i} in [\plc{0}, \plc{k-1}] over \plc{binary_op}, 

where \plc{N} is the total number of logical iterations, \plc{binary_op} denotes the \plc{reduction-identifier},  
\plc{scan}(\plc{binary_op}, \plc{X}\textsubscript{0}, \ldots, \plc{X}\textsubscript{N-1}) 
is defined as follows: 

\begin{itemize}
\item if \plc{N} = 1, \plc{X}\textsubscript{0}
\item if \plc{N} > 1, \plc{binary_op}(\plc{scan}(\plc{binary_op}, \plc{X}\textsubscript{0}, \ldots, \plc{X}\textsubscript{k-1}),
      \plc{scan}(\plc{binary_op}, \plc{X}\textsubscript{k}, \ldots, \plc{X}\textsubscript{N-1}), for any \plc{k} where 0 < \plc{k} <= \plc{N - 1}
\end{itemize}

\plc{X}\textsubscript{k} denotes the value of list item of the \plc{k}-th loop logical iteration associated
with the loop directive. In other words, the \plc{scan} operations may be performed in arbitrary order,
and the behavior is nondeterministic if \plc{binary_op} is not associative.

All iterations of an enclosing loop (or parallel loop) construct, \code{simd} or loop SIMD 
construct are permitted to execute in an unordered fashion in unspecified threads, and 
indeterminately sequenced within each thread when dependencies imposed by 
the \code{scan} directive are retained. 

%%%The \code{scan} directive with an \code{inclusive} clause specifies a 
%%%boundary between lexically last \plc{scan} writes 
%%%and other reads of one or more list items that needs to preserve 
%%%the dependencies. 

The \plc{inclusive} means that the \plc{k}-th iteration 
input data is included in the \plc{k}-th \plc{scan}.
%%%, which defines, when the thread 
%%%(or SIMD lane) executing the \plc{k}-th iteration of the loop 
%%%encounters a \code{scan} directive, for each list item, its computation  
%%%waits until its dependences on all valid iterations imposed by the \code{scan} 
%%%construct are satisfied before it completes execution of the \plc{k}-th 
%%%iteration, thus, the specific dependence inhereent by \plc{scan} 
%%%is satisfied when a thread (or SIMD lane) executing the corresponding iteration 
%encounters a \code{scan} directive.
If a \code{scan} directive with the \code{inclusive} clause appears in the loop
body, any references to the list item that follow the directive
are replaced with the scan copy instead of the private copy.  The
private copy is initialized at the beginning of each iteration
with the initializer value of the reduction-identifier specified
on the reduction clause for the construct. The iteration value of
a list item is, for a given iteration, the value of its private
copy after the completion of all writes to the private copy that
precede the scan directive. Prior to any read of the scan copy of
a list item that follows the scan directive, the scan copy is
assigned the result of the scan operation over the iteration
values of the corresponding private copies for the set of
iterations comprised of all prior iterations and the current
iteration.
 
%%The \code{scan} directive with an \code{exclusive} clause specifies a boundary that  
%%does not permit dependencies between statements that precede this \code{scan} 
%%construct and statements that appears after this \code{scan} directive, except dependencies write to and 
%%read from list items specified in the \code{exclusive} clause, and all writes to list items must 
%%appear after this \code{scan} directive. 

The \plc{exclusive} means that the \plc{k}-th iteration input data is 
excluded in the \plc{k}-th \plc{scan}.
%%%, which defines, when the thread (or SIMD lane) executing 
%%%the \plc{k}-th iteration of the loop encounters a \code{scan} directive, for each list 
%%%item, the read of its value shall happen before the \plc{scan} computation that appears 
%%%after this \code{scan} directive, completes and writes its produced result to this list 
%%%item in the \plc{k}-th iteration, thus, the specific dependence inhereent by \plc{scan}
%%%is satisfied when a thread (or SIMD lane) executing the corresponding iteration
%%%encounters a \code{scan} directive.
If a \plc{scan} directive with the \code{exclusive} clause appears in the loop
body, any references to the list item that precede the directive
are replaced with the scan copy instead of the private copy. The
private copy is initialized immediately after the scan directive
on each iteration with the initializer value of the
reduction-identifier specified on the reduction clause for the
construct. The iteration value of a list item is, for a given
iteration, the value of its private copy after the completion of
all writes to the private copy that follow the scan directive.
Prior to any read of the scan copy of a list item that precedes
the scan directive, the scan copy is assigned the result of the
scan operation over the iteration values of the corresponding
private copies for the set of iterations comprised of all prior
iterations.


\restrictions
Restrictions to the \code{scan} directive are as follows:

\begin{itemize}

\item A \code{scan} directive must be closely nested inside a loop
(or parallel loop) construct, \code{simd}, or loop SIMD construct.

\item A list item that appears in the \code{inclusive} or \code{exclusive} clause must
appear in \code{reduction} clause with \code{inscan} modifier of the enclosing loop
(or parallel loop) construct, \code{simd} or loop SIMD construct.

\item A list item that appears in the \code{reduction} clause with \code{inscan} modifier 
of the enclosing loop (or parallel loop) construct, \code{simd} or loop SIMD construct, it must appear in an \code{inclusive} 
or \code{exclusive} clause on the enclosed \code{scan} directive.  

\item A list item must not appear in more than one \code{inclusive} or \code{exclusive} clause of \code{scan} directives 
inside the enclosing loop (or parallel loop) construct, \code{simd} or loop SIMD construct.

\item At most one \code{scan} directive specified with \code{exclusive} is permitted within 
the enclosing loop (or parallel loop) construct, \code{simd} or loop SIMD construct.

\item A \code{inclusive} clause must not appear on the \code{scan} directive specified with \code{exclusive} clause. 

\end{itemize}

\crossreferences
\begin{itemize}
\item loop construct, see
\specref{subsec:Loop Construct}.

%%%\item \code{taskloop} constrict, see 
%%%\specref{subsec:taskloop Construct}.

\item \code{simd} construct, see
\specref{subsec:simd Construct}.

\item parallel loop construct, see
\specref{subsec:Parallel Loop Construct}.
\end{itemize}

\pagebreak
