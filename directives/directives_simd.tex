% This is an included file. See the master file for more information.
%
% When editing this file:
%
%    1. To change formatting, appearance, or style, please edit openmp.sty.
%
%    2. Custom commands and macros are defined in openmp.sty.
%
%    3. Be kind to other editors -- keep a consistent style by copying-and-pasting to
%       create new content.
%
%    4. We use semantic markup, e.g. (see openmp.sty for a full list):
%         \code{}     % for bold monospace keywords, code, operators, etc.
%         \plc{}      % for italic placeholder names, grammar, etc.
%
%    5. There are environments that provide special formatting, e.g. language bars.
%       Please use them whereever appropriate.  Examples are:
%
%         \begin{fortranspecific}
%         This is text that appears enclosed in blue language bars for Fortran.
%         \end{fortranspecific}
%
%         \begin{note}
%         This is a note.  The "Note -- " header appears automatically.
%         \end{note}
%
%    6. Other recommendations:
%         Use the convenience macros defined in openmp.sty for the minor headers
%         such as Comments, Syntax, etc.
%
%         To keep items together on the same page, prefer the use of
%         \begin{samepage}.... Avoid \parbox for text blocks as it interrupts line numbering.
%         When possible, avoid \filbreak, \pagebreak, \newpage, \clearpage unless that's
%         what you mean. Use \needspace{} cautiously for troublesome paragraphs.
%
%         Avoid absolute lengths and measures in this file; use relative units when possible.
%         Vertical space can be relative to \baselineskip or ex units. Horizontal space
%         can be relative to \linewidth or em units.
%
%         Prefer \emph{} to italicize terminology, e.g.:
%             This is a \emph{definition}, not a placeholder.
%             This is a \plc{var-name}.
%

\section{\hcode{declare}~\hcode{simd} Directive}
\index{declare simd@{\code{declare}~\code{simd}}}
\index{directives!declare simd@{\code{declare}~\code{simd}}}
\label{sec:declare simd Directive}
\summary
The \code{declare}~\code{simd} directive can be applied to a function (C, C++ and Fortran) or a
subroutine (Fortran) to enable the creation of one or more versions that can process
multiple arguments using SIMD instructions from a single invocation in a SIMD
loop. The \code{declare}~\code{simd} directive is a declarative directive. There may be multiple
\code{declare}~\code{simd} directives for a function (C, C++, Fortran) or subroutine (Fortran).

\syntax
The syntax of the \code{declare}~\code{simd} directive is as follows:

\begin{ccppspecific}
\begin{ompcPragma}
#pragma omp declare simd \plc{[clause[ [},\plc{] clause] ... ] new-line}
\plc{[}#pragma omp declare simd \plc{[clause[ [},\plc{] clause] ... ] new-line]}
\plc{[ ... ]}
   \plc{function definition or declaration}
\end{ompcPragma}

where \plc{clause} is one of the following:

\begin{indentedcodelist}
simdlen(\plc{length})
linear(\plc{linear-list[ }:\plc{ linear-step]})
aligned(\plc{argument-list[ }:\plc{ alignment]})
uniform(\plc{argument-list})
inbranch
notinbranch
\end{indentedcodelist}
\end{ccppspecific}


\begin{fortranspecific}
\begin{ompfPragma}
!$omp declare simd \plc{[}(\plc{proc-name})\plc{] [clause[ [},\plc{] clause] ... ]}
\end{ompfPragma}

where \plc{clause} is one of the following:
\begin{indentedcodelist}
simdlen(\plc{length})
linear(\plc{linear-list[ }:\plc{ linear-step]})
aligned(\plc{argument-list[ }:\plc{ alignment]})
uniform(\plc{argument-list})
inbranch
notinbranch
\end{indentedcodelist}
\end{fortranspecific}


\descr
\begin{ccppspecific}
The use of one or more \code{declare}~\code{simd} directives immediately prior
to a function declaration or definition enables the
creation of corresponding SIMD
versions of the associated function that can be used to process multiple arguments from
a single invocation in a SIMD loop concurrently.

The expressions appearing in the clauses of each directive are evaluated in the scope of
the arguments of the function declaration or definition.
\end{ccppspecific}

\begin{samepage}
\begin{fortranspecific}
The use of one or more \code{declare}~\code{simd} directives for a specified
subroutine or function enables the creation of corresponding SIMD versions of the
subroutine or function that can be used to process multiple arguments from a
single invocation in a SIMD loop concurrently.
\end{fortranspecific}
\end{samepage}

If a SIMD version is created, the number of concurrent arguments for the function is
determined by the \code{simdlen} clause. If the \code{simdlen} clause is used its value
corresponds to the number of concurrent arguments of the function. The parameter of
the \code{simdlen} clause must be a constant positive integer expression. Otherwise, the
number of concurrent arguments for the function is implementation defined.

\begin{cppspecific}
The special \plc{this} pointer can be used as if was one of the arguments to the function in any of the \code{linear}, \code{aligned}, or \code{uniform} clauses.
\end{cppspecific}

The \code{uniform} clause declares one or more arguments to have an invariant value for all
concurrent invocations of the function in the execution of a single SIMD loop.

\begin{samepage}
\begin{ccppspecific}
The \code{aligned} clause declares that the object to which each list item points is aligned to
the number of bytes expressed in the optional parameter of the \code{aligned} clause.
\end{ccppspecific}
\end{samepage}

\needspace{15\baselineskip}\begin{samepage}
\begin{fortranspecific}
The \code{aligned} clause declares that the target of each list item is aligned to the number
of bytes expressed in the optional parameter of the \code{aligned} clause.
\end{fortranspecific}
\end{samepage}

The optional parameter of the \code{aligned} clause, \plc{alignment}, must be a constant positive
integer expression. If no optional parameter is specified, implementation-defined default
alignments for SIMD instructions on the target platforms are assumed.

The \code{inbranch} clause specifies that the SIMD version of the function will always be called from inside a
conditional statement of a SIMD loop. The \code{notinbranch} clause specifies that the
SIMD version of the function will never be called from inside a conditional statement of a SIMD loop. If
neither clause is specified, then the SIMD version of the function may or may not be called from inside a
conditional statement of a SIMD loop.

\restrictions
\begin{itemize}
\item Each argument can appear in at most one \code{uniform} or \code{linear} clause.

\item At most one \code{simdlen} clause can appear in a \code{declare}~\code{simd} directive.

\item Either \code{inbranch} or \code{notinbranch} may be specified, but not both.

\item When a \plc{linear-step} expression is specified in a \code{linear} clause it must be
either a constant integer expression or an integer-typed parameter that is specified in
a \code{uniform} clause on the directive.

\item The function or subroutine body must be a structured block.

\item The execution of the function or subroutine, when called from a SIMD loop, cannot result in the execution of an OpenMP construct except for an \code{ordered} construct with the \code{simd} clause or an \code{atomic} construct.

\item The execution of the function or subroutine cannot have any side effects that would
alter its execution for concurrent iterations of a SIMD chunk.

\item A program that branches into or out of the function is non-conforming.

\begin{ccppspecific}
\item If the function has any declarations, then the \code{declare}~\code{simd} construct for any
declaration that has one must be equivalent to the one specified for the definition.
Otherwise, the result is unspecified.

\item The function cannot contain calls to the \code{longjmp} or \code{setjmp} functions.
\end{ccppspecific}

\begin{cspecific}
\item The type of list items appearing in the \code{aligned} clause must be array or pointer.
\end{cspecific}

\begin{cppspecific}
\item The function cannot contain any calls to \code{throw}.

\item The type of list items appearing in the \code{aligned} clause must be array, pointer,
reference to array, or reference to pointer.
\end{cppspecific}

\begin{fortranspecific}
\item \plc{proc-name} must not be a generic name, procedure pointer or entry name.

\item If \plc{proc-name} is omitted, the \code{declare}~\code{simd}
  directive must appear in the specification part of a subroutine
  subprogram or a function subprogram for which creation of the SIMD
  versions is enabled.

\item Any \code{declare}~\code{simd} directive must appear in the specification part of a subroutine
subprogram, function subprogram or interface body to which it applies.

\item If a \code{declare}~\code{simd} directive is specified in an interface block for a procedure, it
must match a \code{declare}~\code{simd} directive in the definition of the procedure.

\item If a procedure is declared via a procedure declaration statement, the procedure
\plc{proc-name} should appear in the same specification.

\item If a \code{declare}~\code{simd} directive is specified for a procedure name with explicit
interface and a \code{declare}~\code{simd} directive is also specified for the definition of the
procedure then the two \code{declare}~\code{simd} directives must match. Otherwise the result
is unspecified.

\item Procedure pointers may not be used to access versions created by the \code{declare}~\code{simd} directive.

\item The type of list items appearing in the \code{aligned} clause must be \code{C_PTR} or Cray
pointer, or the list item must have the \code{POINTER} or \code{ALLOCATABLE} attribute.
\end{fortranspecific}
\end{itemize}

\crossreferences
\begin{itemize}
\item \code{reduction} clause, see
\specref{subsubsec:reduction clause}.

\item \code{linear} clause, see
\specref{subsubsec:linear clause}.
\end{itemize}
