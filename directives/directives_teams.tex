%% TODOs:
%% - do we need a barrier or join operation on the host at the end of the "teams"
%%   region?

% This is an included file. See the master file for more information.
%
% When editing this file:
%
%    1. To change formatting, appearance, or style, please edit openmp.sty.
%
%    2. Custom commands and macros are defined in openmp.sty.
%
%    3. Be kind to other editors -- keep a consistent style by copying-and-pasting to
%       create new content.
%
%    4. We use semantic markup, e.g. (see openmp.sty for a full list):
%         \code{}     % for bold monospace keywords, code, operators, etc.
%         \plc{}      % for italic placeholder names, grammar, etc.
%
%    5. There are environments that provide special formatting, e.g. language bars.
%       Please use them whereever appropriate.  Examples are:
%
%         \begin{fortranspecific}
%         This is text that appears enclosed in blue language bars for Fortran.
%         \end{fortranspecific}
%
%         \begin{note}
%         This is a note.  The "Note -- " header appears automatically.
%         \end{note}
%
%    6. Other recommendations:
%         Use the convenience macros defined in openmp.sty for the minor headers
%         such as Comments, Syntax, etc.
%
%         To keep items together on the same page, prefer the use of
%         \begin{samepage}.... Avoid \parbox for text blocks as it interrupts line numbering.
%         When possible, avoid \filbreak, \pagebreak, \newpage, \clearpage unless that's
%         what you mean. Use \needspace{} cautiously for troublesome paragraphs.
%
%         Avoid absolute lengths and measures in this file; use relative units when possible.
%         Vertical space can be relative to \baselineskip or ex units. Horizontal space
%         can be relative to \linewidth or em units.
%
%         Prefer \emph{} to italicize terminology, e.g.:
%             This is a \emph{definition}, not a placeholder.
%             This is a \plc{var-name}.
%


\section{\hcode{teams} Construct}
\index{teams@{\code{teams}}}
\index{constructs!teams@{\code{teams}}}
\index{device constructs!teams@{\code{teams}}}
\label{sec:teams Construct}
\summary
The \code{teams} construct creates a league of initial teams and the initial thread in each
team executes the region.

\syntax
\begin{ccppspecific}
The syntax of the \code{teams} construct is as follows:

\begin{ompcPragma}
#pragma omp teams \plc{[clause[ [},\plc{] clause] ... ] new-line}
    \plc{structured-block}
\end{ompcPragma}

where \plc{clause} is one of the following:

\begin{indentedcodelist}
num_teams(\plc{integer-expression})
thread_limit(\plc{integer-expression})
default(shared \textnormal{|} none)
private(\plc{list})
firstprivate(\plc{list})
shared(\plc{list})
reduction(\plc{[}default ,\plc{] reduction-identifier }:\plc{ list})
allocate(\plc{[allocator }:\plc{] list})
\end{indentedcodelist}
\end{ccppspecific}

\begin{fortranspecific}
The syntax of the \code{teams} construct is as follows:

\begin{ompfPragma}
!$omp teams \plc{[clause[ [},\plc{] clause] ... ]}
    \plc{structured-block}
!$omp end teams
\end{ompfPragma}


where \plc{clause} is one of the following:

\begin{indentedcodelist}
num_teams(\plc{scalar-integer-expression})
thread_limit(\plc{scalar-integer-expression})
default(shared \textnormal{|} firstprivate \textnormal{|} private \textnormal{|} none)
private(\plc{list})
firstprivate(\plc{list})
shared(\plc{list})
reduction(\plc{[}default ,\plc{] reduction-identifier }:\plc{ list})
allocate(\plc{[allocator }:\plc{] list})
\end{indentedcodelist}

\end{fortranspecific}

\begin{samepage}

\binding
The binding thread set for a \code{teams} region is the encountering thread.

\descr
When a thread encounters a \code{teams} construct, a league of teams is
created. Each team is an initial team, and the initial thread in each team
executes the \code{teams} region.

The number of teams created is implementation defined, but is less than or 
equal to the value specified in the \code{num_teams} clause. A thread may 
obtain the number of initial teams created by the construct by a call to 
the \code{omp_get_num_teams} routine.

\end{samepage}

The maximum number of threads participating in the contention group that 
each team initiates is implementation defined, but is less than or equal 
to the value specified in the \code{thread_limit} clause.

On a combined or composite construct that includes \code{target} and
\code{teams} constructs, the expressions in \code{num_teams} and
\code{thread_limit} clauses are evaluated on the host device on
entry to the \code{target} construct.

Once the teams are created, the number of initial teams remains constant 
for the duration of the \code{teams} region.

Within a \code{teams} region, initial team numbers uniquely identify each
initial team. Initial team numbers are consecutive whole numbers ranging 
from zero to one less than the number of initial teams. A thread may obtain 
its own initial team number by a call to the \code{omp_get_team_num} library
routine. The policy for assigning the initial threads to places is 
implementation defined. The \code{teams} construct sets the 
\plc{place-partition-var} and \plc{default-device-var} ICVs for each initial 
thread to an implementation-defined value.

After the teams have completed execution of the \code{teams} region, the 
encountering task resumes execution of the enclosing task region.

\events

The \plc{teams-begin} event occurs in a thread that encounters a
\code{teams} construct before any initial task is created for the
corresponding \code{teams} region.

Upon creation of each initial task, an \plc{initial-task-begin} event
occurs in the thread that executes the initial task after the initial
task is fully initialized but before the thread begins to execute the
structured block of the \code{teams} construct.

If the \code{teams} region creates a native thread, a \plc{native-thread-begin}
event occurs as the first event in the context of the new thread prior to the 
\plc{initial-task-begin} event.

When a thread finishes an initial task, an \plc{initial-task-end}
event occurs in the thread.

The \plc{teams-end} event occurs in the thread that encounters the
\code{teams} construct after the thread executes its \plc{initial-task-end} 
event but before it resumes execution of the encountering task.

If a native thread is destroyed at the end of a \code{teams} region, a
\plc{native-thread-end} event occurs in the thread as the last event prior 
to destruction of the thread.

\tools

A thread dispatches a registered \code{ompt_callback_parallel_begin}
callback for each occurrence of a \plc{teams-begin} event in that
thread.  The callback occurs in the task that encounters the 
\code{teams} construct.  This callback has the type signature
\code{ompt_callback_parallel_begin_t}. In the dispatched callback, 
\code{(flag)}~\code{&}~\code{ompt_parallel_league)} evaluates to \plc{true}.

A thread dispatches a registered \code{ompt_callback_implicit_task}
callback with \code{ompt_scope_begin} as its \plc{endpoint} argument
for each occurrence of an \plc{initial-task-begin} in that thread.
Similarly, a thread dispatches a registered \code{ompt_callback_implicit_task}
callback with \code{ompt_scope_end} as its \plc{endpoint} argument
for each occurence of an \plc{initial-task-end} event in that thread. 
The callbacks occur in the context of the initial task.  The callbacks 
have type signature \code{ompt_callback_implicit_task_t}.

A thread dispatches a registered \code{ompt_callback_parallel_end}
callback for each occurrence of a \plc{teams-end} event in that
thread.  The callback occurs in the task that encounters
the \code{teams} construct.  This callback has the type signature
\code{ompt_callback_parallel_end_t}.

A thread dispatches a registered \code{ompt_callback_thread_begin}
callback for the \plc{native-thread-begin} event in that thread.
The callback occurs in the context of the thread.  The callback 
has type signature \code{ompt_callback_thread_begin_t}.

A thread dispatches a registered \code{ompt_callback_thread_end}
callback for the \plc{native-thread-end} event in that thread. The callback 
occurs in the context of the thread.  The callback has type signature
\code{ompt_callback_thread_end_t}.

\restrictions
Restrictions to the \code{teams} construct are as follows:

\begin{itemize}
\item A program that branches into or out of a \code{teams} region is non-conforming.
\item A program must not depend on any ordering of the evaluations of the clauses 
      of the \code{teams} directive, or on any side effects of the evaluation of 
      the clauses.
\item At most one \code{thread_limit} clause can appear on the directive. The
      \code{thread_limit} expression must evaluate to a positive integer value.
\item At most one \code{num_teams} clause can appear on the directive. The 
      \code{num_teams} expression must evaluate to a positive integer value.
\item A \code{teams} region can only be strictly nested within the implicit 
      parallel region or a \code{target} region. If a \code{teams} construct 
      is nested within a \code{target} construct, that \code{target} construct 
      must contain no statements, declarations or directives outside of the 
      \code{teams} construct. \item \code{distribute}, \code{distribute simd}, 
      distribute parallel worksharing-loop, distribute parallel worksharing-loop 
      SIMD, \code{parallel} regions, including any \code{parallel} regions arising 
      from combined constructs, \code{omp_get_num_teams()} regions, and 
      \code{omp_get_team_num()} regions are the only OpenMP regions that may be 
      strictly nested inside the \code{teams} region.
\end{itemize}

\crossreferences
\begin{itemize}

\item \code{parallel} construct, see \specref{sec:parallel Construct}.

\item \code{distribute} construct, see \specref{subsec:distribute Construct}.

\item \code{distribute simd} construct, see \specref{subsec:distribute simd Construct}.

\item \code{allocate} clause, see
\specref{subsec:allocate Clause}.

\item \code{target} construct, see \specref{subsec:target Construct}.

\item \code{default}, \code{shared}, \code{private}, \code{firstprivate}, and \code{reduction} clauses, see \specref{subsec:Data-Sharing Attribute Clauses}.

\item \code{omp_get_num_teams} routine, see
\specref{subsec:omp_get_num_teams}.

\item \code{omp_get_team_num} routine, see
\specref{subsec:omp_get_team_num}.

\item \code{ompt_callback_thread_begin_t}, see
  \specref{sec:ompt_callback_thread_begin_t}.

\item \code{ompt_callback_thread_end_t}, see
  \specref{sec:ompt_callback_thread_end_t}.

\item \code{ompt_callback_parallel_begin_t}, see
  \specref{sec:ompt_callback_parallel_begin_t}.

\item \code{ompt_callback_parallel_end_t},
  see \specref{sec:ompt_callback_parallel_end_t}.

\item \code{ompt_callback_implicit_task_t}, see
  \specref{sec:ompt_callback_implicit_task_t}.
\end{itemize}
