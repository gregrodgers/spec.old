% This is an included file. See the master file for more information.
%
% When editing this file:
%
%    1. To change formatting, appearance, or style, please edit openmp.sty.
%
%    2. Custom commands and macros are defined in openmp.sty.
%
%    3. Be kind to other editors -- keep a consistent style by copying-and-pasting to
%       create new content.
%
%    4. We use semantic markup, e.g. (see openmp.sty for a full list):
%         \code{}     % for bold monospace keywords, code, operators, etc.
%         \plc{}      % for italic placeholder names, grammar, etc.
%
%    5. There are environments that provide special formatting, e.g. language bars.
%       Please use them whereever appropriate.  Examples are:
%
%         \begin{fortranspecific}
%         This is text that appears enclosed in blue language bars for Fortran.
%         \end{fortranspecific}
%
%         \begin{note}
%         This is a note.  The "Note -- " header appears automatically.
%         \end{note}
%
%    6. Other recommendations:
%         Use the convenience macros defined in openmp.sty for the minor headers
%         such as Comments, Syntax, etc.
%
%         To keep items together on the same page, prefer the use of
%         \begin{samepage}.... Avoid \parbox for text blocks as it interrupts line numbering.
%         When possible, avoid \filbreak, \pagebreak, \newpage, \clearpage unless that's
%         what you mean. Use \needspace{} cautiously for troublesome paragraphs.
%
%         Avoid absolute lengths and measures in this file; use relative units when possible.
%         Vertical space can be relative to \baselineskip or ex units. Horizontal space
%         can be relative to \linewidth or em units.
%
%         Prefer \emph{} to italicize terminology, e.g.:
%             This is a \emph{definition}, not a placeholder.
%             This is a \plc{var-name}.
%


\section{Worksharing Constructs}
\label{sec:Worksharing Constructs}
\index{worksharing constructs}
\index{constructs!worksharing}
\index{worksharing!constructs}
A worksharing construct distributes the execution of the corresponding region among the
members of the team that encounters it. Threads execute portions of the region in the
context of the implicit tasks each one is executing. If the team consists of only one
thread then the worksharing region is not executed in parallel.

A worksharing region has no barrier on entry; however, an implied barrier exists at the
end of the worksharing region, unless a \code{nowait} clause is specified. If a \code{nowait}
clause is present, an implementation may omit the barrier at the end of the worksharing
region. In this case, threads that finish early may proceed straight to the instructions
following the worksharing region without waiting for the other members of the team to
finish the worksharing region, and without performing a flush operation.

The OpenMP API defines the worksharing constructs that are described
in this section.  The worksharing-loop construct is described in \specref{subsec:Worksharing-Loop Construct}.

\begin{samepage}
\restrictions
The following restrictions apply to worksharing constructs:

\begin{itemize}
\item Each worksharing region must be encountered by all threads in a team or by none at
all, unless cancellation has been requested for the innermost enclosing parallel
region.

\item The sequence of worksharing regions and \code{barrier} regions encountered must be the
same for every thread in a team
\end{itemize}
\end{samepage}











\subsection{\hcode{sections} Construct}
\label{subsec:sections Construct}
\index{sections@{\code{sections}}}
\index{constructs!sections@{\code{sections}}}
\summary
The \code{sections} construct is a non-iterative worksharing construct that contains a set of
structured blocks that are to be distributed among and executed by the threads in a team.
Each structured block is executed once by one of the threads in the team in the context
of its implicit task.

\syntax
\begin{ccppspecific}
The syntax of the \code{sections} construct is as follows:

\begin{ompcPragma}
#pragma omp sections \plc{[clause[ [},\plc{] clause] ... ] new-line}
   {
   \plc{[}#pragma omp section \plc{new-line}\plc{]}
      \plc{structured-block}
   \plc{[}#pragma omp section \plc{new-line}
      \plc{structured-block]}
   \plc{...}
   }
\end{ompcPragma}

where \plc{clause} is one of the following:

{}
\begin{indentedcodelist}
private(\plc{list})
firstprivate(\plc{list})
lastprivate(\plc{[ lastprivate-modifier}:\plc{] list})
reduction(\plc{[ reduction-modifier},\plc{]reduction-identifier }:\plc{ list})
nowait
allocate(\plc{[allocator: ]}\plc{list})
\end{indentedcodelist}
\end{ccppspecific}

\needspace{16\baselineskip}
\begin{fortranspecific}
The syntax of the \code{sections} construct is as follows:

\begin{ompfPragma}
!$omp sections \plc{[clause[ [},\plc{] clause] ... ]}
   \plc{[}!$omp section\plc{]}
      \plc{structured-block}
   \plc{[}!$omp section
      \plc{structured-block]}
   \plc{...}
!$omp end sections \plc{[}nowait\plc{]}
\end{ompfPragma}

\begin{samepage}
where \plc{clause} is one of the following:

{}
\begin{indentedcodelist}
private(\plc{list})
firstprivate(\plc{list})
lastprivate(\plc{[ lastprivate-modifier}:\plc{] list})
reduction(\plc{[ reduction-modifier},\plc{]reduction-identifier }:\plc{ list})
allocate(\plc{[allocator: ]}\plc{list})
\end{indentedcodelist}
\end{samepage}
\end{fortranspecific}

\binding
The binding thread set for a \code{sections} region is the current team. A \code{sections}
region binds to the innermost enclosing \code{parallel} region. Only the threads of the team
executing the binding \code{parallel} region participate in the execution of the structured
blocks and the implied barrier of the \code{sections} region if the barrier is not eliminated
by a \code{nowait} clause.

\descr
Each structured block in the \code{sections} construct is preceded by a \code{section} directive
except possibly the first block, for which a preceding \code{section} directive is optional.

The method of scheduling the structured blocks among the threads in the team is
implementation defined.

There is an implicit barrier at the end of a \code{sections} construct unless a \code{nowait}
clause is specified.

%\tools

\omptWorksharingEvents{sections}

The \plc{section-begin} event occurs before an implicit task starts
executing a structured block in the \code{sections} construct.

\omptWorksharingTools{sections}{ompt_work_sections}

A thread dispatches a registered \code{ompt_callback_dispatch}
callback for each occurrence of a \plc{section-begin} 
event in that thread. The callback occurs in the
context of the implicit task.  The callback has type signature
\code{ompt_callback_dispatch_t}. 

\restrictions
Restrictions to the \code{sections} construct are as follows:

\begin{itemize}
\item Orphaned \code{section} directives are prohibited. That is, the \code{section} directives must
appear within the \code{sections} construct and must not be encountered elsewhere in the
\code{sections} region.

\item The code enclosed in a \code{sections} construct must be a structured block.

\item Only a single \code{nowait} clause can appear on a \code{sections} directive.

\begin{cppspecific}
\item A throw executed inside a \code{sections} region must cause execution to resume within
the same section of the \code{sections} region, and the same thread that threw the
exception must catch it.
\end{cppspecific}
\end{itemize}

\crossreferences
\begin{itemize}
\item \code{private}, \code{firstprivate}, \code{lastprivate}, and \code{reduction} clauses, see
\specref{subsec:Data-Sharing Attribute Clauses}.

\item \code{ompt_scope_begin} and \code{ompt_scope_end}, see
  \specref{sec:ompt_scope_endpoint_t}.

\item \code{ompt_work_sections}, see \specref{sec:ompt_work_t}.

\item \code{ompt_callback_work_t}, see
\specref{sec:ompt_callback_work_t}.

\item \code{ompt_callback_dispatch_t}, see 
\specref{sec:ompt_callback_dispatch_t}.

\end{itemize}










\subsection{\hcode{single} Construct}
\index{single@{\code{single}}}
\index{constructs!single@{\code{single}}}
\label{subsec:single Construct}
\summary
The \code{single} construct specifies that the associated structured block is executed by only
one of the threads in the team (not necessarily the master thread), in the context of its
implicit task. The other threads in the team, which do not execute the block, wait at an
implicit barrier at the end of the \code{single} construct unless a \code{nowait} clause is specified.

\syntax
\begin{ccppspecific}
The syntax of the single construct is as follows:

\begin{ompcPragma}
#pragma omp single \plc{[clause[ [},\plc{] clause] ... ] new-line}
   \plc{structured-block}
\end{ompcPragma}

\begin{samepage}
where \plc{clause} is one of the following:

\begin{indentedcodelist}
private(\plc{list})
firstprivate(\plc{list})
copyprivate(\plc{list})
nowait
allocate(\plc{[allocator: ]}\plc{list})
\end{indentedcodelist}
\end{samepage}
\end{ccppspecific}

\begin{fortranspecific}
The syntax of the \code{single} construct is as follows:

\begin{ompfPragma}
!$omp single \plc{[clause[ [},\plc{] clause] ... ]}
   \plc{structured-block}
!$omp end single \plc{[end_clause[ [},\plc{] end_clause] ... ]}
\end{ompfPragma}

where \plc{clause} is one of the following:

\begin{indentedcodelist}
private(\plc{list})
firstprivate(\plc{list})
allocate(\plc{[allocator: ]}\plc{list})
\end{indentedcodelist}

and \plc{end_clause} is one of the following:

\begin{indentedcodelist}
copyprivate(\plc{list})
nowait
\end{indentedcodelist}
\end{fortranspecific}

\binding
The binding thread set for a \code{single} region is the current team. A \code{single} region
binds to the innermost enclosing \code{parallel} region. Only the threads of the team
executing the binding \code{parallel} region participate in the execution of the structured
block and the implied barrier of the \code{single} region if the barrier is not eliminated by a
\code{nowait} clause.

\descr
Only one of the encountering threads will execute the structured block associated with the \code{single}
construct. The method of choosing a thread to execute the structured block each time the team encounters the construct
is implementation defined. There is an implicit barrier at the end of the \code{single} construct unless a
\code{nowait} clause is specified.

\events

The \plc{single-begin} event occurs after an \code{implicit task} encounters a
\code{single} construct but before the task starts the execution of the structured
block of the \code{single} region.

The \plc{single-end} event occurs after a \code{single} region finishes execution of the structured block
but before resuming execution of the encountering implicit task.


\tools

A thread dispatches a registered \code{ompt_callback_work}
callback for each occurrence of \plc{single-begin} and
\plc{single-end} events in that thread. The callback has type signature
\code{ompt_callback_work_t}. The callback receives
\code{ompt_scope_begin} or \code{ompt_scope_end}
as its \plc{endpoint} argument, as appropriate, and
\code{ompt_work_single_executor} or \code{ompt_work_single_other}
as its \plc{wstype} argument.

\restrictions
Restrictions to the \code{single} construct are as follows:

\begin{itemize}
\item The \code{copyprivate} clause must not be used with the \code{nowait} clause.

\item At most one \code{nowait} clause can appear on a \code{single} construct.

\begin{cppspecific}
\item A throw executed inside a \code{single} region must cause execution to resume within the
same \code{single} region, and the same thread that threw the exception must catch it.
\end{cppspecific}
\end{itemize}


\crossreferences
\begin{itemize}
\item \code{private} and \code{firstprivate} clauses, see
\specref{subsec:Data-Sharing Attribute Clauses}.

\item \code{copyprivate} clause, see
\specref{subsubsec:copyprivate clause}.

\item \code{ompt_scope_begin} and \code{ompt_scope_end}, see
  \specref{sec:ompt_scope_endpoint_t}.

\item \code{ompt_work_single_executor} and \code{ompt_work_single_other}, see
\specref{sec:ompt_work_t}.

\item \code{ompt_callback_work_t},
\specref{sec:ompt_callback_work_t}.

\end{itemize}














% Here we need to force the blue marker lower, and force the subsection header higher
% in order to reduce the space between the marker and the header, per Richard:
%\begin{samepage}

\begin{fortranspecific}

\subsection{\hcode{workshare} Construct}
\index{workshare@{\code{workshare}}}
\index{constructs!workshare@{\code{workshare}}}
\label{subsec:workshare Construct}
\summary
The \code{workshare} construct divides the execution of the enclosed structured block into
separate units of work, and causes the threads of the team to share the work such that
each unit is executed only once by one thread, in the context of its implicit task.
%\end{samepage}

%\begin{samepage}
\syntax
The syntax of the \code{workshare} construct is as follows:

\begin{ompfPragma}
!$omp workshare
    \plc{structured-block}
!$omp end workshare \plc{[}nowait\plc{]}
\end{ompfPragma}
%\end{samepage}

The enclosed structured block must consist of only the following:

\begin{itemize}
\item array assignments

\item scalar assignments

\item \code{FORALL} statements

\item \code{FORALL} constructs

\item \code{WHERE} statements

\item \code{WHERE} constructs

\item \code{atomic} constructs

\item \code{critical} constructs

\item \code{parallel} constructs
\end{itemize}

Statements contained in any enclosed \code{critical} construct are also subject to these
restrictions. Statements in any enclosed \code{parallel} construct are not restricted.

\binding
The binding thread set for a \code{workshare} region is the current team. A \code{workshare}
region binds to the innermost enclosing \code{parallel} region. Only the threads of the team
executing the binding \code{parallel} region participate in the execution of the units of
work and the implied barrier of the \code{workshare} region if the barrier is not eliminated
by a \code{nowait} clause.

\descr
There is an implicit barrier at the end of a \code{workshare} construct unless a \code{nowait}
clause is specified.

An implementation of the \code{workshare} construct must insert any synchronization that is
required to maintain standard Fortran semantics. For example, the effects of one
statement within the structured block must appear to occur before the execution of
succeeding statements, and the evaluation of the right hand side of an assignment must
appear to complete prior to the effects of assigning to the left hand side.

The statements in the \code{workshare} construct are divided into units of work as follows:

\begin{itemize}
\item For array expressions within each statement, including transformational array
intrinsic functions that compute scalar values from arrays:

\begin{itemize} % nested level
\item Evaluation of each element of the array expression, including any references to
\code{ELEMENTAL} functions, is a unit of work.

\item Evaluation of transformational array intrinsic functions may be freely subdivided
into any number of units of work.
\end{itemize}

\item For an array assignment statement, the assignment of each element is a unit of work.

\item For a scalar assignment statement, the assignment operation is a unit of work.

\item For a \code{WHERE} statement or construct, the evaluation of the mask expression and the
masked assignments are each a unit of work.

\item For a \code{FORALL} statement or construct, the evaluation of the mask expression,
expressions occurring in the specification of the iteration space, and the masked
assignments are each a unit of work

\item For an \code{atomic} construct, the atomic operation on the storage location designated as
\plc{x} is a unit of work.

\item For a \code{critical} construct, the construct is a single unit of work.

\item For a \code{parallel} construct, the construct is a unit of work with respect to the
\code{workshare} construct. The statements contained in the \code{parallel} construct are
executed by a new thread team.

\item If none of the rules above apply to a portion of a statement in the structured block,
then that portion is a unit of work.
\end{itemize}

The transformational array intrinsic functions are \code{MATMUL}, \code{DOT_PRODUCT}, \code{SUM},
\code{PRODUCT}, \code{MAXVAL}, \code{MINVAL}, \code{COUNT},
\code{ANY}, \code{ALL}, \code{SPREAD}, \code{PACK}, \code{UNPACK},
\code{RESHAPE}, \code{TRANSPOSE}, \code{EOSHIFT}, \code{CSHIFT}, \code{MINLOC}, and \code{MAXLOC}.

It is unspecified how the units of work are assigned to the threads executing a
\code{workshare} region.

If an array expression in the block references the value, association status, or allocation
status of private variables, the value of the expression is undefined, unless the same
value would be computed by every thread.

If an array assignment, a scalar assignment, a masked array assignment, or a \code{FORALL}
assignment assigns to a private variable in the block, the result is unspecified.

The \code{workshare} directive causes the sharing of work to occur only in the \code{workshare}
construct, and not in the remainder of the \code{workshare} region.

%\tools
\omptWorksharing{workshare}{ompt_work_workshare}

\begin{samepage}
\restrictions
The following restrictions apply to the \code{workshare} construct:

\begin{itemize}
\item All array assignments, scalar assignments, and masked array assignments must be
intrinsic assignments.

\item The construct must not contain any user defined function calls unless the function is
\code{ELEMENTAL}.
\end{itemize}
\end{samepage}

\crossreferences
\begin{itemize}
\item \code{ompt_scope_begin} and \code{ompt_scope_end}, see
  \specref{sec:ompt_scope_endpoint_t}.
\item \code{ompt_work_workshare}, see \specref{sec:ompt_work_t}.
\item \code{ompt_callback_work_t}, see
\specref{sec:ompt_callback_work_t}.
\end{itemize}

%\filbreak
\end{fortranspecific}
