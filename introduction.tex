% This is introduction.tex (Chapter 1 )of the OpenMP specification
% This is an included file. See the master file for more information.
%
% When editing this file:
%
%    1. To change formatting, appearance, or style, please edit openmp.sty.
%
%    2. Custom commands and macros are defined in openmp.sty.
%
%    3. Be kind to other editors -- keep a consistent style by copying-and-pasting to
%       create new content.
%
%    4. We use semantic markup, e.g. (see openmp.sty for a full list):
%         \code{}     % for bold monospace keywords, code, operators, etc.
%         \plc{}      % for italic placeholder names, grammar, etc.
%
%    5. There are environments that provide special formatting, e.g. language bars.
%       Please use them whereever appropriate.  Examples are:
%
%         \begin{fortranspecific}
%         This is text that appears enclosed in blue language bars for Fortran.
%         \end{fortranspecific}
%
%         \begin{note}
%         This is a note.  The "Note -- " header appears automatically.
%         \end{note}
%
%    6. Other recommendations:
%         Use the convenience macros defined in openmp.sty for the minor headers
%         such as Comments, Syntax, etc.
%
%         To keep items together on the same page, prefer the use of 
%         \begin{samepage}.... Avoid \parbox for text blocks as it interrupts line numbering.
%         When possible, avoid \filbreak, \pagebreak, \newpage, \clearpage unless that's
%         what you mean. Use \needspace{} cautiously for troublesome paragraphs.
%
%         Avoid absolute lengths and measures in this file; use relative units when possible.
%         Vertical space can be relative to \baselineskip or ex units. Horizontal space
%         can be relative to \linewidth or em units.
%
%         Prefer \emph{} to italicize terminology, e.g.:
%             This is a \emph{definition}, not a placeholder.
%             This is a \plc{var-name}.
%

\chapter{Introduction}
\index{introduction}
\label{chap:introduction}
The collection of compiler directives, library routines, and environment
variables described in this document collectively define the specification of
the OpenMP Application Program Interface (OpenMP API) for parallelism in C, C++
and Fortran programs.

This specification provides a model for parallel programming that is portable
across architectures from different vendors. Compilers from numerous vendors
support the OpenMP API. More information about the OpenMP API can be found at
the following web site

\code{http://www.openmp.org}

The directives, library routines, and environment variables defined in this document 
allow users to create and to manage parallel programs while permitting portability. The 
directives extend the C, C++ and Fortran base languages with single program multiple 
data (SPMD) constructs, tasking constructs, device constructs, worksharing constructs, 
and synchronization constructs, and they provide support for sharing, mapping and privatizing 
data. The functionality to control the runtime environment is provided by library 
routines and environment variables. Compilers that support the OpenMP API often 
include a command line option to the compiler that activates and allows interpretation of 
all OpenMP directives.







\section{Scope}
\label{sec:Scope}
The OpenMP API covers only user-directed parallelization, wherein the programmer 
explicitly specifies the actions to be taken by the compiler and runtime system in order 
to execute the program in parallel. OpenMP-compliant implementations are not required 
to check for data dependencies, data conflicts, race conditions, or deadlocks, any of 
which may occur in conforming programs. In addition, compliant implementations are 
not required to check for code sequences that cause a program to be classified as 
non-conforming. Application developers are responsible for correctly using the OpenMP API 
to produce a conforming program. The OpenMP API does not cover compiler-generated 
automatic parallelization and directives to the compiler to assist such parallelization.







\section{Glossary}
\label{sec:Glossary}
\index{glossary}
\subsection{Threading Concepts}
\label{subsec:Threading Concepts}
\glossaryterm{thread}
\glossarydefstart
An execution entity with a stack and associated static memory, called 
\emph{threadprivate memory}.
\glossarydefend

\glossaryterm{OpenMP thread}
\glossarydefstart
A \emph{thread} that is managed by the OpenMP implementation.
\glossarydefend

\glossaryterm{idle thread}
\glossarydefstart
An \emph{OpenMP thread} that is not currently part of any \code{parallel} region. 
\glossarydefend

\glossaryterm{thread-safe routine}
\glossarydefstart
A routine that performs the intended function even when executed concurrently 
(by more than one \emph{thread}).
\glossarydefend

\glossaryterm{processor}
\glossarydefstart
Implementation defined hardware unit on which one or more \emph{OpenMP threads} can 
execute.
\glossarydefend

\glossaryterm{device}
\glossarydefstart
An implementation defined logical execution engine.

\begin{quote}
COMMENT: A \emph{device} could have one or more \emph{processors}.
\end{quote}
\glossarydefend

\glossaryterm{host device}
\glossarydefstart
The \emph{device} on which the \emph{OpenMP program} begins execution.
\glossarydefend

\glossaryterm{target device}
\glossarydefstart
A device onto which code and data may be offloaded from the \emph{host device}.
\glossarydefend





\subsection{OpenMP Language Terminology}
\label{subsec:OpenMP Language Terminology}
\glossaryterm{base language}
\glossarydefstart
A programming language that serves as the foundation of the OpenMP 
specification.

\begin{quote}
COMMENT: See \specref{sec:normative references}
for a listing of current \emph{base languages} for the OpenMP API.
\end{quote}
\glossarydefend

\glossaryterm{base program}
\glossarydefstart
A program written in a \emph{base language}.
\glossarydefend

\glossaryterm{program order}
\glossarydefstart
An ordering of operations performed by the same thread as determined by the
execution sequence of operations specified by the \emph{base language}.

\begin{quote}
COMMENT: For C11 and C++11, \emph{program order} corresponds to the sequenced
before relation between operations performed by the same thread.
\end{quote}
\glossarydefend

\glossaryterm{structured block}
\glossarydefstart
For C/C++, an executable statement, possibly compound, with a single entry at the 
top and a single exit at the bottom, or an OpenMP \emph{construct}.

For Fortran, a block of executable statements with a single entry at the top and a 
single exit at the bottom, or an OpenMP \emph{construct}.

\begin{quote}
COMMENTS:

For all \emph{base languages}:

\begin{itemize}
\item Access to the \emph{structured block} must not be the result of a branch; and

\item The point of exit cannot be a branch out of the \emph{structured block}.
\end{itemize}

For C/C++:

\begin{itemize}
\item The point of entry must not be a call to \code{setjmp()};

\item \code{longjmp()} and \code{throw()} must not violate the entry/exit criteria;

\item Calls to \code{exit()} are allowed in a \emph{structured block}; and

\item An expression statement, iteration statement, selection statement, 
or try block is considered to be a \emph{structured block} if the 
corresponding compound statement obtained by enclosing it in \code{\{} 
and \code{\}} would be a \emph{structured block}.
\end{itemize}

For Fortran:

\begin{itemize}
\item \code{STOP} statements are allowed in a \emph{structured block}.
\end{itemize}
\end{quote}
\glossarydefend


\glossaryterm{enclosing context}
\glossarydefstart
In C/C++, the innermost scope enclosing an OpenMP \emph{directive}.

In Fortran, the innermost scoping unit enclosing an OpenMP \emph{directive}.
\glossarydefend

\glossaryterm{directive}
\glossarydefstart
In C/C++, a \code{\#pragma}, and in Fortran, a comment, that specifies \emph{OpenMP
program} behavior.

\begin{quote}
COMMENT: See \specref{sec:Directive Format} for a description of OpenMP \emph{directive} syntax.
\end{quote}
\glossarydefend


\glossaryterm{white space}
\glossarydefstart
A non-empty sequence of space and/or horizontal tab characters.
\glossarydefend

\glossaryterm{OpenMP program}
\glossarydefstart
A program that consists of a \emph{base program}, annotated with OpenMP \emph{directives} and 
runtime library routines.
\glossarydefend

\glossaryterm{conforming program}
\glossarydefstart
An \emph{OpenMP program} that follows all rules and restrictions of the OpenMP 
specification.
\glossarydefend

\glossaryterm{declarative directive}
\glossarydefstart
An OpenMP \emph{directive} that may only be placed in a declarative context. A 
\emph{declarative directive} results in one or more declarations only; it is not associated 
with the immediate execution of any user code.
\glossarydefend

\glossaryterm{executable directive}
\glossarydefstart
An OpenMP \emph{directive} that is not declarative. That is, it may be placed in an 
executable context.
\glossarydefend

\glossaryterm{stand-alone directive}
\glossarydefstart
An OpenMP \emph{executable directive} that has no associated executable user code.
\glossarydefend


\glossaryterm{construct}
\glossarydefstart
An OpenMP \emph{executable directive} (and for Fortran, the paired \code{end} \emph{directive}, if 
any) and the associated statement, loop or \emph{structured block}, if any, not including 
the code in any called routines. That is, the lexical extent of an \emph{executable 
directive}.
\glossarydefend

\glossaryterm{combined construct}
\glossarydefstart
A construct that is a shortcut for specifying one construct immediately nested inside another construct. A combined construct is semantically identical to that of explicitly specifying the first construct containing one instance of the second construct and no other statements.
\glossarydefend

\glossaryterm{composite construct} 
\glossarydefstart 
A construct that is composed of two constructs but does not have identical semantics to specifying one of the constructs immediately nested inside the other. A composite construct either adds semantics not included in the constructs from which it is composed or the nesting of the one construct inside the other is not conforming.
\glossarydefend


\glossaryterm{region}
\glossarydefstart
All code encountered during a specific instance of the execution of a given 
\emph{construct} or of an OpenMP library routine. A \emph{region} includes any code in called 
routines as well as any implicit code introduced by the OpenMP implementation. 
The generation of a \emph{task} at the point where a \emph{task generating construct} is encountered is a 
part of the \emph{region} of the \emph{encountering thread}, but an \emph{explicit task region}
associated with a \emph{task generating construct} is not unless it is an
\emph{included task region}. The point where a \code{target} or \code{teams}
directive is encountered is a part of the \emph{region} of the \emph{encountering thread}, but the 
\emph{region} associated with the \code{target} or \code{teams} directive is not.

\begin{quote}
COMMENTS:

A \emph{region} may also be thought of as the dynamic or runtime extent of a 
\emph{construct} or of an OpenMP library routine.

During the execution of an \emph{OpenMP program}, a \emph{construct} may give 
rise to many \emph{regions}.
\end{quote}
\glossarydefend

\glossaryterm{active parallel region}
\glossarydefstart
A \code{parallel} \emph{region} that is executed by a \emph{team} consisting of more than one 
\emph{thread}.
\glossarydefend

\smallskip
\glossaryterm{inactive parallel region}
\glossarydefstart
A \code{parallel} \emph{region} that is executed by a \emph{team} of only one \emph{thread}.
\glossarydefend

\glossaryterm{sequential part}
\glossarydefstart
All code encountered during the execution of an \emph{initial task region} that is not part 
of a \code{parallel} \emph{region} corresponding to a \code{parallel} \emph{construct} or a \code{task}
\emph{region} corresponding to a \code{task} \emph{construct}.

\begin{quote}
COMMENTS: 

A \emph{sequential part} is enclosed by an \emph{implicit parallel region}.

Executable statements in called routines may be in both a \emph{sequential 
part} and any number of explicit \code{parallel} \emph{regions} at different points 
in the program execution.
\end{quote}
\glossarydefend

\glossaryterm{master thread}
\glossarydefstart
An \emph{OpenMP thread} that has  \emph{thread} number 0. A \emph{master
thread} may be an \emph{initial thread} or the \emph{thread} that encounters a
\code{parallel} \emph{construct}, creates a \emph{team}, generates a set of
\emph{implicit tasks}, and then executes one of those \emph{tasks} as
\emph{thread} number 0.
\glossarydefend

\glossaryterm{parent thread}
\glossarydefstart
The \emph{thread} that encountered the \code{parallel} \emph{construct} and generated a 
\code{parallel} \emph{region} is the \emph{parent thread} of each of the 
\emph{threads} in the \emph{team} of that 
\code{parallel} \emph{region}. The \emph{master thread} 
of a \code{parallel} \emph{region} is the same \emph{thread} 
as its \emph{parent thread} with respect to any resources associated with an \emph{OpenMP thread}.
\glossarydefend

\glossaryterm{child thread}
\glossarydefstart
When a thread encounters a \code{parallel} construct, each of the threads in the 
generated \code{parallel} region's team are \emph{child threads} of the encountering \emph{thread}. 
The \code{target} or \code{teams} region's \emph{initial thread} is not a \emph{child thread} of the thread 
that encountered the \code{target} or \code{teams} construct. 
\glossarydefend

\glossaryterm{ancestor thread}
\glossarydefstart
For a given \emph{thread}, its \emph{parent thread} or one of its \emph{parent thread's ancestor threads}.
\glossarydefend

\glossaryterm{descendent thread}
\glossarydefstart
For a given \emph{thread}, one of its \emph{child threads} or one of 
its \emph{child threads' descendent threads}.
\glossarydefend

\glossaryterm{team}
\glossarydefstart
A set of one or more \emph{threads} participating in the execution of a \code{parallel}
\emph{region}.

\begin{quote}
COMMENTS:

For an \emph{active parallel region}, the team comprises the \emph{master thread} 
and at least one additional \emph{thread}.

For an \emph{inactive parallel region}, the \emph{team} comprises only the \emph{master thread}.
\end{quote}
\glossarydefend

\glossaryterm{league}
\glossarydefstart
The set of \emph{teams} created by a \code{teams} construct.
\glossarydefend

\glossaryterm{contention group}
\glossarydefstart
An initial \emph{thread} and its \emph{descendent threads}.
\glossarydefend

\glossaryterm{implicit parallel region}
\glossarydefstart
An \emph{inactive parallel region} that is not generated from a
\code{parallel} \emph{construct}. \emph{Implicit parallel regions} surround the whole
\emph{OpenMP program}, all \code{target} \emph{regions}, and all \code{teams}
\emph{regions}.

\glossarydefend

\glossaryterm{initial thread}
\glossarydefstart
A \emph{thread} that executes an \emph{implicit parallel region}.
\glossarydefend

\glossaryterm{initial team}
\glossarydefstart
A \emph{team} that comprises an \emph{initial thread} executing an \emph{implicit parallel region}.
\glossarydefend

\glossaryterm{nested construct}
\glossarydefstart
A \emph{construct} (lexically) enclosed by another \emph{construct}.
\glossarydefend

\glossaryterm{closely nested construct}
\glossarydefstart
A \emph{construct} nested inside another \emph{construct} with no other \emph{construct} nested 
between them.
\glossarydefend

\glossaryterm{nested region}
\glossarydefstart
A \emph{region} (dynamically) enclosed by another \emph{region}.  That is, a
\emph{region} generated from the execution of another \emph{region}
or one of its \emph{nested regions}.

\begin{quote}
COMMENT: Some nestings are \emph{conforming} and some are not. 
See \specref{sec:Nesting of Regions} for the restrictions on nesting.
\end{quote}
\glossarydefend

\glossaryterm{closely nested region}
\glossarydefstart
A \emph{region nested} inside another \emph{region} with no \code{parallel} \emph{region nested} between 
them. 
\glossarydefend

\glossaryterm{strictly nested region}
\glossarydefstart
A \emph{region nested} inside another \emph{region} with no other \emph{region nested} between 
them. 
\glossarydefend

\glossaryterm{all threads}
\glossarydefstart
All OpenMP \emph{threads} participating in the \emph{OpenMP program}.
\glossarydefend

\glossaryterm{current team}
\glossarydefstart
All \emph{threads} in the \emph{team} executing the innermost enclosing \code{parallel} \emph{region}.
\glossarydefend

\glossaryterm{encountering thread}
\glossarydefstart
For a given \emph{region}, the \emph{thread} that encounters the 
corresponding \emph{construct}.
\glossarydefend

\glossaryterm{all tasks}
\glossarydefstart
All \emph{tasks} participating in the \emph{OpenMP program}. 
\glossarydefend

\glossaryterm{current team tasks}
\glossarydefstart
All \emph{tasks} encountered by the corresponding \emph{team}. The \emph{implicit tasks}
constituting the \code{parallel} \emph{region} and any \emph{descendent tasks} encountered during 
the execution of these \emph{implicit tasks} are included in this set of tasks. 
\glossarydefend

\glossaryterm{generating task}
\glossarydefstart
For a given \emph{region}, the task for which execution by a \emph{thread} generated the \emph{region}.
\glossarydefend

\glossaryterm{binding thread set}
\glossarydefstart
The set of \emph{threads} that are affected by, or provide the context for, the execution of 
a \emph{region}. 

The \emph{binding thread} set for a given \emph{region} can be \emph{all threads} on a \emph{device}, \emph{all 
threads} in a \emph{contention group}, all \emph{master threads} executing an
enclosing \code{teams} \emph{region}, the \emph{current team}, or the \emph{encountering thread}.

\begin{quote}
COMMENT: The \emph{binding thread set} for a particular \emph{region} is described in its 
corresponding subsection of this specification.
\end{quote}
\glossarydefend

\glossaryterm{binding task set}
\glossarydefstart
The set of \emph{tasks} that are affected by, or provide the context for, the execution of a 
\emph{region}. 

The \emph{binding task} set for a given \emph{region} can be \emph{all tasks}, 
the \emph{current team tasks}, or the \emph{generating task}. 

\begin{quote}
COMMENT: The \emph{binding task} set for a particular \emph{region} (if applicable) is 
described in its corresponding subsection of this specification.
\end{quote}
\glossarydefend

\pagebreak
\glossaryterm{binding region}
\glossarydefstart
The enclosing \emph{region} that determines the execution context and limits the scope of 
the effects of the bound \emph{region} is called the \emph{binding region}.

\emph{Binding region} is not defined for \emph{regions} for which the \emph{binding thread} set is \emph{all threads}
or the \emph{encountering thread}, nor is it defined for \emph{regions} for which the \emph{binding task set} is 
\emph{all tasks}.

\begin{quote}
COMMENTS: 

The \emph{binding region} for an \code{ordered} \emph{region} is the innermost enclosing 
\emph{loop region}.

The \emph{binding region} for a \code{taskwait} \emph{region} is the innermost enclosing 
\emph{task region}.

The \emph{binding region} for a \code{cancel} \emph{region} is the innermost enclosing \emph{region} corresponding to the \plc{construct-type-clause} of the \code{cancel} construct.

The \emph{binding region} for a \code{cancellation point} \emph{region} is the innermost enclosing \emph{region} corresponding to the \plc{construct-type-clause} of the \code{cancellation point} construct.

For all other \emph{regions} for which the \emph{binding thread set} is the \emph{current
team} or the \emph{binding task set} is the \emph{current team tasks}, the \emph{binding 
region} is the innermost enclosing \code{parallel} \emph{region}.

For \emph{regions} for which the \emph{binding task set} is the \emph{generating task}, the 
\emph{binding region} is the \emph{region} of the \emph{generating task}.

A \code{parallel} \emph{region} need not be \emph{active} nor explicit to be a \emph{binding region}.

A \emph{task region} need not be explicit to be a \emph{binding region}.

A \emph{region} never binds to any \emph{region} outside of the innermost enclosing 
\code{parallel} \emph{region}.
\end{quote}
\glossarydefend

\glossaryterm{orphaned construct}
\glossarydefstart
A \emph{construct} that gives rise to a \emph{region} for which the \emph{binding thread set} is the \emph{current 
team}, but is not nested within another \emph{construct} giving rise to the \emph{binding region}.
\glossarydefend

\glossaryterm{worksharing construct}
\glossarydefstart
A \emph{construct} that defines units of work, each of which is executed exactly once by 
one of the \emph{threads} in the \emph{team} executing the \emph{construct}.

For C/C++, \emph{worksharing constructs} are \code{for}, \code{sections}, and \code{single}.

For Fortran, \emph{worksharing constructs} are \code{do}, \code{sections}, \code{single} and 
\code{workshare}.
\glossarydefend

\newpage
\glossaryterm{place}
\glossarydefstart
Unordered set of \emph{processors} on a device that is treated by the execution environment as a 
location unit when dealing with OpenMP thread affinity.
\glossarydefend

\glossaryterm{place list}
\glossarydefstart
The ordered list that describes all OpenMP \emph{places} available to the execution 
environment.
\glossarydefend

\glossaryterm{place partition}
\glossarydefstart
An ordered list that corresponds to a contiguous interval in the OpenMP \emph{place list}. 
It describes the \emph{places} currently available to the execution environment for a given 
parallel \emph{region}.
\glossarydefend

\glossaryterm{place number}
\glossarydefstart
A number that uniquely identifies a \emph{place} in the \emph{place list}, with zero identifying the first \emph{place} in the \emph{place list}, and each consecutive whole number identifying the next \emph{place} in the \emph{place list}.
\glossarydefend

\glossaryterm{SIMD instruction}
\glossarydefstart
A single machine instruction that can operate on multiple data elements.
\glossarydefend

\glossaryterm{SIMD lane}
\glossarydefstart
A software or hardware mechanism capable of processing one data element from a 
\emph{SIMD instruction}.
\glossarydefend

\glossaryterm{SIMD chunk}
\glossarydefstart
A set of iterations executed concurrently, each by a \emph{SIMD lane}, by a single \emph{thread}
by means of \emph{SIMD instructions}.
\glossarydefend


%
% Loop Terminology
%
\subsection{Loop Terminology}
\index{loop terminology}
\label{subsec:Loop Terminology}
\glossaryterm{loop directive}
\glossarydefstart
An OpenMP \emph{executable} directive for which the associated user code must be a loop nest that is a \emph{structured block}.
\glossarydefend

\glossaryterm{associated loop(s)}
\glossarydefstart
The loop(s) controlled by a \emph{loop directive}.
\begin{quote}
COMMENT: If the \emph{loop directive} contains a \code{collapse} or an \code{ordered(}\plc{n}\code{)} clause then it may have more than one \emph{associated loop}.
\end{quote}
\glossarydefend

\glossaryterm{sequential loop}
\glossarydefstart
A loop that is not associated with any OpenMP \emph{loop directive}.
\glossarydefend

\glossaryterm{SIMD loop}
\glossarydefstart
A loop that includes at least one \emph{SIMD chunk}.
\glossarydefend

\glossaryterm{doacross loop nest}
\glossarydefstart
A loop nest that has cross-iteration dependence. An iteration is dependent on one or more lexicographically earlier iterations.
\begin{quote}
COMMENT: The \code{ordered} clause parameter on a loop directive identifies the loop(s) associated with the \emph{doacross loop nest}.
\end{quote}
\glossarydefend

%
% Synchronization Terminology
%
\subsection{Synchronization Terminology}
\index{synchronization terminology}
\label{subsec:Synchronization Terminology}
\glossaryterm{barrier}
\glossarydefstart
A point in the execution of a program encountered by a \emph{team} of \emph{threads}, beyond 
which no \emph{thread} in the team may execute until all \emph{threads} in the \emph{team} have 
reached the barrier and all \emph{explicit tasks} generated by the \emph{team} have executed to 
completion. If \emph{cancellation} has been requested, threads may proceed to the end of 
the canceled \emph{region} even if some threads in the team have not reached the \emph{barrier}.
\glossarydefend

\glossaryterm{cancellation}
\glossarydefstart
An action that cancels (that is, aborts) an OpenMP \emph{region} and causes executing 
\emph{implicit} or \emph{explicit} tasks to proceed to the end of the canceled \emph{region}. 
\glossarydefend

\glossaryterm{cancellation point}
\glossarydefstart
A point at which implicit and explicit tasks check if cancellation has been 
requested. If cancellation has been observed, they perform the \emph{cancellation}. 

\begin{quote}
COMMENT: For a list of cancellation points, see \specref{subsec:cancel Construct}
\end{quote}
\glossarydefend
\bigskip

\glossaryterm{flush}
\glossarydefstart
An operation, applied to a set of variables, that a \emph{thread} performs
to enforce consistency between its view and other \emph{threads}' view of
memory.
\glossarydefend

\glossaryterm{flush-set}
\glossarydefstart
The set of variables to which a given \emph{flush} operation is applied.
\glossarydefend

\glossaryterm{flush property}
\glossarydefstart
Properties that determine the manner in which a \emph{flush} operation enforces
memory consistency. These properties are:
\begin{itemize}
    \item \emph{strong}:  guarantees a completion ordering between memory operations
        on different threads upon which all threads agree;
    \item \emph{write}: makes modifications by the executing thread visible
        to other threads;
    \item \emph{read}: makes modifications by other threads visible to the
        executing thread;
    \item \emph{synchronizable}: allows the \emph{flush} operation to
        synchronize with another \emph{flush} operation.
\end{itemize}

\begin{quote}
COMMENT: A given \emph{flush} operation can have one or more \emph{flush
properties}, but must have at least the write or read flush property.
\end{quote}
\glossarydefend

\glossaryterm{strong flush}
\glossarydefstart
A \emph{flush} operation that has the \emph{strong flush property}.
\glossarydefend

\glossaryterm{weak flush}
\glossarydefstart
A \emph{flush} operation that does not have the \emph{strong flush property}.
\glossarydefend

\glossaryterm{write flush}
\glossarydefstart
A \emph{flush} operation that has the \emph{write flush property}.
\glossarydefend

\glossaryterm{read flush}
\glossarydefstart
A \emph{flush} operation that has the \emph{read flush property}.
\glossarydefend

\glossaryterm{synchronizable flush}
\glossarydefstart
A \emph{flush} operation that has the \emph{synchronizable flush property}.
\glossarydefend

\glossaryterm{non-synchronizable flush}
\glossarydefstart
A \emph{flush} operation that does not have the \emph{synchronizable flush
property}.
\glossarydefend

\glossaryterm{release flush}
\glossarydefstart
A \emph{write flush} that is a \emph{synchronizable flush}.
\glossarydefend

\glossaryterm{acquire flush}
\glossarydefstart
A \emph{read flush} that is a \emph{synchronizable flush}.
\glossarydefend

\glossaryterm{sync-set}
\glossarydefstart
The set of variables associated with a \emph{release flush} or \emph{acquire flush} that a
program may use to force a \emph{release flush} to synchronize with an \emph{acquire flush}.
\glossarydefend





\subsection{Tasking Terminology}
\index{tasking terminology}
\label{subsec:Tasking Terminology}
\glossaryterm{task}
\glossarydefstart
A specific instance of executable code and its data environment that the
OpenMP implementation can schedule for execution by threads.
\glossarydefend

\glossaryterm{task region}
\glossarydefstart
A \emph{region} consisting of all code encountered during the execution of a \emph{task}. 

\begin{quote}
COMMENT: A \code{parallel} \emph{region} consists of one or more implicit \emph{task regions}. 
\end{quote}
\glossarydefend

\glossaryterm{implicit task}
\glossarydefstart
A \emph{task} generated by an \emph{implicit parallel region} or generated when a \code{parallel}
\emph{construct} is encountered during execution.
\glossarydefend

\glossaryterm{explicit task}
\glossarydefstart
A \emph{task} that is not an \emph{implicit task}.
\glossarydefend

\glossaryterm{initial task}
\glossarydefstart
An \emph{implicit task} associated with an \emph{implicit parallel region}.
\glossarydefend

\glossaryterm{current task}
\glossarydefstart
For a given \emph{thread}, the \emph{task} corresponding to the \emph{task region} in which it is 
executing.
\glossarydefend

\glossaryterm{child task}
\glossarydefstart
A \emph{task} is a \emph{child task} of its generating \emph{task region}. 
A \emph{child task region} is not part of its generating \emph{task region}.
\glossarydefend

\glossaryterm{sibling tasks}
\glossarydefstart
\emph{Tasks} that are \emph{child tasks} of the same \emph{task region}.
\glossarydefend

\glossaryterm{descendent task}
\glossarydefstart
A \emph{task} that is the \emph{child task} of a \emph{task region} or of one of its 
\emph{descendent task regions}.
\glossarydefend

\glossaryterm{task completion}
\glossarydefstart
\emph{Task completion} occurs when the end of the \emph{structured block} associated with the 
\emph{construct} that generated the \emph{task} is reached.

\begin{quote}
COMMENT: Completion of the \emph{initial task} that is generated when the program begins occurs at program exit.
\end{quote}
\glossarydefend

\glossaryterm{task scheduling point}
\glossarydefstart
A point during the execution of the current \emph{task region} at which it can be 
suspended to be resumed later; or the point of \emph{task completion}, after which the 
executing thread may switch to a different \emph{task region}. 

\begin{quote}
COMMENT: For a list of \emph{task scheduling points}, see \specref{subsec:Task Scheduling}.
\end{quote}
\glossarydefend

\glossaryterm{task switching}
\glossarydefstart
The act of a \emph{thread} switching from the execution of one \emph{task} to another \emph{task}.
\glossarydefend

\glossaryterm{tied task}
\glossarydefstart
A \emph{task} that, when its \emph{task region} is suspended, can be resumed only by the same 
\emph{thread} that suspended it. That is, the \emph{task} is tied to that \emph{thread}. 
\glossarydefend

\glossaryterm{untied task}
\glossarydefstart
A \emph{task} that, when its \emph{task region} is suspended, can be resumed by any \emph{thread} in 
the team. That is, the \emph{task} is not tied to any \emph{thread}. 
\glossarydefend

\glossaryterm{undeferred task}
\glossarydefstart
A \emph{task} for which execution is not deferred with respect to its generating \emph{task} 
\emph{region}. That is, its generating \emph{task region} is suspended until execution of the 
\emph{undeferred task} is completed.
\glossarydefend

\glossaryterm{included task}
\glossarydefstart
A \emph{task} for which execution is sequentially included in the generating \emph{task region}. 
That is, an \emph{included task} is \emph{undeferred} and executed immediately by the 
\emph{encountering thread}.
\glossarydefend

\glossaryterm{merged task}
\glossarydefstart
A \emph{task} for which the \emph{data environment}, inclusive of ICVs, is the same as that of its 
generating \emph{task region}.
\glossarydefend

\glossaryterm{mergeable task}
\glossarydefstart
A \emph{task} that may be a \emph{merged task} if it is an \emph{undeferred task} or an \emph{included task}.
\glossarydefend

\glossaryterm{final task}
\glossarydefstart
A \emph{task} that forces all of its \emph{child tasks} to become \emph{final} and \emph{included tasks}.
\glossarydefend

\glossaryterm{task dependence}
\glossarydefstart
An ordering relation between two \emph{sibling tasks}: the \emph{dependent task} and a 
previously generated \emph{predecessor task}. The \emph{task dependence} is fulfilled when the 
\emph{predecessor task} has completed.
\glossarydefend

\begin{samepage}
\glossaryterm{dependent task}
\glossarydefstart
A \emph{task} that because of a \emph{task dependence} cannot be executed until its \emph{predecessor 
tasks} have completed.
\glossarydefend
\end{samepage}

\glossaryterm{mutually exclusive tasks}
\glossarydefstart
\emph{Tasks} that may be executed in any order, but not at the same
time.
\glossarydefend
\bigskip

\glossaryterm{predecessor task}
\glossarydefstart
A \emph{task} that must complete before its \emph{dependent tasks} can be executed.
\glossarydefend

\glossaryterm{task synchronization construct}
\glossarydefstart
A \code{taskwait}, \code{taskgroup}, or a \code{barrier} \emph{construct}.
\glossarydefend
\bigskip

\glossaryterm{task generating construct}
\glossarydefstart
A \emph{construct} that generates one or more \emph{explicit tasks}.
\glossarydefend
\bigskip

\glossaryterm{target task}
\glossarydefstart
A \emph{mergeable} and \emph{untied} \emph{task} that is generated by a \code{target}, \code{target enter data}, \code{target exit data}, or \code{target update} \emph{construct}.
\glossarydefend

\glossaryterm{taskgroup set}
\glossarydefstart
A set of tasks that are logically grouped by a \code{taskgroup} \emph{region}.
\glossarydefend

\subsection{Data Terminology}
\index{data terminology}
\label{subsec:Data Terminology} 
\glossaryterm{variable}
\glossarydefstart
A named data storage block, for which the value can be defined and redefined during the 
execution of a program.

\begin{adjustwidth}{-0.75in}{0in}
\begin{note}
An array or structure element is a variable that is part of another variable.
\end{note}
\end{adjustwidth}
\glossarydefend

\glossaryterm{scalar variable}
\glossarydefstart
For C/C++:
\nopagebreak
A scalar variable, as defined by the base language.

For Fortran:
\nopagebreak
A scalar variable with intrinsic type, as defined by the base language,
excluding character type.
\glossarydefend

%\glossaryterm{aggregate variable}
%\glossarydefstart
%A variable, such as an array or structure, composed of other variables.
%\glossarydefend

\glossaryterm{array section}
\glossarydefstart
A designated subset of the elements of an array.
\glossarydefend

\glossaryterm{array item}
\glossarydefstart
An array, an array section, or an array element.
\glossarydefend

\glossaryterm{base expression}
\glossarydefstart
For C/C++: The expression in an array section or an array element that specifies
the address of the initial element of the original array.
\glossarydefend

\glossaryterm{named array}
\glossarydefstart
For C/C++:
\nopagebreak
An expression that is an array but not an array element and appears as the
array referred to by a given array item.

For Fortran:
\nopagebreak
A variable that is an array and appears as the array referred to by a given
array item.
\glossarydefend

\glossaryterm{named pointer}
\glossarydefstart
For C/C++:
\nopagebreak
An lvalue expression that is a pointer and appears as a pointer to the array
implicitly referred to by a given array item.

For Fortran:
\nopagebreak
A variable that has the \code{POINTER} attribute and appears as a pointer to
the array to which a given array item implicitly refers.
%array implicitly referred to by a given array item

\begin{adjustwidth}{-0.75in}{0in}
\begin{note}
A given array item cannot have a \emph{named pointer} if it has a \emph{named array}.
\end{note}
\end{adjustwidth}
\glossarydefend


\glossaryterm{attached pointer}
\glossarydefstart
A pointer variable in a device data environment to which the effect of a \code{map} clause 
assigns the address of
an array section.  The pointer is
an attached pointer for the remainder of its lifetime in the device data environment.
\glossarydefend
\bigskip

\glossaryterm{simply contiguous array section}
\glossarydefstart
An array section that statically can be determined to have contiguous storage.
\glossarydefend
\bigskip

\glossaryterm{structure}
\glossarydefstart
A structure is a variable that contains one or more variables. 

For C/C++: 
\nopagebreak
Implemented using struct types.

For C++: 
\nopagebreak
Implemented using class types.        

For Fortran: 
\nopagebreak
Implemented using derived types.        
\glossarydefend

\glossaryterm{private variable}
\glossarydefstart
With respect to a given set of \emph{task regions} or \emph{SIMD lanes} that bind to the same
\code{parallel} \emph{region}, a \emph{variable} for which the name provides access to a different block of 
storage for each \emph{task region} or \emph{SIMD lane}.

A \emph{variable} that is part of another variable (as an array or structure element) cannot 
be made private independently of other components.
\glossarydefend

\glossaryterm{shared variable}
\glossarydefstart
With respect to a given set of \emph{task regions} that bind to the same \code{parallel} 
\emph{region}, a \emph{variable} for which the name provides access to the same block of storage for 
each \emph{task region}.

A \emph{variable} that is part of another variable (as an array or structure element) cannot 
be \emph{shared} independently of the other components, except for static data members 
of C++ classes.
\glossarydefend

\glossaryterm{threadprivate variable}
\glossarydefstart
A \emph{variable} that is replicated, one instance per \emph{thread}, by the OpenMP 
implementation. Its name then provides access to a different block of storage for 
each \emph{thread}.

A \emph{variable} that is part of another variable (as an array or structure element) cannot 
be made \emph{threadprivate} independently of the other components, except for static 
data members of C++ classes. 
\glossarydefend

\glossaryterm{threadprivate memory}
\glossarydefstart
The set of \emph{threadprivate variables} associated with each \emph{thread}.
\glossarydefend

\glossaryterm{data environment}
\glossarydefstart
The \emph{variables} associated with the execution of a given \emph{region}. 
\glossarydefend

\glossaryterm{device data environment}
\glossarydefstart
The initial \emph{data environment} associated with a device.
\glossarydefend
\bigskip

\glossaryterm{device address}
\glossarydefstart
An \emph{implementation defined} reference to an address in a \emph{device
  data environment}.
\glossarydefend

\glossaryterm{device pointer}
\glossarydefstart
A \emph{variable} that contains a \emph{device address}.
\glossarydefend


\glossaryterm{mapped variable}
\glossarydefstart
An original \emph{variable} in a \emph{data environment} with a corresponding \emph{variable} in a 
device \emph{data environment}.

\begin{quote}
COMMENT: The original and corresponding \emph{variables} may share storage.
\end{quote}
\glossarydefend

\glossaryterm{mappable type}
\glossarydefstart
A type that is valid for a \emph{mapped variable}. If a type is composed from other types 
(such as the type of an array or structure element) and any of the other types are 
not mappable then the type is not mappable.

\begin{quote}
COMMENT: Pointer types are \emph{mappable} but the memory block to which the pointer refers is not \emph{mapped}.
\end{quote}

For C: 
\nopagebreak
The type must be a complete type.

For C++: 
\nopagebreak
The type must be a complete type.

In addition, for class types:
\begin{itemize}
\item All member functions accessed in any \code{target} region must appear in a 
\code{declare}~\code{target} directive.
\end{itemize}

For Fortran: 
\nopagebreak
No restrictions on the type except that for derived types:

\begin{itemize}
\item All type-bound procedures accessed in any target region must appear in a \code{declare}~\code{target} directive.
\end{itemize}
\glossarydefend

\glossaryterm{defined}
\glossarydefstart
For \emph{variables}, the property of having a valid value.

For C:
\nopagebreak
For the contents of \emph{variables}, the property of having a valid value.

For C++: 
\nopagebreak
For the contents of \emph{variables} of POD (plain old data) type, the property of having 
a valid value.

For \emph{variables} of non-POD class type, the property of having been constructed but 
not subsequently destructed.

For Fortran: 
\nopagebreak
For the contents of \emph{variables}, the property of having a valid value. For the 
allocation or association status of \emph{variables}, the property of having a valid status.

\begin{quote}
COMMENT: Programs that rely upon \emph{variables} that are not \emph{defined} are \emph{non-conforming programs}.
\end{quote}
\glossarydefend

\glossaryterm{class type}
\glossarydefstart
For C++: \emph{Variables} declared with one of the \code{class}, \code{struct}, or \code{union} keywords
\glossarydefend

\glossaryterm{sequentially consistent atomic construct}
\glossarydefstart
An \code{atomic} construct for which the \code{seq\_cst} clause is specified.
\glossarydefend
\bigskip

\glossaryterm{non-sequentially consistent atomic construct}
\glossarydefstart
An \code{atomic} construct for which the \code{seq\_cst} clause is not specified
\glossarydefend
\bigskip
\bigskip
\bigskip





\subsection{Implementation Terminology}
\index{implementation terminology}
\label{subsec:Implementation Terminology}
\glossaryterm{supporting \emph{n} levels of parallelism}
\glossarydefstart
Implies allowing an \emph{active parallel region} to be enclosed by \emph{n-1} \emph{active parallel 
regions}.
\glossarydefend

\glossaryterm{supporting the OpenMP API}
\glossarydefstart
Supporting at least one level of parallelism.
\glossarydefend
\bigskip

\glossaryterm{supporting nested parallelism}
\glossarydefstart
Supporting more than one level of parallelism.  
\glossarydefend
\bigskip

\glossaryterm{internal control variable}
\glossarydefstart
A conceptual variable that specifies runtime behavior of a set of \emph{threads} or \emph{tasks} 
in an \emph{OpenMP program}.

\begin{quote}
COMMENT: The acronym ICV is used interchangeably with the term \emph{internal 
control variable} in the remainder of this specification.
\end{quote}
\glossarydefend

\glossaryterm{compliant implementation}
\glossarydefstart
An implementation of the OpenMP specification that compiles and executes any 
\emph{conforming program} as defined by the specification.

\begin{quote}
COMMENT: A \emph{compliant implementation} may exhibit \emph{unspecified behavior} when 
compiling or executing a \emph{non-conforming program}.
\end{quote}
\glossarydefend

\glossaryterm{unspecified behavior}
\glossarydefstart
A behavior or result that is not specified by the OpenMP specification or not 
known prior to the compilation or execution of an \emph{OpenMP program}.

Such \emph{unspecified behavior} may result from:

\begin{itemize}
\item Issues documented by the OpenMP specification as having \emph{unspecified 
behavior}.

\item A \emph{non-conforming program}.

\item A \emph{conforming program} exhibiting an \emph{implementation defined} behavior.
\end{itemize}
\glossarydefend

\glossaryterm{implementation defined}
\glossarydefstart
Behavior that must be documented by the implementation, and is allowed to vary 
among different \emph{compliant implementations}. An implementation is allowed to 
define this behavior as \emph{unspecified}.

\begin{quote}
COMMENT: All features that have \emph{implementation defined} behavior 
are documented in Appendix~\ref{chap:OpenMP Implementation-Defined Behaviors}.
\end{quote}
\glossarydefend

\glossaryterm{deprecated} 
\glossarydefstart
Implies a construct, clause, or other feature is normative in the current specification but is considered obsolescent and will be removed in the future.
\glossarydefend

\subsection{Tool Terminology}

\glossaryterm{tool} 
\glossarydefstart
Executable code, distinct from application or runtime code, that can observe and/or modify the execution of an application.
\glossarydefend

\glossaryterm{first-party tool} 
\glossarydefstart
A tool that executes in the address space of the program it is monitoring.
\glossarydefend



\glossaryterm{activated tool} 
\glossarydefstart
A first-party tool that successfully completed its initialization.
\glossarydefend

\glossaryterm{event} 
\glossarydefstart
A point of interest in the execution of a thread where the condition
defining that event is true. 
\glossarydefend

\glossaryterm{tool callback} 
\glossarydefstart
A function provided by a tool to an OpenMP implementation that can be invoked when needed.
\glossarydefend

\glossaryterm{registering a callback} 
\glossarydefstart
Providing a callback function to an OpenMP implementation for a particular purpose.
\glossarydefend

\glossaryterm{dispatching a callback at an event} 
\glossarydefstart
Processing a callback when an associated event occurs in a manner consistent with the return code
provided when a \emph{first-party} tool registered the callback.
\glossarydefend

\glossaryterm{thread state} 
\glossarydefstart
An enumeration type that describes what an OpenMP thread is currently doing.  
A thread can be in only one state at any time.
\glossarydefend

\glossaryterm{wait identifier} 
\glossarydefstart
A unique opaque handle associated with each data object (e.g., a lock) used by the OpenMP runtime
to enforce mutual exclusion that may cause a thread to wait actively or passively.
\glossarydefend

\glossaryterm{frame} 
\glossarydefstart
A storage area on a thread's stack associated with a procedure invocation. A frame includes space for 
one or more saved registers and often also includes space for saved arguments, local variables, 
and padding for alignment.
\glossarydefend

\glossaryterm{canonical frame address} 
\glossarydefstart
An address associated with a procedure \emph{frame} on a call stack defined as the value of the stack pointer immediately prior 
to calling the procedure whose invocation the frame represents.
\glossarydefend

\glossaryterm{runtime entry point} 
\glossarydefstart
A function interface provided by an OpenMP runtime for use by a tool. A runtime entry point is
typically not associated with a global function symbol.
\glossarydefend

\glossaryterm{trace record} 
\glossarydefstart
A data structure to store information associated with an occurrence of an \emph{event}.
\glossarydefend

\glossaryterm{native trace record} 
\glossarydefstart
A \emph{trace record} for an OpenMP device that is in a device-specific format. 
\glossarydefend

\glossaryterm{signal} 
\glossarydefstart
A software interrupt delivered to a thread.
\glossarydefend

\glossaryterm{signal handler} 
\glossarydefstart
A function called asynchronously when a \emph{signal} is delivered to a thread.
\glossarydefend

\glossaryterm{async signal safe} 
\glossarydefstart
Guaranteed not to interfere with operations that are being interrupted by \emph{signal} delivery. 
An async signal safe \emph{runtime entry point} is safe to call from a \emph{signal handler}.
\glossarydefend

\glossaryterm{code block} 
\glossarydefstart
A contiguous region of memory that contains code of an OpenMP program to be executed on a device.
\glossarydefend



\section{Execution Model}
\label{sec:Execution Model}
\index{execution model}
The OpenMP API uses the fork-join model of parallel execution. Multiple threads of
execution perform tasks defined implicitly or explicitly by OpenMP directives. The
OpenMP API is intended to support programs that will execute correctly both as parallel
programs (multiple threads of execution and a full OpenMP support library) and as
sequential programs (directives ignored and a simple OpenMP stubs library). However,
it is possible and permitted to develop a program that executes correctly as a parallel
program but not as a sequential program, or that produces different results when 
executed as a parallel program compared to when it is executed as a sequential program. 
Furthermore, using different numbers of threads may result in different numeric results 
because of changes in the association of numeric operations. For example, a serial 
addition reduction may have a different pattern of addition associations than a parallel 
reduction. These different associations may change the results of floating-point addition.

An OpenMP program begins as a single thread of execution, called an initial thread. An 
initial thread executes sequentially, as if the code encountered is part of an implicit task region, called an
initial task region, that is generated by the implicit parallel region surrounding the whole
program.

The thread that executes the implicit parallel region that surrounds the whole program 
executes on the \emph{host device}. An implementation may support 
other \emph{target devices}. If
supported, one or more devices are available to the host device for offloading code and 
data. Each device has its own threads that are distinct from threads that execute on 
another device. Threads cannot migrate from one device to another device. The 
execution model is host-centric such that the host device offloads \code{target} regions to
target devices.

When a \code{target} construct is encountered, a new \emph{target task} is generated.
The \emph{target task} region encloses the \code{target} region. The \emph{target task} is 
complete after the execution of the \code{target} region is complete.

When a \emph{target task} executes, the enclosed \code{target} region is executed by an initial 
thread.  The initial thread may execute on a \emph{target device}.  The initial thread executes 
sequentially, as if the target region is part of an initial task region that is
generated by an implicit parallel region. If the target device does not exist or the implementation does not support the target
device, all \code{target} regions associated with that device execute on the host device.

The implementation must ensure that the \code{target} region executes as if it were executed in the data environment of the target device unless an \code{if} clause is present and the \code{if} clause expression evaluates to \plc{false}.

The \code{teams} construct creates a \emph{league of teams}, where each team
is an initial team that comprises an initial thread that executes the
\code{teams} region. Each initial thread executes sequentially, as if the
code encountered is part of an initial task region that is generated by an
implicit parallel region associated with each team.

If a construct creates a data environment, the data environment is created at the time the
construct is encountered. Whether a construct creates a data environment is defined in 
the description of the construct.

When any thread encounters a \code{parallel} construct, the thread creates a team of itself
and zero or more additional threads and becomes the master of the new team. A set of 
implicit tasks, one per thread, is generated. The code for each task is defined by the code 
inside the \code{parallel} construct. Each task is assigned to a different thread in the team
and becomes tied; that is, it is always executed by the thread to which it is initially 
assigned. The task region of the task being executed by the encountering thread is 
suspended, and each member of the new team executes its implicit task. There is an 
implicit barrier at the end of the \code{parallel} construct. Only the master thread resumes
execution beyond the end of the \code{parallel} construct, resuming the task region that
was suspended upon encountering the \code{parallel} construct. Any number of
\code{parallel} constructs can be specified in a single program.

\code{parallel} regions may be arbitrarily nested inside each other. If nested parallelism is
disabled, or is not supported by the OpenMP implementation, then the new team that is 
created by a thread encountering a \code{parallel} construct inside a \code{parallel} region
will consist only of the encountering thread. However, if nested parallelism is supported 
and enabled, then the new team can consist of more than one thread. A \code{parallel}
construct may include a \code{proc\_bind} clause to specify the places to use for the threads
in the team within the \code{parallel} region.

When any team encounters a worksharing construct, the work inside the construct is 
divided among the members of the team, and executed cooperatively instead of being 
executed by every thread. There is a default barrier at the end of each worksharing 
construct unless the \code{nowait} clause is present. Redundant execution of code by every
thread in the team resumes after the end of the worksharing construct.

When any thread encounters a \emph{task generating construct}, one or more explicit tasks are generated.
Execution of explicitly generated tasks is assigned to one of the threads in the current 
team, subject to the thread's availability to execute work. Thus, execution of the new 
task could be immediate, or deferred until later according to task scheduling constraints 
and thread availability. Threads are allowed to suspend the current task region at a task 
scheduling point in order to execute a different task. If the suspended task region is for
a tied task, the initially assigned thread later resumes execution of the suspended task
region. If the suspended task region is for an untied task, then any thread may resume its
execution. Completion of all explicit tasks bound to a given parallel region is guaranteed
before the master thread leaves the implicit barrier at the end of the region. Completion
of a subset of all explicit tasks bound to a given parallel region may be specified through
the use of task synchronization constructs. Completion of all explicit tasks bound to the
implicit parallel region is guaranteed by the time the program exits.

When any thread encounters a \code{simd} construct, the iterations of the loop associated with
the construct may be executed concurrently using the SIMD lanes that are available to
the thread.

When a thread encounters a \code{concurrent} construct, an implementation may
choose to apply optimizations, parallelization, and/or vectorization, including
scheduling loop iterations in ways that cannot be directly represented in
OpenMP. When a \code{concurrent} construct is encountered, an implementation
must guarantee that the logical iterations of the associated loops are executed
exactly once. When encountering a \code{concurrent} construct an implementation
will not generate a new league of thread teams, but may use any thread teams
currently available on which to execute. 

The \code{cancel} construct can alter the previously described flow of execution in an
OpenMP region. The effect of the \code{cancel} construct depends on its 
\plc{construct-type-clause}. If a task encounters a \code{cancel} 
construct with a \code{taskgroup} 
\plc{construct-type-clause}, then the task activates cancellation 
and continues execution at the end of its
\code{task} region, which implies completion of that task. 
Any other task in that \code{taskgroup}
that has begun executing completes execution unless it encounters a \code{cancellation}
\code{point} construct, in which case it continues execution at the end of its \code{task} region,
which implies its completion. Other tasks in that \code{taskgroup} region that have not
begun execution are aborted, which implies their completion.

For all other \plc{construct-type-clause} values, if a 
thread encounters a \code{cancel} construct, it
activates cancellation of the innermost enclosing region of the type specified and the 
thread continues execution at the end of that region. Threads check if cancellation has 
been activated for their region at cancellation points and, if so, also resume execution at 
the end of the canceled region.

If cancellation has been activated regardless of \plc{construct-type-clause}, 
threads that are
waiting inside a barrier other than an implicit barrier at the end of the canceled region 
exit the barrier and resume execution at the end of the canceled region. This action can 
occur before the other threads reach that barrier.

Synchronization constructs and library routines are available in the OpenMP API to 
coordinate tasks and data access in \code{parallel} regions. In addition, library routines and
environment variables are available to control or to query the runtime environment of 
OpenMP programs.

The OpenMP specification makes no guarantee that input or output to the same file is 
synchronous when executed in parallel. In this case, the programmer is responsible for 
synchronizing input and output statements (or routines) using the provided 
synchronization constructs or library routines. For the case where each thread accesses a 
different file, no synchronization by the programmer is necessary.








\section{Memory Model}
\label{sec:Memory Model}
\index{memory model}
\index{modification order}
\subsection{Structure of the OpenMP Memory Model}
\label{subsec:Structure of the OpenMP Memory Model}
The OpenMP API provides a relaxed-consistency, shared-memory model. All OpenMP
threads have access to a place to store and to retrieve variables, 
called the \emph{memory}. In
addition, each thread is allowed to have its own \emph{temporary view} of the memory. The
temporary view of memory for each thread is not a required part of the OpenMP
memory model, but can represent any kind of intervening structure, such as machine
registers, cache, or other local storage, between the thread and the memory. The
temporary view of memory allows the thread to cache variables and thereby to avoid
going to memory for every reference to a variable. Each thread also has access to
another type of memory that must not be accessed by other threads, 
called \emph{threadprivate memory}.

A directive that accepts data-sharing attribute clauses determines two kinds of access to
variables used in the directive's associated structured block: shared and private. Each
variable referenced in the structured block has an original variable, which is the variable  
by the same name that exists in the program immediately outside the construct. Each
reference to a shared variable in the structured block becomes a reference to the original 
variable. For each private variable referenced in the structured block, a new version of
the original variable (of the same type and size) is created in memory for each task or
SIMD lane that contains code associated with the directive. Creation of the new version
does not alter the value of the original variable. However, the impact of attempts to
access the original variable during the region associated with the directive is
unspecified; see \specref{subsubsec:private clause} for additional details. References to a
private variable in the structured block refer to the private version of the original
variable for the current task or SIMD lane. The relationship between the value of the
original variable and the initial or final value of the private version depends on the exact
clause that specifies it. Details of this issue, as well as other issues with privatization,
are provided in \specref{sec:Data Environment}.

The minimum size at which a memory update may also read and write back adjacent 
variables that are part of another variable (as array or structure elements) is
implementation defined but is no larger than required by the base language. 

A single access to a variable may be implemented with multiple load or store
instructions, and hence is not guaranteed to be atomic with respect to other accesses to
the same variable. Accesses to variables smaller than the implementation defined
minimum size or to C or C++ bit-fields may be implemented by reading, modifying, and 
rewriting a larger unit of memory, and may thus interfere with updates of variables or
fields in the same unit of memory.

If multiple threads write without synchronization to the same memory unit, including
cases due to atomicity considerations as described above, then a data race occurs.
Similarly, if at least one thread reads from a memory unit and at least one thread writes
without synchronization to that same memory unit, including cases due to atomicity
considerations as described above, then a data race occurs. If a data race occurs then the
result of the program is unspecified.

Every variable has an associated \emph{modification order}, defined as a
sequential ordering of all operations that store a value into the variable
without causing a data race to occur. 

A private variable in a task region that eventually generates an inner nested \code{parallel}
region is permitted to be made shared by implicit tasks in the inner \code{parallel} region.
A private variable in a task region can be shared by an explicit task region generated
during its execution. However, it is the programmer's responsibility to ensure through
synchronization that the lifetime of the variable does not end before completion of the
explicit task region sharing it. Any other access by one task to the
private variables of another task results in unspecified behavior.




\subsection{Device Data Environments}
\label{subsec:Device Data Environments}
\index{device data environments}
When an OpenMP program begins, an implicit \code{target}~\code{data} region for each device surrounds the whole program. Each device has a device data environment that is defined by its implicit \code{target}~\code{data} region. Any \code{declare}~\code{target} directives and the directives that accept data-mapping attribute clauses determine how an original variable in a data environment is mapped to a corresponding variable in a device data environment.

When an original variable is mapped to a device data environment and the associated corresponding variable is not present in the device data environment, a new corresponding variable (of the same type and size as the original variable) is created in the device data environment. The initial value of the new corresponding variable is determined from the clauses and the data environment of the encountering thread.

The corresponding variable in the device data environment may share storage with the
original variable. Writes to the corresponding variable may alter the value of the original
variable. The impact of this on memory consistency is discussed in 
\specref{subsec:OpenMP Memory Consistency}. 
When a task executes in the context of a device data environment, references to  
the original variable refer to the corresponding variable in the device data environment.

The relationship between the value of the original variable and the initial or final value
of the corresponding variable depends on the \plc{map-type}. Details of this issue, as well as
other issues with mapping a variable, are provided in \specref{subsec:map Clause}.

The original variable in a data environment and the corresponding variable(s) in one or
more device data environments may share storage. Without intervening synchronization  
data races can occur.




\subsection{The Flush Operation}
\label{subsec:The Flush Operation}
\index{flush operation}
\index{flush-set}
\index{sync-set}

The memory model has relaxed-consistency because a thread's temporary view of
memory is not required to be consistent with memory at all times. A value written to a
variable can remain in the thread's temporary view until it is forced to memory at a later
time. Likewise, a read from a variable may retrieve the value from the thread's
temporary view, unless it is forced to read from memory. The OpenMP flush operation
enforces consistency between the temporary view and memory.

The flush operation is applied to a set of variables called the
\emph{flush-set}. It also has an associated \emph{sync-set}
containing a set of synchronization variables that may be used to force
synchronization between the flush operation and another flush operation.  A flush operation
must be a write flush, a read flush, or both a write flush and a
read flush; accordingly, it restricts reordering of memory operations
performed by the same thread that an implementation might otherwise do as
follows:

\begin{itemize}
\item With respect to a write flush, an implementation must not reorder any
    prior memory operation in the thread's program order that reads or writes
    to a variable in its flush-set or any subsequent memory operation in the
    thread's program order that writes to a variable in its sync-set.

\item With respect to a read flush, an implementation must not reorder any
    subsequent memory operation in the thread's program order that reads or
    writes to a variable in its flush-set or any prior memory operation in the
    thread's program order that reads from a variable in its sync-set.

\item With respect to a read flush, an implementation must not reorder any
    subsequent write flush in the thread's program order that is applied to
    the same variable.
\end{itemize}

A strong flush imposes stronger ordering requirements across threads
compared to a weak flush, as described in
\specref{subsec:OpenMP Memory Consistency}.

If a thread has performed a write to its temporary view of a shared variable
since its last write flush of that variable, then when it executes another
write flush of the variable, the flush does not complete until the value of
the variable has been written to the variable in memory.  If a thread performs
multiple writes to the same variable between two write flushes of that
variable, the write flush ensures that it is the value of the last write is
written to the variable in memory. The completion of a write flush of a set of
variables executed by a thread is defined as the point at which all writes to
those variables performed by the thread before the flush are visible in
memory.

If a thread has not performed a write to its temporary view of a shared
variable since its last write flush of that variable, a read flush of the
variable executed by a thread causes its temporary view of the variable to be
discarded. If its next memory operation following the read flush for that
variable is a read, then the thread will read from memory when it may again
capture the value in the temporary view. When a thread executes a read flush,
no later memory operation by that thread for a variable involved in that flush
is allowed to start until the flush completes. The completion of a read flush
of a set of variables executed by a thread is defined as the point at which
the thread's temporary view of all variables involved is discarded.

Flush operations enable a program to provide a guarantee of consistency
between a thread's temporary view and memory. Therefore, the flush operation can
be used to guarantee that a value written to a variable by one thread may be
read by a second thread. To accomplish this, it is sufficient for the
programmer to ensure that the second thread has not written to the variable
since its last flush of the variable, and that the following sequence of
events are completed in the specified order according to the completion order
guarantees defined in \specref{subsec:OpenMP Memory Consistency}: 

\begin{enumerate}
\item The value is written to the variable by the first thread.   

\item The variable is flushed, with a write flush, by the first thread.    

\item The variable is flushed, with a read flush, by the second thread.

\item The value is read from the variable by the second thread.
\end{enumerate}

\begin{note}
OpenMP synchronization operations, described in 
\specref{sec:Master and Synchronization Constructs and Clauses} and in \specref{sec:Lock Routines}, 
are recommended for enforcing this order. Synchronization 
through variables, described in \specref{subsec:happens-before}, is possible
but is not recommended because the proper timing of flushes is difficult.
\end{note}



\subsection{Flush Synchronization and Happens Before}
\label{subsec:happens-before}
\index{flush synchronization}
\index{happens before}

A release flush is a write flush that is synchronizable, and likewise an
acquire flush is a read flush that is synchronizable.  A release flush may
synchronize with an acquire flush using a variable that is in the sync-sets of
both flushes.

For every variable in the sync-set of a release flush, there exists a
\emph{release flush sequence} that is a sequence of consecutive modifications
to the variable taken from its modification order. It is defined to be the
maximal such sequence that is headed by a modification that follows the
release flush in its thread's program order and contains only subsequent
modifications performed on the same thread or read-modify-write atomic
modifications performed by a different thread.  A release flush synchronizes
with an acquire flush on a different thread if there exists a variable in the
sync-sets of both flushes and the value written to it by some modification in
a release flush sequence associated with the release flush is read by an
access to the variable preceding the acquire flush in its thread's
program order.

\begin{note}
The conditions under which a release flush synchronizes with an acquire flush
are intended to match the conditions under which release operations
synchronize with acquire operations in C++11 and C11. A read-modify-write atomic
modification includes any atomic operation specified with an \code{atomic}
construct on which neither the \code{read} or \code{write} clause appears.
\end{note}

An operation \plc{X} \emph{happens~before} an operation \plc{Y} if any of the following conditions are satisfied:
\begin{enumerate}
\item \plc{X} happens before \plc{Y} according to the base language's definition of happens before, if such a definition exists.
\item \plc{X} and \plc{Y} are performed by the same thread, and \plc{X} precedes \plc{Y} in the thread's program order.
\item \plc{X} is a release flush, \plc{Y} is an acquire flush, and \plc{X} synchronizes with \plc{Y} according to the flush synchronization
conditions explained above.
\item There exists another operation \plc{Z}, such that \plc{X} happens before \plc{Z} and \plc{Z} happens before \plc{Y}.
\end{enumerate}

A variable with an initial value is treated as if the value is stored to the
variable by an operation that happens before all operations that access or
modify the variable in the program.

\subsection{OpenMP Memory Consistency}
\label{subsec:OpenMP Memory Consistency}
Given the conditions in \specref{subsec:happens-before} for flush
synchronization and happens before, OpenMP guarantees the following in
accordance with the restrictions in \specref{subsec:The Flush Operation} on
reordering with respect to flush operations:

\begin{itemize}
\item If the intersection of the flush-sets of two strong flushes
    performed by two different threads is non-empty, then the two flushes must
    be completed as if in some sequential order, seen by all threads.

\item If two operations performed by the same thread access, modify, or, with
    a strong flush, flush the same variable, then they must be
    completed as if in that thread's program order, as seen by all threads. 

\item If two operations access, modify, or flush the same variable and one
    happens before the other, then they must be completed in that happens
    before order, as seen by the thread or threads that perform the
    operations.

\item Any two atomic memory operations from different \code{atomic} regions
    must be completed as if in the same order as the strong flushes
    implied in their respective regions, as seen by all threads.
    \end{itemize}

The flush operation can be specified using the \code{flush} directive, and is also implied at 
various locations in an OpenMP program: see \specref{subsec:flush Construct} for details.

\begin{note}
Since flush operations by themselves cannot prevent data races, explicit flush 
operations are only useful in combination with non-sequentially consistent atomic 
directives.
\end{note}

OpenMP programs that:

\begin{itemize}[rightmargin=11ex]
\item do not use non-sequentially consistent atomic directives,

\item do not rely on the accuracy of a \plc{false} result from 
\code{omp\_test\_lock} and \code{omp\_test\_nest\_lock}, and

\item correctly avoid data races as required in \specref{subsec:Structure of the OpenMP Memory Model} 
\end{itemize}

behave as though operations on shared variables were simply interleaved in an order 
consistent with the order in which they are performed by each thread. The relaxed 
consistency model is invisible for such programs, and any explicit flush operations in 
such programs are redundant.

Implementations are allowed to relax the ordering imposed by implicit flush operations 
when the result is only visible to programs using non-sequentially consistent atomic 
directives.




\section{Tool Interface}
\label{subsec:Tool Support}

To enable development of high-quality, portable, \emph{first-party}
tools that support monitoring, performance or correctness
analysis of OpenMP programs developed using any implementation of 
the OpenMP API, the OpenMP API includes a tool interface 
known as OMPT.

The OMPT interface provides the following: 
\begin{itemize}
\item a mechanism to initialize a first-party tool,
\item routines that enable a tool to determine the capabilities of an
  OpenMP implementation,
\item routines that enable a tool to examine OpenMP state information associated with a thread,
\item mechanisms that enable a tool to map implementation-level calling
  contexts back to their source-level representations,
\item a callback interface that enables a tool to receive notification
  of OpenMP \emph{events},
\item a tracing interface that enables a tool to trace activity on OpenMP target devices, and
\item a runtime library routine that an application can use to control a tool.
\end{itemize}

OpenMP implementations may differ with respect to the \emph{thread states} that
they support, the mutual exclusion implementations they employ, 
and the OpenMP events for which tool callbacks are invoked. For some OpenMP events,
OpenMP implementations must guarantee that a registered callback will be invoked for each occurrence of the
event. For other OpenMP events, OpenMP implementations are permitted to invoke a registered callback for some
or no occurrences of the event; for such
OpenMP events, however,
OpenMP implementations are encouraged to invoke tool callbacks on as
many occurrences of the event as is practical to do so.
Section~\ref{sec:ompt-register-callbacks} specifies the subset of OMPT
callbacks that an OpenMP implementation must support for a minimal
implementation of the OMPT interface.

An implementation of the OpenMP API may differ from the
abstract execution model described by its specification.  The ability
of tools using the OMPT interface to observe such differences does not constrain
implementations of the OpenMP API in any way. 

With the exception of the \code{omp\_control\_tool} runtime library routine for tool control, 
all other routines in the OMPT interface are intended for use only by tools and
are not visible to applications.
For that reason, a Fortran binding is provided only 
for \code{omp\_control\_tool};
all other OMPT functionality is described with C syntax only.

\section{OpenMP Compliance}
\label{sec:OpenMP Compliance}
\index{OpenMP compliance}
\index{compliance}
% Note, though, that \specref{chap:ToolsSupport} contains both
% mandatory and optional features; the OpenMP API is compliant if at
% least the mandatory features are provided.
The OpenMP API defines constructs that operate in the context of the
base language that is supported by an implementation. If the
implementation of the base language does not support a language
construct that appears in this document, a compliant OpenMP
implementation is not required to support it, with the exception that
for Fortran, the implementation must allow case insensitivity for
directive and API routines names, and must allow identifiers of more
than six characters. An implementation of the OpenMP API is compliant
if and only if it compiles and executes all other conforming programs,
and supports the tool interface, according to the syntax and semantics
laid out in Chapters 1, 2, 3, 4 and 5. Appendices A, B, C, D, and E,
as well as sections designated as Notes (see \specref{sec:Organization
 of this document}) are for information purposes only and are not
part of the specification.

All library, intrinsic and built-in routines provided by the base language must be
thread-safe in a compliant implementation. In addition, the implementation of the base 
language must also be thread-safe. For example, \code{ALLOCATE} and \code{DEALLOCATE} 
statements must be thread-safe in Fortran. Unsynchronized concurrent use of such 
routines by different threads must produce correct results (although not necessarily the 
same as serial execution results, as in the case of random number generation routines).

Starting with Fortran 90, variables with explicit initialization have the \code{SAVE} attribute 
implicitly. This is not the case in Fortran 77. However, a compliant OpenMP Fortran 
implementation must give such a variable the \code{SAVE} attribute, regardless of the 
underlying base language version.

Appendix~\ref{chap:OpenMP Implementation-Defined Behaviors} 
lists certain aspects of the OpenMP API that are implementation defined. A 
compliant implementation is required to define and document its behavior for each of 
the items in Appendix~\ref{chap:OpenMP Implementation-Defined Behaviors}.






\section{Normative References}
\index{normative references}
\label{sec:normative references}
\begin{itemize}
\item ISO/IEC 9899:1990, \textsl{Information Technology - Programming Languages - C}.

This OpenMP API specification refers to ISO/IEC 9899:1990 as C90.

\item ISO/IEC 9899:1999, \textsl{Information Technology - Programming Languages - C}. 

This OpenMP API specification refers to ISO/IEC 9899:1999 as C99.

\item ISO/IEC 9899:2011, \textsl{Information Technology - Programming Languages - C}. 

This OpenMP API specification refers to ISO/IEC 9899:2011 as C11. The
following features are not supported:

\begin{itemize}
\item Supporting the noreturn property
\item Adding alignment support
\item Creation of complex value
\item Abandoning a process (adding \code{quick\_exit} and \code{at\_quick\_exit})
\item Threads for the C standard library
\item Thread-local storage
\item Parallel memory sequencing model
\item Atomic
\end{itemize}

\item ISO/IEC 14882:1998, \textsl{Information Technology - Programming Languages - C++}. 

This OpenMP API specification refers to ISO/IEC 14882:1998 as C++.

\item ISO/IEC 14882:2011, \textsl{Information Technology - Programming Languages - C++}. 

This OpenMP API specification refers to ISO/IEC 14882:2011 as
C++11. The following features are not supported:

\begin{itemize}
\item Rvalue references
\item Variadic templates
\item Extending variadic template template parameter
\item Lambda expression
\item Declared type of an expression
\item Incomplete return types
\item Default template arguments for function templates
\item Alias templates
\item Generalized constant expressions
\item Alignment support
\item Delegating constructors
\item New character types
\item Standard layout types
\item Defaulted functions
\item Local and unnamed types as template arguments
\item Range-based for
\item Explicit virtual overrides
\item Minimal support for garbage collection and reachability-based leak detection
\item Allowing move constructs to throw
\item Defining move special member functions
\item Concurrency
\item Data-dependency ordering: atomics and memory model
\item Additions to the standard library
\item Thread-local storage
\item Dynamic initialization and destruction with concurrency
\item C++11 library
\item Long long support
\item Extending integral types
\end{itemize}

\item ISO/IEC 14882:2014, \textsl{Information Technology - Programming Languages - C++}.

This OpenMP API specification refers to ISO/IEC 14882:2014 as
C++14. The following features are not supported:

\begin{itemize}
\item Initialized/Generalized lambda captures (init-capture)
\item Variable templates
\item Generic (polymorphic) lambda expressions
\item Sized deallocation
\item Null forward iterators
\item Add make\_unique
\item constexpr library additions: functional
\item What signal handlers can do
\item Prohibiting ``out of thin air''
\end{itemize}

\item ISO/IEC 1539:1980, \textsl{Information Technology - Programming Languages - Fortran}.

This OpenMP API specification refers to ISO/IEC 1539:1980 as Fortran 77.

\item ISO/IEC 1539:1991, \textsl{Information Technology - Programming Languages - Fortran}.

This OpenMP API specification refers to ISO/IEC 1539:1991 as Fortran 90.

\item ISO/IEC 1539-1:1997, \textsl{Information Technology - Programming Languages - Fortran}.

This OpenMP API specification refers to ISO/IEC 1539-1:1997 as Fortran 95.

\item ISO/IEC 1539-1:2004, \textsl{Information Technology - Programming Languages - Fortran}.


This OpenMP API specification refers to ISO/IEC 1539-1:2004 as Fortran 2003. The 
following features are not supported:

\begin{itemize}
\item IEEE Arithmetic issues covered in Fortran 2003 Section 14

\item Parameterized derived types

\item The \code{PASS} attribute

\item Procedures bound to a type as operators

\item Overriding a type-bound procedure

\item \code{SELECT}~\code{TYPE} construct

\item Deferred bindings and abstract types

\item Controlling IEEE underflow

\item Another IEEE class value 
\end{itemize}
\end{itemize}

Where this OpenMP API specification refers to C, C++ or Fortran, reference is made to 
the base language supported by the implementation.








\pagebreak
\section{Organization of this Document}
\label{sec:Organization of this document}
The remainder of this document is structured as follows: 

\begin{itemize}
\item Chapter \ref{chap:Directives} ``Directives''

\item Chapter \ref{chap:Runtime Library Routines} ``Runtime Library Routines''

\item Chapter \ref{chap:ToolsSupport} ``Tool Interface''

\item Chapter \ref{chap:Environment Variables} ``Environment Variables''

\item Appendix \ref{chap:Appendix A} ``Stubs for Runtime Library Routines''

\item Appendix \ref{chap:Interface Declarations} ``Interface Declarations'' 

\item Appendix \ref{chap:OpenMP Implementation-Defined Behaviors} ``OpenMP Implementation-Defined Behaviors''

\item Appendix \ref{chap:frames} ``Task Frame Management for the Tool Interface''

\item Appendix \ref{chap:Features History} ``Features History''
\end{itemize}

Some sections of this document only apply to programs written in a certain base 
language. Text that applies only to programs for which the base language is C or C++ is shown 
as follows: 

\begin{ccppspecific}
C/C++ specific text...
\end{ccppspecific}

Text that applies only to programs for which the base language is C only is shown as follows:

\begin{cspecific}
C specific text...
\end{cspecific}

Text that applies only to programs for which the base language is C90 only is shown as 
follows:

\begin{c90specific}
C90 specific text...
\end{c90specific}

Text that applies only to programs for which the base language is C99 only is shown as 
follows:

\begin{c99specific}
C99 specific text...
\end{c99specific}

Text that applies only to programs for which the base language is C++ only is shown as 
follows:

\begin{cppspecific}
C++ specific text...
\end{cppspecific}

Text that applies only to programs for which the base language is Fortran is shown as follows: 

\begin{fortranspecific}
Fortran specific text......
\end{fortranspecific}

Where an entire page consists of base language specific text, a marker is shown
at the top of the page.  For Fortran-specific text, the marker is:

\bigskip
\linewitharrows{-1}{dashed}{Fortran (cont.)}{10em}
\bigskip

For C/C++-specific text, the marker is:

\bigskip
\linewitharrows{-1}{dashed}{C/C++ (cont.)}{10em}
\bigskip

Some text is for information only, and is not part of the normative specification. Such 
text is designated as a note, like this: 

\needspace{6\baselineskip}\begin{note}
Non-normative text...
\end{note}


% This is the end of ch1-introduction.tex of the OpenMP specification.

