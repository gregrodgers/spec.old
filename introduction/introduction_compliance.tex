% This is an included file. See the master file for more information.
%
% When editing this file:
%
%    1. To change formatting, appearance, or style, please edit openmp.sty.
%
%    2. Custom commands and macros are defined in openmp.sty.
%
%    3. Be kind to other editors -- keep a consistent style by copying-and-pasting to
%       create new content.
%
%    4. We use semantic markup, e.g. (see openmp.sty for a full list):
%         \code{}     % for bold monospace keywords, code, operators, etc.
%         \plc{}      % for italic placeholder names, grammar, etc.
%
%    5. There are environments that provide special formatting, e.g. language bars.
%       Please use them whereever appropriate.  Examples are:
%
%         \begin{fortranspecific}
%         This is text that appears enclosed in blue language bars for Fortran.
%         \end{fortranspecific}
%
%         \begin{note}
%         This is a note.  The "Note -- " header appears automatically.
%         \end{note}
%
%    6. Other recommendations:
%         Use the convenience macros defined in openmp.sty for the minor headers
%         such as Comments, Syntax, etc.
%
%         To keep items together on the same page, prefer the use of
%         \begin{samepage}.... Avoid \parbox for text blocks as it interrupts line numbering.
%         When possible, avoid \filbreak, \pagebreak, \newpage, \clearpage unless that's
%         what you mean. Use \needspace{} cautiously for troublesome paragraphs.
%
%         Avoid absolute lengths and measures in this file; use relative units when possible.
%         Vertical space can be relative to \baselineskip or ex units. Horizontal space
%         can be relative to \linewidth or em units.
%
%         Prefer \emph{} to italicize terminology, e.g.:
%             This is a \emph{definition}, not a placeholder.
%             This is a \plc{var-name}.
%

\section{OpenMP Compliance}
\label{sec:OpenMP Compliance}
\index{OpenMP compliance}
\index{compliance}
The OpenMP API defines constructs that operate in the context of the
base language that is supported by an implementation. If the
implementation of the base language does not support a language
construct that appears in this document, a compliant OpenMP
implementation is not required to support it, with the exception that
for Fortran, the implementation must allow case insensitivity for
directive and API routines names, and must allow identifiers of more
than six characters. An implementation of the OpenMP API is compliant
if and only if it compiles and executes all other conforming programs,
and supports the tool interface, according to the syntax and semantics
laid out in Chapters 1, 2, 3, 4 and 5. Appendices A, B, C, and D, 
as well as sections designated as Notes (see \specref{sec:Organization
 of this document}) are for information purposes only and are not
part of the specification.

All library, intrinsic and built-in routines provided by the base language must be
thread-safe in a compliant implementation. In addition, the implementation of the base
language must also be thread-safe. For example, \code{ALLOCATE} and \code{DEALLOCATE}
statements must be thread-safe in Fortran. Unsynchronized concurrent use of such
routines by different threads must produce correct results (although not necessarily the
same as serial execution results, as in the case of random number generation routines).

Starting with Fortran 90, variables with explicit initialization have the \code{SAVE} attribute
implicitly. This is not the case in Fortran 77. However, a compliant OpenMP Fortran
implementation must give such a variable the \code{SAVE} attribute, regardless of the
underlying base language version.

Appendix~\ref{chap:OpenMP Implementation-Defined Behaviors}
lists certain aspects of the OpenMP API that are implementation defined. A
compliant implementation must define and document its behavior for each of
the items in Appendix~\ref{chap:OpenMP Implementation-Defined Behaviors}.

