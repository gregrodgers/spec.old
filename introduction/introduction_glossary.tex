% This is an included file. See the master file for more information.
%
% When editing this file:
%
%    1. To change formatting, appearance, or style, please edit openmp.sty.
%
%    2. Custom commands and macros are defined in openmp.sty.
%
%    3. Be kind to other editors -- keep a consistent style by copying-and-pasting to
%       create new content.
%
%    4. We use semantic markup, e.g. (see openmp.sty for a full list):
%         \code{}     % for bold monospace keywords, code, operators, etc.
%         \plc{}      % for italic placeholder names, grammar, etc.
%
%    5. There are environments that provide special formatting, e.g. language bars.
%       Please use them whereever appropriate.  Examples are:
%
%         \begin{fortranspecific}
%         This is text that appears enclosed in blue language bars for Fortran.
%         \end{fortranspecific}
%
%         \begin{note}
%         This is a note.  The "Note -- " header appears automatically.
%         \end{note}
%
%    6. Other recommendations:
%         Use the convenience macros defined in openmp.sty for the minor headers
%         such as Comments, Syntax, etc.
%
%         To keep items together on the same page, prefer the use of
%         \begin{samepage}.... Avoid \parbox for text blocks as it interrupts line numbering.
%         When possible, avoid \filbreak, \pagebreak, \newpage, \clearpage unless that's
%         what you mean. Use \needspace{} cautiously for troublesome paragraphs.
%
%         Avoid absolute lengths and measures in this file; use relative units when possible.
%         Vertical space can be relative to \baselineskip or ex units. Horizontal space
%         can be relative to \linewidth or em units.
%
%         Prefer \emph{} to italicize terminology, e.g.:
%             This is a \emph{definition}, not a placeholder.
%             This is a \plc{var-name}.
%

\section{Glossary}
\label{sec:Glossary}
\index{glossary}
\subsection{Threading Concepts}
\label{subsec:Threading Concepts}
\glossaryterm{thread}
\glossarydefstart
An execution entity with a stack and associated static memory, called
\emph{threadprivate memory}.
\glossarydefend

\glossaryterm{OpenMP thread}
\glossarydefstart
A \emph{thread} that is managed by the OpenMP implementation.
\glossarydefend

\glossaryterm{idle thread}
\glossarydefstart
An \emph{OpenMP thread} that is not currently part of any 
\code{parallel} region.
\glossarydefend

\glossaryterm{thread-safe routine}
\glossarydefstart
A routine that performs the intended function even when executed concurrently
(by more than one \emph{thread}).
\glossarydefend

\glossaryterm{processor}
\glossarydefstart
Implementation defined hardware unit on which one or more 
\emph{OpenMP threads} can execute.
\glossarydefend

\glossaryterm{device}
\glossarydefstart
An implementation defined logical execution engine.

\begin{quote}
COMMENT: A \emph{device} could have one or more \emph{processors}.
\end{quote}
\glossarydefend

\glossaryterm{host device}
\glossarydefstart
The \emph{device} on which the \emph{OpenMP program} begins execution.
\glossarydefend

\glossaryterm{target device}
\glossarydefstart
A device onto which code and data may be offloaded from the \emph{host device}.
\glossarydefend

\glossaryterm{parent device}
\glossarydefstart
For a given \code{target} region, the device on which the corresponding 
\code{target} construct was encountered.
\glossarydefend

% 
\subsection{OpenMP Language Terminology}
\label{subsec:OpenMP Language Terminology}
\glossaryterm{base language}
\glossarydefstart
A programming language that serves as the foundation of the OpenMP
specification.

\begin{quote}
COMMENT: See \specref{sec:normative references}
for a listing of current \emph{base languages} for the OpenMP API.
\end{quote}
\glossarydefend

\glossaryterm{base program}
\glossarydefstart
A program written in a \emph{base language}.
\glossarydefend

\glossaryterm{program order}
\glossarydefstart
An ordering of operations performed by the same thread as determined by the
execution sequence of operations specified by the \emph{base language}.

\begin{quote}
COMMENT: For C11 and C++11, \emph{program order} corresponds to the 
\emph{sequenced before} relation between operations performed by the 
same thread.
\end{quote}
\glossarydefend

\glossaryterm{structured block}
\glossarydefstart
For C/C++, an executable statement, possibly compound, with a single 
entry at the top and a single exit at the bottom, or an OpenMP 
\emph{construct}.

For Fortran, a block of executable statements with a single entry at 
the top and a single exit at the bottom, or an OpenMP \emph{construct}.

\begin{quote}
COMMENT:
See \specref{sec:Directive Format} for restrictions on \emph{structured blocks}.
\end{quote}
\glossarydefend

\glossaryterm{compilation unit}
\glossarydefstart
For C/C++, a translation unit.

For Fortran, a program unit.
\glossarydefend

\glossaryterm{enclosing context}
\glossarydefstart
For C/C++, the innermost scope enclosing an OpenMP \emph{directive}.

For Fortran, the innermost scoping unit enclosing an OpenMP \emph{directive}.
\glossarydefend

\glossaryterm{directive}
\glossarydefstart
For C/C++, a \pcode{\#pragma}, and for Fortran, a comment, that 
specifies \emph{OpenMP program} behavior.

\begin{quote}
COMMENT: See \specref{sec:Directive Format} for a description of OpenMP 
\emph{directive} syntax. 
\end{quote}
\glossarydefend

\glossaryterm{metadirective}
\glossarydefstart
A \emph{directive} that conditionally resolves to another \emph{directive} 
at compile time.
\glossarydefend


\glossaryterm{white space}
\glossarydefstart
A non-empty sequence of space and/or horizontal tab characters.
\glossarydefend

\glossaryterm{OpenMP program}
\glossarydefstart
A program that consists of a \emph{base program} that is annotated with OpenMP
\emph{directives} or that calls OpenMP API runtime library routines
\glossarydefend

\glossaryterm{conforming program}
\glossarydefstart
An \emph{OpenMP program} that follows all rules and restrictions of the OpenMP
specification.
\glossarydefend

\glossaryterm{declarative directive}
\glossarydefstart
An OpenMP \emph{directive} that may only be placed in a declarative context. A
\emph{declarative directive} results in one or more declarations only; it is 
not associated with the immediate execution of any user code.
\glossarydefend

\glossaryterm{executable directive}
\glossarydefstart
An OpenMP \emph{directive} that is not declarative. That is, it may be 
placed in an executable context.
\glossarydefend

\glossaryterm{stand-alone directive}
\glossarydefstart
An OpenMP \emph{executable directive} that has no associated user
code except for that which appears in clauses in the directive.
\glossarydefend


\glossaryterm{construct}
\glossarydefstart
An OpenMP \emph{executable directive} (and for Fortran, the paired 
\code{end} \emph{directive}, if any) and the associated statement, 
loop or \emph{structured block}, if any, not including the code in 
any called routines. That is, the lexical extent of an \emph{executable
directive}.
\glossarydefend

\glossaryterm{combined construct}
\glossarydefstart
A construct that is a shortcut for specifying one construct immediately 
nested inside another construct. A combined construct is semantically 
identical to that of explicitly specifying the first construct containing 
one instance of the second construct and no other statements.
\glossarydefend

\glossaryterm{composite construct}
\glossarydefstart
A construct that is composed of two constructs but does not have identical 
semantics to specifying one of the constructs immediately nested inside the 
other. A composite construct either adds semantics not included in the 
constructs from which it is composed or the nesting of the one construct 
inside the other is not conforming.
\glossarydefend

\glossaryterm{combined target construct}
\glossarydefstart
A \emph{combined construct} that is composed of a \code{target} construct 
along with another construct.
\glossarydefend


\glossaryterm{region}
\glossarydefstart
All code encountered during a specific instance of the execution of a given
\emph{construct} or of an OpenMP library routine. A \emph{region} includes 
any code in called routines as well as any implicit code introduced by the 
OpenMP implementation. The generation of a \emph{task} at the point where 
a \emph{task generating construct} is encountered is a part of the 
\emph{region} of the \emph{encountering thread}.  However, an 
\emph{explicit task region} corresponding to a 
\emph{task generating construct} is not part of the \emph{region} of the 
\emph{encountering thread} unless it is an \emph{included task region}. 
The point where a \code{target} or \code{teams} directive is encountered 
is a part of the \emph{region} of the \emph{encountering thread}, but the
\emph{region} corresponding to the \code{target} or \code{teams} directive 
is not.

\begin{quote}
COMMENTS:

A \emph{region} may also be thought of as the dynamic or runtime extent of a
\emph{construct} or of an OpenMP library routine.

During the execution of an \emph{OpenMP program}, a \emph{construct} may give
rise to many \emph{regions}.
\end{quote}
\glossarydefend

\glossaryterm{active parallel region}
\glossarydefstart
A \code{parallel} \emph{region} that is executed by a \emph{team} consisting 
of more than one
\emph{thread}.
\glossarydefend

\smallskip
\glossaryterm{inactive parallel region}
\glossarydefstart
A \code{parallel} \emph{region} that is executed by a \emph{team} of only 
one \emph{thread}.
\glossarydefend

\glossaryterm{active target region}
\glossarydefstart
A \code{target} \emph{region} that is executed on a \emph{device} other 
than the \emph{device} that encountered the \code{target} \emph{construct}. 
\glossarydefend

\smallskip
\glossaryterm{inactive target region}
\glossarydefstart
A \code{target} \emph{region} that is executed on the same \emph{device} 
that encountered the \code{target} \emph{construct}.
\glossarydefend

\glossaryterm{sequential part}
\glossarydefstart
All code encountered during the execution of an \emph{initial task region} 
that is not part of a \code{parallel} \emph{region} corresponding to a 
\code{parallel} \emph{construct} or a \code{task}
\emph{region} corresponding to a \code{task} \emph{construct}.

\begin{quote}
COMMENTS:

A \emph{sequential part} is enclosed by an \emph{implicit parallel region}.

Executable statements in called routines may be in both a \emph{sequential
part} and any number of explicit \code{parallel} \emph{regions} at different 
points in the program execution.
\end{quote}
\glossarydefend

\glossaryterm{master thread}
\glossarydefstart
An \emph{OpenMP thread} that has  \emph{thread} number 0. A \emph{master
thread} may be an \emph{initial thread} or the \emph{thread} that encounters a
\code{parallel} \emph{construct}, creates a \emph{team}, generates a set of
\emph{implicit tasks}, and then executes one of those \emph{tasks} as
\emph{thread} number 0.
\glossarydefend

\glossaryterm{parent thread}
\glossarydefstart
The \emph{thread} that encountered the \code{parallel} \emph{construct} 
and generated a \code{parallel} \emph{region} is the \emph{parent thread} 
of each of the \emph{threads} in the \emph{team} of that
\code{parallel} \emph{region}. The \emph{master thread}
of a \code{parallel} \emph{region} is the same \emph{thread}
as its \emph{parent thread} with respect to any resources associated with 
an \emph{OpenMP thread}.
\glossarydefend

\glossaryterm{child thread}
\glossarydefstart
When a thread encounters a \code{parallel} construct, each of the threads 
in the generated \code{parallel} region's team are \emph{child threads} of 
the encountering \emph{thread}. The \code{target} or \code{teams} region's 
\emph{initial thread} is not a \emph{child thread} of the thread
that encountered the \code{target} or \code{teams} construct.
\glossarydefend

\glossaryterm{ancestor thread}
\glossarydefstart
For a given \emph{thread}, its \emph{parent thread} or one of its 
\emph{parent thread's ancestor threads}.
\glossarydefend

\glossaryterm{descendent thread}
\glossarydefstart
For a given \emph{thread}, one of its \emph{child threads} or one of
its \emph{child threads' descendent threads}.
\glossarydefend

\glossaryterm{team}
\glossarydefstart
A set of one or more \emph{threads} participating in the execution 
of a \code{parallel}
\emph{region}.

\begin{quote}
COMMENTS:

For an \emph{active parallel region}, the team comprises the 
\emph{master thread} and at least one additional \emph{thread}.

For an \emph{inactive parallel region}, the \emph{team} comprises 
only the \emph{master thread}.
\end{quote}
\glossarydefend

\glossaryterm{league}
\glossarydefstart
The set of \emph{teams} created by a \code{teams} construct.
\glossarydefend

\glossaryterm{contention group}
\glossarydefstart
An initial \emph{thread} and its \emph{descendent threads}.
\glossarydefend

\glossaryterm{implicit parallel region}
\glossarydefstart
An \emph{inactive parallel region} that is not generated from a
\code{parallel} \emph{construct}. \emph{Implicit parallel regions} 
surround the whole \emph{OpenMP program}, all \code{target} 
\emph{regions}, and all \code{teams} \emph{regions}.

\glossarydefend

\glossaryterm{initial thread}
\glossarydefstart
The \emph{thread} that executes an \emph{implicit parallel region}.
\glossarydefend

\glossaryterm{initial team}
\glossarydefstart
The \emph{team} that comprises an \emph{initial thread} executing 
an \emph{implicit parallel region}.
\glossarydefend

\glossaryterm{nested construct}
\glossarydefstart
A \emph{construct} (lexically) enclosed by another \emph{construct}.
\glossarydefend

\glossaryterm{closely nested construct}
\glossarydefstart
A \emph{construct} nested inside another \emph{construct} with no 
other \emph{construct} nested between them.
\glossarydefend

\glossaryterm{nested region}
\glossarydefstart
A \emph{region} (dynamically) enclosed by another \emph{region}.  That is, a
\emph{region} generated from the execution of another \emph{region}
or one of its \emph{nested regions}.

\begin{quote}
COMMENT: Some nestings are \emph{conforming} and some are not.
See \specref{sec:Nesting of Regions} for the restrictions on nesting.
\end{quote}
\glossarydefend

\glossaryterm{closely nested region}
\glossarydefstart
A \emph{region nested} inside another \emph{region} with no \code{parallel} 
\emph{region nested} between them.
\glossarydefend

\glossaryterm{strictly nested region}
\glossarydefstart
A \emph{region nested} inside another \emph{region} with no other 
\emph{region nested} between them.
\glossarydefend

\glossaryterm{all threads}
\glossarydefstart
All OpenMP \emph{threads} participating in the \emph{OpenMP program}.
\glossarydefend

\glossaryterm{current team}
\glossarydefstart
All \emph{threads} in the \emph{team} executing the innermost 
enclosing \code{parallel} \emph{region}.
\glossarydefend

\glossaryterm{encountering thread}
\glossarydefstart
For a given \emph{region}, the \emph{thread} that encounters the
corresponding \emph{construct}.
\glossarydefend

\glossaryterm{all tasks}
\glossarydefstart
All \emph{tasks} participating in the \emph{OpenMP program}.
\glossarydefend

\glossaryterm{current team tasks}
\glossarydefstart
All \emph{tasks} encountered by the corresponding \emph{team}. The 
\emph{implicit tasks} constituting the \code{parallel} \emph{region} 
and any \emph{descendent tasks} encountered during the execution of 
these \emph{implicit tasks} are included in this set of tasks.
\glossarydefend

\glossaryterm{generating task}
\glossarydefstart
For a given \emph{region}, the task for which execution by a 
\emph{thread} generated the \emph{region}.
\glossarydefend

\glossaryterm{binding thread set}
\glossarydefstart
The set of \emph{threads} that are affected by, or provide the context 
for, the execution of a \emph{region}.

The \emph{binding thread} set for a given \emph{region} can be 
\emph{all threads} on a \emph{device}, \emph{all threads} in a 
\emph{contention group}, all \emph{master threads} executing an
enclosing \code{teams} \emph{region}, the \emph{current team}, 
or the \emph{encountering thread}.

\begin{quote}
COMMENT: The \emph{binding thread set} for a particular \emph{region} 
is described in its corresponding subsection of this specification.
\end{quote}
\glossarydefend

\glossaryterm{binding task set}
\glossarydefstart
The set of \emph{tasks} that are affected by, or provide the context for, 
the execution of a \emph{region}.

The \emph{binding task} set for a given \emph{region} can be \emph{all tasks},
the \emph{current team tasks}, the \emph{binding implicit task}, or the 
\emph{generating task}.

\begin{quote}
COMMENT: The \emph{binding task} set for a particular \emph{region} (if 
applicable) is described in its corresponding subsection of this specification.
\end{quote}
\glossarydefend

\glossaryterm{binding region}
\glossarydefstart
The enclosing \emph{region} that determines the execution context and limits 
the scope of the effects of the bound \emph{region} is called the 
\emph{binding region}.

\emph{Binding region} is not defined for \emph{regions} for which the 
\emph{binding thread} set is \emph{all threads} or the 
\emph{encountering thread}, nor is it defined for \emph{regions} for 
which the \emph{binding task set} is \emph{all tasks}.

\begin{quote}
COMMENTS:

The \emph{binding region} for an \code{ordered} \emph{region} is the 
innermost enclosing \emph{loop region}.

The \emph{binding region} for a \code{taskwait} \emph{region} is the 
innermost enclosing \emph{task region}.

The \emph{binding region} for a \code{cancel} \emph{region} is the 
innermost enclosing \emph{region} corresponding to the 
\plc{construct-type-clause} of the \code{cancel} construct.

The \emph{binding region} for a \code{cancellation point} \emph{region} 
is the innermost enclosing \emph{region} corresponding to the 
\plc{construct-type-clause} of the \code{cancellation point} construct.

For all other \emph{regions} for which the \emph{binding thread set} is 
the \emph{current team} or the \emph{binding task set} is the 
\emph{current team tasks}, the \emph{binding region} is the innermost 
enclosing \code{parallel} \emph{region}.

For \emph{regions} for which the \emph{binding task set} is the 
\emph{generating task}, the \emph{binding region} is the \emph{region} 
of the \emph{generating task}.

A \code{parallel} \emph{region} need not be \emph{active} nor explicit 
to be a \emph{binding region}.

A \emph{task region} need not be explicit to be a \emph{binding region}.

A \emph{region} never binds to any \emph{region} outside of the innermost 
enclosing \code{parallel} \emph{region}.
\end{quote}
\glossarydefend

\glossaryterm{orphaned construct}
\glossarydefstart
A \emph{construct} that gives rise to a \emph{region} for which the 
\emph{binding thread set} is the \emph{current team}, but is not nested 
within another \emph{construct} giving rise to the \emph{binding region}.
\glossarydefend

\glossaryterm{worksharing construct}
\glossarydefstart
A \emph{construct} that defines units of work, each of which is executed 
exactly once by one of the \emph{threads} in the \emph{team} executing 
the \emph{construct}.

For C/C++, \emph{worksharing constructs} are \code{for}, \code{sections}, 
and \code{single}.

For Fortran, \emph{worksharing constructs} are \code{do}, \code{sections}, 
\code{single} and \code{workshare}.
\glossarydefend

\glossaryterm{device construct}
\glossarydefstart
An OpenMP \emph{construct} that accepts the \code{device} clause.
\glossarydefend

\glossaryterm{device routine}
\glossarydefstart
A function (for C/C+ and Fortran) or subroutine (for Fortran) that can be
executed on a \emph{target device}, as part of a \code{target} region.
\glossarydefend

\glossaryterm{place}
\glossarydefstart
Unordered set of \emph{processors} on a device that is treated by the 
execution environment as a location unit when dealing with OpenMP thread 
affinity.
\glossarydefend

\glossaryterm{place list}
\glossarydefstart
The ordered list that describes all OpenMP \emph{places} available to 
the execution environment.
\glossarydefend

\glossaryterm{place partition}
\glossarydefstart
An ordered list that corresponds to a contiguous interval in the OpenMP 
\emph{place list}. It describes the \emph{places} currently available to 
the execution environment for a given parallel \emph{region}.
\glossarydefend

\glossaryterm{place number}
\glossarydefstart
A number that uniquely identifies a \emph{place} in the \emph{place list}, 
with zero identifying the first \emph{place} in the \emph{place list}, and 
each consecutive whole number identifying the next \emph{place} in the 
\emph{place list}.
\glossarydefend

\glossaryterm{thread affinity}
\glossarydefstart
A mapping of \emph{threads} to hardware execution units that are defined
as \emph{places} within the current \emph{place partition}.
\glossarydefend

\glossaryterm{SIMD instruction}
\glossarydefstart
A single machine instruction that can operate on multiple data elements.
\glossarydefend

\glossaryterm{SIMD lane}
\glossarydefstart
A software or hardware mechanism capable of processing one data element from a
\emph{SIMD instruction}.
\glossarydefend

\glossaryterm{SIMD chunk}
\glossarydefstart
A set of iterations executed concurrently, each by a \emph{SIMD lane}, by 
a single \emph{thread} by means of \emph{SIMD instructions}.
\glossarydefend

\glossaryterm{memory}
\glossarydefstart
A storage resource to store and to retrieve variables accessible by 
OpenMP threads.
\glossarydefend

\glossaryterm{memory space}
\glossarydefstart
A representation of storage resources from which \emph{memory} can 
be allocated or deallocated.  More than one memory space may exist.
\glossarydefend

\glossaryterm{memory allocator}
\glossarydefstart
An OpenMP object that fulfills requests to allocate and to deallocate 
\emph{memory} for program variables from the storage resources of its 
associated \emph{memory space}.
\glossarydefend

%
% Loop Terminology
%
% 
\subsection{Loop Terminology}
\index{loop terminology}
\label{subsec:Loop Terminology}
\glossaryterm{loop-associated directive}
\glossarydefstart
An OpenMP \emph{executable} directive for which the associated user 
code must be a loop nest that is a \emph{structured block}.
\glossarydefend

\glossaryterm{associated loop(s)}
\glossarydefstart
The loop(s) controlled by a \emph{loop-associated directive}.
\begin{quote}
COMMENT: If the \emph{loop-associated directive} contains a \code{collapse} or 
an \code{ordered(}\plc{n}\code{)} clause then it may have more than 
one \emph{associated loop}.
\end{quote}
\glossarydefend

\glossaryterm{sequential loop}
\glossarydefstart
A loop that is not associated with any OpenMP \emph{loop-associated directive}.
\glossarydefend

\glossaryterm{SIMD loop}
\glossarydefstart
A loop that includes at least one \emph{SIMD chunk}.
\glossarydefend

\glossaryterm{non-rectangular loop nest}
\glossarydefstart
A loop nest for which the iteration count of a loop inside the loop nest is the
not same for all occurrences of the loop in the loop nest.
\glossarydefend

\glossaryterm{doacross loop nest}
\glossarydefstart
A loop nest that has cross-iteration dependence. An iteration is dependent 
on one or more lexicographically earlier iterations.
\begin{quote}
COMMENT: The \code{ordered} clause parameter on a worksharing-loop directive identifies 
the loop(s) associated with the \emph{doacross loop nest}.
\end{quote}
\glossarydefend

%
% Synchronization Terminology
%
\subsection{Synchronization Terminology}
\index{synchronization terminology}
\label{subsec:Synchronization Terminology}
\glossaryterm{barrier}
\glossarydefstart
A point in the execution of a program encountered by a \emph{team} 
of \emph{threads}, beyond which no \emph{thread} in the team may 
execute until all \emph{threads} in the \emph{team} have reached 
the barrier and all \emph{explicit tasks} generated by the \emph{team} 
have executed to completion. If \emph{cancellation} has been requested, 
threads may proceed to the end of the canceled \emph{region} even if 
some threads in the team have not reached the \emph{barrier}.
\glossarydefend

\glossaryterm{cancellation}
\glossarydefstart
An action that cancels (that is, aborts) an OpenMP \emph{region} and 
causes executing \emph{implicit} or \emph{explicit} tasks to proceed 
to the end of the canceled \emph{region}.
\glossarydefend

\glossaryterm{cancellation point}
\glossarydefstart
A point at which implicit and explicit tasks check if cancellation has been
requested. If cancellation has been observed, they perform the \emph{cancellation}.

\begin{quote}
COMMENT: For a list of cancellation points, see \specref{subsec:cancel Construct}.
\end{quote}
\glossarydefend
\bigskip

\glossaryterm{flush}
\glossarydefstart
An operation that a \emph{thread} performs to enforce consistency between its
view and other \emph{threads}' view of memory.
\glossarydefend

\glossaryterm{flush property}
\glossarydefstart
Properties that determine the manner in which a \emph{flush} operation enforces
memory consistency. These properties are:
\begin{itemize}
    \item \emph{strong}: flushes a set of variables from the current thread's
        temporary view of the memory to the memory;
    \item \emph{release}: orders memory operations that precede the flush
        before memory operations performed by a different thread with which it
        synchronizes;
    \item \emph{acquire}: orders memory operations that follow the flush after
        memory operations performed by a different thread that synchronizes
        with it.
\end{itemize}

\begin{quote}
COMMENT: Any \emph{flush} operation has one or more \emph{flush properties}.
\end{quote}
\glossarydefend

\glossaryterm{strong flush}
\glossarydefstart
A \emph{flush} operation that has the \emph{strong flush property}.
\glossarydefend

\glossaryterm{release flush}
\glossarydefstart
A \emph{flush} operation that has the \emph{release flush property}.
\glossarydefend

\glossaryterm{acquire flush}
\glossarydefstart
A \emph{flush} operation that has the \emph{acquire flush property}.
\glossarydefend

\glossaryterm{atomic operation}
\glossarydefstart
An operation that is specified by an \code{atomic} construct and atomically
accesses and/or modifies a specific storage location.
\glossarydefend

\glossaryterm{atomic read}
\glossarydefstart
An \emph{atomic operation} that is specified by an \code{atomic} construct 
on which the \code{read} clause is present.
\glossarydefend

\glossaryterm{atomic write}
\glossarydefstart
An \emph{atomic operation} that is specified by an \code{atomic} construct 
on which the \code{write} clause is present.
\glossarydefend

\glossaryterm{atomic update}
\glossarydefstart
An \emph{atomic operation} that is specified by an \code{atomic} construct 
on which the \code{update} clause is present.
\glossarydefend

\glossaryterm{atomic captured update}
\glossarydefstart
An \emph{atomic operation} that is specified by an \code{atomic} construct 
on which the \code{capture} clause is present.
\glossarydefend

\glossaryterm{read-modify-write}
\glossarydefstart
An \emph{atomic operation} that reads and writes to a given storage location.

\begin{quote}
COMMENT: All \emph{atomic update} and \emph{atomic captured update} operations
are \emph{read-modify-write} operations.
\end{quote}
\glossarydefend

\glossaryterm{sequentially consistent atomic construct}
\glossarydefstart
An \code{atomic} construct for which the \code{seq_cst} clause is specified.
\glossarydefend
\bigskip

\glossaryterm{non-sequentially consistent atomic construct}
\glossarydefstart
An \code{atomic} construct for which the \code{seq_cst} clause is not specified
\glossarydefend
\bigskip
\bigskip

\glossaryterm{sequentially consistent atomic operation}
\glossarydefstart
An \emph{atomic operation} that is specified by a 
\emph{sequentially consistent atomic construct}. 
\glossarydefend
\bigskip
\bigskip


% 
\subsection{Tasking Terminology}
\index{tasking terminology}
\label{subsec:Tasking Terminology}
\glossaryterm{task}
\glossarydefstart
A specific instance of executable code and its data environment that the
OpenMP implementation can schedule for execution by threads.
\glossarydefend

\glossaryterm{task region}
\glossarydefstart
A \emph{region} consisting of all code encountered during the execution 
of a \emph{task}.

\begin{quote}
COMMENT: A \code{parallel} \emph{region} consists of one or more implicit 
\emph{task regions}.
\end{quote}
\glossarydefend

\glossaryterm{implicit task}
\glossarydefstart
A \emph{task} generated by an \emph{implicit parallel region} or generated 
when a \code{parallel} \emph{construct} is encountered during execution.
\glossarydefend

\glossaryterm{binding implicit task}
\glossarydefstart
The \emph{implicit task} of the current thread team assigned to the 
encountering thread.
\glossarydefend

\glossaryterm{explicit task}
\glossarydefstart
A \emph{task} that is not an \emph{implicit task}.
\glossarydefend

\glossaryterm{initial task}
\glossarydefstart
An \emph{implicit task} associated with an \emph{implicit parallel region}.
\glossarydefend

\glossaryterm{current task}
\glossarydefstart
For a given \emph{thread}, the \emph{task} corresponding to the 
\emph{task region} in which it is executing.
\glossarydefend

\glossaryterm{child task}
\glossarydefstart
A \emph{task} is a \emph{child task} of its generating \emph{task region}.
A \emph{child task region} is not part of its generating \emph{task region}.
\glossarydefend

\glossaryterm{sibling tasks}
\glossarydefstart
\emph{Tasks} that are \emph{child tasks} of the same \emph{task region}.
\glossarydefend

\glossaryterm{descendent task}
\glossarydefstart
A \emph{task} that is the \emph{child task} of a \emph{task region} or 
of one of its \emph{descendent task regions}.
\glossarydefend

\glossaryterm{task completion}
\glossarydefstart
\emph{Task completion} occurs when the end of the \emph{structured block} 
associated with the \emph{construct} that generated the \emph{task} is reached.

\begin{quote}
COMMENT: Completion of the \emph{initial task} that is generated when the 
program begins occurs at program exit.
\end{quote}
\glossarydefend

\glossaryterm{task scheduling point}
\glossarydefstart
A point during the execution of the current \emph{task region} at which it 
can be suspended to be resumed later; or the point of \emph{task completion}, 
after which the executing thread may switch to a different \emph{task region}.

\begin{quote}
COMMENT: For a list of \emph{task scheduling points}, see 
\specref{subsec:Task Scheduling}.
\end{quote}
\glossarydefend

\glossaryterm{task switching}
\glossarydefstart
The act of a \emph{thread} switching from the execution of one \emph{task} 
to another \emph{task}.
\glossarydefend

\glossaryterm{tied task}
\glossarydefstart
A \emph{task} that, when its \emph{task region} is suspended, can be resumed 
only by the same \emph{thread} that suspended it. That is, the \emph{task} 
is tied to that \emph{thread}.
\glossarydefend

\glossaryterm{untied task}
\glossarydefstart
A \emph{task} that, when its \emph{task region} is suspended, can be resumed 
by any \emph{thread} in the team. That is, the \emph{task} is not tied to 
any \emph{thread}.
\glossarydefend

\glossaryterm{undeferred task}
\glossarydefstart
A \emph{task} for which execution is not deferred with respect to its 
generating \emph{task} \emph{region}. That is, its generating 
\emph{task region} is suspended until execution of the
\emph{undeferred task} is completed.
\glossarydefend

\glossaryterm{included task}
\glossarydefstart
A \emph{task} for which execution is sequentially included in the 
generating \emph{task region}. That is, an \emph{included task} is 
\emph{undeferred} and executed immediately by the \emph{encountering thread}.
\glossarydefend

\glossaryterm{merged task}
\glossarydefstart
A \emph{task} for which the \emph{data environment}, inclusive of ICVs, 
is the same as that of its generating \emph{task region}.
\glossarydefend

\glossaryterm{mergeable task}
\glossarydefstart
A \emph{task} that may be a \emph{merged task} if it is an 
\emph{undeferred task} or an \emph{included task}.
\glossarydefend

\glossaryterm{final task}
\glossarydefstart
A \emph{task} that forces all of its \emph{child tasks} to become 
\emph{final} and \emph{included tasks}.
\glossarydefend

\glossaryterm{task dependence}
\glossarydefstart
An ordering relation between two \emph{sibling tasks}: the 
\emph{dependent task} and a previously generated \emph{predecessor task}. 
The \emph{task dependence} is fulfilled when the
\emph{predecessor task} has completed.
\glossarydefend

\begin{samepage}
\glossaryterm{dependent task}
\glossarydefstart
A \emph{task} that because of a \emph{task dependence} cannot be executed 
until its \emph{predecessor tasks} have completed.
\glossarydefend
\end{samepage}

\glossaryterm{mutually exclusive tasks}
\glossarydefstart
\emph{Tasks} that may be executed in any order, but not at the same time.
\glossarydefend
\bigskip

\glossaryterm{predecessor task}
\glossarydefstart
A \emph{task} that must complete before its \emph{dependent tasks} can be 
executed.
\glossarydefend

\glossaryterm{task synchronization construct}
\glossarydefstart
A \code{taskwait}, \code{taskgroup}, or a \code{barrier} \emph{construct}.
\glossarydefend
\bigskip

\glossaryterm{task generating construct}
\glossarydefstart
A \emph{construct} that generates one or more \emph{explicit tasks}.
\glossarydefend
\bigskip

\glossaryterm{target task}
\glossarydefstart
A \emph{mergeable} and \emph{untied} \emph{task} that is generated by 
a \code{target}, \code{target enter data}, \code{target exit data}, or 
\code{target update} \emph{construct}.
\glossarydefend

\glossaryterm{taskgroup set}
\glossarydefstart
A set of tasks that are logically grouped by a \code{taskgroup} \emph{region}.
\glossarydefend


\subsection{Data Terminology}
\index{data terminology}
\label{subsec:Data Terminology}
\glossaryterm{variable}
\glossarydefstart
A named data storage block, for which the value can be defined and 
redefined during the execution of a program.

\begin{adjustwidth}{-0.75in}{0in}
\begin{note}
An array element or structure element is a variable that is part of 
another variable.
\end{note}
\end{adjustwidth}
\glossarydefend

\glossaryterm{scalar variable}
\glossarydefstart
For C/C++, a scalar variable, as defined by the base language.

For Fortran, a scalar variable with intrinsic type, as defined by 
the base language, excluding character type.
\glossarydefend

\glossaryterm{aggregate variable}
\glossarydefstart
A variable, such as an array or structure, composed of other variables.
\glossarydefend

\glossaryterm{array section}
\glossarydefstart
A designated subset of the elements of an array that is specified using a
subscript notation that can select more than one element.
\glossarydefend

\glossaryterm{array item}
\glossarydefstart
An array, an array section, or an array element.
\glossarydefend

\glossaryterm{shape-operator}
\glossarydefstart
For C/C++, an array shaping operator that reinterprets a pointer expression as
an array with one or more specified dimensions.
\glossarydefend

\glossaryterm{implicit array}
\glossarydefstart
For C/C++, the set of array elements of non-array type \emph{T} that may be
accessed by applying a sequence of [] operators to a given pointer that is
either a pointer to type \emph{T} or a pointer to a multidimensional array of
elements of type \emph{T}.

For Fortran, the set of array elements for a given array pointer.
\begin{quote}
COMMENT:

For C/C++, the implicit array for pointer p with type \emph{T} (*)[10] 
consists of all accessible elements p[\emph{i}][\emph{j}], for all 
\emph{i} and \emph{j}=0..9.
\end{quote}
\glossarydefend

\glossaryterm{base pointer}
\glossarydefstart
For C/C++, an lvalue pointer expression that is used by a given lvalue
expression or array section to refer indirectly to its storage, where
the lvalue expression or array section is part of the implicit array for that
lvalue pointer expression.

For Fortran, a data pointer that appears last in the designator for a given
variable or array section, where the variable or array section is part
of the pointer target for that data pointer.

\begin{quote}
COMMENT: 

For the array section
(*p0).x0[k1].p1->p2[k2].x1[k3].x2[4][0:n],
where identifiers p\emph{i} have a pointer type declaration
and identifiers x\emph{i} have an array type declaration, the 
\emph{base pointer} is: (*p0).x0[k1].p1->p2.
\end{quote}
\glossarydefend

\glossaryterm{named pointer}
\glossarydefstart
For C/C++, the \emph{base pointer} of a given lvalue expression or 
array section, or the \emph{base pointer} of one of its \emph{named pointers}.

For Fortran, the \emph{base pointer} of a given variable or array section, or
the \emph{base pointer} of one of its \emph{named pointers}.

\begin{quote}
COMMENT: 

For the array section
(*p0).x0[k1].p1->p2[k2].x1[k3].x2[4][0:n],
where identifiers p\emph{i} have a pointer type declaration
and identifiers x\emph{i} have an array type declaration, the 
\emph{named pointers} are: 
p0, 
(*p0).x0[k1].p1,
and 
(*p0).x0[k1].p1->p2.
\end{quote}
\glossarydefend

\glossaryterm{containing array}
\glossarydefstart
For C/C++, a non-subscripted array (a \emph{containing array}) that appears in
a given lvalue expression or array section, where the lvalue expression or
array section is part of that \emph{containing array}.

For Fortran, an array (a \emph{containing array}) without the \code{POINTER}
attribute and without a subscript list that appears in the designator of a
given variable or array section, where the variable or array section is part
of that \emph{containing array}.

\begin{quote}
COMMENT: 

For the array section
(*p0).x0[k1].p1->p2[k2].x1[k3].x2[4][0:n],
where identifiers p\emph{i} have a pointer type declaration
and identifiers x\emph{i} have an array type declaration, 
the \emph{containing arrays} are:
    (*p0).x0[k1].p1->p2[k2].x1
and 
(*p0).x0[k1].p1->p2[k2].x1[k3].x2.
\end{quote}
\glossarydefend

\glossaryterm{base array}
\glossarydefstart
For C/C++, a \emph{containing array} of a given lvalue expression or 
array section that does not appear in the expression of any of its 
other \emph{containing arrays}.

For Fortran, a \emph{containing array} of a given variable or array section
that does not appear in the designator of any of its other 
\emph{containing arrays}.

\begin{quote}
COMMENT: 

For the array section
(*p0).x0[k1].p1->p2[k2].x1[k3].x2[4][0:n],
where identifiers p\emph{i} have a pointer type declaration
and identifiers x\emph{i} have an array type declaration, 
the \emph{base array} is: (*p0).x0[k1].p1->p2[k2].x1[k3].x2.
\end{quote}
\glossarydefend

\glossaryterm{named array}
\glossarydefstart
For C/C++, a \emph{containing array} of a given lvalue expression or array 
section, or a \emph{containing array} of one of its \emph{named pointers}.

For Fortran, a \emph{containing array} of a given variable or array section, or
a \emph{containing array} of one of its \emph{named pointers}.

\begin{quote}
COMMENT: 

For the array section
(*p0).x0[k1].p1->p2[k2].x1[k3].x2[4][0:n],
where identifiers p\emph{i} have a pointer type declaration
and identifiers x\emph{i} have an array type declaration, 
the \emph{named arrays} are:
(*p0).x0,
    (*p0).x0[k1].p1->p2[k2].x1,
and 
(*p0).x0[k1].p1->p2[k2].x1[k3].x2.
\end{quote}

\glossarydefend

\glossaryterm{base expression}
\glossarydefstart
The \emph{base array} of a given array section or array element, if
it exists; otherwise, the \emph{base pointer} of the array section or array
element.

\begin{quote}
COMMENT: 

For the array section
(*p0).x0[k1].p1->p2[k2].x1[k3].x2[4][0:n],
where identifiers p\emph{i} have a pointer type declaration and
identifiers x\emph{i} have an array type declaration, the
\emph{base expression} is:
(*p0).x0[k1].p1->p2[k2].x1[k3].x2.

More examples for C/C++:
\begin{itemize}
\item The \emph{base expression} for x[i] and for 
    x[i:n] is x, if x is an array or pointer.
\item The \emph{base expression} for x[5][i] and for x[5][i:n] is x, if x is a
    pointer to an array or x is 2-dimensional array.
\item The \emph{base expression} for y[5][i] and for y[5][i:n] is y[5], if
    y is an array of pointers or y is a pointer to a pointer.
\end{itemize}
Examples for Fortran:
\begin{itemize}
\item The \emph{base expression} for x(i) and for x(i:j) is x.
\end{itemize}
\end{quote}
\glossarydefend


\glossaryterm{attached pointer}
\glossarydefstart
A pointer variable in a device data environment to which the effect of a
\code{map} clause assigns the address of an object, minus some offset,  that
is created in the device data environment. The pointer is an attached pointer
for the remainder of its lifetime in the device data environment.
\glossarydefend
\bigskip

\glossaryterm{simply contiguous array section}
\glossarydefstart
An array section that statically can be determined to have contiguous 
storage or that, in Fortran, has the \code{CONTIGUOUS} attribute.
\glossarydefend
\bigskip

\glossaryterm{structure}
\glossarydefstart
A structure is a variable that contains one or more variables.

For C/C++:
\nopagebreak
Implemented using struct types.

For C++:
\nopagebreak
Implemented using class types.

For Fortran:
\nopagebreak
Implemented using derived types.
\glossarydefend

\glossaryterm{private variable}
\glossarydefstart
With respect to a given set of \emph{task regions} or \emph{SIMD lanes} 
that bind to the same \code{parallel} \emph{region}, a \emph{variable} 
for which the name provides access to a different block of
storage for each \emph{task region} or \emph{SIMD lane}.

A \emph{variable} that is part of another variable (as an array or 
structure element) cannot be made private independently of other components.
\glossarydefend

\glossaryterm{shared variable}
\glossarydefstart
With respect to a given set of \emph{task regions} that bind to the same 
\code{parallel} \emph{region}, a \emph{variable} for which the name 
provides access to the same block of storage for each \emph{task region}.

A \emph{variable} that is part of another variable (as an array or 
structure element) cannot be \emph{shared} independently of the other 
components, except for static data members of C++ classes.
\glossarydefend

\glossaryterm{threadprivate variable}
\glossarydefstart
A \emph{variable} that is replicated, one instance per \emph{thread}, by the 
OpenMP implementation. Its name then provides access to a different block of 
storage for each \emph{thread}.

A \emph{variable} that is part of another variable (as an array or structure 
element) cannot be made \emph{threadprivate} independently of the other 
components, except for static data members of C++ classes.
\glossarydefend

\glossaryterm{threadprivate memory}
\glossarydefstart
The set of \emph{threadprivate variables} associated with each \emph{thread}.
\glossarydefend

\glossaryterm{data environment}
\glossarydefstart
The \emph{variables} associated with the execution of a given \emph{region}.
\glossarydefend

\glossaryterm{device data environment}
\glossarydefstart
The initial \emph{data environment} associated with a device.
\glossarydefend
\bigskip

\glossaryterm{device address}
\glossarydefstart
An \emph{implementation defined} reference to an address in a 
\emph{device data environment}.
\glossarydefend

\glossaryterm{device pointer}
\glossarydefstart
A \emph{variable} that contains a \emph{device address}.
\glossarydefend


\glossaryterm{mapped variable}
\glossarydefstart
An original \emph{variable} in a \emph{data environment} with a corresponding 
\emph{variable} in a device \emph{data environment}.

\begin{quote}
COMMENT: The original and corresponding \emph{variables} may share storage.
\end{quote}
\glossarydefend

\begin{table}[b]
\centering
\caption{Map-Type Decay of Map Type Combinations\label{tab:map-type_decay}}
\begin{tabular}{l|c|c|c|c|c|c}
  & alloc & to    & from  & tofrom & release & delete \\
  \hline
alloc  & alloc & alloc & alloc & alloc  & release & delete \\
to     & alloc & to    & alloc & to     & release & delete \\
from   & alloc & alloc & from  & from   & release & delete \\
tofrom & alloc & to    & from  & tofrom & release & delete \\
\end{tabular}
\end{table}

\glossaryterm{map-type decay}
\glossarydefstart
The process used to determine the final map type when mapping a variable
with a user defined mapper. Table~\ref{tab:map-type_decay} shows the final 
map type that the combination of the two map types determines.
\glossarydefend

\glossaryterm{mappable type}
\glossarydefstart
A type that is valid for a \emph{mapped variable}. If a type is composed 
from other types (such as the type of an array or structure element) and 
any of the other types are not mappable then the type is not mappable.

\begin{quote}
COMMENT: Pointer types are \emph{mappable} but the memory block to which 
the pointer refers is not \emph{mapped}.
\end{quote}

For C, the type must be a complete type.

For C++, the type must be a complete type.

In addition, for class types:
\begin{itemize}
\item All member functions accessed in any \code{target} region must appear in a
\code{declare}~\code{target} directive.
\end{itemize}

For Fortran, no restrictions on the type except that for derived types:

\begin{itemize}
\item All type-bound procedures accessed in any target region must appear in a 
\code{declare}~\code{target} directive.
\end{itemize}
\glossarydefend

\glossaryterm{defined}
\glossarydefstart
For \emph{variables}, the property of having a valid value.

For C, for the contents of \emph{variables}, the property of having a valid 
value.

For C++, for the contents of \emph{variables} of POD (plain old data) type, 
the property of having a valid value.

For \emph{variables} of non-POD class type, the property of having been 
constructed but not subsequently destructed.

For Fortran, for the contents of \emph{variables}, the property of having 
a valid value. For the allocation or association status of \emph{variables}, 
the property of having a valid status.

\begin{quote}
COMMENT: Programs that rely upon \emph{variables} that are not \emph{defined} 
are \emph{non-conforming programs}.
\end{quote}
\glossarydefend

\glossaryterm{class type}
\glossarydefstart
For C++, \emph{variables} declared with one of the \code{class}, \code{struct}, 
or \code{union} keywords.
\glossarydefend





\subsection{Implementation Terminology}
\index{implementation terminology}
\label{subsec:Implementation Terminology}
\glossaryterm{supporting \emph{n} active levels of parallelism}
\glossarydefstart
Implies allowing an \emph{active parallel region} to be enclosed by \emph{n-1} 
\emph{active parallel regions}.
\glossarydefend

\glossaryterm{supporting the OpenMP API}
\glossarydefstart
Supporting at least one active level of parallelism.
\glossarydefend
\bigskip

\glossaryterm{supporting nested parallelism}
\glossarydefstart
Supporting more than one active level of parallelism.
\glossarydefend
\bigskip

\glossaryterm{internal control variable}
\glossarydefstart
A conceptual variable that specifies runtime behavior of a set of \emph{threads} 
or \emph{tasks} in an \emph{OpenMP program}.

\begin{quote}
COMMENT: The acronym ICV is used interchangeably with the term 
\emph{internal control variable} in the remainder of this specification.
\end{quote}
\glossarydefend

\glossaryterm{compliant implementation}
\glossarydefstart
An implementation of the OpenMP specification that compiles and executes any
\emph{conforming program} as defined by the specification.

\begin{quote}
COMMENT: A \emph{compliant implementation} may exhibit 
\emph{unspecified behavior} when
compiling or executing a \emph{non-conforming program}.
\end{quote}
\glossarydefend

\glossaryterm{unspecified behavior}
\glossarydefstart
A behavior or result that is not specified by the OpenMP specification or not
known prior to the compilation or execution of an \emph{OpenMP program}.

Such \emph{unspecified behavior} may result from:

\begin{itemize}
\item Issues documented by the OpenMP specification as having 
\emph{unspecified behavior}.

\item A \emph{non-conforming program}.

\item A \emph{conforming program} exhibiting an 
\emph{implementation defined} behavior.
\end{itemize}
\glossarydefend

\glossaryterm{implementation defined}
\glossarydefstart
Behavior that must be documented by the implementation, and is allowed to vary
among different \emph{compliant implementations}. An implementation is allowed to
define this behavior as \emph{unspecified}.

\begin{quote}
COMMENT: All features that have \emph{implementation defined} behavior
are documented in Appendix~\ref{chap:OpenMP Implementation-Defined Behaviors}.
\end{quote}
\glossarydefend

\glossaryterm{deprecated}
\glossarydefstart
For a construct, clause, or other feature, the property that it is normative 
in the current specification but is considered obsolescent and will be 
removed in the future.
\glossarydefend


\subsection{Tool Terminology}

\glossaryterm{tool}
\glossarydefstart
Executable code, distinct from application or runtime code, that can observe 
and/or modify the execution of an application.
\glossarydefend

\glossaryterm{first-party tool}
\glossarydefstart
A tool that executes in the address space of the program that it is monitoring.
\glossarydefend

\glossaryterm{third-party tool}
\glossarydefstart
A tool that executes as a separate process from the process that it is 
monitoring and potentially controlling.
\glossarydefend

\glossaryterm{activated tool}
\glossarydefstart
A \emph{first-party tool} that successfully completed its initialization.
\glossarydefend

\glossaryterm{event}
\glossarydefstart
A point of interest in the execution of a thread.
\glossarydefend

\glossaryterm{native thread}
\glossarydefstart
A thread defined by an underlying thread implementation.
\glossarydefend

\glossaryterm{tool callback}
\glossarydefstart
A function that a tool provides to an OpenMP implementation to 
invoke when an associated event occurs.
\glossarydefend

\glossaryterm{registering a callback}
\glossarydefstart
Providing a \emph{tool callback} to an OpenMP implementation.
\glossarydefend

\glossaryterm{dispatching a callback at an event}
\glossarydefstart
Processing a callback when an associated \emph{event} occurs in a manner 
consistent with the return code provided when a \emph{first-party tool} 
registered the callback.
\glossarydefend

\glossaryterm{thread state}
\glossarydefstart
An enumeration type that describes the current OpenMP activity  of a 
\emph{thread}. A \emph{thread} can be in only one state at any time.
\glossarydefend

\glossaryterm{wait identifier}
\glossarydefstart
A unique opaque handle associated with each data object (for example, a lock) 
used by the OpenMP runtime to enforce mutual exclusion that may cause a thread 
to wait actively or passively.
\glossarydefend

\glossaryterm{frame}
\glossarydefstart
A storage area on a thread's stack associated with a procedure invocation. 
A frame includes space for one or more saved registers and often also 
includes space for saved arguments, local variables,
and padding for alignment.
\glossarydefend

\glossaryterm{canonical frame address}
\glossarydefstart
An address associated with a procedure \emph{frame} on a call stack that 
was the value of the stack pointer immediately prior to calling the procedure 
for which the invocation is represented by the frame.
\glossarydefend

\glossaryterm{runtime entry point}
\glossarydefstart
A function interface provided by an OpenMP runtime for use by a tool. 
A runtime entry point is typically not associated with a global function symbol.
\glossarydefend

\glossaryterm{trace record}
\glossarydefstart
A data structure in which to store information associated with an 
occurrence of an \emph{event}.
\glossarydefend

\glossaryterm{native trace record}
\glossarydefstart
A \emph{trace record} for an OpenMP device that is in a device-specific format.
\glossarydefend

\glossaryterm{signal}
\glossarydefstart
A software interrupt delivered to a \emph{thread}.
\glossarydefend

\glossaryterm{signal handler}
\glossarydefstart
A function called asynchronously when a \emph{signal} is delivered to 
a \emph{thread}.
\glossarydefend

\glossaryterm{async signal safe}
\glossarydefstart
The guarantee that interruption by \emph{signal} delivery will not
interfere with a set of operations.
An async signal safe \emph{runtime entry point} is safe to call 
from a \emph{signal handler}.
\glossarydefend

\glossaryterm{code block}
\glossarydefstart
A contiguous region of memory that contains code of an OpenMP program 
to be executed on a device.
\glossarydefend


\glossaryterm{OMPT}
\glossarydefstart
An interface that helps a \emph{first-party tool} monitor the execution 
of an OpenMP program.
\glossarydefend

\glossaryterm{OMPD}
\glossarydefstart
An interface that helps a \emph{third-party tool} inspect the OpenMP state 
of a program that has begun execution.
\glossarydefend

\glossaryterm{OMPD library}
\glossarydefstart
A dynamically loadable library that implements the \emph{OMPD} interface.
\glossarydefend

\glossaryterm{image file}
\glossarydefstart
An executable or shared library.
\glossarydefend

\glossaryterm{address space}
\glossarydefstart
A collection of logical, virtual, or physical memory address ranges 
that contain code, stack, and/or data. 
Address ranges within an address space need not be contiguous.  
An address space consists of one or more \emph{segments}.
\glossarydefend

\glossaryterm{segment}
\glossarydefstart
A portion of an address space associated with a set of address ranges.
\glossarydefend

\begin{comment}
\glossaryterm{architecture}
\glossarydefstart
A combination of the processor and the Application Binary Interface (ABI) used by
threads and address spaces.
\glossarydefend
\end{comment}

\glossaryterm{OpenMP architecture}
\glossarydefstart
The architecture on which an OpenMP \emph{region} executes.
\glossarydefend

\glossaryterm{tool architecture}
\glossarydefstart
The architecture on which an \emph{OMPD} tool executes.
\glossarydefend

\glossaryterm{OpenMP process}
\glossarydefstart
A collection of one or more \emph{threads} and \emph{address spaces}. 
A process may contain \emph{threads} and \emph{address spaces} for 
multiple \emph{OpenMP architectures}. At least one thread in 
an OpenMP process is an OpenMP \emph{thread}.
A process may be live or a core file.
\glossarydefend

\glossaryterm{handle}
\glossarydefstart
An opaque reference provided by an \emph{OMPD} library to a using tool. 
A handle uniquely identifies an abstraction.
\glossarydefend

\glossaryterm{address space handle}
\glossarydefstart
A \emph{handle} that refers to an \emph{address space} within an OpenMP process.
\glossarydefend

\glossaryterm{thread handle}
\glossarydefstart
A \emph{handle} that refers to an OpenMP \emph{thread}.
\glossarydefend

\glossaryterm{parallel handle}
\glossarydefstart
A \emph{handle} that refers to an OpenMP parallel \emph{region}.
\glossarydefend

\glossaryterm{task handle}
\glossarydefstart
A \emph{handle} that refers to an OpenMP task \emph{region}.
\glossarydefend

\glossaryterm{descendent handle}
\glossarydefstart
An output \emph{handle} that is returned from the \emph{OMPD} library 
in a function that accepts an input \emph{handle}: the output \emph{handle} 
is a descendent of the input emph{handle}.
\glossarydefend

\glossaryterm{ancestor handle}
\glossarydefstart
An input \emph{handle} that is passed to the \emph{OMPD} library in a 
function that returns an output \emph{handle}: the input emph{handle} 
is an ancestor of the output \emph{handle}. For a given \emph{handle}, 
the ancestors of the \emph{handle} are also the ancestors of the handle's 
descendent.

\begin{quote}
	COMMENT: A \emph{handle} cannot be used by the tool in an \emph{OMPD}
        call if any ancestor of the \emph{handle} has been released, except 
        for \emph{OMPD} calls 	that release the \emph{handle}.
\end{quote}
\glossarydefend

\glossaryterm{tool context}
\glossarydefstart
An opaque reference provided by a tool to an \emph{OMPD} library. 
A \emph{tool context} uniquely identifies an abstraction.
\glossarydefend

\glossaryterm{address space context}
\glossarydefstart
A \emph{tool context} that refers to an \emph{address space} within a process.
\glossarydefend

\glossaryterm{thread context}
\glossarydefstart
A \emph{tool context} that refers to a \emph{native thread}.
\glossarydefend

\glossaryterm{thread identifier}
\glossarydefstart
An identifier for a native thread defined by a thread implementation.
\glossarydefend


