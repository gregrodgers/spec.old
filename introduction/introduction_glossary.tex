% This is an included file. See the master file for more information.
%
% When editing this file:
%
%    1. To change formatting, appearance, or style, please edit openmp.sty.
%
%    2. Custom commands and macros are defined in openmp.sty.
%
%    3. Be kind to other editors -- keep a consistent style by copying-and-pasting to
%       create new content.
%
%    4. We use semantic markup, e.g. (see openmp.sty for a full list):
%         \code{}     % for bold monospace keywords, code, operators, etc.
%         \plc{}      % for italic placeholder names, grammar, etc.
%
%    5. There are environments that provide special formatting, e.g. language bars.
%       Please use them whereever appropriate.  Examples are:
%
%         \begin{fortranspecific}
%         This is text that appears enclosed in blue language bars for Fortran.
%         \end{fortranspecific}
%
%         \begin{note}
%         This is a note.  The "Note -- " header appears automatically.
%         \end{note}
%
%    6. Other recommendations:
%         Use the convenience macros defined in openmp.sty for the minor headers
%         such as Comments, Syntax, etc.
%
%         To keep items together on the same page, prefer the use of
%         \begin{samepage}.... Avoid \parbox for text blocks as it interrupts line numbering.
%         When possible, avoid \filbreak, \pagebreak, \newpage, \clearpage unless that's
%         what you mean. Use \needspace{} cautiously for troublesome paragraphs.
%
%         Avoid absolute lengths and measures in this file; use relative units when possible.
%         Vertical space can be relative to \baselineskip or ex units. Horizontal space
%         can be relative to \linewidth or em units.
%
%         Prefer \emph{} to italicize terminology, e.g.:
%             This is a \emph{definition}, not a placeholder.
%             This is a \plc{var-name}.
%

\section{Glossary}
\label{sec:Glossary}
\index{glossary}
\subsection{Threading Concepts}
\label{subsec:Threading Concepts}
\glossaryterm{thread}
\glossarydefstart
An execution entity with a stack and associated static memory, called
\emph{threadprivate memory}.
\glossarydefend

\glossaryterm{OpenMP thread}
\glossarydefstart
A \emph{thread} that is managed by the OpenMP implementation.
\glossarydefend

\glossaryterm{idle thread}
\glossarydefstart
An \emph{OpenMP thread} that is not currently part of any \code{parallel} region.
\glossarydefend

\glossaryterm{thread-safe routine}
\glossarydefstart
A routine that performs the intended function even when executed concurrently
(by more than one \emph{thread}).
\glossarydefend

\glossaryterm{processor}
\glossarydefstart
Implementation defined hardware unit on which one or more \emph{OpenMP threads} can
execute.
\glossarydefend

\glossaryterm{device}
\glossarydefstart
An implementation defined logical execution engine.

\begin{quote}
COMMENT: A \emph{device} could have one or more \emph{processors}.
\end{quote}
\glossarydefend

\glossaryterm{host device}
\glossarydefstart
The \emph{device} on which the \emph{OpenMP program} begins execution.
\glossarydefend

\glossaryterm{target device}
\glossarydefstart
A device onto which code and data may be offloaded from the \emph{host device}.
\glossarydefend




% \vspace{5ex}
\subsection{OpenMP Language Terminology}
\label{subsec:OpenMP Language Terminology}
\glossaryterm{base language}
\glossarydefstart
A programming language that serves as the foundation of the OpenMP
specification.

\begin{quote}
COMMENT: See \specref{sec:normative references}
for a listing of current \emph{base languages} for the OpenMP API.
\end{quote}
\glossarydefend

\glossaryterm{base program}
\glossarydefstart
A program written in a \emph{base language}.
\glossarydefend

\glossaryterm{program order}
\glossarydefstart
An ordering of operations performed by the same thread as determined by the
execution sequence of operations specified by the \emph{base language}.

\begin{quote}
COMMENT: For C11 and C++11, \emph{program order} corresponds to the sequenced
before relation between operations performed by the same thread.
\end{quote}
\glossarydefend

\glossaryterm{structured block}
\glossarydefstart
For C/C++, an executable statement, possibly compound, with a single entry at the
top and a single exit at the bottom, or an OpenMP \emph{construct}.

For Fortran, a block of executable statements with a single entry at the top and a
single exit at the bottom, or an OpenMP \emph{construct}.

\begin{quote}
COMMENTS:

For all \emph{base languages}:

\begin{itemize}
\item Access to the \emph{structured block} must not be the result of a branch; and

\item The point of exit cannot be a branch out of the \emph{structured block}.
\end{itemize}

For C/C++:

\begin{itemize}
\item The point of entry must not be a call to \code{setjmp()};

\item \code{longjmp()} and \code{throw()} must not violate the entry/exit criteria;

\item Calls to \code{exit()} are allowed in a \emph{structured block}; and

\item An expression statement, iteration statement, selection statement,
or try block is considered to be a \emph{structured block} if the
corresponding compound statement obtained by enclosing it in \code|{|
and \code|}| would be a \emph{structured block}.
\end{itemize}

For Fortran:

\begin{itemize}
\item \code{STOP} statements are allowed in a \emph{structured block}.
\end{itemize}
\end{quote}
\glossarydefend


\glossaryterm{enclosing context}
\glossarydefstart
In C/C++, the innermost scope enclosing an OpenMP \emph{directive}.

In Fortran, the innermost scoping unit enclosing an OpenMP \emph{directive}.
\glossarydefend

\glossaryterm{directive}
\glossarydefstart
In C/C++, a \code{#pragma}, and in Fortran, a comment, that specifies \emph{OpenMP
program} behavior.

\begin{quote}
COMMENT: See \specref{sec:Directive Format} for a description of OpenMP \emph{directive} syntax.
\end{quote}
\glossarydefend


\glossaryterm{white space}
\glossarydefstart
A non-empty sequence of space and/or horizontal tab characters.
\glossarydefend

\glossaryterm{OpenMP program}
\glossarydefstart
A program that consists of a \emph{base program}, annotated with OpenMP \emph{directives} and
runtime library routines.
\glossarydefend

\glossaryterm{conforming program}
\glossarydefstart
An \emph{OpenMP program} that follows all rules and restrictions of the OpenMP
specification.
\glossarydefend

\glossaryterm{declarative directive}
\glossarydefstart
An OpenMP \emph{directive} that may only be placed in a declarative context. A
\emph{declarative directive} results in one or more declarations only; it is not associated
with the immediate execution of any user code.
\glossarydefend

\glossaryterm{executable directive}
\glossarydefstart
An OpenMP \emph{directive} that is not declarative. That is, it may be placed in an
executable context.
\glossarydefend

\glossaryterm{stand-alone directive}
\glossarydefstart
An OpenMP \emph{executable directive} that has no associated executable user code.
\glossarydefend


\glossaryterm{construct}
\glossarydefstart
An OpenMP \emph{executable directive} (and for Fortran, the paired \code{end} \emph{directive}, if
any) and the associated statement, loop or \emph{structured block}, if any, not including
the code in any called routines. That is, the lexical extent of an \emph{executable
directive}.
\glossarydefend

\glossaryterm{combined construct}
\glossarydefstart
A construct that is a shortcut for specifying one construct immediately nested inside another construct. A combined construct is semantically identical to that of explicitly specifying the first construct containing one instance of the second construct and no other statements.
\glossarydefend

\glossaryterm{composite construct}
\glossarydefstart
A construct that is composed of two constructs but does not have identical semantics to specifying one of the constructs immediately nested inside the other. A composite construct either adds semantics not included in the constructs from which it is composed or the nesting of the one construct inside the other is not conforming.
\glossarydefend


\glossaryterm{region}
\glossarydefstart
All code encountered during a specific instance of the execution of a given
\emph{construct} or of an OpenMP library routine. A \emph{region} includes any code in called
routines as well as any implicit code introduced by the OpenMP implementation.
The generation of a \emph{task} at the point where a \emph{task generating construct} is encountered is a
part of the \emph{region} of the \emph{encountering thread}, but an \emph{explicit task region}
associated with a \emph{task generating construct} is not unless it is an
\emph{included task region}. The point where a \code{target} or \code{teams}
directive is encountered is a part of the \emph{region} of the \emph{encountering thread}, but the
\emph{region} associated with the \code{target} or \code{teams} directive is not.

\begin{quote}
COMMENTS:

A \emph{region} may also be thought of as the dynamic or runtime extent of a
\emph{construct} or of an OpenMP library routine.

During the execution of an \emph{OpenMP program}, a \emph{construct} may give
rise to many \emph{regions}.
\end{quote}
\glossarydefend

\glossaryterm{active parallel region}
\glossarydefstart
A \code{parallel} \emph{region} that is executed by a \emph{team} consisting of more than one
\emph{thread}.
\glossarydefend

\smallskip
\glossaryterm{inactive parallel region}
\glossarydefstart
A \code{parallel} \emph{region} that is executed by a \emph{team} of only one \emph{thread}.
\glossarydefend

\glossaryterm{sequential part}
\glossarydefstart
All code encountered during the execution of an \emph{initial task region} that is not part
of a \code{parallel} \emph{region} corresponding to a \code{parallel} \emph{construct} or a \code{task}
\emph{region} corresponding to a \code{task} \emph{construct}.

\begin{quote}
COMMENTS:

A \emph{sequential part} is enclosed by an \emph{implicit parallel region}.

Executable statements in called routines may be in both a \emph{sequential
part} and any number of explicit \code{parallel} \emph{regions} at different points
in the program execution.
\end{quote}
\glossarydefend

\glossaryterm{master thread}
\glossarydefstart
An \emph{OpenMP thread} that has  \emph{thread} number 0. A \emph{master
thread} may be an \emph{initial thread} or the \emph{thread} that encounters a
\code{parallel} \emph{construct}, creates a \emph{team}, generates a set of
\emph{implicit tasks}, and then executes one of those \emph{tasks} as
\emph{thread} number 0.
\glossarydefend

\glossaryterm{parent thread}
\glossarydefstart
The \emph{thread} that encountered the \code{parallel} \emph{construct} and generated a
\code{parallel} \emph{region} is the \emph{parent thread} of each of the
\emph{threads} in the \emph{team} of that
\code{parallel} \emph{region}. The \emph{master thread}
of a \code{parallel} \emph{region} is the same \emph{thread}
as its \emph{parent thread} with respect to any resources associated with an \emph{OpenMP thread}.
\glossarydefend

\glossaryterm{child thread}
\glossarydefstart
When a thread encounters a \code{parallel} construct, each of the threads in the
generated \code{parallel} region's team are \emph{child threads} of the encountering \emph{thread}.
The \code{target} or \code{teams} region's \emph{initial thread} is not a \emph{child thread} of the thread
that encountered the \code{target} or \code{teams} construct.
\glossarydefend

\glossaryterm{ancestor thread}
\glossarydefstart
For a given \emph{thread}, its \emph{parent thread} or one of its \emph{parent thread's ancestor threads}.
\glossarydefend

\glossaryterm{descendent thread}
\glossarydefstart
For a given \emph{thread}, one of its \emph{child threads} or one of
its \emph{child threads' descendent threads}.
\glossarydefend

\glossaryterm{team}
\glossarydefstart
A set of one or more \emph{threads} participating in the execution of a \code{parallel}
\emph{region}.

\begin{quote}
COMMENTS:

For an \emph{active parallel region}, the team comprises the \emph{master thread}
and at least one additional \emph{thread}.

For an \emph{inactive parallel region}, the \emph{team} comprises only the \emph{master thread}.
\end{quote}
\glossarydefend

\glossaryterm{league}
\glossarydefstart
The set of \emph{teams} created by a \code{teams} construct.
\glossarydefend

\glossaryterm{contention group}
\glossarydefstart
An initial \emph{thread} and its \emph{descendent threads}.
\glossarydefend

\glossaryterm{implicit parallel region}
\glossarydefstart
An \emph{inactive parallel region} that is not generated from a
\code{parallel} \emph{construct}. \emph{Implicit parallel regions} surround the whole
\emph{OpenMP program}, all \code{target} \emph{regions}, and all \code{teams}
\emph{regions}.

\glossarydefend

\glossaryterm{initial thread}
\glossarydefstart
A \emph{thread} that executes an \emph{implicit parallel region}.
\glossarydefend

\glossaryterm{initial team}
\glossarydefstart
A \emph{team} that comprises an \emph{initial thread} executing an \emph{implicit parallel region}.
\glossarydefend

\glossaryterm{nested construct}
\glossarydefstart
A \emph{construct} (lexically) enclosed by another \emph{construct}.
\glossarydefend

\glossaryterm{closely nested construct}
\glossarydefstart
A \emph{construct} nested inside another \emph{construct} with no other \emph{construct} nested
between them.
\glossarydefend

\glossaryterm{nested region}
\glossarydefstart
A \emph{region} (dynamically) enclosed by another \emph{region}.  That is, a
\emph{region} generated from the execution of another \emph{region}
or one of its \emph{nested regions}.

\begin{quote}
COMMENT: Some nestings are \emph{conforming} and some are not.
See \specref{sec:Nesting of Regions} for the restrictions on nesting.
\end{quote}
\glossarydefend

\glossaryterm{closely nested region}
\glossarydefstart
A \emph{region nested} inside another \emph{region} with no \code{parallel} \emph{region nested} between
them.
\glossarydefend

\glossaryterm{strictly nested region}
\glossarydefstart
A \emph{region nested} inside another \emph{region} with no other \emph{region nested} between
them.
\glossarydefend

\glossaryterm{all threads}
\glossarydefstart
All OpenMP \emph{threads} participating in the \emph{OpenMP program}.
\glossarydefend

\glossaryterm{current team}
\glossarydefstart
All \emph{threads} in the \emph{team} executing the innermost enclosing \code{parallel} \emph{region}.
\glossarydefend

\glossaryterm{encountering thread}
\glossarydefstart
For a given \emph{region}, the \emph{thread} that encounters the
corresponding \emph{construct}.
\glossarydefend

\glossaryterm{all tasks}
\glossarydefstart
All \emph{tasks} participating in the \emph{OpenMP program}.
\glossarydefend

\glossaryterm{current team tasks}
\glossarydefstart
All \emph{tasks} encountered by the corresponding \emph{team}. The \emph{implicit tasks}
constituting the \code{parallel} \emph{region} and any \emph{descendent tasks} encountered during
the execution of these \emph{implicit tasks} are included in this set of tasks.
\glossarydefend

\glossaryterm{generating task}
\glossarydefstart
For a given \emph{region}, the task for which execution by a \emph{thread} generated the \emph{region}.
\glossarydefend

\glossaryterm{binding thread set}
\glossarydefstart
The set of \emph{threads} that are affected by, or provide the context for, the execution of
a \emph{region}.

The \emph{binding thread} set for a given \emph{region} can be \emph{all threads} on a \emph{device}, \emph{all
threads} in a \emph{contention group}, all \emph{master threads} executing an
enclosing \code{teams} \emph{region}, the \emph{current team}, or the \emph{encountering thread}.

\begin{quote}
COMMENT: The \emph{binding thread set} for a particular \emph{region} is described in its
corresponding subsection of this specification.
\end{quote}
\glossarydefend

\glossaryterm{binding task set}
\glossarydefstart
The set of \emph{tasks} that are affected by, or provide the context for, the execution of a
\emph{region}.

The \emph{binding task} set for a given \emph{region} can be \emph{all tasks},
the \emph{current team tasks}, the \emph{binding implicit task} or the \emph{generating task}.

\begin{quote}
COMMENT: The \emph{binding task} set for a particular \emph{region} (if applicable) is
described in its corresponding subsection of this specification.
\end{quote}
\glossarydefend

%\pagebreak
\glossaryterm{binding region}
\glossarydefstart
The enclosing \emph{region} that determines the execution context and limits the scope of
the effects of the bound \emph{region} is called the \emph{binding region}.

\emph{Binding region} is not defined for \emph{regions} for which the \emph{binding thread} set is \emph{all threads}
or the \emph{encountering thread}, nor is it defined for \emph{regions} for which the \emph{binding task set} is
\emph{all tasks}.

\begin{quote}
COMMENTS:

The \emph{binding region} for an \code{ordered} \emph{region} is the innermost enclosing
\emph{loop region}.

The \emph{binding region} for a \code{taskwait} \emph{region} is the innermost enclosing
\emph{task region}.

The \emph{binding region} for a \code{cancel} \emph{region} is the innermost enclosing \emph{region} corresponding to the \plc{construct-type-clause} of the \code{cancel} construct.

The \emph{binding region} for a \code{cancellation point} \emph{region} is the innermost enclosing \emph{region} corresponding to the \plc{construct-type-clause} of the \code{cancellation point} construct.

For all other \emph{regions} for which the \emph{binding thread set} is the \emph{current
team} or the \emph{binding task set} is the \emph{current team tasks}, the \emph{binding
region} is the innermost enclosing \code{parallel} \emph{region}.

For \emph{regions} for which the \emph{binding task set} is the \emph{generating task}, the
\emph{binding region} is the \emph{region} of the \emph{generating task}.

A \code{parallel} \emph{region} need not be \emph{active} nor explicit to be a \emph{binding region}.

A \emph{task region} need not be explicit to be a \emph{binding region}.

A \emph{region} never binds to any \emph{region} outside of the innermost enclosing
\code{parallel} \emph{region}.
\end{quote}
\glossarydefend

\glossaryterm{orphaned construct}
\glossarydefstart
A \emph{construct} that gives rise to a \emph{region} for which the \emph{binding thread set} is the \emph{current
team}, but is not nested within another \emph{construct} giving rise to the \emph{binding region}.
\glossarydefend

\glossaryterm{worksharing construct}
\glossarydefstart
A \emph{construct} that defines units of work, each of which is executed exactly once by
one of the \emph{threads} in the \emph{team} executing the \emph{construct}.

For C/C++, \emph{worksharing constructs} are \code{for}, \code{sections}, and \code{single}.

For Fortran, \emph{worksharing constructs} are \code{do}, \code{sections}, \code{single} and
\code{workshare}.
\glossarydefend

%\newpage
\glossaryterm{place}
\glossarydefstart
Unordered set of \emph{processors} on a device that is treated by the execution environment as a
location unit when dealing with OpenMP thread affinity.
\glossarydefend

\glossaryterm{place list}
\glossarydefstart
The ordered list that describes all OpenMP \emph{places} available to the execution
environment.
\glossarydefend

\glossaryterm{place partition}
\glossarydefstart
An ordered list that corresponds to a contiguous interval in the OpenMP \emph{place list}.
It describes the \emph{places} currently available to the execution environment for a given
parallel \emph{region}.
\glossarydefend

\glossaryterm{place number}
\glossarydefstart
A number that uniquely identifies a \emph{place} in the \emph{place list}, with zero identifying the first \emph{place} in the \emph{place list}, and each consecutive whole number identifying the next \emph{place} in the \emph{place list}.
\glossarydefend

\glossaryterm{SIMD instruction}
\glossarydefstart
A single machine instruction that can operate on multiple data elements.
\glossarydefend

\glossaryterm{SIMD lane}
\glossarydefstart
A software or hardware mechanism capable of processing one data element from a
\emph{SIMD instruction}.
\glossarydefend

\glossaryterm{SIMD chunk}
\glossarydefstart
A set of iterations executed concurrently, each by a \emph{SIMD lane}, by a single \emph{thread}
by means of \emph{SIMD instructions}.
\glossarydefend

\glossaryterm{memory allocator}
\glossarydefstart
An OpenMP object that fulfills requests to allocate and deallocate storage for program variables.
\glossarydefend

%
% Loop Terminology
%
% \vspace{5ex}
\subsection{Loop Terminology}
\index{loop terminology}
\label{subsec:Loop Terminology}
\glossaryterm{loop directive}
\glossarydefstart
An OpenMP \emph{executable} directive for which the associated user code must be a loop nest that is a \emph{structured block}.
\glossarydefend

\glossaryterm{associated loop(s)}
\glossarydefstart
The loop(s) controlled by a \emph{loop directive}.
\begin{quote}
COMMENT: If the \emph{loop directive} contains a \code{collapse} or an \code{ordered(}\plc{n}\code{)} clause then it may have more than one \emph{associated loop}.
\end{quote}
\glossarydefend

\glossaryterm{sequential loop}
\glossarydefstart
A loop that is not associated with any OpenMP \emph{loop directive}.
\glossarydefend

\glossaryterm{SIMD loop}
\glossarydefstart
A loop that includes at least one \emph{SIMD chunk}.
\glossarydefend

\glossaryterm{doacross loop nest}
\glossarydefstart
A loop nest that has cross-iteration dependence. An iteration is dependent on one or more lexicographically earlier iterations.
\begin{quote}
COMMENT: The \code{ordered} clause parameter on a loop directive identifies the loop(s) associated with the \emph{doacross loop nest}.
\end{quote}
\glossarydefend

%
% Synchronization Terminology
%
\subsection{Synchronization Terminology}
\index{synchronization terminology}
\label{subsec:Synchronization Terminology}
\glossaryterm{barrier}
\glossarydefstart
A point in the execution of a program encountered by a \emph{team} of \emph{threads}, beyond
which no \emph{thread} in the team may execute until all \emph{threads} in the \emph{team} have
reached the barrier and all \emph{explicit tasks} generated by the \emph{team} have executed to
completion. If \emph{cancellation} has been requested, threads may proceed to the end of
the canceled \emph{region} even if some threads in the team have not reached the \emph{barrier}.
\glossarydefend

\glossaryterm{cancellation}
\glossarydefstart
An action that cancels (that is, aborts) an OpenMP \emph{region} and causes executing
\emph{implicit} or \emph{explicit} tasks to proceed to the end of the canceled \emph{region}.
\glossarydefend

\glossaryterm{cancellation point}
\glossarydefstart
A point at which implicit and explicit tasks check if cancellation has been
requested. If cancellation has been observed, they perform the \emph{cancellation}.

\begin{quote}
COMMENT: For a list of cancellation points, see \specref{subsec:cancel Construct}
\end{quote}
\glossarydefend
\bigskip

\glossaryterm{flush}
\glossarydefstart
An operation, applied to a set of variables, that a \emph{thread} performs
to enforce consistency between its view and other \emph{threads}' view of
memory.
\glossarydefend

\glossaryterm{flush-set}
\glossarydefstart
The set of variables to which a given \emph{flush} operation is applied.
\glossarydefend

\glossaryterm{flush property}
\glossarydefstart
Properties that determine the manner in which a \emph{flush} operation enforces
memory consistency. These properties are:
\begin{itemize}
    \item \emph{strong}:  guarantees a completion ordering between memory operations
        on different threads upon which all threads agree;
    \item \emph{write}: makes modifications by the executing thread visible
        to other threads;
    \item \emph{read}: makes modifications by other threads visible to the
        executing thread;
    \item \emph{synchronizable}: allows the \emph{flush} operation to
        synchronize with another \emph{flush} operation.
\end{itemize}

\begin{quote}
COMMENT: A given \emph{flush} operation can have one or more \emph{flush
properties}, but must have at least the write or read flush property.
\end{quote}
\glossarydefend

\glossaryterm{strong flush}
\glossarydefstart
A \emph{flush} operation that has the \emph{strong flush property}.
\glossarydefend

\glossaryterm{weak flush}
\glossarydefstart
A \emph{flush} operation that does not have the \emph{strong flush property}.
\glossarydefend

\glossaryterm{write flush}
\glossarydefstart
A \emph{flush} operation that has the \emph{write flush property}.
\glossarydefend

\glossaryterm{read flush}
\glossarydefstart
A \emph{flush} operation that has the \emph{read flush property}.
\glossarydefend

\glossaryterm{synchronizable flush}
\glossarydefstart
A \emph{flush} operation that has the \emph{synchronizable flush property}.
\glossarydefend

\glossaryterm{non-synchronizable flush}
\glossarydefstart
A \emph{flush} operation that does not have the \emph{synchronizable flush
property}.
\glossarydefend
\medskip

\glossaryterm{release flush}
\glossarydefstart
A \emph{write flush} that is a \emph{synchronizable flush}.
\glossarydefend

\glossaryterm{acquire flush}
\glossarydefstart
A \emph{read flush} that is a \emph{synchronizable flush}.
\glossarydefend

\glossaryterm{sync-set}
\glossarydefstart
The set of variables associated with a \emph{release flush} or \emph{acquire flush} that a
program may use to force a \emph{release flush} to synchronize with an \emph{acquire flush}.
\glossarydefend




% \vspace{5ex}
\subsection{Tasking Terminology}
\index{tasking terminology}
\label{subsec:Tasking Terminology}
\glossaryterm{task}
\glossarydefstart
A specific instance of executable code and its data environment that the
OpenMP implementation can schedule for execution by threads.
\glossarydefend

\glossaryterm{task region}
\glossarydefstart
A \emph{region} consisting of all code encountered during the execution of a \emph{task}.

\begin{quote}
COMMENT: A \code{parallel} \emph{region} consists of one or more implicit \emph{task regions}.
\end{quote}
\glossarydefend

\glossaryterm{implicit task}
\glossarydefstart
A \emph{task} generated by an \emph{implicit parallel region} or generated when a \code{parallel}
\emph{construct} is encountered during execution.
\glossarydefend

\glossaryterm{binding implicit task}
\glossarydefstart
The \emph{implicit task} of the current thread team assigned to the encountering thread.
\glossarydefend

\glossaryterm{explicit task}
\glossarydefstart
A \emph{task} that is not an \emph{implicit task}.
\glossarydefend

\glossaryterm{initial task}
\glossarydefstart
An \emph{implicit task} associated with an \emph{implicit parallel region}.
\glossarydefend

\glossaryterm{current task}
\glossarydefstart
For a given \emph{thread}, the \emph{task} corresponding to the \emph{task region} in which it is
executing.
\glossarydefend

\glossaryterm{child task}
\glossarydefstart
A \emph{task} is a \emph{child task} of its generating \emph{task region}.
A \emph{child task region} is not part of its generating \emph{task region}.
\glossarydefend

\glossaryterm{sibling tasks}
\glossarydefstart
\emph{Tasks} that are \emph{child tasks} of the same \emph{task region}.
\glossarydefend

\glossaryterm{descendent task}
\glossarydefstart
A \emph{task} that is the \emph{child task} of a \emph{task region} or of one of its
\emph{descendent task regions}.
\glossarydefend

\glossaryterm{task completion}
\glossarydefstart
\emph{Task completion} occurs when the end of the \emph{structured block} associated with the
\emph{construct} that generated the \emph{task} is reached.

\begin{quote}
COMMENT: Completion of the \emph{initial task} that is generated when the program begins occurs at program exit.
\end{quote}
\glossarydefend

\glossaryterm{task scheduling point}
\glossarydefstart
A point during the execution of the current \emph{task region} at which it can be
suspended to be resumed later; or the point of \emph{task completion}, after which the
executing thread may switch to a different \emph{task region}.

\begin{quote}
COMMENT: For a list of \emph{task scheduling points}, see \specref{subsec:Task Scheduling}.
\end{quote}
\glossarydefend

\glossaryterm{task switching}
\glossarydefstart
The act of a \emph{thread} switching from the execution of one \emph{task} to another \emph{task}.
\glossarydefend

\glossaryterm{tied task}
\glossarydefstart
A \emph{task} that, when its \emph{task region} is suspended, can be resumed only by the same
\emph{thread} that suspended it. That is, the \emph{task} is tied to that \emph{thread}.
\glossarydefend

\glossaryterm{untied task}
\glossarydefstart
A \emph{task} that, when its \emph{task region} is suspended, can be resumed by any \emph{thread} in
the team. That is, the \emph{task} is not tied to any \emph{thread}.
\glossarydefend

\glossaryterm{undeferred task}
\glossarydefstart
A \emph{task} for which execution is not deferred with respect to its generating \emph{task}
\emph{region}. That is, its generating \emph{task region} is suspended until execution of the
\emph{undeferred task} is completed.
\glossarydefend

\glossaryterm{included task}
\glossarydefstart
A \emph{task} for which execution is sequentially included in the generating \emph{task region}.
That is, an \emph{included task} is \emph{undeferred} and executed immediately by the
\emph{encountering thread}.
\glossarydefend

\glossaryterm{merged task}
\glossarydefstart
A \emph{task} for which the \emph{data environment}, inclusive of ICVs, is the same as that of its
generating \emph{task region}.
\glossarydefend

\glossaryterm{mergeable task}
\glossarydefstart
A \emph{task} that may be a \emph{merged task} if it is an \emph{undeferred task} or an \emph{included task}.
\glossarydefend

\glossaryterm{final task}
\glossarydefstart
A \emph{task} that forces all of its \emph{child tasks} to become \emph{final} and \emph{included tasks}.
\glossarydefend

\glossaryterm{task dependence}
\glossarydefstart
An ordering relation between two \emph{sibling tasks}: the \emph{dependent task} and a
previously generated \emph{predecessor task}. The \emph{task dependence} is fulfilled when the
\emph{predecessor task} has completed.
\glossarydefend

\begin{samepage}
\glossaryterm{dependent task}
\glossarydefstart
A \emph{task} that because of a \emph{task dependence} cannot be executed until its \emph{predecessor
tasks} have completed.
\glossarydefend
\end{samepage}

\glossaryterm{mutually exclusive tasks}
\glossarydefstart
\emph{Tasks} that may be executed in any order, but not at the same
time.
\glossarydefend
\bigskip

\glossaryterm{predecessor task}
\glossarydefstart
A \emph{task} that must complete before its \emph{dependent tasks} can be executed.
\glossarydefend

\glossaryterm{task synchronization construct}
\glossarydefstart
A \code{taskwait}, \code{taskgroup}, or a \code{barrier} \emph{construct}.
\glossarydefend
\bigskip

\glossaryterm{task generating construct}
\glossarydefstart
A \emph{construct} that generates one or more \emph{explicit tasks}.
\glossarydefend
\bigskip

\glossaryterm{target task}
\glossarydefstart
A \emph{mergeable} and \emph{untied} \emph{task} that is generated by a \code{target}, \code{target enter data}, \code{target exit data}, or \code{target update} \emph{construct}.
\glossarydefend

\glossaryterm{taskgroup set}
\glossarydefstart
A set of tasks that are logically grouped by a \code{taskgroup} \emph{region}.
\glossarydefend

%\newpage %% HACK
\subsection{Data Terminology}
\index{data terminology}
\label{subsec:Data Terminology}
\glossaryterm{variable}
\glossarydefstart
A named data storage block, for which the value can be defined and redefined during the
execution of a program.

\begin{adjustwidth}{-0.75in}{0in}
\begin{note}
An array or structure element is a variable that is part of another variable.
\end{note}
\end{adjustwidth}
\glossarydefend

\glossaryterm{scalar variable}
\glossarydefstart
For C/C++:
\nopagebreak
A scalar variable, as defined by the base language.

For Fortran:
\nopagebreak
A scalar variable with intrinsic type, as defined by the base language,
excluding character type.
\glossarydefend

%\glossaryterm{aggregate variable}
%\glossarydefstart
%A variable, such as an array or structure, composed of other variables.
%\glossarydefend

\glossaryterm{array section}
\glossarydefstart
A designated subset of the elements of an array.
\glossarydefend

\glossaryterm{array item}
\glossarydefstart
An array, an array section, or an array element.
\glossarydefend

\glossaryterm{base expression}
\glossarydefstart
For C/C++: The expression in an array section or an array element that specifies
the address of the initial element of the original array.
\glossarydefend

\glossaryterm{named array}
\glossarydefstart
For C/C++:
\nopagebreak
An expression that is an array but not an array element and appears as the
array referred to by a given array item.

For Fortran:
\nopagebreak
A variable that is an array and appears as the array referred to by a given
array item.
\glossarydefend

\glossaryterm{named pointer}
\glossarydefstart
For C/C++:
\nopagebreak
An lvalue expression that is a pointer and appears as a pointer to the array
implicitly referred to by a given array item.

For Fortran:
\nopagebreak
A variable that has the \code{POINTER} attribute and appears as a pointer to
the array to which a given array item implicitly refers.
%array implicitly referred to by a given array item

\begin{adjustwidth}{-0.75in}{0in}
\begin{note}
A given array item cannot have a \emph{named pointer} if it has a \emph{named array}.
\end{note}
\end{adjustwidth}
\glossarydefend


\glossaryterm{attached pointer}
\glossarydefstart
A pointer variable in a device data environment to which the effect of a \code{map} clause
assigns the address of
an array section.  The pointer is
an attached pointer for the remainder of its lifetime in the device data environment.
\glossarydefend
\bigskip

\glossaryterm{simply contiguous array section}
\glossarydefstart
An array section that statically can be determined to have contiguous storage.
\glossarydefend
\bigskip

\glossaryterm{structure}
\glossarydefstart
A structure is a variable that contains one or more variables.

For C/C++:
\nopagebreak
Implemented using struct types.

For C++:
\nopagebreak
Implemented using class types.

For Fortran:
\nopagebreak
Implemented using derived types.
\glossarydefend

\glossaryterm{private variable}
\glossarydefstart
With respect to a given set of \emph{task regions} or \emph{SIMD lanes} that bind to the same
\code{parallel} \emph{region}, a \emph{variable} for which the name provides access to a different block of
storage for each \emph{task region} or \emph{SIMD lane}.

A \emph{variable} that is part of another variable (as an array or structure element) cannot
be made private independently of other components.
\glossarydefend

\glossaryterm{shared variable}
\glossarydefstart
With respect to a given set of \emph{task regions} that bind to the same \code{parallel}
\emph{region}, a \emph{variable} for which the name provides access to the same block of storage for
each \emph{task region}.

A \emph{variable} that is part of another variable (as an array or structure element) cannot
be \emph{shared} independently of the other components, except for static data members
of C++ classes.
\glossarydefend

\glossaryterm{threadprivate variable}
\glossarydefstart
A \emph{variable} that is replicated, one instance per \emph{thread}, by the OpenMP
implementation. Its name then provides access to a different block of storage for
each \emph{thread}.

A \emph{variable} that is part of another variable (as an array or structure element) cannot
be made \emph{threadprivate} independently of the other components, except for static
data members of C++ classes.
\glossarydefend

\glossaryterm{threadprivate memory}
\glossarydefstart
The set of \emph{threadprivate variables} associated with each \emph{thread}.
\glossarydefend

\glossaryterm{data environment}
\glossarydefstart
The \emph{variables} associated with the execution of a given \emph{region}.
\glossarydefend

\glossaryterm{device data environment}
\glossarydefstart
The initial \emph{data environment} associated with a device.
\glossarydefend
\bigskip

\glossaryterm{device address}
\glossarydefstart
An \emph{implementation defined} reference to an address in a \emph{device
  data environment}.
\glossarydefend

\glossaryterm{device pointer}
\glossarydefstart
A \emph{variable} that contains a \emph{device address}.
\glossarydefend


\glossaryterm{mapped variable}
\glossarydefstart
An original \emph{variable} in a \emph{data environment} with a corresponding \emph{variable} in a
device \emph{data environment}.

\begin{quote}
COMMENT: The original and corresponding \emph{variables} may share storage.
\end{quote}
\glossarydefend

\glossaryterm{map-type decay}
\glossarydefstart
The process used to determine the final map type used when mapping a variable
with a user defined mapper.  The combination of the two map types determines the
final map type based on the following table.

\begin{tabular}{l|c|c|c|c|c|c}
  & alloc & to    & from  & tofrom & release & delete \\
  \hline
alloc  & alloc & alloc & alloc & alloc  & release & delete \\
to     & alloc & to    & alloc & to     & release & delete \\
from   & alloc & alloc & from  & from   & release & delete \\
tofrom & alloc & to    & from  & tofrom & release & delete \\
\end{tabular}
\vspace{2ex} %% HACK
\glossarydefend

\glossaryterm{mappable type}
\glossarydefstart
A type that is valid for a \emph{mapped variable}. If a type is composed from other types
(such as the type of an array or structure element) and any of the other types are
not mappable then the type is not mappable.

\begin{quote}
COMMENT: Pointer types are \emph{mappable} but the memory block to which the pointer refers is not \emph{mapped}.
\end{quote}

For C:
\nopagebreak
The type must be a complete type.

For C++:
\nopagebreak
The type must be a complete type.

In addition, for class types:
\begin{itemize}
\item All member functions accessed in any \code{target} region must appear in a
\code{declare}~\code{target} directive.
\end{itemize}

For Fortran:
\nopagebreak
No restrictions on the type except that for derived types:

\begin{itemize}
\item All type-bound procedures accessed in any target region must appear in a \code{declare}~\code{target} directive.
\end{itemize}
\glossarydefend

\glossaryterm{defined}
\glossarydefstart
For \emph{variables}, the property of having a valid value.

For C:
\nopagebreak
For the contents of \emph{variables}, the property of having a valid value.

For C++:
\nopagebreak
For the contents of \emph{variables} of POD (plain old data) type, the property of having
a valid value.

For \emph{variables} of non-POD class type, the property of having been constructed but
not subsequently destructed.

For Fortran:
\nopagebreak
For the contents of \emph{variables}, the property of having a valid value. For the
allocation or association status of \emph{variables}, the property of having a valid status.

\begin{quote}
COMMENT: Programs that rely upon \emph{variables} that are not \emph{defined} are \emph{non-conforming programs}.
\end{quote}
\glossarydefend

\glossaryterm{class type}
\glossarydefstart
For C++: \emph{Variables} declared with one of the \code{class}, \code{struct}, or \code{union} keywords
\glossarydefend

\glossaryterm{sequentially consistent atomic construct}
\glossarydefstart
An \code{atomic} construct for which the \code{seq_cst} clause is specified.
\glossarydefend
\bigskip

\glossaryterm{non-sequentially consistent atomic construct}
\glossarydefstart
An \code{atomic} construct for which the \code{seq_cst} clause is not specified
\glossarydefend
\bigskip
\bigskip
\bigskip





\subsection{Implementation Terminology}
\index{implementation terminology}
\label{subsec:Implementation Terminology}
\glossaryterm{supporting \emph{n} levels of parallelism}
\glossarydefstart
Implies allowing an \emph{active parallel region} to be enclosed by \emph{n-1} \emph{active parallel
regions}.
\glossarydefend

\glossaryterm{supporting the OpenMP API}
\glossarydefstart
Supporting at least one level of parallelism.
\glossarydefend
\bigskip

\glossaryterm{supporting nested parallelism}
\glossarydefstart
Supporting more than one level of parallelism.
\glossarydefend
\bigskip

\glossaryterm{internal control variable}
\glossarydefstart
A conceptual variable that specifies runtime behavior of a set of \emph{threads} or \emph{tasks}
in an \emph{OpenMP program}.

\begin{quote}
COMMENT: The acronym ICV is used interchangeably with the term \emph{internal
control variable} in the remainder of this specification.
\end{quote}
\glossarydefend

\glossaryterm{compliant implementation}
\glossarydefstart
An implementation of the OpenMP specification that compiles and executes any
\emph{conforming program} as defined by the specification.

\begin{quote}
COMMENT: A \emph{compliant implementation} may exhibit \emph{unspecified behavior} when
compiling or executing a \emph{non-conforming program}.
\end{quote}
\glossarydefend

\glossaryterm{unspecified behavior}
\glossarydefstart
A behavior or result that is not specified by the OpenMP specification or not
known prior to the compilation or execution of an \emph{OpenMP program}.

Such \emph{unspecified behavior} may result from:

\begin{itemize}
\item Issues documented by the OpenMP specification as having \emph{unspecified
behavior}.

\item A \emph{non-conforming program}.

\item A \emph{conforming program} exhibiting an \emph{implementation defined} behavior.
\end{itemize}
\glossarydefend

\glossaryterm{implementation defined}
\glossarydefstart
Behavior that must be documented by the implementation, and is allowed to vary
among different \emph{compliant implementations}. An implementation is allowed to
define this behavior as \emph{unspecified}.

\begin{quote}
COMMENT: All features that have \emph{implementation defined} behavior
are documented in Appendix~\ref{chap:OpenMP Implementation-Defined Behaviors}.
\end{quote}
\glossarydefend

\glossaryterm{deprecated}
\glossarydefstart
Implies a construct, clause, or other feature is normative in the current specification but is considered obsolescent and will be removed in the future.
\glossarydefend

%\newpage %% HACK
\subsection{Tool Terminology}

\glossaryterm{tool}
\glossarydefstart
Executable code, distinct from application or runtime code, that can observe and/or modify the execution of an application.
\glossarydefend

\glossaryterm{first-party tool}
\glossarydefstart
A tool that executes in the address space of the program it is monitoring.
\glossarydefend

\glossaryterm{third-party tool}
\glossarydefstart
A tool that executes as a separate process from that which it is monitoring and potentially controlling.
\glossarydefend

\glossaryterm{activated tool}
\glossarydefstart
A first-party tool that successfully completed its initialization.
\glossarydefend

\glossaryterm{event}
\glossarydefstart
A point of interest in the execution of a thread where the condition
defining that event is true.
\glossarydefend

\glossaryterm{tool callback}
\glossarydefstart
A function provided by a tool to an OpenMP implementation that can be invoked when needed.
\glossarydefend

\glossaryterm{registering a callback}
\glossarydefstart
Providing a callback function to an OpenMP implementation for a particular purpose.
\glossarydefend

\glossaryterm{dispatching a callback at an event}
\glossarydefstart
Processing a callback when an associated event occurs in a manner consistent with the return code
provided when a \emph{first-party} tool registered the callback.
\glossarydefend

\glossaryterm{thread state}
\glossarydefstart
An enumeration type that describes what an OpenMP thread is currently doing.
A thread can be in only one state at any time.
\glossarydefend

\glossaryterm{wait identifier}
\glossarydefstart
A unique opaque handle associated with each data object (e.g., a lock) used by the OpenMP runtime
to enforce mutual exclusion that may cause a thread to wait actively or passively.
\glossarydefend

\glossaryterm{frame}
\glossarydefstart
A storage area on a thread's stack associated with a procedure invocation. A frame includes space for
one or more saved registers and often also includes space for saved arguments, local variables,
and padding for alignment.
\glossarydefend

\glossaryterm{canonical frame address}
\glossarydefstart
An address associated with a procedure \emph{frame} on a call stack defined as the value of the stack pointer immediately prior
to calling the procedure whose invocation the frame represents.
\glossarydefend

\glossaryterm{runtime entry point}
\glossarydefstart
A function interface provided by an OpenMP runtime for use by a tool. A runtime entry point is
typically not associated with a global function symbol.
\glossarydefend

\glossaryterm{trace record}
\glossarydefstart
A data structure to store information associated with an occurrence of an \emph{event}.
\glossarydefend

\glossaryterm{native trace record}
\glossarydefstart
A \emph{trace record} for an OpenMP device that is in a device-specific format.
\glossarydefend

\glossaryterm{signal}
\glossarydefstart
A software interrupt delivered to a thread.
\glossarydefend

\glossaryterm{signal handler}
\glossarydefstart
A function called asynchronously when a \emph{signal} is delivered to a thread.
\glossarydefend

\glossaryterm{async signal safe}
\glossarydefstart
Guaranteed not to interfere with operations that are being interrupted by \emph{signal} delivery.
An async signal safe \emph{runtime entry point} is safe to call from a \emph{signal handler}.
\glossarydefend

\glossaryterm{code block}
\glossarydefstart
A contiguous region of memory that contains code of an OpenMP program to be executed on a device.
\glossarydefend


\glossaryterm{OMPT}
\glossarydefstart
An interface that helps a first-party tool monitor the execution of an OpenMP program.
\glossarydefend

\glossaryterm{OMPD}
\glossarydefstart
An interface that helps a third-party tool inspect the OpenMP state of a program that has begun execution.
\glossarydefend

\glossaryterm{OMPD library}
\glossarydefstart
A shared library that implements the OMPD interface.
\glossarydefend

\glossaryterm{image file}
\glossarydefstart
An executable or shared library.
\glossarydefend

% ilaguna: not sure if we need this one
%\paragraph{thread}
%A thread is an execution entity running within a specific address
%space within a process.

\glossaryterm{address space}
\glossarydefstart
%An address space is a collection of logical, virtual, or physical memory address ranges containing
%code, stack, and data. The memory address ranges within an address space need not be
%contiguous.  An address space may be segmented, where a segmented address consists of a
%segment identifier and an address in that segment. An address space has associated with it a
%collection of image files that have been loaded into it. For example, an OpenMP program running
%on a system with GPUs may consist of multiple address spaces: one for the host program and one
%for each GPU device. In practical terms, on such systems an OpenMP \emph{device} may be
%implemented as a CUDA context, which \emph{is} an address space into which CUDA image files
%are loaded and CUDA kernels are launched.
A collection of logical, virtual, or physical memory address ranges containing code, stack, and/or data.
Address ranges within an address space need not be contiguous.  An address space consists of one or more \emph{segments}.
% An address space has associated with it a collection of image files that have been loaded into it.
\glossarydefend

\glossaryterm{segment}
\glossarydefstart
A region of an address space associated with a set of address ranges.
\glossarydefend

\begin{comment}
\glossaryterm{architecture}
\glossarydefstart
%A target architecture is defined by the processor (CPU or GPU) and the Application Binary
%Interface (ABI) used by threads and address spaces. A process may contain threads and address
%spaces for multiple target architectures. For example, a process may contain a host address
%space
%and threads for an x86\_64, 64-bit CPU architecture, along with accelerator address spaces and
%threads for an NVIDIA\textsuperscript{\textregistered} GPU architecture or for an
%Intel\textsuperscript{\textregistered} Xeon
%Phi\textsuperscript{\texttrademark} architecture.
A combination of the processor and the Application Binary Interface (ABI) used by
threads and address spaces.
\glossarydefend
\end{comment}

\glossaryterm{OpenMP architecture}
\glossarydefstart
The architecture on which an OpenMP region executes.
\glossarydefend

\glossaryterm{tool architecture}
\glossarydefstart
The architecture on which an OMPD tool executes.
\glossarydefend

\glossaryterm{OpenMP process}
\glossarydefstart
A collection of one or more threads and address spaces. A process may contain threads and address spaces for multiple OpenMP architectures.
At least one thread in an OpenMP process is an OpenMP thread.
%The collection may be homogeneous or heterogeneous, containing, for example, threads or
%address spaces from host programs or accelerator devices.
A process may be live or a core file.
\glossarydefend

\glossaryterm{handle}
\glossarydefstart
%OMPD handles identify OpenMP entities during the execution of an OpenMP program. Handles
%are opaque to the debugger, and defined internally by the OMPD implementation. Below we
%define these handles and the conditions under which they are guaranteed to be valid.
An opaque reference provided by an OMPD library implementation to a using tool. A handle uniquely identifies an abstraction.

% An identifier for an OpenMP entity that is valid during the execution of an OpenMP program. OMPD  handles are created by an OMPD library upon requests from a third-party tool and are opaque to  the third-party tool. Handles remain valid until they are released by third-party tools.
\glossarydefend

\glossaryterm{address space handle}
\glossarydefstart
%A tool handle \emph{address space handle} identifies a portion of an instance of
%an \emph{OpenMP program} that is running on a host device or a target
%device.
%The host address space handle is allocated and initialized with the per
%process or core file initialization
%call to \refdef{\texttt{ompd\_process\_initialize}}{process-initialize:def}.
%A process or core file is initialized by passing the host address
%space context to that function to obtain an address space handle for
%the process or core file.
%The handle remains valid until it is released by the debugger.
%The handle is created by the OMPD implementation, which passes ownership
%to the debugger which is responsible for indicating when it no longer
%needs the handle.
%The debugger releases the handle when it calls
%\refdef{\texttt{ompd\_release\_address\_space\_handle}}{release-address-space-handle:def}.
%The OMPD implementation can use the handle to cache invariant
%address-space-specific data (e.g., symbol addresses), and to retain a copy of
%the debugger's address space context pointer.
%The handle is passed into subsequent API function calls.
%In the OMPD API, an address space handle is represented by the opaque type
%\texttt{ompd\_address\_space\_handle\_t}.
%\emph{Future versions of this API will support address space handles
%	for target devices, which will be allocated and initialized by various
%	OMPD API calls.}
A handle that refers to an address space within an OpenMP process.
%a portion of an instance of an OpenMP program that is running on a host or a target device.
\glossarydefend

\glossaryterm{thread handle}
\glossarydefstart
%The \emph{thread handle} identifies an \emph{OpenMP thread}.
%Thread handles are allocated and initialized by various OMPD API calls.
%A handle is valid for the life time of the corresponding system thread.
%Thread handles are represented by \texttt{ompd\_thread\_handle\_t},
%and created by the OMPD implementation which passes ownership to the
%debugger which is responsible for indicating when it no longer
%needs the handle.
%The debugger releases the thread handle by calling
%\refdef{\texttt{ompd\_release\_thread\_handle}}{release-thread-handle:def}.
A handle that refers to an OpenMP thread.
%A handle is valid for the lifetime of the system thread corresponding to the OpenMP thread.
\glossarydefend

\glossaryterm{parallel handle}
\glossarydefstart
%The \emph{parallel handle} identifies an \emph{OpenMP parallel region}.
%It is allocated and initialized by various OMPD API calls.
%The handle is valid for the life time of the parallel region.
%The handle is guaranteed to be valid if at least one thread
%in the parallel region is paused, or if a thread in a nested
%parallel region is paused.
%Parallel handles are represented by the opaque type
%\texttt{ompd\_parallel\_handle\_t}, and created by the OMPD implementation
%which passes ownership to the debugger which is responsible for
%indicating when it no longer needs the handle.
%The debugger releases the parallel handle by calling
%\refdef{\texttt{ompd\_release\_parallel\_handle}}{release-parallel-handle:def}.
A handle that refers to an OpenMP parallel region.
%The handle is guaranteed to be valid if at least one thread in the parallel region is paused, or if a thread in a nested parallel region is  paused.
\glossarydefend

\glossaryterm{task handle}
\glossarydefstart
%The \emph{task handle} identifies an \emph{OpenMP task region}.
%It is allocated and initialized by various OMPD API calls.
%The handle is valid for the life time of the task region.
%The handle is guaranteed to be valid if all threads
%in the task team are paused.
%Task handles are represented by the opaque type
%\texttt{ompd\_task\_handle\_t}, and created by the OMPD implementation
%which passes ownership to the debugger which is responsible for
%indicating when it no longer needs the handle.
%The debugger releases the task handle by calling
%\refdef{\texttt{ompd\_release\_task\_handle}}{release-task-handle:def}.
A handle that refers to an OpenMP task region.
% The handle is valid for the life time of the  task region. The handle is guaranteed to be valid if all threads in the task team are paused.
\glossarydefend

\glossaryterm{tool context}
\glossarydefstart
%Debugger contexts are used to identify a process, address space, or thread
%object in the debugger. Contexts are passed from the debugger into various OMPD API calls,
%and then from the OMPD implementation back to the debugger's callback functions.
%For example, symbol lookup and memory accesses are done in the ``context''
%of a particular address space and possibly thread in the debugger.
%Contexts are opaque to the OMPD implementation, and defined by the debugger.
An opaque reference provided by a tool to an OMPD library implementation. A tool context uniquely identifies an abstraction.
% a process, address space, or thread that resides in a third-party tool. Contexts are opaque to an OMPD library, and are defined by a third-party tool.
\glossarydefend

\glossaryterm{address space context}
\glossarydefstart
%The \emph{address space context} identifies the debugger object for a
%portion of an instance of an \emph{OpenMP program} that is running on
%a host or target device.
%An address space is contained within a process, and has an associated
%target architecture.
%The address space context must be valid for the life time of its
%associated address space handle.
%The host address space context is passed into the process initialization call
%\refdef{\texttt{ompd\_process\_initialize}}{process-initialize:def}
%to associate the host address space context with the address space handle.
%The OMPD implementation can assume that the address space context is valid
%until
%\refdef{\texttt{ompd\_release\_address\_space\_handle}}{release-address-space-handle:def}
%is called for the address space context passed into the initialization routine.
A tool context that refers to an address space within a process.
\glossarydefend

\glossaryterm{thread context}
\glossarydefstart
%The \emph{thread context} identifies the debugger object for a thread.
%The debugger owns and initializes the thread context.
%The OMPD implementation obtains a thread context using the
%\texttt{get\_thread\_context} callback.
%This callback allows the OMPD implementation to map an operating
%system thread ID to a debugger thread context.
%The OMPD implementation can assume that the thread context is valid
%for as long as the debugger is holding any references to thread handles
%that may contain the thread context.
A tool context that refers to a thread.
\glossarydefend

\glossaryterm{thread identifier}
\glossarydefstart
An identifier for a native thread defined by a thread implementation.
%on a device.
\glossarydefend

