% This is an included file. See the master file for more information.
%
% When editing this file:
%
%    1. To change formatting, appearance, or style, please edit openmp.sty.
%
%    2. Custom commands and macros are defined in openmp.sty.
%
%    3. Be kind to other editors -- keep a consistent style by copying-and-pasting to
%       create new content.
%
%    4. We use semantic markup, e.g. (see openmp.sty for a full list):
%         \code{}     % for bold monospace keywords, code, operators, etc.
%         \plc{}      % for italic placeholder names, grammar, etc.
%
%    5. There are environments that provide special formatting, e.g. language bars.
%       Please use them whereever appropriate.  Examples are:
%
%         \begin{fortranspecific}
%         This is text that appears enclosed in blue language bars for Fortran.
%         \end{fortranspecific}
%
%         \begin{note}
%         This is a note.  The "Note -- " header appears automatically.
%         \end{note}
%
%    6. Other recommendations:
%         Use the convenience macros defined in openmp.sty for the minor headers
%         such as Comments, Syntax, etc.
%
%         To keep items together on the same page, prefer the use of 
%         \begin{samepage}.... Avoid \parbox for text blocks as it interrupts line numbering.
%         When possible, avoid \filbreak, \pagebreak, \newpage, \clearpage unless that's
%         what you mean. Use \needspace{} cautiously for troublesome paragraphs.
%
%         Avoid absolute lengths and measures in this file; use relative units when possible.
%         Vertical space can be relative to \baselineskip or ex units. Horizontal space
%         can be relative to \linewidth or em units.
%
%         Prefer \emph{} to italicize terminology, e.g.:
%             This is a \emph{definition}, not a placeholder.
%             This is a \plc{var-name}.
%

\section{Memory Model}
\label{sec:Memory Model}
\index{memory model}
\index{modification order}
\subsection{Structure of the OpenMP Memory Model}
\label{subsec:Structure of the OpenMP Memory Model}
The OpenMP API provides a relaxed-consistency, shared-memory model. All OpenMP
threads have access to a place to store and to retrieve variables, 
called the \emph{memory}. In
addition, each thread is allowed to have its own \emph{temporary view} of the memory. The
temporary view of memory for each thread is not a required part of the OpenMP
memory model, but can represent any kind of intervening structure, such as machine
registers, cache, or other local storage, between the thread and the memory. The
temporary view of memory allows the thread to cache variables and thereby to avoid
going to memory for every reference to a variable. Each thread also has access to
another type of memory that must not be accessed by other threads, 
called \emph{threadprivate memory}.

A directive that accepts data-sharing attribute clauses determines two kinds of access to
variables used in the directive's associated structured block: shared and private. Each
variable referenced in the structured block has an original variable, which is the variable  
by the same name that exists in the program immediately outside the construct. Each
reference to a shared variable in the structured block becomes a reference to the original 
variable. For each private variable referenced in the structured block, a new version of
the original variable (of the same type and size) is created in memory for each task or
SIMD lane that contains code associated with the directive. Creation of the new version
does not alter the value of the original variable. However, the impact of attempts to
access the original variable during the region associated with the directive is
unspecified; see \specref{subsubsec:private clause} for additional details. References to a
private variable in the structured block refer to the private version of the original
variable for the current task or SIMD lane. The relationship between the value of the
original variable and the initial or final value of the private version depends on the exact
clause that specifies it. Details of this issue, as well as other issues with privatization,
are provided in \specref{sec:Data Environment}.

The minimum size at which a memory update may also read and write back adjacent 
variables that are part of another variable (as array or structure elements) is
implementation defined but is no larger than required by the base language. 

A single access to a variable may be implemented with multiple load or store
instructions, and hence is not guaranteed to be atomic with respect to other accesses to
the same variable. Accesses to variables smaller than the implementation defined
minimum size or to C or C++ bit-fields may be implemented by reading, modifying, and 
rewriting a larger unit of memory, and may thus interfere with updates of variables or
fields in the same unit of memory.

If multiple threads write without synchronization to the same memory unit, including
cases due to atomicity considerations as described above, then a data race occurs.
Similarly, if at least one thread reads from a memory unit and at least one thread writes
without synchronization to that same memory unit, including cases due to atomicity
considerations as described above, then a data race occurs. If a data race occurs then the
result of the program is unspecified.

Every variable has an associated \emph{modification order}, defined as a
sequential ordering of all operations that store a value into the variable
without causing a data race to occur. 

A private variable in a task region that eventually generates an inner nested \code{parallel}
region is permitted to be made shared by implicit tasks in the inner \code{parallel} region.
A private variable in a task region can be shared by an explicit task region generated
during its execution. However, it is the programmer's responsibility to ensure through
synchronization that the lifetime of the variable does not end before completion of the
explicit task region sharing it. Any other access by one task to the
private variables of another task results in unspecified behavior.




\subsection{Device Data Environments}
\label{subsec:Device Data Environments}
\index{device data environments}
When an OpenMP program begins, an implicit \code{target}~\code{data} region for each device surrounds the whole program. Each device has a device data environment that is defined by its implicit \code{target}~\code{data} region. Any \code{declare}~\code{target} directives and the directives that accept data-mapping attribute clauses determine how an original variable in a data environment is mapped to a corresponding variable in a device data environment.

When an original variable is mapped to a device data environment and the associated corresponding variable is not present in the device data environment, a new corresponding variable (of the same type and size as the original variable) is created in the device data environment. The initial value of the new corresponding variable is determined from the clauses and the data environment of the encountering thread.

The corresponding variable in the device data environment may share storage with the
original variable. Writes to the corresponding variable may alter the value of the original
variable. The impact of this on memory consistency is discussed in 
\specref{subsec:OpenMP Memory Consistency}. 
When a task executes in the context of a device data environment, references to  
the original variable refer to the corresponding variable in the device data environment.
If a corresponding variable does not exist in the device data
environment then accesses to the original variable result in
unspecified behavior unless the unified\_shared\_memory requirement
is specified.

The relationship between the value of the original variable and the initial or final value
of the corresponding variable depends on the \plc{map-type}. Details of this issue, as well as
other issues with mapping a variable, are provided in \specref{subsec:map Clause}.

The original variable in a data environment and the corresponding variable(s) in one or
more device data environments may share storage. Without intervening synchronization  
data races can occur.

\subsection{Memory Management}

The host device, and target devices that an implementation may support, have attached storage resources where program variables are stored. These resources can be of different kinds. 

An OpenMP program can use a memory allocator to allocate storage for its variables. Memory allocators are associated with certain storage resources and use that storage to allocate variables. Memory allocators are also used to deallocate variables and free the storage in the resources. When an OpenMP memory allocator is not used variables can be allocated in any storage resource. 
%The behavior of a memory management construct, clause or API is unspecified if the variable that is applied to was not allocated with an OpenMP memory allocator.


\subsection{The Flush Operation}
\label{subsec:The Flush Operation}
\index{flush operation}
\index{flush-set}
\index{sync-set}

The memory model has relaxed-consistency because a thread's temporary view of
memory is not required to be consistent with memory at all times. A value written to a
variable can remain in the thread's temporary view until it is forced to memory at a later
time. Likewise, a read from a variable may retrieve the value from the thread's
temporary view, unless it is forced to read from memory. The OpenMP flush operation
enforces consistency between the temporary view and memory.

The flush operation is applied to a set of variables called the
\emph{flush-set}. It also has an associated \emph{sync-set}
containing a set of synchronization variables that may be used to force
synchronization between the flush operation and another flush operation.  A flush operation
must be a write flush, a read flush, or both a write flush and a
read flush; accordingly, it restricts reordering of memory operations
performed by the same thread that an implementation might otherwise do as
follows:

\begin{itemize}
\item With respect to a write flush, an implementation must not reorder any
    prior memory operation in the thread's program order that reads or writes
    to a variable in its flush-set or any subsequent memory operation in the
    thread's program order that writes to a variable in its sync-set.

\item With respect to a read flush, an implementation must not reorder any
    subsequent memory operation in the thread's program order that reads or
    writes to a variable in its flush-set or any prior memory operation in the
    thread's program order that reads from a variable in its sync-set.

\item With respect to a read flush, an implementation must not reorder any
    subsequent write flush in the thread's program order that is applied to
    the same variable.
\end{itemize}

A strong flush imposes stronger ordering requirements across threads
compared to a weak flush, as described in
\specref{subsec:OpenMP Memory Consistency}.

If a thread has performed a write to its temporary view of a shared variable
since its last write flush of that variable, then when it executes another
write flush of the variable, the flush does not complete until the value of
the variable has been written to the variable in memory.  If a thread performs
multiple writes to the same variable between two write flushes of that
variable, the write flush ensures that it is the value of the last write is
written to the variable in memory. The completion of a write flush of a set of
variables executed by a thread is defined as the point at which all writes to
those variables performed by the thread before the flush are visible in
memory.

If a thread has not performed a write to its temporary view of a shared
variable since its last write flush of that variable, a read flush of the
variable executed by a thread causes its temporary view of the variable to be
discarded. If its next memory operation following the read flush for that
variable is a read, then the thread will read from memory when it may again
capture the value in the temporary view. When a thread executes a read flush,
no later memory operation by that thread for a variable involved in that flush
is allowed to start until the flush completes. The completion of a read flush
of a set of variables executed by a thread is defined as the point at which
the thread's temporary view of all variables involved is discarded.

Flush operations enable a program to provide a guarantee of consistency
between a thread's temporary view and memory. Therefore, the flush operation can
be used to guarantee that a value written to a variable by one thread may be
read by a second thread. To accomplish this, it is sufficient for the
programmer to ensure that the second thread has not written to the variable
since its last flush of the variable, and that the following sequence of
events are completed in the specified order according to the completion order
guarantees defined in \specref{subsec:OpenMP Memory Consistency}: 

\begin{enumerate}
\item The value is written to the variable by the first thread.   

\item The variable is flushed, with a write flush, by the first thread.    

\item The variable is flushed, with a read flush, by the second thread.

\item The value is read from the variable by the second thread.
\end{enumerate}

\begin{note}
OpenMP synchronization operations, described in 
\specref{sec:Master and Synchronization Constructs and Clauses} and in \specref{sec:Lock Routines}, 
are recommended for enforcing this order. Synchronization 
through variables, described in \specref{subsec:happens-before}, is possible
but is not recommended because the proper timing of flushes is difficult.
\end{note}



\subsection{Flush Synchronization and Happens Before}
\label{subsec:happens-before}
\index{flush synchronization}
\index{happens before}

A release flush is a write flush that is synchronizable, and likewise an
acquire flush is a read flush that is synchronizable.  A release flush may
synchronize with an acquire flush using a variable that is in the sync-sets of
both flushes.

For every variable in the sync-set of a release flush, there exists a
\emph{release flush sequence} that is a sequence of consecutive modifications
to the variable taken from its modification order. It is defined to be the
maximal such sequence that is headed by a modification that follows the
release flush in its thread's program order and contains only subsequent
modifications performed on the same thread or read-modify-write atomic
modifications performed by a different thread.  A release flush synchronizes
with an acquire flush on a different thread if there exists a variable in the
sync-sets of both flushes and the value written to it by some modification in
a release flush sequence associated with the release flush is read by an
access to the variable preceding the acquire flush in its thread's
program order.

\newpage %% HACK
\begin{note}
The conditions under which a release flush synchronizes with an acquire flush
are intended to match the conditions under which release operations
synchronize with acquire operations in C++11 and C11. A read-modify-write atomic
modification includes any atomic operation specified with an \code{atomic}
construct on which neither the \code{read} or \code{write} clause appears.
\end{note}

An operation \plc{X} \emph{happens~before} an operation \plc{Y} if any of the following conditions are satisfied:
\begin{enumerate}
\item \plc{X} happens before \plc{Y} according to the base language's definition of happens before, if such a definition exists.
\item \plc{X} and \plc{Y} are performed by the same thread, and \plc{X} precedes \plc{Y} in the thread's program order.
\item \plc{X} is a release flush, \plc{Y} is an acquire flush, and \plc{X} synchronizes with \plc{Y} according to the flush synchronization
conditions explained above.
\item There exists another operation \plc{Z}, such that \plc{X} happens before \plc{Z} and \plc{Z} happens before \plc{Y}.
\end{enumerate}

A variable with an initial value is treated as if the value is stored to the
variable by an operation that happens before all operations that access or
modify the variable in the program.

\subsection{OpenMP Memory Consistency}
\label{subsec:OpenMP Memory Consistency}
Given the conditions in \specref{subsec:happens-before} for flush
synchronization and happens before, OpenMP guarantees the following in
accordance with the restrictions in \specref{subsec:The Flush Operation} on
reordering with respect to flush operations:

\begin{itemize}
\item If the intersection of the flush-sets of two strong flushes
    performed by two different threads is non-empty, then the two flushes must
    be completed as if in some sequential order, seen by all threads.

\item If two operations performed by the same thread access, modify, or, with
    a strong flush, flush the same variable, then they must be
    completed as if in that thread's program order, as seen by all threads. 

\item If two operations access, modify, or flush the same variable and one
    happens before the other, then they must be completed in that happens
    before order, as seen by the thread or threads that perform the
    operations.

\item Any two atomic memory operations from different \code{atomic} regions
    must be completed as if in the same order as the strong flushes
    implied in their respective regions, as seen by all threads.
    \end{itemize}

The flush operation can be specified using the \code{flush} directive, and is also implied at 
various locations in an OpenMP program: see \specref{subsec:flush Construct} for details.

\begin{note}
Since flush operations by themselves cannot prevent data races, explicit flush 
operations are only useful in combination with non-sequentially consistent atomic 
directives.
\end{note}

OpenMP programs that:

\begin{itemize}[rightmargin=11ex]
\item do not use non-sequentially consistent atomic directives,

\item do not rely on the accuracy of a \plc{false} result from 
\code{omp\_test\_lock} and \code{omp\_test\_nest\_lock}, and

\item correctly avoid data races as required in \specref{subsec:Structure of the OpenMP Memory Model} 
\end{itemize}

behave as though operations on shared variables were simply interleaved in an order 
consistent with the order in which they are performed by each thread. The relaxed 
consistency model is invisible for such programs, and any explicit flush operations in 
such programs are redundant.

Implementations are allowed to relax the ordering imposed by implicit flush operations 
when the result is only visible to programs using non-sequentially consistent atomic 
directives.

