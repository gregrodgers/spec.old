% This is an included file. See the master file for more information.
%
% When editing this file:
%
%    1. To change formatting, appearance, or style, please edit openmp.sty.
%
%    2. Custom commands and macros are defined in openmp.sty.
%
%    3. Be kind to other editors -- keep a consistent style by copying-and-pasting to
%       create new content.
%
%    4. We use semantic markup, e.g. (see openmp.sty for a full list):
%         \code{}     % for bold monospace keywords, code, operators, etc.
%         \plc{}      % for italic placeholder names, grammar, etc.
%
%    5. There are environments that provide special formatting, e.g. language bars.
%       Please use them whereever appropriate.  Examples are:
%
%         \begin{fortranspecific}
%         This is text that appears enclosed in blue language bars for Fortran.
%         \end{fortranspecific}
%
%         \begin{note}
%         This is a note.  The "Note -- " header appears automatically.
%         \end{note}
%
%    6. Other recommendations:
%         Use the convenience macros defined in openmp.sty for the minor headers
%         such as Comments, Syntax, etc.
%
%         To keep items together on the same page, prefer the use of
%         \begin{samepage}.... Avoid \parbox for text blocks as it interrupts line numbering.
%         When possible, avoid \filbreak, \pagebreak, \newpage, \clearpage unless that's
%         what you mean. Use \needspace{} cautiously for troublesome paragraphs.
%
%         Avoid absolute lengths and measures in this file; use relative units when possible.
%         Vertical space can be relative to \baselineskip or ex units. Horizontal space
%         can be relative to \linewidth or em units.
%
%         Prefer \emph{} to italicize terminology, e.g.:
%             This is a \emph{definition}, not a placeholder.
%             This is a \plc{var-name}.
%

\section{Memory Model}
\label{sec:Memory Model}
\index{memory model}
\subsection{Structure of the OpenMP Memory Model}
\label{subsec:Structure of the OpenMP Memory Model}
The OpenMP API provides a relaxed-consistency, shared-memory model. All OpenMP
threads have access to a place to store and to retrieve variables,
called the \emph{memory}. In addition, each thread is allowed to have 
its own \emph{temporary view} of the memory. The temporary view of memory 
for each thread is not a required part of the OpenMP memory model, but 
can represent any kind of intervening structure, such as machine
registers, cache, or other local storage, between the thread and 
the memory. The temporary view of memory allows the thread to cache 
variables and thereby to avoid going to memory for every reference 
to a variable. Each thread also has access to another type of memory 
that must not be accessed by other threads, called \emph{threadprivate memory}.

A directive that accepts data-sharing attribute clauses determines 
two kinds of access to variables used in the directive's associated 
structured block: shared and private. Each variable referenced in 
the structured block has an original variable, which is the variable
by the same name that exists in the program immediately outside the 
construct. Each reference to a shared variable in the structured block 
becomes a reference to the original variable. For each private variable 
referenced in the structured block, a new version of the original 
variable (of the same type and size) is created in memory for each task or
SIMD lane that contains code associated with the directive. Creation of 
the new version does not alter the value of the original variable. 
However, the impact of attempts to access the original variable during 
the region corresponding to the directive is unspecified; see 
\specref{subsubsec:private clause} for additional details. References to a
private variable in the structured block refer to the private version 
of the original variable for the current task or SIMD lane. The 
relationship between the value of the original variable and the 
initial or final value of the private version depends on the exact
clause that specifies it. Details of this issue, as well as other 
issues with privatization, are provided in \specref{sec:Data Environment}.

The minimum size at which a memory update may also read and write back adjacent
variables that are part of another variable (as array or structure elements) is
implementation defined but is no larger than required by the base language.

A single access to a variable may be implemented with multiple load or store
instructions and, thus, is not guaranteed to be atomic with respect to other 
accesses to the same variable. Accesses to variables smaller than the 
implementation defined minimum size or to C or C++ bit-fields may be 
implemented by reading, modifying, and rewriting a larger unit of memory, 
and may thus interfere with updates of variables or fields in the same 
unit of memory.

If multiple threads write without synchronization to the same memory 
unit, including cases due to atomicity considerations as described above, 
then a data race occurs. Similarly, if at least one thread reads from a 
memory unit and at least one thread writes without synchronization to 
that same memory unit, including cases due to atomicity considerations 
as described above, then a data race occurs. If a data race occurs then the
result of the program is unspecified.

A private variable in a task region that subsequently generates an inner 
nested \code{parallel} region is permitted to be made shared by implicit 
tasks in the inner \code{parallel} region. A private variable in a task 
region can also be shared by an explicit task region generated during its 
execution. However, it is the programmer's responsibility to ensure through
synchronization that the lifetime of the variable does not end before 
completion of the explicit task region sharing it. Any other access by 
one task to the private variables of another task results in unspecified 
behavior.



\subsection{Device Data Environments}
\label{subsec:Device Data Environments}
\index{device data environments}
When an OpenMP program begins, an implicit \code{target}~\code{data} 
region for each device surrounds the whole program. Each device has 
a device data environment that is defined by its implicit 
\code{target}~\code{data} region. Any \code{declare}~\code{target} 
directives and the directives that accept data-mapping attribute 
clauses determine how an original variable in a data environment 
is mapped to a corresponding variable in a device data environment.

When an original variable is mapped to a device data environment 
and the corresponding variable is not present in the device data 
environment, a new corresponding variable (of the same type and 
size as the original variable) is created in the device data 
environment. Conversely the original variable becomes the new 
variable's corresponding variable in the device data environment 
of the device that performs the mapping operation.

The corresponding variable in the device data environment may share 
storage with the original variable. Writes to the corresponding 
variable may alter the value of the original variable. The impact 
of this possibility on memory consistency is discussed in 
\specref{subsec:OpenMP Memory Consistency}. When a task executes 
in the context of a device data environment, references to the 
original variable refer to the corresponding variable in the device 
data environment. If an original variable has not been mapped and a 
corresponding variable does not exist in the device data environment 
then accesses to the original variable result in unspecified behavior 
unless the \code{unified_shared_memory} clause is specified on a 
\code{requires} directive for the compilation unit.

The relationship between the value of the original variable and the 
initial or final value of the corresponding variable depends on the 
\plc{map-type}. Details of this issue, as well as other issues with 
mapping a variable, are provided in \specref{subsec:map Clause}.

The original variable in a data environment and the corresponding 
variable(s) in one or more device data environments may share storage. 
Without intervening synchronization data races can occur.

\subsection{Memory Management}
The host device, and target devices that an implementation may support, 
have attached storage resources where program variables are stored. 
These resources can have different traits. A memory space in an OpenMP 
program represents a set of these storage resources. Memory spaces are 
defined according to a set of traits, and a single resource may be exposed 
as multiple memory spaces with different traits or may be part of multiple 
memory spaces. In any device, at least one memory space is guaranteed to exist.

An OpenMP program can use a \emph{memory allocator} to allocate \emph{memory} 
in which to store variables. This \emph{memory} will be allocated from the 
storage resources of the \emph{memory space} associated with the memory 
allocator. Memory allocators are also used to deallocate previously allocated
\emph{memory}. When an OpenMP memory allocator is not used to allocate memory,
OpenMP does not prescribe the storage resource for the allocation; the memory 
for the variables may be allocated in any storage resource.



\subsection{The Flush Operation}
\label{subsec:The Flush Operation}
\index{flush operation}
\index{strong flush}
\index{flush-set}


The memory model has relaxed-consistency because a thread's temporary view of
memory is not required to be consistent with memory at all times. A value
written to a variable can remain in the thread's temporary view until it is
forced to memory at a later time. Likewise, a read from a variable may
retrieve the value from the thread's temporary view, unless it is forced to
read from memory.  The OpenMP flush operations enforce consistency between 
a thread's temporary view of memory and memory, as well as between multiple 
threads' view of memory.

If a flush operation is a strong flush, it enforces consistency between a
thread's temporary view and memory.  A strong flush operation is applied to a
set of variables called the \emph{flush-set}. A strong flush restricts
reordering of memory operations that an implementation might otherwise do.
Implementations must not reorder the code for a memory operation for a given
variable, or the code for a flush operation for the variable, with respect to
a strong flush operation that refers to the same variable.

If a thread has performed a write to its temporary view of a shared variable
since its last strong flush of that variable, then when it executes another
strong flush of the variable, the strong flush does not complete until the
value of the variable has been written to the variable in memory. If a thread
performs multiple writes to the same variable between two strong flushes of
that variable, the strong flush ensures that the value of the last write is
written to the variable in memory. A strong flush of a variable executed by a
thread also causes its temporary view of the variable to be discarded, so that
if its next memory operation for that variable is a read, then the thread will
read from memory and capture the value in its temporary view.
When a thread executes a strong flush, no later memory operation by that
thread for a variable involved in that strong flush is allowed to start until
the strong flush completes.  The completion of a strong flush 
executed by a thread is defined as the point at which 
all writes to the flush-set performed by the thread before the strong 
flush are visible in memory to all other threads, and at which that 
thread's temporary view of the flush-set is discarded.

A strong flush operation provides a guarantee of consistency between a thread’s
temporary view and memory. Therefore, a strong flush can be used to
guarantee that a value written to a variable by one thread may be read by a
second thread. To accomplish this, the programmer must ensure that the second
thread has not written to the variable since its last strong flush of the
variable, and that the following sequence of events are completed in this 
specific order:

\begin{enumerate}
    \item The value is written to the variable by the first thread;
    \item The variable is flushed, with a strong flush, by the first thread;
    \item The variable is flushed, with a strong flush, by the second thread; and
    \item The value is read from the variable by the second thread.
\end{enumerate}

If a flush operation is a release flush or acquire flush, it can enforce
consistency between the views of memory of two synchronizing threads.  A release
flush guarantees that any prior operation that writes or reads a shared
variable will appear to be completed before any operation that writes or reads
the same shared variable that follows an acquire flush with which it
synchronizes (see Section \specref{subsec:happens-before} for more details on
flush synchronization).  A release flush will propagate the values of all
shared variables in its temporary view to memory prior to the thread
performing any subsequent atomic operation that may establish a
synchronization. An acquire flush will discard any value of a shared variable
in its temporary view to which the thread has not written since last performing
a release flush, so that it may subsequently read a value propagated by a
release flush that synchronizes with it.   Therefore, release and acquire
flushes may also be used to guarantee that a value written to a variable by
one thread may be read by a second thread. To accomplish this, the programmer
must ensure that the second thread has not written to the variable since its
last acquire flush, and that the following sequence of events happen in this
specific order:

\begin{enumerate}
    \item The value is written to the variable by the first thread;
    \item The first thread performs a release flush;
    \item The second thread performs an acquire flush; and
    \item The value is read from the variable by the second thread.
\end{enumerate}

\begin{note}
OpenMP synchronization operations, described in 
\specref{sec:Synchronization Constructs and Clauses} and in 
\specref{sec:Lock Routines}, are recommended for enforcing this 
order. Synchronization through variables is possible but is not 
recommended because the proper timing of flushes is difficult.
\end{note}


\subsection{Flush Synchronization and \emph{Happens Before}}
\label{subsec:happens-before}
\index{flush synchronization}
\index{release flush}
\index{acquire flush}
\index{happens before}

OpenMP supports thread synchronization with the use of release flushes and
acquire flushes. For any such synchronization, a release flush is the source
of the synchronization and an acquire flush is the sink of the
synchronization, such that the release flush \emph{synchronizes with} the
acquire flush.

A release flush has one or more associated \emph{release sequences} that
define the set of a modifications that may be used to establish a
synchronization. A release sequence starts with an atomic operation
that follows the release flush and modifies a shared variable and additionally
includes any read-modify-write atomic operations that read a value taken from
some modification in the release sequence. The following rules determine the 
atomic operation that starts an associated release sequence.

\begin{itemize}
\item If a release flush is performed on entry to an atomic operation, that
    atomic operation starts its release sequence.
\item If a release flush is performed in an implicit \code{flush} region, an 
    atomic operation that is provided by the implementation and that modifies
    an internal synchronization variable, starts its release sequence.
\item If a release flush is performed by an explicit \code{flush} region, any
    atomic operation that modifies a shared variable and follows the
    \code{flush} region in its thread's program order starts an associated
    release sequence.
\end{itemize}

An acquire flush is associated with one or more prior atomic operations that
read a shared variable and that may be used to establish a synchronization.
The associated atomic operations that may establish a synchronization are
determined as follows:

\begin{itemize}
\item If an acquire flush is performed on exit from an atomic operation, that
    atomic operation is its associated atomic operation.
\item If an acquire flush is performed in an implicit \code{flush} region, an  
    atomic operation that is provided by the implementation and that reads an 
    internal synchronization variable is its associated atomic operation.
\item If an acquire flush is performed by an explicit \code{flush} region, any
    atomic operation that reads a shared variable and precedes the
    \code{flush} region in its thread's program order is an associated atomic
    operation.
\end{itemize}

A release flush synchronizes with an acquire flush if an atomic operation
associated with the acquire flush reads a value written by a modification from
a release sequence associated with the release flush.

An operation \plc{X} \emph{simply~happens~before} an operation \plc{Y} 
if any of the following conditions are satisfied:
\begin{enumerate}
\item \plc{X} and \plc{Y} are performed by the same thread, and \plc{X} 
      precedes \plc{Y} in the thread's program order;
\item \plc{X} synchronizes with \plc{Y} according to the flush 
      synchronization conditions explained above or according
      to the base language's definition of \emph{synchronizes~with}, 
      if such a definition exists; or
\item There exists another operation \plc{Z}, such that \plc{X} simply 
      happens before \plc{Z} and \plc{Z} simply happens before \plc{Y}.
\end{enumerate}

An operation \plc{X} \emph{happens~before} an operation \plc{Y} 
if any of the following conditions are satisfied:
\begin{enumerate}
\item \plc{X} happens before \plc{Y} according to the base language's 
      definition of \emph{happens~before}, if such a definition exists; or
\item \plc{X} simply happens before \plc{Y}.
\end{enumerate}

A variable with an initial value is treated as if the value is stored to the
variable by an operation that happens before all operations that access or
modify the variable in the program.

\subsection{OpenMP Memory Consistency}
\label{subsec:OpenMP Memory Consistency}

The observable completion order of memory operations, as seen by all threads, 
is guaranteed according to the following rules:

\begin{itemize}
\item If two operations performed by different threads are sequentially
    consistent atomic operations or they are strong flushes that flush the
    same variable, then they must be completed as if in some sequential order,
    seen by all threads.
\item If two operations performed by the same thread are sequentially
    consistent atomic operations or they access, modify, or, with a strong
    flush, flush the same variable, then they must be completed as if in that
    thread's program order, as seen by all threads.
\item If two operations are performed by different threads and one happens
    before the other, then they must be completed as if in that 
    \emph{happens before} order, as seen by all threads, if:
    \begin{itemize}
        \item both operations access or modify the same variable;
        \item both operations are strong flushes that flush the same variable; or
        \item both operations are sequentially consistent atomic operations.
    \end{itemize}
\item Any two atomic memory operations from different \code{atomic} regions
    must be completed as if in the same order as the strong flushes
    implied in their respective regions, as seen by all threads.
\end{itemize}

The flush operation can be specified using the \code{flush} directive, 
and is also implied at various locations in an OpenMP program: see 
\specref{subsec:flush Construct} for details.

\begin{note}
Since flush operations by themselves cannot prevent data races, explicit 
flush operations are only useful in combination with non-sequentially 
consistent atomic directives.
\end{note}

OpenMP programs that:

\begin{itemize}[rightmargin=11ex]
\item Do not use non-sequentially consistent atomic directives;
\item Do not rely on the accuracy of a \plc{false} result from
      \code{omp_test_lock} and \code{omp_test_nest_lock}; and
\item Correctly avoid data races as required in 
      \specref{subsec:Structure of the OpenMP Memory Model},
\end{itemize}

behave as though operations on shared variables were simply interleaved 
in an order consistent with the order in which they are performed by 
each thread. The relaxed consistency model is invisible for such programs, 
and any explicit flush operations in such programs are redundant.
