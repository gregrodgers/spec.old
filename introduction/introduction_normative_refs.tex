\section{Normative References}
\index{normative references}
\label{sec:normative references}
\begin{itemize}
\item ISO/IEC 9899:1990, \textsl{Information Technology - Programming Languages - C}.

This OpenMP API specification refers to ISO/IEC 9899:1990 as C90.

\item ISO/IEC 9899:1999, \textsl{Information Technology - Programming Languages - C}.

This OpenMP API specification refers to ISO/IEC 9899:1999 as C99.

\item ISO/IEC 9899:2011, \textsl{Information Technology - Programming Languages - C}.

This OpenMP API specification refers to ISO/IEC 9899:2011 as C11.
While future versions of the OpenMP specification are expected to 
address the following features, currently their use may result in 
unspecified behavior.

\begin{itemize}
\item Threads for the C standard library
\item Thread-local storage
\item Parallel memory sequencing model
\item Atomic
\end{itemize}

\item ISO/IEC 9899:2018, \textsl{Information Technology - Programming Languages - C}.

This OpenMP API specification refers to ISO/IEC 9899:2018 as C18.

\item ISO/IEC 14882:1998, \textsl{Information Technology - Programming Languages - C++}.

This OpenMP API specification refers to ISO/IEC 14882:1998 as C++98.

\item ISO/IEC 14882:2011, \textsl{Information Technology - Programming Languages - C++}.

This OpenMP API specification refers to ISO/IEC 14882:2011 as C++11. 
While future versions of the OpenMP specification are expected to
address the following features, currently their use may result in
unspecified behavior.

\begin{itemize}
\item Concurrency
\item Data-dependency ordering: atomics and memory model
\item Thread-local storage
\item Dynamic initialization and destruction with concurrency
\item \code{<atomic>}, \code{<thread>}, \code{<mutex>},
  \code{<future>} and \code{<condition_variable>}
\end{itemize}

\item ISO/IEC 14882:2014, \textsl{Information Technology - Programming Languages - C++}.

This OpenMP API specification refers to ISO/IEC 14882:2014 as C++14. 
While future versions of the OpenMP specification are expected to
address the following features, currently their use may result in
unspecified behavior.

\begin{itemize}
\item What signal handlers can do
\end{itemize}

\item ISO/IEC 14882:2017, \textsl{Information Technology - Programming Languages - C++}.

This OpenMP API specification refers to ISO/IEC 14882:2017 as
C++17. 

\item ISO/IEC 1539:1980, \textsl{Information Technology - Programming Languages - Fortran}.

This OpenMP API specification refers to ISO/IEC 1539:1980 as Fortran 77.

\item ISO/IEC 1539:1991, \textsl{Information Technology - Programming Languages - Fortran}.

This OpenMP API specification refers to ISO/IEC 1539:1991 as Fortran 90.

\item ISO/IEC 1539-1:1997, \textsl{Information Technology - Programming Languages - Fortran}.

This OpenMP API specification refers to ISO/IEC 1539-1:1997 as Fortran 95.

\item ISO/IEC 1539-1:2004, \textsl{Information Technology - Programming Languages - Fortran}.

This OpenMP API specification refers to ISO/IEC 1539-1:2004 as Fortran 2003.

\item ISO/IEC 1539-1:2010, \textsl{Information Technology - Programming Languages - Fortran}.

This OpenMP API specification refers to ISO/IEC 1539-1:2010 as Fortran 2008. 
While future versions of the OpenMP specification are expected to
address the following features, currently their use may result in
unspecified behavior.

\begin{itemize}
\item DO CONCURRENT
\item Allocatable components of recursive type
\item Pointer initialization
\item Value attribute is permitted for any nonallocatable nonpointer nonarray
\item Simply contiguous arrays rank remapping to rank>1 target
\item Polymorphic assignment
\item Accessing real and imaginary parts
\item Pointer function reference is a variable
\item Recursive I/O
\item The BLOCK construct
\item EXIT statement (to terminate a non-DO construct)
\item Internal procedure as an actual argument
\item Generic resolution by procedureness
\item Generic resolution by pointer vs. allocatable
\item Impure elemental procedures
\end{itemize}

\end{itemize}

Where this OpenMP API specification refers to C, C++ or Fortran, reference is made to
the base language supported by the implementation.

