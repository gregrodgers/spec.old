% This is an included file. See the master file for more information.
%
% When editing this file:
%
%    1. To change formatting, appearance, or style, please edit openmp.sty.
%
%    2. Custom commands and macros are defined in openmp.sty.
%
%    3. Be kind to other editors -- keep a consistent style by copying-and-pasting to
%       create new content.
%
%    4. We use semantic markup, e.g. (see openmp.sty for a full list):
%         \code{}     % for bold monospace keywords, code, operators, etc.
%         \plc{}      % for italic placeholder names, grammar, etc.
%
%    5. There are environments that provide special formatting, e.g. language bars.
%       Please use them whereever appropriate.  Examples are:
%
%         \begin{fortranspecific}
%         This is text that appears enclosed in blue language bars for Fortran.
%         \end{fortranspecific}
%
%         \begin{note}
%         This is a note.  The "Note -- " header appears automatically.
%         \end{note}
%
%    6. Other recommendations:
%         Use the convenience macros defined in openmp.sty for the minor headers
%         such as Comments, Syntax, etc.
%
%         To keep items together on the same page, prefer the use of
%         \begin{samepage}.... Avoid \parbox for text blocks as it interrupts line numbering.
%         When possible, avoid \filbreak, \pagebreak, \newpage, \clearpage unless that's
%         what you mean. Use \needspace{} cautiously for troublesome paragraphs.
%
%         Avoid absolute lengths and measures in this file; use relative units when possible.
%         Vertical space can be relative to \baselineskip or ex units. Horizontal space
%         can be relative to \linewidth or em units.
%
%         Prefer \emph{} to italicize terminology, e.g.:
%             This is a \emph{definition}, not a placeholder.
%             This is a \plc{var-name}.
%

\section{Normative References}
\index{normative references}
\label{sec:normative references}
\begin{itemize}
\item ISO/IEC 9899:1990, \textsl{Information Technology - Programming Languages - C}.

This OpenMP API specification refers to ISO/IEC 9899:1990 as C90.

\item ISO/IEC 9899:1999, \textsl{Information Technology - Programming Languages - C}.

This OpenMP API specification refers to ISO/IEC 9899:1999 as C99.

\item ISO/IEC 9899:2011, \textsl{Information Technology - Programming Languages - C}.

This OpenMP API specification refers to ISO/IEC 9899:2011 as C11. The
following features are not supported:

\begin{itemize}
\item Supporting the noreturn property
\item Adding alignment support
\item Creation of complex value
\item Abandoning a process (adding \code{quick_exit} and \code{at_quick_exit})
\item Threads for the C standard library
\item Thread-local storage
\item Parallel memory sequencing model
\item Atomic
\end{itemize}

\item ISO/IEC 14882:1998, \textsl{Information Technology - Programming Languages - C++}.

This OpenMP API specification refers to ISO/IEC 14882:1998 as C++.

\item ISO/IEC 14882:2011, \textsl{Information Technology - Programming Languages - C++}.

This OpenMP API specification refers to ISO/IEC 14882:2011 as
C++11. The following features are not supported:

\begin{itemize}
\item Alignment support
\item Standard layout types
\item Allowing move constructs to throw
\item Defining move special member functions
\item Concurrency
\item Data-dependency ordering: atomics and memory model
\item Additions to the standard library
\item Thread-local storage
\item Dynamic initialization and destruction with concurrency
\item C++11 library
\end{itemize}

\item ISO/IEC 14882:2014, \textsl{Information Technology - Programming Languages - C++}.

This OpenMP API specification refers to ISO/IEC 14882:2014 as
C++14. The following features are not supported:

\begin{itemize}
\item Sized deallocation
\item What signal handlers can do
\end{itemize}

\item ISO/IEC 14882:2017, \textsl{Information Technology - Programming Languages - C++}.

This OpenMP API specification refers to ISO/IEC 14882:2017 as
C++17. 

\item ISO/IEC 1539:1980, \textsl{Information Technology - Programming Languages - Fortran}.

This OpenMP API specification refers to ISO/IEC 1539:1980 as Fortran 77.

\item ISO/IEC 1539:1991, \textsl{Information Technology - Programming Languages - Fortran}.

This OpenMP API specification refers to ISO/IEC 1539:1991 as Fortran 90.

\item ISO/IEC 1539-1:1997, \textsl{Information Technology - Programming Languages - Fortran}.

This OpenMP API specification refers to ISO/IEC 1539-1:1997 as Fortran 95.

\item ISO/IEC 1539-1:2004, \textsl{Information Technology - Programming Languages - Fortran}.

This OpenMP API specification refers to ISO/IEC 1539-1:2004 as Fortran 2003.

\item ISO/IEC 1539-1:2010, \textsl{Information Technology - Programming Languages - Fortran}.

This OpenMP API specification refers to ISO/IEC 1539-1:2010 as Fortran 2008. The
following features are not supported:

\begin{itemize}
\item Submodules
\item Coarrays
\item DO CONCURRENT
\item Allocatable components of recursive type
\item Pointer initialization
\item Value attribute is permitted for any nonallocatable nonpointer nonarray
\item Simply contiguous arrays rank remapping to rank>1 target
\item Polymorphic assignment
\item Accessing real and imaginary parts
\item Pointer function reference is a variable
\item Recursive I/O
\item The BLOCK construct
\item EXIT statement
\item ERROR STOP
\item Internal procedure as an actual argument
\item Generic resolution by procedureness
\item Generic resolution by pointer vs. allocatable
\item Impure elemental procedures
\end{itemize}

\end{itemize}

Where this OpenMP API specification refers to C, C++ or Fortran, reference is made to
the base language supported by the implementation.

