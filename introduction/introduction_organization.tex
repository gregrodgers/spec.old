% This is an included file. See the master file for more information.
%
% When editing this file:
%
%    1. To change formatting, appearance, or style, please edit openmp.sty.
%
%    2. Custom commands and macros are defined in openmp.sty.
%
%    3. Be kind to other editors -- keep a consistent style by copying-and-pasting to
%       create new content.
%
%    4. We use semantic markup, e.g. (see openmp.sty for a full list):
%         \code{}     % for bold monospace keywords, code, operators, etc.
%         \plc{}      % for italic placeholder names, grammar, etc.
%
%    5. There are environments that provide special formatting, e.g. language bars.
%       Please use them whereever appropriate.  Examples are:
%
%         \begin{fortranspecific}
%         This is text that appears enclosed in blue language bars for Fortran.
%         \end{fortranspecific}
%
%         \begin{note}
%         This is a note.  The "Note -- " header appears automatically.
%         \end{note}
%
%    6. Other recommendations:
%         Use the convenience macros defined in openmp.sty for the minor headers
%         such as Comments, Syntax, etc.
%
%         To keep items together on the same page, prefer the use of
%         \begin{samepage}.... Avoid \parbox for text blocks as it interrupts line numbering.
%         When possible, avoid \filbreak, \pagebreak, \newpage, \clearpage unless that's
%         what you mean. Use \needspace{} cautiously for troublesome paragraphs.
%
%         Avoid absolute lengths and measures in this file; use relative units when possible.
%         Vertical space can be relative to \baselineskip or ex units. Horizontal space
%         can be relative to \linewidth or em units.
%
%         Prefer \emph{} to italicize terminology, e.g.:
%             This is a \emph{definition}, not a placeholder.
%             This is a \plc{var-name}.
%

\section{Organization of this Document}
\label{sec:Organization of this document}
The remainder of this document is structured as follows:

\begin{itemize}
\item Chapter \ref{chap:Directives} ``Directives''

\item Chapter \ref{chap:Runtime Library Routines} ``Runtime Library Routines''

\item Chapter \ref{chap:ToolsSupport} ``Tool Support''

\item Chapter \ref{chap:Environment Variables} ``Environment Variables''

\item Appendix \ref{chap:OpenMP Implementation-Defined Behaviors} ``OpenMP Implementation-Defined Behaviors''

\item Appendix \ref{chap:frames} ``Task Frame Management for the Tool Interface''

\item Appendix \ref{chap:ompd_diagram} ``Interaction Diagram of OMPD Components''

\item Appendix \ref{chap:Features History} ``Features History''
\end{itemize}

Some sections of this document only apply to programs written in a certain base
language. Text that applies only to programs for which the base language is C or C++ is shown
as follows:

\begin{ccppspecific}
C/C++ specific text...
\end{ccppspecific}

Text that applies only to programs for which the base language is C only is shown as follows:

\begin{cspecific}
C specific text...
\end{cspecific}

Text that applies only to programs for which the base language is C90 only is shown as
follows:

\begin{c90specific}
C90 specific text...
\end{c90specific}

Text that applies only to programs for which the base language is C99 only is shown as
follows:

\begin{c99specific}
C99 specific text...
\end{c99specific}

Text that applies only to programs for which the base language is C++ only is shown as
follows:

\begin{cppspecific}
C++ specific text...
\end{cppspecific}

Text that applies only to programs for which the base language is Fortran is shown as follows:

\begin{fortranspecific}
Fortran specific text......
\end{fortranspecific}

Where an entire page consists of base language specific text, a marker is shown
at the top of the page.  For Fortran-specific text, the marker is:

\bigskip
\linewitharrows{-1}{dashed}{Fortran (cont.)}
\bigskip

For C/C++-specific text, the marker is:

\bigskip
\linewitharrows{-1}{dashed}{C/C++ (cont.)}
\bigskip

Some text is for information only, and is not part of the normative specification. Such
text is designated as a note, like this:

\needspace{6\baselineskip}\begin{note}
Non-normative text...
\end{note}

