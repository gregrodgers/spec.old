% This is an included file. See the master file for more information.
%
% When editing this file:
%
%    1. To change formatting, appearance, or style, please edit openmp.sty.
%
%    2. Custom commands and macros are defined in openmp.sty.
%
%    3. Be kind to other editors -- keep a consistent style by copying-and-pasting to
%       create new content.
%
%    4. We use semantic markup, e.g. (see openmp.sty for a full list):
%         \code{}     % for bold monospace keywords, code, operators, etc.
%         \plc{}      % for italic placeholder names, grammar, etc.
%
%    5. There are environments that provide special formatting, e.g. language bars.
%       Please use them whereever appropriate.  Examples are:
%
%         \begin{fortranspecific}
%         This is text that appears enclosed in blue language bars for Fortran.
%         \end{fortranspecific}
%
%         \begin{note}
%         This is a note.  The "Note -- " header appears automatically.
%         \end{note}
%
%    6. Other recommendations:
%         Use the convenience macros defined in openmp.sty for the minor headers
%         such as Comments, Syntax, etc.
%
%         To keep items together on the same page, prefer the use of
%         \begin{samepage}.... Avoid \parbox for text blocks as it interrupts line numbering.
%         When possible, avoid \filbreak, \pagebreak, \newpage, \clearpage unless that's
%         what you mean. Use \needspace{} cautiously for troublesome paragraphs.
%
%         Avoid absolute lengths and measures in this file; use relative units when possible.
%         Vertical space can be relative to \baselineskip or ex units. Horizontal space
%         can be relative to \linewidth or em units.
%
%         Prefer \emph{} to italicize terminology, e.g.:
%             This is a \emph{definition}, not a placeholder.
%             This is a \plc{var-name}.
%

\section{Tool Interfaces}
\label{subsec:Tool Support}

The OpenMP API includes two tool interfaces, OMPT and OMPD,
to enable development of high-quality, portable, tools that support
monitoring, performance, or correctness analysis and debugging of OpenMP programs
developed using any implementation of the OpenMP API,


\subsection{OMPT}

The OMPT interface, which is intended for \emph{first-party} tools,
provides the following:
\begin{itemize}
\item A mechanism to initialize a first-party tool;
\item Routines that enable a tool to determine the capabilities of an
      OpenMP implementation;
\item Routines that enable a tool to examine OpenMP state information 
      associated with a thread;
\item Mechanisms that enable a tool to map implementation-level calling
      contexts back to their source-level representations;
\item A callback interface that enables a tool to receive notification
      of OpenMP \emph{events};
\item A tracing interface that enables a tool to trace activity on OpenMP 
      target devices; and
\item A runtime library routine that an application can use to control a tool.
\end{itemize}

OpenMP implementations may differ with respect to the \emph{thread states} that
they support, the mutual exclusion implementations that they employ,
and the OpenMP events for which tool callbacks are invoked. For some OpenMP events,
OpenMP implementations must guarantee that a registered callback will be invoked 
for each occurrence of the event. For other OpenMP events, OpenMP implementations 
are permitted to invoke a registered callback for some or no occurrences of the 
event; for such OpenMP events, however, OpenMP implementations are encouraged to 
invoke tool callbacks on as many occurrences of the event as is practical.
Section~\ref{sec:ompt-register-callbacks} specifies the subset of OMPT
callbacks that an OpenMP implementation must support for a minimal
implementation of the OMPT interface.

An implementation of the OpenMP API may differ from the
abstract execution model described by its specification.  The ability
of tools that use the OMPT interface to observe such differences does 
not constrain implementations of the OpenMP API in any way.

With the exception of the \code{omp_control_tool} runtime library routine for 
tool control, all other routines in the OMPT interface are intended for use 
only by tools and are not visible to applications. For that reason, a Fortran 
binding is provided only for \code{omp_control_tool}; all other OMPT functionality 
is described with C syntax only.

\subsection{OMPD}

The OMPD interface is intended for a \emph{third-party} tool, which runs as a 
separate process. An OpenMP implementation must provide an OMPD
library that can be dynamically loaded and used by a third-party tool.
A third-party tool, such as a debugger, uses the OMPD library to access
OpenMP state of a program that has begun execution. OMPD defines the following:

\begin{itemize}
\item An interface that an OMPD library exports, which a tool can use 
      to access OpenMP state of a program that has begun execution;
\item A callback interface that a tool provides to the OMPD library so 
      that the library can use it to access the OpenMP state of a program 
      that has begun execution; and
\item A small number of symbols that must be defined by an OpenMP 
      implementation to help the tool find the correct OMPD library to use 
      for that OpenMP implementation and to facilitate notification of events.
\end{itemize}
Section~\ref{sec:ompd-overview} describes OMPD in detail.

