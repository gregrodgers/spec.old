\chapter{OMPD Interface}
\label{sec:OMPD Interface}
\label{chap:OMPD Interface}
\label{sec:ompd-overview}

This chapter describes OMPD, which is an interface for \emph{third-party} tools.
\emph{Third-party} tools exist in separate processes from the OpenMP program.
To provide OMPD support, an OpenMP implementation must provide an OMPD library to
be loaded by the \emph{third-party} tool. An OpenMP implementation does not need
to maintain any extra information to support OMPD inquiries from \emph{third-party}
tools \emph{unless} it is explicitly instructed to do so.

OMPD allows \emph{third-party tools} such as a debuggers to inspect the 
OpenMP state of a live program or core file in an implementation-agnostic 
manner. That is, a tool that uses OMPD should work with any conforming 
OpenMP implementation. An OpenMP implementor provides a library for OMPD 
that a third-party tool can dynamically load. Using the interface exported 
by the OMPD library, the external tool can inspect the OpenMP state of a 
program. In order to satisfy requests from the third-party tool, the OMPD 
library may need to read data from, or to find the addresses of symbols in 
the OpenMP program. The OMPD library provides this functionality through a 
callback interface that the third-party tool must instantiate for the OMPD library.

To use OMPD, the third-party tool loads the OMPD library. The OMPD library exports 
the API that is defined throughout this section and that the tool uses to 
determine OpenMP information about the OpenMP program. The OMPD library must
look up the symbols and read data out of the program. It does not perform
these operations directly, but instead it uses the callback interface that the
tool exports to cause the tool to perform them.

The OMPD architecture insulates tools from the internal structure of the 
OpenMP runtime while the OMPD library is insulated from the details of how 
to access the OpenMP program. This decoupled design allows for flexibility in how 
the OpenMP program and tool are deployed, so that, for example, the tool and the 
OpenMP program are not required to execute on the same machine.

Generally the tool does not interact directly with the OpenMP runtime and, 
instead, interacts with it through the OMPD library. However, a few cases
require the tool to access the OpenMP runtime directly. These cases fall into 
two broad categories. The first is during initialization, where the tool must
look up symbols and read variables in the OpenMP runtime in order to identify 
the OMPD library that it should use, which is discussed in 
\specref{subsubsec:ompd_dll_locations} and 
\specref{subsubsec:ompd_dll_locations_valid}. The second category relates to 
arranging for the tool to be notified when certain events occur during the 
execution of the OpenMP program. For this purpose, the OpenMP implementation
must define certain symbols in the runtime code, as is discussed in 
\specref{subsec:runtime-entry-points-for-ompd}. Each of these symbols corresponds 
to an event type. The runtime must ensure that control passes through the 
appropriate named location when events occur. If the tool requires notification 
of an event, it can plant a breakpoint at the matching location. The location 
can, but may not, be a function. It can, for example, simply be a label. However, 
the names of the locations must have external \code{C} linkage.

\section{OMPD Interfaces Definitions}
\index{tool interfaces definitions}
\index{tools header files}
\label{sec:OMPD Interfaces Definitions}

\begin{ccppspecific}
A compliant implementation must supply a set of definitions for the OMPD runtime
entry points, OMPD tool
callback signatures, OMPD tool interface routines, and the special data types of
their parameters and return values. These definitions, which are listed throughout
this chapter, and their associated declarations shall be provided in a header file
named \code{omp-tools.h}. In addition, the set of definitions may specify other
implementation-specific values.

The \code{ompd_dll_locations} function,
all OMPD tool interface functions, and all OMPD runtime entry points are
external functions with \code{C} linkage.
\end{ccppspecific}






\section{Activating an OMPD Tool}
\label{subsec:activating}

The tool and the OpenMP program exist as separate processes. 
Thus, coordination is required between the OpenMP runtime
and the external tool for OMPD.

\subsection{Enabling the Runtime for OMPD}
\label{subsubsec:enabling-ompd}

In order to support third-party tools, the OpenMP runtime may need to collect
and to maintain information that it might not otherwise. The OpenMP runtime 
collects whatever information is necessary to support OMPD if the environment 
variable \code{OMP_DEBUG} is set to \plc{enabled}.

\begin{crossrefs}
\item Activating an OMPT Tool, \specref{sec:ompt-initialization}

\item   \code{OMP_DEBUG}, \specref{sec:OMP_DEBUG}
\end{crossrefs}

\subsection{\hcode{ompd_dll_locations}}
\label{subsubsec:ompd_dll_locations}
\index{ompd_dll_locations@{\code{ompd_dll_locations}}}

\summary
The \code{ompd_dll_locations} global variable indicates the location
of OMPD libraries that are compatible with the OpenMP implementation.

\format
\begin{cspecific}
\begin{ompSyntax}
const char **ompd_dll_locations;
\end{ompSyntax}
\end{cspecific}

\descr
An OpenMP runtime may have more than one OMPD library. The tool must be able 
to locate the right library to use for the OpenMP program that it is examining.
The OpenMP runtime system must provide a public variable \code{ompd_dll_locations},
which is an \code{argv}-style vector of filename string pointers that provides 
the name(s) of any compatible OMPD library. This variable must have \code{C} 
linkage. The tool uses the name of the variable verbatim and, in particular, 
does not apply any name mangling before performing the look up.

The programming model or architecture of the tool and, thus, that of OMPD does 
not have to match that of the OpenMP program that is being examined. The tool
must interpret the contents of \code{ompd_dll_locations} to find a suitable OMPD 
that matches its own architectural characteristics. On platforms that support 
different programming models (for example, 32-bit vs 64-bit), OpenMP 
implementations are encouraged to provide OMPD libraries for all models, and 
that can handle OpenMP programs of any model. Thus, for example, a 32-bit 
debugger that uses OMPD should be able to debug a 64-bit OpenMP program by 
loading a 32-bit OMPD implementation that can manage a 64-bit OpenMP runtime.

\code{ompd_dll_locations} points to a NULL-terminated vector of zero or more 
NULL-terminated pathname strings that do not have any filename conventions. 
This vector must be fully initialized \emph{before} \code{ompd_dll_locations} 
is set to a non-null value, such that if a tool, such as a debugger, stops 
execution of the OpenMP program at any point at which \code{ompd_dll_locations} 
is non-null, then the vector of strings to which it points is valid and complete.

\begin{crossrefs}
\item   \code{ompd_dll_locations_valid}, 
see \specref{subsubsec:ompd_dll_locations_valid}
\end{crossrefs}



\subsection{\hcode{ompd_dll_locations_valid}}
\label{subsubsec:ompd_dll_locations_valid}
\index{ompd_dll_locations@{\code{ompd_dll_locations_valid}}}

\summary
The OpenMP runtime notifies third-party tools that \code{ompd_dll_locations}
is valid by allowing execution to pass through a location that the symbol 
\code{ompd_dll_locations_valid} identifies.

\format
\begin{cspecific}
\begin{ompSyntax}
void ompd_dll_locations_valid(void);
\end{ompSyntax}
\end{cspecific}

\descr
Since \code{ompd_dll_locations} may not be a static variable, it may require
runtime initialization. The OpenMP runtime notifies third-party tools that 
\code{ompd_dll_locations} is valid by having execution pass through a location 
that the symbol \code{ompd_dll_locations_valid} identifies. If 
\code{ompd_dll_locations} is NULL, a third-party tool can place a breakpoint 
at \code{ompd_dll_locations_valid} to be notified that \code{ompd_dll_locations} 
is initialized. In practice, the symbol \code{ompd_dll_locations_valid} 
may not be a function; instead, it may be a labeled machine instruction through
which execution passes once the vector is valid.







\section{OMPD Data Types}
\label{subsec:ompd-data-types}

This section defines the OMPD types.

\subsection{Size Type}
\label{subsubsubsec:ompd_size_t}

\summary
The \code{ompd_size_t} type specifies the number of bytes in opaque 
data objects that are passed across the OMPD API.

\format
\begin{ccppspecific}
\begin{ompSyntax}
typedef uint64_t ompd_size_t;
\end{ompSyntax}
\end{ccppspecific}



\subsection{Wait ID Type}
\label{subsubsubsec:ompd_wait_id_t}

\summary
This \code{ompd_wait_id_t} type identifies the object on which a thread.

\format
\begin{ccppspecific}
\begin{ompSyntax}
typedef uint64_t ompd_wait_id_t;
\end{ompSyntax}
\end{ccppspecific}



\subsection{Basic Value Types}
\label{subsubsubsec:ompd_addr_t}
\label{subsubsubsec:ompd_word_t}
\label{subsubsubsec:ompd_seg_t}

\summary
These definitions represent a word, address, and segment value types.

\format

\begin{ccppspecific}
\begin{ompSyntax}
typedef uint64_t ompd_addr_t;
typedef int64_t  ompd_word_t;
typedef uint64_t ompd_seg_t;
\end{ompSyntax}
\end{ccppspecific}

\descr
The \plc{ompd_addr_t} type represents an unsigned integer address in an
OpenMP process. The \plc{ompd_word_t} type represents a signed version of 
\plc{ompd_addr_t} to hold a signed integer of the OpenMP process. The 
\plc{ompd_seg_t} type represents an unsigned integer segment value.



\subsection{Address Type}
\label{subsubsubsec:ompd_address_t}

\summary
The \code{ompd_address_t} type is used to specify device addresses.

\format
\begin{ccppspecific}
\begin{ompEnv}
typedef struct ompd_address_t {
  ompd_seg_t \plc{segment};
  ompd_addr_t \plc{address};
} ompd_address_t;
\end{ompEnv}
\end{ccppspecific}

\descr
The \code{ompd_address_t} type is a structure that OMPD uses to specify 
device addresses, which may or may not be segmented. For non-segmented 
architectures, \code{ompd_segment_none} is used in the \plc{segment} 
field of \code{ompd_address_t}; it is an instance of the \code{ompd_seg_t} 
type that has the value 0.

\subsection{Frame Information Type}
\label{subsubsubsec:ompd_frame_info_t}

\summary
The \code{ompd_frame_info_t} type is used to specify frame information.

\format
\begin{ccppspecific}
\begin{ompEnv}
typedef struct ompd_frame_info_t {
  ompd_address_t \plc{frame_address};
  ompd_word_t \plc{frame_flag};
} ompd_frame_info_t;
\end{ompEnv}
\end{ccppspecific}

\descr
The \code{ompd_frame_info_t} type is a structure that OMPD uses to specify 
frame information. The \plc{frame_address} field of \code{ompd_frame_info_t} 
identifies a frame.
The \plc{frame_flag} field of \code{ompd_frame_info_t} indicates what type
of information is provided in \plc{frame_address}. The values and meaning is
the same as defined for the \code{ompt_frame_t} enumeration type.

\begin{crossrefs}
\item \code{ompt_frame_t}, 
see \specref{subsubsubsec:ompt_frame_t}
\end{crossrefs}

\subsection{System Device Identifiers}
\label{ompd:omp_device_t}

\summary
The \code{ompd_device_t} type provides information about OpenMP devices.

\format
\begin{ccppspecific}
\begin{ompSyntax}
typedef uint64_t ompd_device_t;
\end{ompSyntax}
\end{ccppspecific}
\descr
Different OpenMP runtimes may utilize different underlying devices. The 
Device identifiers can vary in size and format and, thus, are not explicitly 
represented in OMPD. Instead, device identifiers are passed across the interface 
via the \code{ompd_device_t} type, which is a pointer to where the device 
identifier is stored, and the size of the device identifier in bytes. The 
OMPD library and a tool that uses it must agree on the format of the object 
that is passed. Each different kind of device identifier uses a unique 
unsigned 64-bit integer value.

Recommended values of \code{ompd_device_t} are defined in the \code{ompd-types.h}
header file, which is available on \textbf{\url{http://www.openmp.org/}}.



\subsection{Native Thread Identifiers}
\label{ompd:ompd_thread_id_t}

\summary
The \code{ompd_thread_id_t} type provides information about native threads.

\format
\begin{ccppspecific}
\begin{ompSyntax}
typedef uint64_t ompd_thread_id_t;
\end{ompSyntax}
\end{ccppspecific}

\descr
Different OpenMP runtimes may use different native thread implementations.
Native thread identifiers can vary in size and format and, thus, are not 
explicitly represented in the OMPD API. Instead, native thread identifiers 
are passed across the interface via the \code{ompd_thread_id_t} type, which 
is a pointer to where the native thread identifier is stored, and the size 
of the native thread identifier in bytes. The OMPD library and a tool that 
uses it must agree on the format of the object that is passed. Each different 
kind of native thread identifier uses a unique unsigned 64-bit integer value.

Recommended values of \code{ompd_thread_id_t} are defined in the 
\code{ompd-types.h} header file, which is available on 
\textbf{\url{http://www.openmp.org/}}.



\subsection{OMPD Handle Types}
\label{subsubsec:ompd_address_space_handle_t}
\label{subsubsec:ompd_thread_handle_t}
\label{subsubsec:ompd_parallel_handle_t}
\label{subsubsec:ompd_task_handle_t}

\summary
OMPD handle types are opaque types.

\format
\begin{ccppspecific}
\begin{ompSyntax}
typedef struct _ompd_aspace_handle ompd_address_space_handle_t;
typedef struct _ompd_thread_handle ompd_thread_handle_t;
typedef struct _ompd_parallel_handle ompd_parallel_handle_t;
typedef struct _ompd_task_handle ompd_task_handle_t;
\end{ompSyntax}
\end{ccppspecific}

\descr
OMPD uses handles for address spaces (\code{ompd_address_space_handle_t}),
threads (\code{ompd_thread_handle_t}), parallel regions 
(\code{ompd_parallel_handle_t}), and tasks (\code{ompd_task_handle_t}).
Each operation of the OMPD interface that applies to a particular address 
space, thread, parallel region, or task must explicitly specify a corresponding 
handle. A handle for an entity is constant while the entity itself is alive. 
Handles are defined by the OMPD library, and are opaque to the tool.

Defining externally visible type names in this way introduces type safety to 
the interface, and helps to catch instances where incorrect handles are passed 
by the tool to the OMPD library. The structures do not need to be defined;
instead, the OMPD library must cast incoming (pointers to) handles to the 
appropriate internal, private types.



\subsection{OMPD Scope Types}
\label{subsubsec:ompd_scope_t}

\summary
The \code{ompd_scope_t} type identifies OMPD scopes.

\format
\begin{ccppspecific}
\begin{ompSyntax}
typedef enum ompd_scope_t {
  ompd_scope_global = 1,
  ompd_scope_address_space = 2,
  ompd_scope_thread = 3,
  ompd_scope_parallel = 4,
  ompd_scope_implicit_task = 5,
  ompd_scope_task = 6
} ompd_scope_t;
\end{ompSyntax}
\end{ccppspecific}

\descr
The \code{ompd_scope_t} type identifies OpenMP scopes, including those
related to parallel regions and tasks. When used in an OMPD interface 
function call, the scope type and the ompd handle must match 
according to Table~\ref{table:scope-types}.

\begin{table}[h!]
\caption{Mapping of Scope Type and OMPD Handles\label{table:scope-types}}
\begin{tabular}{p{1.7in} p{3.0in}}
\hline
\textsf{\textbf{Scope types}} & \textsf{\textbf{Handles}}\\
\hline
{\splc{ompd_scope_global}}        & Address space handle for the host device \\
{\splc{ompd_scope_address_space}} & Any address space handle \\
{\splc{ompd_scope_thread}}        & Any thread handle \\
{\splc{ompd_scope_parallel}}      & Any parallel handle \\
{\splc{ompd_scope_implicit_task}} & Task handle for an implicit task \\
{\splc{ompd_scope_task}}          & Any task handle \\
\hline
\end{tabular}%
\end{table}%



\subsection{ICV ID Type}
\label{subsubsec:ompd_icv_id_t}

\summary
The \code{ompd_icv_id_t} type identifies an OpenMP implementation ICV.

\format
\begin{ccppspecific}
\begin{ompSyntax}
typedef uint64_t ompd_icv_id_t;
\end{ompSyntax}
\end{ccppspecific}

The \code{ompd_icv_id_t} type identifies OpenMP implementation ICVs.
\code{ompd_icv_undefined} is an instance of this type with the value 0.



\subsection{Tool Context Types}
\label{subsubsec:ompd_address_space_context_t}
\label{subsubsec:ompd_thread_context_t}

\summary
A third-party tool uses contexts to uniquely identify abstractions. These 
contexts are opaque to the OMPD library and are defined as follows:

\format
\begin{ccppspecific}
\begin{ompSyntax}
typedef struct _ompd_aspace_cont ompd_address_space_context_t;
typedef struct _ompd_thread_cont ompd_thread_context_t;
\end{ompSyntax}
\end{ccppspecific}



\subsection{Return Code Types}
\label{subsubsec:ompd_rc_t}

\summary
The \code{ompd_rc_t} type is the return code type of OMPD operations

\format
\begin{ccppspecific}
\begin{ompSyntax}
typedef enum ompd_rc_t {
  ompd_rc_ok = 0,
  ompd_rc_unavailable = 1,
  ompd_rc_stale_handle = 2,
  ompd_rc_bad_input = 3,
  ompd_rc_error = 4,
  ompd_rc_unsupported = 5,
  ompd_rc_needs_state_tracking = 6,
  ompd_rc_incompatible = 7,
  ompd_rc_device_read_error = 8,
  ompd_rc_device_write_error = 9,
  ompd_rc_nomem = 10,
} ompd_rc_t;
\end{ompSyntax}
\end{ccppspecific}


\descr
The \code{ompd_rc_t} type is used for the return codes of OMPD operations. 
The return code types and their semantics are defined as follows:

\begin{itemize}
\label{ompd:ompd_rc_ok}
\item \code{ompd_rc_ok} is returned when the operation is successful;

\label{ompd:ompd_rc_unavailable}
\item \code{ompd_rc_unavailable} is returned when information is not 
      available for the specified context;

\label{ompd:ompd_rc_stale_handle}
\item \code{ompd_rc_stale_handle} is returned when the specified handle 
      is no longer valid;

\label{ompd:ompd_rc_bad_input}
\item \code{ompd_rc_bad_input} is returned when the input parameters 
      (other than handle) are invalid;

\label{ompd:ompd_rc_error}
\item \code{ompd_rc_error} is returned when a fatal error occurred;

\label{ompd:ompd_rc_unsupported}
\item \code{ompd_rc_unsupported} is returned when the requested 
      operation is not supported;

\label{ompd:ompd_rc_needs_state_tracking}
\item \code{ompd_rc_needs_state_tracking} is returned when the state 
      tracking operation failed because state tracking is not currently enabled;

\label{ompd:ompd_rc_device_read_error}
\item \code{ompd_rc_device_read_error} is returned when a read operation 
      failed on the device;

\label{ompd:ompd_rc_device_write_error}
\item \code{ompd_rc_device_write_error} is returned when a write operation 
      failed on the device;

\label{ompd:ompd_rc_incompatible}
\item \code{ompd_rc_incompatible} is returned when this OMPD library is
      incompatible with, or is not capable of handling, the OpenMP program; and

\label{ompd:ompd_rc_nomem}
\item \code{ompd_rc_nomem} is returned when a memory allocation fails.
\end{itemize}



\subsection{Primitive Type Sizes}
\label{subsubsec:ompd_device_type_sizes_t}

\summary
The \code{ompd_device_type_sizes_t} type provides the ``sizeof'' of 
primitive types in the OpenMP architecture address space.

\format

\begin{ccppspecific}
\begin{ompSyntax}
typedef struct ompd_device_type_sizes_t {
  uint8_t \plc{sizeof_char};
  uint8_t \plc{sizeof_short};
  uint8_t \plc{sizeof_int};
  uint8_t \plc{sizeof_long};
  uint8_t \plc{sizeof_long_long};
  uint8_t \plc{sizeof_pointer};
} ompd_device_type_sizes_t;
\end{ompSyntax}
\end{ccppspecific}


\descr
The \code{ompd_device_type_sizes_t} type is used in operations through which 
the OMPD library can interrogate the tool about the ``sizeof'' of primitive 
types in the OpenMP architecture address space. The fields of 
\code{ompd_device_type_sizes_t} give the sizes of the eponymous basic types 
used by the OpenMP runtime. As the tool and the OMPD library, by definition, 
have the same architecture and programming model, the size of the fields can 
be given as \code{uint8_t}.

\begin{crossrefs}
\item \code{ompd_callback_sizeof_fn_t}, 
see \specref{subsubsubsec:ompd_callback_sizeof_fn_t}
\end{crossrefs}
