\section{Runtime Entry Points for OMPD}
\label{subsec:runtime-entry-points-for-ompd}

The OpenMP implementation must define several entry point symbols 
through which execution must pass when particular events occur
\emph{and} data collection for OMPD is enabled. A tool can enable
notification of an event by setting a breakpoint at the address of 
the entry point symbol.

Entry point symbols have external \code{C} linkage and do not
require demangling or other transformations to look up their 
names to obtain the address in the OpenMP program. While each 
entry point symbol conceptually has a function type signature, 
it may not be a function. It may be a labeled location



\subsection{Beginning Parallel Regions}
\label{subsubsec:ompd_bp_parallel_begin}
\index{ompd_bp_parallel_begin@{\code{ompd_bp_parallel_begin}}}

\summary
Before starting the execution of an OpenMP parallel region, the 
implementation executes \code{ompd_bp_parallel_begin}.

\format
\begin{cspecific}
\begin{ompSyntax}
void ompd_bp_parallel_begin(void);
\end{ompSyntax}
\end{cspecific}

\descr
The OpenMP implementation must execute \code{ompd_bp_parallel_begin} 
at every \plc{parallel-begin} event. At the point that the implementation 
reaches \code{ompd_bp_parallel_begin}, the binding for 
\code{ompd_get_curr_parallel_handle} is the parallel region that is 
beginning and the binding for \code{ompd_get_curr_task_handle}
is the task that encountered the \code{parallel} construct.

\begin{crossrefs}
\item \code{parallel} construct, see \specref{sec:parallel Construct}.

\item \code{ompd_get_curr_parallel_handle}, 
see \specref{subsubsubsec:ompd_get_curr_parallel_handle}.

\item \code{ompd_get_curr_task_handle}, 
see \specref{subsubsubsec:ompd_get_curr_task_handle}.
\end{crossrefs}



\subsection{Ending Parallel Regions}
\label{subsubsec:ompd_bp_parallel_end}
\index{ompd_bp_parallel_end@{\code{ompd_bp_parallel_end}}}

\summary
After finishing the execution of an OpenMP parallel region, 
the implementation executes \code{ompd_bp_parallel_end}.

\format
\begin{cspecific}
\begin{ompSyntax}
void ompd_bp_parallel_end(void);
\end{ompSyntax}
\end{cspecific}


\descr
The OpenMP implementation must execute \code{ompd_bp_parallel_end} at 
every \plc{parallel-end} event. At the point that the implementation reaches 
\code{ompd_bp_parallel_end}, the binding for \code{ompd_get_curr_parallel_handle} 
is the \code{parallel} region that is ending and the binding for 
\code{ompd_get_curr_task_handle} is the task that encountered the 
\code{parallel} construct. After execution of \code{ompd_bp_parallel_end}, any 
\plc{parallel_handle} that was acquired for the \code{parallel} region is 
invalid and should be released.

\begin{crossrefs}
\item \code{parallel} construct, see \specref{sec:parallel Construct}.

\item \code{ompd_get_curr_parallel_handle}, 
see \specref{subsubsubsec:ompd_get_curr_parallel_handle}.

\item \code{ompd_rel_parallel_handle}, 
see \specref{subsubsubsec:ompd_rel_parallel_handle}.

\item \code{ompd_get_curr_task_handle}, 
see \specref{subsubsubsec:ompd_get_curr_task_handle}.
\end{crossrefs}



\subsection{Beginning Task Regions}
\label{subsubsec:ompd_bp_task_begin}
\index{ompd_bp_task_begin@{\code{ompd_bp_task_begin}}}

\summary
Before starting the execution of an OpenMP task region, 
the implementation executes \code{ompd_bp_task_begin}.

\format
\begin{cspecific}
\begin{ompSyntax}
void ompd_bp_task_begin(void);
\end{ompSyntax}
\end{cspecific}

\descr
The OpenMP implementation must execute \code{ompd_bp_task_begin} 
immediately before starting execution of a \plc{structured-block} that is
associated with a non-merged task. At the point that the implementation 
reaches \code{ompd_bp_task_begin}, the binding for 
\code{ompd_get_curr_task_handle} is the task that is scheduled to execute.

\begin{crossrefs}
\item \code{ompd_get_curr_task_handle}, 
see \specref{subsubsubsec:ompd_get_curr_task_handle}.
\end{crossrefs}



\subsection{Ending Task Regions}
\label{subsubsec:ompd_bp_task_end}
\index{ompd_bp_task_end@{\code{ompd_bp_task_end}}}

\summary
After finishing the execution of an OpenMP task region, 
the implementation executes \code{ompd_bp_task_end}.

\format
\begin{cspecific}
\begin{ompSyntax}
void ompd_bp_task_end(void);
\end{ompSyntax}
\end{cspecific}

\descr
The OpenMP implementation must execute \code{ompd_bp_task_end} immediately
after completion of a \plc{structured-block} that is associated with a 
non-merged task. At the point that the implementation reaches 
\code{ompd_bp_task_end}, the binding for \code{ompd_get_curr_task_handle} 
is the task that finished execution. After execution of \code{ompd_bp_task_end}, 
any \plc{task_handle} that was acquired for the task region is invalid and 
should be released.

\begin{crossrefs}
\item \code{ompd_get_curr_task_handle}, 
see \specref{subsubsubsec:ompd_get_curr_task_handle}.

\item \code{ompd_rel_task_handle}, see \specref{subsubsubsec:ompd_rel_task_handle}.
\end{crossrefs}



\subsection{Beginning OpenMP Threads}
\label{subsubsec:ompd_bp_thread_begin}
\index{ompd_bp_thread_begin@{\code{ompd_bp_thread_begin}}}

\summary
When starting an OpenMP thread, the implementation executes
\code{ompd_bp_thread_begin}.

\format
\begin{cspecific}
\begin{ompSyntax}
void ompd_bp_thread_begin(void);
\end{ompSyntax}
\end{cspecific}

\descr
The OpenMP implementation must execute \code{ompd_bp_thread_begin} at 
every \plc{native-thread-begin} and \plc{initial-thread-begin} event.
This execution occurs before the thread starts the execution of any 
OpenMP region.

\begin{crossrefs}
\item \code{parallel} construct, see \specref{sec:parallel Construct}.

\item Initial task, see \specref{subsec:Initial Task}.
\end{crossrefs}



\subsection{Ending OpenMP Threads}
\label{subsubsec:ompd_bp_thread_end}
\index{ompd_bp_thread_end@{\code{ompd_bp_thread_end}}}

\summary
When terminating an OpenMP thread, the implementation 
executes \code{ompd_bp_thread_end}.

\format
\begin{cspecific}
\begin{ompSyntax}
void ompd_bp_thread_end(void);
\end{ompSyntax}
\end{cspecific}

\descr
The OpenMP implementation must execute \code{ompd_bp_thread_end} 
at every \plc{native-thread-end} and the \plc{initial-thread-end} event.
This execution occurs after the thread completes the execution of 
all OpenMP regions. After executing \code{ompd_bp_thread_end}, any 
\plc{thread_handle} that was acquired for this thread is invalid 
and should be released.

\begin{crossrefs}
\item \code{parallel} construct, see \specref{sec:parallel Construct}.

\item Initial task, see \specref{subsec:Initial Task}.

\item \code{ompd_rel_thread_handle}, 
see \specref{subsubsubsec:ompd_rel_thread_handle}.
\end{crossrefs}



\subsection{Initializing OpenMP Devices}
\label{subsubsec:ompd_bp_device_begin}
\index{ompd_bp_device_begin@{\code{ompd_bp_device_begin}}}

\summary
The OpenMP implementation must execute \code{ompd_bp_device_begin} 
at every \plc{device-initialize} event.

\format
\begin{cspecific}
\begin{ompSyntax}
void ompd_bp_device_begin(void);
\end{ompSyntax}
\end{cspecific}

\descr
When initializing a device for execution of a \code{target} region, 
the implementation must execute \code{ompd_bp_device_begin}.
This execution occurs before the work associated with any 
OpenMP region executes on the device.

\begin{crossrefs}
\item Device Initialization, see \specref{subsec:Device Initialization}.
\end{crossrefs}



\subsection{Finalizing OpenMP Devices}
\label{subsubsec:ompd_bp_device_end}
\index{ompd_bp_device_end@{\code{ompd_bp_device_end}}}

\summary
When terminating an OpenMP thread, the implementation 
executes \code{ompd_bp_device_end}.

\format
\begin{cspecific}
\begin{ompSyntax}
void ompd_bp_device_end(void);
\end{ompSyntax}
\end{cspecific}

\descr
The OpenMP implementation must execute \code{ompd_bp_device_end} 
at every \plc{device-finalize} event. This execution occurs after 
the thread executes all OpenMP regions. After execution of 
\code{ompd_bp_device_end}, any \plc{address_space_handle} that 
was acquired for this device is invalid and should be released.

\begin{crossrefs}
\item Device Initialization, see \specref{subsec:Device Initialization}.

\item \code{ompd_rel_address_space_handle}, 
see \specref{subsubsubsec:ompd_rel_address_space_handle}.
\end{crossrefs}

