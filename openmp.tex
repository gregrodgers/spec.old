% Welcome to openmp.tex. This is the master LaTex file for the OpenMP specification.
%
% The files in this set include:
%
%    openmp.tex              - this file, the master file
%    Makefile                    - makes the document
%    openmp.sty                  - the main style file
%    openmp-index.ist            - the index style file
%    titlepage.tex               - the title page
%    ch*.tex                     - the main chapters
%    appendix*.tex               - the appendices
%    worksharing-schedule-loop.* - a flow chart diagram
%    openmplogo.png              - the logo
%
% When editing this file:
%
%    1. To change formatting, appearance, or style, please edit openmp.sty.
%
%    2. Custom commands and macros are defined in openmp.sty.
%
%    3. Be kind to other editors -- keep a consistent style by copying-and-pasting to
%       create new content.
%
%    4. We use semantic markup, e.g. (see openmp.sty for a full list):
%         \code{}     % for bold monospace keywords, code, operators, etc.
%         \plc{}      % for italic placeholder names, grammar, etc.
%
%    5. There are environments that provide special formatting, e.g. language bars.
%       Please use them whereever appropriate.  Examples are:
%
%         \begin{fortranspecific}
%         This is text that appears enclosed in blue language bars for Fortran.
%         \end{fortranspecific}
%
%         \begin{note}
%         This is a note.  The "Note -- " header appears automatically.
%         \end{note}
%
%    6. Other recommendations:
%         Use the convenience macros defined in openmp.sty for the minor headers
%         such as Comments, Syntax, etc.
%
%         To keep items together on the same page, prefer the use of
%         \begin{samepage}.... Avoid \parbox for text blocks as it interrupts line numbering.
%         When possible, avoid \filbreak, \pagebreak, \newpage, \clearpage unless that's
%         what you mean. Use \needspace{} cautiously for troublesome paragraphs.
%
%         Avoid absolute lengths and measures in this file; use relative units when possible.
%         Vertical space can be relative to \baselineskip or ex units. Horizontal space
%         can be relative to \linewidth or em units.
%
%         Prefer \emph{} to italicize terminology, e.g.:
%             This is a \emph{definition}, not a placeholder.
%             This is a \plc{var-name}.
%

% The following says letter size, but the style sheet may change the size
\documentclass[10pt,letterpaper,twoside,makeidx,hidelinks]{scrreprt}

% Text to appear in the footer on even-numbered pages:
\newcommand{\footerText}{OpenMP API -- Version 5.0 rev 1, November 2016}

% Style sheet:
% This is openmp.sty, the preamble and style definitions for the OpenMP specification.
% This is an include file. Please see the master file for more information.
%
% When editing this file:
%
%    1. To change formatting, appearance, or style, please edit openmp.sty.
%
%    2. Custom commands and macros are defined in openmp.sty.
%
%    3. Be kind to other editors -- keep a consistent style by copying-and-pasting to
%       create new content.
%
%    4. We use semantic markup, e.g. (see openmp.sty for a full list):
%         \code{}     % for bold monospace keywords, code, operators, etc.
%         \plc{}      % for italic placeholder names, grammar, etc.
%
%    5. Other recommendations:
%         Use the convenience macros defined in openmp.sty for the minor headers
%         such as Comments, Syntax, etc.
%
%         To keep items together on the same page, prefer the use of
%         \begin{samepage}.... Avoid \parbox for text blocks as it interrupts line numbering.
%         When possible, avoid \filbreak, \pagebreak, \newpage, \clearpage unless that's
%         what you mean. Use \needspace{} cautiously for troublesome paragraphs.
%
%         Avoid absolute lengths and measures in this file; use relative units when possible.
%         Vertical space can be relative to \baselineskip or ex units. Horizontal space
%         can be relative to \linewidth or em units.
%
%         Prefer \emph{} to italicize terminology, e.g.:
%             This is a \emph{definition}, not a placeholder.
%             This is a \plc{var-name}.
%
% Quick list of the environments, commands and macros supported. Search below for more details.
%
%    \binding                   % makes header of the same name
%    \comments
%    \constraints
%    \crossreferences
%    \descr
%    \argdesc
%    \effect
%    \format
%    \restrictions
%    \summary
%    \syntax
%
%    \code{}                    % monospace, bold
%       \scode{}                % code to be used in supertabular environment
%       \hcode{}                % code to be used in section headers
%      \pcode{}                 % code to be used when nesting plc inside (cannot use "_")
%
%    \plc{}                     % for any kind of placeholder: italic
%       \splc{}                 % plc to be used in supertabular environment
%       \hplc{}                 % plc to be used in section headers
%
%    \begin{codepar}            % for blocks of verbatim code: monospace, bold
%    \begin{boxedcode}          % outlined verbatim code for syntax definitions, prototypes, etc.
%    \begin{indentedcodelist}   % used with,e.g., "where clause is one of the following:"
%
%    \specref{}                 % formats the cross-reference "Section X on page Y"
%
%    \notestart                 % black horizontal rule for Notes
%    \noteend
%
%    \cspecificstart            % blue horizontal rule for C-specific text
%    \cspecificend
%
%    \cppspecificstart          % blue horizontal rule for C++ -specific text
%    \cppspecificend
%
%    \ccppspecificstart         % blue horizontal rule for C / C++ -specific text
%    \ccppspecificend
%
%    \fortranspecificstart      % blue horizontal rule for Fortran-specific text
%    \fortranspecificend
%
%    \glossaryterm              % for use in formatting glossary entries
%    \glossarydefstart
%    \glossarydefend
%
%    \compactitem               % single-spaced itemized lists for the Examples doc
%    \cexample                  % C/C++ code example for the Examples doc
%    \fexample                  % Fortran code example for the Examples doc

\newcommand{\PAR}{\par}

\usepackage{ifpdf,ifthen}       % allow conditional tests in LaTeX definitions
\usepackage{color,fancyvrb}  % for \VerbatimInput
\usepackage{xparse} %for argument passing to boxedcode


%%%%%%%%%%%%%%%%%%%%%%%%%%%%%%%%%%%%%%%%%%%%%%%%%%%%%%%%%%%%%%%%%%%%%%%%%%%%%%%%%%%%%%%%%%%%%%
% Document data
%
\author{}


%%%%%%%%%%%%%%%%%%%%%%%%%%%%%%%%%%%%%%%%%%%%%%%%%%%%%%%%%%%%%%%%%%%%%%%%%%%%%%%%%%%%%%%%%%%%%
% Fonts

\usepackage{amsmath}
\usepackage{amsfonts}
\usepackage{amssymb}
\usepackage{bm}
\usepackage{courier}
\usepackage{helvet}
\usepackage[utf8]{inputenc}

% Main body serif font:
\usepackage{tgtermes}
\usepackage[T1]{fontenc}


%%%%%%%%%%%%%%%%%%%%%%%%%%%%%%%%%%%%%%%%%%%%%%%%%%%%%%%%%%%%%%%%%%%%%%%%%%%%%%%%%%%%%%%%%%%%%
% Graphic elements

\usepackage{graphicx}
\usepackage{framed}    % for making boxes with \begin{framed}
\usepackage{tikz}      % for flow charts, diagrams, arrows
\usepackage{backnaur}  % for BNF grammar


%%%%%%%%%%%%%%%%%%%%%%%%%%%%%%%%%%%%%%%%%%%%%%%%%%%%%%%%%%%%%%%%%%%%%%%%%%%%%%%%%%%%%%%%%%%%%
% Page formatting

\usepackage[paperwidth=7.5in, paperheight=9in,
            top=0.75in, bottom=1.0in, left=1.4in, right=0.6in]{geometry}

\usepackage{changepage}   % allows left/right-page margin readjustments

\setlength{\oddsidemargin}{0.45in}
\setlength{\evensidemargin}{0.185in}
\raggedbottom


%%%%%%%%%%%%%%%%%%%%%%%%%%%%%%%%%%%%%%%%%%%%%%%%%%%%%%%%%%%%%%%%%%%%%%%%%%%%%%%%%%%%%%%%%%%%%
% Paragraph formatting

\usepackage{setspace}     % allows use of \singlespacing, \onehalfspacing
\usepackage{needspace}    % allows use of \needspace to keep lines together
\usepackage{parskip}      % removes paragraph indenting

\raggedright
\usepackage[raggedrightboxes]{ragged2e}  % is this needed?

\lefthyphenmin=60         % only hyphenate if the left part is >= this many chars
\righthyphenmin=60        % only hyphenate if the right part is >= this many chars


%%%%%%%%%%%%%%%%%%%%%%%%%%%%%%%%%%%%%%%%%%%%%%%%%%%%%%%%%%%%%%%%%%%%%%%%%%%%%%%%%%%%%%%%%%%%%%
% Bulleted (itemized) lists
%    Align bullets with section header
%    Align text left
%    Small bullets
%    \compactitem for single-spaced lists (used in the Examples doc)

\usepackage{enumitem}     % for setting margins on lists
\setlist{leftmargin=*}    % don't indent bullet items
\renewcommand{\labelitemi}{{\normalsize$\bullet$}} % bullet size

% There is a \compactitem defined in package parlist (and perhaps others), however,
% we'll define our own version of compactitem in terms of package enumitem that
% we already use:
\newenvironment{compactitem}
{\begin{itemize}[itemsep=-1.2ex]}
{\end{itemize}}


%%%%%%%%%%%%%%%%%%%%%%%%%%%%%%%%%%%%%%%%%%%%%%%%%%%%%%%%%%%%%%%%%%%%%%%%%%%%%%%%%%%%%%%%%%%%%%
% Tables

% This allows tables to flow across page breaks, headers on each new page, etc.
\usepackage{supertabular}
\usepackage{caption}


%%%%%%%%%%%%%%%%%%%%%%%%%%%%%%%%%%%%%%%%%%%%%%%%%%%%%%%%%%%%%%%%%%%%%%%%%%%%%%%%%%%%%%%%%%%%%
% Line numbering

\usepackage[pagewise]{lineno}       % for line numbers on left side of the page
\pagewiselinenumbers
\setlength\linenumbersep{6em}
\renewcommand\linenumberfont{\normalfont\small\sffamily}
\nolinenumbers            % start with line numbers off


%%%%%%%%%%%%%%%%%%%%%%%%%%%%%%%%%%%%%%%%%%%%%%%%%%%%%%%%%%%%%%%%%%%%%%%%%%%%%%%%%%%%%%%%%%%%%
% Footers

\usepackage{fancyhdr}     % makes right/left footers
\pagestyle{fancy}
\fancyhead{} % clear all header fields
\cfoot{}
\renewcommand{\headrulewidth}{0pt}

% Left side on even pages:
% This requires that \footerText be defined in the master document:
\fancyfoot[LE]{\bfseries \thepage \mdseries \hspace{2em} \footerText}
%\fancyhfoffset[E]{4em}
\fancyfootoffset[E]{4em}

% Right side on odd pages:
\fancyfoot[RO]{\mdseries  \leftmark \hspace{2em} \bfseries \thepage}

% Automatic continuation bars: Set style on begin, reset to empty on end.
% The page with the begin also has no cont. bar.

\newcommand{\specificstartmultipage}[1]{\pagestyle{#1}\thispagestyle{emptyhead}}
\newcommand{\specificendmultipage}[1]{\pagestyle{emptyhead}}

\fancypagestyle{emptyhead}
{
    \fancyhead{}
}

\fancypagestyle{fcont}
{
    \setlength{\headheight}{30pt}
    \setlength{\headsep}{3pt}
    \fancyhead[L]{\linewitharrows{-1}{dashed}{Fortran (cont.)}}
}

\fancypagestyle{ccppcont}
{
    \setlength{\headheight}{30pt}
    \setlength{\headsep}{3pt}
    \fancyhead[L]{\linewitharrows{-1}{dashed}{C/C++ (cont.)}}
}


%%%%%%%%%%%%%%%%%%%%%%%%%%%%%%%%%%%%%%%%%%%%%%%%%%%%%%%%%%%%%%%%%%%%%%%%%%%%%%%%%%%%%%%%%%%%%
% Ubuntu 16/04 has a broken titlesec package:

\usepackage{etoolbox}
\makeatletter
\patchcmd{\ttlh@hang}{\parindent\z@}{\parindent\z@\leavevmode}{}{}
\patchcmd{\ttlh@hang}{\noindent}{}{}{}
\makeatother

% END of Ubuntu fix

\ifdef{\ompversion}{}{\def\ompversion{DIFF}}

% Found these numbers empirically by looking at the upper limits of
% tocnumwidth + 2.5pt .
% Critical are 2.16.17. or 2.18.10.

\newlength{\headingIndent}
\setlength{\headingIndent}{-0.7in}
\newlength{\headingOffset}
\setlength{\headingOffset}{0pt}%
\addtolength{\headingOffset}{-1\headingIndent}%


%auxilary command to format the section numbering in a way, that the section
%number shows up in the margin and the section name starts with the text.
\newcommand{\headingformat}[1]{%
%\newlength{\myOffset}\setlength{\myOffset}{\headingOffset}\addtolength{\myOffset}{-.5em}%
% To make the numbering right aligned, use \myOffset instead of
% \headingOffset, use \raggedleft and uncomment the \hspace
\parbox[t]{\headingOffset}{\raggedright{#1}\enskip}%\hspace{.5em}%
}

\KOMAoption{chapterprefix}{true}
\renewcommand*{\chapterformat}{\hspace{\headingIndent}\normalfont\sffamily\upshape\Huge\bfseries\fontsize{16}{16}\selectfont\chapapp~\thechapter~}
\renewcommand*{\sectionformat}{\headingformat{\thesection}}
\renewcommand*{\subsectionformat}{\headingformat{\thesubsection}}
\renewcommand*{\subsubsectionformat}{\headingformat{\thesubsubsection}}
\RedeclareSectionCommand[tocnumwidth=10pt,font=\fontsize{20}{20}\selectfont,%
                         beforeskip=1\baselineskip,afterskip=1\baselineskip]{chapter}

\renewcommand\chapterlineswithprefixformat[3]{%
  \MakeUppercase{#2}\vspace{3\baselineskip}#3\par
  \nobreak\vspace{-\parskip}\vspace{.6\baselineskip}\rule{0.9\textwidth}{1pt}%
}


\RedeclareSectionCommand[tocindent=20pt,tocnumwidth=22.5pt,%
                         beforeskip=5em plus 1em minus 1em,%
                         afterskip=1em plus 0.5em minus 0em,%
                         indent=\headingIndent,%
                         font=\bfseries\sffamily\fontsize{16}{16}\selectfont]%
                         {section}
\RedeclareSectionCommand[tocindent=30pt,tocnumwidth=35pt,%
                         beforeskip=4em plus 1em minus 2em,%
                         afterskip=.75em plus 0.5em minus 0em,%
                         indent=\headingIndent,%
                         font=\bfseries\sffamily\fontsize{14}{14}\selectfont]%
                         {subsection}
\RedeclareSectionCommand[tocindent=40pt,tocnumwidth=42pt,%
                         beforeskip=3em plus 1em minus 1em,%
                         afterskip=.5em plus 0.5em minus 0em,%
                         indent=\headingIndent,%
                         font=\bfseries\sffamily\fontsize{12}{12}\selectfont]%
                         {subsubsection}
\RedeclareSectionCommand[tocindent=50pt,tocnumwidth=49pt,%
                         beforeskip=3em plus 1em minus 1em,%
                         afterskip=.5em plus 0.5em minus 0em,%
                         indent=\headingIndent,%
                         font=\bfseries\sffamily\fontsize{11}{11}\selectfont]
                         {paragraph}

\RedeclareSectionCommand[beforeskip=1.5em plus .55em minus .5em,%
                         afterskip=1sp plus 0.55em minus 0em,%
                         indent=0pt,%
                         font=\bfseries\sffamily\fontsize{11}{11}\selectfont]
                         {subparagraph}

\DeclareNewSectionCommand[style=section, level=5, beforeskip=1.5em plus .55em minus .5em,%
                         afterskip=1sp plus 0.55em minus 0em,%
                         indent=0pt,tocnumwidth=0pt,%
                         tocstyle={},tocindent=0pt,%
                         font=\bfseries\sffamily\fontsize{11}{11}\selectfont]
                         {little}

\widowpenalty=10000
\clubpenalty=10000


%%%%%%%%%%%%%%%%%%%%%%%%%%%%%%%%%%%%%%%%%%%%%%%%%%%%%%%%%%%%%%%%%%%%%%%%%%%%%%%%%%%%%%%%%%%%%%
% Macros for minor headers: Summary, Syntax, Description, etc.
% These headers are defined in terms of \paragraph

%\titleformat{\paragraph}[hang]{\needspace{4\baselineskip}%
%\Large\bfseries\sffamily\fontsize{11}{11}\selectfont}{\theparagraph}{1.0em}{}
%\titlespacing{\paragraph}{-5em}{3em plus 1em minus 1em}{0.5em plus 0.5em minus 0em}[10em]

\newcommand{\subsubsubsection}[1]{\paragraph{#1}}

% Allow latexdiff to access the counter for our redefined subsubsubsections
\usepackage{aliascnt}
\newaliascnt{subsubsubsection}{paragraph}

%\titleformat{\subparagraph}[block]{\needspace{3\baselineskip}\large\bfseries\sffamily\fontsize{11}{11}\selectfont}{}{}{}
%\titlespacing{\subparagraph}{0em}{1.5em plus 0.55em minus 0.5em}{0.0em plus 0.55em minus 0.0em}


% Format and spacing for chapter, section, subsection, and subsubsection headers:

%\setcounter{secnumdepth}{5}          % show numbers down to subsubsection level

\setcounter{tocdepth}{5}
\setcounter{secnumdepth}{5}
% \usepackage{tocloft}



% Use one of the convenience macros below, or \littleheader{} for an arbitrary header
\newcommand{\littleheader}[1] {\subparagraph*{#1}}
\newcommand{\DELlittleheader}[1] {\subparagraph*{\DIFdel{#1}}}
\newcommand{\ADDlittleheader}[1] {\subparagraph*{\DIFadd{#1}}}

\newcommand{\binding} {\littleheader{Binding}}
\newcommand{\comments} {\littleheader{Comments}}
\newcommand{\constraints} {\littleheader{Constraints on Arguments}}
\newcommand{\crossreferences} {\littleheader{Cross References}}
\newcommand{\descr} {\littleheader{Description}}
\newcommand{\argdesc} {\littleheader{Description of Arguments}}
\newcommand{\effect} {\littleheader{Effect}}
\newcommand{\format} {\littleheader{Format}}
\newcommand{\restrictions} {\littleheader{Restrictions}}
\newcommand{\summary} {\littleheader{Summary}}
% test to automatically fix the issue with pagebreak after syntax heading
% (make it section instead of part)
\newcommand{\syntax} {\little*{Syntax}}
\newcommand{\events} {\littleheader{Execution Model Events}}
\newcommand{\tools} {\littleheader{Tool Callbacks}}
\newcommand{\record} {\littleheader{Trace Record}}
\newcommand{\DELbinding} {\DELlittleheader{Binding}}
\newcommand{\DELcomments} {\DELlittleheader{Comments}}
\newcommand{\DELconstraints} {\DELlittleheader{Constraints on Arguments}}
\newcommand{\DELcrossreferences} {\DELlittleheader{Cross References}}
\newcommand{\DELdescr} {\DELlittleheader{Description}}
\newcommand{\DELargdesc} {\DELlittleheader{Description of Arguments}}
\newcommand{\DELeffect} {\DELlittleheader{Effect}}
\newcommand{\DELformat} {\DELlittleheader{Format}}
\newcommand{\DELrestrictions} {\DELlittleheader{Restrictions}}
\newcommand{\DELsummary} {\DELlittleheader{Summary}}
\newcommand{\DELsyntax} {\DELlittleheader{Syntax}}
\newcommand{\DELevents} {\DELlittleheader{Events}}
\newcommand{\DELtools} {\DELlittleheader{Tool Callbacks}}
\newcommand{\DELrecord} {\DELlittleheader{Trace Record}}

\newcommand{\ADDbinding} {\ADDlittleheader{Binding}}
\newcommand{\ADDcomments} {\ADDlittleheader{Comments}}
\newcommand{\ADDconstraints} {\ADDlittleheader{Constraints on Arguments}}
\newcommand{\ADDcrossreferences} {\ADDlittleheader{Cross References}}
\newcommand{\ADDdescr} {\ADDlittleheader{Description}}
\newcommand{\ADDargdesc} {\ADDlittleheader{Description of Arguments}}
\newcommand{\ADDeffect} {\ADDlittleheader{Effect}}
\newcommand{\ADDformat} {\ADDlittleheader{Format}}
\newcommand{\ADDrestrictions} {\ADDlittleheader{Restrictions}}
\newcommand{\ADDsummary} {\ADDlittleheader{Summary}}
\newcommand{\ADDsyntax} {\ADDlittleheader{Syntax}}
\newcommand{\ADDevents} {\ADDlittleheader{Events}}
\newcommand{\ADDtools} {\ADDlittleheader{Tool Callbacks}}
\newcommand{\ADDrecord} {\ADDlittleheader{Trace Record}}



\usepackage{adjustbox}
\usepackage{array}
\usepackage{booktabs}
\usepackage{multirow}
\newcolumntype{R}[2]{%
    >{\adjustbox{angle=#1,lap=\width-(#2)}\bgroup}%
    l%
    <{\egroup}%
}
\newcommand*\rot[1]{\multicolumn{1}{R{60}{3em}}{\begin{minipage}{4cm}#1\end{minipage}}}% no optional argument here, please!


%%%%%%%%%%%%%%%%%%%%%%%%%%%%%%%%%%%%%%%%%%%%%%%%%%%%%%%%%%%%%%%%%%%%%%%%%%%%%%%%%%%%%%%%%%%%%%
% Code and placeholder semantic tagging.
%
% When possible, prefer semantic tags instead of typographic tags. The
% following semantics tags are defined here:
%
%     \code{}     % for bold monospace keywords, code, operators, etc.
%     \plc{}      % for italic placeholder names, grammar, etc.
%
% For function prototypes or other code snippets, you can use \code{} as
% the outer wrapper, and use \plc{{} inside. Example:
%
%     \code{\#pragma omp directive ( \plc{some-placeholder-identifier} :}
%
% To format text in italics for emphasis (rather than text as a placeholder),
% use the generic \emph{} command. Example:
%
%     This sentence \emph{emphasizes some non-placeholder words}.
% Enable \alltt{} for formatting blocks of code:
\usepackage{alltt}
\usepackage{listings}
\usepackage[listings]{tcolorbox}

% This sets the default \code{} font to tt (monospace) and bold:
\RecustomVerbatimCommand{\verb}{Verb}{fontseries=b, commandchars=\\\{\}}
%\newcommand{\code}[1]{{\texttt{\textbf#1}}}}

% This defines the \plc{} placeholder font to be tt normal slanted:
%\newcommand{\plc}[1] {{\textrm{\textmd{\itshape{#1}}}}}

% Environment for a paragraph of literal code, single-spaced, no outline, no indenting:
%\newenvironment{codepar}[1]
%{\begin{alltt}\bfseries #1}
%{\end{alltt}}

% For blocks of code inside a box frame:
\DefineVerbatimEnvironment{OMPVerbatim}{Verbatim}{fontseries=b, commandchars=\\\{\}}
%\newenvironment{boxedcode}
%{ \VerbatimEnvironment
%  \vspace{0.25em plus 5em minus 0.25em}
%  \begin{framed}
%  \begin{minipage}[t]{\textwidth}
%  \begin{OMPVerbatim}
%  }
%  {
%  \end{OMPVerbatim}
%\end{minipage}\end{framed}\vspace{0.25em plus 5em minus 0.25em}}

% Fallback solution for environments that cannot handle listings inside
% (tabulars !)



\lstnewenvironment{boxedcode}
{
\lstset{backgroundcolor=\color{lightgray},caption="boxedcode"}
  }
  {
\lstset{backgroundcolor=\color{white},caption=none}
}


    \definecolor{OMPblue}{RGB}{0,83,159}
    \definecolor{OMPblack}{RGB}{0,0,0}
    \definecolor{OMPwhite}{RGB}{255,255,255}
    \definecolor{OMPlightblue}{RGB}{142,186,226}
    \definecolor{OMPgrey}{RGB}{51,51,51}
    \definecolor{OMPlightgrey}{RGB}{204,204,204}
    \definecolor{OMPsuperlightgrey}{RGB}{247,247,247}
    \definecolor{OMPpetrol}{RGB}{0,97,101}
    \definecolor{OMPteal}{RGB}{0,152,161}
    \definecolor{OMPmaygreen}{RGB}{189,205,0}
    \definecolor{OMPgreen}{RGB}{87,171,39}
    \definecolor{OMPyellow}{RGB}{255,237,0}
    \definecolor{OMPorange}{RGB}{246,168,0}
    \definecolor{OMPmagenta}{RGB}{227,0,102}
    \definecolor{OMPred}{RGB}{204,7,30}
    \definecolor{OMPbordeaux}{RGB}{161,16,53}
    \definecolor{OMPviolet}{RGB}{97,33,88}
    \definecolor{OMPpurple}{RGB}{122,111,172}


% code defines an ompt callback function
\lstnewenvironment{omptCallback}
{}{}%{\lstset{backgroundcolor=\color{OMPpetrol!40}}}
%{\lstset{backgroundcolor=\color{white}}}

% code defines an ompt record struct
\lstnewenvironment{omptRecord}
{}{}%{\lstset{backgroundcolor=\color{OMPteal!40}}}
%{\lstset{backgroundcolor=\color{white}}}

% code defines an ompt inquiry function
\lstnewenvironment{omptInquiry}
{}{}%{\lstset{backgroundcolor=\color{OMPmaygreen!40}}}
%{\lstset{backgroundcolor=\color{white}}}

% code defines an ompt enum
\lstnewenvironment{omptEnum}
{}{}%{\lstset{backgroundcolor=\color{OMPgreen!40}}}
%{\lstset{backgroundcolor=\color{white}}}

% code contains ompt (and none of above)
\lstnewenvironment{omptOther}
{}{}%{\lstset{backgroundcolor=\color{OMPyellow!40}}}
%{\lstset{backgroundcolor=\color{white}}}

% code contains /#pragma/
\lstnewenvironment{ompcPragma}
{}{}%{\lstset{backgroundcolor=\color{OMPviolet!40}%,breaklines=true,prebreak=\textbackslash
%}}
%{\lstset{backgroundcolor=\color{white}}}

% code contains /!$/
\lstnewenvironment{ompfPragma}
{}{}%{\lstset{backgroundcolor=\color{OMPpurple!40}}}
%{\lstset{backgroundcolor=\color{white}}}

% code contains other fortran syntax
\lstnewenvironment{ompfSyntax}
{}{}%{\lstset{backgroundcolor=\color{OMPpurple!80}}}
%{\lstset{backgroundcolor=\color{white}}}


% code contains /function\s+omp/
\lstnewenvironment{ompfFunction}
{}{}%{\lstset{backgroundcolor=\color{OMPyellow!40}}}
%{\lstset{backgroundcolor=\color{white}}}

% code contains /subroutine\s+omp/
\lstnewenvironment{ompfSubroutine}
{}{}%{\lstset{backgroundcolor=\color{OMPorange!40}}}
%{\lstset{backgroundcolor=\color{white}}}

% code contains /typedef enum/
\lstnewenvironment{ompcEnum}
{}{}%{\lstset{backgroundcolor=\color{OMPred!40}}}
%{\lstset{backgroundcolor=\color{white}}}

% code contains function name with prefix omp_ followed by (
\lstnewenvironment{ompcFunction}
{}{}%{\lstset{backgroundcolor=\color{OMPmagenta!40}}}
%{\lstset{backgroundcolor=\color{white}}}

% code contains / :: /
\lstnewenvironment{ompfEnum}
{}{}%{\lstset{backgroundcolor=\color{OMPred!80}}}
%{\lstset{backgroundcolor=\color{white}}}

% code contains OMP?_
\lstnewenvironment{ompEnv}
{}{}%{\lstset{backgroundcolor=\color{OMPblue!40}}}
%{\lstset{backgroundcolor=\color{white}}}

% none of above
\lstnewenvironment{ompSyntax}
{}{}%{\lstset{backgroundcolor=\color{OMPblue!80}}}
%{\lstset{backgroundcolor=\color{white}}}


\lstnewenvironment{indentedcodelist}
{\lstset{xleftmargin=0.08\textwidth,lineskip=1ex}}{}%{\lstset{backgroundcolor=\color{OMPyellow!40},xleftmargin=0.08\textwidth,lineskip=1ex}}
%{\lstset{style=plc}}

\lstnewenvironment{codepar}
{
%\lstset{backgroundcolor=\color{lightgray}}
  }
  {
%\lstset{backgroundcolor=\color{white}}
}

%\tcbuselibrary{listings}
%\tcbset{listing engine=listings}

%\newtcblisting{boxedcodeTCB}
%{
%colback=red!5!white,
%listing only,
%listing engine=listings
%  }

%% Make sure that - in listings is not displayed as --
\makeatletter
\lst@CCPutMacro\lst@ProcessOther {"2D}{\lst@ttfamily{-{}}{-{}}}
\@empty\z@\@empty
\makeatother

%\def\plc{\lstinline[style=plc,basicstyle=\rmfamily\itshape\color{OMPteal}]}
%\def\nspace{\lstinline[showstringspaces=false,basicstyle=\rmfamily\color{OMPgreen},breaklines=true]}
%\def\code{\lstinline[style=openmp,basicstyle=\linespread{1.1}\ttfamily\bfseries\color{OMPpetrol},breaklines=true]}
%\def\plc{\lstinline[style=plc,basicstyle=\rmfamily\itshape]}
%\def\nspace{\lstinline[showstringspaces=false,basicstyle=\rmfamily\color{OMPgreen},breaklines=true]}
%\def\code{\lstinline[style=openmp,basicstyle=\linespread{1.1}\ttfamily\bfseries,breaklines=true]}
%\newcommand\code{\lstinline[style=openmp,basicstyle=\linespread{1.1}\ttfamily\bfseries,breaklines=true]}
\newcommand\hcode[1]{\protect\textbf{\protect\texttt{\protect\detokenize{#1}}}}
\newcommand\scode[1]{\protect\textbf{\protect\texttt{\protect\detokenize{#1}}}}
\newcommand\pcode[1]{\textbf{\texttt{#1}}}
\newcommand\splc[1]{\protect\textit{\protect\textrm{\protect\detokenize{#1}}}}
\newcommand\code[1]{\protect\textbf{\protect\texttt{\protect\detokenize{#1}}}}
\newcommand\tcode[1]{\textbf{\texttt{#1}}}
\newcommand\plc[1]{\protect\textit{\protect\textrm{\protect\detokenize{#1}}}}

\lstdefinestyle{plc}{
  showstringspaces=false,
  columns=fullflexible,
  escapechar=@,
  basicstyle=\rmfamily\itshape,
  keywordstyle=\rmfamily\itshape,
  commentstyle=\rmfamily\itshape,
  breaklines=true,
}

\lstdefinestyle{openmp}{
  showstringspaces=false,
  basicstyle=\ttfamily\bfseries,
  linewidth=.99\textwidth,
  xleftmargin=0.01\textwidth,
%  keywordstyle=\ttfamily\bfseries,
%  commentstyle=\ttfamily\bfseries,
  columns=fullflexible,
  keepspaces=true,
  escapechar=@,
  float,
  backgroundcolor=\color{white!90!black},
framesep=1ex,
frame=l,
framerule=3pt,
breaklines=true,
  floatplacement=H,
  belowskip=\smallskipamount,
  aboveskip=\smallskipamount,
  morecomment=[l][\color{red}\sout]{\%DIF\ <},         % deleted empty lines
  morecomment=[l][\color{blue}\uwave]{\%DIF\ >},       % added empty lines
  moredelim=[il][\color{red}\sout]{\%DIF\ <\ },         % deleted lines
  moredelim=[il][\color{blue}\uwave]{\%DIF\ >\ },       % added lines
  moredelim=**[is][\rmfamily\mdseries\itshape]{\\plc\{}{\}},
%  moredelim=**[is][\mdseries]{\\textnormal\{}{\}},
  moredelim=**[is][\textsubscript]{\\textsubscript\{}{\}},
  moredelim=**[is][]{\\textnormal\{}{\}},
  moredelim=**[is][\rmfamily\mdseries\itshape]{\\textsl\{}{\}},
  moredelim=**[is][]{\\code\{}{\}},
%  moredelim=**[is][\color{green!60!black}]{\\DIFadd\{}{\}},
%  moredelim=**[is][\color{red!80!black}]{\\DIFdel\{}{\}},
  moredelim=*[is][\color{red}\sout]{*!----}{----!*},
  moredelim=*[is][\color{blue}\uwave]{*!++++}{++++!*},
%  moredelim**=[is][\color{green!60!black}\uwave]{*+*}{+*+},
%  moredelim=[is][\color{red}\sout]{-*-}{-*-},
%  moredelim=[is][\color{blue}\uwave]{+*+}{+*+},
}
\lstset{style=openmp}



% This sets the margins in the framed box:
\setlength{\FrameSep}{0.6em}

% For indented lists of verbatim code at a relaxed line spacing,
% e.g., for use after "where clause is one of the following:"
\usepackage{setspace}
%\newenvironment{indentedcodelist}{%
%    \begin{adjustwidth}{0.25in}{}\begin{spacing}{1.5}\begin{alltt}\bfseries}
%    {\end{alltt}\end{spacing}\vspace{-0.25\baselineskip}\end{adjustwidth}}




%% Helper to check all begin/end of *specific
%\renewcommand{\cspecificstart}{\needspace{\sbns}\linewitharrows{-1}{solid}{SPECIF}}
%\renewcommand{\cspecificend}{\linewitharrows{1}{solid}{SPECIE}\bigskip}
%\renewenvironment{cspecific}{\cspecificstart}{\cspecificend}
%\renewenvironment{ccppspecific}{\cspecificstart}{\cspecificend}
%\renewenvironment{cppspecific}{\cspecificstart}{\cspecificend}
%\renewenvironment{c90specific}{\cspecificstart}{\cspecificend}
%\renewenvironment{c99specific}{\cspecificstart}{\cspecificend}
%\renewenvironment{fortranspecific}{\cspecificstart}{\cspecificend}

%%%%%%%%%%%%%%%%%%%%%%%%%%%%%%%%%%%%%%%%%%%%%%%%%%%%%%%%%%%%%%%%%%%%%%%%%%%%%%%%%%%%%%%%%%%%%%
% Glossary formatting

\newcommand{\glossaryterm}[1]{\needspace{1ex}
\begin{adjustwidth}{-0.75in}{0.0in}
\nolinenumbers\parbox[b][-0.95\baselineskip][t]{1.4in}{\flushright \textbf{#1}}
\end{adjustwidth}\linenumbers}

\newcommand{\ADDglossaryterm}[1]{\needspace{1ex}
\begin{adjustwidth}{-0.75in}{0.0in}
\nolinenumbers\parbox[b][-0.95\baselineskip][t]{1.4in}{\flushright \textbf{\DIFadd{#1}}}
\end{adjustwidth}\linenumbers}

\newcommand{\DELglossaryterm}[1]{\needspace{1ex}
\begin{adjustwidth}{-0.75in}{0.0in}
\nolinenumbers\parbox[b][-0.95\baselineskip][t]{1.4in}{\flushright \textbf{\DIFdel{#1}}}
\end{adjustwidth}\linenumbers}

\newcommand{\glossarydefstart}{
\begin{adjustwidth}{0.79in}{0.0in}}

\newcommand{\glossarydefend}{
\end{adjustwidth}\vspace{-1.5\baselineskip}}


%%%%%%%%%%%%%%%%%%%%%%%%%%%%%%%%%%%%%%%%%%%%%%%%%%%%%%%%%%%%%%%%%%%%%%%%%%%%%%%%%%%%%%%%%%%%%
% Indexing and Table of Contents

\usepackage{makeidx}
\usepackage[nodotinlabels]{titletoc}   % required for its [nodotinlabels] option

% Clickable links in TOC and index:
\usepackage[hyperindex=true,linktocpage=true]{hyperref}
\hypersetup{
  colorlinks  = true, % Colors links instead of red boxes
  urlcolor    = blue, % Color for external links
  linkcolor   = blue  % Color for internal links
}


%%%%%%%%%%%%%%%%%%%%%%%%%%%%%%%%%%%%%%%%%%%%%%%%%%%%%%%%%%%%%%%%%%%%%%%%%%%%%%%%%%%%%%%%%%%%%
% Formats a cross reference label as "Section X on page Y".

\newcommand{\specref}[1]{Section~\ref{#1} on page~\pageref{#1}}

% Formats a cross reference label as "Table X on page Y".

\newcommand{\tabref}[1]{Table~\ref{#1} on page~\pageref{#1}}

% Formats a cross reference label as "Chapter X on page Y".

\newcommand{\specchapterref}[1]{Chapter~\ref{#1} on page~\pageref{#1}}

% For caption for supertabular and figure, by yanyh15
\captionsetup[table]{labelfont={sf,sc,bf},textfont=normalfont,singlelinecheck=off,labelformat=simple,labelsep=colon,aboveskip=00pt,belowskip=10pt}
% \renewcommand{\thetable}{\thechapter-\arabic{table}}
%
\captionsetup[figure]{labelfont={sf,sc,bf},textfont=normalfont,singlelinecheck=off,labelformat=simple,labelsep=colon}
% \renewcommand{\thefigure}{\thechapter-\arabic{figure}}
%
%%%%%%%%%%%%%%%%%%%%%%%%%%%%%%%%%%%%%%%%%%%%%%%%%%%%%%%%%%%%%%%%%%%%%%%%%%%%%%%%%%%%%%%%%%%%%
% Code example formatting for the Examples document
% This defines:
%     /cexample       formats blue markers, caption, and code for C/C++ examples
%     /fexample       formats blue markers, caption, and code for Fortran examples
% Thanks to Jin, Haoqiang H. for the original definitions of the following:

\usepackage{toolbox}         % for \toolboxMakeSplit

\newcommand{\myreplace}[3]{\bgroup\toolboxMakeSplit*{#1}{DoSplit}%
   \long\def\DoReplace##1{\DoSplit{##1}\lefttext\righttext
   \lefttext
   \toolboxIfElse{\ifx\righttext\undefined}{}%
      {#2\expandafter\DoReplace\expandafter{\righttext}}}%
   \DoReplace{#3}\egroup}

\newcommand{\escstr}[1]{\myreplace{_}{\_}{#1}}

\def\exampleheader#1#2{%
   \ifthenelse{ \equal{#1}{} }{
      \def\cname{#2}
      \def\ename\cname
   }{
      \def\cname{#1.#2}
% Use following line for old numbering
%      \def\ename{\thesection.#2}
% Use following for mneumonics
      \def\ename{\escstr{#1}.#2}
   }
   \noindent
   \textit{Example \ename}
   %\vspace*{-3mm}
}

\def\cnexample#1#2{%
   \exampleheader{#1}{#2}
   %\code{\VerbatimInput[numbers=left,numbersep=14ex,firstnumber=\thelinenumber,firstline=8,fontsize=\small]%
   %\code{\VerbatimInput[numbers=left,firstnumber=1,firstline=8,fontsize=\small]%
   \code{\VerbatimInput[firstline=8,fontsize=\small]%
      {sources/Example_\cname.c}}
}

\def\fnexample#1#2{%
   \exampleheader{#1}{#2}
   %\code{\VerbatimInput[numbers=left,numbersep=14ex,firstnumber=\thelinenumber,firstline=6,fontsize=\small]%
   %\code{\VerbatimInput[numbers=left,firstnumber=1,firstline=8,fontsize=\small]%
   \code{\VerbatimInput[firstline=8,fontsize=\small]%
      {sources/Example_\cname.f}}
}

\newcommand\cexample[2]{%
\needspace{5\baselineskip}\ccppspecificstart
\cnexample{#1}{#2}
\ccppspecificend
}

\newcommand\fexample[2]{%
\needspace{5\baselineskip}\fortranspecificstart
\fnexample{#1}{#2}
\fortranspecificend
}

% Commands for "bold" ceiling and floor symbols

\newcommand\blfloor{\bm{\lfloor\mkern-10mu\lfloor}}
\newcommand\brfloor{\bm{\rfloor\mkern-10mu\rfloor}}
\newcommand\blceil{\bm{\lceil\mkern-10mu\lceil}}
\newcommand\brceil{\bm{\rceil\mkern-10mu\rceil}}

% Set default fonts:
\rmfamily\mdseries\upshape\normalsize

\usepackage{scrlayer}
\DeclareNewLayer[foreground,textarea,contents={
\phantom{a}
\emph{This page intentionally left blank}
    }
]{intentionally.text}
\DeclareNewPageStyleByLayers{intentionally}{intentionally.text}

\renewcommand{\cleardoublepage}{\cleardoubleoddpageusingstyle{intentionally}}

% This is the end of openmp.sty of the OpenMP specification.


% Enable the next line to display labels and cross-references for debugging:
%\usepackage{showkeys}

\makeindex

\begin{document}
    \pagenumbering{roman}
    %%%%%%%%%%%%%%%%%%%%%%%%%%%%%%%%%%%%%%%%%%%%%%%%%%%%%%%%%%%%%%%%%%%%%%%%%%%%%%%%%%%%%%%%%%%%%%
% Title page

  \begin{titlepage}
    \begin{flushleft}
     \hspace{-6em} \includegraphics[width=0.4\textwidth]{openmp-logo.png}
    \end{flushleft}

    \begin{adjustwidth}{-0.75in}{0in}
    \begin{center}
      \Huge
      \textsf{OpenMP\\Application Programming\\Interface}

      % An optional subtitle can go here:
      \vspace{0.5in}\textsf{    }\vspace{-0.7in}
      \normalsize

      \vspace{1.0in}

      \textbf{Version 5.0 rev 2, November 2017}
    \end{center}
    \end{adjustwidth}

    \vspace{3.0in}

\begin{adjustwidth}{0pt}{1em}\setlength{\parskip}{0.25\baselineskip}%
Copyright \copyright 1997-2017 OpenMP Architecture Review Board.\\
Permission to copy without fee all or part of this material is granted,
provided the OpenMP Architecture Review Board copyright notice and
the title of this document appear. Notice is given that copying is by
permission of OpenMP Architecture Review Board.\end{adjustwidth}

  \end{titlepage}

% Blank page

\clearpage
\thispagestyle{empty}
\phantom{a}
This page intentionally left blank in published version.

%% Commented out for official release.
This is Revision 2+ (Unofficial Draft) (21 February 2018) and
includes the following internal tickets applied to the 4.5 LaTeX sources:
50, 134, 354, 389, 399, 408, 425, 426, 430, 445, 452, 458, 459, 461-463, 
465-467, 484, 486, 489, 491-504, 508, 510, 511, 514, 518, 520, 521, 523, 
524, 530-533, 536, 542, 545, 546, 548, 551, 555-562, 564, 568-578, 581-584, 
586, 589-592, 594, 597, 598, 601, 603, 606, 608, 610-618, 620-623, 625-644, 
648-651, 656, 661-666, 668-673, 675-677, 679, 680, 681, 684, 685, 691-694, 
696, 699-712, 721, 725, 729, 731-736, 745, 752, 753, 755, 763, 765, 773, 
775, 776, 783

%% DO NOT DISTRIBUTE
This is a draft; contents will change in official release
%% This is an internal draft; contents will change in official release

\vfill



    \setcounter{page}{0}

% Four levels in Toc  e.g.  2.14.3.2  (rchrd)
    \setcounter{tocdepth}{3}

    \begin{spacing}{1.3}
        \tableofcontents
    \end{spacing}

    \linenumbers\pagewiselinenumbers
    \newpage\pagenumbering{arabic}

    % This is introduction.tex (Chapter 1 )of the OpenMP specification
% This is an included file. See the master file for more information.
%
% When editing this file:
%
%    1. To change formatting, appearance, or style, please edit openmp.sty.
%
%    2. Custom commands and macros are defined in openmp.sty.
%
%    3. Be kind to other editors -- keep a consistent style by copying-and-pasting to
%       create new content.
%
%    4. We use semantic markup, e.g. (see openmp.sty for a full list):
%         \code{}     % for bold monospace keywords, code, operators, etc.
%         \plc{}      % for italic placeholder names, grammar, etc.
%
%    5. There are environments that provide special formatting, e.g. language bars.
%       Please use them whereever appropriate.  Examples are:
%
%         \begin{fortranspecific}
%         This is text that appears enclosed in blue language bars for Fortran.
%         \end{fortranspecific}
%
%         \begin{note}
%         This is a note.  The "Note -- " header appears automatically.
%         \end{note}
%
%    6. Other recommendations:
%         Use the convenience macros defined in openmp.sty for the minor headers
%         such as Comments, Syntax, etc.
%
%         To keep items together on the same page, prefer the use of
%         \begin{samepage}.... Avoid \parbox for text blocks as it interrupts line numbering.
%         When possible, avoid \filbreak, \pagebreak, \newpage, \clearpage unless that's
%         what you mean. Use \needspace{} cautiously for troublesome paragraphs.
%
%         Avoid absolute lengths and measures in this file; use relative units when possible.
%         Vertical space can be relative to \baselineskip or ex units. Horizontal space
%         can be relative to \linewidth or em units.
%
%         Prefer \emph{} to italicize terminology, e.g.:
%             This is a \emph{definition}, not a placeholder.
%             This is a \plc{var-name}.
%

\chapter{Introduction}
\index{introduction}
\label{chap:introduction}
The collection of compiler directives, library routines, and environment
variables described in this document collectively define the specification of
the OpenMP Application Program Interface (OpenMP API) for parallelism in C, C++
and Fortran programs.

This specification provides a model for parallel programming that is portable
across architectures from different vendors. Compilers from numerous vendors
support the OpenMP API. More information about the OpenMP API can be found at
the following web site

\code{http://www.openmp.org}

The directives, library routines, environment variables, and tool support 
defined in this document allow users to create, to manage, to debug and to 
analyze parallel programs while permitting portability. The directives extend 
the C, C++ and Fortran base languages with single program multiple
data (SPMD) constructs, tasking constructs, device constructs, worksharing 
constructs,and synchronization constructs, and they provide support for 
sharing, mapping and privatizing data. The functionality to control the 
runtime environment is provided by library routines and environment 
variables. Compilers that support the OpenMP API often include a command 
line option to the compiler that activates and allows interpretation of
all OpenMP directives.



\section{Scope}
\label{sec:Scope}
The OpenMP API covers only user-directed parallelization, wherein the 
programmer explicitly specifies the actions to be taken by the compiler 
and runtime system in order to execute the program in parallel. 
OpenMP-compliant implementations are not required to check for data 
dependencies, data conflicts, race conditions, or deadlocks, any of
which may occur in conforming programs. In addition, compliant 
implementations are not required to check for code sequences that 
cause a program to be classified as non-conforming. Application 
developers are responsible for correctly using the OpenMP API
to produce a conforming program. The OpenMP API does not cover 
compiler-generated automatic parallelization.


% This is an included file. See the master file for more information.
%
% When editing this file:
%
%    1. To change formatting, appearance, or style, please edit openmp.sty.
%
%    2. Custom commands and macros are defined in openmp.sty.
%
%    3. Be kind to other editors -- keep a consistent style by copying-and-pasting to
%       create new content.
%
%    4. We use semantic markup, e.g. (see openmp.sty for a full list):
%         \code{}     % for bold monospace keywords, code, operators, etc.
%         \plc{}      % for italic placeholder names, grammar, etc.
%
%    5. There are environments that provide special formatting, e.g. language bars.
%       Please use them whereever appropriate.  Examples are:
%
%         \begin{fortranspecific}
%         This is text that appears enclosed in blue language bars for Fortran.
%         \end{fortranspecific}
%
%         \begin{note}
%         This is a note.  The "Note -- " header appears automatically.
%         \end{note}
%
%    6. Other recommendations:
%         Use the convenience macros defined in openmp.sty for the minor headers
%         such as Comments, Syntax, etc.
%
%         To keep items together on the same page, prefer the use of
%         \begin{samepage}.... Avoid \parbox for text blocks as it interrupts line numbering.
%         When possible, avoid \filbreak, \pagebreak, \newpage, \clearpage unless that's
%         what you mean. Use \needspace{} cautiously for troublesome paragraphs.
%
%         Avoid absolute lengths and measures in this file; use relative units when possible.
%         Vertical space can be relative to \baselineskip or ex units. Horizontal space
%         can be relative to \linewidth or em units.
%
%         Prefer \emph{} to italicize terminology, e.g.:
%             This is a \emph{definition}, not a placeholder.
%             This is a \plc{var-name}.
%

\section{Glossary}
\label{sec:Glossary}
\index{glossary}
\subsection{Threading Concepts}
\label{subsec:Threading Concepts}
\glossaryterm{thread}
\glossarydefstart
An execution entity with a stack and associated static memory, called
\emph{threadprivate memory}.
\glossarydefend

\glossaryterm{OpenMP thread}
\glossarydefstart
A \emph{thread} that is managed by the OpenMP implementation.
\glossarydefend

\glossaryterm{idle thread}
\glossarydefstart
An \emph{OpenMP thread} that is not currently part of any \code{parallel} region.
\glossarydefend

\glossaryterm{thread-safe routine}
\glossarydefstart
A routine that performs the intended function even when executed concurrently
(by more than one \emph{thread}).
\glossarydefend

\glossaryterm{processor}
\glossarydefstart
Implementation defined hardware unit on which one or more \emph{OpenMP threads} can
execute.
\glossarydefend

\glossaryterm{device}
\glossarydefstart
An implementation defined logical execution engine.

\begin{quote}
COMMENT: A \emph{device} could have one or more \emph{processors}.
\end{quote}
\glossarydefend

\glossaryterm{host device}
\glossarydefstart
The \emph{device} on which the \emph{OpenMP program} begins execution.
\glossarydefend

\glossaryterm{target device}
\glossarydefstart
A device onto which code and data may be offloaded from the \emph{host device}.
\glossarydefend

\glossaryterm{parent device}
\glossarydefstart
For a given \code{target} region, the device on which the corresponding \code{target} construct was encountered.
\glossarydefend

% 
\subsection{OpenMP Language Terminology}
\label{subsec:OpenMP Language Terminology}
\glossaryterm{base language}
\glossarydefstart
A programming language that serves as the foundation of the OpenMP
specification.

\begin{quote}
COMMENT: See \specref{sec:normative references}
for a listing of current \emph{base languages} for the OpenMP API.
\end{quote}
\glossarydefend

\glossaryterm{base program}
\glossarydefstart
A program written in a \emph{base language}.
\glossarydefend

\glossaryterm{program order}
\glossarydefstart
An ordering of operations performed by the same thread as determined by the
execution sequence of operations specified by the \emph{base language}.

\begin{quote}
COMMENT: For C11 and C++11, \emph{program order} corresponds to the sequenced
before relation between operations performed by the same thread.
\end{quote}
\glossarydefend

\glossaryterm{structured block}
\glossarydefstart
For C/C++, an executable statement, possibly compound, with a single entry at the
top and a single exit at the bottom, or an OpenMP \emph{construct}.

For Fortran, a block of executable statements with a single entry at the top and a
single exit at the bottom, or an OpenMP \emph{construct}.

\begin{quote}
COMMENTS:

For all \emph{base languages}:

\begin{itemize}
\item Access to the \emph{structured block} must not be the result of a branch;

\item The point of exit cannot be a branch out of the \emph{structured block};

\item Infinite loops where the point of exit is never reached are
allowed in a \emph{structured block}; and

\item Halting caused by an IEEE exception is allowed in a \emph{structured block}.
\end{itemize}

For C/C++:

\begin{itemize}
\item The point of entry must not be a call to \code{setjmp()};

\item \code{longjmp()} and \code{throw()} must not violate the entry/exit criteria;

\item A \emph{structured block} may contain calls to \code{exit()},
\code{_Exit()}, \code{quick_exit()}, \code{abort()} or functions
with a \code{_Noreturn} specifier (in C) or a \code{noreturn} attribute (in C/C++); and
  
\item An expression statement, iteration statement, selection statement,
or try block is considered to be a \emph{structured block} if the
corresponding compound statement obtained by enclosing it in \tcode{\{}
and \tcode{\}} would be a \emph{structured block}.
\end{itemize}

For Fortran:

\begin{itemize}
\item \code{STOP} statements are allowed in a \emph{structured block}.
\end{itemize}
\end{quote}
\glossarydefend

\glossaryterm{compilation unit}
\glossarydefstart
For C/C++, a translation unit.

For Fortran, a program unit.
\glossarydefend

\glossaryterm{enclosing context}
\glossarydefstart
For C/C++, the innermost scope enclosing an OpenMP \emph{directive}.

For Fortran, the innermost scoping unit enclosing an OpenMP \emph{directive}.
\glossarydefend

\glossaryterm{directive}
\glossarydefstart
For C/C++, a \pcode{\#pragma}, and for Fortran, a comment, that specifies \emph{OpenMP
program} behavior.

\begin{quote}
COMMENT: See \specref{sec:Directive Format} for a description of OpenMP \emph{directive} syntax.
\end{quote}
\glossarydefend

\glossaryterm{metadirective}
\glossarydefstart
A \emph{directive} that conditionally resolves to another \emph{directive} at compile time.
\glossarydefend


\glossaryterm{white space}
\glossarydefstart
A non-empty sequence of space and/or horizontal tab characters.
\glossarydefend

\glossaryterm{OpenMP program}
\glossarydefstart
A program that consists of a \emph{base program} that is annotated with OpenMP
\emph{directives} or that calls OpenMP API runtime library routines
\glossarydefend

\glossaryterm{conforming program}
\glossarydefstart
An \emph{OpenMP program} that follows all rules and restrictions of the OpenMP
specification.
\glossarydefend

\glossaryterm{declarative directive}
\glossarydefstart
An OpenMP \emph{directive} that may only be placed in a declarative context. A
\emph{declarative directive} results in one or more declarations only; it is not associated
with the immediate execution of any user code.
\glossarydefend

\glossaryterm{executable directive}
\glossarydefstart
An OpenMP \emph{directive} that is not declarative. That is, it may be placed in an
executable context.
\glossarydefend

\glossaryterm{stand-alone directive}
\glossarydefstart
An OpenMP \emph{executable directive} that has no associated executable user code.
\glossarydefend


\glossaryterm{construct}
\glossarydefstart
An OpenMP \emph{executable directive} (and for Fortran, the paired \code{end} \emph{directive}, if
any) and the associated statement, loop or \emph{structured block}, if any, not including
the code in any called routines. That is, the lexical extent of an \emph{executable
directive}.
\glossarydefend

\glossaryterm{combined construct}
\glossarydefstart
A construct that is a shortcut for specifying one construct immediately nested inside another construct. A combined construct is semantically identical to that of explicitly specifying the first construct containing one instance of the second construct and no other statements.
\glossarydefend

\glossaryterm{composite construct}
\glossarydefstart
A construct that is composed of two constructs but does not have identical semantics to specifying one of the constructs immediately nested inside the other. A composite construct either adds semantics not included in the constructs from which it is composed or the nesting of the one construct inside the other is not conforming.
\glossarydefend

\glossaryterm{combined target construct}
\glossarydefstart
A \emph{combined construct} that is composed of a \code{target} construct and another construct.
\glossarydefend


\glossaryterm{region}
\glossarydefstart
All code encountered during a specific instance of the execution of a given
\emph{construct} or of an OpenMP library routine. A \emph{region} includes any code in called
routines as well as any implicit code introduced by the OpenMP implementation.
The generation of a \emph{task} at the point where a \emph{task generating construct} is encountered is a
part of the \emph{region} of the \emph{encountering thread}, but an \emph{explicit task region}
corresponding to a \emph{task generating construct} is not unless it is an
\emph{included task region}. The point where a \code{target} or \code{teams}
directive is encountered is a part of the \emph{region} of the \emph{encountering thread}, but the
\emph{region} corresponding to the \code{target} or \code{teams} directive is not.

\begin{quote}
COMMENTS:

A \emph{region} may also be thought of as the dynamic or runtime extent of a
\emph{construct} or of an OpenMP library routine.

During the execution of an \emph{OpenMP program}, a \emph{construct} may give
rise to many \emph{regions}.
\end{quote}
\glossarydefend

\glossaryterm{active parallel region}
\glossarydefstart
A \code{parallel} \emph{region} that is executed by a \emph{team} consisting of more than one
\emph{thread}.
\glossarydefend

\smallskip
\glossaryterm{inactive parallel region}
\glossarydefstart
A \code{parallel} \emph{region} that is executed by a \emph{team} of only one \emph{thread}.
\glossarydefend

\glossaryterm{active target region}
\glossarydefstart
A \code{target} \emph{region} that is executed on a \emph{device} other than the \emph{device} that encountered the \code{target} \emph{construct}. 
\glossarydefend

\smallskip
\glossaryterm{inactive target region}
\glossarydefstart
A \code{target} \emph{region} that is executed on the same \emph{device} that encountered the \code{target} \emph{construct}.
\glossarydefend

\glossaryterm{sequential part}
\glossarydefstart
All code encountered during the execution of an \emph{initial task region} that is not part
of a \code{parallel} \emph{region} corresponding to a \code{parallel} \emph{construct} or a \code{task}
\emph{region} corresponding to a \code{task} \emph{construct}.

\begin{quote}
COMMENTS:

A \emph{sequential part} is enclosed by an \emph{implicit parallel region}.

Executable statements in called routines may be in both a \emph{sequential
part} and any number of explicit \code{parallel} \emph{regions} at different points
in the program execution.
\end{quote}
\glossarydefend

\glossaryterm{master thread}
\glossarydefstart
An \emph{OpenMP thread} that has  \emph{thread} number 0. A \emph{master
thread} may be an \emph{initial thread} or the \emph{thread} that encounters a
\code{parallel} \emph{construct}, creates a \emph{team}, generates a set of
\emph{implicit tasks}, and then executes one of those \emph{tasks} as
\emph{thread} number 0.
\glossarydefend

\glossaryterm{parent thread}
\glossarydefstart
The \emph{thread} that encountered the \code{parallel} \emph{construct} and generated a
\code{parallel} \emph{region} is the \emph{parent thread} of each of the
\emph{threads} in the \emph{team} of that
\code{parallel} \emph{region}. The \emph{master thread}
of a \code{parallel} \emph{region} is the same \emph{thread}
as its \emph{parent thread} with respect to any resources associated with an \emph{OpenMP thread}.
\glossarydefend

\glossaryterm{child thread}
\glossarydefstart
When a thread encounters a \code{parallel} construct, each of the threads in the
generated \code{parallel} region's team are \emph{child threads} of the encountering \emph{thread}.
The \code{target} or \code{teams} region's \emph{initial thread} is not a \emph{child thread} of the thread
that encountered the \code{target} or \code{teams} construct.
\glossarydefend

\glossaryterm{ancestor thread}
\glossarydefstart
For a given \emph{thread}, its \emph{parent thread} or one of its \emph{parent thread's ancestor threads}.
\glossarydefend

\glossaryterm{descendent thread}
\glossarydefstart
For a given \emph{thread}, one of its \emph{child threads} or one of
its \emph{child threads' descendent threads}.
\glossarydefend

\glossaryterm{team}
\glossarydefstart
A set of one or more \emph{threads} participating in the execution of a \code{parallel}
\emph{region}.

\begin{quote}
COMMENTS:

For an \emph{active parallel region}, the team comprises the \emph{master thread}
and at least one additional \emph{thread}.

For an \emph{inactive parallel region}, the \emph{team} comprises only the \emph{master thread}.
\end{quote}
\glossarydefend

\glossaryterm{league}
\glossarydefstart
The set of \emph{teams} created by a \code{teams} construct.
\glossarydefend

\glossaryterm{contention group}
\glossarydefstart
An initial \emph{thread} and its \emph{descendent threads}.
\glossarydefend

\glossaryterm{implicit parallel region}
\glossarydefstart
An \emph{inactive parallel region} that is not generated from a
\code{parallel} \emph{construct}. \emph{Implicit parallel regions} surround the whole
\emph{OpenMP program}, all \code{target} \emph{regions}, and all \code{teams}
\emph{regions}.

\glossarydefend

\glossaryterm{initial thread}
\glossarydefstart
The \emph{thread} that executes an \emph{implicit parallel region}.
\glossarydefend

\glossaryterm{initial team}
\glossarydefstart
The \emph{team} that comprises an \emph{initial thread} executing an \emph{implicit parallel region}.
\glossarydefend

\glossaryterm{nested construct}
\glossarydefstart
A \emph{construct} (lexically) enclosed by another \emph{construct}.
\glossarydefend

\glossaryterm{closely nested construct}
\glossarydefstart
A \emph{construct} nested inside another \emph{construct} with no other \emph{construct} nested
between them.
\glossarydefend

\glossaryterm{nested region}
\glossarydefstart
A \emph{region} (dynamically) enclosed by another \emph{region}.  That is, a
\emph{region} generated from the execution of another \emph{region}
or one of its \emph{nested regions}.

\begin{quote}
COMMENT: Some nestings are \emph{conforming} and some are not.
See \specref{sec:Nesting of Regions} for the restrictions on nesting.
\end{quote}
\glossarydefend

\glossaryterm{closely nested region}
\glossarydefstart
A \emph{region nested} inside another \emph{region} with no \code{parallel} \emph{region nested} between
them.
\glossarydefend

\glossaryterm{strictly nested region}
\glossarydefstart
A \emph{region nested} inside another \emph{region} with no other \emph{region nested} between
them.
\glossarydefend

\glossaryterm{all threads}
\glossarydefstart
All OpenMP \emph{threads} participating in the \emph{OpenMP program}.
\glossarydefend

\glossaryterm{current team}
\glossarydefstart
All \emph{threads} in the \emph{team} executing the innermost enclosing \code{parallel} \emph{region}.
\glossarydefend

\glossaryterm{encountering thread}
\glossarydefstart
For a given \emph{region}, the \emph{thread} that encounters the
corresponding \emph{construct}.
\glossarydefend

\glossaryterm{all tasks}
\glossarydefstart
All \emph{tasks} participating in the \emph{OpenMP program}.
\glossarydefend

\glossaryterm{current team tasks}
\glossarydefstart
All \emph{tasks} encountered by the corresponding \emph{team}. The \emph{implicit tasks}
constituting the \code{parallel} \emph{region} and any \emph{descendent tasks} encountered during
the execution of these \emph{implicit tasks} are included in this set of tasks.
\glossarydefend

\glossaryterm{generating task}
\glossarydefstart
For a given \emph{region}, the task for which execution by a \emph{thread} generated the \emph{region}.
\glossarydefend

\glossaryterm{binding thread set}
\glossarydefstart
The set of \emph{threads} that are affected by, or provide the context for, the execution of
a \emph{region}.

The \emph{binding thread} set for a given \emph{region} can be \emph{all threads} on a \emph{device}, \emph{all
threads} in a \emph{contention group}, all \emph{master threads} executing an
enclosing \code{teams} \emph{region}, the \emph{current team}, or the \emph{encountering thread}.

\begin{quote}
COMMENT: The \emph{binding thread set} for a particular \emph{region} is described in its
corresponding subsection of this specification.
\end{quote}
\glossarydefend

\glossaryterm{binding task set}
\glossarydefstart
The set of \emph{tasks} that are affected by, or provide the context for, the execution of a
\emph{region}.

The \emph{binding task} set for a given \emph{region} can be \emph{all tasks},
the \emph{current team tasks}, the \emph{binding implicit task} or the \emph{generating task}.

\begin{quote}
COMMENT: The \emph{binding task} set for a particular \emph{region} (if applicable) is
described in its corresponding subsection of this specification.
\end{quote}
\glossarydefend

%\pagebreak
\glossaryterm{binding region}
\glossarydefstart
The enclosing \emph{region} that determines the execution context and limits the scope of
the effects of the bound \emph{region} is called the \emph{binding region}.

\emph{Binding region} is not defined for \emph{regions} for which the \emph{binding thread} set is \emph{all threads}
or the \emph{encountering thread}, nor is it defined for \emph{regions} for which the \emph{binding task set} is
\emph{all tasks}.

\begin{quote}
COMMENTS:

The \emph{binding region} for an \code{ordered} \emph{region} is the innermost enclosing
\emph{loop region}.

The \emph{binding region} for a \code{taskwait} \emph{region} is the innermost enclosing
\emph{task region}.

The \emph{binding region} for a \code{cancel} \emph{region} is the innermost enclosing \emph{region} corresponding to the \plc{construct-type-clause} of the \code{cancel} construct.

The \emph{binding region} for a \code{cancellation point} \emph{region} is the innermost enclosing \emph{region} corresponding to the \plc{construct-type-clause} of the \code{cancellation point} construct.

For all other \emph{regions} for which the \emph{binding thread set} is the \emph{current
team} or the \emph{binding task set} is the \emph{current team tasks}, the \emph{binding
region} is the innermost enclosing \code{parallel} \emph{region}.

For \emph{regions} for which the \emph{binding task set} is the \emph{generating task}, the
\emph{binding region} is the \emph{region} of the \emph{generating task}.

A \code{parallel} \emph{region} need not be \emph{active} nor explicit to be a \emph{binding region}.

A \emph{task region} need not be explicit to be a \emph{binding region}.

A \emph{region} never binds to any \emph{region} outside of the innermost enclosing
\code{parallel} \emph{region}.
\end{quote}
\glossarydefend

\glossaryterm{orphaned construct}
\glossarydefstart
A \emph{construct} that gives rise to a \emph{region} for which the \emph{binding thread set} is the \emph{current
team}, but is not nested within another \emph{construct} giving rise to the \emph{binding region}.
\glossarydefend

\glossaryterm{worksharing construct}
\glossarydefstart
A \emph{construct} that defines units of work, each of which is executed exactly once by
one of the \emph{threads} in the \emph{team} executing the \emph{construct}.

For C/C++, \emph{worksharing constructs} are \code{for}, \code{sections}, and \code{single}.

For Fortran, \emph{worksharing constructs} are \code{do}, \code{sections}, \code{single} and
\code{workshare}.
\glossarydefend

\glossaryterm{device construct}
\glossarydefstart
An OpenMP \emph{construct} that accepts the \code{device} clause.
\glossarydefend

\glossaryterm{device routine}
\glossarydefstart
A function (for C/C+ and Fortran) or subroutine (for Fortran) that can be
executed on a \emph{target device}, as part of a \code{target} region.
\glossarydefend

\glossaryterm{place}
\glossarydefstart
Unordered set of \emph{processors} on a device that is treated by the execution environment as a
location unit when dealing with OpenMP thread affinity.
\glossarydefend

\glossaryterm{place list}
\glossarydefstart
The ordered list that describes all OpenMP \emph{places} available to the execution
environment.
\glossarydefend

\glossaryterm{place partition}
\glossarydefstart
An ordered list that corresponds to a contiguous interval in the OpenMP \emph{place list}.
It describes the \emph{places} currently available to the execution environment for a given
parallel \emph{region}.
\glossarydefend

\glossaryterm{place number}
\glossarydefstart
A number that uniquely identifies a \emph{place} in the \emph{place list}, with zero identifying the first \emph{place} in the \emph{place list}, and each consecutive whole number identifying the next \emph{place} in the \emph{place list}.
\glossarydefend

\glossaryterm{SIMD instruction}
\glossarydefstart
A single machine instruction that can operate on multiple data elements.
\glossarydefend

\glossaryterm{SIMD lane}
\glossarydefstart
A software or hardware mechanism capable of processing one data element from a
\emph{SIMD instruction}.
\glossarydefend

\glossaryterm{SIMD chunk}
\glossarydefstart
A set of iterations executed concurrently, each by a \emph{SIMD lane}, by a single \emph{thread}
by means of \emph{SIMD instructions}.
\glossarydefend

\glossaryterm{memory}
\glossarydefstart
A storage resource to store and to retrieve variables accessible by OpenMP threads.
\glossarydefend

\glossaryterm{memory space}
\glossarydefstart
A representation of storage resources from which \emph{memory} can be allocated or deallocated.
\glossarydefend

\glossaryterm{memory allocator}
\glossarydefstart
An OpenMP object that fulfills requests to allocate and to deallocate \emph{memory} for program variables from the storage resources of its associated \emph{memory space}.
\glossarydefend

%
% Loop Terminology
%
% 
\subsection{Loop Terminology}
\index{loop terminology}
\label{subsec:Loop Terminology}
\glossaryterm{loop directive}
\glossarydefstart
An OpenMP \emph{executable} directive for which the associated user code must be a loop nest that is a \emph{structured block}.
\glossarydefend

\glossaryterm{associated loop(s)}
\glossarydefstart
The loop(s) controlled by a \emph{loop directive}.
\begin{quote}
COMMENT: If the \emph{loop directive} contains a \code{collapse} or an \code{ordered(}\plc{n}\code{)} clause then it may have more than one \emph{associated loop}.
\end{quote}
\glossarydefend

\glossaryterm{sequential loop}
\glossarydefstart
A loop that is not associated with any OpenMP \emph{loop directive}.
\glossarydefend

\glossaryterm{SIMD loop}
\glossarydefstart
A loop that includes at least one \emph{SIMD chunk}.
\glossarydefend

\glossaryterm{non-rectangular loop nest}
\glossarydefstart
A loop nest for which the iteration count of a loop inside the loop nest is the
not same for all occurrences of the loop in the loop nest.
\glossarydefend

\glossaryterm{doacross loop nest}
\glossarydefstart
A loop nest that has cross-iteration dependence. An iteration is dependent on one or more lexicographically earlier iterations.
\begin{quote}
COMMENT: The \code{ordered} clause parameter on a loop directive identifies the loop(s) associated with the \emph{doacross loop nest}.
\end{quote}
\glossarydefend

%
% Synchronization Terminology
%
\subsection{Synchronization Terminology}
\index{synchronization terminology}
\label{subsec:Synchronization Terminology}
\glossaryterm{barrier}
\glossarydefstart
A point in the execution of a program encountered by a \emph{team} of \emph{threads}, beyond
which no \emph{thread} in the team may execute until all \emph{threads} in the \emph{team} have
reached the barrier and all \emph{explicit tasks} generated by the \emph{team} have executed to
completion. If \emph{cancellation} has been requested, threads may proceed to the end of
the canceled \emph{region} even if some threads in the team have not reached the \emph{barrier}.
\glossarydefend

\glossaryterm{cancellation}
\glossarydefstart
An action that cancels (that is, aborts) an OpenMP \emph{region} and causes executing
\emph{implicit} or \emph{explicit} tasks to proceed to the end of the canceled \emph{region}.
\glossarydefend

\glossaryterm{cancellation point}
\glossarydefstart
A point at which implicit and explicit tasks check if cancellation has been
requested. If cancellation has been observed, they perform the \emph{cancellation}.

\begin{quote}
COMMENT: For a list of cancellation points, see \specref{subsec:cancel Construct}.
\end{quote}
\glossarydefend
\bigskip

\glossaryterm{flush}
\glossarydefstart
An operation that a \emph{thread} performs to enforce consistency between its
view and other \emph{threads}' view of memory.
\glossarydefend

\glossaryterm{flush property}
\glossarydefstart
Properties that determine the manner in which a \emph{flush} operation enforces
memory consistency. These properties are:
\begin{itemize}
    \item \emph{strong}: flushes a set of variables from the current thread's
        temporary view of the memory to the memory;
    \item \emph{release}: orders memory operations that precede the flush
        before memory operations performed by a different thread with which it
        synchronizes;
    \item \emph{acquire}: orders memory operations that follow the flush after
        memory operations performed by a different thread that synchronizes
        with it.
\end{itemize}

\begin{quote}
COMMENT: Any \emph{flush} operation has one or more \emph{flush properties}.
\end{quote}
\glossarydefend

\glossaryterm{strong flush}
\glossarydefstart
A \emph{flush} operation that has the \emph{strong flush property}.
\glossarydefend

\glossaryterm{release flush}
\glossarydefstart
A \emph{flush} operation that has the \emph{release flush property}.
\glossarydefend

\glossaryterm{acquire flush}
\glossarydefstart
A \emph{flush} operation that has the \emph{acquire flush property}.
\glossarydefend

\glossaryterm{atomic operation}
\glossarydefstart
An operation that is specified by an \code{atomic} construct and atomically
accesses and/or modifies a specific storage location.
\glossarydefend

\glossaryterm{atomic read}
\glossarydefstart
An \emph{atomic operation} that is specified by an \code{atomic} construct on which the
\code{read} clause is present.
\glossarydefend

\glossaryterm{atomic write}
\glossarydefstart
An \emph{atomic operation} that is specified by an \code{atomic} construct on which the
\code{write} clause is present.
\glossarydefend

\glossaryterm{atomic update}
\glossarydefstart
An \emph{atomic operation} that is specified by an \code{atomic} construct on which the
\code{update} clause is present.
\glossarydefend

\glossaryterm{atomic captured update}
\glossarydefstart
An \emph{atomic operation} that is specified by an \code{atomic} construct on which the
\code{capture} clause is present.
\glossarydefend

\glossaryterm{read-modify-write}
\glossarydefstart
An \emph{atomic operation} that reads and writes to a given storage location.

\begin{quote}
COMMENT: All \emph{atomic update} and \emph{atomic captured update} operations
are \emph{read-modify-write} operations.
\end{quote}
\glossarydefend

\glossaryterm{sequentially consistent atomic construct}
\glossarydefstart
An \code{atomic} construct for which the \code{seq_cst} clause is specified.
\glossarydefend
\bigskip

\glossaryterm{non-sequentially consistent atomic construct}
\glossarydefstart
An \code{atomic} construct for which the \code{seq_cst} clause is not specified
\glossarydefend
\bigskip
\bigskip

\glossaryterm{sequentially consistent atomic operation}
\glossarydefstart
An \emph{atomic operation} that is specified by a \emph{sequentially consistent atomic construct}. 
\glossarydefend
\bigskip
\bigskip


% 
\subsection{Tasking Terminology}
\index{tasking terminology}
\label{subsec:Tasking Terminology}
\glossaryterm{task}
\glossarydefstart
A specific instance of executable code and its data environment that the
OpenMP implementation can schedule for execution by threads.
\glossarydefend

\glossaryterm{task region}
\glossarydefstart
A \emph{region} consisting of all code encountered during the execution of a \emph{task}.

\begin{quote}
COMMENT: A \code{parallel} \emph{region} consists of one or more implicit \emph{task regions}.
\end{quote}
\glossarydefend

\glossaryterm{implicit task}
\glossarydefstart
A \emph{task} generated by an \emph{implicit parallel region} or generated when a \code{parallel}
\emph{construct} is encountered during execution.
\glossarydefend

\glossaryterm{binding implicit task}
\glossarydefstart
The \emph{implicit task} of the current thread team assigned to the encountering thread.
\glossarydefend

\glossaryterm{explicit task}
\glossarydefstart
A \emph{task} that is not an \emph{implicit task}.
\glossarydefend

\glossaryterm{initial task}
\glossarydefstart
An \emph{implicit task} associated with an \emph{implicit parallel region}.
\glossarydefend

\glossaryterm{current task}
\glossarydefstart
For a given \emph{thread}, the \emph{task} corresponding to the \emph{task region} in which it is
executing.
\glossarydefend

\glossaryterm{child task}
\glossarydefstart
A \emph{task} is a \emph{child task} of its generating \emph{task region}.
A \emph{child task region} is not part of its generating \emph{task region}.
\glossarydefend

\glossaryterm{sibling tasks}
\glossarydefstart
\emph{Tasks} that are \emph{child tasks} of the same \emph{task region}.
\glossarydefend

\glossaryterm{descendent task}
\glossarydefstart
A \emph{task} that is the \emph{child task} of a \emph{task region} or of one of its
\emph{descendent task regions}.
\glossarydefend

\glossaryterm{task completion}
\glossarydefstart
\emph{Task completion} occurs when the end of the \emph{structured block} associated with the
\emph{construct} that generated the \emph{task} is reached.

\begin{quote}
COMMENT: Completion of the \emph{initial task} that is generated when the program begins occurs at program exit.
\end{quote}
\glossarydefend

\glossaryterm{task scheduling point}
\glossarydefstart
A point during the execution of the current \emph{task region} at which it can be
suspended to be resumed later; or the point of \emph{task completion}, after which the
executing thread may switch to a different \emph{task region}.

\begin{quote}
COMMENT: For a list of \emph{task scheduling points}, see \specref{subsec:Task Scheduling}.
\end{quote}
\glossarydefend

\glossaryterm{task switching}
\glossarydefstart
The act of a \emph{thread} switching from the execution of one \emph{task} to another \emph{task}.
\glossarydefend

\glossaryterm{tied task}
\glossarydefstart
A \emph{task} that, when its \emph{task region} is suspended, can be resumed only by the same
\emph{thread} that suspended it. That is, the \emph{task} is tied to that \emph{thread}.
\glossarydefend

\glossaryterm{untied task}
\glossarydefstart
A \emph{task} that, when its \emph{task region} is suspended, can be resumed by any \emph{thread} in
the team. That is, the \emph{task} is not tied to any \emph{thread}.
\glossarydefend

\glossaryterm{undeferred task}
\glossarydefstart
A \emph{task} for which execution is not deferred with respect to its generating \emph{task}
\emph{region}. That is, its generating \emph{task region} is suspended until execution of the
\emph{undeferred task} is completed.
\glossarydefend

\glossaryterm{included task}
\glossarydefstart
A \emph{task} for which execution is sequentially included in the generating \emph{task region}.
That is, an \emph{included task} is \emph{undeferred} and executed immediately by the
\emph{encountering thread}.
\glossarydefend

\glossaryterm{merged task}
\glossarydefstart
A \emph{task} for which the \emph{data environment}, inclusive of ICVs, is the same as that of its
generating \emph{task region}.
\glossarydefend

\glossaryterm{mergeable task}
\glossarydefstart
A \emph{task} that may be a \emph{merged task} if it is an \emph{undeferred task} or an \emph{included task}.
\glossarydefend

\glossaryterm{final task}
\glossarydefstart
A \emph{task} that forces all of its \emph{child tasks} to become \emph{final} and \emph{included tasks}.
\glossarydefend

\glossaryterm{task dependence}
\glossarydefstart
An ordering relation between two \emph{sibling tasks}: the \emph{dependent task} and a
previously generated \emph{predecessor task}. The \emph{task dependence} is fulfilled when the
\emph{predecessor task} has completed.
\glossarydefend

\begin{samepage}
\glossaryterm{dependent task}
\glossarydefstart
A \emph{task} that because of a \emph{task dependence} cannot be executed until its \emph{predecessor
tasks} have completed.
\glossarydefend
\end{samepage}

\glossaryterm{mutually exclusive tasks}
\glossarydefstart
\emph{Tasks} that may be executed in any order, but not at the same
time.
\glossarydefend
\bigskip

\glossaryterm{predecessor task}
\glossarydefstart
A \emph{task} that must complete before its \emph{dependent tasks} can be executed.
\glossarydefend

\glossaryterm{task synchronization construct}
\glossarydefstart
A \code{taskwait}, \code{taskgroup}, or a \code{barrier} \emph{construct}.
\glossarydefend
\bigskip

\glossaryterm{task generating construct}
\glossarydefstart
A \emph{construct} that generates one or more \emph{explicit tasks}.
\glossarydefend
\bigskip

\glossaryterm{target task}
\glossarydefstart
A \emph{mergeable} and \emph{untied} \emph{task} that is generated by a \code{target}, \code{target enter data}, \code{target exit data}, or \code{target update} \emph{construct}.
\glossarydefend

\glossaryterm{taskgroup set}
\glossarydefstart
A set of tasks that are logically grouped by a \code{taskgroup} \emph{region}.
\glossarydefend


\subsection{Data Terminology}
\index{data terminology}
\label{subsec:Data Terminology}
\glossaryterm{variable}
\glossarydefstart
A named data storage block, for which the value can be defined and redefined during the
execution of a program.

\begin{adjustwidth}{-0.75in}{0in}
\begin{note}
An array or structure element is a variable that is part of another variable.
\end{note}
\end{adjustwidth}
\glossarydefend

\glossaryterm{scalar variable}
\glossarydefstart
For C/C++, a scalar variable, as defined by the base language.

For Fortran, a scalar variable with intrinsic type, as defined by the base language,
excluding character type.
\glossarydefend

\glossaryterm{aggregate variable}
\glossarydefstart
A variable, such as an array or structure, composed of other variables.
\glossarydefend

\glossaryterm{array section}
\glossarydefstart
A designated subset of the elements of an array.
\glossarydefend

\glossaryterm{array item}
\glossarydefstart
An array, an array section, or an array element.
\glossarydefend

\glossaryterm{shape-operator}
\glossarydefstart
For C/C++: An array shaping operator that reinterprets a pointer expression as
an array with one or more specified dimensions.
\glossarydefend

\glossaryterm{base expression}
\glossarydefstart
For C/C++, an expression in an array section or array element that specifies
the address of the original array.

\begin{quote}
COMMENT: The \emph{base expression} is \emph{x} for \emph{array
element} x[i] and for \emph{array section} x[i:j].
\end{quote}
\glossarydefend

\glossaryterm{named array}
\glossarydefstart
For C/C++, an expression that is an array but not an array element and appears as the
array referred to by a given array item.

For Fortran, a variable that is an array and appears as the array referred to by a given
array item.
\glossarydefend

\glossaryterm{named pointer}
\glossarydefstart
For C/C++, an lvalue expression that is a pointer and appears as a pointer to the array
implicitly referred to by a given array item.

For Fortran,
a variable that has the \code{POINTER} attribute and appears as a pointer to
the array to which a given array item implicitly refers.
%array implicitly referred to by a given array item

\begin{quote}
COMMENT: A given array item cannot have a \emph{named pointer} if it has a \emph{named array}.
\end{quote}
\glossarydefend


\glossaryterm{attached pointer}
\glossarydefstart
A pointer variable in a device data environment to which the effect of a \code{map} clause
assigns the address of
an array section.  The pointer is
an attached pointer for the remainder of its lifetime in the device data environment.
\glossarydefend
\bigskip

\glossaryterm{simply contiguous array section}
\glossarydefstart
An array section that statically can be determined to have contiguous storage or that, in Fortran, has the \code{CONTIGUOUS} attribute.
\glossarydefend
\bigskip

\glossaryterm{structure}
\glossarydefstart
A structure is a variable that contains one or more variables.

For C/C++:
\nopagebreak
Implemented using struct types.

For C++:
\nopagebreak
Implemented using class types.

For Fortran:
\nopagebreak
Implemented using derived types.
\glossarydefend

\glossaryterm{private variable}
\glossarydefstart
With respect to a given set of \emph{task regions} or \emph{SIMD lanes} that bind to the same
\code{parallel} \emph{region}, a \emph{variable} for which the name provides access to a different block of
storage for each \emph{task region} or \emph{SIMD lane}.

A \emph{variable} that is part of another variable (as an array or structure element) cannot
be made private independently of other components.
\glossarydefend

\glossaryterm{shared variable}
\glossarydefstart
With respect to a given set of \emph{task regions} that bind to the same \code{parallel}
\emph{region}, a \emph{variable} for which the name provides access to the same block of storage for
each \emph{task region}.

A \emph{variable} that is part of another variable (as an array or structure element) cannot
be \emph{shared} independently of the other components, except for static data members
of C++ classes.
\glossarydefend

\glossaryterm{threadprivate variable}
\glossarydefstart
A \emph{variable} that is replicated, one instance per \emph{thread}, by the OpenMP
implementation. Its name then provides access to a different block of storage for
each \emph{thread}.

A \emph{variable} that is part of another variable (as an array or structure element) cannot
be made \emph{threadprivate} independently of the other components, except for static
data members of C++ classes.
\glossarydefend

\glossaryterm{threadprivate memory}
\glossarydefstart
The set of \emph{threadprivate variables} associated with each \emph{thread}.
\glossarydefend

\glossaryterm{data environment}
\glossarydefstart
The \emph{variables} associated with the execution of a given \emph{region}.
\glossarydefend

\glossaryterm{device data environment}
\glossarydefstart
The initial \emph{data environment} associated with a device.
\glossarydefend
\bigskip

\glossaryterm{device address}
\glossarydefstart
An \emph{implementation defined} reference to an address in a \emph{device
  data environment}.
\glossarydefend

\glossaryterm{device pointer}
\glossarydefstart
A \emph{variable} that contains a \emph{device address}.
\glossarydefend


\glossaryterm{mapped variable}
\glossarydefstart
An original \emph{variable} in a \emph{data environment} with a corresponding \emph{variable} in a
device \emph{data environment}.

\begin{quote}
COMMENT: The original and corresponding \emph{variables} may share storage.
\end{quote}
\glossarydefend

\glossaryterm{map-type decay}
\glossarydefstart
The process used to determine the final map type when mapping a variable
with a user defined mapper.  The combination of the two map types determines the
final map type based on the following table.
\begin{table}[h]
\centering
\label{tab:map-type_decay}
\begin{tabular}{l|c|c|c|c|c|c}
  & alloc & to    & from  & tofrom & release & delete \\
  \hline
alloc  & alloc & alloc & alloc & alloc  & release & delete \\
to     & alloc & to    & alloc & to     & release & delete \\
from   & alloc & alloc & from  & from   & release & delete \\
tofrom & alloc & to    & from  & tofrom & release & delete \\
\end{tabular}
\end{table}

\glossarydefend

\glossaryterm{mappable type}
\glossarydefstart
A type that is valid for a \emph{mapped variable}. If a type is composed from other types
(such as the type of an array or structure element) and any of the other types are
not mappable then the type is not mappable.

\begin{quote}
COMMENT: Pointer types are \emph{mappable} but the memory block to which the pointer refers is not \emph{mapped}.
\end{quote}

For C, the type must be a complete type.

For C++, the type must be a complete type.

In addition, for class types:
\begin{itemize}
\item All member functions accessed in any \code{target} region must appear in a
\code{declare}~\code{target} directive.
\end{itemize}

For Fortran, no restrictions on the type except that for derived types:

\begin{itemize}
\item All type-bound procedures accessed in any target region must appear in a \code{declare}~\code{target} directive.
\end{itemize}
\glossarydefend

\glossaryterm{defined}
\glossarydefstart
For \emph{variables}, the property of having a valid value.

For C, for the contents of \emph{variables}, the property of having a valid value.

For C++, for the contents of \emph{variables} of POD (plain old data) type, the property of having
a valid value.

For \emph{variables} of non-POD class type, the property of having been constructed but
not subsequently destructed.

For Fortran, for the contents of \emph{variables}, the property of having a valid value. For the
allocation or association status of \emph{variables}, the property of having a valid status.

\begin{quote}
COMMENT: Programs that rely upon \emph{variables} that are not \emph{defined} are \emph{non-conforming programs}.
\end{quote}
\glossarydefend

\glossaryterm{class type}
\glossarydefstart
For C++, \emph{variables} declared with one of the \code{class}, \code{struct}, or \code{union} keywords.
\glossarydefend





\subsection{Implementation Terminology}
\index{implementation terminology}
\label{subsec:Implementation Terminology}
\glossaryterm{supporting \emph{n} levels of parallelism}
\glossarydefstart
Implies allowing an \emph{active parallel region} to be enclosed by \emph{n-1} \emph{active parallel
regions}.
\glossarydefend

\glossaryterm{supporting the OpenMP API}
\glossarydefstart
Supporting at least one level of parallelism.
\glossarydefend
\bigskip

\glossaryterm{supporting nested parallelism}
\glossarydefstart
Supporting more than one level of parallelism.
\glossarydefend
\bigskip

\glossaryterm{internal control variable}
\glossarydefstart
A conceptual variable that specifies runtime behavior of a set of \emph{threads} or \emph{tasks}
in an \emph{OpenMP program}.

\begin{quote}
COMMENT: The acronym ICV is used interchangeably with the term \emph{internal
control variable} in the remainder of this specification.
\end{quote}
\glossarydefend

\glossaryterm{compliant implementation}
\glossarydefstart
An implementation of the OpenMP specification that compiles and executes any
\emph{conforming program} as defined by the specification.

\begin{quote}
COMMENT: A \emph{compliant implementation} may exhibit \emph{unspecified behavior} when
compiling or executing a \emph{non-conforming program}.
\end{quote}
\glossarydefend

\glossaryterm{unspecified behavior}
\glossarydefstart
A behavior or result that is not specified by the OpenMP specification or not
known prior to the compilation or execution of an \emph{OpenMP program}.

Such \emph{unspecified behavior} may result from:

\begin{itemize}
\item Issues documented by the OpenMP specification as having \emph{unspecified
behavior}.

\item A \emph{non-conforming program}.

\item A \emph{conforming program} exhibiting an \emph{implementation defined} behavior.
\end{itemize}
\glossarydefend

\glossaryterm{implementation defined}
\glossarydefstart
Behavior that must be documented by the implementation, and is allowed to vary
among different \emph{compliant implementations}. An implementation is allowed to
define this behavior as \emph{unspecified}.

\begin{quote}
COMMENT: All features that have \emph{implementation defined} behavior
are documented in Appendix~\ref{chap:OpenMP Implementation-Defined Behaviors}.
\end{quote}
\glossarydefend

\glossaryterm{deprecated}
\glossarydefstart
Implies a construct, clause, or other feature is normative in the current specification but is considered obsolescent and will be removed in the future.
\glossarydefend


\subsection{Tool Terminology}

\glossaryterm{tool}
\glossarydefstart
Executable code, distinct from application or runtime code, that can observe and/or modify the execution of an application.
\glossarydefend

\glossaryterm{first-party tool}
\glossarydefstart
A tool that executes in the address space of the program it is monitoring.
\glossarydefend

\glossaryterm{third-party tool}
\glossarydefstart
A tool that executes as a separate process from that which it is monitoring and potentially controlling.
\glossarydefend

\glossaryterm{activated tool}
\glossarydefstart
A first-party tool that successfully completed its initialization.
\glossarydefend

\glossaryterm{event}
\glossarydefstart
A point of interest in the execution of a thread where the condition
defining that event is true.
\glossarydefend

\glossaryterm{tool callback}
\glossarydefstart
A function provided by a tool to an OpenMP implementation that can be invoked when needed.
\glossarydefend

\glossaryterm{registering a callback}
\glossarydefstart
Providing a callback function to an OpenMP implementation for a particular purpose.
\glossarydefend

\glossaryterm{dispatching a callback at an event}
\glossarydefstart
Processing a callback when an associated event occurs in a manner consistent with the return code
provided when a \emph{first-party} tool registered the callback.
\glossarydefend

\glossaryterm{thread state}
\glossarydefstart
An enumeration type that describes what an OpenMP thread is currently doing.
A thread can be in only one state at any time.
\glossarydefend

\glossaryterm{wait identifier}
\glossarydefstart
A unique opaque handle associated with each data object (for example, a lock) used by the OpenMP runtime
to enforce mutual exclusion that may cause a thread to wait actively or passively.
\glossarydefend

\glossaryterm{frame}
\glossarydefstart
A storage area on a thread's stack associated with a procedure invocation. A frame includes space for
one or more saved registers and often also includes space for saved arguments, local variables,
and padding for alignment.
\glossarydefend

\glossaryterm{canonical frame address}
\glossarydefstart
An address associated with a procedure \emph{frame} on a call stack defined as the value of the stack pointer immediately prior
to calling the procedure whose invocation the frame represents.
\glossarydefend

\glossaryterm{runtime entry point}
\glossarydefstart
A function interface provided by an OpenMP runtime for use by a tool. A runtime entry point is
typically not associated with a global function symbol.
\glossarydefend

\glossaryterm{trace record}
\glossarydefstart
A data structure to store information associated with an occurrence of an \emph{event}.
\glossarydefend

\glossaryterm{native trace record}
\glossarydefstart
A \emph{trace record} for an OpenMP device that is in a device-specific format.
\glossarydefend

\glossaryterm{signal}
\glossarydefstart
A software interrupt delivered to a thread.
\glossarydefend

\glossaryterm{signal handler}
\glossarydefstart
A function called asynchronously when a \emph{signal} is delivered to a thread.
\glossarydefend

\glossaryterm{async signal safe}
\glossarydefstart
Guaranteed not to interfere with operations that are being interrupted by \emph{signal} delivery.
An async signal safe \emph{runtime entry point} is safe to call from a \emph{signal handler}.
\glossarydefend

\glossaryterm{code block}
\glossarydefstart
A contiguous region of memory that contains code of an OpenMP program to be executed on a device.
\glossarydefend


\glossaryterm{OMPT}
\glossarydefstart
An interface that helps a first-party tool monitor the execution of an OpenMP program.
\glossarydefend

\glossaryterm{OMPD}
\glossarydefstart
An interface that helps a third-party tool inspect the OpenMP state of a program that has begun execution.
\glossarydefend

\glossaryterm{OMPD library}
\glossarydefstart
A dynamically loadable library that implements the OMPD interface.
\glossarydefend

\glossaryterm{image file}
\glossarydefstart
An executable or shared library.
\glossarydefend

% ilaguna: not sure if we need this one
%\paragraph{thread}
%A thread is an execution entity running within a specific address
%space within a process.

\glossaryterm{address space}
\glossarydefstart
%An address space is a collection of logical, virtual, or physical memory address ranges containing
%code, stack, and data. The memory address ranges within an address space need not be
%contiguous.  An address space may be segmented, where a segmented address consists of a
%segment identifier and an address in that segment. An address space has associated with it a
%collection of image files that have been loaded into it. For example, an OpenMP program running
%on a system with GPUs may consist of multiple address spaces: one for the host program and one
%for each GPU device. In practical terms, on such systems an OpenMP \emph{device} may be
%implemented as a CUDA context, which \emph{is} an address space into which CUDA image files
%are loaded and CUDA kernels are launched.
A collection of logical, virtual, or physical memory address ranges containing code, stack, and/or data.
Address ranges within an address space need not be contiguous.  An address space consists of one or more \emph{segments}.
% An address space has associated with it a collection of image files that have been loaded into it.
\glossarydefend

\glossaryterm{segment}
\glossarydefstart
A region of an address space associated with a set of address ranges.
\glossarydefend

\begin{comment}
\glossaryterm{architecture}
\glossarydefstart
%A target architecture is defined by the processor (CPU or GPU) and the Application Binary
%Interface (ABI) used by threads and address spaces. A process may contain threads and address
%spaces for multiple target architectures. For example, a process may contain a host address
%space
%and threads for an x86\_64, 64-bit CPU architecture, along with accelerator address spaces and
%threads for an NVIDIA\textsuperscript{\textregistered} GPU architecture or for an
%Intel\textsuperscript{\textregistered} Xeon
%Phi\textsuperscript{\texttrademark} architecture.
A combination of the processor and the Application Binary Interface (ABI) used by
threads and address spaces.
\glossarydefend
\end{comment}

\glossaryterm{OpenMP architecture}
\glossarydefstart
The architecture on which an OpenMP region executes.
\glossarydefend

\glossaryterm{tool architecture}
\glossarydefstart
The architecture on which an OMPD tool executes.
\glossarydefend

\glossaryterm{OpenMP process}
\glossarydefstart
A collection of one or more threads and address spaces. A process may contain threads and address spaces for multiple OpenMP architectures.
At least one thread in an OpenMP process is an OpenMP thread.
%The collection may be homogeneous or heterogeneous, containing, for example, threads or
%address spaces from host programs or accelerator devices.
A process may be live or a core file.
\glossarydefend

\glossaryterm{handle}
\glossarydefstart
%OMPD handles identify OpenMP entities during the execution of an OpenMP program. Handles
%are opaque to the debugger, and defined internally by the OMPD implementation. Below we
%define these handles and the conditions under which they are guaranteed to be valid.
An opaque reference provided by an OMPD library to a using tool. A handle uniquely identifies an abstraction.

% An identifier for an OpenMP entity that is valid during the execution of an OpenMP program. OMPD  handles are created by an OMPD library upon requests from a third-party tool and are opaque to  the third-party tool. Handles remain valid until they are released by third-party tools.
\glossarydefend

\glossaryterm{address space handle}
\glossarydefstart
%A tool handle \emph{address space handle} identifies a portion of an instance of
%an \emph{OpenMP program} that is running on a host device or a target
%device.
%The host address space handle is allocated and initialized with the per
%process or core file initialization
%call to \refdef{\texttt{ompd\_process\_initialize}}{process-initialize:def}.
%A process or core file is initialized by passing the host address
%space context to that function to obtain an address space handle for
%the process or core file.
%The handle remains valid until it is released by the debugger.
%The handle is created by the OMPD implementation, which passes ownership
%to the debugger which is responsible for indicating when it no longer
%needs the handle.
%The debugger releases the handle when it calls
%\refdef{\texttt{ompd\_release\_address\_space\_handle}}{release-address-space-handle:def}.
%The OMPD implementation can use the handle to cache invariant
%address-space-specific data (e.g., symbol addresses), and to retain a copy of
%the debugger's address space context pointer.
%The handle is passed into subsequent API function calls.
%In the OMPD API, an address space handle is represented by the opaque type
%\texttt{ompd\_address\_space\_handle\_t}.
%\emph{Future versions of this API will support address space handles
%	for target devices, which will be allocated and initialized by various
%	OMPD API calls.}
A handle that refers to an address space within an OpenMP process.
%a portion of an instance of an OpenMP program that is running on a host or a target device.
\glossarydefend

\glossaryterm{thread handle}
\glossarydefstart
%The \emph{thread handle} identifies an \emph{OpenMP thread}.
%Thread handles are allocated and initialized by various OMPD API calls.
%A handle is valid for the life time of the corresponding system thread.
%Thread handles are represented by \texttt{ompd\_thread\_handle\_t},
%and created by the OMPD implementation which passes ownership to the
%debugger which is responsible for indicating when it no longer
%needs the handle.
%The debugger releases the thread handle by calling
%\refdef{\texttt{ompd\_release\_thread\_handle}}{release-thread-handle:def}.
A handle that refers to an OpenMP thread.
%A handle is valid for the lifetime of the system thread corresponding to the OpenMP thread.
\glossarydefend

\glossaryterm{parallel handle}
\glossarydefstart
%The \emph{parallel handle} identifies an \emph{OpenMP parallel region}.
%It is allocated and initialized by various OMPD API calls.
%The handle is valid for the life time of the parallel region.
%The handle is guaranteed to be valid if at least one thread
%in the parallel region is paused, or if a thread in a nested
%parallel region is paused.
%Parallel handles are represented by the opaque type
%\texttt{ompd\_parallel\_handle\_t}, and created by the OMPD implementation
%which passes ownership to the debugger which is responsible for
%indicating when it no longer needs the handle.
%The debugger releases the parallel handle by calling
%\refdef{\texttt{ompd\_release\_parallel\_handle}}{release-parallel-handle:def}.
A handle that refers to an OpenMP parallel region.
%The handle is guaranteed to be valid if at least one thread in the parallel region is paused, or if a thread in a nested parallel region is  paused.
\glossarydefend

\glossaryterm{task handle}
\glossarydefstart
%The \emph{task handle} identifies an \emph{OpenMP task region}.
%It is allocated and initialized by various OMPD API calls.
%The handle is valid for the life time of the task region.
%The handle is guaranteed to be valid if all threads
%in the task team are paused.
%Task handles are represented by the opaque type
%\texttt{ompd\_task\_handle\_t}, and created by the OMPD implementation
%which passes ownership to the debugger which is responsible for
%indicating when it no longer needs the handle.
%The debugger releases the task handle by calling
%\refdef{\texttt{ompd\_release\_task\_handle}}{release-task-handle:def}.
A handle that refers to an OpenMP task region.
% The handle is valid for the life time of the  task region. The handle is guaranteed to be valid if all threads in the task team are paused.
\glossarydefend

\glossaryterm{descendent handle}
\glossarydefstart
An output handle that is returned from the OMPD library in a function that 
accepts an input handle: the output handle is a descendent of the input handle.
\glossarydefend

\glossaryterm{ancestor handle}
\glossarydefstart
An input handle that is passed to the OMPD library in a function that 
returns an output handle: the input handle is an ancestor of the 
output handle. For a given handle, the ancestors of the handle are 
also the ancestors of the handle's descendent.

\begin{quote}
	COMMENT: A handle cannot be used by the tool in an OMPD call if at least 
	one ancestor of the handle has been released, except for OMPD calls 
	that release the handle.
\end{quote}
\glossarydefend

\glossaryterm{tool context}
\glossarydefstart
%Debugger contexts are used to identify a process, address space, or thread
%object in the debugger. Contexts are passed from the debugger into various OMPD API calls,
%and then from the OMPD implementation back to the debugger's callback functions.
%For example, symbol lookup and memory accesses are done in the ``context''
%of a particular address space and possibly thread in the debugger.
%Contexts are opaque to the OMPD implementation, and defined by the debugger.
An opaque reference provided by a tool to an OMPD library. A tool context uniquely identifies an abstraction.
% a process, address space, or thread that resides in a third-party tool. Contexts are opaque to an OMPD library, and are defined by a third-party tool.
\glossarydefend

\glossaryterm{address space context}
\glossarydefstart
%The \emph{address space context} identifies the debugger object for a
%portion of an instance of an \emph{OpenMP program} that is running on
%a host or target device.
%An address space is contained within a process, and has an associated
%target architecture.
%The address space context must be valid for the life time of its
%associated address space handle.
%The host address space context is passed into the process initialization call
%\refdef{\texttt{ompd\_process\_initialize}}{process-initialize:def}
%to associate the host address space context with the address space handle.
%The OMPD implementation can assume that the address space context is valid
%until
%\refdef{\texttt{ompd\_release\_address\_space\_handle}}{release-address-space-handle:def}
%is called for the address space context passed into the initialization routine.
A tool context that refers to an address space within a process.
\glossarydefend

\glossaryterm{thread context}
\glossarydefstart
%The \emph{thread context} identifies the debugger object for a thread.
%The debugger owns and initializes the thread context.
%The OMPD implementation obtains a thread context using the
%\texttt{get\_thread\_context} callback.
%This callback allows the OMPD implementation to map an operating
%system thread ID to a debugger thread context.
%The OMPD implementation can assume that the thread context is valid
%for as long as the debugger is holding any references to thread handles
%that may contain the thread context.
A tool context that refers to a thread.
\glossarydefend

\glossaryterm{thread identifier}
\glossarydefstart
An identifier for a native thread defined by a thread implementation.
%on a device.
\glossarydefend


% This is an included file. See the master file for more information.
%
% When editing this file:
%
%    1. To change formatting, appearance, or style, please edit openmp.sty.
%
%    2. Custom commands and macros are defined in openmp.sty.
%
%    3. Be kind to other editors -- keep a consistent style by copying-and-pasting to
%       create new content.
%
%    4. We use semantic markup, e.g. (see openmp.sty for a full list):
%         \code{}     % for bold monospace keywords, code, operators, etc.
%         \plc{}      % for italic placeholder names, grammar, etc.
%
%    5. There are environments that provide special formatting, e.g. language bars.
%       Please use them whereever appropriate.  Examples are:
%
%         \begin{fortranspecific}
%         This is text that appears enclosed in blue language bars for Fortran.
%         \end{fortranspecific}
%
%         \begin{note}
%         This is a note.  The "Note -- " header appears automatically.
%         \end{note}
%
%    6. Other recommendations:
%         Use the convenience macros defined in openmp.sty for the minor headers
%         such as Comments, Syntax, etc.
%
%         To keep items together on the same page, prefer the use of
%         \begin{samepage}.... Avoid \parbox for text blocks as it interrupts line numbering.
%         When possible, avoid \filbreak, \pagebreak, \newpage, \clearpage unless that's
%         what you mean. Use \needspace{} cautiously for troublesome paragraphs.
%
%         Avoid absolute lengths and measures in this file; use relative units when possible.
%         Vertical space can be relative to \baselineskip or ex units. Horizontal space
%         can be relative to \linewidth or em units.
%
%         Prefer \emph{} to italicize terminology, e.g.:
%             This is a \emph{definition}, not a placeholder.
%             This is a \plc{var-name}.
%

\section{Execution Model}
\label{sec:Execution Model}
\index{execution model}
The OpenMP API uses the fork-join model of parallel execution. Multiple threads of
execution perform tasks defined implicitly or explicitly by OpenMP directives. The
OpenMP API is intended to support programs that will execute correctly both as parallel
programs (multiple threads of execution and a full OpenMP support library) and as
sequential programs (directives ignored and a simple OpenMP stubs library). However,
it is possible and permitted to develop a program that executes correctly as a parallel
program but not as a sequential program, or that produces different results when
executed as a parallel program compared to when it is executed as a sequential program.
Furthermore, using different numbers of threads may result in different numeric results
because of changes in the association of numeric operations. For example, a serial
addition reduction may have a different pattern of addition associations than a parallel
reduction. These different associations may change the results of floating-point addition.

An OpenMP program begins as a single thread of execution, called an initial thread. An
initial thread executes sequentially, as if the code encountered is part of an implicit task region, called an
initial task region, that is generated by the implicit parallel region surrounding the whole
program.

The thread that executes the implicit parallel region that surrounds the whole program
executes on the \emph{host device}. An implementation may support
other \emph{target devices}. If
supported, one or more devices are available to the host device for offloading code and
data. Each device has its own threads that are distinct from threads that execute on
another device. Threads cannot migrate from one device to another device. The
execution model is host-centric such that the host device offloads \code{target} regions to
target devices.

When a \code{target} construct is encountered, a new \emph{target task} is generated.
The \emph{target task} region encloses the \code{target} region. The \emph{target task} is
complete after the execution of the \code{target} region is complete.

When a \emph{target task} executes, the enclosed \code{target} region is executed by an initial
thread.  The initial thread may execute on a \emph{target device}.  The initial thread executes
sequentially, as if the target region is part of an initial task region that is
generated by an implicit parallel region. If the target device does not exist or the implementation does not support the target
device, all \code{target} regions associated with that device execute on the host device.

The implementation must ensure that the \code{target} region executes as if it were executed in the data environment of the target device unless an \code{if} clause is present and the \code{if} clause expression evaluates to \plc{false}.

The \code{teams} construct creates a \emph{league of teams}, where each team
is an initial team that comprises an initial thread that executes the
\code{teams} region. Each initial thread executes sequentially, as if the
code encountered is part of an initial task region that is generated by an
implicit parallel region associated with each team.

If a construct creates a data environment, the data environment is created at the time the
construct is encountered. Whether a construct creates a data environment is defined in
the description of the construct.

When any thread encounters a \code{parallel} construct, the thread creates a team of itself
and zero or more additional threads and becomes the master of the new team. A set of
implicit tasks, one per thread, is generated. The code for each task is defined by the code
inside the \code{parallel} construct. Each task is assigned to a different thread in the team
and becomes tied; that is, it is always executed by the thread to which it is initially
assigned. The task region of the task being executed by the encountering thread is
suspended, and each member of the new team executes its implicit task. There is an
implicit barrier at the end of the \code{parallel} construct. Only the master thread resumes
execution beyond the end of the \code{parallel} construct, resuming the task region that
was suspended upon encountering the \code{parallel} construct. Any number of
\code{parallel} constructs can be specified in a single program.

\code{parallel} regions may be arbitrarily nested inside each other. If nested parallelism is
disabled, or is not supported by the OpenMP implementation, then the new team that is
created by a thread encountering a \code{parallel} construct inside a \code{parallel} region
will consist only of the encountering thread. However, if nested parallelism is supported
and enabled, then the new team can consist of more than one thread. A \code{parallel}
construct may include a \code{proc_bind} clause to specify the places to use for the threads
in the team within the \code{parallel} region.

When any team encounters a worksharing construct, the work inside the construct is
divided among the members of the team, and executed cooperatively instead of being
executed by every thread. There is a default barrier at the end of each worksharing
construct unless the \code{nowait} clause is present. Redundant execution of code by every
thread in the team resumes after the end of the worksharing construct.

When any thread encounters a \emph{task generating construct}, one or more explicit tasks are generated.
Execution of explicitly generated tasks is assigned to one of the threads in the current
team, subject to the thread's availability to execute work. Thus, execution of the new
task could be immediate, or deferred until later according to task scheduling constraints
and thread availability. Threads are allowed to suspend the current task region at a task
scheduling point in order to execute a different task. If the suspended task region is for
a tied task, the initially assigned thread later resumes execution of the suspended task
region. If the suspended task region is for an untied task, then any thread may resume its
execution. Completion of all explicit tasks bound to a given parallel region is guaranteed
before the master thread leaves the implicit barrier at the end of the region. Completion
of a subset of all explicit tasks bound to a given parallel region may be specified through
the use of task synchronization constructs. Completion of all explicit tasks bound to the
implicit parallel region is guaranteed by the time the program exits.

When any thread encounters a \code{simd} construct, the iterations of the loop associated with
the construct may be executed concurrently using the SIMD lanes that are available to
the thread.

When a \code{loop} construct is encountered, the iterations of the loop(s)
associated with the construct are executed in the context of its binding
region, if defined, and otherwise are executed once per thread that encounters
the construct. If the \code{loop} region binds to a \code{teams} region, the
encountering thread(s) may continue execution after the \code{loop} region
without waiting for all iterations to complete; the iterations are
guaranteed to complete before the end of \code{teams} region.  Otherwise, all
iterations must complete before the encountering thread(s) continue execution
after the \code{loop} region.

Any thread that encounters the \code{loop} construct may participate in the
execution of the iterations of its associated loop(s). If the \code{nodep}
clause is present on the construct, any participating thread may create a
team of threads, as if from a \code{parallel} construct, that may also
participate in the execution of the iterations.

%The set of iterations are partitioned into iteration subsets,
%where the iterations in each subset execute on the same thread or SIMD lane.
%Iterations in a given iteration subset execute as if enclosed by the same task.


%an implementation may
%choose to apply optimizations, parallelization, and/or vectorization, including
%scheduling loop iterations in ways that cannot be directly represented in
%OpenMP. When encountering a \code{concurrent} construct an implementation
%will not generate a new league of thread teams, but may use any thread teams
%currently available on which to execute.

The \code{cancel} construct can alter the previously described flow of execution in an
OpenMP region. The effect of the \code{cancel} construct depends on its
\plc{construct-type-clause}. If a task encounters a \code{cancel}
construct with a \code{taskgroup}
\plc{construct-type-clause}, then the task activates cancellation
and continues execution at the end of its
\code{task} region, which implies completion of that task.
Any other task in that \code{taskgroup}
that has begun executing completes execution unless it encounters a \code{cancellation}
\code{point} construct, in which case it continues execution at the end of its \code{task} region,
which implies its completion. Other tasks in that \code{taskgroup} region that have not
begun execution are aborted, which implies their completion.

For all other \plc{construct-type-clause} values, if a
thread encounters a \code{cancel} construct, it
activates cancellation of the innermost enclosing region of the type specified and the
thread continues execution at the end of that region. Threads check if cancellation has
been activated for their region at cancellation points and, if so, also resume execution at
the end of the canceled region.

If cancellation has been activated regardless of \plc{construct-type-clause},
threads that are
waiting inside a barrier other than an implicit barrier at the end of the canceled region
exit the barrier and resume execution at the end of the canceled region. This action can
occur before the other threads reach that barrier.

Synchronization constructs and library routines are available in the OpenMP API to
coordinate tasks and data access in \code{parallel} regions. In addition, library routines and
environment variables are available to control or to query the runtime environment of
OpenMP programs.

The OpenMP specification makes no guarantee that input or output to the same file is
synchronous when executed in parallel. In this case, the programmer is responsible for
synchronizing input and output statements (or routines) using the provided
synchronization constructs or library routines. For the case where each thread accesses a
different file, no synchronization by the programmer is necessary.

% This is an included file. See the master file for more information.
%
% When editing this file:
%
%    1. To change formatting, appearance, or style, please edit openmp.sty.
%
%    2. Custom commands and macros are defined in openmp.sty.
%
%    3. Be kind to other editors -- keep a consistent style by copying-and-pasting to
%       create new content.
%
%    4. We use semantic markup, e.g. (see openmp.sty for a full list):
%         \code{}     % for bold monospace keywords, code, operators, etc.
%         \plc{}      % for italic placeholder names, grammar, etc.
%
%    5. There are environments that provide special formatting, e.g. language bars.
%       Please use them whereever appropriate.  Examples are:
%
%         \begin{fortranspecific}
%         This is text that appears enclosed in blue language bars for Fortran.
%         \end{fortranspecific}
%
%         \begin{note}
%         This is a note.  The "Note -- " header appears automatically.
%         \end{note}
%
%    6. Other recommendations:
%         Use the convenience macros defined in openmp.sty for the minor headers
%         such as Comments, Syntax, etc.
%
%         To keep items together on the same page, prefer the use of
%         \begin{samepage}.... Avoid \parbox for text blocks as it interrupts line numbering.
%         When possible, avoid \filbreak, \pagebreak, \newpage, \clearpage unless that's
%         what you mean. Use \needspace{} cautiously for troublesome paragraphs.
%
%         Avoid absolute lengths and measures in this file; use relative units when possible.
%         Vertical space can be relative to \baselineskip or ex units. Horizontal space
%         can be relative to \linewidth or em units.
%
%         Prefer \emph{} to italicize terminology, e.g.:
%             This is a \emph{definition}, not a placeholder.
%             This is a \plc{var-name}.
%

\section{Memory Model}
\label{sec:Memory Model}
\index{memory model}
\subsection{Structure of the OpenMP Memory Model}
\label{subsec:Structure of the OpenMP Memory Model}
The OpenMP API provides a relaxed-consistency, shared-memory model. All OpenMP
threads have access to a place to store and to retrieve variables,
called the \emph{memory}. In
addition, each thread is allowed to have its own \emph{temporary view} of the memory. The
temporary view of memory for each thread is not a required part of the OpenMP
memory model, but can represent any kind of intervening structure, such as machine
registers, cache, or other local storage, between the thread and the memory. The
temporary view of memory allows the thread to cache variables and thereby to avoid
going to memory for every reference to a variable. Each thread also has access to
another type of memory that must not be accessed by other threads,
called \emph{threadprivate memory}.

A directive that accepts data-sharing attribute clauses determines two kinds of access to
variables used in the directive's associated structured block: shared and private. Each
variable referenced in the structured block has an original variable, which is the variable
by the same name that exists in the program immediately outside the construct. Each
reference to a shared variable in the structured block becomes a reference to the original
variable. For each private variable referenced in the structured block, a new version of
the original variable (of the same type and size) is created in memory for each task or
SIMD lane that contains code associated with the directive. Creation of the new version
does not alter the value of the original variable. However, the impact of attempts to
access the original variable during the region corresponding to the directive is
unspecified; see \specref{subsubsec:private clause} for additional details. References to a
private variable in the structured block refer to the private version of the original
variable for the current task or SIMD lane. The relationship between the value of the
original variable and the initial or final value of the private version depends on the exact
clause that specifies it. Details of this issue, as well as other issues with privatization,
are provided in \specref{sec:Data Environment}.

The minimum size at which a memory update may also read and write back adjacent
variables that are part of another variable (as array or structure elements) is
implementation defined but is no larger than required by the base language.

A single access to a variable may be implemented with multiple load or store
instructions, and hence is not guaranteed to be atomic with respect to other accesses to
the same variable. Accesses to variables smaller than the implementation defined
minimum size or to C or C++ bit-fields may be implemented by reading, modifying, and
rewriting a larger unit of memory, and may thus interfere with updates of variables or
fields in the same unit of memory.

If multiple threads write without synchronization to the same memory unit, including
cases due to atomicity considerations as described above, then a data race occurs.
Similarly, if at least one thread reads from a memory unit and at least one thread writes
without synchronization to that same memory unit, including cases due to atomicity
considerations as described above, then a data race occurs. If a data race occurs then the
result of the program is unspecified.

A private variable in a task region that eventually generates an inner nested \code{parallel}
region is permitted to be made shared by implicit tasks in the inner \code{parallel} region.
A private variable in a task region can be shared by an explicit task region generated
during its execution. However, it is the programmer's responsibility to ensure through
synchronization that the lifetime of the variable does not end before completion of the
explicit task region sharing it. Any other access by one task to the
private variables of another task results in unspecified behavior.




\subsection{Device Data Environments}
\label{subsec:Device Data Environments}
\index{device data environments}
When an OpenMP program begins, an implicit \code{target}~\code{data} region for each device surrounds the whole program. Each device has a device data environment that is defined by its implicit \code{target}~\code{data} region. Any \code{declare}~\code{target} directives and the directives that accept data-mapping attribute clauses determine how an original variable in a data environment is mapped to a corresponding variable in a device data environment.

When an original variable is mapped to a device data environment and the associated corresponding variable is not present in the device data environment, a new corresponding variable (of the same type and size as the original variable) is created in the device data environment. The initial value of the new corresponding variable is determined from the clauses and the data environment of the encountering thread.

The corresponding variable in the device data environment may share storage with the
original variable. Writes to the corresponding variable may alter the value of the original
variable. The impact of this on memory consistency is discussed in
\specref{subsec:OpenMP Memory Consistency}.
When a task executes in the context of a device data environment, references to
the original variable refer to the corresponding variable in the device data environment.
If a corresponding variable does not exist in the device data
environment then accesses to the original variable result in
unspecified behavior unless the unified\_shared\_memory requirement
is specified.

The relationship between the value of the original variable and the initial or final value
of the corresponding variable depends on the \plc{map-type}. Details of this issue, as well as
other issues with mapping a variable, are provided in \specref{subsec:map Clause}.

The original variable in a data environment and the corresponding variable(s) in one or
more device data environments may share storage. Without intervening synchronization
data races can occur.

\subsection{Memory Management}

The host device, and target devices that an implementation may support, have attached storage resources where program variables are stored. These resources can be of different kinds.

An OpenMP program can use a memory allocator to allocate storage for its variables. Memory allocators are associated with certain storage resources and use that storage to allocate variables. Memory allocators are also used to deallocate variables and free the storage in the resources. When an OpenMP memory allocator is not used variables can be allocated in any storage resource.
%The behavior of a memory management construct, clause or API is unspecified if the variable that is applied to was not allocated with an OpenMP memory allocator.


\subsection{The Flush Operation}
\label{subsec:The Flush Operation}
\index{flush operation}
\index{strong flush}
\index{flush-set}


The memory model has relaxed-consistency because a thread's temporary view of
memory is not required to be consistent with memory at all times. A value
written to a variable can remain in the thread's temporary view until it is
forced to memory at a later time. Likewise, a read from a variable may
retrieve the value from the thread's temporary view, unless it is forced to
read from memory. The OpenMP flush operation enforces consistency between
multiple threads' view of memory.

If a flush operation is a strong flush, it enforces consistency between a
thread's temporary view and memory.  A strong flush operation is applied to a
set of variables called the \emph{flush-set}. A strong flush restricts
reordering of memory operations that an implementation might otherwise do.
Implementations must not reorder the code for a memory operation for a given
variable, or the code for a flush operation for the variable, with respect to
a strong flush operation that refers to the same variable.

If a thread has performed a write to its temporary view of a shared variable
since its last strong flush of that variable, then when it executes another
strong flush of the variable, the strong flush does not complete until the
value of the variable has been written to the variable in memory. If a thread
performs multiple writes to the same variable between two strong flushes of
that variable, the strong flush ensures that the value of the last write is
written to the variable in memory. A strong flush of a variable executed by a
thread also causes its temporary view of the variable to be discarded, so that
if its next memory operation for that variable is a read, then the thread will
read from memory when it may again capture the value in the temporary view.
When a thread executes a strong flush, no later memory operation by that
thread for a variable involved in that strong flush is allowed to start until
the strong flush completes.  The completion of a strong flush of a set of
variables executed by a thread is defined as the point at which all writes to
those variables performed by the thread before the strong flush are visible in
memory to all other threads and that thread’s temporary view of all variables
involved is discarded.

A strong flush operation provides a guarantee of consistency between a thread’s
temporary view and memory. Therefore, a strong flush can be used to
guarantee that a value written to a variable by one thread may be read by a
second thread. To accomplish this, the programmer must ensure that the second
thread has not written to the variable since its last strong flush of the
variable, and that the following sequence of events are completed in the specified
order:

\begin{enumerate}
    \item The value is written to the variable by the first thread.
    \item The variable is flushed, with a strong flush, by the first thread.
    \item The variable is flushed, with a strong flush, by the second thread.
    \item The value is read from the variable by the second thread.
\end{enumerate}

If a flush operation is a release flush or acquire flush, it can enforce
consistency between two synchronizing threads' view of memory.  A release
flush guarantees that any prior operation that writes or reads a shared
variable will appear to be completed before any operation that writes or reads
the same shared variable that follows an acquire flush with which it
synchronizes (see Section \specref{subsec:happens-before} for more details on
flush synchronization).  A release flush will propagate the values of all
shared variables in its temporary view to memory prior to the thread
performing any subsequent atomic operation that may establish a
synchronization. An acquire flush will discard any value of a shared variable
in its temporary view to which the thread has not written since last performing
a release flush, so that it may subsequently read a value propagated by a
release flush that synchronizes with it.   Therefore, release and acquire
flushes may also be used to guarantee that a value written to a variable by
one thread may be read by a second thread. To accomplish this, the programmer
must ensure that the second thread has not written to the variable since its
last acquire flush, and that the following sequence of events happen in the
specified order:

\begin{enumerate}
    \item The value is written to the variable by the first thread.
    \item The first thread performs a release flush.
    \item The second thread performs an acquire flush.
    \item The value is read from the variable by the second thread.
\end{enumerate}

\begin{note}
OpenMP synchronization operations, described in 
\specref{sec:Synchronization Constructs and Clauses} and in \specref{sec:Lock Routines}, 
are recommended for enforcing this order. Synchronization 
through variables is possible but is not recommended because the proper timing
of flushes is difficult.
\end{note}


\subsection{Flush Synchronization and Happens Before}
\label{subsec:happens-before}
\index{flush synchronization}
\index{release flush}
\index{acquire flush}
\index{happens before}

OpenMP supports thread synchronization with the use of release flushes and
acquire flushes. For any such synchronization, a release flush is the source
of the synchronization and an acquire flush is the sink of the
synchronization, such that the release flush \emph{synchronizes with} the
acquire flush.

A release flush has one or more associated \emph{release sequences} that
define the set of a modifications that may be used to establish a
synchronization. Any such release sequence starts with an atomic operation
that follows the release flush and modifies a shared variable and additionally
includes any read-modify-write atomic operations that read a value taken from
some modification in the release sequence. The atomic operations that start an
associated release sequence are determined as follows:

\begin{itemize}
\item If a release flush is performed on entry to an atomic operation, that
    atomic operation starts its release sequence.
\item If a release flush is performed by an implicit \code{flush} region, some
    atomic operation performed by the implementation on an internal
    synchronization variable starts its release sequence.
\item If a release flush is performed by an explicit \code{flush} region, any
    atomic operation that modifies a shared variable and follows the
    \code{flush} region in its thread's program order starts an associated
    release sequence.
\end{itemize}

An acquire flush is associated with one or more prior atomic operations that
read a shared variable and that may be used to establish a synchronization.
The associated atomic operations that may establish a synchronization are
determined as follows:

\begin{itemize}
\item If an acquire flush is performed on exit from an atomic operation, that
    atomic operation is its associated atomic operation.
\item If an acquire flush is performed by an implicit \code{flush} region, some
    atomic operation performed by the implementation that reads an internal
    synchronization variable is its associated atomic operation.
\item If an acquire flush is performed by an explicit \code{flush} region, any
    atomic operation that reads a shared variable and precedes the
    \code{flush} region in its thread's program order is an associated atomic
    operation.
\end{itemize}

A release flush synchronizes with an acquire flush if an atomic operation
associated with the acquire flush reads a value written by a modification from
a release sequence associated with the release flush.

An operation \plc{X} \emph{simply~happens~before} an operation \plc{Y} if any of the following conditions are satisfied:
\begin{enumerate}
\item \plc{X} and \plc{Y} are performed by the same thread, and \plc{X} precedes \plc{Y} in the thread's program order.
\item \plc{X} synchronizes with \plc{Y} according to the flush synchronization conditions explained above or according
    to the base language's definition of \emph{synchronizes~with}, if such a definition exists.
\item There exists another operation \plc{Z}, such that \plc{X} simply happens before \plc{Z} and \plc{Z} simply happens before \plc{Y}.
\end{enumerate}

An operation \plc{X} \emph{happens~before} an operation \plc{Y} if any of the following conditions are satisfied:
\begin{enumerate}
\item \plc{X} happens before \plc{Y} according to the base language's definition of \emph{happens~before}, if such a definition exists.
\item \plc{X} simply happens before \plc{Y}.
\end{enumerate}

A variable with an initial value is treated as if the value is stored to the
variable by an operation that happens before all operations that access or
modify the variable in the program.

\subsection{OpenMP Memory Consistency}
\label{subsec:OpenMP Memory Consistency}

The observable completion order of memory operations, as seen by all threads, is
guaranteed according to the following rules:

\begin{itemize}
\item If two operations performed by different threads are sequentially
    consistent atomic operations or they are strong flushes that flush the
    same variable, then they must be completed as if in some sequential order,
    seen by all threads.

\item If two operations performed by the same thread are sequentially
    consistent atomic operations or they access, modify, or, with a strong
    flush, flush the same variable, then they must be completed as if in that
    thread's program order, as seen by all threads.

\item If two operations are performed by different threads and one happens
    before the other, then they must be completed as if in that happens before
    order, as seen by all threads, if:
    \begin{itemize}
        \item both operations access or modify the same variable,
        \item both operations are strong flushes that flush the same variable, or
        \item both operations are sequentially consistent atomic operations.
    \end{itemize}

\item Any two atomic memory operations from different \code{atomic} regions
    must be completed as if in the same order as the strong flushes
    implied in their respective regions, as seen by all threads.
\end{itemize}

The flush operation can be specified using the \code{flush} directive, and is also implied at
various locations in an OpenMP program: see \specref{subsec:flush Construct} for details.

\begin{note}
Since flush operations by themselves cannot prevent data races, explicit flush
operations are only useful in combination with non-sequentially consistent atomic
directives.
\end{note}

OpenMP programs that:

\begin{itemize}[rightmargin=11ex]
\item do not use non-sequentially consistent atomic directives,

\item do not rely on the accuracy of a \plc{false} result from
\code{omp_test_lock} and \code{omp_test_nest_lock}, and

\item correctly avoid data races as required in \specref{subsec:Structure of the OpenMP Memory Model}
\end{itemize}

behave as though operations on shared variables were simply interleaved in an order
consistent with the order in which they are performed by each thread. The relaxed
consistency model is invisible for such programs, and any explicit flush operations in
such programs are redundant.

% This is an included file. See the master file for more information.
%
% When editing this file:
%
%    1. To change formatting, appearance, or style, please edit openmp.sty.
%
%    2. Custom commands and macros are defined in openmp.sty.
%
%    3. Be kind to other editors -- keep a consistent style by copying-and-pasting to
%       create new content.
%
%    4. We use semantic markup, e.g. (see openmp.sty for a full list):
%         \code{}     % for bold monospace keywords, code, operators, etc.
%         \plc{}      % for italic placeholder names, grammar, etc.
%
%    5. There are environments that provide special formatting, e.g. language bars.
%       Please use them whereever appropriate.  Examples are:
%
%         \begin{fortranspecific}
%         This is text that appears enclosed in blue language bars for Fortran.
%         \end{fortranspecific}
%
%         \begin{note}
%         This is a note.  The "Note -- " header appears automatically.
%         \end{note}
%
%    6. Other recommendations:
%         Use the convenience macros defined in openmp.sty for the minor headers
%         such as Comments, Syntax, etc.
%
%         To keep items together on the same page, prefer the use of
%         \begin{samepage}.... Avoid \parbox for text blocks as it interrupts line numbering.
%         When possible, avoid \filbreak, \pagebreak, \newpage, \clearpage unless that's
%         what you mean. Use \needspace{} cautiously for troublesome paragraphs.
%
%         Avoid absolute lengths and measures in this file; use relative units when possible.
%         Vertical space can be relative to \baselineskip or ex units. Horizontal space
%         can be relative to \linewidth or em units.
%
%         Prefer \emph{} to italicize terminology, e.g.:
%             This is a \emph{definition}, not a placeholder.
%             This is a \plc{var-name}.
%

\section{Tool Interfaces}
\label{subsec:Tool Support}

The OpenMP API includes two tool interfaces, OMPT and OMPD,
to enable development of high-quality, portable, tools that support
monitoring, performance, or correctness analysis and debugging of OpenMP programs
developed using any implementation of the OpenMP API,


\subsection{OMPT}

The OMPT interface, which is intended for \emph{first-party} tools,
provides the following:
\begin{itemize}
\item A mechanism to initialize a first-party tool;
\item Routines that enable a tool to determine the capabilities of an
      OpenMP implementation;
\item Routines that enable a tool to examine OpenMP state information 
      associated with a thread;
\item Mechanisms that enable a tool to map implementation-level calling
      contexts back to their source-level representations;
\item A callback interface that enables a tool to receive notification
      of OpenMP \emph{events};
\item A tracing interface that enables a tool to trace activity on OpenMP 
      target devices; and
\item A runtime library routine that an application can use to control a tool.
\end{itemize}

OpenMP implementations may differ with respect to the \emph{thread states} that
they support, the mutual exclusion implementations that they employ,
and the OpenMP events for which tool callbacks are invoked. For some OpenMP events,
OpenMP implementations must guarantee that a registered callback will be invoked 
for each occurrence of the event. For other OpenMP events, OpenMP implementations 
are permitted to invoke a registered callback for some or no occurrences of the 
event; for such OpenMP events, however, OpenMP implementations are encouraged to 
invoke tool callbacks on as many occurrences of the event as is practical.
Section~\ref{sec:ompt-register-callbacks} specifies the subset of OMPT
callbacks that an OpenMP implementation must support for a minimal
implementation of the OMPT interface.

An implementation of the OpenMP API may differ from the
abstract execution model described by its specification.  The ability
of tools that use the OMPT interface to observe such differences does 
not constrain implementations of the OpenMP API in any way.

With the exception of the \code{omp_control_tool} runtime library routine for 
tool control, all other routines in the OMPT interface are intended for use 
only by tools and are not visible to applications. For that reason, a Fortran 
binding is provided only for \code{omp_control_tool}; all other OMPT functionality 
is described with C syntax only.

\subsection{OMPD}

The OMPD interface is intended for a \emph{third-party} tool, which runs as a 
separate process. An OpenMP implementation must provide an OMPD
library that can be dynamically loaded and used by a third-party tool.
A third-party tool, such as a debugger, uses the OMPD library to access
OpenMP state of a program that has begun execution. OMPD defines the following:

\begin{itemize}
\item An interface that an OMPD library exports, which a tool can use 
      to access OpenMP state of a program that has begun execution;
\item A callback interface that a tool provides to the OMPD library so 
      that the library can use it to access the OpenMP state of a program 
      that has begun execution; and
\item A small number of symbols that must be defined by an OpenMP 
      implementation to help the tool find the correct OMPD library to use 
      for that OpenMP implementation and to facilitate notification of events.
\end{itemize}
Section~\ref{sec:ompd-overview} describes OMPD in detail.


% This is an included file. See the master file for more information.
%
% When editing this file:
%
%    1. To change formatting, appearance, or style, please edit openmp.sty.
%
%    2. Custom commands and macros are defined in openmp.sty.
%
%    3. Be kind to other editors -- keep a consistent style by copying-and-pasting to
%       create new content.
%
%    4. We use semantic markup, e.g. (see openmp.sty for a full list):
%         \code{}     % for bold monospace keywords, code, operators, etc.
%         \plc{}      % for italic placeholder names, grammar, etc.
%
%    5. There are environments that provide special formatting, e.g. language bars.
%       Please use them whereever appropriate.  Examples are:
%
%         \begin{fortranspecific}
%         This is text that appears enclosed in blue language bars for Fortran.
%         \end{fortranspecific}
%
%         \begin{note}
%         This is a note.  The "Note -- " header appears automatically.
%         \end{note}
%
%    6. Other recommendations:
%         Use the convenience macros defined in openmp.sty for the minor headers
%         such as Comments, Syntax, etc.
%
%         To keep items together on the same page, prefer the use of
%         \begin{samepage}.... Avoid \parbox for text blocks as it interrupts line numbering.
%         When possible, avoid \filbreak, \pagebreak, \newpage, \clearpage unless that's
%         what you mean. Use \needspace{} cautiously for troublesome paragraphs.
%
%         Avoid absolute lengths and measures in this file; use relative units when possible.
%         Vertical space can be relative to \baselineskip or ex units. Horizontal space
%         can be relative to \linewidth or em units.
%
%         Prefer \emph{} to italicize terminology, e.g.:
%             This is a \emph{definition}, not a placeholder.
%             This is a \plc{var-name}.
%

\section{OpenMP Compliance}
\label{sec:OpenMP Compliance}
\index{OpenMP compliance}
\index{compliance}
The OpenMP API defines constructs that operate in the context of the
base language that is supported by an implementation. If the
implementation of the base language does not support a language
construct that appears in this document, a compliant OpenMP
implementation is not required to support it, with the exception that
for Fortran, the implementation must allow case insensitivity for
directive and API routines names, and must allow identifiers of more
than six characters. An implementation of the OpenMP API is compliant
if and only if it compiles and executes all other conforming programs,
and supports the tool interface, according to the syntax and semantics
laid out in Chapters 1, 2, 3, 4 and 5. Appendices A, B, C, and D, 
as well as sections designated as Notes (see \specref{sec:Organization
 of this document}) are for information purposes only and are not
part of the specification.

All library, intrinsic and built-in routines provided by the base language must be
thread-safe in a compliant implementation. In addition, the implementation of the base
language must also be thread-safe. For example, \code{ALLOCATE} and \code{DEALLOCATE}
statements must be thread-safe in Fortran. Unsynchronized concurrent use of such
routines by different threads must produce correct results (although not necessarily the
same as serial execution results, as in the case of random number generation routines).

Starting with Fortran 90, variables with explicit initialization have the \code{SAVE} attribute
implicitly. This is not the case in Fortran 77. However, a compliant OpenMP Fortran
implementation must give such a variable the \code{SAVE} attribute, regardless of the
underlying base language version.

Appendix~\ref{chap:OpenMP Implementation-Defined Behaviors}
lists certain aspects of the OpenMP API that are implementation defined. A
compliant implementation must define and document its behavior for each of
the items in Appendix~\ref{chap:OpenMP Implementation-Defined Behaviors}.


% This is an included file. See the master file for more information.
%
% When editing this file:
%
%    1. To change formatting, appearance, or style, please edit openmp.sty.
%
%    2. Custom commands and macros are defined in openmp.sty.
%
%    3. Be kind to other editors -- keep a consistent style by copying-and-pasting to
%       create new content.
%
%    4. We use semantic markup, e.g. (see openmp.sty for a full list):
%         \code{}     % for bold monospace keywords, code, operators, etc.
%         \plc{}      % for italic placeholder names, grammar, etc.
%
%    5. There are environments that provide special formatting, e.g. language bars.
%       Please use them whereever appropriate.  Examples are:
%
%         \begin{fortranspecific}
%         This is text that appears enclosed in blue language bars for Fortran.
%         \end{fortranspecific}
%
%         \begin{note}
%         This is a note.  The "Note -- " header appears automatically.
%         \end{note}
%
%    6. Other recommendations:
%         Use the convenience macros defined in openmp.sty for the minor headers
%         such as Comments, Syntax, etc.
%
%         To keep items together on the same page, prefer the use of
%         \begin{samepage}.... Avoid \parbox for text blocks as it interrupts line numbering.
%         When possible, avoid \filbreak, \pagebreak, \newpage, \clearpage unless that's
%         what you mean. Use \needspace{} cautiously for troublesome paragraphs.
%
%         Avoid absolute lengths and measures in this file; use relative units when possible.
%         Vertical space can be relative to \baselineskip or ex units. Horizontal space
%         can be relative to \linewidth or em units.
%
%         Prefer \emph{} to italicize terminology, e.g.:
%             This is a \emph{definition}, not a placeholder.
%             This is a \plc{var-name}.
%

\section{Normative References}
\index{normative references}
\label{sec:normative references}
\begin{itemize}
\item ISO/IEC 9899:1990, \textsl{Information Technology - Programming Languages - C}.

This OpenMP API specification refers to ISO/IEC 9899:1990 as C90.

\item ISO/IEC 9899:1999, \textsl{Information Technology - Programming Languages - C}.

This OpenMP API specification refers to ISO/IEC 9899:1999 as C99.

\item ISO/IEC 9899:2011, \textsl{Information Technology - Programming Languages - C}.

This OpenMP API specification refers to ISO/IEC 9899:2011 as C11. The
following features are not supported:

\begin{itemize}
\item Supporting the noreturn property
\item Adding alignment support
\item Creation of complex value
\item Abandoning a process (adding \code{quick_exit} and \code{at_quick_exit})
\item Threads for the C standard library
\item Thread-local storage
\item Parallel memory sequencing model
\item Atomic
\end{itemize}

\item ISO/IEC 14882:1998, \textsl{Information Technology - Programming Languages - C++}.

This OpenMP API specification refers to ISO/IEC 14882:1998 as C++.

\item ISO/IEC 14882:2011, \textsl{Information Technology - Programming Languages - C++}.

This OpenMP API specification refers to ISO/IEC 14882:2011 as
C++11. The following features are not supported:

\begin{itemize}
\item Alignment support
\item Standard layout types
\item Allowing move constructs to throw
\item Defining move special member functions
\item Concurrency
\item Data-dependency ordering: atomics and memory model
\item Additions to the standard library
\item Thread-local storage
\item Dynamic initialization and destruction with concurrency
\item C++11 library
\end{itemize}

\item ISO/IEC 14882:2014, \textsl{Information Technology - Programming Languages - C++}.

This OpenMP API specification refers to ISO/IEC 14882:2014 as
C++14. The following features are not supported:

\begin{itemize}
\item Sized deallocation
\item What signal handlers can do
\end{itemize}

\item ISO/IEC 14882:2017, \textsl{Information Technology - Programming Languages - C++}.

This OpenMP API specification refers to ISO/IEC 14882:2017 as
C++17. 

\item ISO/IEC 1539:1980, \textsl{Information Technology - Programming Languages - Fortran}.

This OpenMP API specification refers to ISO/IEC 1539:1980 as Fortran 77.

\item ISO/IEC 1539:1991, \textsl{Information Technology - Programming Languages - Fortran}.

This OpenMP API specification refers to ISO/IEC 1539:1991 as Fortran 90.

\item ISO/IEC 1539-1:1997, \textsl{Information Technology - Programming Languages - Fortran}.

This OpenMP API specification refers to ISO/IEC 1539-1:1997 as Fortran 95.

\item ISO/IEC 1539-1:2004, \textsl{Information Technology - Programming Languages - Fortran}.

This OpenMP API specification refers to ISO/IEC 1539-1:2004 as Fortran 2003.

\item ISO/IEC 1539-1:2010, \textsl{Information Technology - Programming Languages - Fortran}.

This OpenMP API specification refers to ISO/IEC 1539-1:2010 as Fortran 2008. The
following features are not supported:

\begin{itemize}
\item Submodules
\item Coarrays
\item DO CONCURRENT
\item Allocatable components of recursive type
\item Pointer initialization
\item Value attribute is permitted for any nonallocatable nonpointer nonarray
\item Simply contiguous arrays rank remapping to rank>1 target
\item Polymorphic assignment
\item Accessing real and imaginary parts
\item Pointer function reference is a variable
\item Recursive I/O
\item The BLOCK construct
\item EXIT statement
\item ERROR STOP
\item Internal procedure as an actual argument
\item Generic resolution by procedureness
\item Generic resolution by pointer vs. allocatable
\item Impure elemental procedures
\end{itemize}

\end{itemize}

Where this OpenMP API specification refers to C, C++ or Fortran, reference is made to
the base language supported by the implementation.


% This is an included file. See the master file for more information.
%
% When editing this file:
%
%    1. To change formatting, appearance, or style, please edit openmp.sty.
%
%    2. Custom commands and macros are defined in openmp.sty.
%
%    3. Be kind to other editors -- keep a consistent style by copying-and-pasting to
%       create new content.
%
%    4. We use semantic markup, e.g. (see openmp.sty for a full list):
%         \code{}     % for bold monospace keywords, code, operators, etc.
%         \plc{}      % for italic placeholder names, grammar, etc.
%
%    5. There are environments that provide special formatting, e.g. language bars.
%       Please use them whereever appropriate.  Examples are:
%
%         \begin{fortranspecific}
%         This is text that appears enclosed in blue language bars for Fortran.
%         \end{fortranspecific}
%
%         \begin{note}
%         This is a note.  The "Note -- " header appears automatically.
%         \end{note}
%
%    6. Other recommendations:
%         Use the convenience macros defined in openmp.sty for the minor headers
%         such as Comments, Syntax, etc.
%
%         To keep items together on the same page, prefer the use of
%         \begin{samepage}.... Avoid \parbox for text blocks as it interrupts line numbering.
%         When possible, avoid \filbreak, \pagebreak, \newpage, \clearpage unless that's
%         what you mean. Use \needspace{} cautiously for troublesome paragraphs.
%
%         Avoid absolute lengths and measures in this file; use relative units when possible.
%         Vertical space can be relative to \baselineskip or ex units. Horizontal space
%         can be relative to \linewidth or em units.
%
%         Prefer \emph{} to italicize terminology, e.g.:
%             This is a \emph{definition}, not a placeholder.
%             This is a \plc{var-name}.
%

\section{Organization of this Document}
\label{sec:Organization of this document}
The remainder of this document is structured as follows:

\begin{itemize}
\item Chapter \ref{chap:Directives} ``Directives''

\item Chapter \ref{chap:Runtime Library Routines} ``Runtime Library Routines''

\item Chapter \ref{chap:ToolsSupport} ``Tool Interface''

\item Chapter \ref{chap:Environment Variables} ``Environment Variables''

\item Appendix \ref{chap:OpenMP Implementation-Defined Behaviors} ``OpenMP Implementation-Defined Behaviors''

\item Appendix \ref{chap:frames} ``Task Frame Management for the Tool Interface''

\item Appendix \ref{chap:ompd_diagram} ``Interaction Diagram of OMPD Components''

\item Appendix \ref{chap:Features History} ``Features History''
\end{itemize}

Some sections of this document only apply to programs written in a certain base
language. Text that applies only to programs for which the base language is C or C++ is shown
as follows:

\begin{ccppspecific}
C/C++ specific text...
\end{ccppspecific}

Text that applies only to programs for which the base language is C only is shown as follows:

\begin{cspecific}
C specific text...
\end{cspecific}

Text that applies only to programs for which the base language is C90 only is shown as
follows:

\begin{c90specific}
C90 specific text...
\end{c90specific}

Text that applies only to programs for which the base language is C99 only is shown as
follows:

\begin{c99specific}
C99 specific text...
\end{c99specific}

Text that applies only to programs for which the base language is C++ only is shown as
follows:

\begin{cppspecific}
C++ specific text...
\end{cppspecific}

Text that applies only to programs for which the base language is Fortran is shown as follows:

\begin{fortranspecific}
Fortran specific text......
\end{fortranspecific}

Where an entire page consists of base language specific text, a marker is shown
at the top of the page.  For Fortran-specific text, the marker is:

\bigskip
\linewitharrows{-1}{dashed}{Fortran (cont.)}
\bigskip

For C/C++-specific text, the marker is:

\bigskip
\linewitharrows{-1}{dashed}{C/C++ (cont.)}
\bigskip

Some text is for information only, and is not part of the normative specification. Such
text is designated as a note, like this:

\needspace{6\baselineskip}\begin{note}
Non-normative text...
\end{note}




    % This is ch2-directives.tex of the OpenMP specification.
% This is an included file. See the master file for more information.
%
% When editing this file:
%
%    1. To change formatting, appearance, or style, please edit openmp.sty.
%
%    2. Custom commands and macros are defined in openmp.sty.
%
%    3. Be kind to other editors -- keep a consistent style by copying-and-pasting to
%       create new content.
%
%    4. We use semantic markup, e.g. (see openmp.sty for a full list):
%         \code{}     % for bold monospace keywords, code, operators, etc.
%         \plc{}      % for italic placeholder names, grammar, etc.
%
%    5. Other recommendations:
%         Use the convenience macros defined in openmp.sty for the minor headers
%         such as Comments, Syntax, etc.
%
%         To keep items together on the same page, prefer the use of 
%         \begin{samepage}.... Avoid \parbox for text blocks as it interrupts line numbering.
%         When possible, avoid \filbreak, \pagebreak, \newpage, \clearpage unless that's
%         what you mean. Use \needspace{} cautiously for troublesome paragraphs.
%
%         Avoid absolute lengths and measures in this file; use relative units when possible.
%         Vertical space can be relative to \baselineskip or ex units. Horizontal space
%         can be relative to \linewidth or em units.
%
%         Prefer \emph{} to italicize terminology, e.g.:
%             This is a \emph{definition}, not a placeholder.
%             This is a \plc{var-name}.
%
\chapter{Directives}
\index{directives}
\label{chap:Directives}
This chapter describes the syntax and behavior of OpenMP directives, and is divided 
into the following sections:

\begin{itemize}
\item The language-specific directive format 
(\specref{sec:Directive Format})

\item Mechanisms to control conditional compilation 
(\specref{sec:Conditional Compilation})

\item Control of OpenMP API ICVs 
(\specref{sec:Internal Control Variables})

\item How to specify and to use array sections for all base languages 
(\specref{sec:Array Sections}) 

\item Details of each OpenMP directive, including associated events and tool callbacks 
(\specref{sec:parallel Construct} to 
\specref{sec:Nesting of Regions}) 
\end{itemize}

\ccppspecificstart
In C/C++, OpenMP directives are specified by using the \code{\#pragma} mechanism provided 
by the C and C++ standards. 
\ccppspecificend

\fortranspecificstart
In Fortran, OpenMP directives are specified by using special comments that are 
identified by unique sentinels. Also, a special comment form is available for conditional 
compilation. 
\fortranspecificend

Compilers can therefore ignore OpenMP directives and conditionally compiled code if 
support of the OpenMP API is not provided or enabled. A compliant implementation 
must provide an option or interface that ensures that underlying support of all OpenMP 
directives and OpenMP conditional compilation mechanisms is enabled. In the 
remainder of this document, the phrase \emph{OpenMP compilation} is used to mean a 
compilation with these OpenMP features enabled.

\begin{samepage}
\fortranspecificstart
\restrictions
The following restriction applies to all OpenMP directives: 
\begin{itemize}
\item OpenMP directives, except SIMD and \code{declare}~\code{target} directives,
 may not appear in pure procedures.
\end{itemize}
\fortranspecificend
\end{samepage}


% This is an included file. See the master file for more information.
%
% When editing this file:
%
%    1. To change formatting, appearance, or style, please edit openmp.sty.
%
%    2. Custom commands and macros are defined in openmp.sty.
%
%    3. Be kind to other editors -- keep a consistent style by copying-and-pasting to
%       create new content.
%
%    4. We use semantic markup, e.g. (see openmp.sty for a full list):
%         \code{}     % for bold monospace keywords, code, operators, etc.
%         \plc{}      % for italic placeholder names, grammar, etc.
%
%    5. There are environments that provide special formatting, e.g. language bars.
%       Please use them whereever appropriate.  Examples are:
%
%         \begin{fortranspecific}
%         This is text that appears enclosed in blue language bars for Fortran.
%         \end{fortranspecific}
%
%         \begin{note}
%         This is a note.  The "Note -- " header appears automatically.
%         \end{note}
%
%    6. Other recommendations:
%         Use the convenience macros defined in openmp.sty for the minor headers
%         such as Comments, Syntax, etc.
%
%         To keep items together on the same page, prefer the use of
%         \begin{samepage}.... Avoid \parbox for text blocks as it interrupts line numbering.
%         When possible, avoid \filbreak, \pagebreak, \newpage, \clearpage unless that's
%         what you mean. Use \needspace{} cautiously for troublesome paragraphs.
%
%         Avoid absolute lengths and measures in this file; use relative units when possible.
%         Vertical space can be relative to \baselineskip or ex units. Horizontal space
%         can be relative to \linewidth or em units.
%
%         Prefer \emph{} to italicize terminology, e.g.:
%             This is a \emph{definition}, not a placeholder.
%             This is a \plc{var-name}.
%


\section{Directive Format}
\label{sec:Directive Format}
\index{directive format}
\begin{ccppspecific}
OpenMP directives for C/C++ are specified with the \code{pragma} preprocessing directive.
The syntax of an OpenMP directive is as follows:

\begin{ompcPragma}
#pragma omp\plc{ directive-name [clause[ [},\plc{] clause] ... ] new-line}
\end{ompcPragma}

Each directive  starts with \pcode{\#pragma} \code{omp}. The remainder of the directive follows the
conventions of the C and C++ standards for compiler directives. In particular, white
space can be used before and after the \pcode{\#}, and sometimes white space must be used to
separate the words in a directive. Preprocessing tokens following the \pcode{\#pragma} \code{omp}
are subject to macro replacement.

Some OpenMP directives may be composed of consecutive \pcode{\#pragma} preprocessing
directives if specified in their syntax.

Directives are case-sensitive.

An OpenMP executable directive applies to at most one succeeding statement, which
must be a structured block.
\end{ccppspecific}

\begin{cppspecific}
Directives may not appear in constexpr functions or in constant expressions.
Variadic parameter packs cannot be expanded into a directive or its clauses
except as part of an expression argument to be evaluated by the base language,
such as into a function call inside an \code{if()} clause.
\end{cppspecific}

\begin{fortranspecific}
OpenMP directives for Fortran are specified as follows:

\begin{ompfPragma}
\plc{sentinel directive-name [clause[ [},\plc{] clause]...]}
\end{ompfPragma}

All OpenMP compiler directives must begin with a directive \emph{sentinel}. The format of a
sentinel differs between fixed and free-form source files, as described in
\specref{subsec:Fixed Source Form Directives} and \specref{subsec:Free Source Form Directives}.

Directives are case insensitive. Directives cannot be embedded within continued
statements, and statements cannot be embedded within directives.

In order to simplify the presentation, free form is used for the syntax of OpenMP
directives for Fortran in the remainder of this document, except as noted.
\end{fortranspecific}

Only one \emph{directive-name} can be specified per directive (note that this includes combined
directives, see \specref{sec:Combined Constructs}).  The order in which clauses appear on directives
is not significant. Clauses on directives may be repeated as needed, subject to the
restrictions listed in the description of each clause.

Some data-sharing attribute clauses (\specref{subsec:Data-Sharing Attribute Clauses}),
data copying clauses (\specref{subsec:Data Copying Clauses}), the
\code{threadprivate} directive (\specref{subsec:threadprivate Directive}),
the \code{flush} directive (\specref{subsec:flush Construct}), and the
\code{link} clause of the \code{declare}~\code{target} directive
(\specref{subsec:declare target Directive}) accept a \emph{list}. The
\code{to} clause of the \code{declare}~\code{target} directive
(\specref{subsec:declare target Directive}) accepts an \plc{extended-list}.
The \code{depend} clause (\specref{subsec:depend Clause}), when used to specify task dependences,
accepts a \plc{locator-list}.  A \plc{list} consists of a
comma-separated collection of one or more \plc{list items}. A
\plc{extended-list} consists of a comma-separated collection of one or more
\plc{extended list items}. A \plc{locator-list} consists of a comma-separated
collection of one or more \plc{locator list items}.

\begin{ccppspecific}
A \plc{list item} is a variable or array section. An \plc{extended list item} is a \plc{list item} or a function name.  A \plc{locator list item} is any \plc{lvalue}
expression, including variables, or an array section.
\end{ccppspecific}

\begin{fortranspecific}
A \plc{list item} is a variable, array section or common block name
(enclosed in slashes). An \plc{extended list item} is a \plc{list item}
or a procedure name. A \plc{locator list item} is a \plc{list item}.

When a named common block appears in a \plc{list}, it has the same
meaning as if every explicit member of the common block appeared in
the list.  An explicit member of a common block is a variable that is
named in a \code{COMMON} statement that specifies the common block
name and is declared in the same scoping unit in which the clause
appears.

Although variables in common blocks can be accessed by use association
or host association, common block names cannot.  As a result, a common
block name specified in a data-sharing attribute, a data copying or
a data-mapping attribute clause must be declared to be a common block in
the same scoping unit in which the clause appears.

If a list item that appears in a directive or clause is an optional
dummy argument that is not present, the directive or clause for that
list item is ignored.

If the variable referenced inside a construct is an optional dummy
argument that is not present, any explicitly determined, implicitly
determined, or predetermined data-sharing and data-mapping attribute
rules for that variable are ignored.  Otherwise, if the variable is an
optional dummy argument that is present, it is present inside the
construct.
\end{fortranspecific}

For all base languages, a \plc{list item}  or an \plc{extended list item}
is subject to the restrictions specified in \specref{sec:Array Sections}
and in each of the sections describing clauses and directives for which
the \plc{list} or \plc{extended-list} appears.



%\pagebreak
% Force the blue floater bar down, and force the subsection header up, to
% bring the blue bar closer to the header:
\vspace{2\baselineskip}
\begin{fortranspecific}
\vspace{-1\baselineskip}
\subsection{Fixed Source Form Directives}
\label{subsec:Fixed Source Form Directives}
\index{fixed source form directives}
The following sentinels are recognized in fixed form source files:

\begin{ompfPragma}
!$omp \textnormal{|} c$omp \textnormal{|} *$omp
\end{ompfPragma}

Sentinels must start in column 1 and appear as a single word with no intervening
characters. Fortran fixed form line length, white space, continuation, and column rules
apply to the directive line. Initial directive lines must have a space or zero in column 6,
and continuation directive lines must have a character other than a space or a zero in
column 6.

Comments may appear on the same line as a directive. The exclamation point initiates a
comment when it appears after column 6. The comment extends to the end of the source
line and is ignored. If the first non-blank character after the directive sentinel of an
initial or continuation directive line is an exclamation point, the line is ignored.

\begin{note}
in the following example, the three formats for specifying the directive are
equivalent (the first line represents the position of the first 9 columns):

\begin{ompfPragma}
c23456789
!$omp parallel do shared(a,b,c)

c$omp parallel do
c$omp+shared(a,b,c)

c$omp paralleldoshared(a,b,c)
\end{ompfPragma}
\end{note}










\subsection{Free Source Form Directives}
\label{subsec:Free Source Form Directives}
\index{free source form directives}

The following sentinel is recognized in free form source files:

\begin{ompfPragma}
!$omp
\end{ompfPragma}

The sentinel can appear in any column as long as it is preceded only by white space
(spaces and tab characters). It must appear as a single word with no intervening
character. Fortran free form line length, white space, and continuation rules apply to the
directive line. Initial directive lines must have a space after the sentinel. Continued
directive lines must have an ampersand (\code{&}) as the last non-blank character on the line,
prior to any comment placed inside the directive. Continuation directive lines can have
an ampersand after the directive sentinel with optional white space before and after the
ampersand.

Comments may appear on the same line as a directive. The exclamation point (\code{!})
initiates a comment. The comment extends to the end of the source line and is ignored.
If the first non-blank character after the directive sentinel is an exclamation point, the
line is ignored.

One or more blanks or horizontal tabs are optional to separate adjacent
keywords in \plc{directive-names} unless otherwise specified.

\begin{note}
in the following example the three formats for specifying the directive are
equivalent (the first line represents the position of the first 9 columns):

\begin{ompfPragma}
!23456789
       !$omp parallel do &
                 !$omp shared(a,b,c)

       !$omp parallel &
      !$omp&do shared(a,b,c)

!$omp paralleldo shared(a,b,c)
\end{ompfPragma}
\end{note}
\bigskip
\end{fortranspecific}








\subsection{Stand-Alone Directives}
\label{subsec:Stand-Alone Directives}
\index{stand-alone directives}
\summary
Stand-alone directives are executable directives that have no associated user code.

\descr
Stand-alone directives do not have any associated executable user code. Instead, they
represent executable statements that typically do not have succinct equivalent statements
in the base languages. There are some restrictions on the placement of a stand-alone
directive within a program. A stand-alone directive may be placed only at a point where
a base language executable statement is allowed.

\restrictions
\begin{ccppspecific}
For C/C++, a stand-alone directive may not be used in place of the statement following
an \code{if}, \code{while}, \code{do}, \code{switch}, or \code{label}.
\end{ccppspecific}

\begin{fortranspecific}
For Fortran, a stand-alone directive may not be used as the action statement in an \code{if}
statement or as the executable statement following a label if the label is referenced in
the program.
\end{fortranspecific}


\section{Conditional Compilation}
\label{sec:Conditional Compilation}
\index{conditional compilation}
\index{_OPENMP@{\code{_OPENMP} macro}}
In implementations that support a preprocessor, the \code{_OPENMP} macro name is defined to
have the decimal value \plc{yyyymm} where \plc{yyyy} and \plc{mm} are the year and month designations
of the version of the OpenMP API that the implementation supports.

If this macro is the subject of a \pcode{\#define} or a \pcode{\#undef} preprocessing directive, the
behavior is unspecified.

\begin{fortranspecific}
The OpenMP API requires Fortran lines to be compiled conditionally, as described in
the following sections.




\subsection{Fixed Source Form Conditional Compilation Sentinels}
\label{subsec:Fixed Source Form Conditional Compilation Sentinels}
\index{fixed source form conditional compilation sentinels}
\index{compilation sentinels}
The following conditional compilation sentinels are recognized in fixed form source
files:

\begin{ompfPragma}
!$ \textnormal{|} *$ \textnormal{|} c$
\end{ompfPragma}

To enable conditional compilation, a line with a conditional compilation sentinel must
satisfy the following criteria:

\begin{itemize}
\item The sentinel must start in column 1 and appear as a single word with no intervening
white space.

\item After the sentinel is replaced with two spaces, initial lines must have a space or zero
in column 6 and only white space and numbers in columns 1 through 5.

\item After the sentinel is replaced with two spaces, continuation lines must have a
character other than a space or zero in column 6 and only white space in columns 1
through 5.
\end{itemize}

If these criteria are met, the sentinel is replaced by two spaces. If these criteria are not
met, the line is left unchanged.

\begin{note}
in the following example, the two forms for specifying conditional compilation
in fixed source form are equivalent (the first line represents the position of the first 9
columns):

\begin{ompfPragma}
c23456789
!$ 10 iam = omp_get_thread_num() +
!$   &          index

#ifdef _OPENMP
   10 iam = omp_get_thread_num() +
     &            index
#endif
\end{ompfPragma}
\end{note}





\subsection{Free Source Form Conditional Compilation Sentinel}
\label{subsec:Free Source Form Conditional Compilation Sentinel}
\index{free source form conditional compilation sentinel}
\index{compilation sentinels}
The following conditional compilation sentinel is recognized in free form source files:

\begin{ompfPragma}
!$
\end{ompfPragma}

To enable conditional compilation, a line with a conditional compilation sentinel must
satisfy the following criteria:

\begin{itemize}
\item The sentinel can appear in any column but must be preceded only by white space.

\item The sentinel must appear as a single word with no intervening white space.

\item Initial lines must have a space after the sentinel.

\item Continued lines must have an ampersand as the last non-blank character on the line,
prior to any comment appearing on the conditionally compiled line. Continuation lines
can have an ampersand after the sentinel, with optional white space before and after
the ampersand.
\end{itemize}

If these criteria are met, the sentinel is replaced by two spaces. If these criteria are not
met, the line is left unchanged.

\begin{note}
in the following example, the two forms for specifying conditional compilation
in free source form are equivalent (the first line represents the position of the first 9
columns):

\begin{ompfPragma}
c23456789
 !$ iam = omp_get_thread_num() +     &
 !$&    index

#ifdef _OPENMP
    iam = omp_get_thread_num() +     &
        index
#endif
\end{ompfPragma}
\end{note}
\bigskip
\end{fortranspecific}





\section{\hcode{requires} Directive}
\label{sec:requires Directive}
\index{requires@{\code{requires}}}
\index{directives!requires@{\code{requires}}}

\summary The \code{requires} directive specifies the features an implementation
must provide in order for the code to compile and to execute correctly.
The \code{requires} directive is a declarative directive.




\syntax
\begin{ccppspecific}
  The syntax of the \code{requires} directive is as follows:

\begin{ompcPragma}
  #pragma omp requires \plc{clause[ [ [},\plc{] clause] ... ] new-line}

\end{ompcPragma}

\end{ccppspecific}

\begin{fortranspecific}
  The syntax of the \code{requires} directive is as follows:

\begin{ompfPragma}
!$omp requires \plc{clause[ [ [},\plc{] clause] ... ]}

\end{ompfPragma}

\end{fortranspecific}

Where \plc{clause} is either one of the requirement clauses listed below or a
clause of the form {\scode{ext_}\plc{implementation-defined-requirement}} for an
implementation defined requirement clause.

\begin{indentedcodelist}
unified_address
unified_shared_memory
atomic_default_mem_order(seq_cst \textnormal{|} acq_rel \textnormal{|} relaxed)
dynamic_allocators
\end{indentedcodelist}

\descr

The \code{requires} directive specifies features an implementation must
support for correct execution. The behavior specified by a requirement clause
may override the normal behavior specified elsewhere in this document.

\begin{ccppspecific}
The \code{requires} directive specifies requirements for the execution of all
code in the current translation unit.
\end{ccppspecific}

\begin{fortranspecific}
The \code{requires} directive specifies requirements for the execution of all
code in the current program unit.
\end{fortranspecific}

\begin{note}
Use of this directive makes your code less portable. Users should be aware that not all
devices or implementations support all requirements.
\end{note}

When the \code{unified_address} clause appears on a \code{requires}
directive, the implementation guarantees that all devices accessible through
OpenMP API routines and directives use a unified address space. In this
address space, a pointer will always refer to the same location in memory
from all devices accessible through OpenMP.  The pointers returned by
\code{omp_target_alloc} and accessed through \code{use_device_ptr} are
guaranteed to be pointer values that can support pointer arithmetic while
still being native device pointers. The \code{is_device_ptr} clause is not
necessary for device pointers to be translated in \code{target} regions, and
pointers found not present are not set to null but keep their original value.
Memory local to a specific execution context may be exempt from this,
following the restrictions of locality to a given execution context, thread or
contention group.  Target devices may still have discrete memories and
dereferencing a device pointer on the host device remains unspecified
behavior.

The \code{unified_shared_memory} clause implies the \code{unified_address}
requirement, inheriting all of its behaviors.  Additionally memory in the
device data environment of any device visible to OpenMP, including but not
limited to the host, is considered part of the device data environment of all
devices accessible through OpenMP except as noted below.  Every device address
allocated through OpenMP device memory routines is a valid host pointer. Memory
local to an execution context as defined in \code{unified_address} above may remain
part of distinct device data environments as long as the execution context is
local to the device containing that environment.

The \code{unified_shared_memory} clause makes the \code{map} clause optional
on \code{target} constructs as well as the \code{declare}~\code{target}
directive on static lifetime variables accessed as part of
\code{declare}~\code{target} functions.  Scalar variables are still made
\code{firstprivate} by default for \code{target} regions.  Values stored into
memory by one device may not be visible to other devices until those two
devices synchronize with each other or both synchronize with the host.

The \code{atomic_default_mem_order} clause specifies the default memory ordering
behavior for \code{atomic} constructs that must be provided by an
implementation. If the default memory ordering is specified as \code{seq_cst}, all
\code{atomic} constructs on which \plc{memory-order-clause} is not specified
behave as if the \code{seq_cst} clause appears. If the default memory
ordering is specified as \code{relaxed}, all \code{atomic} constructs on which
\plc{memory-order-clause} is not specified behave as if the \code{relaxed}
clause appears.

If the default memory ordering is specified as \code{acq_rel}, \code{atomic}
constructs on which \plc{memory-order-clause} is not specified behave in the
following manner:

\begin{itemize}
    \item as if the \code{release} clause is present if the construct
        specifies an atomic write or atomic update operation;
    \item as if the \code{acquire} clause is present if the construct
        specifies an atomic read operation;
    \item as if the \code{acq_rel} clause is present if the construct
        specifies an atomic captured update operation.
\end{itemize}

The \code{dynamic_allocators} clause has the following effects:
\begin{itemize}
 \item makes the \code{uses_allocators} clause optional on \code{target} constructs for the purpose of using allocators in the corresponding \code{target} regions,
 \item allows the \code{omp_init_allocator} and \code{omp_destroy_allocator} API routines in \code{target} regions,
 \item allows default allocators to be used by \code{allocate} directives, \code{allocate} clauses and \code{omp_alloc} API routines in \code{target} regions.
\end{itemize}

Implementers are allowed to include additional implementation defined
requirement clauses.  Requirement names that do not start with \code{ext_} are
reserved. All implementation-defined requirements should begin with
\code{ext_}.

\restrictions

The restrictions for the \code{requires} directive are as follows:

\begin{itemize}

\item Each of the clauses can appear at most once on the directive.

\item At most one \code{requires} directive with
\code{atomic_default_mem_order} clause can appear in a single compilation
unit.

\item A \code{requires} directive with a \code{unified_address},
  \code{unified_shared_memory} or \code{reverse_offload} clause shall appear
  lexically before any device constructs or device routines.

\item The \code{requires} directive with \code{atomic_default_mem_order}
clause may not appear lexically after any \code{atomic} construct on which
\plc{memory-order-clause} is not specified.

\end{itemize}


\section{Internal Control Variables}
\label{sec:Internal Control Variables}
\index{internal control variables (ICVs)}
\index{ICVs (internal control variables)}
An OpenMP implementation must act as if there are internal control variables (ICVs)
that control the behavior of an OpenMP program. These ICVs store information such as
the number of threads to use for future \code{parallel} regions, the schedule to use for
worksharing loops and whether nested parallelism is enabled or not. The ICVs are given
values at various times (described below) during the execution of the program. They are
initialized by the implementation itself and may be given values through OpenMP
environment variables and through calls to OpenMP API routines. The program can
retrieve the values of these ICVs only through OpenMP API routines.

For purposes of exposition, this document refers to the ICVs by certain names, but an
implementation is not required to use these names or to offer any way to access the
variables other than through the ways shown in
\specref{subsec:ICV Initialization}.








\subsection{ICV Descriptions}
\label{subsec:ICV Descriptions}
The following ICVs store values that affect the operation of \code{parallel} regions.

\begin{itemize}
\item \plc{dyn-var} - controls whether dynamic adjustment of the number of threads is enabled
for encountered \code{parallel} regions. There is one copy of this ICV per data
environment.

\item \plc{nest-var} - controls whether nested parallelism is enabled for encountered \code{parallel}
regions. There is one copy of this ICV per data environment. The \plc{nest-var} ICV has been deprecated.

\item \plc{nthreads-var} - controls the number of threads requested for encountered \code{parallel}
regions. There is one copy of this ICV per data environment.

\item \plc{thread-limit-var} - controls the maximum number of threads participating in the
contention group. There is one copy of this ICV per data environment.

\item \plc{max-active-levels-var} - controls the maximum number of nested active \code{parallel}
regions. There is one copy of this ICV per device.

\item \plc{place-partition-var} -- controls the place partition available to the execution
environment for encountered \code{parallel} regions. There is one copy of this ICV per
implicit task.

\item \plc{active-levels-var} - the number of nested, active parallel regions enclosing the current
task such that all of the \code{parallel} regions are enclosed by the outermost initial task
region on the current device. There is one copy of this ICV per data environment.

\item \plc{levels-var} - the number of nested parallel regions enclosing the current task such that
all of the \code{parallel} regions are enclosed by the outermost initial task region on the
current device. There is one copy of this ICV per data environment.

\item \plc{bind-var} - controls the binding of OpenMP threads to places. When binding is
requested, the variable indicates that the execution environment is advised not to
move threads between places. The variable can also provide default thread affinity
policies. There is one copy of this ICV per data environment.
\end{itemize}

The following ICVs store values that affect the operation of loop regions.

\begin{itemize}
\item \plc{run-sched-var} - controls the schedule that the \code{runtime} schedule clause uses for
loop regions. There is one copy of this ICV per data environment.

\item \plc{def-sched-var} - controls the implementation defined default scheduling of loop
regions. There is one copy of this ICV per device.
\end{itemize}

The following ICVs store values that affect program execution.

\begin{itemize}
\item \plc{stacksize-var} - controls the stack size for threads that the OpenMP implementation
creates. There is one copy of this ICV per device.

\item \plc{wait-policy-var} - controls the desired behavior of waiting threads. There is one copy
of this ICV per device.

\item \plc{display-affinity-var} - controls whether to display thread affinity. There is one copy of this ICV for the whole program.

\item \plc{affinity-format-var} - controls the thread affinity format when displaying thread affinity. There is one copy of this ICV per device.

\item \plc{cancel-var} - controls the desired behavior of the \code{cancel} construct and cancellation
points. There is one copy of this ICV for the whole program.

\item \plc{default-device-var} - controls the default target device. There is one copy of this ICV
per data environment.

\item \plc{target-offload-var} - controls the offloading behavior. There is one copy of this ICV
for the whole program.

\item \plc{max-task-priority-var} - controls the maximum priority value that can be specified in the
\code{priority} clause of the \code{task} construct. There is one copy of this ICV for the whole program.

\end{itemize}

The following ICVs store values that affect the operation of the
first-party tool interface.

\begin{itemize}

\item \plc{tool-var} - determines whether an OpenMP implementation will
try to register a tool.  There is one copy of this ICV for the whole program.

\item \plc{tool-libraries-var} - specifies a list of absolute paths to tool libraries for OpenMP devices.
There is one copy of this ICV for the whole program.

\end{itemize}

The following ICVs store values that relate to the operation of the
OMPD tool interface.

\begin{itemize}
\item
  \plc{debug-var} - determines whether an OpenMP implementation will collect
  information that an OMPD library can access to satisfy requests from
  a tool.
  There is one copy of this ICV for the whole program.
\end{itemize}

The following ICVs store values that affect default memory allocation.

\begin{itemize}

\item \plc{def-allocator-var} - determines the memory allocator to be used by memory allocation routines, directives and clauses when a memory allocator is not specified by the user. There is one copy of this ICV per implicit task.

\end{itemize}




\subsection{ICV Initialization}
\label{subsec:ICV Initialization}
\index{modifying ICV's}
Table~\ref{tab:ICV Initial Values} shows the ICVs, associated
environment variables, and initial values.

\nolinenumbers
\renewcommand{\arraystretch}{1.5}
\tablefirsthead{%
\hline
\textsf{\textbf{ICV}} & \textsf{\textbf{Environment Variable}} & \textsf{\textbf{Initial value}}\\
\hline\\[-3ex]
}
\tablehead{%
\multicolumn{2}{l}{\small\slshape table continued from previous page}\\
\hline
\textsf{\textbf{ICV}} & \textsf{\textbf{Environment Variable}} & \textsf{\textbf{Initial value}}\\
\hline\\[-3ex]
}
\tabletail{%
\hline\\[-4ex]
\multicolumn{2}{l}{\small\slshape table continued on next page}\\
}
\tablelasttail{\hline}
\tablecaption{ICV Initial Values\label{tab:ICV Initial Values}}
\begin{supertabular}{p{1.4in} p{1.8in} p{1.5in}}
{\splc{dyn-var}} & {\scode{OMP_DYNAMIC}} & See description below\\
{\splc{nest-var}} & {\scode{OMP_NESTED}} & Implementation defined\\
{\splc{nthreads-var}} & {\scode{OMP_NUM_THREADS}} & Implementation defined\\
{\splc{run-sched-var}} & {\scode{OMP_SCHEDULE}} & Implementation defined\\
{\splc{def-sched-var}} & (none) & Implementation defined\\
{\splc{bind-var}} & {\scode{OMP_PROC_BIND}} & Implementation defined\\
{\splc{stacksize-var}} & {\scode{OMP_STACKSIZE}} & Implementation defined\\
{\splc{wait-policy-var}} & {\scode{OMP_WAIT_POLICY}} & Implementation defined\\
{\splc{thread-limit-var}} & {\scode{OMP_THREAD_LIMIT}} & Implementation defined\\
{\splc{max-active-levels-var}} & {\scode{OMP_MAX_ACTIVE_LEVELS}} & See description below\\
{\splc{active-levels-var}} & (none) & {\splc{zero}}\\
{\splc{levels-var}} & (none) & {\splc{zero}}\\
{\splc{place-partition-var}} & {\scode{OMP_PLACES}} & Implementation defined\\
{\splc{cancel-var}} & {\scode{OMP_CANCELLATION}} & {\splc{false}}\\
{\splc{display-affinity-var}} & {\scode{OMP_DISPLAY_AFFINITY}} & {\splc{false}}\\
{\splc{affinity-format-var}} & {\scode{OMP_AFFINITY_FORMAT}} & Implementation defined \\
{\splc{default-device-var}} & {\scode{OMP_DEFAULT_DEVICE}} & Implementation defined\\
{\splc{target-offload-var}} & {\scode{OMP_TARGET_OFFLOAD}} & {\scode{DEFAULT}} \\
{\splc{max-task-priority-var}} & {\scode{OMP_MAX_TASK_PRIORITY}} & {\splc{zero}}\\
{\splc{tool-var}} & {\scode{OMP_TOOL}} & {\splc{enabled}}\\
{\splc{tool-libraries-var}} & {\scode{OMP_TOOL_LIBRARIES}} & {\splc{empty string}}\\
{\splc{debug-var}} & {\scode{OMP_DEBUG}} & {\splc{disabled}} \\
{\splc{def-allocator-var}} & {\scode{OMP_ALLOCATOR}} & Implementation defined\\
\end{supertabular}

\linenumbers

\descr
\begin{itemize}
\item Each device has its own ICVs.

\item The value of the \plc{nthreads-var} ICV is a list.

\item The value of the \plc{bind-var} ICV is a list.

\item The initial value of \plc{dyn-var} is implementation defined if the implementation supports
dynamic adjustment of the number of threads; otherwise, the initial value is \plc{false}.

\item The initial value of \plc{max-active-levels-var} is the number of levels of parallelism that
the implementation supports. See the definition of \emph{supporting n levels of parallelism}
in \specref{subsec:Implementation Terminology} for further details.
\end{itemize}

The host and target device ICVs are initialized before any OpenMP API construct or
OpenMP API routine executes. After the initial values are assigned, the values of any
OpenMP environment variables that were set by the user are read and the associated
ICVs for the host device are modified accordingly. The method for initializing a target
device's ICVs is implementation defined.

\crossreferences
\begin{itemize}
\item \code{OMP_SCHEDULE} environment variable, see \specref{sec:OMP_SCHEDULE}.

\item \code{OMP_NUM_THREADS} environment variable, see \specref{sec:OMP_NUM_THREADS}.

\item \code{OMP_DYNAMIC} environment variable, see \specref{sec:OMP_DYNAMIC}.

\item \code{OMP_PROC_BIND} environment variable, see \specref{sec:OMP_PROC_BIND}.

\item \code{OMP_PLACES} environment variable, see \specref{sec:OMP_PLACES}.

\item \code{OMP_NESTED} environment variable, see \specref{sec:OMP_NESTED}.

\item \code{OMP_STACKSIZE} environment variable, see \specref{sec:OMP_STACKSIZE}.

\item \code{OMP_WAIT_POLICY} environment variable, see \specref{sec:OMP_WAIT_POLICY}.

\item \code{OMP_MAX_ACTIVE_LEVELS} environment variable, see \specref{sec:OMP_MAX_ACTIVE_LEVELS}.

\item \code{OMP_THREAD_LIMIT} environment variable, see \specref{sec:OMP_THREAD_LIMIT}.

\item \code{OMP_CANCELLATION} environment variable, see \specref{sec:OMP_CANCELLATION}.

\item \code{OMP_DISPLAY_AFFINITY} environment variable, see \specref{sec:OMP_DISPLAY_AFFINITY}.

\item \code{OMP_AFFINITY_FORMAT} environment variable, see \specref{sec:OMP_AFFINITY_FORMAT}.

\item \code{OMP_DEFAULT_DEVICE} environment variable, see \specref{sec:OMP_DEFAULT_DEVICE}.

\item \code{OMP_TARGET_OFFLOAD} environment variable, see \specref{sec:OMP_TARGET_OFFLOAD}.

\item \code{OMP_MAX_TASK_PRIORITY} environment variable, see \specref{sec:OMP_MAX_TASK_PRIORITY}.

\item \code{OMP_TOOL} environment variable, see \specref{sec:OMP_TOOL}.

\item \code{OMP_TOOL_LIBRARIES} environment variable, see \specref{sec:OMP_TOOL_LIBRARIES}.

\item \code{OMP_DEBUG} environment variable, see \specref{sec:OMP_DEBUG}.

\item \code{OMP_ALLOCATOR} environment variable, see \specref{sec:OMP_ALLOCATOR}.
\end{itemize}








\subsection{Modifying and Retrieving ICV Values}
\label{subsec:Modifying and Retrieving ICV Values}
\index{modifying and retrieving ICV values}
Table~\ref{tab:Ways to Modify and to Retrieve ICV Values} shows the method for modifying and retrieving the values of ICVs
through OpenMP API routines.

%\newpage %% HACK to force table title onto same page...
{\small%
\nolinenumbers
\renewcommand{\arraystretch}{1.5}
\tablefirsthead{%
\hline
\textsf{\textbf{ICV}} & \textsf{\textbf{Ways to modify value}} & \textsf{\textbf{Ways to retrieve value}}\\
\hline\\[-3ex]
}
\tablehead{%
\multicolumn{2}{l}{\small\slshape table continued from previous page}\\
\hline
\textsf{\textbf{ICV}} & \textsf{\textbf{Ways to modify value}} & \textsf{\textbf{Ways to retrieve value}}\\
\hline\\[-3ex]
}
\tabletail{%
\hline\\[-4ex]
\multicolumn{2}{l}{\small\slshape table continued on next page}\\
}
\tablelasttail{\hline}
\tablecaption{Ways to Modify and to Retrieve ICV Values\label{tab:Ways to Modify and to Retrieve ICV Values}}
\begin{supertabular}{ p{1.2in} p{2.0in} p{1.5in}}
{\splc{dyn-var}} & {\scode{omp_set_dynamic()}} & {\scode{omp_get_dynamic()}}\\

{\splc{nest-var}} & {\scode{omp_set_nested()}} & {\scode{omp_get_nested()}}\\

{\splc{nthreads-var}} & {\scode{omp_set_num_threads()}} & {\scode{omp_get_max_threads()}}\\

{\splc{run-sched-var}} & {\scode{omp_set_schedule()}} & {\scode{omp_get_schedule()}}\\

{\splc{def-sched-var}} & (none) & (none)\\

{\splc{bind-var}} & (none) & {\scode{omp_get_proc_bind()}}\\

{\splc{stacksize-var}} & (none) & (none)\\

{\splc{wait-policy-var}} & (none) & (none)\\

{\splc{thread-limit-var}} & {\scode{thread_limit}} clause & {\scode{omp_get_thread_limit()}}\\

{\splc{max-active-levels-var}} & {\scode{omp_set_max_active_levels()}} & {\scode{omp_get_max_active_levels()}}\\

{\splc{active-levels-var}} & (none) & {\scode{omp_get_active_level()}}\\

{\splc{levels-var}} & (none) & {\scode{omp_get_level()}}\\

{\splc{place-partition-var}} & (none) & See description below \\

{\splc{cancel-var}} & (none) & {\scode{omp_get_cancellation()}}\\

{\splc{display-affinity-var}} & (none) & (none)\\

{\splc{affinity-format-var}} & {\scode{omp_set_affinity_format()}} & {\scode{omp_get_affinity_format()}}\\

{\splc{default-device-var}} & {\scode{omp_set_default_device()}} & {\scode{omp_get_default_device()}}\\

{\splc{target-offload-var}} & (none) & (none)\\

{\splc{max-task-priority-var}} & (none) & {\scode{omp_get_max_task_priority()}}\\
{\splc{tool-var}} & (none) & (none)\\
{\splc{tool-libraries-var}} & (none) & (none)\\
{\splc{debug-var}} & (none) & (none) \\

{\splc{def-allocator-var}} & {\scode{omp_set_default_allocator()}} & {\scode{omp_get_default_allocator()}}\\

\end{supertabular}

\linenumbers} % end of \small block

\descr
\begin{itemize}
\item The value of the \plc{nthreads-var} ICV is a list. The runtime call
\code{omp_set_num_threads} sets the value of the first element of this list, and
\code{omp_get_max_threads} retrieves the value of the first element of this list.

\item The value of the \plc{bind-var} ICV is a list. The runtime call \code{omp_get_proc_bind}
retrieves the value of the first element of this list.

\item
Detailed values in the \plc{place-partition-var} ICV are retrieved using the runtime calls
\code{omp_get_partition_num_places}, \code{omp_get_partition_place_nums},
\code{omp_get_place_num_procs}, and \code{omp_get_place_proc_ids}.
\end{itemize}

\crossreferences
\begin{itemize}
\item \code{thread_limit} clause of the \code{teams} construct, see \specref{subsec:teams Construct}.

\item \code{omp_set_num_threads} routine, see \specref{subsec:omp_set_num_threads}.

\item \code{omp_get_max_threads} routine, see \specref{subsec:omp_get_max_threads}.

\item \code{omp_set_dynamic} routine, see \specref{subsec:omp_set_dynamic}.

\item \code{omp_get_dynamic} routine, see \specref{subsec:omp_get_dynamic}.

\item \code{omp_get_cancellation} routine, see \specref{subsec:omp_get_cancellation}.

\item \code{omp_set_nested} routine, see \specref{subsec:omp_set_nested}.

\item \code{omp_get_nested} routine, see \specref{subsec:omp_get_nested}.

\item \code{omp_set_schedule} routine, see \specref{subsec:omp_set_schedule}.

\item \code{omp_get_schedule} routine, see \specref{subsec:omp_get_schedule}.

\item \code{omp_get_thread_limit} routine, see \specref{subsec:omp_get_thread_limit}.

\item \code{omp_set_max_active_levels} routine, see \specref{subsec:omp_set_max_active_levels}.

\item \code{omp_get_max_active_levels} routine, see \specref{subsec:omp_get_max_active_levels}.

\item \code{omp_get_level} routine, see \specref{subsec:omp_get_level}.

\item \code{omp_get_active_level} routine, see \specref{subsec:omp_get_active_level}.

\item \code{omp_get_proc_bind} routine, see \specref{subsec:omp_get_proc_bind}.

\item \code{omp_get_place_num_procs} routine, see \specref{subsec:omp_get_place_num_procs}.

\item \code{omp_get_place_proc_ids} routine, see \specref{subsec:omp_get_place_proc_ids}.

\item \code{omp_get_partition_num_places} routine, see \specref{subsec:omp_get_partition_num_places}.

\item \code{omp_get_partition_place_nums} routine, see \specref{subsec:omp_get_partition_place_nums}.

\item \code{omp_set_affinity_format} routine, see \specref{subsec:omp_set_affinity_format}.

\item \code{omp_get_affinity_format} routine, see \specref{subsec:omp_get_affinity_format}.

\item \code{omp_set_default_device} routine, see \specref{subsec:omp_set_default_device}.

\item \code{omp_get_default_device} routine, see \specref{subsec:omp_get_default_device}.

\item \code{omp_get_max_task_priority} routine, see \specref{subsec:omp_get_max_task_priority}.

\item \code{omp_set_default_allocator} routine, see \specref{subsec:omp_set_default_allocator}.

\item \code{omp_get_default_allocator} routine, see \specref{subsec:omp_get_default_allocator}.
\end{itemize}









\subsection{How ICVs are Scoped}
\label{subsec:How ICVs are Scoped}
Table~\ref{tab:Scopes of ICVs} shows the ICVs and their scope.

\nolinenumbers
\tablefirsthead{%
\hline
\textsf{\textbf{ICV}} & \textsf{\textbf{Scope}}\\
\hline \\[-3ex]
}
\tablehead{%
\multicolumn{2}{l}{\small\slshape table continued from previous page}\\
\hline
\textsf{\textbf{ICV}} & \textsf{\textbf{Scope}}\\
\hline \\[-3ex]
}
\tabletail{%
\hline\\[-4ex]
\multicolumn{2}{l}{\small\slshape table continued on next page}\\
}
\tablelasttail{\hline}
\tablecaption{Scopes of ICVs\label{tab:Scopes of ICVs}}
\begin{supertabular}{p{1.5in} p{2.5in}}
{\splc{dyn-var}} & data environment\\
{\splc{nest-var}} & data environment\\
{\splc{nthreads-var}} & data environment\\
{\splc{run-sched-var}} & data environment\\
{\splc{def-sched-var}} & device\\
{\splc{bind-var}} & data environment\\
{\splc{stacksize-var}} & device\\
{\splc{wait-policy-var}} & device\\
{\splc{thread-limit-var}} & data environment\\
{\splc{max-active-levels-var}} & device\\
{\splc{active-levels-var}} & data environment\\
{\splc{levels-var}} & data environment\\
{\splc{place-partition-var}} & implicit task\\
{\splc{cancel-var}} & global\\
{\splc{display-affinity-var}} & global \\
{\splc{affinity-format-var}} & device \\
{\splc{default-device-var}} & data environment\\
{\splc{target-offload-var}} & global\\
{\splc{max-task-priority-var}} & global\\
{\splc{tool-var}} & global\\
{\splc{tool-libraries-var}} & global\\
{\splc{debug-var}} & global \\
{\splc{third-party-tool-var}} & global \\
{\splc{def-allocator-var}} & implicit task\\
\end{supertabular}

\linenumbers

\descr
\begin{itemize}
\item There is one copy per device of each ICV with device scope

\item Each data environment has its own copies of ICVs with data environment scope

\item Each implicit task has its own copy of ICVs with implicit task scope
\end{itemize}

Calls to OpenMP API routines retrieve or modify data environment scoped ICVs in the
data environment of their binding tasks.










\subsubsection{How the Per-Data Environment ICVs Work}
\label{subsubsec:How the Per-Data Environment ICVs Work}
When a \code{task} construct or \code{parallel} construct is encountered, the generated task(s)
inherit the values of the data environment scoped ICVs from the generating task's ICV
values.

When a \code{parallel} construct is encountered, the value of each ICV witch implicit task scope is inherited, unless otherwise specified, from the implicit binding task of the generating task unless otherwise specified.

When a \code{task} construct is encountered, the generated task inherits the value of
\plc{nthreads-var} from the generating task's \plc{nthreads-var} value. When a \code{parallel}
construct is encountered, and the generating task's \plc{nthreads-var} list contains a single
element, the generated task(s) inherit that list as the value of \plc{nthreads-var}. When a
\code{parallel} construct is encountered, and the generating task's \plc{nthreads-var} list contains
multiple elements, the generated task(s) inherit the value of \plc{nthreads-var} as the list
obtained by deletion of the first element from the generating task's \plc{nthreads-var} value.
The \plc{bind-var} ICV is handled in the same way as the \plc{nthreads-var} ICV.

When a \plc{target task} executes a \code{target} region, the generated initial task uses the values of the data environment scoped ICVs from the device data environment ICV values of the device that will execute the region.

If a \code{teams} construct with a \code{thread_limit} clause is encountered,
the \plc{thread-limit-var} ICV of the construct's data environment is instead set to a value that is less than or equal to the value specified in the clause.

When encountering a loop worksharing region with \code{schedule(runtime)}, all
implicit task regions that constitute the binding parallel region must have the same value
for \plc{run-sched-var} in their data environments. Otherwise, the behavior is unspecified.








\subsection{ICV Override Relationships}
\label{subsec:ICV Override Relationships}
Table~\ref{tab:ICV Override Relationships} shows the override relationships
among construct clauses and ICVs.

\nolinenumbers
\renewcommand{\arraystretch}{1.5}
\tablefirsthead{%
\hline
\textsf{\textbf{ICV}} & \textsf{\textbf{construct clause, if used}}\\
\hline\\[-3ex]
}
\tablehead{%
\multicolumn{2}{l}{\small\slshape table continued from previous page}\\
\hline
\textsf{\textbf{ICV}} & \textsf{\textbf{construct clause, if used}}\\
\hline\\[-3ex]
}
\tabletail{%
\hline\\[-4ex]
\multicolumn{2}{l}{\small\slshape table continued on next page}\\
}
\tablelasttail{\hline}
\tablecaption{ICV Override Relationships\label{tab:ICV Override Relationships}}
\begin{supertabular}{ p{1.3in} p{2.0in}}
{\splc{dyn-var}} & (none)\\
{\splc{nest-var}} & (none)\\
{\splc{nthreads-var}} & {\scode{num_threads}}\\
{\splc{run-sched-var}} & {\scode{schedule}}\\
{\splc{def-sched-var}} & {\scode{schedule}}\\
{\splc{bind-var}} & {\scode{proc_bind}}\\
{\splc{stacksize-var}} & (none)\\
{\splc{wait-policy-var}} & (none)\\
{\splc{thread-limit-var}} & (none)\\
{\splc{max-active-levels-var}} & (none)\\
{\splc{active-levels-var}} & (none)\\
{\splc{levels-var}} & (none)\\
{\splc{place-partition-var}} & (none)\\
{\splc{cancel-var}} & (none)\\
{\splc{display-affinity-var}} & (none) \\
{\splc{affinity-format-var}} & (none) \\
{\splc{default-device-var}} & (none)\\
{\splc{target-offload-var}} & (none)\\
{\splc{max-task-priority-var}} & (none)\\
{\splc{tool-var}} & (none)\\
{\splc{tool-libraries-var}} & (none)\\
{\splc{debug-var}} & (none) \\
{\splc{def-allocator-var}} & {\scode{allocator}}\\
\end{supertabular}

\linenumbers

\descr
\begin{itemize}
\item The \code{num_threads} clause overrides the value of the first element of the
\plc{nthreads-var} ICV.

\item If \plc{bind-var} is not set to \plc{false} then the \code{proc_bind} clause overrides the value of the
first element of the \plc{bind-var} ICV; otherwise, the \code{proc_bind} clause has no effect.
\end{itemize}

\crossreferences
\begin{itemize}
\item \code{parallel} construct, see
\specref{sec:parallel Construct}.

\item \code{proc_bind} clause,
\specref{sec:parallel Construct}.

\item \code{num_threads} clause, see
\specref{subsec:Determining the Number of Threads for a parallel Region}.

\item Loop construct, see
\specref{subsec:Loop Construct}.

\item \code{schedule} clause, see
\specref{subsubsec:Determining the Schedule of a Worksharing Loop}.
\end{itemize}





%% \filbreak
\begin{ccppspecific}
\section{Array Shaping}
\label{sec:Array Shaping}
\index{array shaping}


If an expression has a pointer to \plc{T} type, then a shape-operator can be
used to specify the extent of that pointer. In other words, the
shape-operator is used to reinterpret, as an n-dimensional array, the region of
memory pointed by that expression.

Formally, the syntax of the shape-operator is as follows:
\begin{indentedcodelist}
\plc{ shaped-expression } := ([\plc{s}@\textsubscript{\plc{1}}@\plc{}][\plc{s}@\textsubscript{\plc{2}}@]...[\plc{s}@\textsubscript{\plc{n}}@])\plc{expression}
\end{indentedcodelist}

The result of applying the shape-operator to an expression is an lvalue
expression with an n-dimensional array type with dimensions
\plc{s}\textsubscript{\plc{1}} $\times$ \plc{s}\textsubscript{\plc{2}} ...
$\times$ \plc{s}\textsubscript{\plc{n}} and element type \plc{T}.

The precedence of the shape operator is the same as a type cast.

Each $\plc{s}_\plc{i}$ is an integral type expression that must evaluate to a positive integer.

\restrictions
Restrictions on the shape-operator are as follows:

\begin{itemize}
\item The \plc{T} type must be a complete type.

\item The shape-operator can appear only in clauses where it is explicitly allowed.

\item The type of the expression upon which a shape-operator is applied must be a pointer type.

\begin{cppspecific}
\item If the \plc{T} type is a reference to a type \plc{T'} then the type will be considered to be \plc{T'}
for all purposes of the designated array.
\end{cppspecific}

\end{itemize}
\end{ccppspecific}





%% \filbreak
\section{Array Sections}
\label{sec:Array Sections}
\index{array sections}
An array section designates a subset of the elements in an array.

\begin{ccppspecific}
To specify an array section in an OpenMP construct, array subscript expressions are
extended with the following syntax:
\begin{indentedcodelist}
[\plc{ lower-bound }:\plc{ length }:\plc{ stride}] \textnormal{or}

[\plc{ lower-bound }:\plc{ length }:\plc{ }] \textnormal{or}

[\plc{ lower-bound }:\plc{ length }] \textnormal{or}

[\plc{ lower-bound }:\plc{ }:\plc{ stride}] \textnormal{or}

[\plc{ lower-bound }:\plc{ }:\plc{ }] \textnormal{or}

[\plc{ lower-bound }:\plc{ }] \textnormal{or}

[ :\plc{ length }:\plc{ stride}] \textnormal{or}

[ :\plc{ length }:\plc{ }] \textnormal{or}

[ :\plc{ length }] \textnormal{or}

[\plc{ }:\plc{ }:\plc{ stride}]

[\plc{ }:\plc{ }:\plc{ }]

[\plc{ }:\plc{ }]
\end{indentedcodelist}

% TODO: consider removing this in future ticket
The array section must be a subset of the original array.

% This is now in the glossary.
%The \emph{base expression} of an array section or an array element is an expression that specifies
%the address of the initial element of the original array.

Array sections are allowed on multidimensional arrays. Base language array subscript
expressions can be used to specify length-one dimensions of multidimensional array
sections.

The \plc{lower-bound}, \plc{length} and \plc{stride} are integral type 
expressions. When evaluated they represent a set of integer values as follows:

\{ \plc{lower-bound}, \plc{lower-bound} + \plc{stride}, \plc{lower-bound} + 2 * \plc{stride},... , \plc{lower-bound} + ((\plc{length} - 1) * \plc{stride}) \}

The \plc{length} must evaluate to a non-negative integer.

The \plc{stride} must to a positive integer.

When the size of the array dimension is not known, the \plc{length} must 
be specified explicitly.

When the \plc{stride} is absent it defaults to 1.

When the \plc{length} is absent, it defaults to 
$(\plc{size} - \plc{lower-bound})/\plc{stride}$ where \plc{size} is
the size of the array dimension 

When the \plc{lower-bound} is absent it defaults to 0.

The precedence of an array section is the same as the subscript operator.

% TODO: consider adding this in future ticket
%Each array element specified by an array section is determined according to
%the base expression of the array section and the array subscript expressions
%derived from its array section subscripts.

\begin{note}
The following are examples of array sections:

\begin{indentedcodelist}
a[0:6]
a[:6]
a[1:10]
a[1:]
b[10][:][:0]
c[1:10][42][0:6]
S.c[:100]
p->y[:10]
this->a[:N]
\end{indentedcodelist}

The first two examples are equivalent. If \code{a} is declared to be an eleven
element array, the third and fourth examples are equivalent. The fifth example
is a zero-length array section. The sixth example is not contiguous.  The
remaining examples show array sections that are formed from more general base
expressions.
\end{note}
\medskip
\end{ccppspecific}

\begin{fortranspecific}
Fortran has built-in support for array sections although some
restrictions apply to their use, as enumerated in the following section.
\end{fortranspecific}

\restrictions
Restrictions to array sections are as follows:

\begin{itemize}
\item An array section can appear only in clauses where it is explicitly allowed.

\item A \plc{stride} expression may not be specified unless otherwise stated.

\begin{ccppspecific}

%\item An array section can only be specified for a base expression.
\item An element of an array section with a non-zero size must have a complete type.

\item The type of the base expression appearing in an array section must be
    an array or pointer type.

\end{ccppspecific}

\begin{cppspecific}
\item If the type of the base expression of an array section is a reference to a type \plc{T} then the type will be considered to be \plc{T} for all purposes of the array section.

\item An array section cannot be used in a C++ user-defined \code{[]}-operator.
\end{cppspecific}

\begin{fortranspecific}

\item If a stride expression is specified, it must be positive.

\item The upper bound for the last dimension of an assumed-size dummy
  array must be specified.

\item If a list item is an array section with vector subscripts, the
  first array element must be the lowest in the array element order of
  the array section.


\end{fortranspecific}


\end{itemize}




\section{Iterators}
\index{iterators}
\label{sec:iterators}

Iterators are identifiers that expand to multiple values in the clause on which they appear.

The syntax of an \plc{iterators-definition} is the following:
\begin{ompSyntax}
\plc{iterator-specifier [}, \plc{iterators-definition ]}
\end{ompSyntax}

The syntax of an \plc{iterator-specifier} is one of the following:
\begin{indentedcodelist}
\plc{[ iterator-type ] } \plc{identifier} = \plc{range-specification}
\end{indentedcodelist}

where:
\begin{itemize}
\item \plc{identifier} is a base language identifier.
\begin{ccppspecific}
\item \plc{iterator-type} is a type name.
\end{ccppspecific}
\begin{fortranspecific}
\item \plc{iterator-type} is a type specifier.
\end{fortranspecific}

\item \plc{range-specification} is of the form \plc{begin}\code{:}\plc{end[}\code{:}\plc{step]} where \plc{begin}, \plc{end} and \plc{step} are expressions for which their types can be converted to the \plc{iterator-type} type.

\begin{ccppspecific}
\item In an \plc{iterator-specifier}, if the \plc{iterator-type} is not specified then the type of that iterator is of \code{int} type.
\end{ccppspecific}

%\newpage %% HACK

\begin{fortranspecific}
\item In an \plc{iterator-specifier}, if the \plc{iterator-type} is not specified then the type of that iterator is default integer.
\end{fortranspecific}
\end{itemize}

In a \plc{range-specification}, if the \plc{step} is not specified its value is implicitly defined to be 1.

An iterator only exists in the context of the clause on which it appears. An iterator also hides all accessible symbols with the same name in the context of the clause.

The use of a variable in an expression that appears in the \plc{range-specification} causes an implicit reference to the variable in all enclosing constructs.

\begin{ccppspecific}
The values of the iterator are the set of values $i_{0}$...$i_{N-1}$ where $i_{0}=begin$,  $i_{j}=i_{j-1} + step$ and
\begin{itemize}
\item $i_{0} < end$ and $i_{N-1} < end$ and $i_{N-1} + step >= end$ if $step > 0$.
\item $i_{0} > end$ and $i_{N-1} > end$ and $i_{N-1} + step <= end$ if $step < 0$.
\end{itemize}
\end{ccppspecific}
\begin{fortranspecific}
The values of the iterator are the set of values $i_{1}$...$i_{N}$ where $i_{1}=begin$,  $i_{j}=i_{j-1} + step$ and
\begin{itemize}
\item $i_{1} <= end$ and $i_{N} <= end$ and $i_{N} + step > end$ if $step > 0$.
\item $i_{1} >= end$ and $i_{N} >= end$ and $i_{N} + step < end$ if $step < 0$.
\end{itemize}
\end{fortranspecific}
The set of of values will be empty if no possible value complies with the conditions above.

 For those clauses that contain expressions containing iterator identifiers, the
effect is as if the list item is instantiated within the clause for each
value of the iterator in the set defined above, substituting each occurrence of
the iterator identifier in the expression with the iterator value. If the set of values of the iterator is empty then the effect is as if the clause was not specified.

\restrictions

\begin{itemize}
\item An expression containing an iterator identifier can only appear in clauses that explicitly allow expressions containing iterators.
\begin{ccppspecific}
\item The \plc{iterator-type} must be an integral or pointer type.
\end{ccppspecific}
\begin{fortranspecific}
\item The \plc{iterator-type} must be an integer type.
\end{fortranspecific}
\item If the \plc{step} expression of a \plc{range-specification} equals zero the behavior is unspecified.
\item Each iterator identifier can only be defined once in an \plc{iterators-definition}.
\item Iterators cannot appear in the \plc{range-specification}.
\end{itemize}


\section{Variant Directives}
\label{sec:Variant Directives}
\index{variant directives}
\index{directives!variant directives}

\subsection{OpenMP Context}
\label{subsec:OpenMP Context}

At any point in a program, an OpenMP context exists that defines traits describing the active OpenMP constructs, the execution devices, and functionallity supported by the implementation. The traits are grouped in trait sets. The following trait sets exist: \plc{construct}, \plc{device} and \plc{implementation}.

The \plc{construct} set is composed of the directive names, each being a trait, of all enclosing executable directives at that point in the program up to a \code{target} directive. Combined and composite constructs will be added to the set as independent constructs in the same nesting order specified by the original construct. The set is ordered by their nesting level in increasing order. In addition, if the point in the program is not enclosed by a \code{target} directive, the following rules will be applied in order:
\begin{enumerate} 
 \item for functions with a \code{declare simd} directive, the \plc{simd} trait will be added at the beginning of the set for the generated SIMD versions.  
 \item for functions with a \code{declare variant} directive, the selectors of the \code{construct} selector set will be added in the same order at the beginning of the set.
 \item for functions within a \code{declare target} block, the \plc{target} trait will be added at the beginning of the set for the versions of the function being generated for \code{target} regions.
\end{enumerate}

The \plc{simd} trait can be further defined with properties that match the clauses accepted by the \code{declare}~\code{simd} directive with the same name and semantics. The \plc{simd} trait will define at least the \plc{simdlen} property and one of  the \plc{inbrach} or \plc{notinbranch} properties.

The \plc{device} set includes traits that define the characteristics of the device being targeted by the compiler at that point in the program. At least the following traits must be defined:
\begin{itemize}
 \item The \plc{kind(kind-name-list)} trait specifies the general kind of the device. The following \plc{kind-name} values are defined:
 \begin{itemize}
  \item \plc{host} specifies that the device is the host device.
  \item \plc{nohost} specifies that the devices is not the host device. 
  \item Values defined in the ``OpenMP Context Definitions'' document which is available on \url{http://www.openmp.org/}. 
 \end{itemize}
 \item The \plc{isa(isa-name-list)} trait specifies the Instruction Set Architectures supported by the device. The accepted \plc{isa-name} values are implementation defined.
 \item The \plc{arch(arch-name-list)} trait specifies the architectures supported by the device. The accepted \plc{arch-name} values are implementation defined.
\end{itemize}

The \plc{implementation} set includes traits that describe the functionallity supported by the OpenMP implementation at that point in the program. At least the following traits can be defined:
\begin{itemize}
 \item The \plc{vendor(vendor-name)} trait specifies the name of the vendor of the implementation. OpenMP defined values for \plc{vendor-name} are defined in the ``OpenMP Context Definitions'' document which is available on \url{http://www.openmp.org/}. 
 \item The \plc{extension(extension-name-list)} trait specifies vendor specific extensions to the OpenMP specification. The accepted \plc{extension-name} values are implementation defined.
 \item A trait with the same name corresponding to each clause that can be supplied to the \code{requires} directive.
\end{itemize}

Implementations can define further traits in the \plc{device} and \plc{implementation} sets. All implementation defined traits must follow the following syntax:
\begin{ompSyntax}
\plc{identifier[}(\plc{context-element[}, \plc{context-element[}, \plc{...]]})\plc{]}

\plc{context-element}:
  \plc{identifier[}(\plc{context-element[}, \plc{context-element[}, \plc{...]]})\plc{]}
  or
  \plc{context-value}

\plc{context-value}:
  \plc{string}
  or
  \plc{integer expression}
\end{ompSyntax}

where \plc{identifier} is a base language identifier.

\subsection{Context Selectors}
\label{subsec:Context Selectors} 

Context selectors allow to define the properties of an OpenMP context that a directive or clause wants to match. OpenMP defines different sets of selectors, each containing different selectors.

The syntax to define a \plc{context-selector-specification} is the following:

\begin{ompSyntax}
\plc{trait-set-selector[},\plc{trait-set-selector[},\plc{...]]}

\plc{trait-set-selector}:
   \plc{trait-set-selector-name}={\plc{trait-selector[}, \plc{trait-selector[}, \plc{...]]}}

\plc{trait-selector}:
   \plc{trait-selector-name[}(\plc{trait-property[}, \plc{trait-property[}, \plc{...]]})\plc{]}
\end{ompSyntax}

The \code{construct} selector set defines which \plc{construct} traits should be active in the OpenMP context. The following selectors can be defined in the \code{construct} set: \code{target}, \code{teams}, \code{parallel}, \code{for} (in C/C++), \code{do} (in Fortran), and \code{simd}. The properties of each selector are the same defined for the corresponding trait. The \code{construct} selector is an ordered list.

The \code{device} and \code{implementation} selector sets define which traits should be active in the corresponding trait set of the OpenMP context. The same traits defined in the corresponding traits sets can be used as selectors with the same properties. The \code{kind} selector of the \code{device} selector set can also be set to the value \code{any} which is as if no \code{kind} selector was specified. 

The \code{user} selector set defines the \code{condition} selector that provides additional user-defined conditions. 
\begin{cspecific}
The \code{condition(}\plc{boolean-expr}\code{)} selector defines a \plc{constant expression} that must evaluate to true for the selector to be true. 
\end{cspecific}
\begin{cppspecific}
The \code{condition(}\plc{boolean-expr}\code{)} selector defines a \plc{constexpr} expression that must evaluate to true for the selector to be true.  
\end{cppspecific}
\begin{fortranspecific}
The \code{condition(}\plc{logical-expr}\code{)} selector defines a \plc{constant expression} that must evaluate to true for the selector to be true. 
\end{fortranspecific}

Implementations can allow further selectors to be specified. Implementations can ignore specified selectors that are not those described in this section.

\restrictions
\begin{itemize}
 \item Each \plc{trait-set-selector-name} can only be specified once.
 \item Each \plc{trait-selector-name} can only be specified once.
\end{itemize}

\subsection{Matching and Scoring Context Selectors}
\label{subsec:Matching and Scoring Context Selectors}

A given context selector is compatible with a given OpenMP context if:
\begin{itemize}
 \item All selectors in the \code{user} set of the context selector are true,
 \item All selectors in the \code{construct}, \code{device} and \code{implementation} sets of the context selector appear in the corresponding trait set of the OpenMP context,
 \item For each selector in the context selector, its properties are a subset of the properties of the corresponding trait of the OpenMP context,
 \item Selectors in the \code{construct} set of the context selector appear in the same relative order as their correspending traits in the \plc{construct} trait set of the OpenMP context.
\end{itemize}

Some properties of the \code{simd} selector have special rules to match the properties of the \plc{simd} trait:
\begin{itemize}
 \item The \code{simdlen(}\plc{N}\code{)} property of the selector matches the \plc{simdlen(M)} trait of the OpenMP context $M \% N$ equals zero.
 \item The \code{aligned(}\plc{list:N}\code{)} property of the selector matches the \plc{aligned(list:M)} trait of the OpenMP context if $N \% M$ equals zero.
\end{itemize}

Among compatible context selectors a score will be computed using the following algorithm:
\begin{enumerate}
 \item Each trait appearing in the \plc{construct} trait set in the OpenMP context gets assigned the value $2^{p-1}$ where $p$ is the position of trait in the set.
 \item The \code{kind}, \code{arch} and \code{isa} selectors will have the value $2^{l}$, $2^{l+1}$ and $2^{l+2}$ respectively where $l$ is the number of traits in the \plc{construct} set.
 \item Additional implementation allowed selector values are implementation defined.
 \item Other selectors have a value of zero.
 \item Context selectors which are a strict subset of another context selector have a score of zero. For other context selectors, the final score is the addition of the values of all the specified selectors plus $1$. If the traits corrpesding to the \code{construct} selectors appear multiple times in the OpenMP context, the highest valued subset of traits that contains all the selectors in the same order will be used.
\end{enumerate}

\subsection{\hcode{declare variant} Directive}
\index{declare variant@{\code{declare variant}}}
\index{directives!declare variant@{\code{declare variant}}}
\label{subsec:declare variant Directive}
\summary
The \code{declare variant} declares a function to be a specialized variant of another function and in which context it should be used.

\syntax
\begin{ccppspecific}
\begin{samepage}
The syntax of the \code{declare variant} directive is as follows:

\begin{ompcPragma}
#pragma omp declare variant(\plc{base-func-name}) \plc{[clause[ [},\plc{] clause] ... ] new-line}
   \plc{function definition or declaration}
\end{ompcPragma}
\end{samepage}

\begin{samepage}
where \plc{clause} is one of the following{}:

\begin{indentedcodelist}
match(\plc{context-selector-specification})
\end{indentedcodelist}
\end{samepage}
\end{ccppspecific}

\begin{fortranspecific}
The syntax of the \code{declare variant} directive is as follows:

\begin{ompfPragma}
!$omp declare variant(\plc{[proc-name}:\plc{]base-proc-name}) \plc{[clause[ [},\plc{] clause] ... ]}
\end{ompfPragma}

where \plc{clause} is one of the following{}:

\begin{indentedcodelist}
match(\plc{context-selector-specification})
\end{indentedcodelist}
\end{fortranspecific}

\descr

The use of a \code{declare}~\code{variant} directive declares the function to be a function variant of the \plc{base-func-name} or \plc{base-proc-name} function. If no \code{match} clause is specified then the context selector for the variant is empty. If a \code{match} clause is specified then the context selector in the clause will be associated to the variant.

At any point, after the declaration of variant for a given base function, where there is a direct call to that base function the compiler will check if there is any variant that is compatible with OpenMP context at that point. Among the compatible variants, the variant with the highest score according to the algorithm described in Section \ref{subsec:Matching and Scoring Context Selectors} will be selected. If multiple variants have the highest score, it is unspecified which one will be selected. If a compatible variant exists, the original call to the base function will be replaced with a call to the selected variant function. 

The prototype of the variant function shall, in general, match that of the base function. It is implementation defined if for some specific OpenMP context the prototype of the variant should differ, and how, from that of the base function.

\restrictions
Restrictions to the \code{declare variant} directive are as follows:

\begin{itemize}
\item At most one \code{match} clause can appear in a \code{declare variant} directive.
\item If the function definition has a \code{declare}~\code{variant} directive or if a declaration of the function in the same compilation unit has a \code{declare}~\code{variant}, then, calling the variant function directly in an OpenMP context that is different than the one specified by the \code{construct} set of the context selector is non-conforming.

\begin{ccppspecific}
\item If the function has any declarations, then the \code{declare}~\code{variant} directive for any
declaration that has one must be equivalent. If the function definition has a \code{declare}~\code{variant} it must also be equivalent.
Otherwise, the result is unspecified.
\end{ccppspecific}

\begin{cppspecific}
\item \plc{base-func-name} should not designate an overloaded function name. Otherwise, \plc{base-func-name} must be a function declaration without the return type.
\item The \plc{base-func-name} of a \code{declare variant} directive cannot be a template function.
\item The \plc{base-func-name} of a \code{declare variant} directive cannot be a virtual function.
\end{cppspecific}

\begin{fortranspecific}
\item \plc{proc-name} must not be a generic name, procedure pointer or entry name
\item If \plc{proc-name} is omitted, the \code{declare variant} directive must appear in the specification part of a subroutine subprogram or a function subprogram.
\item Any \code{declare variant} directive must appear in the specification part of a subroutine, subprogram, function subprogram or interface body to which it applies.
\item If a \code{declare variant} directive is specified in an interface block for a procedure, it must match a \code{declare variant} directive in the definition of the procedure.
\item If a procedure is declared via a procedure declaration statement, the procedure \plc{proc-name} should appear in the same specification.
\item If a \code{declare variant} diretive is specified for a procedure name with explicit interface and a \code{declare variant} directive is also specified for the definition of the procedure the two \code{declare variant} directives must match. Otherwise the result is unspecified.
\end{fortranspecific}
\end{itemize}

\crossreferences
\begin{itemize}
\item OpenMP Context Specification, see \specref{subsec:OpenMP Context}.
\item Context Selectors, see \specref{subsec:Context Selectors}.
\end{itemize}


\subsection{Metadirective Meta-Directive}
\label{sec:directive variants}
\summary
The metadirective meta-directive can specify multiple directive variants
of which one may be conditionally selected to replace the meta-directive based
on the enclosing context.

\syntax
\begin{ccppspecific}
The syntax of the metadirective meta-directive takes one of the
following forms:
\begin{ompcPragma}
#pragma omp metadirective \plc{[clause[ [},\plc{] clause] ... ] new-line}
\end{ompcPragma}
or
\begin{ompcPragma}
#pragma omp begin metadirective \plc{[clause[ [},\plc{] clause] ... ] new-line}
    \plc{stmt(s)}
#pragma omp end metadirective 
\end{ompcPragma}


\begin{samepage}
where \plc{clause} is:
\begin{indentedcodelist}
    when(\plc{context-selector-specification}: \plc{[directive-variant]})
    default(\plc{directive-variant})
\end{indentedcodelist}
\end{samepage}

\end{ccppspecific}

\begin{fortranspecific}
The syntax of the metadirective meta-directive takes one of the following
forms:

\begin{ompfPragma}
!$omp metadirective \plc{[clause[ [},\plc{] clause] ... ]}
\end{ompfPragma}

or 

\begin{ompfPragma}
!$omp begin metadirective \plc{[clause[ [},\plc{] clause] ... ]}
    \plc{stmt(s)}
!$omp end metadirective 
\end{ompfPragma}

\begin{samepage}
where \plc{clause} is:

\begin{indentedcodelist}
    when(\plc{context-selector-specification}: \plc{[directive-variant]})
    default(\plc{directive-variant})
\end{indentedcodelist}
\end{samepage}

\end{fortranspecific}

In the \code{when} clause, \plc{context-selector-specification} specifies a context
selector (see Section~\ref{subsec:Context Selectors}).

In the \code{when} and \code{default} clauses, \plc{directive-variant}
has the following form and specifies a directive variant that is an OpenMP
directive that has the same directive name and clauses.

\begin{indentedcodelist}
\plc{ directive-name [clause[ [},\plc{] clause] ... ]}
\end{indentedcodelist}

\descr

The metadirective directive is a meta-directive that
behaves as if it is either ignored or replaced by the directive variant
specified in one of the \code{when} or \code{default} clauses that appears on
the directive.

The OpenMP context for a given meta-directive is defined according to Section
\ref{subsec:OpenMP Context}.  For each \code{when} clause that appears on the
meta-directive, the specified directive variant, if present, is a candidate to
replace the meta-directive if the corresponding context selector is compatible
with the OpenMP context according to the matching rules defined in
Section~\ref{subsec:Matching and Scoring Context Selectors}.  If only one
compatible context selector specified by a \code{when} clause has the highest
score and it specifies a directive variant, the directive variant will replace
the meta-directive. If more than one \code{when} clause specifies a compatible
context selector that has the highest computed score and at least one
specifies a directive variant, the first directive variant specified in the
lexical order of those \code{when} clauses will replace the meta-directive.

If no context selector from any \code{when} clause is compatible with the
OpenMP context and a \code{default} clause is present, the directive variant
specified in the \code{default} clause will replace the meta-directive.

If a directive variant is not selected to replace the meta-directive according
to the above rules, the meta-directive has no effect on the execution of program. 

The \code{begin}~\code{metadirective} directive behaves
identically to the \code{metadirective} directive, except that the
directive syntax for the specified directive variants must accept a paired
\code{end}~\plc{directive}.  For any directive variant that is selected to
replace the \code{begin}~\code{metadirective} meta-directive, the
\code{end}~\code{metadirective} directive will be implicitly
replaced by its paired \code{end}~\plc{directive} to demarcate the statements
that are affected by or are associated with the directive variant. If no
directive variant is selected to replace the meta-directive, its paired
\code{end}~\code{metadirective} directive is ignored.

\restrictions
Restrictions for the metadirective directive are as follows:

\begin{itemize}
    \item The directive variant appearing in a \code{when} or \code{default}
        clause must not specify a \code{metadirective},
        \code{begin}~\code{metadirective}, or \code{end}~\code{metadirective}
        directive.

    \item The context selector that appears in a \code{when} clause must not
        specify any properties for the \code{simd} selector.

    \item Any replacement that occurs for a metadirective meta-directive must
        not result in a non-conforming OpenMP program.

    \item Any directive variant that is specified by a \code{when} or \code{default}
        clause on a \code{begin}~\code{metadirective}
        meta-directive must be an OpenMP directive that has a paired \code{end}~\plc{directive}, and
        the \code{begin}~\code{metadirective} directive must have a paired
        \code{end}~\code{metadirective} directive.

    \item The \code{default} clause may appear at most once on the directive.
\end{itemize}

% This is an included file. See the master file for more information.
%
% When editing this file:
%
%    1. To change formatting, appearance, or style, please edit openmp.sty.
%
%    2. Custom commands and macros are defined in openmp.sty.
%
%    3. Be kind to other editors -- keep a consistent style by copying-and-pasting to
%       create new content.
%
%    4. We use semantic markup, e.g. (see openmp.sty for a full list):
%         \code{}     % for bold monospace keywords, code, operators, etc.
%         \plc{}      % for italic placeholder names, grammar, etc.
%
%    5. There are environments that provide special formatting, e.g. language bars.
%       Please use them whereever appropriate.  Examples are:
%
%         \begin{fortranspecific}
%         This is text that appears enclosed in blue language bars for Fortran.
%         \end{fortranspecific}
%
%         \begin{note}
%         This is a note.  The "Note -- " header appears automatically.
%         \end{note}
%
%    6. Other recommendations:
%         Use the convenience macros defined in openmp.sty for the minor headers
%         such as Comments, Syntax, etc.
%
%         To keep items together on the same page, prefer the use of
%         \begin{samepage}.... Avoid \parbox for text blocks as it interrupts line numbering.
%         When possible, avoid \filbreak, \pagebreak, \newpage, \clearpage unless that's
%         what you mean. Use \needspace{} cautiously for troublesome paragraphs.
%
%         Avoid absolute lengths and measures in this file; use relative units when possible.
%         Vertical space can be relative to \baselineskip or ex units. Horizontal space
%         can be relative to \linewidth or em units.
%
%         Prefer \emph{} to italicize terminology, e.g.:
%             This is a \emph{definition}, not a placeholder.
%             This is a \plc{var-name}.
%


\section{Canonical Loop Form}
\label{sec:Canonical Loop Form}
\index{canonical loop form}
\begin{cppspecific}
A range-based for loop with random access iterator has a \emph{canonical loop form}.
\end{cppspecific}
\begin{ccppspecific}
A loop has \emph{canonical loop form} if it conforms to the following:

\medskip
\nolinenumbers
\renewcommand{\arraystretch}{1.0}
\tablefirsthead{%
    \hline\\[-2ex]
    \multicolumn{2}{l}{\hspace*{-5pt}%
        {\scode{for (}\splc{init-expr}\scode{; }\splc{test-expr}\scode{; }\splc{incr-expr}\scode{) }\splc{structured-block}}}\\[2pt]
    \hline\\[-2ex]
}
\tablehead{%
    \multicolumn{2}{l}{\small\slshape continued from previous page}\\
    \hline\\[-2ex]
}
\tabletail{%
    \hline\\[-2ex]
    \multicolumn{2}{l}{\small\slshape continued on next page}\\
}
\tablelasttail{\hline}
\begin{supertabular}{ p{0.8in} p{4.5in}}
    {\splc{init-expr}} & One of the following:\\
    & {\splc{var}} = {\splc{lb}}\\
    & {\splc{integer-type}} {\splc{var}} = {\splc{lb}}\\
    & {\splc{random-access-iterator-type}} {\splc{var}} = {\splc{lb}}\\
    & {\splc{pointer-type}} {\splc{var}} = {\splc{lb}}\\
    & \\
    {\splc{test-expr}} & One of the following:\\
    & {\splc{var}} {\splc{relational-op}} {\splc{b}}\\
    & {\splc{b}} {\splc{relational-op}} {\splc{var}}\\
    & \\
    {\splc{incr-expr}} & One of the following:\\
    & ++{\splc{var}}\\
    & {\splc{var}}++\\
    & {-} {-} {\splc{var}}\\
    & {\splc{var {-} {-}}}\\
    & {\splc{var}} += {\splc{incr}}\\
    & {\splc{var}} {-} = {\splc{incr}}\\
    & {\splc{var}} = {\splc{var}} + {\splc{incr}}\\
    & {\splc{var}} = {\splc{incr}} + {\splc{var}}\\
    & {\splc{var}} = {\splc{var}} - {\splc{incr}}\\
    & \\
    {\splc{var}} & One of the following:\\
    & \hspace{1.5em}A variable of a signed or unsigned integer type.\\
    & \hspace{1.5em}For C++, a variable of a random access iterator type.\\
    & \hspace{1.5em}For C, a variable of a pointer type.\\
    & If this variable would otherwise be shared, it is implicitly made private in the loop
    construct. This variable must not be modified during the execution of the {\splc{for-loop}}
    other than in {\splc{incr-expr}}. Unless the variable is specified {\scode{lastprivate}}
    or {\scode{linear}} on the loop construct, its value after the loop is unspecified.\\
    {\splc{relational-op}} & One of the following:\\
    & {\scode{<}}\\
    & {\scode{<=}}\\
    & {\scode{>}}\\
    & {\scode{>=}}\\
    & {\scode{!=}}\\
    & \\
    {\splc{lb}} and {\splc{b}} & Loop invariant expressions of a type compatible with the type of {\splc{var}}.\\
    & \\
    {\splc{incr}} & A loop invariant integer expression.\\
\end{supertabular}
\medskip

\linenumbers

The canonical form allows the iteration count of all associated loops to be computed
before executing the outermost loop. The computation is performed for each loop in an
integer type. This type is derived from the type of \plc{var} as follows:

\begin{itemize}
    \item If \plc{var} is of an integer type, then the type is the type of \plc{var}.

    \item For C++, if \plc{var} is of a random access iterator type, then the type is the type that
    would be used by \plc{std::distance} applied to variables of the type of \plc{var}.

    \item For C, if \plc{var} is of a pointer type, then the type is \code{ptrdiff_t}.
\end{itemize}

The behavior is unspecified if any intermediate result required to compute the iteration
count cannot be represented in the type determined above.

There is no implied synchronization during the evaluation of the \plc{lb}, \plc{b}, or \plc{incr}
expressions. It is unspecified whether, in what order, or how many times any side effects
within the \plc{lb}, \plc{b}, or \plc{incr} expressions occur.

\begin{note}
Random access iterators are required to support random access to elements in
constant time. Other iterators are precluded by the restrictions since they can take linear
time or offer limited functionality. It is therefore advisable to use tasks to parallelize
those cases.

% The word "Restrictions" seems out of place; was it meant to be a header outside of the Note?

%Restrictions
\end{note}

\restrictions
The following restrictions also apply:

\begin{itemize}
    \item If \plc{test-expr} is of the form \plc{var} \plc{relational-op}
    \plc{b} and \plc{relational-op} is < or <= then \plc{incr-expr} must cause \plc{var} to increase on each
    iteration of the loop. If \plc{test-expr} is of
    the form \plc{var} \plc{relational-op} \plc{b} and \plc{relational-op}
    is > or >= then \plc{incr-expr} must cause \plc{var} to decrease on each iteration of the loop.

    \item If \plc{test-expr} is of the form \plc{b} \plc{relational-op}
    \plc{var} and \plc{relational-op} is < or <= then
    \plc{incr-expr} must cause \plc{var} to decrease on each iteration of the loop. If \plc{test-expr} is of
    the form \plc{b} \plc{relational-op} \plc{var} and \plc{relational-op}
    is > or >= then \plc{incr-expr} must cause \plc{var} to increase on each iteration of the loop.

    \item If \plc{test-expr} is of the form \plc{b} != \plc{var} or
    \plc{var} != \plc{b} then \plc{incr-expr} must cause \plc{var}
    either to increase on each iteration of the loop or to decrease on
    each iteration of the loop.

    \item For C++, in the \code{simd} construct the only random access iterator types that are
    allowed for \plc{var} are pointer types.

    \item The \plc{b}, \plc{lb} and \plc{incr} expressions may not reference
    \plc{var} of any of the associated loops.

    \item If \plc{relational-op} is != and \plc{incr-expr} is of the
    form that has \plc{incr} then \plc{incr} must be a constant expression and
    evaluate to -1 or 1.
\end{itemize}
\end{ccppspecific}





\section{Worksharing Constructs}
\label{sec:Worksharing Constructs}
\index{worksharing constructs}
\index{constructs!worksharing}
\index{worksharing!constructs}
A worksharing construct distributes the execution of the corresponding region among the
members of the team that encounters it. Threads execute portions of the region in the
context of the implicit tasks each one is executing. If the team consists of only one
thread then the worksharing region is not executed in parallel.

A worksharing region has no barrier on entry; however, an implied barrier exists at the
end of the worksharing region, unless a \code{nowait} clause is specified. If a \code{nowait}
clause is present, an implementation may omit the barrier at the end of the worksharing
region. In this case, threads that finish early may proceed straight to the instructions
following the worksharing region without waiting for the other members of the team to
finish the worksharing region, and without performing a flush operation.

The OpenMP API defines the following worksharing constructs, and these are described
in the sections that follow:

\begin{itemize}
\item loop construct

\item \code{sections} construct

\item \code{single} construct

\item \code{workshare} construct
\end{itemize}

\begin{samepage}
\restrictions
The following restrictions apply to worksharing constructs:

\begin{itemize}
\item Each worksharing region must be encountered by all threads in a team or by none at
all, unless cancellation has been requested for the innermost enclosing parallel
region.

\item The sequence of worksharing regions and \code{barrier} regions encountered must be the
same for every thread in a team
\end{itemize}
\end{samepage}










\subsection{Loop Construct}
\label{subsec:Loop Construct}
\index{loop@{\code{loop}}}
\index{constructs!loop@{\emph{loop}}}
\index{constructs!do@{\code{do} \emph{Fortran}}}
\index{do@{\code{do}, \emph{Fortran}}}
\index{for@{\code{for}, \emph{C/C++}}}
\index{constructs!for@{\code{for}, \emph{C/C++}}}
\summary
The loop construct specifies that the iterations of one or more associated loops will be
executed in parallel by threads in the team in the context of their implicit tasks. The
iterations are distributed across threads that already exist in the team executing the
\code{parallel} region to which the loop region binds.

\syntax
\begin{ccppspecific}
The syntax of the loop construct is as follows:

\begin{ompcPragma}
#pragma omp for \plc{[clause[ [},\plc{] clause] ... ] new-line}
    \plc{for-loops}
\end{ompcPragma}

where clause is one of the following:
\index{clauses!collapse@{\code{collapse}}}

\begin{indentedcodelist}
private(\plc{list})
firstprivate(\plc{list})
lastprivate(\plc{[ lastprivate-modifier}:\plc{] list})
linear(\plc{list[ }:\plc{ linear-step]})
reduction(\plc{reduction-identifier }:\plc{ list})
schedule(\plc{[modifier [}, \plc{modifier]}:\plc{]kind[},\plc{ chunk_size]})
collapse(\plc{n})
ordered\plc{[}(\plc{n})\plc{]}
nowait
allocate(\plc{[allocator: ]}\plc{list})
\end{indentedcodelist}

The \code{for} directive places restrictions on the structure of all associated \plc{for-loops}.
Specifically, all associated \plc{for-loops} must have \emph{canonical loop form} (see
\specref{sec:Canonical Loop Form}).
\end{ccppspecific}

\begin{fortranspecific}
The syntax of the loop construct is as follows:

\begin{ompfPragma}
!$omp do \plc{[clause[ [},\plc{] clause] ... ]}
   \plc{do-loops}
\textsl{[}!$omp end do \textsl{[}nowait\textsl{]]}
\end{ompfPragma}

where \plc{clause} is one of the following:

\begin{indentedcodelist}
private(\plc{list})
firstprivate(\plc{list})
lastprivate(\plc{[ lastprivate-modifier}:\plc{] list})
linear(\plc{list[ }:\plc{ linear-step]})
reduction(\plc{reduction-identifier }:\plc{ list})
schedule(\plc{[modifier [}, \plc{modifier]}:\plc{]kind[},\plc{ chunk_size]})
collapse(\plc{n})
ordered\plc{[}(\plc{n})\plc{]}
allocate(\plc{[allocator: ]}\plc{list})
\end{indentedcodelist}

If an \code{end}~\code{do} directive is not specified, an \code{end}~\code{do} directive is assumed at the end of the
\plc{do-loops}.

Any associated \plc{do-loop} must be a \plc{do-construct} or an
\plc{inner-shared-do-construct} as defined by the Fortran standard. If
an \code{end}~\code{do} directive follows a \plc{do-construct} in
which several loop statements share a \code{DO} termination statement,
then the directive can only be specified for the outermost of these
\code{DO} statements.

If any of the loop iteration variables would otherwise be shared, they are implicitly
made private on the loop construct.
\end{fortranspecific}


\binding
The binding thread set for a loop region is the current team. A loop region binds to the
innermost enclosing \code{parallel} region. Only the threads of the team executing the
binding \code{parallel} region participate in the execution of the loop iterations and the
implied barrier of the loop region if the barrier is not eliminated by a \code{nowait} clause.

\descr
The loop construct is associated with a loop nest consisting of one or more loops that
follow the directive.

There is an implicit barrier at the end of a loop construct unless a \code{nowait} clause is
specified.

\index{clauses!collapse@{\code{collapse}}}
The \code{collapse} clause may be used to specify how many loops are
associated with the loop construct. The parameter of the \code{collapse}
clause must be a constant positive integer expression. If a \code{collapse}
clause is specified with a parameter value greater than 1, then the
iterations of the associated loops to which the clause applies are collapsed
into one larger iteration space that is then divided according
to the \code{schedule} clause. The sequential execution of the iterations
in these associated loops determines the order of the iterations in the
collapsed iteration space. If no \code{collapse} clause is present or its
parameter is 1, the only loop that is associated with the loop construct
for the purposes of determining how the iteration space is divided according
to the \code{schedule} clause is the one that immediately follows the
loop directive.

If more than one loop is associated with the loop construct then the
number of times that any intervening code between any two associated
loops will be executed is unspecified but will be at least once per
iteration of the loop enclosing the intervening code and at most once
per iteration of the innermost loop associated with the construct. If the
iteration count of any loop associated with the loop construct that does not
enclose the intervening code is zero then the behavior is unspecified.

The iteration count for each associated loop is computed before entry to the
outermost loop. If execution of any associated loop changes any of the values
used to compute any of the iteration counts, then the behavior is unspecified.

The integer type (or kind, for Fortran) used to compute the iteration count
for the collapsed loop is implementation defined.

\index{clauses!schedule@{\code{schedule}}} A worksharing loop has
logical iterations numbered 0,1,...,N-1 where N is the number of loop
iterations, and the logical numbering denotes the sequence in which
the iterations would be executed if a set of associated loop(s) were
executed sequentially.  At the beginning of each logical iteration,
the loop iteration variable of each associated loop has the value that
it would have if the set of the associated loop(s) were executed
sequentially.  The \code{schedule} clause specifies how iterations of
these associated loops are divided into contiguous non-empty subsets,
called chunks, and how these chunks are distributed among threads of
the team. Each thread executes its assigned chunk(s) in the context of
its implicit task.  The iterations of a given chunk are executed in
sequential order by the assigned thread.  The \plc{chunk_size}
expression is evaluated using the original list items of any variables
that are made private in the loop construct. It is unspecified
whether, in what order, or how many times, any side effects of the
evaluation of this expression occur. The use of a variable in a
\code{schedule} clause expression of a loop construct causes an
implicit reference to the variable in all enclosing constructs.

Different loop regions with the same schedule and iteration count, even if
they occur in the same parallel region, can distribute iterations among
threads differently. The only exception is for the \code{static} schedule
as specified in Table~\ref{tab:Schedule-Values}. Programs that depend
on which thread executes a particular iteration under any other circumstances
are non-conforming.

See \specref{subsubsec:Determining the Schedule of a Worksharing Loop}
for details of how the schedule for a worksharing loop is
determined.

The schedule \plc{kind} can be one of those specified in
Table~\ref{tab:Schedule-Values}.

The schedule \plc{modifier} can be one of those specified in
Table~\ref{tab:Schedule Clause Modifier Values}. If the
\code{static} schedule kind is specified or if the \code{ordered}
clause is specified, and if the \code{nonmonotonic} modifier is
not specified, the effect is as if the \code{monotonic} modifier
is specified. Otherwise, unless the \code{monotonic} modifier is
specified, the effect is as if the \code{nonmonotonic} modifier
is specified.

The \code{ordered} clause with the parameter may also be used to specify
how many loops are associated with the loop construct. The parameter of
the \code{ordered} clause must be a constant positive integer expression
if specified. The parameter of the \code{ordered} clause does not
affect how the logical iteration space is then divided. If an \code{ordered}
clause with the parameter is specified for the loop construct, then those
associated loops form a \emph{doacross loop nest}.

If the value of the parameter in the \code{collapse} or \code{ordered}
clause is larger than the number of nested loops following the construct,
the behavior is unspecified.

\nolinenumbers
\vspace{1ex}\renewcommand{\arraystretch}{1.5}
\tablefirsthead{%
\hline\\[-3ex]
}
\tablehead{%
\multicolumn{2}{l}{\small\slshape table continued from previous page}\\
\hline\\[-3ex]
}
\tabletail{%
\hline\\[-4ex]
\multicolumn{2}{l}{\small\slshape table continued on next page}\\
}
\tablelasttail{\hline}
\tablecaption{\code{schedule} Clause \plc{kind} Values\label{tab:Schedule-Values}}
\begin{supertabular}{ p{0.8in} p{4.3in} }
{\scode{static}} & When {\scode{schedule(static,}\splc{ chunk_size}\scode{)}} is specified, iterations are divided
into chunks of size {\splc{chunk_size}}, and the chunks are assigned to the threads in
the team in a round-robin fashion in the order of the thread number.\\

 & When no {\splc{chunk_size}} is specified, the iteration space is divided into chunks that
are approximately equal in size, and at most one chunk is distributed to each
thread. The size of the chunks is unspecified in this case.\\

 & A compliant implementation of the {\scode{static}} schedule must ensure that the
same assignment of logical iteration numbers to threads will be used in two
loop regions if the following conditions are satisfied: 1) both loop regions have
the same number of loop iterations, 2) both loop regions have the same value
of {\splc{chunk_size}} specified, or both loop regions have no {\splc{chunk_size}} specified, 3)
both loop regions bind to the same parallel region, and 4) neither loop is
associated with a SIMD construct. A data dependence between the same
logical iterations in two such loops is guaranteed to be satisfied allowing safe
use of the {\scode{nowait}} clause.\\

\index{dynamic@{{\scode{dynamic}}}}
{\scode{dynamic}} & When {\scode{schedule(dynamic,}\splc{ chunk_size}\scode{)}} is specified, the iterations are
distributed to threads in the team in chunks. Each
thread executes a chunk of iterations, then requests another chunk, until no
chunks remain to be distributed. \\

 & Each chunk contains {\splc{chunk_size}} iterations, except for the
chunk that contains the sequentially last iteration, which may have fewer iterations.\\

 & When no {\splc{chunk_size}} is specified, it defaults to 1.\\

\index{guided@{{\scode{guided}}}}
{\scode{guided}} & When {\scode{schedule(guided,}\splc{ chunk_size}\scode{)}} is specified, the iterations are
assigned to threads in the team in chunks. Each thread executes a
chunk of iterations, then requests another chunk, until no chunks remain to be assigned.\\

 & For a {\splc{chunk_size}} of 1, the size of each chunk is proportional to the
number of unassigned iterations divided by the number of threads in the team,
decreasing to 1. For a {\splc{chunk_size}} with value $k$ (greater than 1), the
size of each chunk is determined in the same way, with the restriction
that the chunks do not contain fewer than $k$ iterations (except for the
chunk that contains the sequentially last iteration, which may have fewer
than $k$ iterations).
\\

 & When no {\splc{chunk_size}} is specified, it defaults to 1.\\

{\scode{auto}} & When {\scode{schedule(auto)}} is specified, the decision regarding scheduling is
\index{auto@{{\scode{auto}}}}
delegated to the compiler and/or runtime system. The programmer gives the
implementation the freedom to choose any possible mapping of iterations to
threads in the team.\\

{\scode{runtime}} & When {\scode{schedule(runtime)}} is specified, the decision regarding scheduling
is deferred until run time, and the schedule and chunk size are taken from the
{\splc{run-sched-var}} ICV. If the ICV is set to {\scode{auto}}, the schedule is implementation
defined.\\
\end{supertabular}

\linenumbers
\bigskip\bigskip


\begin{note}
For a team of $p$ threads and a loop of $n$ iterations, let $\blceil n/p \brceil$ be the integer $q$
that satisfies $n = p*q - r$, with $0 <= r < p$. One compliant implementation of the \code{static}
schedule (with no specified \plc{chunk_size}) would behave as though \plc{chunk_size} had been
specified with value $q$. Another compliant implementation would assign $q$ iterations to
the first $p-r$ threads, and $q-1$ iterations to the remaining $r$ threads. This illustrates why a
conforming program must not rely on the details of a particular implementation.

A compliant implementation of the \code{guided} schedule with a \plc{chunk_size} value of $k$
would assign $q = \blceil n/p \brceil$ iterations to the first available thread and set $n$ to the larger of
$n-q$ and $p*k$. It would then repeat this process until $q$ is greater than or equal to the
number of remaining iterations, at which time the remaining iterations form the final
chunk. Another compliant implementation could use the same method, except with
$q = \blceil n/(2p) \brceil$, and set $n$ to the larger of $n-q$ and $2*p*k$.
\end{note}

\nolinenumbers
\vspace{1ex}\renewcommand{\arraystretch}{1.5}
\tablefirsthead{%
\hline\\[-3ex]
}
\tablehead{%
\multicolumn{2}{l}{\small\slshape table continued from previous page}\\
\hline\\[-3ex]
}
\tabletail{%
\hline\\[-4ex]
\multicolumn{2}{l}{\small\slshape table continued on next page}\\
}
\tablelasttail{\hline}
\tablecaption{\code{schedule} Clause \plc{modifier} Values\label{tab:Schedule Clause Modifier Values}}
%% \vspace{1ex}
\begin{supertabular}{ p{1in} p{4.1in} }
{\scode{monotonic}} & When the {\scode{monotonic}} modifier is specified then each thread executes the chunks
that it is assigned in increasing logical iteration order.\\
{\scode{nonmonotonic}} & When the {\scode{nonmonotonic}} modifier is specified then chunks are assigned to threads
in any order and the behavior of an application that depends on any execution order of the chunks is unspecified.\\
{\scode{simd}} & When the {\scode{simd}} modifier is specified and the loop is associated with a SIMD construct, the {\splc{chunk_size}} for all chunks except the first and last chunks  is  $new\_chunk\_size = \blceil chunk\_size / simd\_width \brceil * simd\_width $ where {\splc{simd_width}} is an implementation-defined value. The first chunk will have at least {\splc{new_chunk_size}} iterations except if it is also the last chunk. The last chunk may have fewer iterations than {\splc{new_chunk_size}}. If the {\scode{simd}} modifier is specified and the loop is not associated  with a SIMD construct, the modifier is ignored.\\
\end{supertabular}
\linenumbers
\medskip

\def\omptWorksharing#1#2
{
\events

The \plc{#1-begin} event occurs after an implicit task encounters a
\code{#1} construct but before the task starts the execution of the structured
block of the \code{#1} region.

The \plc{#1-end} event occurs after a \code{#1} region finishes execution
but before resuming execution of the encountering task.

\tools

A thread dispatches a registered \code{ompt_callback_work}
callback for each occurrence of a \plc{#1-begin} and
\plc{#1-end} event in that thread. The callback occurs in the
context of the implicit task.  The callback has type signature
\code{ompt_callback_work_t}. The callback receives
\code{ompt_scope_begin} or \code{ompt_scope_end}
as its \plc{endpoint} argument, as appropriate, and
\code{#2} as its \plc{wstype} argument.
}
\omptWorksharing{loop}{ompt_work_loop}

\restrictions
Restrictions to the loop construct are as follows:

\begin{itemize}
\item There must be no OpenMP directive in the region between any
associated loops.

\item If a \code{collapse} clause is specified, exactly one loop must
occur in the region at each nesting level up to the number of loops
specified by the parameter of the \code{collapse} clause.

\item If the \code{ordered} clause is present, all loops associated
with the construct must be perfectly nested; that is there must be
no intervening code between any two loops.

\item The values of the loop control expressions of the loops associated with the loop
construct must be the same for all threads in the team.

\item Only one \code{schedule} clause can appear on a loop directive.

\item Only one \code{collapse} clause can appear on a loop directive.

\item \plc{chunk_size} must be a loop invariant integer expression with a positive value.

\item The value of the \plc{chunk_size} expression must be the same for all threads in the team.

\item The value of the \plc{run-sched-var} ICV must be the same for all threads in the team.

\item When \code{schedule(runtime)} or \code{schedule(auto)} is specified, \plc{chunk_size} must
not be specified.

\item A \plc{modifier} may not be specified on a \code{linear} clause.

\item Only one \code{ordered} clause can appear on a loop directive.

\item The \code{ordered} clause must be present on the loop construct if any \code{ordered} region
ever binds to a loop region arising from the loop construct.

\item The \code{nonmonotonic} modifier cannot be specified if an \code{ordered} clause is specified.

\item Either the \code{monotonic} modifier or the \code{nonmonotonic} modifier can be specified but not both.

\item The loop iteration variable may not appear in a \code{threadprivate} directive.

\item If both the \code{collapse} and \code{ordered} clause with a parameter are specified,
the parameter of the \code{ordered} clause must be greater than or equal to the parameter of the
\code{collapse} clause.

\item A \code{linear} clause or an \code{ordered} clause with a parameter can be specified on a loop directive but not both.
\end{itemize}

\begin{ccppspecific}
\begin{itemize}
\item The associated \plc{for-loops} must be structured blocks.

\item Only an iteration of the innermost associated loop may be curtailed by a \code{continue}
statement.

\item No statement can branch to any associated \code{for} statement.

\item Only one \code{nowait} clause can appear on a \code{for} directive.

\item A throw executed inside a loop region must cause execution to resume within the
same iteration of the loop region, and the same thread that threw the exception must
catch it.
\end{itemize}
\end{ccppspecific}

\begin{fortranspecific}
\begin{itemize}
\item The associated \plc{do-loops} must be structured blocks.

\item Only an iteration of the innermost associated loop may be curtailed by a \code{CYCLE}
statement.

\item No statement in the associated loops other than the \code{DO} statements can cause a branch
out of the loops.

\item The \plc{do-loop} iteration variable must be of type integer.

\item The \plc{do-loop} cannot be a \code{DO WHILE} or a \code{DO} loop without loop control.
\end{itemize}
\end{fortranspecific}

\crossreferences
\begin{itemize}
\item \code{private}, \code{firstprivate}, \code{lastprivate}, \code{linear}, and \code{reduction} clauses, see
\specref{subsec:Data-Sharing Attribute Clauses}.

\item \code{OMP_SCHEDULE} environment variable, see
\specref{sec:OMP_SCHEDULE}.

\item \code{ordered} construct, see
\specref{subsec:ordered Construct}.

\item \code{depend} clause, see
\specref{subsec:depend Clause}.

\item \code{ompt_scope_begin} and \code{ompt_scope_end}, see
  \specref{sec:ompt_scope_endpoint_t}.
\item \code{ompt_work_loop}, see \specref{sec:ompt_work_t}.

\item \code{ompt_callback_work_t}, see
\specref{sec:ompt_callback_work_t}.

\end{itemize}








\subsubsection{Determining the Schedule of a Worksharing Loop}
\label{subsubsec:Determining the Schedule of a Worksharing Loop}
\index{worksharing!scheduling}
When execution encounters a loop directive, the \code{schedule} clause (if any) on the
directive, and the \plc{run-sched-var} and \plc{def-sched-var} ICVs are used to determine how loop
iterations are assigned to threads. See
\specref{sec:Internal Control Variables}
for details of how the
values of the ICVs are determined. If the loop directive does not have a \code{schedule}
clause then the current value of the \mbox{\plc{def-sched-var}} ICV determines the schedule. If the
loop directive has a \code{schedule} clause that specifies the \code{runtime} schedule kind then
the current value of the \plc{run-sched-var} ICV determines the schedule. Otherwise, the
value of the \code{schedule} clause determines the schedule. Figure~\ref{fig:schedule loop}
describes how the schedule for a worksharing loop is determined.

% Figure 2-1: The process for editing a .dia diagram is:
%    1. Use dia to edit the .dia file
%    2. Export to a .tex file
%    3. Edit the .tex file and manually add the \code{} and \plc{} markup.

\begin{figure}[h]
\begin{quote} % to indent the diagram
% Graphic for TeX using PGF
% Title: worksharing-schedule-loop.dia
% Creator: Dia v0.97.2
% CreationDate: Wed Mar 12 03:33:08 2014
% For: dm
% \usepackage{tikz}
% The following commands are not supported in PSTricks at present
% We define them conditionally, so when they are implemented,
% this pgf file will use them.
\ifx\du\undefined
  \newlength{\du}
\fi
\setlength{\du}{15\unitlength}
\begin{tikzpicture}
\pgftransformxscale{1.000000}
\pgftransformyscale{-1.000000}
\definecolor{dialinecolor}{rgb}{0.000000, 0.000000, 0.000000}
\pgfsetstrokecolor{dialinecolor}
\definecolor{dialinecolor}{rgb}{1.000000, 1.000000, 1.000000}
\pgfsetfillcolor{dialinecolor}
\definecolor{dialinecolor}{rgb}{1.000000, 1.000000, 1.000000}
\pgfsetfillcolor{dialinecolor}
\fill (18.500000\du,7.000000\du)--(22.000000\du,9.500000\du)--(18.500000\du,12.000000\du)--(15.000000\du,9.500000\du)--cycle;
\pgfsetlinewidth{0.080000\du}
\pgfsetdash{}{0pt}
\pgfsetdash{}{0pt}
\pgfsetmiterjoin
\definecolor{dialinecolor}{rgb}{0.000000, 0.000000, 0.000000}
\pgfsetstrokecolor{dialinecolor}
\draw (18.500000\du,7.000000\du)--(22.000000\du,9.500000\du)--(18.500000\du,12.000000\du)--(15.000000\du,9.500000\du)--cycle;
% setfont left to latex
\definecolor{dialinecolor}{rgb}{0.000000, 0.000000, 0.000000}
\pgfsetstrokecolor{dialinecolor}
\node at (18.500000\du,9.695000\du){};
\pgfsetlinewidth{0.080000\du}
\pgfsetdash{}{0pt}
\pgfsetdash{}{0pt}
\pgfsetbuttcap
{
\definecolor{dialinecolor}{rgb}{0.000000, 0.000000, 0.000000}
\pgfsetfillcolor{dialinecolor}
% was here!!!
\pgfsetarrowsend{latex}
\definecolor{dialinecolor}{rgb}{0.000000, 0.000000, 0.000000}
\pgfsetstrokecolor{dialinecolor}
\draw (18.500000\du,12.000000\du)--(18.500000\du,14.000000\du);
}
\pgfsetlinewidth{0.080000\du}
\pgfsetdash{}{0pt}
\pgfsetdash{}{0pt}
\pgfsetbuttcap
{
\definecolor{dialinecolor}{rgb}{0.000000, 0.000000, 0.000000}
\pgfsetfillcolor{dialinecolor}
% was here!!!
\pgfsetarrowsend{latex}
\definecolor{dialinecolor}{rgb}{0.000000, 0.000000, 0.000000}
\pgfsetstrokecolor{dialinecolor}
\draw (22.000000\du,16.500000\du)--(26.000000\du,16.500000\du);
}
% setfont left to latex
\definecolor{dialinecolor}{rgb}{0.000000, 0.000000, 0.000000}
\pgfsetstrokecolor{dialinecolor}
\node[anchor=west] at (20.000000\du,10.000000\du){};
\pgfsetlinewidth{0.080000\du}
\pgfsetdash{}{0pt}
\pgfsetdash{}{0pt}
\pgfsetbuttcap
{
\definecolor{dialinecolor}{rgb}{0.000000, 0.000000, 0.000000}
\pgfsetfillcolor{dialinecolor}
% was here!!!
\pgfsetarrowsend{latex}
\definecolor{dialinecolor}{rgb}{0.000000, 0.000000, 0.000000}
\pgfsetstrokecolor{dialinecolor}
\draw (18.500000\du,5.500000\du)--(18.500000\du,7.000000\du);
}
% setfont left to latex
\definecolor{dialinecolor}{rgb}{0.000000, 0.000000, 0.000000}
\pgfsetstrokecolor{dialinecolor}
\node[anchor=west] at (31.000000\du,12.000000\du){};
\definecolor{dialinecolor}{rgb}{1.000000, 1.000000, 1.000000}
\pgfsetfillcolor{dialinecolor}
\fill (18.500000\du,14.000000\du)--(22.000000\du,16.500000\du)--(18.500000\du,19.000000\du)--(15.000000\du,16.500000\du)--cycle;
\pgfsetlinewidth{0.080000\du}
\pgfsetdash{}{0pt}
\pgfsetdash{}{0pt}
\pgfsetmiterjoin
\definecolor{dialinecolor}{rgb}{0.000000, 0.000000, 0.000000}
\pgfsetstrokecolor{dialinecolor}
\draw (18.500000\du,14.000000\du)--(22.000000\du,16.500000\du)--(18.500000\du,19.000000\du)--(15.000000\du,16.500000\du)--cycle;
% setfont left to latex
\definecolor{dialinecolor}{rgb}{0.000000, 0.000000, 0.000000}
\pgfsetstrokecolor{dialinecolor}
\node at (18.500000\du,16.695000\du){};
\pgfsetlinewidth{0.080000\du}
\pgfsetdash{}{0pt}
\pgfsetdash{}{0pt}
\pgfsetbuttcap
{
\definecolor{dialinecolor}{rgb}{0.000000, 0.000000, 0.000000}
\pgfsetfillcolor{dialinecolor}
% was here!!!
\pgfsetarrowsend{latex}
\definecolor{dialinecolor}{rgb}{0.000000, 0.000000, 0.000000}
\pgfsetstrokecolor{dialinecolor}
\draw (22.000000\du,9.500000\du)--(26.000000\du,9.500000\du);
}
\pgfsetlinewidth{0.080000\du}
\pgfsetdash{}{0pt}
\pgfsetdash{}{0pt}
\pgfsetbuttcap
{
\definecolor{dialinecolor}{rgb}{0.000000, 0.000000, 0.000000}
\pgfsetfillcolor{dialinecolor}
% was here!!!
\definecolor{dialinecolor}{rgb}{0.000000, 0.000000, 0.000000}
\pgfsetstrokecolor{dialinecolor}
\draw (18.500000\du,19.040000\du)--(18.500000\du,21.000000\du);
}
\pgfsetlinewidth{0.080000\du}
\pgfsetdash{}{0pt}
\pgfsetdash{}{0pt}
\pgfsetbuttcap
{
\definecolor{dialinecolor}{rgb}{0.000000, 0.000000, 0.000000}
\pgfsetfillcolor{dialinecolor}
% was here!!!
\pgfsetarrowsend{latex}
\definecolor{dialinecolor}{rgb}{0.000000, 0.000000, 0.000000}
\pgfsetstrokecolor{dialinecolor}
\draw (18.500000\du,21.000000\du)--(26.000000\du,21.000000\du);
}
% setfont left to latex
\definecolor{dialinecolor}{rgb}{0.000000, 0.000000, 0.000000}
\pgfsetstrokecolor{dialinecolor}
\node[anchor=west] at (17.500000\du,5.000000\du){START};
% setfont left to latex
\definecolor{dialinecolor}{rgb}{0.000000, 0.000000, 0.000000}
\pgfsetstrokecolor{dialinecolor}
\node at (18.471967\du,9.021967\du){\code{schedule}
};
% setfont left to latex
\definecolor{dialinecolor}{rgb}{0.000000, 0.000000, 0.000000}
\pgfsetstrokecolor{dialinecolor}
\node at (18.471967\du,9.821967\du){clause present?};
% setfont left to latex
\definecolor{dialinecolor}{rgb}{0.000000, 0.000000, 0.000000}
\pgfsetstrokecolor{dialinecolor}
\node at (18.408579\du,15.687353\du){schedule};
% setfont left to latex
\definecolor{dialinecolor}{rgb}{0.000000, 0.000000, 0.000000}
\pgfsetstrokecolor{dialinecolor}
\node at (18.408579\du,16.487353\du){kind value is};
% setfont left to latex
\definecolor{dialinecolor}{rgb}{0.000000, 0.000000, 0.000000}
\pgfsetstrokecolor{dialinecolor}
\node at (18.408579\du,17.287353\du){\code{runtime}?};
% setfont left to latex
\definecolor{dialinecolor}{rgb}{0.000000, 0.000000, 0.000000}
\pgfsetstrokecolor{dialinecolor}
\node[anchor=west] at (26.000000\du,9.500000\du){Use \plc{def-sched-var} schedule kind};
% setfont left to latex
\definecolor{dialinecolor}{rgb}{0.000000, 0.000000, 0.000000}
\pgfsetstrokecolor{dialinecolor}
\node[anchor=west] at (26.000000\du,16.500000\du){Use schedule kind specified in
};
% setfont left to latex
\definecolor{dialinecolor}{rgb}{0.000000, 0.000000, 0.000000}
\pgfsetstrokecolor{dialinecolor}
\node[anchor=west] at (26.000000\du,17.300000\du){\code{schedule} clause};
% setfont left to latex
\definecolor{dialinecolor}{rgb}{0.000000, 0.000000, 0.000000}
\pgfsetstrokecolor{dialinecolor}
\node[anchor=west] at (26.000000\du,21.000000\du){Use \plc{run-sched-var} schedule kind};
% setfont left to latex
\definecolor{dialinecolor}{rgb}{0.000000, 0.000000, 0.000000}
\pgfsetstrokecolor{dialinecolor}
\node[anchor=west] at (22.000000\du,9.000000\du){No};
% setfont left to latex
\definecolor{dialinecolor}{rgb}{0.000000, 0.000000, 0.000000}
\pgfsetstrokecolor{dialinecolor}
\node[anchor=west] at (19.000000\du,12.500000\du){Yes};
% setfont left to latex
\definecolor{dialinecolor}{rgb}{0.000000, 0.000000, 0.000000}
\pgfsetstrokecolor{dialinecolor}
\node[anchor=west] at (22.000000\du,16.000000\du){No};
% setfont left to latex
\definecolor{dialinecolor}{rgb}{0.000000, 0.000000, 0.000000}
\pgfsetstrokecolor{dialinecolor}
\node[anchor=west] at (19.500000\du,20.500000\du){Yes};
\end{tikzpicture}

\end{quote}
\caption{Determining the \code{schedule} for a Worksharing Loop\label{fig:schedule loop}}
\end{figure}

\crossreferences
\begin{itemize}
\item ICVs, see
\specref{sec:Internal Control Variables}
\end{itemize}











\subsection{\hcode{sections} Construct}
\label{subsec:sections Construct}
\index{sections@{\code{sections}}}
\index{constructs!sections@{\code{sections}}}
\summary
The \code{sections} construct is a non-iterative worksharing construct that contains a set of
structured blocks that are to be distributed among and executed by the threads in a team.
Each structured block is executed once by one of the threads in the team in the context
of its implicit task.

\syntax
\begin{ccppspecific}
The syntax of the \code{sections} construct is as follows:

\begin{ompcPragma}
#pragma omp sections \plc{[clause[ [},\plc{] clause] ... ] new-line}
   {
   \plc{[}#pragma omp section \plc{new-line}\plc{]}
      \plc{structured-block}
   \plc{[}#pragma omp section \plc{new-line}
      \plc{structured-block]}
   \plc{...}
   }
\end{ompcPragma}

where \plc{clause} is one of the following:

\begin{indentedcodelist}
private(\plc{list})
firstprivate(\plc{list})
lastprivate(\plc{[ lastprivate-modifier}:\plc{] list})
reduction(\plc{reduction-identifier }:\plc{ list})
nowait
allocate(\plc{[allocator: ]}\plc{list})
\end{indentedcodelist}
\end{ccppspecific}

\needspace{16\baselineskip}
\begin{fortranspecific}
The syntax of the \code{sections} construct is as follows:

\begin{ompfPragma}
!$omp sections \plc{[clause[ [},\plc{] clause] ... ]}
   \plc{[}!$omp section\plc{]}
      \plc{structured-block}
   \plc{[}!$omp section
      \plc{structured-block]}
   \plc{...}
!$omp end sections \plc{[}nowait\plc{]}
\end{ompfPragma}

\begin{samepage}
where \plc{clause} is one of the following:

\begin{indentedcodelist}
private(\plc{list})
firstprivate(\plc{list})
lastprivate(\plc{[ lastprivate-modifier}:\plc{] list})
reduction(\plc{reduction-identifier }:\plc{ list})
allocate(\plc{[allocator: ]}\plc{list})
\end{indentedcodelist}
\end{samepage}
\end{fortranspecific}

\binding
The binding thread set for a \code{sections} region is the current team. A \code{sections}
region binds to the innermost enclosing \code{parallel} region. Only the threads of the team
executing the binding \code{parallel} region participate in the execution of the structured
blocks and the implied barrier of the \code{sections} region if the barrier is not eliminated
by a \code{nowait} clause.

\descr
Each structured block in the \code{sections} construct is preceded by a \code{section} directive
except possibly the first block, for which a preceding \code{section} directive is optional.

The method of scheduling the structured blocks among the threads in the team is
implementation defined.

There is an implicit barrier at the end of a \code{sections} construct unless a \code{nowait}
clause is specified.

%\tools
\omptWorksharing{sections}{ompt_work_sections}

\restrictions
Restrictions to the \code{sections} construct are as follows:

\begin{itemize}
\item Orphaned \code{section} directives are prohibited. That is, the \code{section} directives must
appear within the \code{sections} construct and must not be encountered elsewhere in the
\code{sections} region.

\item The code enclosed in a \code{sections} construct must be a structured block.

\item Only a single \code{nowait} clause can appear on a \code{sections} directive.

\begin{cppspecific}
\item A throw executed inside a \code{sections} region must cause execution to resume within
the same section of the \code{sections} region, and the same thread that threw the
exception must catch it.
\end{cppspecific}
\end{itemize}

\crossreferences
\begin{itemize}
\item \code{private}, \code{firstprivate}, \code{lastprivate}, and \code{reduction} clauses, see
\specref{subsec:Data-Sharing Attribute Clauses}.

\item \code{ompt_scope_begin} and \code{ompt_scope_end}, see
  \specref{sec:ompt_scope_endpoint_t}.

\item \code{ompt_work_sections}, see \specref{sec:ompt_work_t}.

\item \code{ompt_callback_work_t}, see
\specref{sec:ompt_callback_work_t}.
\end{itemize}










\subsection{\hcode{single} Construct}
\index{single@{\code{single}}}
\index{constructs!single@{\code{single}}}
\label{subsec:single Construct}
\summary
The \code{single} construct specifies that the associated structured block is executed by only
one of the threads in the team (not necessarily the master thread), in the context of its
implicit task. The other threads in the team, which do not execute the block, wait at an
implicit barrier at the end of the \code{single} construct unless a \code{nowait} clause is specified.

\syntax
\begin{ccppspecific}
The syntax of the single construct is as follows:

\begin{ompcPragma}
#pragma omp single \plc{[clause[ [},\plc{] clause] ... ] new-line}
   \plc{structured-block}
\end{ompcPragma}

\begin{samepage}
where \plc{clause} is one of the following:

\begin{indentedcodelist}
private(\plc{list})
firstprivate(\plc{list})
copyprivate(\plc{list})
nowait
allocate(\plc{[allocator: ]}\plc{list})
\end{indentedcodelist}
\end{samepage}
\end{ccppspecific}

\begin{fortranspecific}
The syntax of the \code{single} construct is as follows:

\begin{ompfPragma}
!$omp single \plc{[clause[ [},\plc{] clause] ... ]}
   \plc{structured-block}
!$omp end single \plc{[end_clause[ [},\plc{] end_clause] ... ]}
\end{ompfPragma}

where \plc{clause} is one of the following:

\begin{indentedcodelist}
private(\plc{list})
firstprivate(\plc{list})
allocate(\plc{[allocator: ]}\plc{list})
\end{indentedcodelist}

and \plc{end_clause} is one of the following:

\begin{indentedcodelist}
copyprivate(\plc{list})
nowait
\end{indentedcodelist}
\end{fortranspecific}

\binding
The binding thread set for a \code{single} region is the current team. A \code{single} region
binds to the innermost enclosing \code{parallel} region. Only the threads of the team
executing the binding \code{parallel} region participate in the execution of the structured
block and the implied barrier of the \code{single} region if the barrier is not eliminated by a
\code{nowait} clause.

\descr
Only one of the encountering threads will execute the structured block associated with the \code{single}
construct. The method of choosing a thread to execute the structured block each time the team encounters the construct
is implementation defined. There is an implicit barrier at the end of the \code{single} construct unless a
\code{nowait} clause is specified.

\events

The \plc{single-begin} event occurs after an \code{implicit task} encounters a
\code{single} construct but before the task starts the execution of the structured
block of the \code{single} region.

The \plc{single-end} event occurs after a \code{single} region finishes execution of the structured block
but before resuming execution of the encountering implicit task.


\tools

A thread dispatches a registered \code{ompt_callback_work}
callback for each occurrence of \plc{single-begin} and
\plc{single-end} events in that thread. The callback has type signature
\code{ompt_callback_work_t}. The callback receives
\code{ompt_scope_begin} or \code{ompt_scope_end}
as its \plc{endpoint} argument, as appropriate, and
\code{ompt_work_single_executor} or \code{ompt_work_single_other}
as its \plc{wstype} argument.

\restrictions
Restrictions to the \code{single} construct are as follows:

\begin{itemize}
\item The \code{copyprivate} clause must not be used with the \code{nowait} clause.

\item At most one \code{nowait} clause can appear on a \code{single} construct.

\begin{cppspecific}
\item A throw executed inside a \code{single} region must cause execution to resume within the
same \code{single} region, and the same thread that threw the exception must catch it.
\end{cppspecific}
\end{itemize}


\crossreferences
\begin{itemize}
\item \code{private} and \code{firstprivate} clauses, see
\specref{subsec:Data-Sharing Attribute Clauses}.

\item \code{copyprivate} clause, see
\specref{subsubsec:copyprivate clause}.

\item \code{ompt_scope_begin} and \code{ompt_scope_end}, see
  \specref{sec:ompt_scope_endpoint_t}.

\item \code{ompt_work_single_executor} and \code{ompt_work_single_other}, see
\specref{sec:ompt_work_t}.

\item \code{ompt_callback_work_t},
\specref{sec:ompt_callback_work_t}.

\end{itemize}












%\newpage %% HACK

% Here we need to force the blue marker lower, and force the subsection header higher
% in order to reduce the space between the marker and the header, per Richard:
%\begin{samepage}
\vspace{3\baselineskip}
\begin{fortranspecific}
\vspace{-1\baselineskip}
\subsection{\hcode{workshare} Construct}
\index{workshare@{\code{workshare}}}
\index{constructs!workshare@{\code{workshare}}}
\label{subsec:workshare Construct}
\summary
The \code{workshare} construct divides the execution of the enclosed structured block into
separate units of work, and causes the threads of the team to share the work such that
each unit is executed only once by one thread, in the context of its implicit task.
%\end{samepage}

%\begin{samepage}
\syntax
The syntax of the \code{workshare} construct is as follows:

\begin{ompfPragma}
!$omp workshare
    \plc{structured-block}
!$omp end workshare \plc{[}nowait\plc{]}
\end{ompfPragma}
%\end{samepage}

The enclosed structured block must consist of only the following:

\begin{itemize}
\item array assignments

\item scalar assignments

\item \code{FORALL} statements

\item \code{FORALL} constructs

\item \code{WHERE} statements

\item \code{WHERE} constructs

\item \code{atomic} constructs

\item \code{critical} constructs

\item \code{parallel} constructs
\end{itemize}

Statements contained in any enclosed \code{critical} construct are also subject to these
restrictions. Statements in any enclosed \code{parallel} construct are not restricted.

\binding
The binding thread set for a \code{workshare} region is the current team. A \code{workshare}
region binds to the innermost enclosing \code{parallel} region. Only the threads of the team
executing the binding \code{parallel} region participate in the execution of the units of
work and the implied barrier of the \code{workshare} region if the barrier is not eliminated
by a \code{nowait} clause.

\descr
There is an implicit barrier at the end of a \code{workshare} construct unless a \code{nowait}
clause is specified.

An implementation of the \code{workshare} construct must insert any synchronization that is
required to maintain standard Fortran semantics. For example, the effects of one
statement within the structured block must appear to occur before the execution of
succeeding statements, and the evaluation of the right hand side of an assignment must
appear to complete prior to the effects of assigning to the left hand side.

The statements in the \code{workshare} construct are divided into units of work as follows:

\begin{itemize}
\item For array expressions within each statement, including transformational array
intrinsic functions that compute scalar values from arrays:

\begin{itemize} % nested level
\item Evaluation of each element of the array expression, including any references to
\code{ELEMENTAL} functions, is a unit of work.

\item Evaluation of transformational array intrinsic functions may be freely subdivided
into any number of units of work.
\end{itemize}

\item For an array assignment statement, the assignment of each element is a unit of work.

\item For a scalar assignment statement, the assignment operation is a unit of work.

\item For a \code{WHERE} statement or construct, the evaluation of the mask expression and the
masked assignments are each a unit of work.

\item For a \code{FORALL} statement or construct, the evaluation of the mask expression,
expressions occurring in the specification of the iteration space, and the masked
assignments are each a unit of work

\item For an \code{atomic} construct, the atomic operation on the storage location designated as
\plc{x} is a unit of work.

\item For a \code{critical} construct, the construct is a single unit of work.

\item For a \code{parallel} construct, the construct is a unit of work with respect to the
\code{workshare} construct. The statements contained in the \code{parallel} construct are
executed by a new thread team.

\item If none of the rules above apply to a portion of a statement in the structured block,
then that portion is a unit of work.
\end{itemize}

The transformational array intrinsic functions are \code{MATMUL}, \code{DOT_PRODUCT}, \code{SUM},
\code{PRODUCT}, \code{MAXVAL}, \code{MINVAL}, \code{COUNT},
\code{ANY}, \code{ALL}, \code{SPREAD}, \code{PACK}, \code{UNPACK},
\code{RESHAPE}, \code{TRANSPOSE}, \code{EOSHIFT}, \code{CSHIFT}, \code{MINLOC}, and \code{MAXLOC}.

It is unspecified how the units of work are assigned to the threads executing a
\code{workshare} region.

If an array expression in the block references the value, association status, or allocation
status of private variables, the value of the expression is undefined, unless the same
value would be computed by every thread.

If an array assignment, a scalar assignment, a masked array assignment, or a \code{FORALL}
assignment assigns to a private variable in the block, the result is unspecified.

The \code{workshare} directive causes the sharing of work to occur only in the \code{workshare}
construct, and not in the remainder of the \code{workshare} region.

%\tools
\omptWorksharing{workshare}{ompt_work_workshare}

\begin{samepage}
\restrictions
The following restrictions apply to the \code{workshare} construct:

\begin{itemize}
\item All array assignments, scalar assignments, and masked array assignments must be
intrinsic assignments.

\item The construct must not contain any user defined function calls unless the function is
\code{ELEMENTAL}.
\end{itemize}
\end{samepage}

\crossreferences
\begin{itemize}
\item \code{ompt_scope_begin} and \code{ompt_scope_end}, see
  \specref{sec:ompt_scope_endpoint_t}.
\item \code{ompt_work_workshare}, see \specref{sec:ompt_work_t}.
\item \code{ompt_callback_work_t}, see
\specref{sec:ompt_callback_work_t}.
\end{itemize}

%\filbreak
\end{fortranspecific}

% This is an included file. See the master file for more information.
%
% When editing this file:
%
%    1. To change formatting, appearance, or style, please edit openmp.sty.
%
%    2. Custom commands and macros are defined in openmp.sty.
%
%    3. Be kind to other editors -- keep a consistent style by copying-and-pasting to
%       create new content.
%
%    4. We use semantic markup, e.g. (see openmp.sty for a full list):
%         \code{}     % for bold monospace keywords, code, operators, etc.
%         \plc{}      % for italic placeholder names, grammar, etc.
%
%    5. There are environments that provide special formatting, e.g. language bars.
%       Please use them whereever appropriate.  Examples are:
%
%         \begin{fortranspecific}
%         This is text that appears enclosed in blue language bars for Fortran.
%         \end{fortranspecific}
%
%         \begin{note}
%         This is a note.  The "Note -- " header appears automatically.
%         \end{note}
%
%    6. Other recommendations:
%         Use the convenience macros defined in openmp.sty for the minor headers
%         such as Comments, Syntax, etc.
%
%         To keep items together on the same page, prefer the use of
%         \begin{samepage}.... Avoid \parbox for text blocks as it interrupts line numbering.
%         When possible, avoid \filbreak, \pagebreak, \newpage, \clearpage unless that's
%         what you mean. Use \needspace{} cautiously for troublesome paragraphs.
%
%         Avoid absolute lengths and measures in this file; use relative units when possible.
%         Vertical space can be relative to \baselineskip or ex units. Horizontal space
%         can be relative to \linewidth or em units.
%
%         Prefer \emph{} to italicize terminology, e.g.:
%             This is a \emph{definition}, not a placeholder.
%             This is a \plc{var-name}.
%

\section{\hcode{declare}~\hcode{simd} Directive}
\index{declare simd@{\code{declare}~\code{simd}}}
\index{directives!declare simd@{\code{declare}~\code{simd}}}
\label{sec:declare simd Directive}
\summary
The \code{declare}~\code{simd} directive can be applied to a function (C, C++ and Fortran) or a
subroutine (Fortran) to enable the creation of one or more versions that can process
multiple arguments using SIMD instructions from a single invocation in a SIMD
loop. The \code{declare}~\code{simd} directive is a declarative directive. There may be multiple
\code{declare}~\code{simd} directives for a function (C, C++, Fortran) or subroutine (Fortran).

\syntax
The syntax of the \code{declare}~\code{simd} directive is as follows:

\begin{ccppspecific}
\begin{ompcPragma}
#pragma omp declare simd \plc{[clause[ [},\plc{] clause] ... ] new-line}
\plc{[}#pragma omp declare simd \plc{[clause[ [},\plc{] clause] ... ] new-line]}
\plc{[ ... ]}
   \plc{function definition or declaration}
\end{ompcPragma}

where \plc{clause} is one of the following:

\begin{indentedcodelist}
simdlen(\plc{length})
linear(\plc{linear-list[ }:\plc{ linear-step]})
aligned(\plc{argument-list[ }:\plc{ alignment]})
uniform(\plc{argument-list})
inbranch
notinbranch
\end{indentedcodelist}
\end{ccppspecific}


\begin{fortranspecific}
\begin{ompfPragma}
!$omp declare simd \plc{[}(\plc{proc-name})\plc{] [clause[ [},\plc{] clause] ... ]}
\end{ompfPragma}

where \plc{clause} is one of the following:
\begin{indentedcodelist}
simdlen(\plc{length})
linear(\plc{linear-list[ }:\plc{ linear-step]})
aligned(\plc{argument-list[ }:\plc{ alignment]})
uniform(\plc{argument-list})
inbranch
notinbranch
\end{indentedcodelist}
\end{fortranspecific}


\descr
\begin{ccppspecific}
The use of one or more \code{declare}~\code{simd} directives immediately prior
to a function declaration or definition enables the
creation of corresponding SIMD
versions of the associated function that can be used to process multiple arguments from
a single invocation in a SIMD loop concurrently.

The expressions appearing in the clauses of each directive are evaluated in the scope of
the arguments of the function declaration or definition.
\end{ccppspecific}

\begin{samepage}
\begin{fortranspecific}
The use of one or more \code{declare}~\code{simd} directives for a specified
subroutine or function enables the creation of corresponding SIMD versions of the
subroutine or function that can be used to process multiple arguments from a
single invocation in a SIMD loop concurrently.
\end{fortranspecific}
\end{samepage}

If a SIMD version is created, the number of concurrent arguments for the function is
determined by the \code{simdlen} clause. If the \code{simdlen} clause is used its value
corresponds to the number of concurrent arguments of the function. The parameter of
the \code{simdlen} clause must be a constant positive integer expression. Otherwise, the
number of concurrent arguments for the function is implementation defined.

\begin{cppspecific}
The special \plc{this} pointer can be used as if was one of the arguments to the function in any of the \code{linear}, \code{aligned}, or \code{uniform} clauses.
\end{cppspecific}

The \code{uniform} clause declares one or more arguments to have an invariant value for all
concurrent invocations of the function in the execution of a single SIMD loop.

\begin{samepage}
\begin{ccppspecific}
The \code{aligned} clause declares that the object to which each list item points is aligned to
the number of bytes expressed in the optional parameter of the \code{aligned} clause.
\end{ccppspecific}
\end{samepage}

\needspace{15\baselineskip}\begin{samepage}
\begin{fortranspecific}
The \code{aligned} clause declares that the target of each list item is aligned to the number
of bytes expressed in the optional parameter of the \code{aligned} clause.
\end{fortranspecific}
\end{samepage}

The optional parameter of the \code{aligned} clause, \plc{alignment}, must be a constant positive
integer expression. If no optional parameter is specified, implementation-defined default
alignments for SIMD instructions on the target platforms are assumed.

The \code{inbranch} clause specifies that the SIMD version of the function will always be called from inside a
conditional statement of a SIMD loop. The \code{notinbranch} clause specifies that the
SIMD version of the function will never be called from inside a conditional statement of a SIMD loop. If
neither clause is specified, then the SIMD version of the function may or may not be called from inside a
conditional statement of a SIMD loop.

\restrictions
\begin{itemize}
\item Each argument can appear in at most one \code{uniform} or \code{linear} clause.

\item At most one \code{simdlen} clause can appear in a \code{declare}~\code{simd} directive.

\item Either \code{inbranch} or \code{notinbranch} may be specified, but not both.

\item When a \plc{linear-step} expression is specified in a \code{linear} clause it must be
either a constant integer expression or an integer-typed parameter that is specified in
a \code{uniform} clause on the directive.

\item The function or subroutine body must be a structured block.

\item The execution of the function or subroutine, when called from a SIMD loop, cannot result in the execution of an OpenMP construct except for an \code{ordered} construct with the \code{simd} clause or an \code{atomic} construct.

\item The execution of the function or subroutine cannot have any side effects that would
alter its execution for concurrent iterations of a SIMD chunk.

\item A program that branches into or out of the function is non-conforming.

\begin{ccppspecific}
\item If the function has any declarations, then the \code{declare}~\code{simd} construct for any
declaration that has one must be equivalent to the one specified for the definition.
Otherwise, the result is unspecified.

\item The function cannot contain calls to the \code{longjmp} or \code{setjmp} functions.
\end{ccppspecific}

\begin{cspecific}
\item The type of list items appearing in the \code{aligned} clause must be array or pointer.
\end{cspecific}

\begin{cppspecific}
\item The function cannot contain any calls to \code{throw}.

\item The type of list items appearing in the \code{aligned} clause must be array, pointer,
reference to array, or reference to pointer.
\end{cppspecific}

\begin{fortranspecific}
\item \plc{proc-name} must not be a generic name, procedure pointer or entry name.

\item If \plc{proc-name} is omitted, the \code{declare}~\code{simd}
  directive must appear in the specification part of a subroutine
  subprogram or a function subprogram for which creation of the SIMD
  versions is enabled.

\item Any \code{declare}~\code{simd} directive must appear in the specification part of a subroutine
subprogram, function subprogram or interface body to which it applies.

\item If a \code{declare}~\code{simd} directive is specified in an interface block for a procedure, it
must match a \code{declare}~\code{simd} directive in the definition of the procedure.

\item If a procedure is declared via a procedure declaration statement, the procedure
\plc{proc-name} should appear in the same specification.

\item If a \code{declare}~\code{simd} directive is specified for a procedure name with explicit
interface and a \code{declare}~\code{simd} directive is also specified for the definition of the
procedure then the two \code{declare}~\code{simd} directives must match. Otherwise the result
is unspecified.

\item Procedure pointers may not be used to access versions created by the \code{declare}~\code{simd} directive.

\item The type of list items appearing in the \code{aligned} clause must be \code{C_PTR} or Cray
pointer, or the list item must have the \code{POINTER} or \code{ALLOCATABLE} attribute.
\end{fortranspecific}
\end{itemize}

\crossreferences
\begin{itemize}
\item \code{reduction} clause, see
\specref{subsubsec:reduction clause}.

\item \code{linear} clause, see
\specref{subsubsec:linear clause}.
\end{itemize}

% This is an included file. See the master file for more information.
%
% When editing this file:
%
%    1. To change formatting, appearance, or style, please edit openmp.sty.
%
%    2. Custom commands and macros are defined in openmp.sty.
%
%    3. Be kind to other editors -- keep a consistent style by copying-and-pasting to
%       create new content.
%
%    4. We use semantic markup, e.g. (see openmp.sty for a full list):
%         \code{}     % for bold monospace keywords, code, operators, etc.
%         \plc{}      % for italic placeholder names, grammar, etc.
%
%    5. There are environments that provide special formatting, e.g. language bars.
%       Please use them whereever appropriate.  Examples are:
%
%         \begin{fortranspecific}
%         This is text that appears enclosed in blue language bars for Fortran.
%         \end{fortranspecific}
%
%         \begin{note}
%         This is a note.  The "Note -- " header appears automatically.
%         \end{note}
%
%    6. Other recommendations:
%         Use the convenience macros defined in openmp.sty for the minor headers
%         such as Comments, Syntax, etc.
%
%         To keep items together on the same page, prefer the use of 
%         \begin{samepage}.... Avoid \parbox for text blocks as it interrupts line numbering.
%         When possible, avoid \filbreak, \pagebreak, \newpage, \clearpage unless that's
%         what you mean. Use \needspace{} cautiously for troublesome paragraphs.
%
%         Avoid absolute lengths and measures in this file; use relative units when possible.
%         Vertical space can be relative to \baselineskip or ex units. Horizontal space
%         can be relative to \linewidth or em units.
%
%         Prefer \emph{} to italicize terminology, e.g.:
%             This is a \emph{definition}, not a placeholder.
%             This is a \plc{var-name}.
%


\section{Tasking Constructs}
\label{sec:Tasking Constructs}
\index{tasking constructs}
\index{constructs!tasking constructs}
\subsection{\code{task} Construct}
\index{task@{\code{task}}}
\index{constructs!task@{\code{task}}}
\label{subsec:task Construct}
\summary
The \code{task} construct defines an explicit task.

\syntax
\begin{ccppspecific}
\begin{samepage}
The syntax of the \code{task} construct is as follows: 

\begin{boxedcode}
\#pragma omp task \plc{[clause[ [},\plc{] clause] ... ] new-line}
    \plc{structured-block}
\end{boxedcode}
\end{samepage}

\begin{samepage}
where \plc{clause} is one of the following: 

\begin{indentedcodelist}
if(\plc{[} task :\plc{] scalar-expression})
final(\plc{scalar-expression})
untied
default(shared \textnormal{|} none)
mergeable
private(\plc{list})
firstprivate(\plc{list})
shared(\plc{list})
in_reduction(\plc{reduction-identifier }:\plc{ list})
depend(\plc{dependence-type }:\plc{ locator-list}[:\plc{iterators-definition}])
priority(\plc{priority-value})
\end{indentedcodelist}
\end{samepage}
\end{ccppspecific}

\begin{fortranspecific}
The syntax of the \code{task} construct is as follows: 

\begin{boxedcode}
!\$omp task \plc{[clause[ [},\plc{] clause] ... ]}
    \plc{structured-block}
!\$omp end task
\end{boxedcode}

where \plc{clause} is one of the following:

\begin{indentedcodelist}
if(\plc{[} task :\plc{] scalar-logical-expression})
final(\plc{scalar-logical-expression})
untied
default(private \textnormal{|} firstprivate \textnormal{|} shared \textnormal{|} none)
mergeable
private(\plc{list})
firstprivate(\plc{list})
shared(\plc{list})
in_reduction(\plc{reduction-identifier }:\plc{ list})
depend(\plc{dependence-type }:\plc{ locator-list}[:\plc{iterators-definition}])
priority(\plc{priority-value})
\end{indentedcodelist}
\end{fortranspecific}

\binding
The binding thread set of the \code{task} region is the current team. A \code{task} region binds to 
the innermost enclosing \code{parallel} region. 

\descr

The \code{task} construct is a \emph{task generating construct}. When a thread
encounters a \code{task} construct, an explicit task is generated from the code
for the associated structured block. The data environment of the task is
created according to the data-sharing attribute clauses on the \code{task}
construct, per-data environment ICVs, and any defaults that apply.

The encountering thread may immediately execute the task, or defer its execution. In the 
latter case, any thread in the team may be assigned the task. Completion of the task can 
be guaranteed using task synchronization constructs. 
If a \code{task} construct is encountered during execution of an outer
task, the generated \code{task} region associated with this construct is not a
part of the outer task region unless the generated task is
an included task.

When an \code{if} clause is present on a \code{task} construct, and the \code{if} clause expression 
evaluates to \plc{false}, an undeferred task is generated, and the encountering thread must 
suspend the current task region, for which execution cannot be resumed until the 
generated task is completed. The use of a variable in an \code{if} clause expression 
of a \code{task} construct causes an implicit reference to the variable in all enclosing 
constructs.

When a \code{final} clause is present on a \code{task} construct and the \code{final} clause expression 
evaluates to \plc{true}, the generated task will be a final task. All \code{task} constructs 
encountered during execution of a final task will generate final and included tasks. Note 
that the use of a variable in a \code{final} clause expression of a \code{task} construct causes an 
implicit reference to the variable in all enclosing constructs.

The \code{if} clause expression and the \code{final} clause expression are evaluated in the context 
outside of the \code{task} construct, and no ordering of those evaluations is specified.

A thread that encounters a task scheduling point within the \code{task} region may 
temporarily suspend the \code{task} region. By default, a task is tied and its suspended \code{task} 
region can only be resumed by the thread that started its execution. If the \code{untied} 
clause is present on a \code{task} construct, any thread in the team can resume the \code{task} 
region after a suspension. The \code{untied} clause is ignored if a \code{final} clause is present 
on the same \code{task} construct and the \code{final} clause expression evaluates to \plc{true}, or if a 
task is an included task.

The \code{task} construct includes a task scheduling point in the task region of its generating 
task, immediately following the generation of the explicit task. Each explicit \code{task} 
region includes a task scheduling point at its point of completion. 

When the \code{mergeable} clause is present on a \code{task} construct, the generated task is a \plc{mergeable task}. 

The \code{priority} clause is a hint for the priority of the generated task. The \plc{priority-value} is a
non-negative integer expression that provides a hint for task execution order. Among all
tasks ready to be executed, higher priority tasks (those with a higher numerical value in the
\code{priority} clause expression) are recommended to execute before lower priority ones. The default
\plc{priority-value} when no \code{priority} clause is specified is zero (the lowest priority). If a value is
specified in the \code{priority} clause that is higher than the \plc{max-task-priority-var} ICV then the
implementation will use the value of that ICV. A program that relies on task execution order
being determined by this \plc{priority-value} may have unspecified behavior.

\begin{note}
When storage is shared by an explicit \code{task} region, the 
programmer must ensure, by adding proper synchronization, that the storage does not 
reach the end of its lifetime before the explicit \code{task} region completes its execution.
\end{note}

\events

The \plc{task-create} event occurs when a thread encounters a construct 
that causes a new task to be created. 
The event occurs after the task is initialized but before 
it begins execution or is deferred. 

\tools

A thread dispatches a registered \code{ompt\_callback\_task\_create}
callback for each occurrence of a \plc{task-create} event
in the context of the encountering task.  
This callback has the type signature \code{ompt\_callback\_task\_create\_t}.
In the dispatched callback, \code{(task\_type \& ompt\_task\_explicit)} always
evaluates to \plc{true}.
If the task is an undeferred task, then \code{(task\_type \& ompt\_task\_undeferred)} evaluates to \plc{true}. 
If the task is a final task, \code{(task\_type \& ompt\_task\_final)} evaluates to \plc{true}.
If the task is an untied task, \code{(task\_type \& ompt\_task\_untied)} evaluates to \plc{true}. 
If the task is a mergeable task, \code{(task\_type \& ompt\_task\_mergeable)} evaluates to \plc{true}. 
If the task is a merged task, \code{(task\_type \& ompt\_task\_merged)} evaluates to \plc{true}. 

\restrictions
Restrictions to the \code{task} construct are as follows:

\begin{itemize}
\item A program that branches into or out of a \code{task} region is non-conforming. 

\item A program must not depend on any ordering of the evaluations of the clauses of the 
\code{task} directive, or on any side effects of the evaluations of the clauses. 

\item At most one \code{if} clause can appear on the directive. 

\item At most one \code{final} clause can appear on the directive.

\item At most one \code{priority} clause can appear on the directive.

\begin{ccppspecific}
\item A throw executed inside a \code{task} region must cause execution to resume within the 
same \code{task} region, and the same thread that threw the exception must catch it.
\end{ccppspecific}

\begin{fortranspecific}
\item Unsynchronized use of Fortran I/O statements by multiple tasks on the same unit has 
unspecified behavior
\end{fortranspecific}
\end{itemize}

\crossreferences
\begin{itemize}
\item Task scheduling constraints, see \specref{subsec:Task Scheduling}. 
\item \code{depend} clause, see \specref{subsec:depend Clause}.
\item \code{if} Clause, see \specref{sec:if Clause}.
\item Data-sharing attribute clauses, \specref{subsec:Data-Sharing Attribute Clauses}.
\item \code{ompt\_callback\_task\_create\_t}, see
\specref{sec:ompt_callback_task_create_t}.
%\item \code{ompt\_callback\_task\_schedule\_t}, see
%\specref{sec:ompt_callback_task_schedule_t}.
\end{itemize}











% TASKLOOP 
\subsection{\code{taskloop} Construct}
\index{taskloop@{\code{taskloop}}}
\index{constructs!taskloop@{\code{taskloop}}}
\label{subsec:taskloop Construct}
\summary
The \code{taskloop} construct specifies that the iterations of one or more associated loops will be executed in parallel using explicit tasks. The iterations are distributed across tasks generated by the construct and scheduled to be executed.
\syntax
\begin{ccppspecific}
The syntax of the \code{taskloop} construct is as follows:
\begin{boxedcode}
\#pragma omp taskloop \plc{[clause[[,] clause] ...] new-line}
    \plc{for-loops}
\end{boxedcode}
where \plc{clause} is one of the following:
\begin{indentedcodelist}
if(\plc{[} taskloop :\plc{] scalar-expr})
shared(\plc{list})
private(\plc{list})
firstprivate(\plc{list})
lastprivate(\plc{list})
reduction(\plc{reduction-identifier }:\plc{ list})
in_reduction(\plc{reduction-identifier }:\plc{ list})
default(shared \textnormal{|} none)
grainsize(\plc{grain-size})
num_tasks(\plc{num-tasks})
collapse(\plc{n})
final(\plc{scalar-expr})
priority(\plc{priority-value})
untied
mergeable
nogroup
\end{indentedcodelist}

The \code{taskloop} directive places restrictions on the structure of all associated \plc{for-loops}. Specifically, all associated \plc{for-loops} must have canonical loop form (see \specref{sec:Canonical Loop Form}).
\end{ccppspecific}
\begin{fortranspecific}
The syntax of the \code{taskloop} construct is as follows:
\begin{boxedcode}
!\$omp taskloop \plc{[clause[[,] clause] ...]}
    \plc{do-loops}
\plc{[}!\$omp end taskloop\plc{]}
\end{boxedcode}
where \plc{clause} is one of the following:
\begin{indentedcodelist}
if(\plc{[} taskloop :\plc{] scalar-logical-expr})
shared(\plc{list})
private(\plc{list})
firstprivate(\plc{list})
lastprivate(\plc{list})
reduction(\plc{reduction-identifier }:\plc{ list})
in_reduction(\plc{reduction-identifier }:\plc{ list})
default(private \textnormal{|} firstprivate \textnormal{|} shared \textnormal{|} none)
grainsize(\plc{grain-size})
num_tasks(\plc{num-tasks})
collapse(\plc{n})
final(\plc{scalar-logical-expr})
priority(\plc{priority-value})
untied
mergeable
nogroup
\end{indentedcodelist}

If an \code{end}~\code{taskloop} directive is not specified, an 
\code{end}~\code{taskloop} directive is assumed at the end of the 
\plc{do-loops}.

Any associated \plc{do-loop} must be \plc{do-construct} or an 
\plc{inner-shared-do-construct} as defined by the Fortran standard. 
If an \code{end}~\code{taskloop} directive follows a \plc{do-construct} 
in which several loop statements share a \code{DO} 
termination statement, then the directive can only be specified for the 
outermost of these \code{DO} statements.

If any of the loop iteration variables would otherwise be shared, they are implicitly made private for the loop-iteration tasks generated by the \code{taskloop} construct. Unless the loop iteration variables are specified in a \code{lastprivate} clause on the \code{taskloop} construct, their values after the loop are unspecified.
\end{fortranspecific}

\binding
The binding thread set of the \code{taskloop} region is the current team. A \code{taskloop} region binds to the innermost enclosing \code{parallel} region.

\descr
The \code{taskloop} construct is a \emph{task generating construct}. When a thread encounters a \code{taskloop} construct, the construct partitions the associated loops into explicit tasks for parallel execution of the loops' iterations. The data environment of each generated task is created according to the data-sharing attribute clauses on the \code{taskloop} construct, per-data environment ICVs, and any defaults that apply. The order of the creation of the loop tasks is unspecified.
Programs that rely on any execution order of the logical loop iterations are non-conforming. 

By default, the \code{taskloop} construct executes as if it was enclosed in a \code{taskgroup} construct with no statements or directives outside of the \code{taskloop} construct. Thus, the \code{taskloop} construct creates an implicit \code{taskgroup} region. If the \code{nogroup} clause is present, no implicit \code{taskgroup} region is created.

If a \code{reduction} clause is present on the \code{taskloop} construct, the behavior is as if a \code{task\_reduction} clause with the same reduction operator and list items was applied to the implicit \code{taskgroup} construct enclosing the \code{taskloop} construct.
Furthermore, the \code{taskloop} construct executes as if each generated task was defined by a \code{task} construct on which an \code{in\_reduction} clause with the same reduction operator and list items is present. Thus, the generated tasks are participants of the reduction defined by the \code{task\_reduction} clause that was applied to the implicit \code{taskgroup} construct.

If an \code{in\_reduction} clause is present on the \code{taskloop} construct, the behavior is as if each generated task was defined by a \code{task} construct on which an \code{in\_reduction} clause with the same reduction operator and list items is present. Thus, the generated tasks are participants of a reduction previously defined by a reduction scoping clause.

If a \code{grainsize} clause is present on the \code{taskloop} construct, the number of logical loop iterations assigned to each generated task is greater than or equal to the minimum of the value of the \plc{grain-size} expression and the number of logical loop iterations, but less than two times the value of the \plc{grain-size} expression.

The parameter of the \code{grainsize} clause must be a positive integer expression.
If \code{num\_tasks} is specified, the \code{taskloop} construct creates as many tasks as the minimum of the \plc{num-tasks} expression and the number of logical loop iterations.  
Each task must have at least one logical loop iteration.
The parameter of the \code{num\_tasks} clause must evaluate to a positive integer.
If neither a \code{grainsize} nor \code{num\_tasks} clause is present, the number of loop tasks generated and the number of logical loop iterations assigned to these tasks is implementation defined.

The \code{collapse} clause may be used to specify how many loops are associated with the \code{taskloop} construct. The parameter of the \code{collapse} clause must be a constant positive integer expression. If no \code{collapse} clause is present, the only loop that is associated with the \code{taskloop} construct is the one that immediately follows the \code{taskloop} directive.

If more than one loop is associated with the \code{taskloop} construct, then the iterations of all associated loops are collapsed into one larger iteration space that is then divided according to the \code{grainsize} and \code{num\_tasks} clauses. The sequential execution of the iterations in all associated loops determines the order of the iterations in the collapsed iteration space. 
%% TODO: Does this conflict with the note about the independence of loop iterations below?

The iteration count for each associated loop is computed before entry to the outermost loop. If execution of any associated loop changes any of the values used to compute any of the iteration counts, then the behavior is unspecified. 

The integer type (or kind, for Fortran) used to compute the iteration count for the collapsed loop is implementation defined.

When an \code{if} clause is present on a \code{taskloop} construct, and if the \code{if} clause expression evaluates to \plc{false}, undeferred tasks are generated. The use of a variable in an \code{if} clause expression of a \code{taskloop} construct causes an implicit reference to the variable in all enclosing constructs.

When a \code{final} clause is present on a \code{taskloop} construct and the \code{final} clause expression evaluates to \plc{true}, the generated tasks will be final tasks. The use of a variable in a \code{final} clause expression of a \code{taskloop} construct causes an implicit reference to the variable in all enclosing constructs.

When a \code{priority} clause is present on a \code{taskloop} construct,
the generated tasks have the \plc{priority-value} as if it was 
specified for each individual task.  
If the \code{priority} clause is not specified, tasks generated by 
the \code{taskloop} construct have the default task priority (zero).

If the \code{untied} clause is specified, all tasks generated by the \code{taskloop} construct are untied tasks.

When the \code{mergeable} clause is present on a \code{taskloop} construct, each generated task is a \plc{mergeable task}. 

\begin{cppspecific}
For \code{firstprivate} variables of class type, the number of invocations of copy constructors to perform the initialization  is implementation-defined.
\end{cppspecific}

\begin{note}
When storage is shared by a \code{taskloop} region, the programmer must ensure, by adding proper synchronization, that the storage does not reach the end of its lifetime before the \code{taskloop} region and its descendant tasks complete their execution.
\end{note}

\def\omptWorksharingLoop#1#2
{
\events

The \plc{#1{-begin}} event occurs after a task encounters a 
\code{#1} construct but before the task starts the execution of the structured 
block of the \code{#1} region.

The \plc{#1{-end}} event occurs after a \code{#1} region finishes execution 
but before resuming execution of the encountering task.

\tools

A thread dispatches a registered \code{ompt\_callback\_work}
callback for each occurrence of a \plc{#1{-begin}} and
\plc{#1{-end}} event in that thread. The callback occurs in the
context of the encountering task.  The callback has type signature
\code{ompt\_callback\_work\_t}. The callback receives
\code{ompt\_scope\_begin} or \code{ompt\_scope\_end} 
as its \plc{endpoint} argument, as appropriate, and 
\code{#2} as its \plc{wstype} argument.
}
% \omptWorksharingLoop{taskloop}{ompt\_work\_taskloop}

\events

The \plc{taskloop-begin} event occurs after a task encounters a
\code{taskloop} construct but before any other events that may 
trigger as a consequence of executing the \code{taskloop}.
Specifically, a \plc{taskloop-begin} event for a \code{taskloop}
will precede the \plc{taskgroup-begin} that occurs unless a
\code{nogroup} clause is present.  Regardless of whether an implicit
taskgroup is present, a \plc{taskloop-begin} will always precede
any \plc{task-create} events for generated tasks.

The \plc{taskloop-end} event occurs after a \code{taskloop} region finishes execution
but before resuming execution of the encountering task.

\tools

A thread dispatches a registered \code{ompt\_callback\_work}
callback for each occurrence of a \plc{taskloop-begin} and
\plc{taskloop-end} event in that thread. The callback occurs in the
context of the encountering task.  The callback has type signature
\code{ompt\_callback\_work\_t}. The callback receives
\code{ompt\_scope\_begin} or \code{ompt\_scope\_end}
as its \plc{endpoint} argument, as appropriate, and
\code{ompt\_work\_taskloop} as its \plc{wstype} argument.

\restrictions
The restrictions of the \code{taskloop} construct are as follows:
\begin{itemize}
\item A program that branches into or out of a \code{taskloop} region is non-conforming.
\item All loops associated with the \code{taskloop} construct must be perfectly nested; that is, there must be no intervening code nor any OpenMP directive between any two loops.
\item If a \code{reduction} clause is present on the \code{taskloop} directive, the \code{nogroup} clause must not be specified.
\item The same list item cannot appear in both a \code{reduction} and an \code{in\_reduction} clause.
\item At most one \code{grainsize} clause can appear on a \code{taskloop} directive.
\item At most one \code{num\_tasks} clause can appear on a \code{taskloop} directive.
\item The \code{grainsize} clause and \code{num\_tasks} clause are mutually exclusive and may not appear on the same \code{taskloop} directive.
\item At most one \code{collapse} clause can appear on a \code{taskloop} directive.
\item At most one \code{if} clause can appear on the directive.
\item At most one \code{final} clause can appear on the directive. 
\item At most one \code{priority} clause can appear on the directive.
\end{itemize}

\crossreferences
\begin{itemize}
\item \code{task} construct, \specref{subsec:task Construct}.
\item \code{taskgroup} construct, \specref{subsec:taskgroup Construct}.
\item Data-sharing attribute clauses, \specref{subsec:Data-Sharing Attribute Clauses}. 
\item \code{if} Clause, see \specref{sec:if Clause}.
\item \code{ompt\_scope\_begin} and \code{ompt\_scope\_end}, see
  \specref{sec:ompt_scope_endpoint_t}.
\item \code{ompt\_work\_taskloop}, see \specref{sec:ompt_work_type_t}.
\item \code{ompt\_callback\_work\_t}, see
\specref{sec:ompt_callback_work_t}.

\end{itemize}






%
%TASKLOOP SIMD
%
\subsection{\code{taskloop}~\code{simd} Construct}
\index{taskloop simd@{\code{taskloop}~\code{simd}}}
\index{constructs!taskloop simd@{\code{taskloop}~\code{simd}}}
\label{subsec:taskloop simd Construct}
\summary
The \code{taskloop}~\code{simd} construct specifies a loop that can be 
executed concurrently using SIMD instructions and that those iterations 
will also be executed in parallel using explicit tasks. The \code{taskloop}
\code{simd} construct is a composite construct.

\syntax
\begin{ccppspecific}
The syntax of the \code{taskloop}~\code{simd} construct is as follows:
\begin{boxedcode}
\#pragma omp taskloop simd \plc{[clause[[,] clause] ...] new-line}
    \plc{for-loops}
\end{boxedcode}
where \plc{clause} can be any of the clauses accepted by the \code{taskloop} or \code{simd} directives with identical meanings and restrictions.
\end{ccppspecific}
\begin{fortranspecific}
The syntax of the \code{taskloop}~\code{simd} construct is as follows:
\begin{boxedcode}
!\$omp taskloop simd \plc{[clause[[,] clause] ...]}
    \plc{do-loops}
\plc{[}!\$omp end taskloop simd\plc{]}
\end{boxedcode}
where \plc{clause} can be any of the clauses accepted by the \code{taskloop} or \code{simd} directives with identical meanings and restrictions.

If an \code{end}~\code{taskloop}~\code{simd} directive is not specified, an \code{end}~\code{taskloop}~\code{simd} directive is assumed at the end of the \plc{do-loops}.
\end{fortranspecific}

\binding
The binding thread set of the \code{taskloop}~\code{simd} region is the current team. A \code{taskloop}~\code{simd} region binds to the innermost enclosing parallel region.

\descr
The \code{taskloop}~\code{simd} construct will first distribute the iterations of the associated loop(s) across tasks in a manner consistent with any clauses that apply to the \code{taskloop} construct. The resulting tasks will then be converted to a SIMD loop in a manner consistent with any clauses that apply to the \code{simd} construct, except for the \code{collapse} clause. For the purposes of each task's conversion to a SIMD loop, the \code{collapse} clause is ignored and the effect of any \code{in\_reduction} clause is as if a \code{reduction} clause with the same reduction operator and list items is present on the construct.

\events

This composite construct generates the same events as the \code{taskloop} construct.

\tools

This composite construct dispatches the same callbacks as the \code{taskloop} construct.

\restrictions
\begin{itemize}
\item The restrictions for the \code{taskloop} and \code{simd} constructs apply.
\end{itemize}

\crossreferences
\begin{itemize}
\item \code{taskloop} construct, see \specref{subsec:taskloop Construct}.
\item \code{simd} construct, see \specref{subsec:simd Construct}.
\item Data-sharing attribute clauses, see \specref{subsec:Data-Sharing Attribute Clauses}. 
\item Events and tool callbacks for \code{taskloop} construct, see
\specref{subsec:taskloop Construct}.

\end{itemize}




%
%   TASKYIELD CONSTRUCT
%
\subsection{\code{taskyield} Construct}
\index{taskyield@{\code{taskyield}}}
\index{constructs!taskyield@{\code{taskyield}}}
\label{subsec:taskyield Construct}
\summary
The \code{taskyield} construct specifies that the current task can be suspended in favor of 
execution of a different task. The \code{taskyield} construct is a stand-alone directive.

\syntax
\begin{ccppspecific}
The syntax of the \code{taskyield} construct is as follows:

\begin{boxedcode}
\#pragma omp taskyield \plc{new-line}
\end{boxedcode}
\end{ccppspecific}

\begin{fortranspecific}
The syntax of the \code{taskyield} construct is as follows:

\begin{boxedcode}
!\$omp taskyield
\end{boxedcode}
\end{fortranspecific}

\binding
A \code{taskyield} region binds to the current task region. The binding thread set of the 
\code{taskyield} region is the current team.

\descr
The \code{taskyield} region includes an explicit task scheduling point in the current task 
region.

\crossreferences
\begin{itemize}
\item Task scheduling, see 
\specref{subsec:Task Scheduling}. 
\end{itemize}











\subsection{Initial Task}

\events
No events are associated with the implicit parallel region in each initial thread.

The \plc{initial-thread-begin} event occurs in an initial thread after the OpenMP runtime invokes the tool initializer 
but before the initial thread begins to execute the first OpenMP region in the initial task. 

The \plc{initial-task-create} event occurs after an \plc{initial-thread-begin} event 
but before the first OpenMP region in the initial task begins to execute.

The \plc{initial-thread-end} event occurs as the final event in an initial thread at the end of an initial task 
immediately prior to invocation of the tool finalizer.

\tools

A thread dispatches a registered \code{ompt\_callback\_thread\_begin}
callback for the \plc{initial-thread-begin} event in an initial thread. 
The callback occurs in the context of the initial thread.  
The callback has type signature \code{ompt\_callback\_thread\_begin\_t}. 
The callback receives \code{ompt\_thread\_initial} as its \plc{thread\_type} argument.

A thread dispatches a registered \code{ompt\_callback\_task\_create}
callback for each occurrence of a \plc{initial-task-create} event
in the context of the encountering task.  This callback has the type signature
\code{ompt\_callback\_task\_create\_t}. 
The callback receives \code{ompt\_task\_initial} as its \plc{type} argument. 
The implicit parallel region does not dispatch a \code{ompt\_callback\_parallel\_begin} callback;
however, the implicit parallel region can be initialized within this \code{ompt\_callback\_task\_create}
callback.

A thread dispatches a registered \code{ompt\_callback\_thread\_end}
callback for the \plc{initial-thread-end} event in that thread. 
The callback occurs in the context of the thread.  The callback has type signature
\code{ompt\_callback\_thread\_end\_t}. 
The implicit parallel region does not dispatch a \code{ompt\_callback\_parallel\_end} callback;
however, the implicit parallel region can be finalized within this \code{ompt\_callback\_thread\_end}
callback.

\crossreferences
\begin{itemize}

\item \code{ompt\_task\_initial}, see
\specref{sec:ompt_task_type_t}.

\item \code{ompt\_callback\_thread\_begin\_t}, see
  \specref{sec:ompt_callback_thread_begin_t}.

\item \code{ompt\_callback\_thread\_end\_t}, see
  \specref{sec:ompt_callback_thread_end_t}.

\item \code{ompt\_callback\_task\_create\_t}, see
\specref{sec:ompt_callback_task_create_t}.


\end{itemize}

\subsection{Task Scheduling}
\index{task scheduling}
\index{scheduling}
\label{subsec:Task Scheduling}
Whenever a thread reaches a task scheduling point, the implementation may cause it to 
perform a task switch, beginning or resuming execution of a different task bound to the 
current team. Task scheduling points are implied at the following locations:

\begin{itemize}
\item the point immediately following the generation of an explicit task;

\item after the point of completion of a \code{task} region;

\item in a \code{taskyield} region;

\item in a \code{taskwait} region;

\item at the end of a \code{taskgroup} region;

\item in an implicit and explicit \code{barrier} region;

\item the point immediately following the generation of a \code{target} region;

\item at the beginning and end of a \code{target}~\code{data} region;

\item in a \code{target}~\code{update} region; 

\item in a \code{target}~\code{enter}~\code{data} region; 

\item in a \code{target}~\code{exit}~\code{data} region; 

\item in the \code{omp\_target\_memcpy} routine; 

\item in the \code{omp\_target\_memcpy\_rect} routine;

\end{itemize}

When a thread encounters a task scheduling point it may do one of the following, 
subject to the \emph{Task Scheduling Constraints} (below):

\begin{itemize}
\item begin execution of a tied task bound to the current team

\item resume any suspended task region, bound to the current team, to which it is tied

\item begin execution of an untied task bound to the current team

\item resume any suspended untied task region bound to the current team.
\end{itemize}

If more than one of the above choices is available, it is unspecified as to which will be 
chosen.

\emph{Task Scheduling Constraints} are as follows:

\begin{enumerate}
\item An included task is executed immediately after generation of the task.

\item Scheduling of new tied tasks is constrained by the set of task regions that are currently 
tied to the thread, and that are not suspended in a \code{barrier} region. If this set is empty, 
any new tied task may be scheduled. Otherwise, a new tied task may be scheduled only 
if it is a descendent task of every task in the set.

\item A dependent task shall not be scheduled until its task dependences are fulfilled.

\item When an explicit task is generated by a construct containing an \code{if} clause for which the 
expression evaluated to \plc{false}, and the previous constraints are already met, the task is 
executed immediately after generation of the task.
\end{enumerate}

A program relying on any other assumption about task scheduling is non-conforming.

\begin{note}
Task scheduling points dynamically divide task regions into parts. Each part is 
executed uninterrupted from start to end. Different parts of the same task region are 
executed in the order in which they are encountered. In the absence of task 
synchronization constructs, the order in which a thread executes parts of different 
schedulable tasks is unspecified.

A correct program must behave correctly and consistently with all conceivable 
scheduling sequences that are compatible with the rules above.

For example, if \code{threadprivate} storage is accessed (explicitly in the source code or 
implicitly in calls to library routines) in one part of a task region, its value cannot be 
assumed to be preserved into the next part of the same task region if another schedulable 
task exists that modifies it.

As another example, if a lock acquire and release happen in different parts of a task 
region, no attempt should be made to acquire the same lock in any part of another task 
that the executing thread may schedule. Otherwise, a deadlock is possible. A similar 
situation can occur when a \code{critical} region spans multiple parts of a task and another 
schedulable task contains a \code{critical} region with the same name.

The use of threadprivate variables and the use of locks or critical sections in an explicit 
task with an \code{if} clause must take into account that when the \code{if} clause evaluates to 
\plc{false}, the task is executed immediately, without regard to \emph{Task Scheduling Constraint}~2.
\end{note}

\events

The \plc{task-schedule} event occurs in a thread when the thread switches tasks at a task scheduling point;
no event occurs when switching to or from a merged task.

\tools

A thread dispatches a registered \code{ompt\_callback\_task\_schedule}
callback for each occurrence of a \plc{task-schedule} event in 
the context of the task that begins or resumes. This callback has the type signature
\code{ompt\_callback\_task\_schedule\_t}. 
The argument  \plc{prior\_task\_status} is used to indicate the cause for suspending the prior task. 
This cause may be the completion of the prior task region, the encountering of a 
\code{taskyield} construct, or the encountering of an active cancellation point.

\crossreferences
\begin{itemize}
\item \code{ompt\_callback\_task\_schedule\_t}, see
\specref{sec:ompt_callback_task_schedule_t}.
\end{itemize}

% This is an included file. See the master file for more information.
%
% When editing this file:
%
%    1. To change formatting, appearance, or style, please edit openmp.sty.
%
%    2. Custom commands and macros are defined in openmp.sty.
%
%    3. Be kind to other editors -- keep a consistent style by copying-and-pasting to
%       create new content.
%
%    4. We use semantic markup, e.g. (see openmp.sty for a full list):
%         \code{}     % for bold monospace keywords, code, operators, etc.
%         \plc{}      % for italic placeholder names, grammar, etc.
%
%    5. There are environments that provide special formatting, e.g. language bars.
%       Please use them whereever appropriate.  Examples are:
%
%         \begin{fortranspecific}
%         This is text that appears enclosed in blue language bars for Fortran.
%         \end{fortranspecific}
%
%         \begin{note}
%         This is a note.  The "Note -- " header appears automatically.
%         \end{note}
%
%    6. Other recommendations:
%         Use the convenience macros defined in openmp.sty for the minor headers
%         such as Comments, Syntax, etc.
%
%         To keep items together on the same page, prefer the use of
%         \begin{samepage}.... Avoid \parbox for text blocks as it interrupts line numbering.
%         When possible, avoid \filbreak, \pagebreak, \newpage, \clearpage unless that's
%         what you mean. Use \needspace{} cautiously for troublesome paragraphs.
%
%         Avoid absolute lengths and measures in this file; use relative units when possible.
%         Vertical space can be relative to \baselineskip or ex units. Horizontal space
%         can be relative to \linewidth or em units.
%
%         Prefer \emph{} to italicize terminology, e.g.:
%             This is a \emph{definition}, not a placeholder.
%             This is a \plc{var-name}.
%


\section{Device Directives}
\label{sec:Device Directives}
\index{device directives}
\index{constructs!device constructs}
\index{device constructs!device constructs}

\subsection{Device Initialization}

\events

The \plc{device-initialize} event occurs in a thread that encounters the first 
\code{target}, \code{target data}, or \code{target enter data} construct that 
is associated with a particular target device after the thread initiates 
initialization of OpenMP on the device and the device's OpenMP initialization, 
which may include device-side tool initialization, completes.

The \plc{device-load} event for a code block for a target device occurs in some 
thread before any thread executes code from that code block on that target device.

The \plc{device-unload} event for a target device occurs in some thread
whenever a code block is unloaded from the device.

The \plc{device-finalize} event for a target device that has been initialized
occurs in some thread before an OpenMP implementation shuts down.

\tools

A thread dispatches a registered \code{ompt_callback_device_initialize}
callback for each occurrence of a \plc{device-initialize} event in that 
thread.  This callback has type signature \code{ompt_callback_device_initialize_t}.

A thread dispatches a registered \code{ompt_callback_device_load} callback 
for each occurrence of a \plc{device-load} event in that thread. This callback 
has type signature \code{ompt_callback_device_load_t}.

A thread dispatches a registered \code{ompt_callback_device_unload}
callback for each occurrence of a \plc{device-unload} event in
that thread.  This callback has type signature
\code{ompt_callback_device_unload_t}.

A thread dispatches a registered \code{ompt_callback_device_finalize}
callback for each occurrence of a \plc{device-finalize} event in
that thread.  This callback has type signature
\code{ompt_callback_device_finalize_t}.

\restrictions
No thread may offload execution of an OpenMP construct to a device until a
dispatched \code{ompt_callback_device_initialize} callback completes.

No thread may offload execution of an OpenMP construct to a device after a
dispatched \code{ompt_callback_device_finalize} callback occurs.

\crossreferences
\begin{itemize}
\item \code{ompt_callback_device_load_t}, see
\specref{sec:ompt_callback_device_load_t}.

\item \code{ompt_callback_device_unload_t}, see
\specref{sec:ompt_callback_device_unload_t}.

\item \code{ompt_callback_device_initialize_t}, see
\specref{sec:ompt_callback_device_initialize_t}.

\item \code{ompt_callback_device_finalize_t}, see
\specref{sec:ompt_callback_device_finalize_t}.
\end{itemize}


\subsection{\hcode{target}~\hcode{data} Construct}
\index{target data@{\code{target}~\code{data}}}
\index{constructs!target data@{\code{target}~\code{data}}}
\label{subsec:target data Construct}
\summary
Map variables to a device data environment for the extent of the region.

\syntax
\begin{ccppspecific}
The syntax of the \code{target}~\code{data} construct is as follows:

\begin{ompcPragma}
#pragma omp target data \plc{clause[ [ [},\plc{] clause] ... ] new-line}
    \plc{structured-block}
\end{ompcPragma}

\needspace{10\baselineskip}
\begin{samepage}
where \plc{clause} is one of the following:

\begin{indentedcodelist}
if(\plc{[} target data :\plc{] scalar-expression})
device(\plc{integer-expression})
map(\plc{[[map-type-modifier[},\plc{] [map-type-modifier[},\plc{] ...] map-type}:\plc{ ] locator-list})
use_device_ptr(\plc{ptr-list})
use_device_addr(\plc{list})
\end{indentedcodelist}
\end{samepage}
\end{ccppspecific}
\medskip

\begin{fortranspecific}
The syntax of the \code{target}~\code{data} construct is as follows:

\begin{ompfPragma}
!$omp target data \plc{clause[ [ [},\plc{] clause] ... ]}
    \plc{structured-block}
!$omp end target data
\end{ompfPragma}

where \plc{clause} is one of the following:

\begin{indentedcodelist}
if(\plc{[} target data :\plc{] scalar-logical-expression})
device(\plc{scalar-integer-expression})
map(\plc{[[map-type-modifier[},\plc{] [map-type-modifier[},\plc{] ...] map-type}:\plc{ ] locator-list})
use_device_ptr(\plc{ptr-list})
use_device_addr(\plc{list})
\end{indentedcodelist}

\end{fortranspecific}

\binding
The binding task set for a \code{target}~\code{data} region is the generating task. 
The \code{target}~\code{data} region binds to the region of the generating task.

\descr
When a \code{target}~\code{data} construct is encountered, the encountering task
executes the region. If there is no \code{device} clause, the default device is
determined by the \plc{default-device-var} ICV. When an \code{if} clause is present 
and the \code{if} clause expression evaluates to \plc{false}, the device is the host.
Variables are mapped for the extent of the region, according to any data-mapping 
attribute clauses, from the data environment of the encountering task to the 
device data environment.

Pointers that appear in a \code{use_device_ptr} clause are privatized and the
device pointer to the corresponding list items in the device data environment
are assigned into the private versions.  

List items that appear in a \code{use_device_addr} clause have the address of
the corresponding object in the device data environment inside the construct.
For objects, any reference to the value of the object will be to the
corresponding object on the device, while references to the address will result
in a valid device address that points to that object.  Array sections privatize the
base of the array section and assign the private copy to the address of the
corresponding array section in the device data environment.  

If one or more of the \code{use_device_ptr} or
\code{use_device_addr} clauses and one or more \code{map} clauses are present on the
same construct, the address conversions of \code{use_device_addr} and
\code{use_device_ptr} clauses will occur as if performed after all variables
are mapped according to those \code{map} clauses.

\events

The \plc{target-data-begin} event occurs when a thread enters 
a \code{target}~\code{data} region.

The \plc{target-data-end} event occurs when a thread exits a
\code{target}~\code{data} region.

\tools

A thread dispatches a registered \code{ompt_callback_target} callback with 
\code{ompt_scope_begin} as its \plc{endpoint} argument and 
\code{ompt_target_enter_data} as its \plc{kind} argument for each occurrence 
of a \plc{target-data-begin} event in that thread in the context of the task 
that encounters the construct. Similarly, a thread dispatches a registered 
\code{ompt_callback_target} callback with \code{ompt_scope_end} as its 
\plc{endpoint} argument and \code{ompt_target_exit_data} as its \plc{kind} 
argument for each occurrence of a \plc{target-data-end} event in that thread 
in the context of the task that encounters the construct. These callbacks have 
type signature \code{ompt_callback_target_t}. 

\restrictions
\begin{itemize}
\item A program must not depend on any ordering of the evaluations of the
      clauses of the \code{target}~\code{data} directive, except as explicitly
      stated for \code{map} clauses relative to \code{use_device_ptr} and
      \code{use_device_addr} clauses, or on any side effects of the evaluations 
      of the clauses.
\item At most one \code{device} clause can appear on the directive. The
      \code{device} clause expression  must evaluate to a non-negative integer
      value less than the value of \code{omp_get_num_devices()} or to the value of
      \code{omp_get_initial_device()}.
\item At most one \code{if} clause can appear on the directive.
\item A \plc{map-type} in a \code{map} clause must be \code{to}, \code{from}, 
      \code{tofrom} or \code{alloc}.
\item At least one \code{map}, \code{use_device_addr} or \code{use_device_ptr}
      clause must appear on the directive.
\item A list item in a \code{use_device_ptr} clause must hold the address of
      an object that has a corresponding list item in the device data environment.
\item A list item in a \code{use_device_addr} clause must have a
      corresponding list item in the device data environment.
\item A list item that specifies a given variable may not appear in more than
      one \code{use_device_ptr} clause.
\item A reference to a list item in a \code{use_device_addr} clause must be to
      the address of the list item.
\end{itemize}


\crossreferences
\begin{itemize}
\item \plc{default-device-var}, see
\specref{sec:Internal Control Variables}.

\item \code{if} Clause, see \specref{sec:if Clause}.

\item \code{map} clause, see
\specref{subsec:map Clause}.

\item \code{omp_get_num_devices} routine, see \specref{subsec:omp_get_num_devices}.

\item \code{ompt_callback_target_t}, see
\specref{sec:ompt_callback_target_t}.

\end{itemize}



\subsection{\hcode{target}~\hcode{enter}~\hcode{data} Construct}
\label{subsec:target enter data Construct}
\index{constructs!target enter data@{\code{target}~\code{enter}~\code{data}}}
\index{device data environments}

\summary
The \code{target}~\code{enter}~\code{data} directive specifies that variables 
are mapped to a device data environment. The \code{target}~\code{enter}~\code{data} 
directive is a stand-alone directive.

\syntax

\begin{ccppspecific}
The syntax of the \code{target}~\code{enter}~\code{data} construct is as follows:
\begin{ompcPragma}
#pragma omp target enter data \plc{[ clause[ [},\plc{] clause]...] new-line}
\end{ompcPragma}
where \plc{clause} is one of the following:
\begin{indentedcodelist}
if(\plc{[} target enter data :\plc{] scalar-expression})
device(\plc{integer-expression})
map(\plc{[map-type-modifier[},\plc{] [map-type-modifier[},\plc{] ...] map-type}:\plc{ locator-list})
depend(\plc{[depend-modifier},\plc{] dependence-type }:\plc{ locator-list})
nowait
\end{indentedcodelist}
\end{ccppspecific}

\begin{fortranspecific}
The syntax of the \code{target}~\code{enter}~\code{data} is as follows:
\begin{ompfPragma}
!$omp target enter data \plc{[ clause[ [},\plc{] clause]...]}
\end{ompfPragma}
where clause is one of the following:
\begin{indentedcodelist}
if(\plc{[} target enter data :\plc{] scalar-logical-expression})
device(\plc{scalar-integer-expression})
map(\plc{[map-type-modifier[},\plc{] [map-type-modifier[},\plc{] ...] map-type}:\plc{ locator-list})
depend(\plc{[depend-modifier},\plc{] dependence-type }:\plc{ locator-list})
nowait
\end{indentedcodelist}
\end{fortranspecific}

\binding
The binding task set for a \code{target}~\code{enter}~\code{data} region is
the generating task, which is the \plc{target task} generated by the
\code{target}~\code{enter}~\code{data} construct. The
\code{target}~\code{enter}~\code{data} region binds to the corresponding
\plc{target task} region.

\descr
When a \code{target}~\code{enter}~\code{data} construct is encountered, the 
list items are mapped to the device data environment according to the \code{map} 
clause semantics.

The \code{target}~\code{enter}~\code{data} construct is a task generating 
construct.  The generated task is a \plc{target task}.  The generated task 
region encloses the \code{target}~\code{enter}~\code{data} region.

All clauses are evaluated when the \code{target}~\code{enter}~\code{data} 
construct is encountered.  The data environment of the \plc{target task} 
is created according to the data-sharing attribute clauses on the 
\code{target}~\code{enter}~\code{data} construct, per-data environment ICVs, 
and any default data-sharing attribute rules that apply to the 
\code{target}~\code{enter}~\code{data} construct.  A variable that is 
mapped in the \code{target}~\code{enter}~\code{data} construct has a 
default data-sharing attribute of shared in the data environment of 
the \plc{target task}.

Assignment operations associated with mapping a variable (see 
\specref{subsec:map Clause}) occur when the \plc{target task} executes.

If the \code{nowait} clause is present, execution of the \plc{target task} 
may be deferred.  If the \code{nowait} clause is not present, the 
\plc{target task} is an included task.

If a \code{depend} clause is present, it is associated with the \plc{target task}.

If no \code{device} clause is present, the default device is determined by the 
\plc{default-device-var} ICV.

When an \code{if} clause is present and the \code{if} clause expression 
evaluates to \plc{false}, the device is the host.

\events

Events associated with a \plc{target task} are the same as for the \code{task} 
construct defined in \specref{subsec:task Construct}.

The \plc{target-enter-data-begin} event occurs when a thread enters a
\code{target}~\code{enter}~\code{data} region.

The \plc{target-enter-data-end} event occurs when a thread exits a
\code{target}~\code{enter}~\code{data} region.

\tools
Callbacks associated with events for \plc{target tasks} are the same as 
for the \code{task} construct defined in \specref{subsec:task Construct}.

A thread dispatches a registered \code{ompt_callback_target} callback with
\code{ompt_scope_begin} as its \plc{endpoint} argument and
\code{ompt_target_enter_data} as its \plc{kind} argument for each occurrence
of a \plc{target-enter-data-begin} event in that thread in the context of the 
target task on the host. Similarly, a thread dispatches a registered 
\code{ompt_callback_target} callback with \code{ompt_scope_end} as its
\plc{endpoint} argument and \code{ompt_target_enter_data} as its \plc{kind}
argument for each occurrence of a \plc{target-enter-data-end} event in that thread
in the context of the target task on the host. These callbacks have
type signature \code{ompt_callback_target_t}.

\restrictions
\begin{itemize}
\item A program must not depend on any ordering of the evaluations of the clauses 
      of the \code{target}~\code{enter}~\code{data} directive, or on any side 
      effects of the evaluations of the clauses.
\item At least one \code{map} clause must appear on the directive.
\item At most one \code{device} clause can appear on the directive. The 
      \code{device} clause expression must evaluate to a non-negative integer 
      value less than the value of \code{omp_get_num_devices()} or to the 
      value of \code{omp_get_initial_device()}.
\item At most one \code{if} clause can appear on the directive.
\item A \plc{map-type} must be specified in all \code{map} clauses and must be 
      either \code{to} or \code{alloc}.
\item At most one \code{nowait} clause can appear on the directive.
\end{itemize}

\crossreferences
\begin{itemize}
\item \plc{default-device-var}, see \specref{subsec:ICV Descriptions}.

\item \code{task}, see \specref{subsec:task Construct}.

\item \code{task}~\code{scheduling}~\code{constraints},
see \specref{subsec:Task Scheduling}.

\item \code{target}~\code{data}, see \specref{subsec:target data Construct}.

\item \code{target}~\code{exit}~\code{data},
see \specref{subsec:target exit data Construct}.

\item \code{if} Clause, see \specref{sec:if Clause}.

\item \code{map} clause, see \specref{subsec:map Clause}.

\item \code{omp_get_num_devices} routine, see \specref{subsec:omp_get_num_devices}.

\item \code{ompt_callback_target_t}, see
\specref{sec:ompt_callback_target_t}.
\end{itemize}



\subsection{\hcode{target}~\hcode{exit}~\hcode{data} Construct}
\label{subsec:target exit data Construct}
\index{constructs!target exit data@{\code{target}~\code{exit}~\code{data}}}
\index{device data environments}
\summary
The \code{target}~\code{exit}~\code{data} directive specifies that list items 
are unmapped from a device data environment. The \code{target}~\code{exit}~\code{data}
 directive is a stand-alone directive.

\syntax
\begin{ccppspecific}
The syntax of the \code{target}~\code{exit}~\code{data} construct is as follows:
\begin{ompcPragma}
#pragma omp target exit data \plc{[ clause[ [},\plc{] clause]...] new-line}
\end{ompcPragma}
where \plc{clause} is one of the following:
\begin{indentedcodelist}
if(\plc{[} target exit data :\plc{] scalar-expression})
device(\plc{integer-expression})
map(\plc{[map-type-modifier[},\plc{] [map-type-modifier[},\plc{] ...] map-type}:\plc{ locator-list})
depend(\plc{[depend-modifier},\plc{] dependence-type }:\plc{ locator-list})
nowait
\end{indentedcodelist}
\end{ccppspecific}
\begin{fortranspecific}
The syntax of the \code{target}~\code{exit}~\code{data} is as follows:
\begin{ompfPragma}
!$omp target exit data \plc{[ clause[ [},\plc{] clause]...]}
\end{ompfPragma}
where clause is one of the following:
\begin{indentedcodelist}
if(\plc{[} target exit data :\plc{] scalar-logical-expression})
device(\plc{scalar-integer-expression})
map(\plc{[map-type-modifier[},\plc{] [map-type-modifier[},\plc{] ...] map-type}:\plc{ locator-list})
depend(\plc{[depend-modifier},\plc{] dependence-type }:\plc{ locator-list})
nowait
\end{indentedcodelist}
\end{fortranspecific}

\binding
The binding task set for a \code{target}~\code{exit}~\code{data} region is
the generating task, which is the \plc{target task} generated by the
\code{target}~\code{exit}~\code{data} construct. The
\code{target}~\code{exit}~\code{data} region binds to the corresponding
\plc{target task} region.

\descr
When a \code{target}~\code{exit}~\code{data} construct is encountered, the list 
items in the \code{map} clauses are unmapped from the device data environment 
according to the \code{map} clause semantics.

The \code{target}~\code{exit}~\code{data} construct is a task generating construct.  
The generated task is a \plc{target task}.  The generated task region encloses the 
\code{target}~\code{exit}~\code{data} region.

All clauses are evaluated when the \code{target}~\code{exit}~\code{data} construct 
is encountered.  The data environment of the \plc{target task} is created according 
to the data-sharing attribute clauses on the \code{target}~\code{exit}~\code{data} 
construct, per-data environment ICVs, and any default data-sharing attribute rules 
that apply to the \code{target}~\code{exit}~\code{data} construct.  A variable that 
is mapped in the \code{target}~\code{exit}~\code{data} construct has a default 
data-sharing attribute of shared in the data environment of the \plc{target task}.

Assignment operations associated with mapping a variable (see 
\specref{subsec:map Clause}) occur when the \plc{target task} executes.

If the \code{nowait} clause is present, execution of the \plc{target task} may be 
deferred.  If the \code{nowait} clause is not present, the \plc{target task} is an 
included task.

If a \code{depend} clause is present, it is associated with the \plc{target task}.

If no \code{device} clause is present, the default device is determined by the 
\plc{default-device-var} ICV.

When an \code{if} clause is present and the \code{if} clause expression evaluates 
to \plc{false}, the device is the host.

\events

Events associated with a \plc{target task} are the same as for the \code{task} 
construct defined in \specref{subsec:task Construct}.

The \plc{target-exit-begin} event occurs when a thread enters a
\code{target}~\code{exit}~\code{data} region.

The \plc{target-exit-end} event occurs when a thread exits a
\code{target}~\code{exit}~\code{data} region.

\tools

Callbacks associated with events for \plc{target tasks} are the same as
for the \code{task} construct defined in \specref{subsec:task Construct}.

A thread dispatches a registered \code{ompt_callback_target} callback with
\code{ompt_scope_begin} as its \plc{endpoint} argument and
\code{ompt_target_exit_data} as its \plc{kind} argument for each occurrence
of a \plc{target-exit-data-begin} event in that thread in the context of the
target task on the host. Similarly, a thread dispatches a registered
\code{ompt_callback_target} callback with \code{ompt_scope_end} as its
\plc{endpoint} argument and \code{ompt_target_exit_data} as its \plc{kind}
argument for each occurrence of a \plc{target-exit-data-end} event in that thread
in the context of the target task on the host. These callbacks have
type signature \code{ompt_callback_target_t}. 

\restrictions
\begin{itemize}
\item A program must not depend on any ordering of the evaluations of the clauses 
      of the \code{target}~\code{exit}~\code{data} directive, or on any side effects 
      of the evaluations of the clauses.
\item At least one \code{map} clause must appear on the directive.
\item At most one \code{device} clause can appear on the directive. The 
      \code{device} clause expression must evaluate to a non-negative integer 
      value less than the value of \code{omp_get_num_devices()} or to the 
      value of \code{omp_get_initial_device()}.
\item At most one \code{if} clause can appear on the directive.
\item A \plc{map-type} must be specified in all \code{map} clauses and must be 
      either \code{from}, \code{release}, or \code{delete}.
\item At most one \code{nowait} clause can appear on the directive.
\end{itemize}

\crossreferences
\begin{itemize}
\item \plc{default-device-var}, see \specref{subsec:ICV Descriptions}.

\item \code{task}, see \specref{subsec:task Construct}.

\item \code{task}~\code{scheduling}~\code{constraints},
see \specref{subsec:Task Scheduling}.

\item \code{target}~\code{data}, see \specref{subsec:target data Construct}.

\item \code{target}~\code{enter}~\code{data},
see \specref{subsec:target enter data Construct}.

\item \code{if} Clause, see \specref{sec:if Clause}.

\item \code{map} clause, see \specref{subsec:map Clause}.

\item \code{omp_get_num_devices} routine, see \specref{subsec:omp_get_num_devices}.

\item \code{ompt_callback_target_t}, see
\specref{sec:ompt_callback_target_t}.

\end{itemize}



\subsection{\hcode{target} Construct}
\index{target@{\code{target}}}
\index{constructs!target@{\code{target}}}
\index{device constructs!target@{\code{target}}}
\label{subsec:target Construct}
\summary
Map variables to a device data environment and execute the construct on that device.

\syntax
\begin{ccppspecific}
The syntax of the \code{target} construct is as follows:

\begin{ompcPragma}
#pragma omp target \plc{[clause[ [},\plc{] clause] ... ] new-line}
    \plc{structured-block}
\end{ompcPragma}

where \plc{clause} is one of the following:

\begin{indentedcodelist}
if(\plc{[} target :\plc{] scalar-expression})
device(\plc{[ device-modifier} :\plc{] integer-expression})
private(\plc{list})
firstprivate(\plc{list})
in_reduction(\plc{reduction-identifier }:\plc{ list})
map(\plc{[[map-type-modifier[},\plc{] [map-type-modifier[},\plc{] ...] map-type}:\plc{ ] locator-list})
is_device_ptr(\plc{list})
defaultmap(\plc{implicit-behavior[:variable-category]})
nowait
depend(\plc{[depend-modifier},\plc{] dependence-type }:\plc{ locator-list})
allocate([\plc{[allocator }:\plc{] list})
uses_allocators(\plc{allocator[}(\plc{allocator-traits-array})\plc{]}
	       \plc{[},\plc{allocator[}(\plc{allocator-traits-array})\plc{] ...]})
\end{indentedcodelist}

where \plc{device-modifier} is one of the following:
\begin{indentedcodelist}
ancestor
device_num
\end{indentedcodelist}

and where \plc{allocator} is an identifier of \code{const omp_allocator_t *} type
and \plc{allocator-traits-array} is an identifier of 
\code{const omp_alloctrait_t *} type.
\end{ccppspecific}

\begin{samepage}
\smallskip
\begin{fortranspecific}
The syntax of the \code{target} construct is as follows:

\begin{ompfPragma}
!$omp target \plc{[clause[ [},\plc{] clause] ... ]}
    \plc{structured-block}
!$omp end target
\end{ompfPragma}

where \plc{clause} is one of the following:

\begin{indentedcodelist}
if(\plc{[} target :\plc{] scalar-logical-expression})
device(\plc{[ device-modifier} :\plc{] scalar-integer-expression})
private(\plc{list})
firstprivate(\plc{list})
in_reduction(\plc{reduction-identifier }:\plc{ list})
map(\plc{[[map-type-modifier[},\plc{] [map-type-modifier[},\plc{] ...] map-type}:\plc{ ] locator-list})
is_device_ptr(\plc{list})
defaultmap(\plc{implicit-behavior[:variable-category]})
nowait
depend(\plc{[depend-modifier},\plc{] dependence-type }:\plc{ locator-list})
allocate(\plc{[allocator}:\plc{]list})
uses_allocators(\plc{allocator[}(\plc{allocator-traits-array})\plc{]}
	       \plc{[},\plc{allocator[}(\plc{allocator-traits-array})\plc{] ...]})
\end{indentedcodelist}

where \plc{device-modifier} is one of the following:
\begin{indentedcodelist}
ancestor
device_num
\end{indentedcodelist}

and where \plc{allocator} is an integer expression of \code{omp_allocator_kind} 
\plc{kind} and \plc{allocator-traits-array} is an array of 
\code{type(omp_alloctrait)} type.
\end{fortranspecific}

\end{samepage}

\binding
The binding task set for a \code{target} region is the generating task, which is 
the \plc{target task} generated by the \code{target} construct. The \code{target}
region binds to the corresponding \plc{target task} region.

\descr
The \code{target} construct provides a superset of the functionality provided by 
the \code{target}~\code{data} directive, except for the \code{use_device_ptr} and 
\code{use_device_addr} clauses.

The functionality added to the \code{target} directive is the inclusion of an 
executable region to be executed by a device. That is, the \code{target} directive 
is an executable directive.

The \code{target} construct is a task generating construct.  The generated task is 
a \plc{target task}.  The generated task region encloses the \code{target} region.

All clauses are evaluated when the \code{target} construct is encountered.
The data environment of the \plc{target task} is created according to the
data-sharing attribute clauses on the \code{target} construct, per-data
environment ICVs, and any default data-sharing attribute rules that apply to
the \code{target} construct.  If a variable or part of a variable is mapped by
the \code{target} construct and does not appear as a list item in an
\code{in_reduction} clause on the construct, the variable has a default
data-sharing attribute of shared in the data environment of the \plc{target task}.

Assignment operations associated with mapping a variable (see 
\specref{subsec:map Clause}) occur when the \plc{target task} executes.

If a \code{device} clause in which the \code{device_num} \plc{device-modifier} 
appears is present on the construct, the \code{device} clause expression specifies 
the device number of the target device. If \plc{device-modifier} does not appear 
in the clause, the behavior of the clause is as if \plc{device-modifier} is 
\code{device_num}.

If a \code{device} clause in which the \code{ancestor} \plc{device-modifier} 
appears is present on the \code{target} construct and the \code{device} clause 
expression evaluates to 1, execution of the \code{target} region occurs on the 
parent device of the enclosing \code{target} region.  If the \code{target} 
construct is not encountered in a \code{target} region, the current device is 
treated as the parent device.  The encountering thread waits for completion of the
\code{target} region on the parent device before resuming. For any list item
that appears in a \code{map} clause on the same construct, if the
corresponding list item exists in the device data environment of the parent
device, it is treated as if it has a reference count of positive infinity.

If the \code{nowait} clause is present, execution of the \plc{target task} may be 
deferred.  If the \code{nowait} clause is not present, the \plc{target task} is an 
included task.

If a \code{depend} clause is present, it is associated with the \plc{target task}.

When an \code{if} clause is present and the \code{if} clause expression evaluates 
to \plc{false}, the \code{target} region is executed by the host device in the host 
data environment.

The \code{is_device_ptr} clause is used to indicate that a list item is a device
pointer already in the device data environment and that it should be used
directly.  Support for device pointers created outside of OpenMP, specifically
outside of the \code{omp_target_alloc} routine and the \code{use_device_ptr} clause,
is implementation defined.

If a function (C, C++, Fortran) or subroutine (Fortran) is referenced in a
\code{target} construct then that function or subroutine is treated as if its
name had appeared in a \code{to} clause on a \code{declare}~\code{target}
directive.

Each memory \plc{allocator} specified in the \code{uses_allocators} clause will 
be made available in the \code{target} region. For each non-predefined allocator 
that is specified, a new allocator handle will be associated with an allocator 
that is created with the specified \plc{traits} as if by a call to 
\code{omp_init_allocator} at the beginning of the \code{target} region. Each 
non-predefined allocator will be destroyed as if by a call to 
\code{omp_destroy_allocator} at the end of the \code{target} region.

\begin{ccppspecific}

If a list item in a \code{map} clause has a base pointer and it is a
scalar variable with a predetermined data-sharing attribute of firstprivate
(see \specref{subsubsec:Data-sharing Attribute Rules for Variables Referenced
in a Construct}), then on entry to the \code{target} region:

\begin{itemize}
\item If the list item is not a zero-length array section, the corresponding
private variable is initialized such that the corresponding list item in the
device data environment can be accessed through the pointer in the
\code{target} region.

\item If the list item is a zero-length array section, the corresponding
private variable is initialized such that the corresponding storage location
of the array section can be referenced through the pointer in the \code{target}
region. If the corresponding storage location is not present in the device
data environment, the corresponding private variable is initialized to NULL.
\end{itemize}

\end{ccppspecific}

\events

Events associated with a \plc{target task} are the same as for the \code{task} 
construct defined in \specref{subsec:task Construct}.

The \plc{target-begin} event occurs when a thread enters a \code{target} region.

The \plc{target-end} event occurs when a thread exits a \code{target} region.

The \plc{target-submit} event occurs prior to creating an initial task on 
a target device for a \code{target} region.

\tools

Callbacks associated with events for \plc{target tasks} are the same as
for the \code{task} construct defined in \specref{subsec:task Construct}.

A thread dispatches a registered \code{ompt_callback_target} callback with 
\code{ompt_scope_begin} as its \plc{endpoint} argument and \code{ompt_target} as 
its \plc{kind} argument for each occurrence of a \plc{target-begin} event in that 
thread in the context of the target task on the host. Similarly, a thread dispatches 
a registered \code{ompt_callback_target} callback with \code{ompt_scope_end} as its 
\plc{endpoint} argument and \code{ompt_target} as its \plc{kind} argument for each 
occurrence of a \plc{target-end} event in that thread in the context of the target 
task on the host. These callbacks have type signature \code{ompt_callback_target_t}. 

A thread dispatches a registered \code{ompt_callback_target_submit} callback for 
each occurrence of a \plc{target-submit} event in that thread. The callback has 
type signature \code{ompt_callback_target_submit_t}.

\restrictions
\begin{itemize}
\item If a \code{target}~\code{update}, \code{target}~\code{data}, 
      \code{target}~\code{enter}~\code{data}, or 
      \code{target}~\code{exit}~\code{data} construct is encountered during
      execution of a \code{target} region, the behavior is unspecified.
\item The result of an \code{omp_set_default_device}, \code{omp_get_default_device}, 
      or \code{omp_get_num_devices} routine called within a \code{target} region 
      is unspecified.
\item The effect of an access to a \code{threadprivate} variable in a target region 
      is unspecified.
\item If a list item in a \code{map} clause is a structure element, any other
      element of that structure that is referenced in the \code{target} construct
      must also appear as a list item in a \code{map} clause.
\item A variable referenced in a \code{target} region but not the \code{target} 
      construct that is not declared in the \code{target} region must appear in 
      a \code{declare}~\code{target} directive.
\item At most one \code{defaultmap} clause for each category can appear on the 
      directive.
\item At most one \code{nowait} clause can appear on the directive.
\item A \plc{map-type} in a \code{map} clause must be \code{to}, \code{from}, 
      \code{tofrom} or \code{alloc}.
\item A list item that appears in an \code{is_device_ptr} clause must be a valid 
      device pointer in the device data environment.
\item At most one \code{device} clause can appear on the directive. The \code{device} 
      clause expression must evaluate to a non-negative integer value less than the 
      value of \code{omp_get_num_devices()} or to the value of 
      \code{omp_get_initial_device()}.
\item If a \code{device} clause in which the \code{ancestor} \plc{device-modifier} 
      appears is present on the construct, then the following restrictions apply:

\begin{itemize}
\item A \code{requires} directive with the \code{reverse_offload} clause must 
      be specified;
\item The \code{device} clause expression must evaluate to 1;
\item Only the \code{device}, \code{firstprivate}, \code{private},
      \code{defaultmap}, and \code{map} clauses may appear on the construct.
\item No OpenMP constructs or calls to OpenMP API runtime routines are allowed
      inside the corresponding \code{target} region.
\end{itemize}

\item Memory allocators that do not appear in a \code{uses_allocators} clause 
      cannot appear as an allocator in an \code{allocate} clause or be used in 
      the \code{target} region unless a \code{requires} directive with the 
      \code{dynamic_allocators} clause is present in the same compilation unit.
\item Memory allocators that appear in a \code{uses_allocators} clause cannot 
      appear in other data-sharing attribute clauses or data-mapping attribute 
      clauses in the same construct.
\item Predefined allocators appearing in a \code{uses_allocators} clause cannot 
      have \plc{traits} specified.
\item Non-predefined allocators appearing in a \code{uses_allocators} clause must 
      have \plc{traits} specified.
\item Arrays that contain allocators traits that appear in a \code{uses_allocators} 
      clause must be constant arrays, have constant values and be defined in the 
      same scope as the construct in which the clause appears. 
\item Any IEEE floating-point exception status flag, halting mode, or rounding mode 
      set prior to a \code{target} region is unspecified in the region.
\item Any IEEE floating-point exception status flag, halting mode, or rounding mode 
      set in a \code{target} region is unspecified upon exiting the region.

\begin{ccppspecific}
\item An attached pointer may not be modified in a \code{target} region.
\end{ccppspecific}

\begin{cspecific}
\item A list item that appears in an \code{is_device_ptr} clause must have a type 
      of pointer or array.
\end{cspecific}

\begin{cppspecific}
\item A list item that appears in an \code{is_device_ptr} clause must have a type
      of pointer, array, reference to pointer or reference to array.
\item The effect of invoking a virtual member function of an object on a device 
      other than the device on which the object was constructed is implementation 
      defined.
\item A throw executed inside a \code{target} region must cause execution to resume 
      within the same \code{target} region, and the same thread that threw the 
      exception must catch it.
\end{cppspecific}

\begin{fortranspecific}
\item A list item that appears in an \code{is_device_ptr} clause must be a dummy 
      argument that does not have the \code{ALLOCATABLE}, \code{POINTER} or 
      \code{VALUE} attribute.
\item If a list item in a \code{map} clause is an array section, and the array 
      section is derived from a variable with a \code{POINTER} or \code{ALLOCATABLE} 
      attribute then the behavior is unspecified if the corresponding list item's 
      variable is modified in the region.
\end{fortranspecific}
\end{itemize}

\crossreferences
\begin{itemize}
\item \plc{default-device-var}, see
\specref{sec:Internal Control Variables}.

\item \code{task} construct, see
\specref{subsec:task Construct}.

\item \code{task} scheduling constraints, see
\specref{subsec:Task Scheduling}

\item Memory allocators, see \specref{subsec:Memory Allocators}.

\item \code{target}~\code{data} construct, see
\specref{subsec:target data Construct}.

\item \code{if} Clause, see \specref{sec:if Clause}.

\item \code{private} and \code{firstprivate} clauses, see
\specref{subsec:Data-Sharing Attribute Clauses}.

\item Data-Mapping Attribute Rules and Clauses, see
\specref{subsec:Data-Mapping Attribute Rules, Clauses, and Directives}.

\item \code{omp_get_num_devices} routine, see \specref{subsec:omp_get_num_devices}.

\item \code{omp_alloctrait_t} and \code{omp_alloctrait} types, 
see \specref{subsec:Memory Management Types}.

\item \code{omp_set_default_allocator} routine, see 
\specref{subsec:omp_set_default_allocator}.

\item \code{omp_get_default_allocator} routine, see 
\specref{subsec:omp_get_default_allocator}.

\item \code{ompt_callback_target_t}, see
\specref{sec:ompt_callback_target_t}.

\item \code{ompt_callback_target_submit_t},
\specref{sec:ompt_callback_target_submit_t}.

\end{itemize}



\subsection{\hcode{target}~\hcode{update} Construct}
\index{target update@{\code{target}~\code{update}}}
\index{constructs!target update@{\code{target}~\code{update}}}
\index{device constructs!target update@{\code{target}~\code{update}}}
\label{subsec:target update Construct}
\summary
The \code{target}~\code{update} directive makes the corresponding list items in 
the device data environment consistent with their original list items, according 
to the specified motion clauses. The \code{target}~\code{update} construct is a 
stand-alone directive.

\syntax
\begin{ccppspecific}
The syntax of the \code{target}~\code{update} construct is as follows:

\begin{ompcPragma}
#pragma omp target update \plc{clause[ [ [},\plc{] clause] ... ] new-line}
\end{ompcPragma}
where \plc{clause} is either \plc{motion-clause} or one of the following:

\begin{indentedcodelist}
if(\plc{[} target update :\plc{] scalar-expression})
device(\plc{integer-expression})
nowait
depend(\plc{[depend-modifier},\plc{] dependence-type }:\plc{ locator-list})
\end{indentedcodelist}

and \plc{motion-clause} is one of the following:

\begin{indentedcodelist}
to([mapper(\plc{mapper-identifier}):]\plc{locator-list})
from([mapper(\plc{mapper-identifier}):]\plc{locator-list})
\end{indentedcodelist}
\end{ccppspecific}

\begin{fortranspecific}
The syntax of the \code{target}~\code{update} construct is as follows:

\begin{ompfPragma}
!$omp target update \plc{clause[ [ [},\plc{] clause] ... ]}
\end{ompfPragma}

where \plc{clause} is either \plc{motion-clause} or one of the following:

\begin{indentedcodelist}
if(\plc{[}target update :\plc{] scalar-logical-expression})
device(\plc{scalar-integer-expression})
nowait
depend(\plc{[depend-modifier},\plc{] dependence-type }:\plc{ locator-list})
\end{indentedcodelist}

and \plc{motion-clause} is one of the following:

\begin{indentedcodelist}
to([mapper(\plc{mapper-identifier}):]\plc{locator-list})
from([mapper(\plc{mapper-identifier}):]\plc{locator-list})
\end{indentedcodelist}
\end{fortranspecific}

\binding
The binding task set for a \code{target}~\code{update} region is the
generating task, which is the \plc{target task} generated
by the \code{target}~\code{update} construct. The \code{target}~\code{update}
region binds to the corresponding \plc{target task} region.

\descr
For each list item in a \code{to} or \code{from} clause there is a corresponding 
list item and an original list item. If the corresponding list item is not present 
in the device data environment then no assignment occurs to or from the original 
list item. Otherwise, each corresponding list item in the device data environment 
has an original list item in the current task's data environment.  If a 
\code{mapper()} modifier appears in a \code{to} clause, each list item is replaced 
with the list items that the given mapper specifies are to be mapped
with a \code{to} or \code{tofrom} map-type. If a \code{mapper()} modifier
appears in a \code{from} clause, each list item is replaced with the list items
that the given mapper specifies are to be mapped with a \code{from} or
\code{tofrom} map-type.

For each list item in a \code{from} or a \code{to} clause: 

\begin{itemize}
\item For each part of the list item that is an attached pointer:  

\begin{itemize}
\item On exit from the region that part of the original list item will
      have the value it had on entry to the region;
\item On exit from the region that part of the corresponding list item will
      have the value it had on entry to the region;
\end{itemize}

\item For each part of the list item that is not an attached pointer: 

\begin{itemize}
\item If the clause is \code{from}, the value of that part of the corresponding 
      list item is assigned to that part of the original list item;
\item If the clause is \code{to}, the value of that part of the original list 
      item is assigned to that part of the corresponding list item.
\end{itemize}

\item To avoid race conditions: 

\begin{itemize}
\item Concurrent reads or updates of any part of the original list item must be 
      synchronized with the update of the original list item that occurs as a 
      result of the \code{from} clause;
\item Concurrent reads or updates of any part of the corresponding list item must 
      be synchronized with the update of the corresponding list item that occurs 
      as a result of the \code{to} clause.
\end{itemize}

\end{itemize}


\begin{ccppspecific}
The list items that appear in the \code{to} or \code{from} clauses may
use shape-operators.
\end{ccppspecific}

The list items that appear in the \code{to} or \code{from} clauses may
include array sections with \plc{stride} expressions.

The \code{target}~\code{update} construct is a task generating construct.  
The generated task is a \plc{target task}.  The generated task region encloses 
the \code{target}~\code{update} region.

All clauses are evaluated when the \code{target}~\code{update} construct is 
encountered.  The data environment of the \plc{target task} is created according 
to the data-sharing attribute clauses on the \code{target}~\code{update} construct, 
per-data environment ICVs, and any default data-sharing attribute rules that apply 
to the \code{target}~\code{update} construct.  A variable that is mapped in the 
\code{target}~\code{update} construct has a default data-sharing attribute of 
shared in the data environment of the \plc{target task}.

Assignment operations associated with mapping a variable (see 
\specref{subsec:map Clause}) occur when the \plc{target task} executes.

If the \code{nowait} clause is present, execution of the \plc{target task} may 
be deferred.  If the \code{nowait} clause is not present, the \plc{target task} 
is an included task.

If a \code{depend} clause is present, it is associated with the \plc{target task}.

The device is specified in the \code{device} clause. If there is no \code{device} 
clause, the device is determined by the \plc{default-device-var} ICV. When an 
\code{if} clause is present and the \code{if} clause expression evaluates to 
\plc{false} then no assignments occur.

\events

Events associated with a \plc{target task} are the same as for the \code{task} 
construct defined in \specref{subsec:task Construct}.

The \plc{target-update-begin} event occurs when a thread enters a
\code{target}~\code{update} region.

The \plc{target-update-end} event occurs when a thread exits a
\code{target}~\code{update} region.

\tools

Callbacks associated with events for \plc{target tasks} are the same as
for the \code{task} construct defined in \specref{subsec:task Construct}.

A thread dispatches a registered \code{ompt_callback_target} callback with 
\code{ompt_scope_begin} as its \plc{endpoint} argument and 
\code{ompt_target_update} as its \plc{kind} argument for each occurrence 
of a \plc{target-update-begin} event in that thread in the context of the 
target task on the host. Similarly, a thread dispatches a registered 
\code{ompt_callback_target} callback with \code{ompt_scope_end} as its 
\plc{endpoint} argument and \code{ompt_target_update} as its \plc{kind} 
argument for each occurrence of a \plc{target-update-end} event in that thread 
in the context of the target task on the host. These callbacks have 
type signature \code{ompt_callback_target_t}. 



\restrictions
\begin{itemize}
\item A program must not depend on any ordering of the evaluations of the 
      clauses of the \code{target}~\code{update} directive, or on any side 
      effects of the evaluations of the clauses.
\item At least one \plc{motion-clause} must be specified.
\item A list item can only appear in a \code{to} or \code{from} clause, but not both.
\item A list item in a \code{to} or \code{from} clause must have a mappable type.
\item At most one \code{device} clause can appear on the directive. The
      \code{device} clause expression must evaluate to a non-negative integer 
      value less than the value of \code{omp_get_num_devices()} or to the value of 
      \code{omp_get_initial_device()}.
\item At most one \code{if} clause can appear on the directive.
\item At most one \code{nowait} clause can appear on the directive.
\end{itemize}

\crossreferences
\begin{itemize}
\item \plc{default-device-var}, see
\specref{sec:Internal Control Variables}.

\item Array shaping,
\specref{sec:Array Shaping}

\item Array sections,
\specref{sec:Array Sections}

\item \code{task} construct, see
\specref{subsec:task Construct}.

\item \code{task} scheduling constraints, see
\specref{subsec:Task Scheduling}

\item \code{target}~\code{data}, see
\specref{subsec:target data Construct}.

\item \code{if} Clause, see \specref{sec:if Clause}.

\item \code{omp_get_num_devices} routine, see \specref{subsec:omp_get_num_devices}.

\item \code{ompt_callback_task_create_t}, see
\specref{sec:ompt_callback_task_create_t}.

\item \code{ompt_callback_target_t}, see
\specref{sec:ompt_callback_target_t}.
\end{itemize}



\subsection{\hcode{declare}~\hcode{target} Directive}
\index{declare target@{\code{declare}~\code{target}}}
\index{directives!declare target@{\code{declare}~\code{target}}}
\index{constructs!declare target@{\code{declare}~\code{target}}}
\index{device constructs!declare target@{\code{declare}~\code{target}}}
\label{subsec:declare target Directive}
\summary
The \code{declare}~\code{target} directive specifies that variables,
functions (C, C++ and Fortran), and subroutines (Fortran) are mapped
to a device. The \code{declare}~\code{target} directive is a declarative
directive.

\syntax
\begin{ccppspecific}
The syntax of the \code{declare}~\code{target} directive takes either of
the following forms:

\begin{ompcPragma}
#pragma omp declare target \plc{new-line}
\plc{declaration-definition-seq}
#pragma omp end declare target \plc{new-line}
\end{ompcPragma}

or

\begin{ompcPragma}
#pragma omp declare target (\plc{extended-list}) \plc{new-line}
\end{ompcPragma}

or

\begin{ompcPragma}
#pragma omp declare target \plc{clause[ [},\plc{] clause ... ] new-line}
\end{ompcPragma}

where \plc{clause} is one of the following:

\begin{indentedcodelist}
to(\plc{extended-list})
link(\plc{list})
implements(\plc{function-name})
device_type(host \textnormal{| nohost \textnormal{|} any})
\end{indentedcodelist}
\end{ccppspecific}

\begin{fortranspecific}
The syntax of the \code{declare}~\code{target} directive is as follows:

\begin{ompfPragma}
!$omp declare target (\plc{extended-list})
\end{ompfPragma}

or

\begin{ompfPragma}
!$omp declare target \plc{[clause[ [},\plc{] clause] ... ]}
\end{ompfPragma}

where \plc{clause} is one of the following:

\begin{indentedcodelist}
to(\plc{extended-list})
link(\plc{list})
implements(\plc{subroutine-name})
device_type(host \textnormal{| nohost \textnormal{|} any})
\end{indentedcodelist}
\end{fortranspecific}

\descr

The \code{declare} \code{target} directive ensures that procedures
and global variables can be executed or accessed on a device.
Variables are mapped for all device executions, or for specific
device executions through a \code{link} clause.

If an \plc{extended-list} is present with no clause then the \code{to}
clause is assumed.

The \code{implements} clause specifies that an alternate version of a 
procedure should be used.

The \code{device_type} clause specifies if a version of the procedure should be made
available on host, device or both. If \code{host} is specified only a host version
of the procedure is made available.  If \code{nohost} is specified then only a device
version of the procedure is made available.  If \code{any} is specified then both
device and host versions of the procedure are made available.

\begin{ccppspecific}
If a function appears in a \code{to} clause in the same translation unit in which 
the definition of the function occurs then a device-specific version of the function 
is created.

If a variable appears in a \code{to} clause in the same translation unit in which 
the definition of the variable occurs then the original list item is allocated a 
corresponding list item in the device data environment of all devices.

All calls in \code{target} constructs to the function in the \code{implements} clause
are replaced by the function following the \code{declare} \code{target} constructs.
\end{ccppspecific}

\begin{fortranspecific}
If an internal procedure appears in a \code{to} clause
then a device-specific version of the procedure is created.

If a variable that is host associated appears in a \code{to} clause
then the original list item is allocated a corresponding list item in the
device data environment of all devices.

All calls in \code{target} constructs to the procedure in the \code{implements}
clause are replaced by the procedure in which \code{declare} \code{target} 
construct appeared.
\end{fortranspecific}

If a variable appears in a \code{to} clause then the corresponding list
item in the device data environment of each device is initialized once, in the
manner specified by the program, but at an unspecified point in the program
prior to the first reference to that list item.  The list item is never removed
from those device data environments as if its reference count is initialized to
positive infinity.

Including list items in a \code{link} clause supports compilation of
functions called in a \code{target} region that refer to the list
items.  The list items are not mapped by the \code{declare}~\code{target}
directive.  Instead, they are mapped according to the data mapping
rules described in 
\specref{subsec:Data-Mapping Attribute Rules, Clauses, and Directives}.

\begin{ccppspecific}
If a function is referenced in a function that appears as a list item in a \code{to}
clause on a \code{declare}~\code{target} directive then the name of the referenced 
function is treated as if it had appeared in a \code{to} clause on a declare target 
directive.

If a variable with static storage duration or a function (except \plc{lambda}
for C++) is referenced in the initializer expression list of a variable with
static storage duration that appears as a list item in a \code{to} clause on a 
\code{declare}~\code{target} directive then the name of the referenced variable 
or function is treated as if it had appeared in a \code{to} clause on a 
\code{declare}~\code{target} directive.

The form of the \code{declare}~\code{target} directive that has no clauses
and requires a matching \code{end}~\code{declare}~\code{target} directive
defines an implicit \plc{extended-list} to an implicit \code{to}
clause. The implicit \plc{extended-list} consists of the variable names
of any variable declarations at file or namespace scope that appear between
the two directives and of the function names of any function declarations at
file, namespace or class scope that appear between the two directives.

The \plc{declaration-definition-seq} defined by a
\code{declare}~\code{target} directive and an
\code{end}~\code{declare}~\code{target} directive may contain
\code{declare}~\code{target} directives. If a \code{device_type} clause
is present on the contained \code{declare}~\code{target} directive, then its
argument determines which versions are made available.
If a list item appears both in an implicit and explicit list, the explicit
list determines which versions are made available.

\end{ccppspecific}

\begin{fortranspecific}
If a procedure is referenced in a procedure that appears as a list item in a \code{to}
clause on a \code{declare}~\code{target} directive then the name of the procedure is 
treated as if it had appeared in a \code{to} clause on a \code{declare}~\code{target}
directive.

If a \code{declare}~\code{target} does not have any clauses then an implicit
\plc{extended-list} to an implicit \code{to} clause of one item is formed from
the name of the enclosing subroutine subprogram, function subprogram or
interface body to which it applies.

If a \code{declare}~\code{target} directive has an \code{implements} or 
\code{device_type} clause then any enclosed internal procedures cannot contain 
any \code{declare}~\code{target} directives. The enclosing \code{device_type} 
clause implicitly applies to internal procedures. 
\end{fortranspecific}


\restrictions
\begin{itemize}
\item A threadprivate variable cannot appear in a
      \code{declare}~\code{target} directive.
\item A variable declared in a \code{declare}~\code{target} directive
      must have a mappable type.
\item The same list item must not appear multiple times in clauses on 
      the same directive.
\item The same list item must not explicitly appear in both a \code{to} clause on one
      \code{declare}~\code{target} directive and a \code{link} clause on
      another \code{declare}~\code{target} directive.
\item A \code{device_type} clause must be specified if an \code{implements} clause 
      is specified. 
\end{itemize}

\begin{cppspecific}
\begin{itemize}
\item The function names of overloaded functions or template functions
      may only be specified within an implicit \plc{extended-list}.
\item If a \plc{lambda declaration and definition} appears between a
      \code{declare target} directive and the matching \code{end declare target}
      directive, all the variables that are captured by the \plc{lambda} expression
      must also appear in a \code{to} clause.
\end{itemize}
\end{cppspecific}

\begin{fortranspecific}
\begin{itemize}
\item If a list item is a procedure name, it must not be a generic name,
      procedure pointer or entry name.
\item Any \code{declare}~\code{target} directive with clauses must appear
      in a specification part of a subroutine subprogram, function subprogram,
      program or module.
\item Any \code{declare}~\code{target} directive without clauses must appear
      in a specification part of a subroutine subprogram, function subprogram
      or interface body to which it applies.
\item If a \code{declare}~\code{target} directive is specified in an
      interface block for a procedure, it must match a
      \code{declare}~\code{target} directive in the definition of the
      procedure.
\item If an external procedure is a type-bound procedure of a derived
      type and a \code{declare}~\code{target} directive is specified in
      the definition of the external procedure, such a directive must
      appear in the interface block that is accessible to the derived
      type definition.
\item If any procedure is declared via a procedure declaration statement
      that is not in the type-bound procedure part of a derived-type
      definition, any \code{declare}~\code{target} with the lure
      name must appear in the same specification part.
\item A variable that is part of another variable (as an array, structure
      element or type parameter inquiry) cannot appear in a
      \code{declare}~\code{target} directive.
\item The \code{declare}~\code{target} directive must appear in the
      declaration section of a scoping unit in which the common block
      or variable is declared. Although variables in common blocks can
      be accessed by use association or host association, common block
      names cannot. This means that a common block name specified in a
      \code{declare}~\code{target} directive must be declared to be a
      common block in the same scoping unit in which the
      \code{declare}~\code{target} directive appears.
\item If a \code{declare}~\code{target} directive specifying a common
      block name appears in one program unit, then such a directive must
      also appear in every other program unit that contains a \code{COMMON}
      statement specifying the same name. It must appear after the last
      such \code{COMMON} statement in the program unit.
\item If a list item is declared with the \code{BIND} attribute, the
      corresponding C entities must also be specified in a
      \code{declare}~\code{target} directive in the C program.
\item A blank common block cannot appear in a \code{declare}~\code{target}
      directive.
\item A variable can only appear in a \code{declare}~\code{target} directive
      in the scope in which it is declared. It must not be an element of a
      common block or appear in an \code{EQUIVALENCE} statement.
\item A variable that appears in a \code{declare}~\code{target} directive
      must be declared in the Fortran scope of a module or have the
      \code{SAVE} attribute, either explicitly or implicitly.
\end{itemize}
\end{fortranspecific}

% This is an included file. See the master file for more information.
%
% When editing this file:
%
%    1. To change formatting, appearance, or style, please edit openmp.sty.
%
%    2. Custom commands and macros are defined in openmp.sty.
%
%    3. Be kind to other editors -- keep a consistent style by copying-and-pasting to
%       create new content.
%
%    4. We use semantic markup, e.g. (see openmp.sty for a full list):
%         \code{}     % for bold monospace keywords, code, operators, etc.
%         \plc{}      % for italic placeholder names, grammar, etc.
%
%    5. There are environments that provide special formatting, e.g. language bars.
%       Please use them whereever appropriate.  Examples are:
%
%         \begin{fortranspecific}
%         This is text that appears enclosed in blue language bars for Fortran.
%         \end{fortranspecific}
%
%         \begin{note}
%         This is a note.  The "Note -- " header appears automatically.
%         \end{note}
%
%    6. Other recommendations:
%         Use the convenience macros defined in openmp.sty for the minor headers
%         such as Comments, Syntax, etc.
%
%         To keep items together on the same page, prefer the use of
%         \begin{samepage}.... Avoid \parbox for text blocks as it interrupts line numbering.
%         When possible, avoid \filbreak, \pagebreak, \newpage, \clearpage unless that's
%         what you mean. Use \needspace{} cautiously for troublesome paragraphs.
%
%         Avoid absolute lengths and measures in this file; use relative units when possible.
%         Vertical space can be relative to \baselineskip or ex units. Horizontal space
%         can be relative to \linewidth or em units.
%
%         Prefer \emph{} to italicize terminology, e.g.:
%             This is a \emph{definition}, not a placeholder.
%             This is a \plc{var-name}.
%


\section{Combined Constructs}
\label{sec:Combined Constructs}
\index{combined constructs}
\index{constructs!combined constructs}
Combined constructs are shortcuts for specifying one construct immediately nested
inside another construct. The semantics of the combined constructs are identical to that
of explicitly specifying the first construct containing one instance of the second
construct and no other statements.

For combined constructs, tool callbacks shall be invoked as if the constructs were
explicitly nested.






\subsection{Parallel Worksharing-Loop Construct}
\label{subsec:Parallel Worksharing-Loop Construct}
\index{parallel worksharing-loop construct}
\index{constructs!parallel worksharing-loop construct}
\index{constructs!parallel for@{\code{parallel}~\code{for} \emph{C/C++}}}
\index{constructs!parallel do@{\code{parallel}~\code{do} \emph{Fortran}}}
\index{combined constructs!parallel worksharing-loop construct}
\index{worksharing!parallel}
\summary
The parallel worksharing-loop construct is a shortcut for specifying a \code{parallel} construct
containing one worksharing-loop construct with one or more associated loops and no other statements.



\syntax
\begin{ccppspecific}
The syntax of the parallel worksharing-loop construct is as follows:

\begin{ompcPragma}
#pragma omp parallel for \plc{[clause[ [},\plc{] clause] ... ] new-line}
   \plc{for-loops}
\end{ompcPragma}

where \plc{clause} can be any of the clauses accepted by the \code{parallel} or \code{for} directives,
except the \code{nowait} clause, with identical meanings and restrictions.
\end{ccppspecific}

\begin{fortranspecific}
The syntax of the parallel worksharing-loop construct is as follows:

\begin{ompfPragma}
!$omp parallel do \plc{[clause[ [},\plc{] clause] ... ]}
   \plc{do-loops}
\plc{[}!$omp end parallel do\plc{]}
\end{ompfPragma}

where \plc{clause} can be any of the clauses accepted by the \code{parallel} or \code{do} directives,
with identical meanings and restrictions.

If an \code{end}~\code{parallel}~\code{do} directive is not specified, an \code{end}~\code{parallel}~\code{do} directive is
assumed at the end of the \plc{do-loops}. \code{nowait} may not be specified on an
\code{end}~\code{parallel}~\code{do} directive.
\end{fortranspecific}

\descr
The semantics are identical to explicitly specifying a \code{parallel} directive immediately
followed by a worksharing-loop directive.

\restrictions
\begin{itemize}
\item The restrictions for the \code{parallel} construct and the
    worksharing-loop construct apply.
\end{itemize}

\crossreferences
\begin{itemize}
\item \code{parallel} construct, see
\specref{sec:parallel Construct}.

\item worksharing-loop SIMD construct, see
\specref{subsubsec:Worksharing-Loop SIMD Construct}.

\item Data attribute clauses, see
\specref{subsec:Data-Sharing Attribute Clauses}.
\end{itemize}




\subsection{Parallel Loop Construct}
\label{subsec:Parallel Loop Construct}
\index{parallel loop construct}
\index{constructs!parallel loop construct}
\index{combined constructs!parallel loop construct}
\index{worksharing!parallel}
\summary
The parallel loop construct is a shortcut for specifying a \code{parallel} construct
containing one \code{loop} construct with one or more associated loops and no other statements.



\syntax
\begin{ccppspecific}
The syntax of the parallel loop construct is as follows:

\begin{ompcPragma}
#pragma omp parallel loop \plc{[clause[ [},\plc{] clause] ... ] new-line}
   \plc{for-loops}
\end{ompcPragma}

where \plc{clause} can be any of the clauses accepted by the \code{parallel} or
  \code{loop} directives, with identical meanings and restrictions.
\end{ccppspecific}

\begin{fortranspecific}
The syntax of the parallel loop construct is as follows:

\begin{ompfPragma}
!$omp parallel loop \plc{[clause[ [},\plc{] clause] ... ]}
   \plc{do-loops}
\plc{[}!$omp end parallel loop\plc{]}
\end{ompfPragma}

where \plc{clause} can be any of the clauses accepted by the \code{parallel} or
  \code{loop} directives, with identical meanings and restrictions.

If an \code{end}~\code{parallel}~\code{loop} directive is not specified, an
  \code{end}~\code{parallel}~\code{loop} directive is assumed at the end of the
  \plc{do-loops}. \code{nowait} may not be specified on an
  \code{end}~\code{parallel}~\code{loop} directive.
\end{fortranspecific}

\descr
The semantics are identical to explicitly specifying a \code{parallel} directive immediately
followed by a \code{loop} directive. 


\restrictions
\begin{itemize}
\item The restrictions for the \code{parallel} construct and the
  \code{loop} construct apply.
\end{itemize}

\crossreferences
\begin{itemize}
\item \code{parallel} construct, see
\specref{sec:parallel Construct}.

\item \code{loop} construct, see
\specref{subsec:loop Construct}.

\item Data attribute clauses, see
\specref{subsec:Data-Sharing Attribute Clauses}.
\end{itemize}




\subsection{\hcode{parallel}~\hcode{sections} Construct}
\index{parallel sections@{\code{parallel}~\code{sections}}}
\index{constructs!parallel sections@{\code{parallel}~\code{sections}}}
\index{combined constructs!parallel sections@{\code{parallel}~\code{sections}}}
\label{subsec:parallel sections Construct}
\summary
The \code{parallel}~\code{sections} construct is a shortcut for specifying a \code{parallel}
construct containing one \code{sections} construct and no other statements.

\syntax
\begin{ccppspecific}
The syntax of the \code{parallel}~\code{sections} construct is as follows:

\begin{ompcPragma}
#pragma omp parallel sections \plc{[clause[ [},\plc{] clause] ... ] new-line}
    {
    \plc{[}#pragma omp section \plc{new-line]}
        \plc{structured-block}
    \plc{[}#pragma omp section \plc{new-line}
        \plc{structured-block]}
    \plc{...}
    }
\end{ompcPragma}

where \plc{clause} can be any of the clauses accepted by the \code{parallel} or \code{sections}
directives, except the \code{nowait} clause, with identical meanings and restrictions.
\end{ccppspecific}

\begin{fortranspecific}
The syntax of the \code{parallel}~\code{sections} construct is as follows:

\begin{ompfPragma}
!$omp parallel sections \plc{[clause[ [},\plc{] clause] ... ]}
    \plc{[}!$omp section\plc{]}
        \plc{structured-block}
    \plc{[}!$omp section
        \plc{structured-block]}
    \plc{...}
!$omp end parallel sections
\end{ompfPragma}

where \plc{clause} can be any of the clauses accepted by the \code{parallel} or \code{sections}
directives, with identical meanings and restrictions.

The last section ends at the \code{end}~\code{parallel}~\code{sections} directive. \code{nowait} cannot be
specified on an \code{end}~\code{parallel}~\code{sections} directive.
\end{fortranspecific}

\descr
\begin{ccppspecific}
The semantics are identical to explicitly specifying a \code{parallel} directive immediately
followed by a \code{sections} directive.
\end{ccppspecific}

\begin{fortranspecific}
The semantics are identical to explicitly specifying a \code{parallel} directive immediately
followed by a \code{sections} directive, and an \code{end}~\code{sections} directive immediately
followed by an \code{end}~\code{parallel} directive.
\end{fortranspecific}

\restrictions
The restrictions for the \code{parallel} construct and the \code{sections} construct apply.

\crossreferences
\begin{itemize}
\item \code{parallel} construct, see
\specref{sec:parallel Construct}.

\item \code{sections} construct, see
\specref{subsec:sections Construct}.

\item Data attribute clauses, see
\specref{subsec:Data-Sharing Attribute Clauses}.
\end{itemize}









% Here we need to force the blue floater down lower and force the subsection
% header higher to reduce the space between the blue floater and the header,
% as per Richard:

\begin{fortranspecific}

\subsection{\hcode{parallel}~\hcode{workshare} Construct}
\index{parallel workshare@{\code{parallel}~\code{workshare}}}
\index{constructs!parallel workshare@{\code{parallel}~\code{workshare}}}
\index{combined constructs!parallel workshare@{\code{parallel}~\code{workshare}}}
\label{subsec:parallel workshare Construct}
\summary
The \code{parallel}~\code{workshare} construct is a shortcut for specifying a \code{parallel}
construct containing one \code{workshare} construct and no other statements.

\syntax
The syntax of the \code{parallel}~\code{workshare} construct is as follows:

\begin{ompfPragma}
!$omp parallel workshare \plc{[clause[ [},\plc{] clause] ... ]}
   \plc{structured-block }
!$omp end parallel workshare
\end{ompfPragma}

where \plc{clause} can be any of the clauses accepted by the \code{parallel} directive, with
identical meanings and restrictions. \code{nowait} may not be specified on an
\code{end}~\code{parallel}~\code{workshare} directive.

\descr
The semantics are identical to explicitly specifying a \code{parallel} directive immediately
followed by a \code{workshare} directive, and an \code{end}~\code{workshare} directive immediately
followed by an \code{end}~\code{parallel} directive.

\restrictions
The restrictions for the \code{parallel} construct and the \code{workshare} construct apply.

\crossreferences
\begin{itemize}
\item \code{parallel} construct, see
\specref{sec:parallel Construct}.

\item \code{workshare} construct, see
\specref{subsec:workshare Construct}.

\item Data attribute clauses, see
\specref{subsec:Data-Sharing Attribute Clauses}.
\end{itemize}
\end{fortranspecific}










\subsection{Parallel Worksharing-Loop SIMD Construct}
\label{subsec:Parallel Worksharing-Loop SIMD Construct}
\index{parallel worksharing-loop SIMD construct}
\index{constructs!parallel worksharing-loop SIMD construct}
\index{combined constructs!parallel worksharing-loop SIMD construct}
\summary
The parallel worksharing-loop SIMD construct is a shortcut for specifying a \code{parallel} construct
containing one worksharing-loop SIMD construct and no other statement.

\begin{samepage}
\syntax
\begin{ccppspecific}
The syntax of the parallel worksharing-loop SIMD construct is as follows:

\begin{ompcPragma}
#pragma omp parallel for simd \plc{[clause[ [},\plc{] clause] ... ] new-line}
    \plc{for-loops}
\end{ompcPragma}

where \plc{clause} can be any of the clauses accepted by the \code{parallel}
or \code{for}~\code{simd} directives, except the \code{nowait} clause, with
identical meanings and restrictions.
\end{ccppspecific}
\end{samepage}

\begin{fortranspecific}
\begin{samepage}
The syntax of the parallel worksharing-loop SIMD construct is as follows:

\begin{ompfPragma}
!$omp parallel do simd \plc{[clause[ [},\plc{] clause] ... ]}
    \plc{do-loops}
\plc{[}!$omp end parallel do simd\plc{]}
\end{ompfPragma}
\end{samepage}

where \plc{clause} can be any of the clauses accepted by the \code{parallel}
or \code{do}~\code{simd} directives, with identical meanings and restrictions.

\begin{samepage}
If an \code{end}~\code{parallel}~\code{do}~\code{simd} directive is not specified, an
\code{end}~\code{parallel}~\code{do}~\code{simd} directive is assumed at the end of the
\plc{do-loops}. \code{nowait} may not be specified on
an \code{end}~\code{parallel}~\code{do}~\code{simd} directive.
\end{samepage}
\end{fortranspecific}

\descr
The semantics of the parallel worksharing-loop SIMD construct are identical to explicitly specifying
a \code{parallel} directive immediately followed by a worksharing-loop SIMD directive.

\restrictions
The restrictions for the \code{parallel} construct and the worksharing-loop SIMD construct apply.

\crossreferences
\begin{itemize}
\item \code{parallel} construct, see
\specref{sec:parallel Construct}.

\item worksharing-loop SIMD construct, see
\specref{subsubsec:Worksharing-Loop SIMD Construct}.

\item Data attribute clauses, see
\specref{subsec:Data-Sharing Attribute Clauses}.
\end{itemize}









\subsection{\hcode{target}~\hcode{parallel} Construct}
\label{subsec:target parallel Construct}
\index{target parallel@{\code{target}~\code{parallel}}}
\index{constructs!target parallel@{\code{target}~\code{parallel}}}
\index{combined constructs!target parallel@{\code{target}~\code{parallel}}}
\summary
The \code{target} \code{parallel} construct is a shortcut for specifying a \code{target}
construct containing a \code{parallel} construct and no other statements.

\syntax
\begin{ccppspecific}
The syntax of the \code{target} \code{parallel} construct is as follows:

\begin{ompcPragma}
#pragma omp target parallel \plc{[clause[ [},\plc{] clause] ... ] new-line}
    \plc{structured-block}
\end{ompcPragma}

where \plc{clause} can be any of the clauses accepted by the \code{target} or
\code{parallel} directives, except for \code{copyin}, with identical meanings and restrictions.
\end{ccppspecific}

\begin{samepage}
\begin{fortranspecific}
The syntax of the \code{target} \code{parallel} construct is as follows:

\begin{ompfPragma}
!$omp target parallel \plc{[clause[ [},\plc{] clause] ... ]}
    \plc{structured-block}
!$omp end target parallel
\end{ompfPragma}

where \plc{clause} can be any of the clauses accepted by the \code{target} or
\code{parallel} directives, except for \code{copyin}, with identical meanings and restrictions.
\end{fortranspecific}
\end{samepage}

\descr
The semantics are identical to explicitly specifying a \code{target} directive
immediately followed by a \code{parallel} directive.

\restrictions

The restrictions for the \code{target} and \code{parallel} constructs apply except for the following explicit modifications:

\begin{itemize}
\item If any \code{if} clause on the directive includes a
      \plc{directive-name-modifier} then all \code{if} clauses
      on the directive must include a \plc{directive-name-modifier}.

\item At most one \code{if} clause without a
      \plc{directive-name-modifier} can appear on the directive.

\item At most one \code{if} clause with the \code{parallel}
      \plc{directive-name-modifier} can appear on the directive.


\item At most one \code{if} clause with the \code{target}
      \plc{directive-name-modifier} can appear on the directive.
\end{itemize}

\crossreferences
\begin{itemize}
\item \code{parallel} construct, see
\specref{sec:parallel Construct}.

\item \code{target} construct, see
\specref{subsec:target Construct}.

\item \code{if} Clause, see \specref{sec:if Clause}.

\item Data attribute clauses, see
\specref{subsec:Data-Sharing Attribute Clauses}.
%% \item Multi-if Clause, see \specref{subsec:Multi-if Clause}.
\end{itemize}









%Similar to Distribute Parallel Worksharing-Loop Construct

\subsection{Target Parallel Worksharing-Loop Construct}
\label{subsec:Target Parallel Worksharing-Loop Construct}
\index{target parallel worksharing-loop construct}
\index{constructs!target parallel worksharing-loop}
\index{constructs!target parallel for@{\code{target}~\code{parallel}~\code{for}}}
\index{constructs!target parallel do@{\code{target}~\code{parallel}~\code{do}}}
\index{combined constructs!target parallel worksharing-loop}
\summary
The target parallel worksharing-loop construct is a shortcut for specifying a \code{target}
construct containing a parallel worksharing-loop construct and no other statements.

\syntax
\begin{ccppspecific}
The syntax of the target parallel worksharing-loop construct is as follows:

\begin{ompcPragma}
#pragma omp target parallel for \plc{[clause[ [},\plc{] clause] ... ] new-line}
    \plc{for-loops}
\end{ompcPragma}

where \plc{clause} can be any of the clauses accepted by the \code{target} or
\code{parallel}~\code{for} directives, except for \code{copyin}, with identical meanings and restrictions.
\end{ccppspecific}

\needspace{6\baselineskip}
\begin{fortranspecific}
The syntax of the target parallel worksharing-loop construct is as follows:

\begin{ompfPragma}
!$omp target parallel do \plc{[clause[ [},\plc{] clause] ... ]}
    \plc{do-loops}
\plc{[}!$omp end target parallel do\plc{]}
\end{ompfPragma}

where \plc{clause} can be any of the clauses accepted by the \code{target} or
\code{parallel}~\code{do} directives, except for \code{copyin}, with identical meanings and restrictions.

If an \code{end}~\code{target}~\code{parallel}~\code{do} directive is not specified, an
\code{end}~\code{target}~\code{parallel}~\code{do} directive is assumed at the end of
the \plc{do-loops}.
\end{fortranspecific}

\descr
The semantics are identical to explicitly specifying a \code{target} directive
immediately followed by a parallel worksharing-loop directive.


\restrictions
The restrictions for the \code{target} and parallel worksharing-loop constructs apply except for the following explicit modifications:

\begin{itemize}
\item If any \code{if} clause on the directive includes a
      \plc{directive-name-modifier} then all \code{if} clauses
      on the directive must include a \plc{directive-name-modifier}.

\item At most one \code{if} clause without a
      \plc{directive-name-modifier} can appear on the directive.

\item At most one \code{if} clause with the \code{parallel}
      \plc{directive-name-modifier} can appear on the directive.


\item At most one \code{if} clause with the \code{target}
      \plc{directive-name-modifier} can appear on the directive.
\end{itemize}

\crossreferences
\begin{itemize}
\item \code{target} construct, see
\specref{subsec:target Construct}.

\item Parallel Worksharing-Loop construct, see
\specref{subsec:Parallel Worksharing-Loop Construct}.

\item \code{if} Clause, see \specref{sec:if Clause}.

\item Data attribute clauses, see
\specref{subsec:Data-Sharing Attribute Clauses}.

%% \item Multi-if Clause, see \specref{subsec:Multi-if Clause}.
\end{itemize}









% Similar to Distribute Parallel Loop SIMD Construct

\subsection{Target Parallel Worksharing-Loop SIMD Construct}
\label{subsec:Target Parallel Worksharing-Loop SIMD Construct}
\index{target parallel worksharing-loop SIMD construct}
\index{constructs!target parallel worksharing-loop SIMD}
\index{constructs!target parallel for simd@{\code{target}~\code{parallel}~\code{for}~\code{simd}}}
\index{constructs!target parallel do simd@{\code{target}~\code{parallel}~\code{do}~\code{simd}}}
\index{combined constructs!target parallel worksharing-loop SIMD}
\summary
The target parallel worksharing-loop SIMD construct is a shortcut for specifying a \code{target}
construct containing a parallel worksharing-loop SIMD construct and no other statements.

\syntax
\begin{ccppspecific}
The syntax of the target parallel worksharing-loop SIMD construct is as follows:

\begin{ompcPragma}
#pragma omp target parallel for simd \plc{[clause[
[},\plc{] clause] ... ] new-line}
    \plc{for-loops}
\end{ompcPragma}

where \plc{clause} can be any of the clauses accepted by the \code{target} or
\code{parallel}~\code{for}~\code{simd} directives, except for \code{copyin}, with identical meanings and restrictions.
\end{ccppspecific}

\needspace{6\baselineskip}
\begin{fortranspecific}
The syntax of the target parallel worksharing-loop SIMD construct is as follows:

\begin{ompfPragma}
!$omp target parallel do simd \plc{[clause[ [},\plc{] clause] ... ]}
    \plc{do-loops}
\plc{[}!$omp end target parallel do simd\plc{]}
\end{ompfPragma}

where \plc{clause} can be any of the clauses accepted by the \code{target} or
\code{parallel}~\code{do}~\code{simd} directives, except for \code{copyin}, with identical meanings and restrictions.

If an \code{end}~\code{target}~\code{parallel}~\code{do}~\code{simd} directive is not specified, an
\code{end}~\code{target}~\code{parallel}~\code{do}~\code{simd} directive is assumed at the end of
the \plc{do-loops}.
\end{fortranspecific}

\descr
The semantics are identical to explicitly specifying a \code{target} directive
immediately followed by a parallel worksharing-loop SIMD directive.


\restrictions
The restrictions for the \code{target} and parallel worksharing-loop SIMD constructs apply except for the following explicit modifications:

\begin{itemize}
\item If any \code{if} clause on the directive includes a
      \plc{directive-name-modifier} then all \code{if} clauses
      on the directive must include a \plc{directive-name-modifier}.

\item At most one \code{if} clause without a
      \plc{directive-name-modifier} can appear on the directive.

\item At most one \code{if} clause with the \code{parallel}
      \plc{directive-name-modifier} can appear on the directive.


\item At most one \code{if} clause with the \code{target}
      \plc{directive-name-modifier} can appear on the directive.
\end{itemize}

\crossreferences
\begin{itemize}
\item \code{target} construct, see
\specref{subsec:target Construct}.

\item Parallel worksharing-loop SIMD construct, see
\specref{subsec:Parallel Worksharing-Loop SIMD Construct}.

\item \code{if} Clause, see \specref{sec:if Clause}.

\item Data attribute clauses, see
\specref{subsec:Data-Sharing Attribute Clauses}.

%% \item Multi-if Clause, see \specref{subsec:Multi-if Clause}.
\end{itemize}










\subsection{\hcode{target}~\hcode{simd} Construct}
\index{target simd@{\code{target}~\code{simd}}}
\index{constructs!target simd@{\code{target}~\code{simd}}}
\index{combined constructs!target simd@{\code{target}~\code{simd}}}
\label{subsec:target simd Construct}

\summary
The \code{target} \code{simd} construct is a shortcut for specifying a \code{target}
construct containing a \code{simd} construct and no other statements.


\syntax
\begin{ccppspecific}
The syntax of the \code{target} \code{simd} construct is as follows:

\begin{ompcPragma}
#pragma omp target simd \plc{[clause[ [},\plc{] clause] ... ] new-line}
    \plc{for-loops}
\end{ompcPragma}

where \plc{clause} can be any of the clauses accepted by the \code{target} or
\code{simd} directives with identical meanings and restrictions.

\end{ccppspecific}

\needspace{6\baselineskip}
\begin{fortranspecific}
The syntax of the \code{target} \code{simd} construct is as follows:

\begin{ompfPragma}
!$omp target simd \plc{[clause[ [},\plc{] clause] ... ]}
    \plc{do-loops}
\plc{[}!$omp end target simd\plc{]}
\end{ompfPragma}

where \plc{clause} can be any of the clauses accepted by the \code{target} or
\code{simd} directives with identical meanings and restrictions.

If an \code{end}~\code{target}~\code{simd} directive is not specified, an
\code{end}~\code{target}~\code{simd} directive is assumed at the end of
the \plc{do-loops}.
\end{fortranspecific}

\descr
The semantics are identical to explicitly specifying a \code{target} directive
immediately followed by a \code{simd} directive.

\restrictions

The restrictions for the \code{target} and \code{simd} constructs apply.

\crossreferences
\begin{itemize}
\item \code{simd} construct, see
\specref{subsubsec:simd Construct}.

\item \code{target} construct, see
\specref{subsec:target Construct}.

\item Data attribute clauses, see
\specref{subsec:Data-Sharing Attribute Clauses}.
\end{itemize}









\subsection{\hcode{target}~\hcode{teams} Construct}
\label{subsec:target teams Construct}
\index{target teams@{\code{target}~\code{teams}}}
\index{constructs!target teams@{\code{target}~\code{teams}}}
\index{combined constructs!target teams@{\code{target}~\code{teams}}}
\summary
The \code{target}~\code{teams} construct is a shortcut for specifying a \code{target} construct
containing a \code{teams} construct and no other statements.

\syntax
\begin{ccppspecific}
The syntax of the \code{target}~\code{teams} construct is as follows:

\begin{ompcPragma}
#pragma omp target teams \plc{[clause[ [},\plc{] clause] ... ] new-line}
   \plc{structured-block}
\end{ompcPragma}

where \plc{clause} can be any of the clauses accepted by the \code{target} or \code{teams} directives
with identical meanings and restrictions.
\end{ccppspecific}

\begin{fortranspecific}
The syntax of the \code{target}~\code{teams} construct is as follows:

\begin{ompfPragma}
!$omp target teams \plc{[clause[ [},\plc{] clause] ... ]}
    \plc{structured-block}
!$omp end target teams
\end{ompfPragma}

where \plc{clause} can be any of the clauses accepted by the \code{target} or \code{teams} directives
with identical meanings and restrictions.
\end{fortranspecific}

\descr

The semantics are identical to explicitly specifying a \code{target} directive
immediately followed by a \code{teams} directive.

\restrictions
The restrictions for the \code{target} and \code{teams} constructs apply.

\crossreferences
\begin{itemize}
\item \code{teams} construct, see \specref{sec:teams Construct}.

\item \code{target} construct, see \specref{subsec:target Construct}.

\item Data attribute clauses, see
\specref{subsec:Data-Sharing Attribute Clauses}.
\end{itemize}









\subsection{\hcode{teams}~\hcode{distribute} Construct}
\index{teams distribute@{\code{teams}~\code{distribute}}}
\index{constructs!teams distribute@{\code{teams}~\code{distribute}}}
\index{combined constructs!teams distribute@{\code{teams}~\code{distribute}}}
\label{subsec:teams distribute Construct}
\summary
The \code{teams}~\code{distribute} construct is a shortcut for specifying a \code{teams} construct
containing a \code{distribute} construct and no other statements.

\syntax
\begin{ccppspecific}
The syntax of the \code{teams}~\code{distribute} construct is as follows:

\begin{ompcPragma}
#pragma omp teams distribute \plc{[clause[ [},\plc{] clause] ... ] new-line}
    \plc{for-loops}
\end{ompcPragma}

where \plc{clause} can be any of the clauses accepted by the \code{teams} or \code{distribute}
directives with identical meanings and restrictions.
\end{ccppspecific}

\begin{fortranspecific}
The syntax of the \code{teams}~\code{distribute} construct is as follows:

\begin{ompfPragma}
!$omp teams distribute \plc{[clause[ [},\plc{] clause] ... ]}
    \plc{do-loops}
\plc{[}!$omp end teams distribute\plc{]}
\end{ompfPragma}

where \plc{clause} can be any of the clauses accepted by the \code{teams} or \code{distribute}
directives with identical meanings and restrictions.

If an \code{end}~\code{teams}~\code{distribute} directive is not specified, an
\code{end}~\code{teams}~\code{distribute} directive is assumed at the end of the \plc{do-loops}.
\end{fortranspecific}

\descr
The semantics are identical to explicitly specifying a \code{teams} directive immediately
followed by a \code{distribute} directive.


\restrictions
The restrictions for the \code{teams} and \code{distribute} constructs apply.

\crossreferences
\begin{itemize}
\item \code{teams} construct, see
\specref{sec:teams Construct}.

\item \code{distribute} construct, see
\specref{subsec:distribute Construct}.

\item Data attribute clauses, see
\specref{subsec:Data-Sharing Attribute Clauses}.
\end{itemize}




\subsection{\hcode{teams}~\hcode{loop} Construct}
\index{teams loop@{\code{teams}~\code{loop}}}
\index{constructs!teams loop@{\code{teams}~\code{loop}}}
\index{combined constructs!teams loop@{\code{teams}~\code{loop}}}
\label{subsec:teams loop Construct}
\summary
The \code{teams}~\code{loop} construct is a shortcut for specifying a \code{teams} construct
containing a \code{loop} construct and no other statements.

\syntax
\begin{ccppspecific}
The syntax of the \code{teams}~\code{loop} construct is as follows:

\begin{ompcPragma}
#pragma omp teams loop \plc{[clause[ [},\plc{] clause] ... ] new-line}
    \plc{for-loops}
\end{ompcPragma}

where \plc{clause} can be any of the clauses accepted by the \code{teams} or
  \code{loop} directives with identical meanings and restrictions.
\end{ccppspecific}

\begin{fortranspecific}
The syntax of the \code{teams}~\code{loop} construct is as follows:

\begin{ompfPragma}
!$omp teams loop \plc{[clause[ [},\plc{] clause] ... ]}
    \plc{do-loops}
\plc{[}!$omp end teams loop\plc{]}
\end{ompfPragma}

where \plc{clause} can be any of the clauses accepted by the \code{teams} or
  \code{loop}
directives with identical meanings and restrictions.

If an \code{end}~\code{teams}~\code{loop} directive is not specified, an
\code{end}~\code{teams}~\code{loop} directive is assumed at the end of the \plc{do-loops}.
\end{fortranspecific}

\descr
The semantics are identical to explicitly specifying a \code{teams} directive immediately
followed by a \code{loop} directive.


\restrictions
The restrictions for the \code{teams} and \code{loop} constructs apply.

\crossreferences
\begin{itemize}
\item \code{teams} construct, see
\specref{sec:teams Construct}.

\item \code{loop} construct, see
\specref{subsec:loop Construct}.

\item Data attribute clauses, see
\specref{subsec:Data-Sharing Attribute Clauses}.
\end{itemize}












\subsection{\hcode{teams}~\hcode{distribute}~\hcode{simd} Construct}
\index{teams distribute simd@{\code{teams}~\code{distribute}~\code{simd}}}
\index{constructs!teams distribute simd@{\code{teams}~\code{distribute}~\code{simd}}}
\index{combined constructs!teams distribute simd@{\code{teams}~\code{distribute}~\code{simd}}}
\label{subsec:teams distribute simd Construct}
\summary
The \code{teams}~\code{distribute}~\code{simd} construct is a shortcut for specifying a \code{teams} construct
containing a \code{distribute}~\code{simd} construct and no other statements.

\syntax
\begin{ccppspecific}
The syntax of the \code{teams}~\code{distribute}~\code{simd} construct is as follows:

\begin{ompcPragma}
#pragma omp teams distribute simd \plc{[clause[ [},\plc{] clause] ... ] new-line}
    \plc{for-loops}
\end{ompcPragma}

where \plc{clause} can be any of the clauses accepted by the \code{teams} or \code{distribute}~\code{simd}
directives with identical meanings and restrictions.
\end{ccppspecific}

\begin{fortranspecific}
The syntax of the \code{teams}~\code{distribute}~\code{simd} construct is as follows:

\begin{ompfPragma}
!$omp teams distribute simd \plc{[clause[ [},\plc{] clause] ... ]}
    \plc{do-loops}
\plc{[}!$omp end teams distribute simd\plc{]}
\end{ompfPragma}

where \plc{clause} can be any of the clauses accepted by the \code{teams} or \code{distribute}~\code{simd}
directives with identical meanings and restrictions.

If an \code{end}~\code{teams}~\code{distribute}~\code{simd} directive is
not specified, an \code{end}~\code{teams}~\code{distribute}~\code{simd}
directive is assumed at the end of the \plc{do-loops}.
\end{fortranspecific}

\descr
The semantics are identical to explicitly specifying a \code{teams} directive immediately
followed by a \code{distribute}~\code{simd} directive.


\restrictions
The restrictions for the \code{teams} and \code{distribute}~\code{simd} constructs apply.

\crossreferences
\begin{itemize}
\item \code{teams} construct, see
\specref{sec:teams Construct}.

\item \code{distribute}~\code{simd} construct, see
\specref{subsec:distribute simd Construct}.

\item Data attribute clauses, see
\specref{subsec:Data-Sharing Attribute Clauses}.
\end{itemize}











\subsection{\hcode{target}~\hcode{teams}~\hcode{distribute} Construct}
\index{target teams distribute@{\code{target}~\code{teams}~\code{distribute}}}
\index{constructs!target teams distribute@{\code{target}~\code{teams}~\code{distribute}}}
\index{combined constructs!target teams distribute@{\code{target}~\code{teams}~\code{distribute}}}
\label{subsec:target teams distribute construct}
\summary
The \code{target}~\code{teams}~\code{distribute} construct is a shortcut for specifying a \code{target} construct
containing a \code{teams}~\code{distribute} construct and no other statements.

\syntax
\begin{ccppspecific}
The syntax of the \code{target}~\code{teams}~\code{distribute} construct is as follows:

\begin{ompcPragma}
#pragma omp target teams distribute \plc{[clause[ [},\plc{] clause] ... ] new-line}
   \plc{for-loops}
\end{ompcPragma}

where \plc{clause} can be any of the clauses accepted by the \code{target} or \code{teams}~\code{distribute} directives
with identical meanings and restrictions.
\end{ccppspecific}

\begin{fortranspecific}
The syntax of the \code{target}~\code{teams}~\code{distribute} construct is as follows:

\begin{ompfPragma}
!$omp target teams distribute \plc{[clause[ [},\plc{] clause] ... ]}
    \plc{do-loops}
\plc{[}!$omp end target teams distribute\plc{]}
\end{ompfPragma}

where \plc{clause} can be any of the clauses accepted by the \code{target} or \code{teams}~\code{distribute} directives
with identical meanings and restrictions.

If an \code{end}~\code{target}~\code{teams}~\code{distribute} directive is not specified, an
\code{end}~\code{target}~\code{teams}~\code{distribute} directive is assumed at the end of the \plc{do-loops}.
\end{fortranspecific}

\descr
The semantics are identical to explicitly specifying a \code{target} directive immediately
followed by a \code{teams}~\code{distribute} directive.

\restrictions
The restrictions for the \code{target} and \code{teams}~\code{distribute} constructs.

\crossreferences
\begin{itemize}
\item \code{target} construct, see
\specref{subsec:target data Construct}.

\item \code{teams}~\code{distribute} construct, see
\specref{subsec:teams distribute Construct}.

\item Data attribute clauses, see
\specref{subsec:Data-Sharing Attribute Clauses}.
\end{itemize}










\subsection{\hcode{target}~\hcode{teams}~\hcode{distribute}~\hcode{simd} Construct}
\index{target teams distribute simd@{\code{target}~\code{teams}~\code{distribute}~\code{simd}}}
\index{constructs!target teams distribute simd@{\code{target}~\code{teams}~\code{distribute}~\code{simd}}}
\index{combined constructs!target teams distribute simd@{\code{target}~\code{teams}~\code{distribute}~\code{simd}}}
\label{subsec:target teams distribute simd construct}
\summary
The \code{target}~\code{teams}~\code{distribute}~\code{simd} construct is a shortcut for specifying a \code{target} construct
containing a \code{teams}~\code{distribute}~\code{simd} construct and no other statements.

\syntax
\begin{ccppspecific}
The syntax of the \code{target}~\code{teams}~\code{distribute}~\code{simd} construct is as follows:

\begin{ompcPragma}
#pragma omp target teams distribute simd \plc{\}
            \plc{[clause[ [},\plc{] clause] ...  ] new-line}
   \plc{for-loops}
\end{ompcPragma}

where \plc{clause} can be any of the clauses accepted by the \code{target} or
\code{teams}~\code{distribute}~\code{simd} directives with identical meanings and restrictions.
\end{ccppspecific}

\begin{fortranspecific}
The syntax of the \code{target}~\code{teams}~\code{distribute}~\code{simd} construct is as follows:

\begin{ompfPragma}
!$omp target teams distribute simd \plc{[clause[ [},\plc{] clause] ... ]}
    \plc{do-loops}
\plc{[}!$omp end target teams distribute simd\plc{]}
\end{ompfPragma}

where \plc{clause} can be any of the clauses accepted by the \code{target} or
\code{teams}~\code{distribute}~\code{simd} directives with identical meanings and restrictions.

If an \code{end}~\code{target}~\code{teams}~\code{distribute}~\code{simd} directive is not specified, an
\code{end}~\code{target}~\code{teams}~\code{distribute}~\code{simd} directive is assumed at the end of the \plc{do-loops}.
\end{fortranspecific}

\descr
The semantics are identical to explicitly specifying a \code{target} directive immediately
followed by a \code{teams}~\code{distribute}~\code{simd} directive.

\restrictions
The restrictions for the \code{target} and \code{teams}~\code{distribute}~\code{simd} constructs apply.


\crossreferences
\begin{itemize}
\item \code{target} construct, see
\specref{subsec:target data Construct}.

\item \code{teams}~\code{distribute}~\code{simd} construct, see
\specref{subsec:teams distribute simd Construct}.

\item Data attribute clauses, see
\specref{subsec:Data-Sharing Attribute Clauses}.
\end{itemize}











\subsection{Teams Distribute Parallel Worksharing-Loop Construct}
\label{subsec:Teams Distribute Parallel Worksharing-Loop Construct}
\index{teams distribute parallel worksharing-loop construct}
\index{constructs!teams distribute parallel worksharing-loop construct}
\index{combined constructs!teams distribute parallel worksharing-loop construct}
\summary
The teams distribute parallel worksharing-loop construct is a shortcut for specifying a \code{teams}
construct containing a distribute parallel worksharing-loop construct and no other statements.

\syntax
\begin{ccppspecific}
The syntax of the teams distribute parallel worksharing-loop construct is as follows:

\begin{ompcPragma}[fontsize=\small]
#pragma omp teams distribute parallel for \plc{\}
            \plc{[clause[ [},\plc{] clause] ...  ] new-line}
    \plc{for-loops}
\end{ompcPragma}

where \plc{clause} can be any of the clauses accepted by the \code{teams} or
\code{distribute}~\code{parallel}~\code{for} directives with identical meanings and restrictions.
\end{ccppspecific}

\begin{fortranspecific}
The syntax of the teams distribute parallel worksharing-loop construct is as follows:

\begin{ompfPragma}
!$omp teams distribute parallel do \plc{[clause[ [},\plc{] clause] ... ]}
   \plc{do-loops}
\plc{[} !$omp end teams distribute parallel do \plc{]}
\end{ompfPragma}

where \plc{clause} can be any of the clauses accepted by the \code{teams} or
\code{distribute}~\code{parallel}~\code{do} directives with identical meanings and restrictions.

If an \code{end}~\code{teams}~\code{distribute}~\code{parallel}~\code{do} directive is not specified, an
\code{end}~\code{teams}~\code{distribute}~\code{parallel}~\code{do} directive is assumed at the end of the \plc{do-loops}.
\end{fortranspecific}

\descr
The semantics are identical to explicitly specifying a \code{teams} directive immediately
followed by a distribute parallel worksharing-loop directive.

\restrictions
The restrictions for the \code{teams} and distribute parallel worksharing-loop constructs apply.

\crossreferences
\begin{itemize}
\item \code{teams} construct, see
\specref{sec:teams Construct}.

\item Distribute parallel worksharing-loop construct, see
\specref{subsec:Distribute Parallel Worksharing-Loop Construct}.

\item Data attribute clauses, see
\specref{subsec:Data-Sharing Attribute Clauses}.
\end{itemize}









\subsection{Target Teams Distribute Parallel Worksharing-Loop Construct}
\label{subsec:Target Teams Distribute Parallel Worksharing-Loop Construct}
\index{target teams distribute parallel worksharing-loop construct}
\index{constructs!target teams distribute parallel worksharing-loop construct}
\index{combined constructs!target teams distribute parallel worksharing-loop construct}
\summary
The target teams distribute parallel worksharing-loop construct is a shortcut for specifying a \code{target}
construct containing a teams distribute parallel worksharing-loop construct and no other statements.

\syntax
\begin{ccppspecific}
The syntax of the target teams distribute parallel worksharing-loop construct is as follows:

\begin{ompcPragma}[fontsize=\small]
#pragma omp target teams distribute parallel for \plc{\}
            \plc{[clause[ [},\plc{] clause] ... ] new-line}
    \plc{for-loops}
\end{ompcPragma}

where \plc{clause} can be any of the clauses accepted by the \code{target} or
\code{teams}~\code{distribute}~\code{parallel}~\code{for} directives with identical
meanings and restrictions.
\end{ccppspecific}

\needspace{6\baselineskip}
\begin{fortranspecific}
The syntax of the target teams distribute parallel worksharing-loop construct is as follows:

\begin{ompfPragma}
!$omp target teams distribute parallel do \plc{[clause[ [},\plc{] clause] ... ]}
    \plc{do-loops}
\plc{[}!$omp end target teams distribute parallel do\plc{]}
\end{ompfPragma}

where \plc{clause} can be any of the clauses accepted by the \code{target} or
\code{teams}~\code{distribute}~\code{parallel}~\code{do} directives with
identical meanings and restrictions.

If an \code{end}~\code{target}~\code{teams}~\code{distribute}~\code{parallel}~\code{do} directive is not specified, an
\code{end}~\code{target}~\code{teams}~\code{distribute}~\code{parallel}~\code{do}
directive is assumed at the end of the \plc{do-loops}.
\end{fortranspecific}

\descr
The semantics are identical to explicitly specifying a \code{target}
directive immediately followed by a teams distribute parallel worksharing-loop directive.


\restrictions
The restrictions for the \code{target} and teams distribute parallel
worksharing-loop constructs apply except for the following explicit modifications:

\begin{itemize}
\item If any \code{if} clause on the directive includes a
      \plc{directive-name-modifier} then all \code{if} clauses
      on the directive must include a \plc{directive-name-modifier}.

\item At most one \code{if} clause without a
      \plc{directive-name-modifier} can appear on the directive.

\item At most one \code{if} clause with the \code{parallel}
      \plc{directive-name-modifier} can appear on the directive.

\item At most one \code{if} clause with the \code{target}
      \plc{directive-name-modifier} can appear on the directive.
\end{itemize}

\crossreferences
\begin{itemize}
\item \code{target} construct, see \specref{subsec:target Construct}.

\item Teams distribute parallel worksharing-loop construct, see
      \specref{subsec:Teams Distribute Parallel Worksharing-Loop Construct}.

\item \code{if} Clause, see \specref{sec:if Clause}.

\item Data attribute clauses, see
      \specref{subsec:Data-Sharing Attribute Clauses}.
\end{itemize}










\subsection{Teams Distribute Parallel Worksharing-Loop SIMD Construct}
\label{subsec:Teams Distribute Parallel Worksharing-Loop SIMD Construct}
\index{teams distribute parallel worksharing-loop SIMD construct}
\index{constructs!teams distribute parallel worksharing-loop SIMD construct}
\index{combined constructs!teams distribute parallel worksharing-loop SIMD construct}
\summary
The teams distribute parallel worksharing-loop SIMD construct is a shortcut for specifying a \code{teams}
construct containing a distribute parallel worksharing-loop SIMD construct and no other statements.


\syntax
\begin{ccppspecific}
The syntax of the teams distribute parallel worksharing-loop construct is as follows:

\begin{ompcPragma}[fontsize=\small]
#pragma omp teams distribute parallel for simd \plc{\}
            \plc{[clause[ [},\plc{] clause] ... ] new-line}
    \plc{for-loops}
\end{ompcPragma}

where \plc{clause} can be any of the clauses accepted by the \code{teams} or
\code{distribute}~\code{parallel}~\code{for}~\code{simd}
directives with identical meanings and restrictions.
\end{ccppspecific}

\begin{fortranspecific}
The syntax of the teams distribute parallel worksharing-loop construct is as follows:

\begin{ompfPragma}
!$omp teams distribute parallel do simd \plc{[clause[ [},\plc{] clause] ... ]}
    \plc{do-loops}
\plc{[}!$omp end teams distribute parallel do simd\plc{]}
\end{ompfPragma}

where \plc{clause} can be any of the clauses accepted by the \code{teams} or
\code{distribute}~\code{parallel}~\code{do}~\code{simd}
directives with identical meanings and restrictions.

If an \code{end}~\code{teams}~\code{distribute}~\code{parallel}~\code{do}~\code{simd} directive is not specified, an
\code{end}~\code{teams}~\code{distribute}~\code{parallel}~\code{do}~\code{simd} directive is assumed at the end of
the \plc{do-loops}.
\end{fortranspecific}

\descr
The semantics are identical to explicitly specifying a \code{teams} directive immediately
followed by a distribute parallel worksharing-loop SIMD directive. 

\restrictions
The restrictions for the \code{teams} and distribute parallel worksharing-loop
SIMD constructs apply.

\crossreferences
\begin{itemize}
\item \code{teams} construct, see
\specref{sec:teams Construct}.

\item Distribute parallel worksharing-loop SIMD construct, see
\specref{subsec:Distribute Parallel Worksharing-Loop SIMD Construct}.

\item Data attribute clauses, see
\specref{subsec:Data-Sharing Attribute Clauses}.
\end{itemize}










\subsection{Target Teams Distribute Parallel Worksharing-Loop SIMD Construct}
\label{subsec:Target Teams Distribute Parallel Loop SIMD Construct}
\index{target teams distribute parallel worksharing-loop SIMD construct}
\index{constructs!target teams distribute parallel worksharing-loop SIMD construct}
\index{combined constructs!target teams distribute parallel worksharing-loop SIMD construct}
\summary
The target teams distribute parallel worksharing-loop SIMD construct is a shortcut for specifying a \code{target}
construct containing a teams distribute parallel worksharing-loop SIMD construct and no other statements.

\syntax
\begin{ccppspecific}
The syntax of the target teams distribute parallel worksharing-loop SIMD construct is as follows:

\begin{ompcPragma}
#pragma omp target teams distribute parallel for simd \plc{\}
            \plc{[clause[ [},\plc{] clause] ... ] new-line}
    \plc{for-loops}
\end{ompcPragma}

where \plc{clause} can be any of the clauses accepted by the \code{target} or
\code{teams}~\code{distribute}~\code{parallel}~\code{for}~\code{simd}
directives with identical meanings and restrictions.
\end{ccppspecific}

\begin{fortranspecific}
The syntax of the target teams distribute parallel worksharing-loop SIMD construct is as follows:

\begin{ompfPragma}
!$omp target teams distribute parallel do simd \plc{[clause[ [},\plc{] clause] ... ]}
    \plc{do-loops}
\plc{[}!$omp end target teams distribute parallel do simd\plc{]}
\end{ompfPragma}

where \plc{clause} can be any of the clauses accepted by the
\code{target} or \code{teams}~\code{distribute}~\code{parallel}~\code{do}~\code{simd}
directives with identical meanings and restrictions.

If an \code{end}~\code{target}~\code{teams}~\code{distribute}~\code{parallel}~\code{do}~\code{simd}
directive is not specified, an
\code{end}~\code{target}~\code{teams}~\code{distribute}~\code{parallel}~\code{do}~\code{simd}
directive is assumed at the end of the \plc{do-loops}.
\end{fortranspecific}

\descr
The semantics are identical to explicitly specifying a \code{target}
directive immediately followed by a teams distribute parallel worksharing-loop
SIMD directive. 


\restrictions
The restrictions for the \code{target} and teams distribute parallel
worksharing-loop SIMD constructs apply except for the following explicit modifications:

\begin{itemize}
\item If any \code{if} clause on the directive includes a
      \plc{directive-name-modifier} then all \code{if} clauses
      on the directive must include a \plc{directive-name-modifier}.

\item At most one \code{if} clause without a
      \plc{directive-name-modifier} can appear on the directive.

\item At most one \code{if} clause with the \code{parallel}
      \plc{directive-name-modifier} can appear on the directive.

\item At most one \code{if} clause with the \code{target}
      \plc{directive-name-modifier} can appear on the directive.
  \end{itemize}

\crossreferences
\begin{itemize}
\item \code{target} construct, see \specref{subsec:target Construct}.

\item Teams distribute parallel worksharing-loop SIMD construct, see
      \specref{subsec:Teams Distribute Parallel Worksharing-Loop SIMD Construct}.

\item \code{if} Clause, see \specref{sec:if Clause}.

\item Data attribute clauses, see
      \specref{subsec:Data-Sharing Attribute Clauses}.
\end{itemize}

%% \filbreak
\section{Clauses on Combined and Composite Constructs}
\label{sec:Clauses on Combined and Composite Constructs}
This section specifies the handling of clauses on combined or composite constructs and handling of implicit clauses from
variables with predetermined data sharing if they are not predetermined only on a particular construct.
Some clauses are permitted only on a single construct from the constructs that
constitute the combined or composite construct, the effect is then as if the
clause is applied to that specific construct.  Other clauses have the effect as if
they are applied to one or more constituent constructs as specified below:

\begin{itemize}
\item The \code{collapse} clause is applied once for the whole combined or composite construct.

\item For the \code{private} clause the effect is as if it is applied to the innermost constituent construct only.

\item For the \code{firstprivate} clause the effect is as if it is applied to one or more
constructs as follows:
\begin{itemize}
\item to the \code{distribute} construct if it is among the constituent constructs,

\item to the \code{teams} construct if it is among the constituent constructs and \code{distribute} construct is not,

\item to the worksharing-loop construct if it is among the constituent constructs,

\item to the \code{parallel} construct if it is among the constituent
  constructs and the worksharing-loop construct is not,

\item to the outermost constituent construct if not already applied to it by the above
rules and the outermost constituent construct is neither \code{teams} nor \code{parallel}
nor \code{target} construct,

\item to the \code{target} construct if it is among the constituent
constructs and the same list item does not appear in \code{lastprivate} or \code{map} clause.
\end{itemize}

If the \code{parallel} construct is among the constituent constructs and the
effect is not as if the \code{firstprivate} clause is applied to it by the
above rules, then the effect is as if the \code{shared} clause with the same
list item is applied to the \code{parallel} construct.

If the \code{teams} construct is among the constituent constructs and the
effect is not as if the \code{firstprivate} clause is applied to it by the
above rules, then the effect is as if the \code{shared} clause with the same
list item is applied to the \code{teams} construct.

\item For the \code{lastprivate} clause the effect is as if it is applied to one or more
constructs as follows:

\begin{itemize}

\item to the worksharing-loop construct if it is among the constituent constructs,

\item to the \code{distribute} construct if it is among the constituent constructs,

\item to the innermost constituent construct that permits it unless it is a
worksharing-loop or \code{distribute} construct.
\end{itemize}

If the \code{parallel} construct is among the constituent constructs and the
list item is not also mentioned in the \code{firstprivate} clause, then the effect is as
if the \code{shared} clause with the same list item is applied to the \code{parallel} construct.

If the \code{teams} construct is among the constituent constructs and the
list item is not also mentioned in the \code{firstprivate} clause, then the effect is as
if the \code{shared} clause with the same list item is applied to the \code{teams} construct.

If the \code{target} construct is among the constituent constructs and the list item doesn't
appear in a \code{map} clause the effect is as if the same list item appears in a \code{map} clause
with a \plc{map-type} of \code{tofrom}.

\item For the \code{shared}, \code{default}, \code{order}, or \code{allocate} clauses the effect is as if it is applied to all
the constituent constructs that permit those clauses.

%\item For the \code{reduction} clause the effect is as if it is applied to all the constructs that permit
%the clause, except that it is not applied to the worksharing-loop or \code{sections} construct if they are
%combined with the \code{parallel} construct and neither the \code{inscan} or \code{task} modifier is present.  
%If the construct is combined with the \code{target} construct, the effect is also as if the same list
%item appears in a \code{map} clause with a \plc{map-type} of \code{tofrom}.

\item For the \code{reduction} clause the effect is as if it is applied to all the constructs that permit
the clause, except for the following constructs:
\begin{itemize}
\item the \code{parallel} construct, when combined with the
    worksharing-loop, \code{loop}, or \code{sections} construct;
\item the \code{teams} construct,  when combined with the \code{loop} construct.
\end{itemize}
For the \code{parallel} and \code{teams} constructs above, the behavior
instead is as if each list item or, for any list item that is an array item, its corresponding
named array or named pointer appears in a \code{shared} clause for the construct.
If \plc{reduction-modifier} is specified, the effect is as if it only modifies
the behavior of the \code{reduction} clause for the innermost construct that
constitutes the combined construct and accepts the modifier (see
Section~\ref{subsubsec:reduction clause}). If the construct is combined with
the \code{target} construct, the effect is also as if the same list item
appears in a \code{map} clause with a \plc{map-type} of \code{tofrom}.

\item The \code{in_reduction} clause is permitted on a single construct
among the combined or composite construct and the effect is as if it is
applied to that construct, but if that construct is \code{target}, the
effect is also as if the same list item appears in a \code{map} clause with
a \plc{map-type} of \code{tofrom} and a \plc{map-type-modifier} of
\code{always}.

\item For the \code{if} clause the effect is described in the \specref{sec:if Clause} section.

\item For the \code{linear} clause the effect is as if it is applied to the innermost
constituent construct.
Additionally, if the list item is not the iteration variable of the
\code{simd} or worksharing-loop SIMD
construct, the effect on the outer constituent constructs is as if the list item was present
in the \code{firstprivate} and \code{lastprivate} clauses on the combined or
composite construct and the rules specified above would apply.  If the list item is the
iteration variable of the \code{simd} or worksharing-loop SIMD construct and it is not declared in the construct,
the effect on the outer constituent constructs is as if the list item was present in the
\code{lastprivate} clause on the combined or composite construct and the rules specified above
would apply.

\item For the \code{nowait} clause the effect is as if it is applied to the
outermost constituent construct that permits it.

\end{itemize}

If the clauses have expressions on them, such as for various clauses where the argument of the
clause is an expression, or \plc{lower-bound}, \plc{length}, or \plc{stride} expressions inside array
sections (or \plc{subscript} and \plc{stride} expressions in \plc{subscript-triple} for Fortran),
or \plc{linear-step} or \plc{alignment} expressions, the expressions are evaluated immediately
before the construct to which the clause has been split or duplicated per the
above rules (therefore inside of the outer constituent constructs), except that the expressions inside
of the \code{num_teams} and \code{thread_limit} clauses are always evaluated before the outermost
constituent construct.

The restriction that a list item may not appear in more than one data
sharing clause with the exception of specifying a variable in both
\code{firstprivate} and \code{lastprivate} clauses applies after the clauses
are split or duplicated per the above rules.

% This is an included file. See the master file for more information.
%
% When editing this file:
%
%    1. To change formatting, appearance, or style, please edit openmp.sty.
%
%    2. Custom commands and macros are defined in openmp.sty.
%
%    3. Be kind to other editors -- keep a consistent style by copying-and-pasting to
%       create new content.
%
%    4. We use semantic markup, e.g. (see openmp.sty for a full list):
%         \code{}     % for bold monospace keywords, code, operators, etc.
%         \plc{}      % for italic placeholder names, grammar, etc.
%
%    5. There are environments that provide special formatting, e.g. language bars.
%       Please use them whereever appropriate.  Examples are:
%
%         \begin{fortranspecific}
%         This is text that appears enclosed in blue language bars for Fortran.
%         \end{fortranspecific}
%
%         \begin{note}
%         This is a note.  The "Note -- " header appears automatically.
%         \end{note}
%
%    6. Other recommendations:
%         Use the convenience macros defined in openmp.sty for the minor headers
%         such as Comments, Syntax, etc.
%
%         To keep items together on the same page, prefer the use of
%         \begin{samepage}.... Avoid \parbox for text blocks as it interrupts line numbering.
%         When possible, avoid \filbreak, \pagebreak, \newpage, \clearpage unless that's
%         what you mean. Use \needspace{} cautiously for troublesome paragraphs.
%
%         Avoid absolute lengths and measures in this file; use relative units when possible.
%         Vertical space can be relative to \baselineskip or ex units. Horizontal space
%         can be relative to \linewidth or em units.
%
%         Prefer \emph{} to italicize terminology, e.g.:
%             This is a \emph{definition}, not a placeholder.
%             This is a \plc{var-name}.
%


\section{Master and Synchronization Constructs and Clauses}
\label{sec:Master and Synchronization Constructs and Clauses}
\index{master and synchronization constructs and clauses}
\index{synchronization constructs}
OpenMP provides the following synchronization constructs:
\begin{itemize}
\item the \code{master} construct;

\item the \code{critical} construct;

\item the \code{barrier} construct;

\item the \code{taskwait} construct;

\item the \code{taskgroup} construct;

\item the \code{atomic} construct;

\item the \code{flush} construct;

\item the \code{ordered} construct.
\end{itemize}


\subsection{\hcode{master} Construct}
\index{maste@{\code{master}}}
\index{constructs!master@{\code{master}}}
\label{subsec:master Construct}
\summary
The \code{master} construct specifies a structured block that is executed by the master thread
of the team.

\syntax
\begin{ccppspecific}
The syntax of the \code{master} construct is as follows:

\begin{ompcPragma}
#pragma omp master \plc{new-line}
   \plc{structured-block}
\end{ompcPragma}
\end{ccppspecific}

\begin{fortranspecific}
The syntax of the \code{master} construct is as follows:

\begin{ompfPragma}
!$omp master
   \plc{structured-block}
!$omp end master
\end{ompfPragma}
\end{fortranspecific}

\binding
The binding thread set for a \code{master} region is the current team. A \code{master} region
binds to the innermost enclosing \code{parallel} region. Only the master thread of the team
executing the binding \code{parallel} region participates in the execution of the structured
block of the \code{master} region.

\descr
Other threads in the team do not execute the associated structured block. There is no
implied barrier either on entry to, or exit from, the \code{master} construct.

\events

The \plc{master-begin} event occurs in the thread encountering the \code{master}
construct on entry to the master region, if it is the master thread of the team.

The \plc{master-end} event occurs in the thread encountering the \code{master}
construct on exit of the master region, if it is the master thread of the team.

\tools

A thread dispatches a registered \code{ompt_callback_master}
callback for each occurrence of a \plc{master-begin} and a
\plc{master-end} event in that thread.

The callback occurs  in the context of the task executed by the master thread.
This callback has the type signature
\code{ompt_callback_master_t}. The callback receives
\code{ompt_scope_begin} or \code{ompt_scope_end}
as its \plc{endpoint} argument, as appropriate.

\restrictions
\begin{cppspecific}
\begin{itemize}
\item A throw executed inside a \code{master} region must cause execution to resume within the
same \code{master} region, and the same thread that threw the exception must catch it
\end{itemize}
\end{cppspecific}

\crossreferences
\begin{itemize}

\item \code{ompt_scope_begin} and \code{ompt_scope_end}, see
\specref{sec:ompt_scope_endpoint_t}.

\item \code{ompt_callback_master_t}, see
\specref{sec:ompt_callback_master_t}.


\end{itemize}









\vspace{-12pt} %% UGLY HACK
\subsection{\hcode{critical} Construct}
\index{critical@{\code{critical}}}
\index{constructs!critical@{\code{critical}}}
\label{subsec:critical Construct}
\summary
The \code{critical} construct restricts execution of the associated structured block to a
single thread at a time.

\syntax
\begin{ccppspecific}
The syntax of the \code{critical} construct is as follows:

\begin{ompcPragma}
  #pragma omp critical \plc{[}(\plc{name}) \plc{[[},\plc{]} hint(\plc{hint-expression})\plc{] ] new-line}
    \plc{structured-block}
\end{ompcPragma}

where \plc{hint-expression} is an integer constant expression that
evaluates to a valid synchronization hint (as described in~\specref{subsec:Synchronization Hints}).
\end{ccppspecific}

\begin{fortranspecific}
The syntax of the \code{critical} construct is as follows:

\begin{ompfPragma}
!$omp critical \plc{[}(\plc{name}) \plc{[[},\plc{]} hint(\plc{hint-expression})\plc{] ]}
    \plc{structured-block}
!$omp end critical \plc{[}(\plc{name})\plc{]}
\end{ompfPragma}

where \plc{hint-expression} is a constant expression that evaluates to
a scalar value with kind \code{omp_sync_hint_kind} and  a value
that is a valid synchronization hint (as described
in~\specref{subsec:Synchronization Hints}).
\end{fortranspecific}

\binding
The binding thread set for a \code{critical} region is all threads in the contention group.
The region is executed as if only a single thread at a time among all threads in the
contention group is entering the region for execution, without regard to the team(s) to which the threads belong.

\descr
An optional \plc{name} may be used to identify the \code{critical} construct. All \code{critical}
constructs without a name are considered to have the same unspecified name.

\begin{ccppspecific}
Identifiers used to identify a \code{critical} construct have external linkage and are in a
name space that is separate from the name spaces used by labels, tags, members, and
ordinary identifiers.
\end{ccppspecific}

\begin{fortranspecific}
The names of \code{critical} constructs are global entities of the program. If a name
conflicts with any other entity, the behavior of the program is unspecified.
\end{fortranspecific}

The threads of a contention group execute the \code{critical} region as if only one thread of the contention group is executing the \code{critical} region at a time.
The \code{critical} construct enforces these execution semantics with respect to all \code{critical} constructs with the same name in all
threads in the contention group, not just those threads in the current team.

If present, the \code{hint} clause gives the implementation additional
information about the expected runtime properties of the \code{critical} region
that can optionally be used to optimize the implementation.
The presence of a \code{hint} clause does not affect the isolation
guarantees provided by the \code{critical} construct. If no
\code{hint} clause is specified, the effect is as if \code{hint(omp_sync_hint_none)}
had been specified.

\def\omptMutex#1#2{
\events
The \plc{#1-acquire} event occurs in the thread encountering the
\code{#1} construct on entry to the #1 region before
initiating synchronization for the region.

The \plc{#1-acquired} event occurs in the thread encountering the
\code{#1} construct after entering the region, but before executing the
structured block of the \code{#1} region.

The \plc{#1-release} event occurs in the thread encountering the
\code{#1} construct after completing any synchronization
on exit from the \code{#1} region.

\tools
A thread dispatches a registered \code{ompt_callback_mutex_acquire}
callback for each occurrence of #2 \plc{#1-acquire} event
in that thread.
This callback has the type signature \code{ompt_callback_mutex_acquire_t}.

A thread dispatches a registered \code{ompt_callback_mutex_acquired}
callback for each occurrence of #2 \plc{#1-acquired} event
in that thread.  This callback has the type signature \code{ompt_callback_mutex_t}.

A thread dispatches a registered \code{ompt_callback_mutex_released}
callback for each occurrence of #2 \plc{#1-release} event
in that thread.  This callback has the type signature \code{ompt_callback_mutex_t}.
The callbacks occur in the task encountering
the #1 construct.  The callbacks should receive \code{ompt_mutex_#1}
as their \plc{kind} argument if practical, but a less specific kind is acceptable.
}

\omptMutex{critical}{a}

\restrictions
\begin{itemize}
\item If the \code{hint} clause is specified, the \code{critical}
      construct must have a \plc{name}.
\item If the \code{hint} clause is specified, each of the
  \code{critical} constructs with the same \plc{name} must have a
  \code{hint} clause for which the \plc{hint-expression} evaluates to the same
  value.

% \item All \code{critical} constructs with the same \plc{name} must
%       have a \code{hint} clause for which the \plc{hint-expression} evaluates
%       to the same value if any of them has a \code{hint} clause.
\end{itemize}

\begin{cppspecific}
\begin{itemize}
\item A throw executed inside a \code{critical} region must cause execution to resume within
the same \code{critical} region, and the same thread that threw the exception must catch
it.
\end{itemize}
\end{cppspecific}

\vspace{-6pt} %% UGLY HACK
\begin{fortranspecific}
The following restrictions apply to the critical construct:

\begin{itemize}
\item If a \plc{name} is specified on a \code{critical} directive, the same \plc{name} must also be
specified on the \code{end}~\code{critical} directive.

\item If no \plc{name} appears on the \code{critical} directive, no \plc{name} can appear on the
\code{end}~\code{critical} directive.
\end{itemize}
\end{fortranspecific}

\crossreferences
\begin{itemize}
\item Synchronization Hints, see
\specref{subsec:Synchronization Hints}.

\item \code{ompt_mutex_critical}, see
\specref{sec:ompt_mutex_t}.

\item \code{ompt_callback_mutex_acquire_t}, see
\specref{sec:ompt_callback_mutex_acquire_t}.

\item \code{ompt_callback_mutex_t}, see
\specref{sec:ompt_callback_mutex_t}.

\end{itemize}









\subsection{\hcode{barrier} Construct}
\index{barrier@{\code{barrier}}}
\index{constructs!barrier@{\code{barrier}}}
\label{subsec:barrier Construct}
\summary
The \code{barrier} construct specifies an explicit barrier at the point at which the construct
appears. The \code{barrier} construct is a stand-alone directive.

\syntax
\begin{ccppspecific}
The syntax of the \code{barrier} construct is as follows:

\begin{ompcPragma}
#pragma omp barrier \plc{new-line}
\end{ompcPragma}
\end{ccppspecific}

\begin{fortranspecific}
The syntax of the \code{barrier} construct is as follows:

\begin{ompfPragma}
!$omp barrier
\end{ompfPragma}
\end{fortranspecific}

\binding
The binding thread set for a \code{barrier} region is the current team. A \code{barrier} region
binds to the innermost enclosing \code{parallel} region.

\descr
All threads of the team executing the binding \code{parallel} region must execute the
\code{barrier} region and complete execution of all explicit tasks bound to this \code{parallel}
region before any are allowed to continue execution beyond the barrier.

The \code{barrier} region includes an implicit task scheduling point in the current task
region.

\def\omptSyncRegionEvents#1{
The \plc{#1-begin} event occurs in each thread encountering the
\code{#1} construct on entry to the \code{#1} region.

The \plc{#1-wait-begin} event occurs when a task begins an interval of active or passive waiting
in a \code{#1} region.

The \plc{#1-wait-end} event occurs when a task ends an interval of active or passive waiting
and resumes execution in a \code{#1} region.

The \plc{#1-end} event occurs in each thread encountering the
\code{#1} construct after the #1 synchronization on exit from the
\code{#1} region.
}

\def\omptSyncRegionTools#1{
A thread dispatches a registered \code{ompt_callback_sync_region}
callback for each occurrence of a \plc{#1-begin} and \plc{#1-end} event
in that thread.  The callback occurs in the task encountering
the #1 construct.  This callback has the type signature
\code{ompt_callback_sync_region_t}.
The callback receives
\code{ompt_sync_region_#1} as its \plc{kind} argument and
\code{ompt_scope_begin} or \code{ompt_scope_end}
as its \plc{endpoint} argument, as appropriate.

A thread dispatches a registered
\code{ompt_callback_sync_region_wait} callback
for each occurrence of a \plc{#1-wait-begin} and \plc{#1-wait-end} event.
This callback has type signature \code{ompt_callback_sync_region_t}.
This callback executes in the context of the task that encountered the
\code{#1} construct. The callback receives
\code{ompt_sync_region_#1} as its \plc{kind} argument and
\code{ompt_scope_begin} or \code{ompt_scope_end}
as its \plc{endpoint} argument, as appropriate.
}

\def\omptSyncRegion#1{
\events
\omptSyncRegionEvents{#1}
\tools
\omptSyncRegionTools{#1}
}

\events
\omptSyncRegionEvents{barrier}

A \plc{cancellation} event occurs if cancellation is activated at an implicit cancellation point in an barrier region.

\tools

\omptSyncRegionTools{barrier}

A thread dispatches a registered \code{ompt_callback_cancel}
callback for each occurrence of a \plc{cancellation} event in that thread.
The callback occurs in the context of the encountering task.  The callback has type signature
\code{ompt_callback_cancel_t}.
The callback receives \code{ompt_cancel_detected} as its \plc{flags} argument.

\restrictions
The following restrictions apply to the \code{barrier} construct:

\begin{itemize}
\item Each \code{barrier} region must be encountered by all threads in a team or by none at all,
unless cancellation has been requested for the innermost enclosing parallel region.

\item The sequence of worksharing regions and \code{barrier} regions encountered must be the
same for every thread in a team.
\end{itemize}

\crossreferences
\begin{itemize}
\item \code{ompt_scope_begin} and \code{ompt_scope_end}, see
\specref{sec:ompt_scope_endpoint_t}.

\item \code{ompt_sync_region_barrier}, see
\specref{sec:ompt_sync_region_t}.

\item \code{ompt_callback_sync_region_t}, see
\specref{sec:ompt_callback_sync_region_t}.

\item \code{ompt_callback_cancel_t}, see
\specref{sec:ompt_callback_cancel_t}.

\end{itemize}






\subsection{Implicit Barriers}
\index{implicit barrier}
\index{barrier, implicit}
\label{subsec:implict-barrier}

Implicit tasks in a parallel region synchronize with one another using
implicit barriers at the end of worksharing constructs and at the end
of the \code{parallel} region. This section describes the OMPT events and
tool callbacks associated with implicit barriers.

Implicit barriers are task scheduling points. For a description of
task sheduling points, associated events, and tool callbacks, see
\specref{subsec:Task Scheduling}.

\events

A \plc{cancellation} event occurs if cancellation is activated at an
implicit cancellation point in an implicit barrier region.

The \plc{implicit-barrier-begin} event occurs in each implicit task
at the beginning of an implicit barrier.

The \plc{implicit-barrier-wait-begin} event occurs when a task begins an interval of
active or passive waiting while executing in an implicit barrier region.

The \plc{implicit-barrier-wait-end} event occurs when a task ends an interval of
active or waiting and resumes execution of an implicit barrier region.

The \plc{implicit-barrier-end} event occurs in each implicit task
at the end of an implicit barrier.

\tools

A thread dispatches a registered \code{ompt_callback_sync_region}
callback for each occurrence of a \plc{implicit-barrier-begin} and
\plc{implicit-barrier-end} event in that thread.  The callback occurs
in the implicit task executing in a parallel region.
This callback has the type signature
\code{ompt_callback_sync_region_t}.  The callback receives
\code{ompt_sync_region_barrier} as its \plc{kind} argument and
\code{ompt_scope_begin} or \code{ompt_scope_end} as its
\plc{endpoint} argument, as appropriate.

A thread dispatches a registered \code{ompt_callback_cancel}
callback for each occurrence of a \plc{cancellation} event in that thread.
The callback occurs in the context of the encountering task.  The
callback has type signature \code{ompt_callback_cancel_t}.  The
callback receives \code{ompt_cancel_detected} as its \plc{flags}
argument.

A thread dispatches a registered
\code{ompt_callback_sync_region_wait} callback for each occurrence
of a \plc{implicit-barrier-wait-begin} and
\plc{implicit-barrier-wait-end} event.  This callback has type
signature \code{ompt_callback_sync_region_t}.  The callback occurs
in each implicit task participating in an implicit barrier.  The
callback receives \code{ompt_sync_region_barrier} as its \plc{kind}
argument and \code{ompt_scope_begin} or \code{ompt_scope_end} as
its \plc{endpoint} argument, as appropriate.

\restrictions
If a thread is in the state \code{omp_state_wait_barrier_implicit_parallel},
a call to \code{ompt_get_parallel_info}
may return a pointer to a copy of the current parallel region's \plc{parallel_data}
rather than a pointer to the data word for the region itself. This convention enables the master thread
for a parallel region to free storage for the region immediately after the region ends, yet
avoid having some other thread in the region's team
potentially reference the region's \plc{parallel_data} object after it has been freed.

\crossreferences
\begin{itemize}

\item \code{ompt_scope_begin} and \code{ompt_scope_end}, see
  \specref{sec:ompt_scope_endpoint_t}.

\item \code{ompt_sync_region_barrier}, see
  \specref{sec:ompt_sync_region_t}

\item \code{ompt_cancel_detected}, see
  \specref{sec:ompt_cancel_flag_t}.

\item \code{ompt_callback_sync_region_t}, see
  \specref{sec:ompt_callback_sync_region_t}.

\item \code{ompt_callback_cancel_t}, see
  \specref{sec:ompt_callback_cancel_t}.

\end{itemize}









\subsection{\hcode{taskwait} Construct}
\index{taskwait@{\code{taskwait}}}
\index{constructs!taskwait@{\code{taskwait}}}
\label{subsec:taskwait Construct}
\summary
The \code{taskwait} construct specifies a wait on the completion of child tasks
of the current task. The \code{taskwait} construct is a stand-alone directive.

\syntax
\begin{ccppspecific}
The syntax of the \code{taskwait} construct is as follows:

\begin{ompcPragma}
#pragma omp taskwait \plc{[clause[ [},\plc{] clause] ... ] new-line}
\end{ompcPragma}

where \plc{clause} is one of the following:

\begin{indentedcodelist}
depend(\plc{dependence-type }:\plc{ locator-list}[:\plc{iterators-definition}])
\end{indentedcodelist}
\end{ccppspecific}

\begin{fortranspecific}
The syntax of the \code{taskwait} construct is as follows:

\begin{ompfPragma}
!$omp taskwait \plc{[clause[ [},\plc{] clause] ... ]}
\end{ompfPragma}

where \plc{clause} is one of the following:

\begin{indentedcodelist}
depend(\plc{dependence-type }:\plc{ locator-list}[:\plc{iterators-definition}])
\end{indentedcodelist}

\end{fortranspecific}

\binding
The \code{taskwait} region binds to the current task region. The binding thread set of the
\code{taskwait} region is the current team.

\descr

If no \code{depend} clause is present on the \code{taskwait} construct, the
current task region is suspended at an implicit task scheduling point
associated with the construct. The current task region remains suspended until
all child tasks that it generated before the \code{taskwait} region complete
execution.

Otherwise, if one or more \code{depend} clauses are present on the
\code{taskwait} construct, the behavior is as if these clauses were applied to
a \code{task} construct with an empty associated structured block that
generates a \emph{mergeable} and \emph{included task}. Thus, the current task region is
suspended until the \emph{predecessor tasks} of this task complete execution.

\omptSyncRegion{taskwait}

\restrictions

The \code{mutexinoutset} \plc{dependence-type} may not appear in a
\code{depend} clause on a \code{taskwait} construct.

\crossreferences
\begin{itemize}
\item \code{task} construct, see \specref{subsec:task Construct}.

\item Task scheduling, see
\specref{subsec:Task Scheduling}.

\item \code{depend} clause, see \specref{subsec:depend Clause}.

\item \code{ompt_scope_begin} and \code{ompt_scope_end}, see
\specref{sec:ompt_scope_endpoint_t}.

\item \code{ompt_sync_region_taskwait}, see
\specref{sec:ompt_sync_region_t}.

\item \code{ompt_callback_sync_region_t}, see
\specref{sec:ompt_callback_sync_region_t}.

\end{itemize}







\subsection{\hcode{taskgroup} Construct}
\index{taskgroup@{\code{taskgroup}}}
\index{constructs!taskgroup@{\code{taskgroup}}}
\label{subsec:taskgroup Construct}
\summary
The \code{taskgroup} construct specifies a wait on completion of child tasks of the current
task and their descendent tasks.

\syntax
\begin{ccppspecific}
The syntax of the \code{taskgroup} construct is as follows:

\begin{ompcPragma}
#pragma omp taskgroup \plc{[clause[[,] clause] ...]} \plc{new-line}
    \plc{structured-block}
\end{ompcPragma}

where \plc{clause} is one of the following:

\begin{indentedcodelist}
task_reduction(\plc{reduction-identifier }:\plc{ list})
allocate(\plc{[allocator: ]}\plc{list})
\end{indentedcodelist}
\end{ccppspecific}

\begin{fortranspecific}
The syntax of the \code{taskgroup} construct is as follows:

\begin{ompfPragma}
!$omp taskgroup \plc{[clause [ [},\plc{] clause] ...]}
    \plc{structured-block}
!$omp end taskgroup
\end{ompfPragma}

where \plc{clause} is one of the following:

\begin{indentedcodelist}
task_reduction(\plc{reduction-identifier }:\plc{ list})
allocate(\plc{[allocator: ]}\plc{list})
\end{indentedcodelist}

\end{fortranspecific}

\binding
A \code{taskgroup} region binds to the current task region. A \code{taskgroup} region binds to
the innermost enclosing \code{parallel} region.

\descr
When a thread encounters a \code{taskgroup} construct, it starts executing
the region. All child tasks generated in the \code{taskgroup} region and all
of their descendants that bind to the same \code{parallel} region as the
\code{taskgroup} region are part of the \plc{taskgroup set} associated with
the \code{taskgroup} region.

There is an implicit task scheduling point at the end of the \code{taskgroup}
region. The current task is suspended at the task scheduling point until all
tasks in the \plc{taskgroup set} complete execution.

\omptSyncRegion{taskgroup}

\crossreferences
\begin{itemize}
\item Task scheduling, see
\specref{subsec:Task Scheduling}.
\item \code{task_reduction} Clause, see \specref{subsubsec:task_reduction clause}.

\item \code{ompt_scope_begin} and \code{ompt_scope_end}, see
\specref{sec:ompt_scope_endpoint_t}.

\item \code{ompt_sync_region_taskgroup}, see
\specref{sec:ompt_sync_region_t}.

\item \code{ompt_callback_sync_region_t}, see
\specref{sec:ompt_callback_sync_region_t}.

\end{itemize}










\subsection{\hcode{atomic} Construct}
\index{atomic@{\code{atomic}}}
\index{constructs!atomic@{\code{atomic}}}
\index{constructs!atomic@{\code{atomic}}}
\index{write, atomic@{\code{write, atomic}}}
\index{read, atomic@{\code{read, atomic}}}
\index{update, atomic@{\code{update, atomic}}}
\index{capture, atomic@{\code{capture, atomic}}}
\label{subsec:atomic Construct}
\summary
The \code{atomic} construct ensures that a specific storage location is accessed atomically,
rather than exposing it to the possibility of multiple, simultaneous reading and writing
threads that may result in indeterminate values.

\syntax
In the following syntax, \plc{atomic-clause} is a clause that indicates
the semantics for which atomicity is enforced and \plc{memory-order-clause} is
a clause that indicates the memory ordering behavior of the construct.
Specifically, \plc{atomic-clause} is one of the following:

\begin{indentedcodelist}
read
write
update
capture
\end{indentedcodelist}

and \plc{memory-order-clause} is one of the following:

\begin{indentedcodelist}
seq_cst
acq_rel
release
acquire
\end{indentedcodelist}

\begin{ccppspecific}
The syntax of the \code{atomic} construct takes one of the following forms:

\begin{ompcPragma}
#pragma omp atomic \plc{[memory-order-clause[},\plc{]]} \plc{atomic-clause}
                   \plc{[[},\plc{]}hint(\plc{hint-expression})\plc{]}
                   \plc{[[},\plc{]memory-order-clause]} \plc{new-line}
    \plc{expression-stmt}
\end{ompcPragma}

%% where \plc{atomic-clause} is one of the following:
%%
%% \begin{indentedcodelist}
%% read
%% write
%% update
%% capture
%% \end{indentedcodelist}

or

\begin{ompcPragma}
#pragma omp atomic \plc{[memory-order-clause]} \plc{[[},\plc{]}hint(\plc{hint-expression})\plc{]} \plc{new-line}
    \plc{expression-stmt}
\end{ompcPragma}

or

\begin{ompcPragma}
#pragma omp atomic \plc{[memory-order-clause[},\plc{]]} capture
                   \plc{[[},\plc{]}\plc{[}hint(\plc{hint-expression})\plc{]}
                   \plc{[[},\plc{]memory-order-clause]} \plc{new-line}
    \plc{structured-block}
\end{ompcPragma}

%% \needspace{6\baselineskip}
where \plc{expression-stmt} is an expression statement with one of the following forms:

\begin{itemize}
\item If \plc{atomic-clause} is \code{read}:
\begin{ompSyntax}
\plc{v} = \plc{x};
\end{ompSyntax}


\item If \plc{atomic-clause} is \code{write}:
\begin{ompSyntax}
\plc{x} = \plc{expr};
\end{ompSyntax}

\item If \plc{atomic-clause} is \code{update} or not present:
\begin{ompSyntax}
\plc{x}++;
\plc{x}--;
++\plc{x};
--\plc{x};
\plc{x} \plc{binop}= \plc{expr};
\plc{x} = \plc{x} \plc{binop} \plc{expr};
\plc{x} = \plc{expr} \plc{binop} \plc{x};
\end{ompSyntax}

\item If \plc{atomic-clause} is \code{capture}:
\begin{ompSyntax}
\plc{v} = \plc{x}++;
\plc{v} = \plc{x}--;
\plc{v} = ++\plc{x};
\plc{v} = --\plc{x};
\plc{v} = \plc{x} \plc{binop}= \plc{expr};
\plc{v} = \plc{x} = \plc{x} \plc{binop} \plc{expr};
\plc{v} = \plc{x} = \plc{expr} \plc{binop} \plc{x};
\end{ompSyntax}

and where \plc{structured-block} is a structured block with one of the following forms:

\begin{ompSyntax}
\plc{v} = \plc{x}; \plc{x} \plc{binop}= \plc{expr};
\plc{x} \plc{binop}= \plc{expr}; \plc{v} = \plc{x};
\plc{v} = \plc{x}; \plc{x} = \plc{x} \plc{binop} \plc{expr};
\plc{v} = \plc{x}; \plc{x} = \plc{expr} \plc{binop} \plc{x};
\plc{x} = \plc{x} \plc{binop} \plc{expr}; \plc{v} = \plc{x};
\plc{x} = \plc{expr} \plc{binop} \plc{x}; \plc{v} = \plc{x};
\plc{v} = \plc{x}; \plc{x} = \plc{expr};
\plc{v} = \plc{x}; \plc{x}++;
\plc{v} = \plc{x}; ++\plc{x};
++\plc{x}; \plc{v} = \plc{x};
\plc{x}++\code{;} \plc{v} = \plc{x};
\plc{v} = \plc{x}; \plc{x}--;
\plc{v} = \plc{x}; --\plc{x};
--\plc{x}; \plc{v} = \plc{x};
\plc{x}--; \plc{v} = \plc{x};
\end{ompSyntax}
\end{itemize}

In the preceding expressions:

\begin{itemize}
\item \plc{x} and \plc{v} (as applicable) are both \plc{l-value} expressions with scalar type.

\item During the execution of an atomic region, multiple syntactic occurrences of \plc{x} must
designate the same storage location.

\item Neither of \plc{v} and \plc{expr} (as applicable) may access the storage location designated by \plc{x}.

\item Neither of \plc{x} and \plc{expr} (as applicable) may access the storage location designated by \plc{v}.

\item \plc{expr} is an expression with scalar type.

\item \plc{binop} is one of \code{+}, \code{*}, \code{-}, \code{/},
\code{&}, \code{^}, \code{|}, \code{<<}, or
\code{>>}.

\item \plc{binop}, \plc{binop}\code{=}, \code{++}, and \code{--} are not overloaded operators.

\item The expression \plc{x} \plc{binop} \plc{expr} must be numerically equivalent to
\plc{x} \plc{binop} \plc{(expr)}. This
requirement is satisfied if the operators in \plc{expr} have precedence greater than \plc{binop},
or by using parentheses around \plc{expr} or subexpressions of \plc{expr}.

\item The expression \plc{expr} \plc{binop} \plc{x} must be numerically equivalent to
\plc{(expr)} \plc{binop} \plc{x}. This
requirement is satisfied if the operators in \plc{expr} have precedence equal to or greater
than \plc{binop}, or by using parentheses around \plc{expr} or subexpressions of \plc{expr}.

\item For forms that allow multiple occurrences of \plc{x}, the number of times that \plc{x} is
evaluated is unspecified.
\end{itemize}

\end{ccppspecific}

\begin{fortranspecific}
The syntax of the \code{atomic} construct takes any of the following forms:

\begin{ompfPragma}
!$omp atomic \plc{[memory-order-clause[},\plc{]]} read \plc{[[},\plc{]}hint(\plc{hint-expression})\plc{]}
             \plc{[[},\plc{]memory-order-clause]}
    \plc{capture-statement }
\plc{[}!$omp end atomic\plc{]}
\end{ompfPragma}

or

\begin{ompfPragma}
!$omp atomic \plc{[memory-order-clause[},\plc{]]} write \plc{[[},\plc{]}hint(\plc{hint-expression})\plc{]}
             \plc{[[},\plc{]memory-order-clause]}
    \plc{write-statement }
\plc{[}!$omp end atomic\plc{]}
\end{ompfPragma}

or

\begin{ompfPragma}
!$omp atomic \plc{[memory-order-clause[},\plc{]]} update \plc{[[},\plc{]}hint(\plc{hint-expression})\plc{]}
             \plc{[[},\plc{]memory-order-clause]}
    \plc{update-statement }
\plc{[}!$omp end atomic\plc{]}
\end{ompfPragma}

or

%\newpage %% HACK
\begin{ompfPragma}
!$omp atomic \plc{[memory-order-clause]} \plc{[[},\plc{]}hint(\plc{hint-expression})\plc{]}
             \plc{[[},\plc{]memory-order-clause]}
    \plc{update-statement}
\plc{[}!$omp end atomic\plc{]}
\end{ompfPragma}

or

\begin{ompfPragma}
!$omp atomic \plc{[memory-order-clause[},\plc{]]} capture \plc{[[},\plc{]}hint(\plc{hint-expression})\plc{]}
             \plc{[[},\plc{]memory-order-clause]}
    \plc{update-statement }
    \plc{capture-statement}
!$omp end atomic
\end{ompfPragma}

or

\begin{ompfPragma}
!$omp atomic \plc{[memory-order-clause[},\plc{]]} capture \plc{[[},\plc{]}hint(\plc{hint-expression})\plc{]}
             \plc{[[},\plc{]memory-order-clause]}
    \plc{capture-statement}
    \plc{update-statement}
!$omp end atomic
\end{ompfPragma}

or

\begin{ompfPragma}
!$omp atomic \plc{[memory-order-clause[},\plc{]]} capture \plc{[[},\plc{]}hint(\plc{hint-expression})\plc{]}
             \plc{[[},\plc{]memory-order-clause]}
    \plc{capture-statement}
    \plc{write-statement}
!$omp end atomic
\end{ompfPragma}

where \plc{write-statement} has the following form (if \plc{atomic-clause}
is \code{capture} or \code{write}):

\begin{indentedcodelist}
\plc{x} = \plc{expr}
\end{indentedcodelist}

where \plc{capture-statement} has the following form (if \plc{atomic-clause}
is \code{capture} or \code{read}):

\begin{indentedcodelist}
\plc{v} = \plc{x}
\end{indentedcodelist}

and where \plc{update-statement} has one of the following forms (if \plc{atomic-clause} is \code{update},
\code{capture}, or not present):

\begin{indentedcodelist}
\plc{x} = \plc{x operator expr}

\plc{x} = \plc{expr operator x}

\plc{x} = \plc{intrinsic_procedure_name} (\plc{x}, \plc{expr_list})

\plc{x} = \plc{intrinsic_procedure_name} (\plc{expr_list}, \plc{x})
\end{indentedcodelist}

In the preceding statements:

\begin{itemize}
\item \plc{x} and \plc{v} (as applicable) are both scalar variables of intrinsic type.

\vspace{-2 pt} %% UGLY HACK
\item \plc{x} must not have the \code{ALLOCATABLE} attribute.

\vspace{-2 pt} %% UGLY HACK
\item During the execution of an atomic region, multiple syntactic occurrences of \plc{x} must
designate the same storage location.

\vspace{-2 pt} %% UGLY HACK
\item None of \plc{v}, \plc{expr}, and \plc{expr_list} (as applicable) may access the same storage location as
\plc{x}.

\vspace{-2 pt} %% UGLY HACK
\item None of \plc{x}, \plc{expr}, and \plc{expr_list} (as applicable) may access the same storage location as
\plc{v}.

\vspace{-2 pt} %% UGLY HACK
\item \plc{expr} is a scalar expression.

\vspace{-2 pt} %% UGLY HACK
\item \plc{expr_list} is a comma-separated, non-empty list of scalar expressions. If
\plc{intrinsic_procedure_name} refers to \code{IAND}, \code{IOR}, or \code{IEOR}, exactly one expression
must appear in \plc{expr_list}.

\vspace{-2 pt} %% UGLY HACK
\item \plc{intrinsic_procedure_name} is one of \code{MAX}, \code{MIN}, \code{IAND}, \code{IOR}, or \code{IEOR}.

\vspace{-2 pt} %% UGLY HACK
\item \plc{operator} is one of \code{+}, \code{*}, \code{-}, \code{/}, \code{.AND.}, \code{.OR.}, \code{.EQV.}, or \code{.NEQV.}.

\vspace{-2 pt} %% UGLY HACK
\item The expression \plc{x operator expr} must be numerically equivalent to \plc{x operator (expr)}.
This requirement is satisfied if the operators in \plc{expr} have precedence greater than
\plc{operator}, or by using parentheses around \plc{expr} or subexpressions of \plc{expr}.

\vspace{-2 pt} %% UGLY HACK
\item The expression \plc{expr operator x} must be numerically equivalent to \plc{(expr) operator  x}. This requirement is satisfied if the operators in \plc{expr} have precedence equal to or
greater than \plc{operator}, or by using parentheses around \plc{expr} or subexpressions of \plc{expr}.

\vspace{-2 pt} %% UGLY HACK
\item \plc{intrinsic_procedure_name} must refer to the intrinsic procedure name and not to other
program entities.

\vspace{-2 pt} %% UGLY HACK
\item \plc{operator} must refer to the intrinsic operator and not to a user-defined operator.

\vspace{-2 pt} %% UGLY HACK
\item All assignments must be intrinsic assignments.

\vspace{-2 pt} %% UGLY HACK
\item For forms that allow multiple occurrences of \plc{x}, the number of times that \plc{x} is
evaluated is unspecified.
%% \item In all \code{atomic} construct forms, the \code{seq_cst} clause and the clause that denotes the
%% type of the atomic construct can appear in any order. In addition, an optional comma
%% may be used to separate the clauses
\end{itemize}
\end{fortranspecific}
%% \end{fortranspecific} %% TODO: this needs to be changed.


\begin{samepage}

\binding
If the size of \plc{x} is 8, 16, 32, or 64 bits and \plc{x} is aligned to a
multiple of its size, the binding thread set for the \code{atomic} region is
all threads on the device. Otherwise, the binding thread set for the
\code{atomic} region is all threads in the contention group.  \code{atomic}
regions enforce exclusive access with respect to other \code{atomic} regions
that access the same storage location \plc{x} among all threads in the
binding thread set without regard to the teams to which the threads belong.

\descr

All \code{atomic} constructs force an atomic operation on the storage location
designated by \plc{x} to be performed by the encountering thread, preceded
and/or followed by implicit flush operations as described in this section. The
implicit flushes are performed as if they are part of the same atomic
operation applied to \plc{x}.  Each non-synchronizable implicit flush is a
strong flush. If \plc{memory-order-clause} is present and is \code{seq_cst},
each synchronizable implicit flush is a strong flush; otherwise,
each synchronizable implicit flush is a weak flush.

The \code{atomic} construct with the \code{read} clause forces an atomic read of the location
designated by \plc{x} regardless of the native machine word size. The atomic
read is immediately preceded by a non-synchronizable read flush of \plc{x}.
\end{samepage}

The \code{atomic} construct with the \code{write} clause forces an atomic write of the location
designated by \plc{x} regardless of the native machine word size. The atomic
write is immediately followed by a non-synchronizable write flush of \plc{x}.

The \code{atomic} construct with the \code{update} clause forces an atomic update of the location
designated by \plc{x} using the designated operator or intrinsic. Note that when no clause is
present, the semantics are equivalent to atomic update. Only the read and write of the
location designated by \plc{x} are performed mutually atomically. The evaluation of \plc{expr} or
\plc{expr_list} need not be atomic with respect to the read or write of the location designated
by \plc{x}. No task scheduling points are allowed between the read and the write of the
location designated by \plc{x}. The atomic update is immediately preceded by
a non-synchronizable read flush of \plc{x} and immediately followed by a
non-synchronizable write flush of \plc{x}.

The \code{atomic} construct with the \code{capture} clause forces an atomic
captured update --- an atomic update of the
location designated by \plc{x} using the designated operator or intrinsic while also capturing
the original or final value of the location designated by \plc{x} with respect to the atomic
update. The original or final value of the location designated by \plc{x} is written in the
location designated by \plc{v} depending on the form of the \code{atomic} construct structured
block or statements following the usual language semantics. Only the read and write of
the location designated by \plc{x} are performed mutually atomically. Neither the evaluation
of \plc{expr} or \plc{expr_list}, nor the write to the location designated by \plc{v}, need be atomic with
respect to the read or write of the location designated by \plc{x}. No task scheduling points
are allowed between the read and the write of the location designated by
\plc{x}. The atomic captured update is immediately preceded by a
non-synchronizable read flush of \plc{x} and immediately followed by a
non-synchronizable write flush of \plc{x}.

A release flush operation applied to all variables, with a
sync-set containing \plc{x}, is implied at entry to the atomic operation when
the \code{read} clause is not present and the \code{release}, \code{acq_rel},
or \code{seq_cst} clause is present.  An acquire flush
operation applied to all variables, with a sync-set containing \plc{x}, is
implied at exit from the atomic operation when the \code{read} or
\code{capture} clause is present and the \code{acquire}, \code{acq_rel}, or
\code{seq_cst} clause is present.

% \begin{note}
% As with other implicit flush regions,
% \specref{subsec:OpenMP Memory Consistency}
% reduces the
% ordering that must be enforced. The intent is that, when the analogous operation exists
% in C++11 or C11, a sequentially consistent \code{atomic} construct has the same semantics as
% a \code{memory_order_seq_cst} atomic operation in C++11/C11. Similarly, a
% non-sequentially consistent \code{atomic} construct on which
% \plc{memory-order-clause} is not present has the same semantics as a
% \code{memory_order_relaxed} atomic operation in C++11/C11.
%
% Unlike non-sequentially consistent \code{atomic} constructs, sequentially consistent \code{atomic}
% constructs preserve the interleaving (sequentially consistent) behavior of correct,
% data race free programs. However, they are not designed to replace the \code{flush} directive
% as a mechanism to enforce ordering for non-sequentially consistent \code{atomic} constructs,
% and attempts to do so require extreme caution. For example, a sequentially consistent
% \code{atomic}~\code{write} construct may appear to be reordered with a subsequent
% non-sequentially consistent \code{atomic}~\code{write} construct that modifies
% a different variable, since such reordering would not be observable by a
% correct program if the second write were outside an \code{atomic} directive.
% \end{note}

For all forms of the \code{atomic} construct, any combination of two or more
of these \code{atomic} constructs enforces mutually exclusive access to the
locations designated by \plc{x} among threads in the binding thread set.  To
avoid race conditions, all accesses of the locations designated by \plc{x}
that could potentially occur in parallel must be protected with an
\code{atomic} construct.

\code{atomic} regions do not guarantee exclusive access with respect to any accesses outside
of \code{atomic} regions to the same storage location \plc{x} even if those accesses occur during a
\code{critical} or \code{ordered} region, while an OpenMP lock is owned by the executing
task, or during the execution of a \code{reduction} clause.

However, other OpenMP synchronization can ensure the desired exclusive access. For
example, a barrier following a series of atomic updates to \plc{x} guarantees that subsequent
accesses do not form a race with the atomic accesses.

A compliant implementation may enforce exclusive access between \code{atomic} regions
that update different storage locations. The circumstances under which this occurs are
implementation defined.

If the storage location designated by \plc{x} is not size-aligned (that is, if the byte alignment
of \plc{x} is not a multiple of the size of \plc{x}), then the behavior of the \code{atomic} region is
implementation defined.

If present, the \code{hint} clause gives the implementation additional
information about the expected properties of the atomic operation
that can optionally be used to optimize the implementation.
The presence of a \code{hint} clause does not affect the semantics of
the \code{atomic} construct, and it is legal to ignore all hints.
If no \code{hint} clause is specified, the effect is as if \code{hint(omp_sync_hint_none)} had been specified.

\needspace{16\baselineskip}\begin{samepage}

\omptMutex{atomic}{an}

\restrictions

The following restrictions apply to the \code{atomic} construct:

\begin{itemize}
\item At most one \plc{memory-order-clause} may appear on the construct.

\item If \plc{atomic-clause} is \code{read} then \plc{memory-order-clause}
    must not be \code{acq_rel} or \code{release}.

\item If \plc{atomic-clause} is \code{write} then \plc{memory-order-clause}
    must not be \code{acq_rel} or \code{acquire}.

\item If \plc{atomic-clause} is \code{update} or not present then \plc{memory-order-clause}
    must not be \code{acq_rel} or \code{acquire}.
\end{itemize}

\newpage  %% HACK
\begin{ccppspecific}
\begin{itemize}
\item All atomic accesses to the storage locations designated by \plc{x} throughout the program
are required to have a compatible type.
\end{itemize}
\end{ccppspecific}
\end{samepage}

\begin{fortranspecific}
\begin{itemize}
\item All atomic accesses to the storage locations designated by \plc{x} throughout the program
are required to have the same type and type parameters.
\end{itemize}
\end{fortranspecific}

% Do we want this? I do not believe that we need it. This is not like
% a critical section where different lock implementations might be used
% by lexically different critical sections that name the same
% lock. Here all implementations must inter-operate anyway.
% \begin{itemize}
% \item When a \code{hint} is present at the construct, all \code{atomic} constructs which
% may simultaneously access the storage locations designated by
% \plc{x} must have a hint expression that evaluates to the same value.
% \end{itemize}

\begin{itemize}
\item OpenMP constructs may not be encountered during execution of an
\code{atomic} region.
\end{itemize}


\crossreferences
\begin{itemize}
\item \code{critical} construct, see
\specref{subsec:critical Construct}.

\item \code{barrier} construct, see
\specref{subsec:barrier Construct}.

\item \code{flush} construct, see
\specref{subsec:flush Construct}.

\item \code{ordered} construct, see
\specref{subsec:ordered Construct}.

\item \code{reduction} clause, see
\specref{subsubsec:reduction clause}.

\item lock routines, see
\specref{sec:Lock Routines}.

\item Synchronization Hints, see
\specref{subsec:Synchronization Hints}.

\item \code{ompt_mutex_atomic}, see
\specref{sec:ompt_mutex_t}.

\item \code{ompt_callback_mutex_acquire_t}, see
\specref{sec:ompt_callback_mutex_acquire_t}.

\item \code{ompt_callback_mutex_t}, see
\specref{sec:ompt_callback_mutex_t}.

\end{itemize}









\subsection{\hcode{flush} Construct}
\index{flush@{\code{flush}}}
\index{constructs!flush@{\code{flush}}}
\label{subsec:flush Construct}
\summary
The \code{flush} construct executes the OpenMP flush operation. This operation makes a
thread's temporary view of memory consistent with memory and enforces an order on
the memory operations of the variables explicitly specified or implied. See the memory
model description in \specref{sec:Memory Model} for more details. The \code{flush} construct is a
stand-alone directive.

\syntax
\begin{ccppspecific}
The syntax of the \code{flush} construct is as follows:

\begin{ompcPragma}
#pragma omp flush \plc{[memory-order-clause]} \plc{[}(\plc{list})\plc{] new-line}
\end{ompcPragma}
\begin{samepage}
where \plc{memory-order-clause} is one of the following:

\begin{indentedcodelist}
acq_rel
release
acquire
\end{indentedcodelist}
\end{samepage}
\end{ccppspecific}

\begin{fortranspecific}
The syntax of the \code{flush} construct is as follows:

\begin{ompfPragma}
!$omp flush \plc{[memory-order-clause]} \plc{[}(\plc{list})\plc{]}
\end{ompfPragma}
\begin{samepage}
where \plc{memory-order-clause} is one of the following:

\begin{indentedcodelist}
acq_rel
release
acquire
\end{indentedcodelist}
\end{samepage}
\end{fortranspecific}

\binding
The binding thread set for a \code{flush} region is the encountering thread. Execution of a
\code{flush} region affects the memory and the temporary view of memory of only the thread
that executes the region. It does not affect the temporary view of other threads. Other
threads must themselves execute a flush operation in order to be guaranteed to observe
the effects of the encountering thread's flush operation

\descr
A \code{flush} construct without a list, executed on a given thread, operates as if the whole
thread-visible data state of the program, as defined by the base language, is flushed. A
\code{flush} construct with a list applies the flush operation to the items in the list, and does
not return until the operation is complete for all specified list items. An implementation
may implement a \code{flush} with a list by ignoring the list, and treating it the same as a
\code{flush} without a list.

If list items are specified on the \code{flush} construct, the flush operation
is a non-synchronizable flush.  Otherwise, the flush operation is a
synchronizable flush for which the sync-set contains all variables that are
flushed.

The flush operation's flush properties are determined according to
\plc{memory-order-clause}, if present:

\begin{itemize}
\item If \plc{memory-order-clause} is not specified, the flush operation has the
strong, write, and read flush properties and is both a release flush and an
acquire flush if the flush does not have a list.

\item If \plc{memory-order-clause} is \code{acq_rel}, the flush operation has the
write and read flush properties and is both a release flush and an acquire flush.

\item If \plc{memory-order-clause} is \code{release}, the flush operation has
the write flush property and is a release flush.

\item If \plc{memory-order-clause} is \code{acquire}, the flush operation has
the read flush property and is an acquire flush.
\end{itemize}


\begin{ccppspecific}
If a pointer is present in the list, the pointer itself is flushed, not the memory block to
which the pointer refers.
\end{ccppspecific}

\begin{fortranspecific}
If the list item or a subobject of the list item has the \code{POINTER} attribute, the allocation
or association status of the \code{POINTER} item is flushed, but the pointer target is not. If the
list item is a Cray pointer, the pointer is flushed, but the object to which it points is not.
If the list item is of type \code{C_PTR}, the variable is flushed, but the storage that corresponds
to that address is not flushed. If the list item or the subobject of the list item has the
\code{ALLOCATABLE} attribute and has an allocation status of allocated, the
allocated variable is flushed; otherwise the allocation status is flushed.
\end{fortranspecific}

\begin{note}
Use of a \code{flush} construct with a list is extremely error prone and users are
strongly discouraged from attempting it. The following examples illustrate the ordering
properties of the flush operation. In the following incorrect pseudocode example, the
programmer intends to prevent simultaneous execution of the protected section by the
two threads, but the program does not work properly because it does not enforce the
proper ordering of the operations on variables \code{a} and \code{b}. Any shared data accessed in the
protected section is not guaranteed to be current or consistent during or after the
protected section. The atomic notation in the pseudocode in the following two examples
indicates that the accesses to \code{a} and \code{b} are \code{ATOMIC} writes and captures. Otherwise both
examples would contain data races and automatically result in unspecified behavior.

% Outlined mixed code and text:

\parbox{\linewidth}{%
\begin{spacing}{0.90}\begin{framed}
\emph{Incorrect example:}\\
\hspace{0.3\textwidth}\code{a = b = 0}
\begin{tabular}{ p{0.5\textwidth} p{0.5\textwidth}}\\
\hspace{0.1\textwidth}{\splc{thread 1}} & \hspace{0.1\textwidth}{\splc{thread 2}}\\[1.0em]
{\scode{atomic(b = 1)}} & {\scode{atomic(a = 1)}}\\
{\splc{flush}\scode{(b)}} & {\splc{flush}\scode{(a)}}\\
{\splc{flush}\scode{(a)}} & {\splc{flush}\scode{(b)}}\\
{\scode{atomic(tmp = a)}} & {\scode{atomic(tmp = b)}}\\
{\scode{if (tmp == 0) then}} & {\scode{if (tmp == 0) then}}\\
\hspace{1.25em}{\splc{protected section}} & \hspace{1.25em}{\splc{protected section}}\\
{\scode{end if}} & {\scode{end if}}\\
\end{tabular}
\end{framed}\end{spacing}} % use \parbox to keep the lines together for code only

The problem with this example is that operations on variables \code{a} and \code{b} are not ordered
with respect to each other. For instance, nothing prevents the compiler from moving the
flush of \code{b} on thread 1 or the flush of \code{a} on thread 2 to a position completely after the
protected section (assuming that the protected section on thread 1 does not reference \code{b} and
the protected section on thread 2 does not reference \code{a}). If either re-ordering happens, both
threads can simultaneously execute the protected section.

The following pseudocode example correctly ensures that the protected section is executed
by not more than one of the two threads at any one time. Execution of the
protected section by neither thread is considered correct in this example. This occurs if
both flushes complete prior to either thread executing its \code{if} statement.

\parbox{\linewidth}{%
\begin{spacing}{0.95}\begin{framed}
\emph{Correct example:}\\
\hspace{0.3\textwidth}\code{a = b = 0}
\begin{tabular}{ p{0.5\textwidth} p{0.5\textwidth}}\\
\hspace{0.1\textwidth}{\splc{thread 1}} & \hspace{0.1\textwidth}{\splc{thread 2}}\\[1.0em]
{\scode{atomic(b = 1)}} & {\scode{atomic(a = 1)}}\\
{\splc{flush}\scode{(a,b)}} & {\splc{flush}\scode{(a,b)}}\\
{\scode{atomic(tmp = a)}} & {\scode{atomic(tmp = b)}}\\
{\scode{if (tmp == 0) then}} & {\scode{if (tmp == 0) then}}\\
\hspace{1.25em}{\splc{protected section}} & \hspace{1.25em}{\splc{protected section}}\\
{\scode{end if}} & {\scode{end if}}\\
\end{tabular}
\end{framed}\end{spacing}} % use \parbox to keep lines together for code only
\bigskip

The compiler is prohibited from moving the flush at all for either thread, ensuring that the
respective assignment is complete and the data is flushed before the \code{if} statement is
executed.
\end{note}
\bigskip

Flush operations implied when executing an \code{atomic} region are described in
Section \ref{subsec:atomic Construct}.

A \code{flush} region arising from a \code{flush} directive without a list and
without \plc{memory-order-clause} present is implied at the following locations:
\begin{itemize}
\item During a barrier region.
\item At entry to a \code{target update} region whose corresponding construct has a \code{to} clause.
\item At exit from a \code{target update} region whose corresponding construct has a \code{from} clause.
\item At entry to and exit from \code{parallel}, \code{critical}, \code{target} and \code{target data} regions.
\item At entry to and exit from an \code{ordered} region, if a \code{threads} clause or a \code{depend} clause is present, or if no clauses are present.
\item At entry to a \code{target enter data} region.
\item At exit from a \code{target exit data} region.
\item At exit from worksharing regions unless a \code{nowait} is present.
\item During \code{omp_set_lock} and \code{omp_unset_lock} regions.
\item During \code{omp_test_lock}, \code{omp_set_nest_lock}, \code{omp_unset_nest_lock}
and \code{omp_test_nest_lock} regions, if the region causes the lock to be set or unset.
\item Immediately before and immediately after every task scheduling point.
\item During a \code{cancel} or \code{cancellation point} region, if the \plc{cancel-var} ICV is \plc{true} and cancellation has been activated.
\end{itemize}

\begin{note}
A \code{flush} region is not implied at the following locations:

\begin{itemize}
\item At entry to worksharing regions.

\item At entry to or exit from a \code{master} region.
\end{itemize}
\end{note}

\events

The \plc{flush} event occurs in a thread encountering the
\code{flush} construct.

\tools

A thread dispatches a registered \code{ompt_callback_flush} callback
for each occurrence of a \plc{flush} event in that thread. This
callback has the type signature \code{ompt_callback_flush_t}.

\restrictions

The following restrictions apply to the \code{flush} construct:

\begin{itemize}
\item If \plc{memory-order-clause} is \code{release}, \code{acquire}, or
    \code{acq_rel}, list items must not be specified on the \code{flush} directive.
\end{itemize}

\crossreferences
\begin{itemize}
\item \code{ompt_callback_flush_t}, see
\specref{sec:ompt_callback_flush_t}.
\end{itemize}





\subsection{\hcode{ordered} Construct}
\index{ordered@{\code{ordered}}}
\index{constructs!ordered@{\code{ordered}}}
\label{subsec:ordered Construct}
\summary
The \code{ordered} construct either specifies a structured block in a loop,
\code{simd}, or loop SIMD region that will be executed in the order of the
loop iterations, or it is a stand-alone directive that specifies
cross-iteration dependences in a doacross loop nest. The \code{ordered}
construct sequentializes and orders the execution of \code{ordered} regions
while allowing code outside the region to run in parallel.

\begin{samepage}
\syntax
\begin{ccppspecific}
The syntax of the \code{ordered} construct is as follows:

\begin{ompcPragma}
#pragma omp ordered \plc{[clause[ [},\plc{] clause] ]} \plc{new-line}
   \plc{structured-block}
\end{ompcPragma}

where \plc{clause} is one of the following:
\begin{indentedcodelist}
threads
simd
\end{indentedcodelist}

or

\begin{ompcPragma}
#pragma omp ordered \plc{clause [[[},\plc{] clause] ... ]} \plc{new-line}
\end{ompcPragma}
where \plc{clause} is one of the following:
\begin{indentedcodelist}
depend(source)
depend(sink : \plc{vec})
\end{indentedcodelist}


\end{ccppspecific}
\end{samepage}

\begin{fortranspecific}
The syntax of the \code{ordered} construct is as follows:

\begin{ompfPragma}
!$omp ordered \plc{[clause[ [},\plc{] clause] ]}
    \plc{structured-block}
!$omp end ordered
\end{ompfPragma}

where \plc{clause} is one of the following:
\begin{indentedcodelist}
threads
simd
\end{indentedcodelist}

or

\begin{ompfPragma}
!$omp ordered \plc{clause [[[},\plc{] clause] ... ]}
\end{ompfPragma}

where \plc{clause} is one of the following:
\begin{indentedcodelist}
depend(source)
depend(sink : \plc{vec})
\end{indentedcodelist}
\end{fortranspecific}

If the \code{depend} clause is specified, the \code{ordered} construct is a stand-alone directive.

\binding
The binding thread set for an \code{ordered} region is the current team. An \code{ordered} region
binds to the innermost enclosing \code{simd} or loop SIMD region if the
\code{simd} clause is present, and otherwise it binds to the innermost
enclosing loop region. \code{ordered} regions that bind to different regions
execute independently of each other.

\descr
If no clause is specified, the \code{ordered} construct behaves as if the
\code{threads} clause had been specified. If the \code{threads} clause is
specified, the threads in the team executing the loop region execute
\code{ordered} regions sequentially in the order of the loop iterations.
If any \code{depend} clauses are specified then those clauses specify the
order in which the threads in the team execute \code{ordered} regions. If
the \code{simd} clause is specified, the \code{ordered} regions encountered by
any thread will use only a single SIMD lane to execute the \code{ordered}
regions in the order of the loop iterations.

When the thread executing the first iteration of the loop encounters an
\code{ordered} construct, it can enter the \code{ordered} region without
waiting. When a thread executing any subsequent iteration encounters an
\code{ordered} construct without a \code{depend} clause, it waits at the
beginning of the \code{ordered} region until execution of all \code{ordered}
regions belonging to all previous iterations has completed. When a thread
executing any subsequent iteration encounters an \code{ordered} construct with
one or more {\pcode{depend(sink:\plc{vec})}} clauses, it waits until its dependences on
all valid iterations specified by the \code{depend} clauses
are satisfied before it completes execution of the \code{ordered} region.
A specific dependence is satisfied when a thread executing the corresponding
iteration encounters an \code{ordered} construct with a \code{depend(source)} clause.

\omptMutex{ordered}{an}


\restrictions
Restrictions to the \code{ordered} construct are as follows:

\begin{itemize}
\item At most one \code{threads} clause can appear on an \code{ordered} construct.

\item At most one \code{simd} clause can appear on an \code{ordered} construct.

\item At most one \code{depend(source)} clause can appear on an \code{ordered} construct.

\item Either {\pcode{depend(sink:\plc{vec})}} clauses or \code{depend(source)}
clauses may appear on an \code{ordered} construct, but not both.

\item The loop or loop SIMD region to which an \code{ordered}
region arising from an \code{ordered} construct without a \code{depend}
clause binds must have an \code{ordered} clause without the parameter
specified on the corresponding loop or loop SIMD directive.

\item The loop region to which an \code{ordered} region arising from an
\code{ordered} construct with any \code{depend}
clauses binds must have an \code{ordered} clause with the parameter specified
on the corresponding loop directive.

\item An \code{ordered} construct with the \code{depend} clause specified must
be closely nested inside a loop (or parallel loop) construct.

\item An \code{ordered} region arising from an \code{ordered} construct with
the \code{simd} clause specified must be closely nested inside a
\code{simd} or loop SIMD region.

\item An \code{ordered} region arising from an \code{ordered} construct with
  both the \code{simd} and \code{threads} clauses must be closely nested inside
  a loop SIMD region.

\item During execution of an iteration of a loop or a loop nest within a loop, \code{simd}, or loop SIMD
region, a thread must not execute more than one \code{ordered} region arising
from an \code{ordered} construct without a \code{depend} clause.
\end{itemize}
\begin{cppspecific}
\begin{itemize}
\item A throw executed inside a \code{ordered} region must cause execution to
resume within the same \code{ordered} region, and the same thread that threw
the exception must catch it.
\end{itemize}
\end{cppspecific}



\crossreferences
\begin{itemize}
\item loop construct, see
\specref{subsec:Loop Construct}.

\item \code{simd} construct, see
\specref{subsec:simd Construct}.

\item parallel loop construct, see
\specref{subsec:Parallel Loop Construct}.

\item \code{depend} Clause, see
\specref{subsec:depend Clause}

\item \code{ompt_mutex_ordered}, see
\specref{sec:ompt_mutex_t}.

\item \code{ompt_callback_mutex_acquire_t}, see
\specref{sec:ompt_callback_mutex_acquire_t}.

\item \code{ompt_callback_mutex_t}, see
\specref{sec:ompt_callback_mutex_t}.

\end{itemize}


\subsection{\hcode{depend} Clause}
\index{depend@{\code{depend}}}
\index{clauses!depend@{\code{depend}}}
\label{subsec:depend Clause}
\summary
The \code{depend} clause enforces additional constraints on the scheduling of tasks or loop iterations.  These
constraints establish dependences only between sibling tasks or between loop iterations.

\syntax
The syntax of the \code{depend} clause is as follows:

\begin{ompSyntax}
depend(\plc{dependence-type }:\plc{ locator-list}[:\plc{iterators-definition}])
\end{ompSyntax}

where \plc{dependence-type} is one of the following:
\begin{indentedcodelist}
in
out
inout
mutexinoutset
\end{indentedcodelist}

or

\begin{ompSyntax}
depend(\plc{dependence-type})
\end{ompSyntax}

where \plc{dependence-type} is:
\begin{indentedcodelist}
source
\end{indentedcodelist}

or

\begin{ompSyntax}
depend(\plc{dependence-type} : \plc{vec})
\end{ompSyntax}

where \plc{dependence-type} is:
\begin{indentedcodelist}
sink
\end{indentedcodelist}

and where \plc{vec} is the iteration vector, which has the form:

x\textsubscript{1} [$\pm $ d\textsubscript{1}], x\textsubscript{2} [$\pm $ d\textsubscript{2}], \ldots, x\textsubscript{\plc{n}} [$\pm $ d\textsubscript{\plc{n}}]

where \plc{n} is the value specified by the \code{ordered} clause in the loop
directive, x\textsubscript{\plc{i}} denotes the loop iteration variable of the \plc{i}-th
nested loop associated with the loop directive, and d\textsubscript{\plc{i}} is a
constant non-negative integer.

\descr
Task dependences are derived from the \plc{dependence-type} of a \code{depend} clause and its list
items when \plc{dependence-type} is \code{in}, \code{out}, \code{inout}, or \code{mutexinoutset}.

For the \code{in} \plc{dependence-type}, if the storage location of at least one
of the list items is the same as the storage location of a list item appearing
in a \code{depend} clause with an \code{out}, \code{inout}, or \code{mutexinoutset} \plc{dependence-type} on a construct
from which a sibling task was previously generated, then the generated task
will be a dependent task of that sibling task.

For the \code{out} and \code{inout} \plc{dependence-types}, if the storage location of at least one
of the list items is the same as the storage location of a list item appearing
in a \code{depend} clause with an \code{in}, \code{out}, \code{inout}, or \code{mutexinoutset} \plc{dependence-type} on
a construct from which a sibling task was previously generated, then the
generated task will be a dependent task of that sibling task.

For the \code{mutexinoutset} \plc{dependence-type}, if the storage location of at least one
of the list items is the same as the storage location of a list item appearing
in a \code{depend} clause with an \code{in}, \code{out}, or \code{inout} \plc{dependence-type} on
a construct from which a sibling task was previously generated, then the
generated task will be a dependent task of that sibling task.

If a list item appearing in a \code{depend} clause with a \code{mutexinoutset}
\plc{dependence-type} on a task-generating construct has the same storage location as
a list item appearing in a \code{depend} clause with a \code{mutexinoutset}
\plc{dependence-type} on a different task generating construct, and both constructs
generate sibling tasks, the sibling tasks will be mutually exclusive tasks.

The list items that appear in the \code{depend} clause may reference iterators
defined by an \plc{iterators-definition} appearing on the same clause.

\begin{fortranspecific}
If a list item has the \code{ALLOCATABLE} attribute and its allocation
status is unallocated, the behavior is unspecified. If a list item has
the \code{POINTER} attribute and its association status is
disassociated or undefined, the behavior is unspecified.
\end{fortranspecific}

The list items that appear in the \code{depend} clause may include array sections.

\begin{note}
The enforced task dependence establishes a synchronization of memory
accesses performed by a dependent task with respect to accesses performed by the
predecessor tasks. However, it is the responsibility of the programmer to synchronize properly with respect to other concurrent accesses that occur outside of those tasks.
\end{note}

The \code{source} \plc{dependence-type} specifies the satisfaction of
cross-iteration dependences that arise from the current iteration.

The \code{sink} \plc{dependence-type} specifies a cross-iteration dependence,
where the iteration vector \plc{vec} indicates the iteration that satisfies
the dependence.

If the iteration vector \plc{vec} does not occur in the iteration space,
the \code{depend} clause is ignored.  If all \code{depend} clauses on an
\code{ordered} construct are ignored then the construct is ignored.

\begin{note}
If the iteration vector \plc{vec} does not indicate a lexicographically earlier iteration, it can cause a deadlock.
\end{note}

\events

The \plc{task-dependences} event occurs in a thread encountering a
tasking construct with a \code{depend} clause immediately after the
\plc{task-create} event for the new task.

The \plc{task-dependence} event indicates an unfulfilled dependence for the generated task.
This event occurs in a thread that observes the unfulfilled dependence before it is satisfied.
%unfulfilled dependence... already implies that if the dependence is satisfied (not unfulfilled)
% then there is no events... so can skip the sentence below.
%A dependence will not cause an event if the
%dependence producing task finishes before a dependence consuming task is created.

\tools

A thread dispatches the \code{ompt_callback_task_dependences} callback
for each occurrence of the \plc{task-dependences} event to
announce its dependences with respect to the list items in the \code{depend} clause.
This callback has type signature
\code{ompt_callback_task_dependences_t}.

A thread dispatches the \code{ompt_callback_task_dependence}
callback for a \plc{task-dependence} event to report a
dependence between a predecessor task  (\plc{src_task_data}) and a dependent task
(\plc{sink_task_data}).  This callback has type signature
\code{ompt_callback_task_dependence_t}.

\restrictions
Restrictions to the \code{depend} clause are as follows:

\begin{itemize}
\item List items used in \code{depend} clauses of the same task or sibling tasks must indicate
identical storage locations or disjoint storage locations.

\item List items used in \code{depend} clauses cannot be zero-length array sections.

\begin{fortranspecific}
\item A common block name cannot appear in a \code{depend} clause.
\end{fortranspecific}

\item For a \plc{vec} element of \code{sink} \plc{dependence-type} of the form
x\textsubscript{i} $+$ d\textsubscript{i} or x\textsubscript{i} $-$ d\textsubscript{i} if the loop iteration variable
x\textsubscript{i} has an integral or pointer type, the expression x\textsubscript{i} $+$ d\textsubscript{i} or
x\textsubscript{i} $-$ d\textsubscript{i} for any value of the loop iteration variable x\textsubscript{i} that
can encounter the \code{ordered} construct must be computable in the
loop iteration variable's type without overflow.

\begin{cppspecific}
\item For a \plc{vec} element of \code{sink} \plc{dependence-type} of the form
x\textsubscript{i} $+$ d\textsubscript{i} or x\textsubscript{i} $-$ d\textsubscript{i} if the loop iteration variable
x\textsubscript{i} is of a random access iterator type other than pointer type,
the expression $($ x\textsubscript{i} - lb\textsubscript{i} $)$ $+$ d\textsubscript{i} or
$($ x\textsubscript{i} - lb\textsubscript{i} $)$ $-$ d\textsubscript{i} for any value of the loop iteration variable
x\textsubscript{i} that can encounter the \code{ordered} construct must be computable in the
type that would be used by \plc{std::distance} applied to variables of the
type of x\textsubscript{i} without overflow.
\end{cppspecific}

\begin{ccppspecific}
\item A bit-field cannot appear in a \code{depend} clause.
\end{ccppspecific}

\end{itemize}

\crossreferences
\begin{itemize}
\item Array sections, see
\specref{sec:Array Sections}.

\item \code{task} construct, see
\specref{subsec:task Construct}.

\item \code{target}~\code{enter}~\code{data} construct, see
\specref{subsec:target enter data Construct}.

\item \code{target}~\code{exit}~\code{data} construct, see
\specref{subsec:target exit data Construct}.

\item \code{target} construct, see
\specref{subsec:target Construct}.

\item \code{target}~\code{update} construct, see
\specref{subsec:target update Construct}.

\item Task scheduling constraints, see
\specref{subsec:Task Scheduling}.

\item \code{ordered} construct, see
\specref{subsec:ordered Construct}.

\item Iterators, see
\specref{sec:iterators}.

\item \code{ompt_callback_task_dependences_t}, see
\specref{sec:ompt_callback_task_dependences_t}.

\item \code{ompt_callback_task_dependence_t}, see
\specref{sec:ompt_callback_task_dependence_t}.
\end{itemize}

\subsection{Synchronization Hints}
\index{synchronization hints}
\index{clauses!hint@{\code{hint}}}
\label{subsec:Synchronization Hints}
Hints about the expected dynamic behavior or suggested implementation
can be provided by the programmer to locks (by using
the \code{omp_init_lock_with_hint} or
\code{omp_init_nest_lock_with_hint} functions to initialize the lock), and to
\code{atomic} and \code{critical} directives by using the \code{hint}
clause.  The effect of a hint is implementation defined. The OpenMP
implementation is free to ignore the hint since doing so cannot change
program semantics.

The C/C++ header file (\code{omp.h}) and the Fortran include file (\code{omp_lib.h}) and/or Fortran~90 module file (\code{omp_lib}) define the valid hint constants.
The valid constants must include the following, which can be extended with implementation-defined values:

\begin{ccppspecific}
\begin{ompcEnum}
typedef enum omp_sync_hint_t {
  omp_sync_hint_none = 0,
  omp_lock_hint_none = omp_sync_hint_none,
  omp_sync_hint_uncontended = 1,
  omp_lock_hint_uncontended = omp_sync_hint_uncontended,
  omp_sync_hint_contended = 2,
  omp_lock_hint_contended = omp_sync_hint_contended,
  omp_sync_hint_nonspeculative = 4,
  omp_lock_hint_nonspeculative = omp_sync_hint_nonspeculative,
  omp_sync_hint_speculative = 8
  omp_lock_hint_speculative = omp_sync_hint_speculative
} omp_sync_hint_t;

typedef omp_sync_hint_t omp_lock_hint_t;
\end{ompcEnum}
\end{ccppspecific}

\begin{fortranspecific}
\begin{ompfEnum}
integer, parameter :: omp_lock_hint_kind = omp_sync_hint_kind

integer (kind=omp_sync_hint_kind), &
  parameter :: omp_sync_hint_none = 0
integer (kind=omp_lock_hint_kind), &
  parameter :: omp_lock_hint_none = omp_sync_hint_none
integer (kind=omp_sync_hint_kind), &
  parameter :: omp_sync_hint_uncontended = 1
integer (kind=omp_lock_hint_kind), &
  parameter :: omp_lock_hint_uncontended = &
                    omp_sync_hint_uncontended
integer (kind=omp_sync_hint_kind), &
  parameter :: omp_sync_hint_contended = 2
integer (kind=omp_lock_hint_kind), &
  parameter :: omp_lock_hint_contended = &
                   omp_sync_hint_contended
integer (kind=omp_sync_hint_kind), &
  parameter :: omp_sync_hint_nonspeculative = 4
integer (kind=omp_lock_hint_kind), &
  parameter :: omp_lock_hint_nonspeculative = &
                   omp_sync_hint_nonspeculative
integer (kind=omp_sync_hint_kind), &
  parameter :: omp_sync_hint_speculative = 8
integer (kind=omp_lock_hint_kind), &
  parameter :: omp_lock_hint_speculative = &
                   omp_sync_hint_speculative
\end{ompfEnum}
\end{fortranspecific}

The hints can be combined by using the \code{+} or \code{|} operators in C/C++ or the \code{+} operator in Fortran.
The effect of the combined hint is implementation defined and can be ignored by the implementation.
Combining \code{omp_sync_hint_none} with any other hint is equivalent to specifying the other hint.
The following restrictions apply to combined hints; violating these
restrictions results in unspecified behavior:
\begin{itemize}
\item the hints \code{omp_sync_hint_uncontended} and \code{omp_sync_hint_contended} cannot be combined,
\item the hints \code{omp_sync_hint_nonspeculative} and \code{omp_sync_hint_speculative} cannot be combined.
\end{itemize}
The rules for combining multiple values of \code{omp_sync_hint} apply equally to the
corresponding values of \code{omp_lock_hint}, and expressions mixing the two types.

The intended meaning of hints is
\begin{itemize}
\item \code{omp_sync_hint_uncontended}: low contention is expected in this operation, that is,
few threads are expected to be performing the operation simultaneously in a manner that requires synchronization.
\item \code{omp_sync_hint_contended}: high contention is expected in this operation, that is,
many threads are expected to be performing the operation simultaneously in a manner that requires synchronization.
\item \code{omp_sync_hint_speculative}: the programmer suggests that the operation should be implemented using
speculative techniques such as transactional memory.
\item \code{omp_sync_hint_nonspeculative}: the programmer suggests that the operation should not be
implemented using speculative techniques such as transactional memory.
\end{itemize}

\begin{note}
Future OpenMP specifications may add additional hints to the
\code{omp_sync_hint_t} type and the \code{omp_sync_hint_kind} kind.
Implementers are advised to add implementation-defined hints starting from
the most significant bit of the \code{omp_sync_hint_t} type and
\code{omp_sync_hint_kind} kind and to include the name of the
implementation in the name of the added hint to avoid name conflicts
with other OpenMP implementations.
\end{note}

The \code{omp_sync_hint_t} and \code{omp_lock_hint_t} enumeration types and the equivalent types in Fortran
are synonyms for each other.
The type \code{omp_lock_hint_t} has been deprecated.

\crossreferences
\begin{itemize}
\item \code{atomic} construct, see
\specref{subsec:atomic Construct}

\item \code{critical} construct, see
\specref{subsec:critical Construct}.
\item \code{omp_init_lock_with_hint} and
  \code{omp_init_nest_lock_with_hint}, see
\specref{subsec:omp_init_lock_with_hint and omp_init_nest_lock_with_hint}.
\end{itemize}

% This is an included file. See the master file for more information.
%
% When editing this file:
%
%    1. To change formatting, appearance, or style, please edit openmp.sty.
%
%    2. Custom commands and macros are defined in openmp.sty.
%
%    3. Be kind to other editors -- keep a consistent style by copying-and-pasting to
%       create new content.
%
%    4. We use semantic markup, e.g. (see openmp.sty for a full list):
%         \code{}     % for bold monospace keywords, code, operators, etc.
%         \plc{}      % for italic placeholder names, grammar, etc.
%
%    5. There are environments that provide special formatting, e.g. language bars.
%       Please use them whereever appropriate.  Examples are:
%
%         \begin{fortranspecific}
%         This is text that appears enclosed in blue language bars for Fortran.
%         \end{fortranspecific}
%
%         \begin{note}
%         This is a note.  The "Note -- " header appears automatically.
%         \end{note}
%
%    6. Other recommendations:
%         Use the convenience macros defined in openmp.sty for the minor headers
%         such as Comments, Syntax, etc.
%
%         To keep items together on the same page, prefer the use of
%         \begin{samepage}.... Avoid \parbox for text blocks as it interrupts line numbering.
%         When possible, avoid \filbreak, \pagebreak, \newpage, \clearpage unless that's
%         what you mean. Use \needspace{} cautiously for troublesome paragraphs.
%
%         Avoid absolute lengths and measures in this file; use relative units when possible.
%         Vertical space can be relative to \baselineskip or ex units. Horizontal space
%         can be relative to \linewidth or em units.
%
%         Prefer \emph{} to italicize terminology, e.g.:
%             This is a \emph{definition}, not a placeholder.
%             This is a \plc{var-name}.
%


\section{Cancellation Constructs}
\label{sec:Cancellation Constructs}
\index{cancellation constructs}
\index{constructs!cancellation constructs}
\subsection{\hcode{cancel} Construct}
\index{cancel@{\code{cancel}}}
\index{constructs!cancel@{\code{cancel}}}
\index{cancellation constructs!cancel@{\code{cancel}}}
\label{subsec:cancel Construct}
\summary
The \code{cancel} construct activates cancellation of the innermost enclosing region of the
type specified. The \code{cancel} construct is a stand-alone directive.

\syntax
\begin{ccppspecific}
The syntax of the \code{cancel} construct is as follows:

\begin{ompcPragma}
#pragma omp cancel \plc{construct-type-clause [ [},\plc{] if-clause] new-line}
\end{ompcPragma}

\begin{samepage}
where \plc{construct-type-clause} is one of the following:

\begin{indentedcodelist}
parallel
sections
for
taskgroup
\end{indentedcodelist}
\end{samepage}

and \plc{if-clause} is
\begin{indentedcodelist}
if (\plc{[} cancel :\plc{] scalar-expression})
\end{indentedcodelist}
\end{ccppspecific}

\begin{fortranspecific}
The syntax of the \code{cancel} construct is as follows:

\begin{ompfPragma}
!$omp cancel \plc{construct-type-clause [ [},\plc{] if-clause]}
\end{ompfPragma}


\begin{samepage}
where \plc{construct-type-clause} is one of the following:
\begin{indentedcodelist}
parallel
sections
do
taskgroup
\end{indentedcodelist}
\end{samepage}

and \plc{if-clause} is
\begin{indentedcodelist}
if (\plc{[} cancel :\plc{] scalar-logical-expression})
\end{indentedcodelist}
\end{fortranspecific}


\binding
The binding thread set of the \code{cancel} region is the current team.
The binding region of the \code{cancel} region
is the innermost enclosing region of the type corresponding to the \plc{construct-type-clause}
specified in the directive (that is, the innermost \code{parallel}, \code{sections}, loop, or
\code{taskgroup} region).

\descr
The \code{cancel} construct activates cancellation of the binding region only if the \plc{cancel-var} ICV
is \plc{true}, in which case the \code{cancel} construct causes the encountering task to continue execution
at the end of the binding region if \plc{construct-type-clause} is \code{parallel}, \code{for}, \code{do}, or \code{sections}.
If the \plc{cancel-var} ICV is \plc{true} and \plc{construct-type-clause} is \code{taskgroup}, the encountering task continues execution at the end of the current \code{task} region.
If the \plc{cancel-var} ICV is \plc{false}, the \code{cancel} construct is
ignored.

Threads check for active cancellation only at cancellation points that are
implied at the following locations:

\begin{itemize}
\item \code{cancel} regions;
\item \code{cancellation}~\code{point} regions;
\item \code{barrier} regions;
\item implicit barriers regions.
\end{itemize}

When a thread reaches one of the above cancellation points and if the \plc{cancel-var} ICV is \plc{true},  then:
\begin{itemize}
\item If the thread is at a \code{cancel} or \code{cancellation point} region
and \plc{construct-type-clause} is \code{parallel}, \code{for}, \code{do}, or
\code{sections}, the thread continues execution at the end of the canceled
region if cancellation has been activated for the innermost enclosing region
of the type specified.

\item If the thread is at a \code{cancel} or \code{cancellation point} region
and \plc{construct-type-clause} is \code{taskgroup}, the encountering task
checks for active cancellation of all of the \plc{taskgroup sets} to which the
encountering task belongs, and continues execution at the end of the current
\code{task} region if cancellation has been activated for any of the \plc{taskgroup sets}.

\item If the encountering task is at a \code{barrier} region, the encountering task
checks for active cancellation of the innermost enclosing \code{parallel}
region. If cancellation has been activated, then the encountering task
continues execution at the end of the canceled region.
\end{itemize}


\begin{note}
If one thread activates cancellation and another thread encounters a cancellation
point, the order of execution between the two threads is non-deterministic.
Whether the thread that encounters a cancellation point detects the activated cancellation
depends on the underlying hardware and operating system.
\end{note}

When cancellation of tasks is activated through a \code{cancel} construct with
the \code{taskgroup} \plc{construct-type-clause}, the tasks that belong to the
\plc{taskgroup set} of the innermost enclosing \code{taskgroup} region
will be canceled. The task that encountered that construct continues execution
at the end of its \code{task} region, which implies completion of that
task. Any task that belongs to the innermost enclosing
\code{taskgroup} and has already begun execution must run to
completion or until a cancellation point is reached. Upon reaching a
cancellation point and if cancellation is active, the task continues
execution at the end of its \code{task} region, which implies the
task's completion. Any task that belongs to the innermost enclosing
\code{taskgroup} and that has not begun execution may be discarded,
which implies its completion.

When cancellation is active for a \code{parallel}, \code{sections}, or
worksharing-loop region, each
thread of the binding thread set resumes execution at the end of the canceled region if a
cancellation point is encountered. If the canceled region is a \code{parallel} region, any
tasks that have been created by a \code{task} construct and their descendent tasks are
canceled according to the above \code{taskgroup} cancellation semantics. If the canceled
region is a \code{sections}, or worksharing-loop region, no task cancellation occurs.

\begin{cppspecific}
The usual C++ rules for object destruction are followed when cancellation is performed.
\end{cppspecific}

\begin{fortranspecific}
All private objects or subobjects with \code{ALLOCATABLE} attribute that are allocated inside
the canceled construct are deallocated.
\end{fortranspecific}

If the canceled construct contains a \code{reduction}, \code{task_reduction} or \code{lastprivate} clause, the final
value of the \plc{list-items} that appeared in those clauses are undefined.

When an \code{if} clause is present on a \code{cancel} construct and the \code{if} expression evaluates
to \plc{false}, the \code{cancel} construct does not activate cancellation. The cancellation point
associated with the \code{cancel} construct is always encountered regardless of the value of
the \code{if} expression.

\begin{note}
The programmer is responsible for releasing locks and
other synchronization data structures that might cause a deadlock when
a \code{cancel} construct is encountered and blocked threads cannot be
canceled. The programmer is also responsible for ensuring proper
synchronizations to avoid deadlocks that might arise from cancellation
of OpenMP regions that contain OpenMP synchronization constructs.
\end{note}

\events

If a task encounters a \code{cancel} construct that will
  activate cancellation then a \plc{cancel} event occurs.

A \plc{discarded-task} event occurs for any discarded tasks.

\tools

A thread dispatches a registered \code{ompt_callback_cancel}
callback for each occurrence of a \plc{cancel} event in that thread.
The callback occurs in the context of the encountering task.  The callback has type signature
\code{ompt_callback_cancel_t}.
The callback receives \code{ompt_cancel_activated} as its \plc{flags} argument.

A thread dispatches a registered \code{ompt_callback_cancel}
callback for each occurrence of a \plc{discarded-task} event.
The callback occurs in the context of the task that discards the task. % task can be discarded lazyly, or not at all.
The callback has type signature \code{ompt_callback_cancel_t}.
The callback receives the \plc{ompt_data_t} associated with the discarded task as its \plc{task_data} argument.
The callback receives \code{ompt_cancel_discarded_task} as its \plc{flags} argument.


\restrictions
The restrictions to the \code{cancel} construct are as follows:

\begin{itemize}
\item The behavior for concurrent cancellation of a region and a region nested within it is
unspecified.

\item If \plc{construct-type-clause} is \code{taskgroup}, the \code{cancel}
construct must be closely nested inside a \code{task} construct and the
\code{cancel} region must be closely nested inside a \code{taskgroup} region. If
\plc{construct-type-clause} is \code{sections}, the \code{cancel} construct
must be closely nested inside a \code{sections} or \code{section} construct.
Otherwise, the \code{cancel} construct must be closely
nested inside an OpenMP construct that matches the type specified in
\plc{construct-type-clause} of the \code{cancel} construct.

\item A worksharing construct that is canceled must not have a \code{nowait} clause.

\item A worksharing-loop construct that is canceled must not have an \code{ordered} clause.

\item During execution of a construct that may be subject to cancellation, a
thread must not encounter an orphaned cancellation point. That is, a
cancellation point must only be encountered within that construct and must
not be encountered elsewhere in its region.
\end{itemize}

\crossreferences
\begin{itemize}
\item \plc{cancel-var} ICV, see
\specref{subsec:ICV Descriptions}.

\item \code{if} Clause, see \specref{sec:if Clause}.

\item \code{cancellation}~\code{point} construct, see
\specref{subsec:cancellation point Construct}.

\item \code{omp_get_cancellation} routine, see
\specref{subsec:omp_get_cancellation}.

\item \code{ompt_callback_cancel_t}, see \specref{sec:ompt_callback_cancel_t}.

\item \code{omp_cancel_flag_t} enumeration type, see \specref{sec:ompt_cancel_flag_t}.

\end{itemize}









\subsection{\hcode{cancellation}~\hcode{point} Construct}
\index{cancellation point@{\code{cancellation}~\code{point}}}
\index{constructs!cancellation point@{\code{cancellation}~\code{point}}}
\label{subsec:cancellation point Construct}
\index{cancellation constructs!cancellation point@{\code{cancellation}~\code{point}}}
\summary
The \code{cancellation}~\code{point} construct introduces a user-defined cancellation point at
which implicit or explicit tasks check if cancellation of the innermost enclosing region
of the type specified has been activated. The \code{cancellation}~\code{point} construct is a
stand-alone directive.

\syntax
\begin{ccppspecific}
The syntax of the \code{cancellation}~\code{point} construct is as follows:

\begin{ompcPragma}
#pragma omp cancellation point \plc{construct-type-clause new-line}
\end{ompcPragma}

where \plc{construct-type-clause} is one of the following:

\begin{indentedcodelist}
parallel
sections
for
taskgroup
\end{indentedcodelist}
\end{ccppspecific}

\begin{fortranspecific}
The syntax of the \code{cancellation}~\code{point} construct is as follows:

\begin{ompfPragma}
!$omp cancellation point \plc{construct-type-clause}
\end{ompfPragma}

where \plc{construct-type-clause} is one of the following:

\begin{indentedcodelist}
parallel
sections
do
taskgroup
\end{indentedcodelist}
\end{fortranspecific}

\binding
The binding thread set of the \code{cancellation point} construct is the current team.
The binding region of the \code{cancellation point} region is the innermost enclosing region of the type corresponding to the \plc{construct-type-clause}
specified in the directive (that is, the innermost \code{parallel}, \code{sections}, loop, or
\code{taskgroup} region).

\descr
This directive introduces a user-defined cancellation point at which an implicit or
explicit task must check if cancellation of the innermost enclosing region of the type
specified in the clause has been requested. This construct does not implement any
synchronization between threads or tasks.

When an implicit or explicit task reaches a user-defined cancellation point and if
the \plc{cancel-var} ICV is \plc{true}, then:
\begin{itemize}
\item If the \plc{construct-type-clause} of the encountered \code{cancellation point} construct is \code{parallel}, \code{for}, \code{do}, or \code{sections},
the thread continues execution at the end of the canceled region if
cancellation has been activated for the innermost enclosing region of
the type specified.

\item If the \plc{construct-type-clause} of the encountered
\code{cancellation point} construct is \code{taskgroup}, the encountering
task checks for active cancellation of all  \plc{taskgroup sets} to which the
encountering task belongs and continues execution at the end of the current
\code{task} region if cancellation has been activated for any of them.
\end{itemize}


\events

The \plc{cancellation} event occurs if a task encounters a
cancellation point and detected the activation of cancellation.

\tools

A thread dispatches a registered \code{ompt_callback_cancel}
callback for each occurrence of a \plc{cancellation} event in that thread.
The callback occurs in the context of the encountering task.  The callback has type signature
\code{ompt_callback_cancel_t}.
The callback receives \code{ompt_cancel_detected} as its \plc{flags} argument.

\restrictions
\begin{itemize}
\item A \code{cancellation}~\code{point} construct for which
\plc{construct-type-clause} is \code{taskgroup} must be closely nested
inside a \code{task} construct, and the \code{cancellation}~\code{point}
region must be closely nested inside a \code{taskgroup} region. A
\code{cancellation}~\code{point} construct for which
\plc{construct-type-clause} is \code{sections} must be closely nested
inside a \code{sections} or \code{section} construct. Otherwise, a
\code{cancellation}~\code{point} construct must be closely nested inside
an OpenMP construct that matches the type specified in
\plc{construct-type-clause}.
\end{itemize}

\begin{samepage}
\crossreferences
\begin{itemize}
\item \plc{cancel-var} ICV, see
\specref{subsec:ICV Descriptions}.

\item \code{cancel} construct, see
\specref{subsec:cancel Construct}.

\item \code{omp_get_cancellation} routine, see
\specref{subsec:omp_get_cancellation}.

\item \code{ompt_callback_cancel_t}, see \specref{sec:ompt_callback_cancel_t}.

\end{itemize}
\end{samepage}

% This is an included file. See the master file for more information.
%
% When editing this file:
%
%    1. To change formatting, appearance, or style, please edit openmp.sty.
%
%    2. Custom commands and macros are defined in openmp.sty.
%
%    3. Be kind to other editors -- keep a consistent style by copying-and-pasting to
%       create new content.
%
%    4. We use semantic markup, e.g. (see openmp.sty for a full list):
%         \code{}     % for bold monospace keywords, code, operators, etc.
%         \plc{}      % for italic placeholder names, grammar, etc.
%
%    5. There are environments that provide special formatting, e.g. language bars.
%       Please use them whereever appropriate.  Examples are:
%
%         \begin{fortranspecific}
%         This is text that appears enclosed in blue language bars for Fortran.
%         \end{fortranspecific}
%
%         \begin{note}
%         This is a note.  The "Note -- " header appears automatically.
%         \end{note}
%
%    6. Other recommendations:
%         Use the convenience macros defined in openmp.sty for the minor headers
%         such as Comments, Syntax, etc.
%
%         To keep items together on the same page, prefer the use of
%         \begin{samepage}.... Avoid \parbox for text blocks as it interrupts line numbering.
%         When possible, avoid \filbreak, \pagebreak, \newpage, \clearpage unless that's
%         what you mean. Use \needspace{} cautiously for troublesome paragraphs.
%
%         Avoid absolute lengths and measures in this file; use relative units when possible.
%         Vertical space can be relative to \baselineskip or ex units. Horizontal space
%         can be relative to \linewidth or em units.
%
%         Prefer \emph{} to italicize terminology, e.g.:
%             This is a \emph{definition}, not a placeholder.
%             This is a \plc{var-name}.
%


\section{Data Environment}
\label{sec:Data Environment}
\index{data environment}
This section presents a directive and several clauses for controlling data environments.

\subsection{Data-sharing Attribute Rules}
\label{subsec:Data-sharing Attribute Rules}
\index{data-sharing attribute rules}
\index{attributes, data-sharing}
This section describes how the data-sharing attributes of variables referenced in
data environments are determined.
The following two cases are described separately:

\begin{itemize}
\item \specref{subsubsec:Data-sharing Attribute Rules for Variables Referenced in a Construct}
describes the data-sharing attribute rules for variables
referenced in a construct.

\item \specref{subsubsec:Data-sharing Attribute Rules for Variables Referenced in a Region but not in a Construct} describes the data-sharing attribute rules for variables
referenced in a region, but outside any construct.
\end{itemize}









\subsubsection{Data-sharing Attribute Rules for Variables Referenced in a Construct}
\label{subsubsec:Data-sharing Attribute Rules for Variables Referenced in a Construct}
The data-sharing attributes of variables that are referenced in a construct can be
\emph{predetermined}, \emph{explicitly determined}, or \emph{implicitly determined}, according to the rules
outlined in this section.

Specifying a variable on a \code{firstprivate}, \code{lastprivate}, \code{linear}, \code{reduction},
or \code{copyprivate} clause of an enclosed construct causes an implicit reference to the
variable in the enclosing construct. Specifying a variable on a \code{map} clause of an enclosed
construct may cause an implicit reference to the variable in the enclosing construct.
Such implicit references are also subject to the data-sharing attribute rules outlined in
this section.

Certain variables and objects have \emph{predetermined} data-sharing attributes as follows:

\begin{ccppspecific}
\begin{itemize}
\item Variables appearing in \code{threadprivate} directives are threadprivate.

\item Variables with automatic storage duration that are declared in a scope inside the
construct are private.

\item Objects with dynamic storage duration are shared.

\item Static data members are shared.

\item The loop iteration variable(s) in the associated \plc{for-loop(s)} of a
  \code{for}, \code{parallel}~\code{for}, \code{taskloop}, or \code{distribute} construct is (are) private.

\item The loop iteration variable in the associated \plc{for-loop} of a
  \code{simd} or \code{concurrent}  construct with just
one associated \plc{for-loop} is linear with a \plc{linear-step} that is the increment of
the associated \plc{for-loop}.

\item The loop iteration variables in the associated \plc{for-loops} of a \code{simd} construct with
multiple associated \plc{for-loops} are lastprivate.

\item Variables with static storage duration that are declared in a scope inside the construct
are shared.

\item If an array section with a named pointer is a list item in a
\code{map} clause on the \code{target} construct and the named pointer is a
scalar variable that does not appear in a \code{map} clause on the construct,
the named pointer is firstprivate.

\end{itemize}
\end{ccppspecific}
%
\begin{fortranspecific}
\begin{itemize}
\item Variables and common blocks appearing in \code{threadprivate} directives are
threadprivate.

\item The loop iteration variable(s) in the associated \plc{do-loop(s)} of a
  \code{do}, \code{parallel}~\code{do},
\code{taskloop}, or \code{distribute} construct is (are) private.

\item The loop iteration variable in the associated \plc{do-loop} of a
  \code{simd} or \code{concurrent}  construct with just
one associated \plc{do-loop} is linear with a \plc{linear-step} that is the increment of
the associated \plc{do-loop}.

\item The loop iteration variables in the associated \plc{do-loops} of a \code{simd} construct with
multiple associated \plc{do-loops} are lastprivate.

\item A loop iteration variable for a sequential loop in a \code{parallel} or task generating construct is
private in the innermost such construct that encloses the loop.

\item Implied-do indices and \code{forall} indices are private.

\item Cray pointees have the same the data-sharing attribute as the storage with which their Cray
pointers are associated.

\item Assumed-size arrays are shared.
\nopagebreak
\item An associate name preserves the association with the selector established at the
\code{ASSOCIATE} statement.
\end{itemize}
\end{fortranspecific}
%
Variables with predetermined data-sharing attributes may not be listed in data-sharing
attribute clauses, except for the cases listed below. For these exceptions only, listing a
predetermined variable in a data-sharing attribute clause is allowed and overrides the
variable's predetermined data-sharing attributes.
%
\begin{ccppspecific}
\begin{itemize}
\item The loop iteration variable(s) in the associated \plc{for-loop(s)} of a \code{for},
\code{parallel}~\code{for}, \code{taskloop}, or \code{distribute} construct may be listed in a \code{private} or \code{lastprivate} clause.

\item The loop iteration variable in the associated \plc{for-loop} of a \code{simd} construct with just
one associated \plc{for-loop} may be listed in a \code{private},
\code{lastprivate}, or \code{linear} clause with a
\plc{linear-step}
that is the increment of the associated \plc{for-loop}.

\item The loop iteration variables in the associated \plc{for-loops} of a \code{simd} construct with
multiple associated \plc{for-loops} may be listed in a \code{private} or \code{lastprivate} clause.

\item Variables with \code{const}-qualified type having no mutable member may be listed in a
\code{firstprivate} clause, even if they are static data members.
\end{itemize}
\end{ccppspecific}
%
\begin{fortranspecific}
\begin{itemize}
\item The loop iteration variable(s) in the associated \plc{do-loop(s)} of a \code{do},
\code{parallel}~\code{do}, \code{taskloop}, or \code{distribute}
construct may be listed in a \code{private} or \code{lastprivate} clause.

\item The loop iteration variable in the associated \plc{d}o-loop of a \code{simd} construct with just
one associated \plc{do-loop} may be listed in a \code{private},
\code{lastprivate}, or \code{linear} clause with a \plc{linear-step}
that is the increment of the associated loop.

\item The loop iteration variables in the associated \plc{do-loops} of a \code{simd} construct with
multiple associated \plc{do-loops} may be listed in a \code{private} or \code{lastprivate} clause.

\item Variables used as loop iteration variables in sequential loops in a \code{parallel}
or task generating construct may be listed in data-sharing attribute clauses on the construct itself, and on
enclosed constructs, subject to other restrictions.

\item Assumed-size arrays may be listed in a \code{shared} clause.
\end{itemize}
\end{fortranspecific}

Additional restrictions on the variables that may appear in individual clauses are
described with each clause in \specref{subsec:Data-Sharing Attribute Clauses}.

Variables with \emph{explicitly determined} data-sharing attributes are those that are referenced
in a given construct and are listed in a data-sharing attribute clause on the construct.

Variables with \emph{implicitly determined} data-sharing attributes are those that are referenced
in a given construct, do not have predetermined data-sharing attributes, and are not
listed in a data-sharing attribute clause on the construct.

Rules for variables with \emph{implicitly determined} data-sharing attributes are as follows:

\begin{itemize}
\item In a \code{parallel}, \code{teams}, or task generating construct, the data-sharing attributes of these variables are
determined by the \code{default} clause, if present (see
\specref{subsubsec:default clause}).

\item In a \code{parallel} construct, if no \code{default} clause is present, these variables are
shared.

\item For constructs other than task generating constructs, if no \code{default} clause is present, these variables reference the variables with the same names that exist in the enclosing context.

\item In a \code{target} construct, variables that are not mapped after applying data-mapping attribute rules (see \specref{subsec:Data-mapping Attribute Rules and Clauses}) are firstprivate.
\end{itemize}

\begin{cppspecific}
\begin{itemize}
\item In an orphaned task generating
construct, if no \code{default} clause is present, formal arguments passed by reference are firstprivate.
\end{itemize}
\end{cppspecific}
%
\begin{fortranspecific}
\begin{itemize}
\item In an orphaned task generating
construct, if no \code{default} clause is present, dummy arguments
are firstprivate.
\end{itemize}
\end{fortranspecific}
%
\begin{itemize}
\item In a task generating construct, if no \code{default} clause is present, a variable
for which the data-sharing attribute is not determined by the rules above
and that in the enclosing context is determined to be shared by all implicit tasks bound
to the current team is shared.

\item In a task generating construct, if no
\code{default} clause is present, a variable for which the data-sharing
attribute is not determined by the rules above is firstprivate.
\end{itemize}

Additional restrictions on the variables for which data-sharing attributes cannot be
implicitly determined in a task generating construct are described in
\specref{subsubsec:firstprivate clause}.









\subsubsection{Data-sharing Attribute Rules for Variables Referenced in a Region but not in a Construct}
\label{subsubsec:Data-sharing Attribute Rules for Variables Referenced in a Region but not in a Construct}
The data-sharing attributes of variables that are referenced in a region, but not in a
construct, are determined as follows:

\begin{ccppspecific}
\begin{itemize}
\item Variables with static storage duration that are declared in called routines in the region
are shared.

\item File-scope or namespace-scope variables referenced in called routines in the region
are shared unless they appear in a \code{threadprivate} directive.

\item Objects with dynamic storage duration are shared.

\item Static data members are shared unless they appear in a \code{threadprivate} directive.

\item In C++, formal arguments of called routines in the region that are passed by reference have the same data-sharing attributes as the associated actual arguments.

\item Other variables declared in called routines in the region are private.
\end{itemize}
\end{ccppspecific}
%
\begin{fortranspecific}
\begin{itemize}
\item Local variables declared in called routines in the region and that have the \code{save}
attribute, or that are data initialized, are shared unless they appear in a
\code{threadprivate} directive.

\item Variables belonging to common blocks, or accessed by host or use association, and referenced in called routines in the region are shared unless they appear in a \code{threadprivate} directive.

\item Dummy arguments of called routines in the region that have the
    \code{VALUE} attribute are private.

\item Dummy arguments of called routines in the region that do not have the
    \code{VALUE} attribute are private if the associated actual argument is not
    shared.

\item Dummy arguments of called routines in the region that do not have the
\code{VALUE} attribute are shared if the actual argument is shared and it
is a scalar variable, structure, an array that is not a pointer or
assumed-shape array, or a simply contiguous array section.  Otherwise, the
data-sharing attribute of the dummy argument is implementation-defined if
the associated actual argument is shared.

\item Cray pointees have the same data-sharing attribute as the storage with which their Cray pointers are associated.

\item Implied-do indices, \code{forall} indices, and other local variables declared in called
routines in the region are private.
\end{itemize}
\end{fortranspecific}
%
%
%
%
%
%
%
%
%\vspace{-12 pt} %% UGLY HACK
\subsection{\hcode{threadprivate} Directive}
\index{threadprivate@{\code{threadprivate}}}
\index{directives!threadprivate@{\code{threadprivate}}}
\label{subsec:threadprivate Directive}
\summary
The \code{threadprivate} directive specifies that variables are replicated, with each thread
having its own copy. The \code{threadprivate} directive is a declarative directive.
\syntax
\begin{ccppspecific}
The syntax of the \code{threadprivate} directive is as follows:

\begin{ompcPragma}
#pragma omp threadprivate(\plc{list}) \plc{new-line}
\end{ompcPragma}

where \plc{list} is a comma-separated list of file-scope, namespace-scope, or static
block-scope variables that do not have incomplete types.
\end{ccppspecific}
%
\begin{fortranspecific}
The syntax of the \code{threadprivate} directive is as follows:

\begin{ompfPragma}
!$omp threadprivate(\plc{list})
\end{ompfPragma}

where \plc{list} is a comma-separated list of named variables and named common blocks.
Common block names must appear between slashes.
\end{fortranspecific}

\descr
Each copy of a threadprivate variable is initialized once, in the manner specified by the
program, but at an unspecified point in the program prior to the first reference to that
copy. The storage of all copies of a threadprivate variable is freed according to how
static variables are handled in the base language, but at an unspecified point in the
program.

A program in which a thread references another thread's copy of a threadprivate variable
is non-conforming.

The content of a threadprivate variable can change across a task scheduling point if the
executing thread switches to another task that modifies the variable. For more details on
task scheduling, see
\specref{sec:Execution Model} and
\specref{sec:Tasking Constructs}.

In \code{parallel} regions, references by the master thread will be to the copy of the
variable in the thread that encountered the \code{parallel} region.

During a sequential part references will be to the initial thread's copy of the variable.
The values of data in the initial thread's copy of a threadprivate variable are guaranteed
to persist between any two consecutive references to the variable in the program.

The values of data in the threadprivate variables of non-initial threads
are guaranteed to persist between two consecutive active \code{parallel}
regions only if all of the following conditions hold:

\begin{itemize}  % L0 vvvvvvvvvvvvvvvvvvvvvvvvv
\item Neither \code{parallel} region is nested inside another explicit \code{parallel} region.

\item The number of threads used to execute both \code{parallel} regions is the same.

\item The thread affinity policies used to execute both \code{parallel} regions are the same.

\item The value of the \plc{dyn-var} internal control variable in the enclosing task region is \plc{false}
at entry to both \code{parallel} regions.

\item Neither the \code{omp_pause_resource} nor \code{omp_pause_resource_all} routine is called.

\end{itemize} % L0 ^^^^^^^^^^^^^^^^^^^^^^

If these conditions all hold, and if a threadprivate variable is referenced in both regions,
then threads with the same thread number in their respective regions will reference the
same copy of that variable.

\begin{ccppspecific}
If the above conditions hold, the storage duration, lifetime, and value of a thread's copy
of a threadprivate variable that does not appear in any \code{copyin} clause on the second
region will be retained. Otherwise, the storage duration, lifetime, and value of a thread's
copy of the variable in the second region is unspecified.

If the value of a variable referenced in an explicit initializer of a threadprivate variable
is modified prior to the first reference to any instance of the threadprivate variable, then
the behavior is unspecified.
\end{ccppspecific}
%
\begin{cppspecific}
The order in which any constructors for different threadprivate variables of class type
are called is unspecified. The order in which any destructors for different threadprivate
variables of class type are called is unspecified.
\end{cppspecific}
%
\begin{fortranspecific}
A variable is affected by a \code{copyin} clause if the variable appears in the \code{copyin} clause
or it is in a common block that appears in the \code{copyin} clause.

If the above conditions hold, the definition, association, or allocation status of a thread's
copy of a threadprivate variable or a variable in a threadprivate common
block, that is not affected by any \code{copyin} clause that appears on the second region, will
be retained. Otherwise, the definition and association status of a thread's copy of the
variable in the second region are undefined, and the allocation status of an allocatable
variable will be implementation defined.

If a threadprivate variable or a variable in a threadprivate common block is
not affected by any \code{copyin} clause that appears on the first \code{parallel} region in which
it is referenced, the variable or any subobject of the variable is initially defined or
undefined according to the following rules:

\begin{itemize} % L0 vvvvvvvvvvvvvvvvvvvvvv
\item If it has the \code{ALLOCATABLE} attribute, each copy created will have an initial
allocation status of unallocated.

\item If it has the \code{POINTER} attribute:
\begin{itemize} % L1 vvvvvvvvvvvvvvvvvvvvvv
\item if it has an initial association status of disassociated, either through explicit
initialization or default initialization, each copy created will have an association
status of disassociated;
\item otherwise, each copy created will have an association status of undefined.
\end{itemize} % l1 ^^^^^^^^^^^^^^^^^^^^

%\newpage %% HACK
\item If it does not have either the \code{POINTER} or the \code{ALLOCATABLE} attribute:

\begin{samepage}\begin{itemize} %L1 vvvvvvvvvvvvvvv
\item if it is initially defined, either through explicit initialization or default
initialization, each copy created is so defined;

\item otherwise, each copy created is undefined.
\end{itemize} % L1 ^^^^^^^^^^^^^^^^^
\end{samepage}

\end{itemize} % L0 ^^^^^^^^^^^^^^^^^^^^
\end{fortranspecific}

\restrictions
The restrictions to the \code{threadprivate} directive are as follows:

\begin{itemize} % L0 vvvvvvvvvvvvvvv
\item A threadprivate variable must not appear in any clause except the \code{copyin},
\code{copyprivate}, \code{schedule}, \code{num_threads}, \code{thread_limit}, and \code{if} clauses.

\item A program in which an untied task accesses threadprivate storage is non-conforming.

\begin{ccppspecific}
\item A variable that is part of another variable (as an array or structure element) cannot
appear in a \code{threadprivate} clause unless it is a static data member of a C++
class.

\item A \code{threadprivate} directive for file-scope variables must appear outside any
definition or declaration, and must lexically precede all references to any of the
variables in its list.

\item A \code{threadprivate} directive for namespace-scope variables must appear outside
any definition or declaration other than the namespace definition itself, and must
lexically precede all references to any of the variables in its list.

\item Each variable in the list of a \code{threadprivate} directive at file, namespace, or class
scope must refer to a variable declaration at file, namespace, or class scope that
lexically precedes the directive.

\item A \code{threadprivate} directive for static block-scope variables must appear in the
scope of the variable and not in a nested scope. The directive must lexically precede
all references to any of the variables in its list.

\item Each variable in the list of a \code{threadprivate} directive in block scope must refer to
a variable declaration in the same scope that lexically precedes the directive. The
variable declaration must use the static storage-class specifier.

\item If a variable is specified in a \code{threadprivate} directive in one translation unit, it
must be specified in a \code{threadprivate} directive in every translation unit in which
it is declared.

\item The address of a threadprivate variable is not an address constant.
\end{ccppspecific}
%
\begin{cppspecific}
\item A \code{threadprivate} directive for static class member variables must appear in the
class definition, in the same scope in which the member variables are declared, and
must lexically precede all references to any of the variables in its list.

\item A threadprivate variable must not have an incomplete type or a reference type.

\item A threadprivate variable with class type must have:

\begin{itemize} % L1 vvvvvvvvvvvvvv
\item an accessible, unambiguous default constructor in case of default initialization
without a given initializer;

\item an accessible, unambiguous constructor accepting the given argument in case of
direct initialization;

\item an accessible, unambiguous copy constructor in case of copy initialization with an
explicit initializer
\end{itemize} % L1 ^^^^^^^^^^^^^
\end{cppspecific}
%
\end{itemize} % L0 ^^^^^^^^^^^^^^
%
\begin{fortranspecific}
\begin{itemize} % L0 vvvvvvvvvvvvvvv
\item A variable that is part of another variable (as an array or structure element) cannot
appear in a \code{threadprivate} clause.

\item The \code{threadprivate} directive must appear in the declaration section of a scoping
unit in which the common block or variable is declared. Although variables in
common blocks can be accessed by use association or host association, common
block names cannot. This means that a common block name specified in a
\code{threadprivate} directive must be declared to be a common block in the same
scoping unit in which the \code{threadprivate} directive appears.

\item If a \code{threadprivate} directive specifying a common block name appears in one
program unit, then such a directive must also appear in every other program unit that
contains a \code{COMMON} statement specifying the same name. It must appear after the last
such \code{COMMON} statement in the program unit.

\item If a threadprivate variable or a threadprivate common block is declared
with the \code{BIND} attribute, the corresponding C entities must also be specified in a
\code{threadprivate} directive in the C program.

\item A blank common block cannot appear in a \code{threadprivate} directive.

\item A variable can only appear in a \code{threadprivate} directive in the scope in which it
is declared. It must not be an element of a common block or appear in an
\code{EQUIVALENCE} statement.

\item A variable that appears in a \code{threadprivate} directive must be declared in the
scope of a module or have the \code{SAVE} attribute, either explicitly or implicitly.
\end{itemize} % L0 ^^^^^^^^^^^^^^^^^^^
\end{fortranspecific}
%
\crossreferences
\begin{itemize}
\item \plc{dyn-var} ICV, see
\specref{sec:Internal Control Variables}.

\item Number of threads used to execute a \code{parallel} region, see
\specref{subsec:Determining the Number of Threads for a parallel Region}.

\item \code{copyin} clause, see
\specref{subsubsec:copyin clause}.
\end{itemize}








\subsection{Data-Sharing Attribute Clauses}
\label{subsec:Data-Sharing Attribute Clauses}
\index{data-sharing attribute clauses}
\index{attribute clauses}
\index{clauses!data-sharing}
\index{clauses!attribute data-sharing}
Several constructs accept clauses that allow a user to control the data-sharing attributes
of variables referenced in the construct. Data-sharing attribute clauses apply only to
variables for which the names are visible in the construct on which the clause appears.

Not all of the clauses listed in this section are valid on all directives. The set of clauses
that is valid on a particular directive is described with the directive.

Most of the clauses accept a comma-separated list of list items (see
\specref{sec:Directive Format}).
All list items appearing in a clause must be visible, according to the scoping rules
of the base language. With the exception of the \code{default} clause, clauses may be
repeated as needed. A list item that specifies a given variable may not appear in more
than one clause on the same directive, except that a variable may be specified in both
\code{firstprivate} and \code{lastprivate} clauses.

The reduction data-sharing attribute clauses are explained in Section \ref{subsec:Reduction Clauses}.

\begin{cppspecific}
If a variable referenced in a data-sharing attribute clause has a type derived from a
template, and there are no other references to that variable in the program, then any
behavior related to that variable is unspecified.
\end{cppspecific}
%
\begin{fortranspecific}
When a named common block appears in a \code{private}, \code{firstprivate},
\code{lastprivate}, or \code{shared} clause of a directive, none of its members may be declared
in another data-sharing attribute clause in that directive. When individual members of a common block appear in a \code{private}, \code{firstprivate},
\code{lastprivate}, \code{reduction}, or \code{linear} clause of a directive, the storage of the specified variables is no longer Fortran associated with the storage of the common block itself.
\end{fortranspecific}










\subsubsection{\hcode{default} Clause}
\label{subsubsec:default clause}
\index{default@{\code{default}}}
\index{clauses!default@{\code{default}}}
\summary
The \code{default} clause explicitly determines the data-sharing attributes of variables that
are referenced in a \code{parallel}, \code{teams}, or task generating construct
and would otherwise be implicitly determined (see
\specref{subsubsec:Data-sharing Attribute Rules for Variables Referenced in a Construct}).

\syntax
\begin{ccppspecific}
The syntax of the \code{default} clause is as follows:

\begin{ompSyntax}
default(shared \textnormal{|} none)
\end{ompSyntax}
\end{ccppspecific}
%
\begin{fortranspecific}
The syntax of the \code{default} clause is as follows:

\begin{ompSyntax}
default(private \textnormal{|} firstprivate \textnormal{|} shared \textnormal{|} none)
\end{ompSyntax}
\end{fortranspecific}
%
\descr
The \code{default(shared)} clause causes all variables referenced in the construct that
have implicitly determined data-sharing attributes to be shared.

\begin{fortranspecific}
The \code{default(firstprivate)} clause causes all variables in the construct that have
implicitly determined data-sharing attributes to be firstprivate.

The \code{default(private)} clause causes all variables referenced in the construct that
have implicitly determined data-sharing attributes to be private.
\end{fortranspecific}

The \code{default(none)} clause requires that each variable that is referenced in the
construct, and that does not have a predetermined data-sharing attribute, must have its
data-sharing attribute explicitly determined by being listed in a data-sharing attribute
clause.

\restrictions
The restrictions to the \code{default} clause are as follows:

\begin{itemize}
\item Only a single \code{default} clause may be specified on a
\code{parallel}, \code{task}, \code{taskloop} or \code{teams} directive.
\end{itemize}









\subsubsection{\hcode{shared} Clause}
\label{subsubsec:shared clause}
\index{shared@{\code{shared}}}
\index{clauses!shared@{\code{shared}}}
\summary
The \code{shared} clause declares one or more list items to be shared by tasks generated by
a \code{parallel}, \code{teams}, or task generating construct.

\syntax
The syntax of the \code{shared} clause is as follows:

\begin{ompSyntax}
shared(\plc{list})
\end{ompSyntax}

\descr
All references to a list item within a task refer to the storage area of the original variable
at the point the directive was encountered.

The programmer must ensure, by adding proper synchronization, that
storage shared by an explicit task region does not reach the end of its lifetime before
the explicit task region completes its execution.


\begin{fortranspecific}
The association status of a shared pointer becomes undefined upon entry to and on exit
from the \code{parallel}, \code{teams}, or task generating construct if it
is associated with a target or a  subobject of a target that is in a \code{private},
\code{firstprivate}, \code{lastprivate}, or \code{reduction} clause in the construct.


\begin{note}
Passing a shared variable to a procedure may result in the use of
temporary storage in place of the actual argument when the corresponding dummy
argument does not have the \code{VALUE} or \code{CONTIGUOUS} attribute and its data-sharing attribute
is implementation-defined as per the rules in
\specref{subsubsec:Data-sharing Attribute Rules for Variables Referenced in a Region but not in a Construct}.
These conditions effectively result in references to, and definitions of, the
temporary storage during the procedure reference.  Furthermore, the value of
the shared variable is copied into the intervening temporary storage before the procedure
reference when the dummy argument does not have the \code{INTENT(OUT)}
attribute, and back out of the temporary storage into the shared variable when
the dummy argument does not have the \code{INTENT(IN)} attribute.  Any
references to (or definitions of) the shared storage that is associated with
the dummy argument by any other task must be synchronized with
the procedure reference to avoid possible race conditions.

\end{note}
\medskip
\end{fortranspecific}


\restrictions
The restrictions for the \code{shared} clause are as follows:
\begin{itemize}
%
\begin{cspecific}
\item A variable that is part of another variable (as an array or structure element) cannot appear in a shared clause.
\end{cspecific}
%
\begin{cppspecific}
\item A variable that is part of another variable (as an array or structure
  element) cannot appear in a \code{shared} clause except if the \code{shared}
  clause is associated with a construct within a class non-static member
  function and the variable is an accessible data member of the object for
  which the non-static member function is invoked.
\end{cppspecific}
%
\begin{fortranspecific}
\item A variable that is part of another variable (as an array or structure element) cannot appear in a shared clause.
\end{fortranspecific}
%
\end{itemize}







\subsubsection{\hcode{private} Clause}
\index{private@{\code{private}}}
\index{clauses!private@{\code{private}}}
\label{subsubsec:private clause}
\summary
The \code{private} clause declares one or more list items to be private to a task or to a
SIMD lane.

\syntax
The syntax of the private clause is as follows:

\begin{ompSyntax}
private(\plc{list})
\end{ompSyntax}

\descr
Each task that references a list item that appears in a \code{private} clause in any statement
in the construct receives a new list item. Each SIMD lane used in a \code{simd} construct that
references a list item that appears in a private clause in any statement in the construct
receives a new list item. For each reference to a list item that appears in a
\code{private} clause on a \code{concurrent} construct, the behavior will be
as if a private copy of the list item is created for each logical loop
iteration. Language-specific attributes for new list items are derived from
the corresponding original list item. Inside the construct, all references to the original
list item are replaced by references to the new list item. In the rest of the region, it is
unspecified whether references are to the new list item or the original list item.

Therefore, if an attempt is made to reference the original item, its value after the region
is also unspecified. If a SIMD construct or a task does not reference a list item that
appears in a \code{private} clause, it is unspecified whether SIMD lanes or the task receive
a new list item.

The value and/or allocation status of the original list item will change only:

\begin{itemize}
\item if accessed and modified via pointer,

\item if possibly accessed in the region but outside of the construct,

\item as a side effect of directives or clauses, or

\begin{fortranspecific}
\item if accessed and modified via construct association.
\end{fortranspecific}
\end{itemize}
%
\begin{cppspecific}
If the construct is contained in a member function, it is unspecified
anywhere in the region if accesses through the implicit \code{this}
pointer refer to the new list item or the original list item.
\end{cppspecific}
%
%\pagebreak %% HACK
List items that appear in a \code{private}, \code{firstprivate}, or
\code{reduction} clause in a \code{parallel} construct may also appear
in a \code{private} clause in an enclosed \code{parallel},
worksharing, \code{task}, \code{taskloop}, \code{simd}, or
\code{target} construct.

List items that appear in a \code{private} or \code{firstprivate}
clause in a \code{task} or \code{taskloop} construct may also appear in a \code{private}
clause in an enclosed \code{parallel}, \code{task}, \code{taskloop}, \code{simd}, or
\code{target} construct.

List items that appear in a \code{private}, \code{firstprivate},
\code{lastprivate}, or \code{reduction} clause in a worksharing
construct may also appear in a \code{private} clause in an enclosed
\code{parallel}, \code{task}, \code{simd}, or \code{target} construct.

List items that appear in a \code{private} or \code{firstprivate} clause on a
\code{concurrent} construct may also appear in a \code{private} or
\code{firstprivate} clause in an enclosed \code{parallel} construct.

\begin{ccppspecific}
A new list item of the same type, with automatic storage duration, is allocated for the
construct. The storage and thus lifetime of these list items lasts until the block in which
they are created exits. The size and alignment of the new list item are determined by the
type of the variable. This allocation occurs once for each task generated by the construct
and once for each SIMD lane used by the construct.

The new list item is initialized, or has an undefined initial value, as if it had been locally
declared without an initializer.
\end{ccppspecific}
%
\begin{cppspecific}
If the type of a list item is a reference to a type \plc{T} then the type will be considered to be
\plc{T} for all purposes of this clause.

The order in which any default constructors for different private variables of class type
are called is unspecified. The order in which any destructors for different private
variables of class type are called is unspecified.
\end{cppspecific}
%
\begin{fortranspecific}
If any statement of the construct references a list item, a new list
item of the same type and type parameters is allocated. This
allocation occurs once for each task generated by the construct and
once for each SIMD lane used by the construct. The initial value of
the new list item is undefined. The initial status of a private
pointer is undefined.

For a list item or the subobject of a list item with the \code{ALLOCATABLE} attribute:

\begin{itemize}
\item if the allocation status is unallocated, the new list item or the subobject
of the new list item will have an initial allocation status of unallocated.

\item if the allocation status is allocated, the new list item or the subobject of
the new list item will have an initial allocation status of allocated.

\item If the new list item or the subobject of the new list item is an array, its bounds will be
the same as those of the original list item or the subobject of the original list item.
\end{itemize}

A list item that appears in a \code{private} clause may be storage-associated with other
variables when the \code{private} clause is encountered. Storage association may exist
because of constructs such as \code{EQUIVALENCE} or \code{COMMON}. If \plc{A} is a variable appearing
in a \code{private} clause on a construct and \plc{B} is a variable that is storage-associated with \plc{A}, then:

\begin{itemize}
\item The contents, allocation, and association status of \plc{B} are undefined on entry to the region.

\item Any definition of \plc{A}, or of its allocation or association status, causes the contents,
allocation, and association status of \plc{B} to become undefined.

\item Any definition of \plc{B}, or of its allocation or association status, causes the contents,
allocation, and association status of \plc{A} to become undefined.
\end{itemize}

A list item that appears in a \code{private} clause may be a selector of an \code{ASSOCIATE}
construct. If the construct association is established prior to a \code{parallel} region, the
association between the associate name and the original list item will be retained in the
region.

Finalization of a list item of a finalizable type or subojects of a
list item of a finalizable type occurs at the end of the region. The
order in which any final subroutines for different variables of a
finalizable type are called is unspecified.
\end{fortranspecific}

\restrictions
The restrictions to the \code{private} clause are as follows:

\begin{itemize}
\begin{cspecific}
\item A variable that is part of another variable (as an array or structure element) cannot
appear in a \code{private} clause.
\end{cspecific}
%
\begin{cppspecific}
\item A variable that is part of another variable (as an array or structure element) cannot
appear in a \code{private} clause except if the \code{private} clause is associated with a construct within a class non-static member function and the variable is an accessible data member of the object for which the non-static member function is invoked.

%\newpage %% HACK
\item A variable of class type (or array thereof) that appears in a \code{private} clause requires
an accessible, unambiguous default constructor for the class type.
\end{cppspecific}
%
\begin{ccppspecific}
\item A variable that appears in a \code{private} clause must not have a \code{const}-qualified type
unless it is of class type with a \code{mutable} member. This restriction does not apply to
the \code{firstprivate} clause.

\item A variable that appears in a \code{private} clause must not have an incomplete type or be a reference to an incomplete type.
\end{ccppspecific}
%
%
\begin{fortranspecific}
\item A variable that is part of another variable (as an array or structure element) cannot
appear in a \code{private} clause.

\item A variable that appears in a \code{private} clause must either be definable, or an
allocatable variable. This restriction does not apply to the \code{firstprivate} clause.

\item Variables that appear in namelist statements, in variable format expressions, and in
expressions for statement function definitions, may not appear in a \code{private} clause.

\item Pointers with the \code{INTENT(IN)} attribute may not appear in a \code{private} clause. This
restriction does not apply to the \code{firstprivate} clause.

\item Assumed-size arrays may not appear in the \code{private}
clause in a \code{target}, \code{teams}, or \code{distribute} construct.
\end{fortranspecific}
\end{itemize}









\vspace{-12 pt} %% UGLY HACK
\subsubsection{\hcode{firstprivate} Clause}
\label{subsubsec:firstprivate clause}
\index{firstprivate@{\code{firstprivate}}}
\index{clauses!firstprivate@{\code{firstprivate}}}
\summary
The \code{firstprivate} clause declares one or more list items to be private to a task, and
initializes each of them with the value that the corresponding original item has when the
construct is encountered.

\syntax
The syntax of the \code{firstprivate} clause is as follows:

\begin{ompSyntax}
firstprivate(\plc{list})
\end{ompSyntax}

\descr
The \code{firstprivate} clause provides a superset of the functionality provided by the
\code{private} clause.

A list item that appears in a \code{firstprivate} clause is subject to the \code{private} clause
semantics described in
\specref{subsubsec:private clause},
except as noted. In addition, the
new list item is initialized from the original list item existing before the construct. The
initialization of the new list item is done once for each task that references the list item
in any statement in the construct. The initialization is done prior to the execution of the
construct.

For a \code{firstprivate} clause on a \code{parallel}, \code{task},
\code{taskloop}, \code{target}, or \code{teams} construct, the initial
value of the new list item is the value of the original list item that
exists immediately prior to the construct in the task region where the
construct is encountered unless otherwise specified. For a
\code{firstprivate} clause on a worksharing construct, the initial
value of the new list item for each implicit task of the threads that
execute the worksharing construct is the value of the original list
item that exists in the implicit task immediately prior to the point
in time that the worksharing construct is encountered unless otherwise
specified.

To avoid race conditions, concurrent updates of the original list item must be
synchronized with the read of the original list item that occurs as a result of the
\code{firstprivate} clause.

If a list item appears in both \code{firstprivate} and \code{lastprivate} clauses, the update
required for \code{lastprivate} occurs after all the initializations for \code{firstprivate}.

\begin{ccppspecific}
For variables of non-array type, the initialization occurs by copy assignment. For an
array of elements of non-array type, each element is initialized as if by assignment from
an element of the original array to the corresponding element of the new array.
\end{ccppspecific}
%
\begin{cppspecific}
For each variable of class type:
\begin{itemize}
\item If the \code{firstprivate} clause is not on a \code{target} construct then a copy constructor is invoked to perform the initialization;
\item If the \code{firstprivate} clause is on a \code{target} construct then it is unspecified how many copy constructors, if any, are invoked.
\end{itemize}
If copy constructors are called, the order in which copy constructors for different variables of class type are called is unspecified.
\end{cppspecific}
%
\begin{fortranspecific}
If the original list item does not have the \code{POINTER} attribute, initialization of the new
list items occurs as if by intrinsic assignment, unless the original list item has the
allocation status of unallocated, in which case the new list items will have the
same status.

If the original list item has the \code{POINTER} attribute, the new list items receive the same
association status of the original list item as if by pointer assignment.
\end{fortranspecific}
%
\restrictions
The restrictions to the \code{firstprivate} clause are as follows:

\begin{itemize}
\item A list item that is private within a \code{parallel} region must
not appear in a \code{firstprivate} clause on a worksharing construct
if any of the worksharing regions arising
from the worksharing construct ever bind to any of the
\code{parallel} regions arising from the \code{parallel} construct.

\item A list item that is private within a \code{teams} region must not appear in a
\code{firstprivate} clause on a \code{distribute} construct if any of the \code{distribute}
regions arising from the \code{distribute} construct ever bind to any of the \code{teams}
regions arising from the \code{teams} construct.

\item A list item that appears in a \code{reduction} clause of a \code{parallel}
construct must not appear in a \code{firstprivate} clause on a worksharing, \code{task},
or \code{taskloop} construct if any of the worksharing or task regions arising from
the worksharing, \code{task}, or \code{taskloop} construct ever bind to any of the
\code{parallel} regions arising from the \code{parallel} construct.

\item A list item that appears in a \code{reduction} clause of a \code{teams} construct must not
appear in a \code{firstprivate} clause on a \code{distribute} construct if any of the
\code{distribute} regions arising from the \code{distribute} construct ever bind to any of
the \code{teams} regions arising from the \code{teams} construct.

\item A list item that appears in a \code{reduction} clause of a worksharing construct must not
appear in a \code{firstprivate} clause in a \code{task} construct encountered during execution
of any of the worksharing regions arising from the worksharing construct.

\begin{cppspecific}
\item A variable of class type (or array thereof) that appears in a \code{firstprivate} clause
requires an accessible, unambiguous copy constructor for the class type.
\end{cppspecific}
%
\begin{ccppspecific}
\item A variable that appears in a \code{firstprivate} clause must not have an incomplete C/C++ type or be a reference to an incomplete type.

\item If a list item in a \code{firstprivate} clause on a worksharing
construct has a reference type then it must bind to the same object for all threads of the team.
\end{ccppspecific}
%
\begin{fortranspecific}
\item Variables that appear in namelist statements, in variable format expressions, or in
expressions for statement function definitions, may not appear in a \code{firstprivate}
clause.

\item Assumed-size arrays may not appear in the \code{firstprivate}
clause in a \code{target}, \code{teams}, or \code{distribute} construct.

\item If the list item is a polymorphic variable with the \code{ALLOCATABLE} attribute, the behavior is unspecified.
\end{fortranspecific}
\end{itemize}










\subsubsection{\hcode{lastprivate} Clause}
\index{lastprivate@{\code{lastprivate}}}
\index{clauses!lastprivate@{\code{lastprivate}}}
\label{subsubsec:lastprivate clause}
\summary
The \code{lastprivate} clause declares one or more list items to be private to an implicit
task or to a SIMD lane, and causes the corresponding original list item to be updated
after the end of the region.

\syntax
The syntax of the \code{lastprivate} clause is as follows:

\begin{ompSyntax}
lastprivate(\plc{[ lastprivate-modifier}:\plc{] list})
\end{ompSyntax}

where \plc{lastprivate-modifier} is:
\begin{indentedcodelist}
conditional
\end{indentedcodelist}

\descr
The \code{lastprivate} clause provides a superset of the functionality provided by the
\code{private} clause.

A list item that appears in a \code{lastprivate} clause is subject to the \code{private} clause
semantics described in
\specref{subsubsec:private clause}.
In addition, when a
\code{lastprivate} clause without the \code{conditional} modifier appears on a directive,
the value of each new list item from the sequentially last iteration
of the associated loops, or the lexically last \code{section} construct, is assigned to the
original list item. When the \code{conditional} modifier appears on the clause,
if an assignment to a list item is encountered in the construct then the
original list item is assigned the value that is assigned to the new list item
in the sequentially last iteration or lexically last section in which such an
assignment is encountered.


\begin{ccppspecific}
For an array of elements of non-array type, each element is assigned to the
corresponding element of the original array.
\end{ccppspecific}
%\bigskip
%
\begin{fortranspecific}
If the original list item does not have the \code{POINTER} attribute, its update occurs as if by
intrinsic assignment.

If the original list item has the \code{POINTER} attribute, its update occurs as if by pointer
assignment.
\end{fortranspecific}

When the \code{conditional} modifier does not appear on the \code{lastprivate} clause, list items that are not
assigned a value by the sequentially last iteration of the loops, or by the
lexically last \code{section} construct, have unspecified values after the
construct.  Unassigned subcomponents also have unspecified values after the
construct.

If the \code{lastprivate} clause is used on a construct to which neither
the \code{nowait} nor the \code{nogroup} clauses are applied, the original list item
becomes defined at the end of the construct.  To avoid race conditions,
concurrent reads or updates of the original list item must be synchronized with
the update of the original list item that occurs as a result of the
\code{lastprivate} clause.

Otherwise, If the \code{lastprivate} clause is used on a construct to which
the \code{nowait} or the \code{nogroup} clauses are applied, accesses to the original
list item may create a data race.  To avoid this, if an assignment to the
original list item occurs then synchronization must be inserted to ensure that
the assignment completes and the original list item is flushed to memory.

If a list item appears in both \code{firstprivate} and \code{lastprivate} clauses, the update
required for \code{lastprivate} occurs after all initializations for \code{firstprivate}.

\restrictions
The restrictions to the \code{lastprivate} clause are as follows:

\begin{itemize}
\item A list item that is private within a \code{parallel} region, or
that appears in the \code{reduction} clause of a \code{parallel}
construct, must not appear in a \code{lastprivate} clause on a
worksharing construct if any of the corresponding
worksharing regions ever binds to any of the corresponding
\code{parallel} regions.

\item If a list item that appears in a \code{lastprivate} clause with the
\code{conditional} modifier is modified in the region by an assignment
outside the construct or not to the list item then the value assigned to
the original list item is unspecified.

\item A list item that appears in a \code{lastprivate} clause with the
\code{conditional} modifier must be a scalar variable.

\begin{cppspecific}
\item A variable of class type (or array thereof) that appears in a \code{lastprivate} clause
requires an accessible, unambiguous default constructor for the class type, unless the
list item is also specified in a \code{firstprivate} clause.

\item A variable of class type (or array thereof) that appears in a \code{lastprivate} clause
requires an accessible, unambiguous copy assignment operator for the class type. The
order in which copy assignment operators for different variables of class type are
called is unspecified.
\end{cppspecific}
%
\begin{ccppspecific}
\item A variable that appears in a \code{lastprivate} clause must not have a \code{const}-qualified
type unless it is of class type with a \code{mutable} member.

\item A variable that appears in a \code{lastprivate} clause must not have an incomplete C/C++ type or be a reference to an incomplete type.

\item If a list item in a \code{lastprivate} clause on a worksharing
construct has a reference type then it must bind to the same object for all threads of the team.
\end{ccppspecific}
%
\begin{fortranspecific}
\item A variable that appears in a \code{lastprivate} clause must be definable.

\item If the original list item has the \code{ALLOCATABLE} attribute, the
    corresponding list item whose value is assigned to the original list item must have an allocation status of allocated upon exit from
    the sequentially last iteration or lexically last \code{section} construct.

\item Variables that appear in namelist statements, in variable format expressions, or in
expressions for statement function definitions, may not appear in a \code{lastprivate}
clause.

\item If the list item is a polymorphic variable with the \code{ALLOCATABLE} attribute, the behavior is unspecified.
\end{fortranspecific}
\end{itemize}










\subsubsection{\hcode{linear} Clause}
\index{linear@{\code{linear}}}
\index{clauses!linear@{\code{linear}}}
\label{subsubsec:linear clause}
\summary
The \code{linear} clause declares one or more list items to be private to a SIMD lane and to
have a linear relationship with respect to the iteration space of a loop.

\syntax


\begin{cspecific}
The syntax of the \code{linear} clause is as follows:
\begin{ompSyntax}
linear(\plc{linear-list[ }:\plc{ linear-step]})
\end{ompSyntax}
where \plc{linear-list} is one of the following
% \vspace{-2ex} %% HACK
\begin{indentedcodelist}
\plc{list}
\plc{modifier}(\plc{list})
\end{indentedcodelist}
where  \plc{modifier} is one of the following:
% \vspace{-2ex} %% HACK
\begin{indentedcodelist}
val
\end{indentedcodelist}
\end{cspecific}
%
\begin{cppspecific}
The syntax of the \code{linear} clause is as follows:
\begin{ompSyntax}
linear(\plc{linear-list[ }:\plc{ linear-step]})
\end{ompSyntax}
where \plc{linear-list} is one of the following
% \vspace{-2ex} %% HACK
\begin{indentedcodelist}
\plc{list}
\plc{modifier}(\plc{list})
\end{indentedcodelist}
where  \plc{modifier} is one of the following:
% \vspace{-2ex} %% HACK
\begin{indentedcodelist}
ref
val
uval
\end{indentedcodelist}
\end{cppspecific}

\begin{fortranspecific}
The syntax of the \code{linear} clause is as follows:
\begin{ompSyntax}
linear(\plc{linear-list[ }:\plc{ linear-step]})
\end{ompSyntax}
where \plc{linear-list} is one of the following
% \vspace{-2ex} %% HACK
\begin{indentedcodelist}
\plc{list}
\plc{modifier}(\plc{list})
\end{indentedcodelist}
where  \plc{modifier} is one of the following:
% \vspace{-2ex} %% HACK
\begin{indentedcodelist}
ref
val
uval
\end{indentedcodelist}
\end{fortranspecific}


\descr
The \code{linear} clause provides a superset of the functionality provided by the \code{private} clause.
A list item that appears in a \code{linear} clause is subject to the \code{private} clause semantics described
in \specref{subsubsec:private clause} except as noted.
If \plc{linear-step} is not specified, it is assumed to be 1.

When a \code{linear} clause is specified on a construct, the value of the new list item on each iteration of the associated loop(s) corresponds to the value of the original list item before entering the construct plus the logical number of the iteration times \plc{linear-step}.
The value corresponding to the sequentially last iteration of the associated loop(s) is assigned to the original list item.

When a \code{linear} clause is specified on a declarative directive, all list items must be formal parameters (or, in Fortran, dummy arguments) of a function that will be invoked concurrently on each SIMD lane.
If no \plc{modifier} is specified or the \code{val} or \code{uval} modifier is specified, the value of each list item on each lane corresponds to the value of the list item upon entry to the function plus the logical number of the lane times \plc{linear-step}.
If the \code{uval} modifier is specified, each invocation uses the same storage location for each SIMD lane; this storage location is updated with the final value of the logically last lane.
If the \code{ref} modifier is specified, the storage location of each list item on each lane corresponds to an array at the storage location upon entry to the function indexed by the logical number of the lane times \plc{linear-step}.


\restrictions
\begin{itemize}
\item The \plc{linear-step} expression must be invariant during the execution of the region
corresponding to the construct. Otherwise, the execution results in unspecified
behavior.

\item A \plc{list-item} cannot appear in more than one \code{linear} clause.

\item A \plc{list-item} that appears in a \code{linear} clause cannot appear in any other data-sharing
attribute clause.

\begin{cspecific}
\item A \plc{list-item} that appears in a \code{linear} clause must be of integral or pointer type.
\end{cspecific}

\begin{cppspecific}
\item A \plc{list-item} that appears in a \code{linear} clause without the \code{ref} modifier must be of integral or pointer type, or must be a reference to an integral or pointer type.
\item The \code{ref} or \code{uval} modifier can only be used if the \plc{list-item} is of a reference type.
\item If a list item in a \code{linear} clause on a worksharing
construct has a reference type then it must bind to the same object for all threads of the team.
%\newpage %% HACK
\item If the list item is of a reference type and the \code{ref} modifier is not specified and if any write to the list item occurs before any read of the list item then the result is unspecified.
\end{cppspecific}

\begin{fortranspecific}
\item A \plc{list-item} that appears in a \code{linear} clause without the \code{ref} modifier must be of type \code{integer}.
\item The \code{ref} or \code{uval} modifier can only be used if the \plc{list-item} is a dummy argument without the \code{VALUE} attribute.
\item Variables that have the \code{POINTER} attribute and Cray pointers may not appear in a linear clause.
\item The list item with the \code{ALLOCATABLE} attribute in the sequentially last iteration must have an allocation status of allocated upon exit from that iteration.
\item If the list item is a dummy argument without the \code{VALUE} attribute and the \code{ref} modifier is not specified and if any write to the list item occurs before any read of the list item then the result is unspecified.
\item A common block name cannot appear in a \code{linear} clause.
\end{fortranspecific}
\end{itemize}










\subsection{Reduction Clauses}
\index{reduction clauses}
\label{subsec:Reduction Clauses}
The reduction clauses are data-sharing attribute clauses that can be used to perform some forms of recurrence
calculations (involving mathematically associative and commutative operators)
in parallel.

Reduction clauses include reduction scoping clauses and reduction participating
clauses. Reduction scoping clauses define the region in which a reduction is
computed. Reduction participating clauses define the participants in the
reduction.

Reduction clauses specify a \plc{reduction-identifier} and one or more list
items.










\subsubsection{Properties Common To All Reduction Clauses}
\label{subsubsec:Properties Common To All Reduction Clauses}

\syntax
The syntax of a \plc{reduction-identifier} is defined as follows:
\begin{cspecific} % L1 vvvvv
A \plc{reduction-identifier} is either an \plc{identifier} or one of the following operators:
\code{+},
\code{-},
\code{*},
\code{&},
\code{|},
\code{^},
\code{&&} and
\code{||}
\end{cspecific} % L1 ^^^^^

\begin{cppspecific} % L1 vvvvv
A \plc{reduction-identifier} is either an \plc{id-expression} or one of the following operators:
\code{+},
\code{-},
\code{*},
\code{&},
\code{|},
\code{^},
\code{&&} and
\code{||}
\end{cppspecific} % L1 ^^^^^

\begin{fortranspecific}
A \plc{reduction-identifier} is either a base language identifier, or a user-defined operator,
or one of the following operators:
\code{+},
\code{-},
\code{*},
\code{.and.},
\code{.or.},
\code{.eqv.},
\code{.neqv.},
 or one of the following intrinsic procedure names:
\code{max},
\code{min},
\code{iand},
\code{ior},
\code{ieor}.
\end{fortranspecific}
%
%
\begin{ccppspecific} % L0 vvvvv
Table~\ref{tab:Implicitly Declared C/C++ Reduction Identifiers} lists each
\plc{reduction-identifier} that is implicitly declared at every scope for
arithmetic types and its semantic initializer value. The actual initializer
value is that value as expressed in the data type of the reduction list item.


%\newpage %% HACK
% Table
\nolinenumbers
\needspace{\sbns}
\renewcommand{\arraystretch}{1.5}
%\begin{center}
\tablecaption{Implicitly Declared C/C++ \plc{reduction-identifiers}\label{tab:Implicitly Declared C/C++ Reduction Identifiers}}
%% \tablecaption*{}
\tablefirsthead{%
\hline
\textsf{\textbf{Identifier}} & \textsf{\textbf{Initializer}} & \textsf{\textbf{Combiner}}\\
\hline \\[-3ex]
}
\tablehead{%
\multicolumn{2}{l}{\small\slshape table continued from previous page}\\
\hline
\textsf{\textbf{Identifier}} & \textsf{\textbf{Initializer}} & \textsf{\textbf{Combiner}}\\
\hline \\[-3ex]
}
\tabletail{%
\hline\\[-4ex]
\multicolumn{2}{l}{\small\slshape table continued on next page}\\
}
\tablelasttail{\hline}
\begin{supertabular}{ p{0.1\textwidth} p{0.3\textwidth} p{0.5\textwidth}}
{\scode{+}} & {\scode{omp_priv = 0}} & {\scode{omp_out += omp_in}}\\
{\scode{*}} & {\scode{omp_priv = 1}} & {\scode{omp_out *= omp_in}}\\
{\scode{-}} & {\scode{omp_priv = 0}} & {\scode{omp_out += omp_in}}\\
{\scode{&}} & {\scode{omp_priv =  ~ 0}} & {\scode{omp_out &= omp_in}}\\
{\scode{|}} & {\scode{omp_priv = 0}} & {\scode{omp_out |= omp_in}}\\
{\scode{^}} & {\scode{omp_priv = 0}} & {\scode{omp_out ^}}{\scode{= omp_in}}\\
{\scode{&&}} & {\scode{omp_priv = 1}} & {\scode{omp_out = omp_in && omp_out}}\\
{\scode{||}} & {\scode{omp_priv = 0}} & {\scode{omp_out = omp_in || omp_out}}\\
{\scode{max}} & {\scode{omp_priv = }\splc{Least representable number in the reduction list item type}} & {\scode{omp_out = omp_in > omp_out ? omp_in : omp_out}}\\
{\scode{min}} & {\scode{omp_priv = }\splc{Largest representable number in the reduction list item type}} & {\scode{omp_out = omp_in < omp_out ? omp_in : omp_out}}\\
\end{supertabular}
\bigskip
\end{ccppspecific} % L0 ^^^^^
\linenumbers
\bigskip

\begin{fortranspecific}
Table~\ref{tab:Implicitly Declared Fortran Reduction Identifiers} lists each
\plc{reduction-identifier} that is implicitly declared for numeric and logical
types and its semantic initializer value. The actual initializer value is that
value as expressed in the data type of the reduction list item.

% Table
\nolinenumbers
\renewcommand{\arraystretch}{1.5}
\tablefirsthead{%
\hline
\textsf{\textbf{Identifier}} & \textsf{\textbf{Initializer}} & \textsf{\textbf{Combiner}}\\
\hline \\[-3ex]
}
\tablehead{%
\multicolumn{2}{l}{\small\slshape table continued from previous page}\\
\hline
\textsf{\textbf{Identifier}} & \textsf{\textbf{Initializer}} & \textsf{\textbf{Combiner}}\\
\hline \\[-3ex]
}
\tabletail{%
\hline\\[-4ex]
\multicolumn{2}{l}{\small\slshape table continued on next page}\\
}
\tablelasttail{\hline}
\tablecaption{Implicitly Declared Fortran \plc{reduction-identifiers}\label{tab:Implicitly Declared Fortran Reduction Identifiers}}
\begin{supertabular}{ p{0.1\textwidth} p{0.30\textwidth} p{0.5\textwidth}}
{\scode{+}} & {\scode{omp_priv = 0}} & {\scode{omp_out = omp_in + omp_out}}\\
{\scode{*}} & {\scode{omp_priv = 1}} & {\scode{omp_out = omp_in * omp_out}}\\
{\scode{-}} & {\scode{omp_priv = 0}} & {\scode{omp_out = omp_in + omp_out}}\\
{\scode{.and.}} & {\scode{omp_priv = .true.}} & {\scode{omp_out = omp_in .and. omp_out}}\\
{\scode{.or.}} & {\scode{omp_priv = .false.}} & {\scode{omp_out = omp_in .or. omp_out}}\\
{\scode{.eqv.}} & {\scode{omp_priv = .true.}} & {\scode{omp_out = omp_in .eqv. omp_out}}\\
{\scode{.neqv.}} & {\scode{omp_priv = .false.}} & {\scode{omp_out = omp_in .neqv. omp_out}}\\
{\scode{max}} & {\scode{omp_priv = }\splc{Least representable number in the reduction list item type}} & {\scode{omp_out = max(omp_in, omp_out)}}\\
{\scode{min}} & {\scode{omp_priv = }\splc{Largest representable number in the reduction list item type}} & {\scode{omp_out = min(omp_in, omp_out)}}\\
{\scode{iand}} & {\scode{omp_priv = }\splc{All bits on}} & {\scode{omp_out = iand(omp_in, omp_out)}}\\
{\scode{ior}} & {\scode{omp_priv = 0}} & {\scode{omp_out = ior(omp_in, omp_out)}}\\
{\scode{ieor}} & {\scode{omp_priv = 0}} & {\scode{omp_out = ieor(omp_in, omp_out)}}\\
\end{supertabular}
\bigskip
\end{fortranspecific}
\linenumbers

\vspace{\baselineskip}

In the above tables, \code{omp_in} and \code{omp_out} correspond to two
identifiers that refer to storage of the type of the list item. \code{omp_out}
holds the final value of the combiner operation.

Any \plc{reduction-identifier} that is defined with the \code{declare}~\code{reduction}
directive is also valid. In that case, the initializer and combiner of the
\plc{reduction-identifier} are specified by the \plc{initializer-clause} and
the \plc{combiner} in the \code{declare}~\code{reduction} directive.




\descr
A reduction clause specifies a \plc{reduction-identifier} and one or more list
items.

The \plc{reduction-identifier} specified in a reduction clause must match a
previously declared \plc{reduction-identifier} of the same name and type for
each of the list items. This match is done by means of a name lookup in the
base language.

The list items that appear in a reduction clause may include array sections.


\begin{cppspecific}
If the type is a derived class, then any \plc{reduction-identifier} that
matches its base classes is also a match, if there is no specific match for the
type.

If the \plc{reduction-identifier} is not an \plc{id-expression}, then it is
implicitly converted to one by prepending the keyword operator (for example,
\code{+} becomes {\plc{operator}\code{+}}).

If the \plc{reduction-identifier} is qualified then a qualified name lookup is
used to find the declaration.

If the \plc{reduction-identifier} is unqualified then an \emph{argument-dependent name lookup}
must be performed using the type of each list item.
\end{cppspecific}

If the list item is an array or array section, it will be treated as
if a reduction clause would be applied to each separate element
of the array section.

Any copies associated with the reduction are initialized with the intializer
value of the \plc{reduction-identifier}.

Any copies are combined using the combiner associated with the
\plc{reduction-identifier}.


\restrictions
The restrictions common to reduction clauses are as follows:

\begin{itemize}
\item Any number of reduction clauses can be specified on the directive, but a
list item (or any array element in an array section) can appear only once in
reduction clauses for that directive.

\item For a \plc{reduction-identifier} declared with the \code{declare}~\code{reduction}
construct, the directive must appear before its use in a reduction clause.

\item If a list item is an array section, its base expression must be a base
language identifier.

\item If a list item is an array section, it must specify contiguous storage
and it cannot be a zero-length array section.

\item If a list item is an array section, accesses to the elements of the array
outside the specified array section result in unspecified behavior.

\begin{cspecific}
\item A variable that is part of another variable, with the exception of array
elements, cannot appear in a reduction clause.
\end{cspecific}

\begin{cppspecific}
\item A variable that is part of another variable, with the exception of array
elements, cannot appear in a reduction clause except if the reduction clause
is associated with a construct within a class non-static member function and
the variable is an accessible data member of the object for which the
non-static member function is invoked.
\end{cppspecific}

\begin{ccppspecific}
\item The type of a list item that appears in a reduction clause must be valid for the
\plc{reduction-identifier}. For a \code{max} or \code{min} reduction in C, the type of the list item must
be an allowed arithmetic data type: \code{char}, \code{int}, \code{float}, \code{double}, or \code{_Bool},
possibly modified with \code{long}, \code{short}, \code{signed}, or \code{unsigned}. For a \code{max} or \code{min}
reduction in C++, the type of the list item must be an allowed arithmetic data type:
\code{char}, \code{wchar_t}, \code{int}, \code{float}, \code{double}, or \code{bool}, possibly modified with \code{long},
\code{short}, \code{signed}, or \code{unsigned}.

\item A list item that appears in a reduction clause must not be \code{const}-qualified.

\item The \plc{reduction-identifier} for any list item must be unambiguous and accessible.
\end{ccppspecific}
\bigskip

\begin{fortranspecific}
\item A variable that is part of another variable, with the exception of array
elements, cannot appear in a reduction clause.

\item The type and the rank of a list item that appears in a reduction clause
must be valid for the \plc{combiner} and \plc{initializer}.

\item A list item that appears in a reduction clause must be definable.

\item A procedure pointer may not appear in a reduction clause.

\item A pointer with the \code{INTENT(IN)} attribute may not appear in the
reduction clause.

\item An original list item with the \code{POINTER} attribute or any pointer
component of an original list item that is referenced in the \plc{combiner}
must be associated at entry to the construct that contains the reduction
clause. Additionally, the list item or the pointer component of the list item
must not be deallocated, allocated, or pointer assigned within the region.

\item An original list item with the \code{ALLOCATABLE} attribute or any
allocatable component of an original list item that corresponds to the special
variable identifier in the \plc{combiner} or the \plc{initializer} must be
in the allocated state at entry to the construct that contains the reduction
clause. Additionally, the list item or the allocatable component of the list
item must be neither deallocated nor allocated, explicitly or implicitly,
within the region.

\item If the \plc{reduction-identifier} is defined in a \code{declare}~\code{reduction}
directive, the \code{declare}~\code{reduction} directive must be in the same
subprogram, or accessible by host or use association.

\item If the \plc{reduction-identifier} is a user-defined operator, the same
explicit interface for that operator must be accessible as at the
\code{declare}~\code{reduction} directive.

\item If the \plc{reduction-identifier} is defined in a \code{declare}~\code{reduction}
directive, any subroutine or function referenced in the initializer clause or
combiner expression must be an intrinsic function, or must have an explicit
interface where the same explicit interface is accessible as at the
\code{declare}~\code{reduction} directive.
\end{fortranspecific}
\end{itemize}










\subsubsection{Reduction Scoping Clauses}
\label{subsubsec:Reduction Scoping Clauses}
Reduction scoping clauses define the region in which a reduction is computed by
tasks or SIMD lanes. All properties common to all reduction clauses,
which are defined in Section~\ref{subsubsec:Properties Common To All Reduction
Clauses}, apply to reduction scoping clauses.

The number of copies created for each list item and the time at which those
copies are initialized are determined by the particular reduction scoping clause
that appears on the construct.

The time at which the original list item contains the result of the reduction
is determined by the particular reduction scoping clause.

\begin{samepage}
\begin{fortranspecific}
If the original list item has the \code{POINTER} attribute, copies of
the list item are associated with private targets.
\end{fortranspecific}
\end{samepage}

If the list item is an array section, the elements of any copy of the array section will
be allocated contiguously.

The location in the OpenMP program at which values are combined and the
order in which values are combined are unspecified. Therefore, when
comparing sequential and parallel runs, or when comparing one parallel run to
another (even if the number of threads used is the same), there is no guarantee
that bit-identical results will be obtained or that side effects (such as
floating-point exceptions) will be identical or take place at the same location
in the OpenMP program.

To avoid race conditions, concurrent reads or updates of the original list item
must be synchronized with the update of the original list item that occurs as a
result of the reduction computation.










\subsubsection{Reduction Participating Clauses}
\label{subsubsec:Reduction Participating Clauses}
A reduction participating clause specifies a task or a SIMD lane as a
participant in a reduction defined by a reduction scoping clause.
All properties common to all reduction clauses, which are defined in
Section~\ref{subsubsec:Properties Common To All Reduction Clauses}, apply to
reduction participating clauses.

Accesses to the original list item may be replaced by accesses to copies of the
original list item created by a region corresponding to a construct with a
reduction scoping clause.

In any case, the final value of the reduction must be determined as if all tasks
or SIMD lanes that participate in the reduction are executed sequentially in
some arbitrary order.










\subsubsection{\hcode{reduction} Clause}
\index{reduction@{\code{reduction}}}
\index{clauses!reduction@{\code{reduction}}}
\label{subsubsec:reduction clause}
\summary
The \code{reduction} clause specifies a \plc{reduction-identifier} and one or
more list items. For each list item, a private copy is created in each implicit
task or SIMD lane and is initialized with the initializer value of the
\plc{reduction-identifier}. After the end of the region, the original list item
is updated with the values of the private copies using the combiner associated
with the \plc{reduction-identifier}.

\syntax
\begin{ompSyntax}
reduction(\plc{reduction-identifier }:\plc{ list})
\end{ompSyntax}
Where \plc{reduction-identifier} is defined in Section
\ref{subsubsec:Properties Common To All Reduction Clauses}.

\descr
The \code{reduction} clause is a reduction scoping clause and a reduction
participating clause, as described in Sections \ref{subsubsec:Reduction Scoping
Clauses} and \ref{subsubsec:Reduction Participating Clauses}.

For \code{parallel} and worksharing constructs, a private copy of each list item is created,
one for each implicit task, as if the \code{private} clause had been used. For the \code{simd}
construct, a private copy of each list item is created, one for each SIMD lane as if the
\code{private} clause had been used.  For the \code{taskloop} construct,
private copies are created according to the rules of the reduction scoping
clauses.  For the \code{teams} construct, a private copy of each list item is
created and initialized, one for each team in the league as if the
\code{private} clause had been used. For the \code{concurrent} construct, the
behavior will be as if a private copy of each list item is created for each
loop iteration. At the end of a region corresponding to an above construct for
which the \code{reduction} clause was specified, the original list item is
updated by combining its original value with the final value of each of the
private copies, using the combiner of the specified \plc{reduction-identifier}.

For the \code{target} construct, each list item is implicitly mapped into the
device data environment of the target device and a private copy of each list
item is created and initialized for the initial task. At the end of a
\code{target} region for which the \code{reduction} clause was specified, the
corresponding list item present in the device data environment is updated by
combining its original value with the final value of the created private copy,
using the combiner of the specified \plc{reduction-identifier}.


If \code{nowait} is not used, the reduction computation will be complete at the end of the
construct; however, if the reduction clause is used on a construct to which \code{nowait} is
also applied, accesses to the original list item will create a race and, thus, have
unspecified effect unless synchronization ensures that they occur after all threads have
executed all of their iterations or \code{section} constructs, and the reduction computation
has completed and stored the computed value of that list item. This can most simply be
ensured through a barrier synchronization.


\restrictions
The restrictions to the \code{reduction} clause are as follows:

\begin{itemize}
\item All the common restrictions to all reduction clauses, which are listed in
Section \ref{subsubsec:Properties Common To All Reduction Clauses}, apply to
this clause.

\item A list item that appears in a \code{reduction} clause of a worksharing
construct must be shared in the \code{parallel} region to which a corresponding
worksharing region binds.

\item A list item that appears in a \code{reduction} clause of the innermost
enclosing worksharing or \code{parallel} construct may not be accessed in an
explicit task generated by a construct for which an \code{in_reduction} clause
over the same list item does not appear.

\begin{ccppspecific}
\item If a list item in a \code{reduction} clause on a worksharing
construct has a reference type then it must bind to the same object for all threads of the team.
\end{ccppspecific}
\end{itemize}










\subsubsection{\hcode{task_reduction} Clause}
\index{task_reduction@{\code{task_reduction}}}
\index{clauses!task_reduction@{\code{task_reduction}}}
\label{subsubsec:task_reduction clause}
\summary
The \code{task_reduction} clause specifies a reduction among tasks.

\syntax
\begin{ompSyntax}
task_reduction(\plc{reduction-identifier }:\plc{ list})
\end{ompSyntax}
Where \plc{reduction-identifier} is defined in Section
\ref{subsubsec:Properties Common To All Reduction Clauses}.

\descr
The \code{task_reduction} clause is a reduction scoping clause, as described in
\ref{subsubsec:Reduction Scoping Clauses}.

For each list item, the number of copies is unspecified. Any copies associated
with the reduction are initialized before they are accessed by the tasks
participating in the reduction. After the end of the region, the original list
item contains the result of the reduction.

\restrictions
The restrictions to the \code{task_reduction} clause are as follows:

\begin{itemize}
\item All the common restrictions to all reduction clauses, which are listed in
Section \ref{subsubsec:Properties Common To All Reduction Clauses}, apply to
this clause.
\end{itemize}










%\newpage %% HACK

\subsubsection{\hcode{in_reduction} Clause}
\index{in_reduction@{\code{in_reduction}}}
\index{clauses!in_reduction@{\code{in_reduction}}}
\label{subsubsec:in_reduction clause}
\summary
The \code{in_reduction} clause specifies that a task participates in a reduction.

\syntax
\begin{ompSyntax}
in_reduction(\plc{reduction-identifier }:\plc{ list})
\end{ompSyntax}
Where \plc{reduction-identifier} is defined in Section \ref{subsubsec:Properties Common To All Reduction Clauses}

\descr
The \code{in_reduction} clause is a reduction participating clause, as
described in Section \ref{subsubsec:Reduction Participating Clauses}, that
defines a task to be a participant in the task reduction defined by an
enclosing \code{taskgroup} region for the list item. 

For the \code{task} construct, the generated task becomes the participating
task. For each list item, a private copy may be created as if the \code{private}
clause had been used.

For the \code{target} construct, the target task becomes the participating
task. For each list item, a private copy will be created in the data
environment of the target task as if the \code{private} clause had been used,
and this private copy will be implicitly mapped into the device data
environment of the target device.

At the end of the task region, if a private copy was created its value is
combined with a copy created by a reduction scoping clause or with the original
list item.

\restrictions
The restrictions to the \code{in_reduction} clause are as follows:

\begin{itemize}
\item All the common restrictions to all reduction clauses, which are listed in
Section \ref{subsubsec:Properties Common To All Reduction Clauses}, apply to
this clause.


\item A list item that appears in an \code{in_reduction} clause of a
task-generating construct must appear in a \code{task_reduction} clause of a construct
corresponding to a taskgroup region that includes the participating task in its
\plc{taskgroup set}. The construct corresponding to the innermost region that meets
this condition must specify the same \plc{reduction-identifier} as the
\code{in_reduction} clause.
\end{itemize}










\subsection{Data Copying Clauses}
\label{subsec:Data Copying Clauses}
\index{data copying clauses}
\index{clauses!data copying}
This section describes the \code{copyin} clause (allowed on the \code{parallel} directive and
combined parallel worksharing directives) and the \code{copyprivate} clause (allowed on
the \code{single} directive).

These clauses support the copying of data values from private or threadprivate variables
on one implicit task or thread to the corresponding variables on other implicit tasks or
threads in the team.

The clauses accept a comma-separated list of list items (see \specref{sec:Directive Format}).
All list items appearing in a clause must be visible, according to the scoping rules of the
base language. Clauses may be repeated as needed, but a list item that specifies a given
variable may not appear in more than one clause on the same directive.

\begin{fortranspecific}
An associate name preserves the association with the selector established at the \code{ASSOCIATE} statement. A list item that appears in a data copying clause may be a selector of an \code{ASSOCIATE} construct. If the construct association is established prior to a parallel region, the association between the associate name and the original list item will be retained in the region.
\end{fortranspecific}







\subsubsection{\hcode{copyin} Clause}
\index{copyin@{\code{copyin}}}
\index{clauses!copyin@{\code{copyin}}}
\label{subsubsec:copyin clause}
\summary
The \code{copyin} clause provides a mechanism to copy the value of the master thread's
threadprivate variable to the threadprivate variable of each other member of the team
executing the \code{parallel} region.

\syntax
The syntax of the \code{copyin} clause is as follows:

\begin{ompSyntax}
copyin(\plc{list})
\end{ompSyntax}

\descr
\begin{ccppspecific}
The copy is done after the team is formed and prior to the start of execution of the
associated structured block. For variables of non-array type, the copy occurs by copy
assignment. For an array of elements of non-array type, each element is copied as if by
assignment from an element of the master thread's array to the corresponding element of
the other thread's array.
\end{ccppspecific}

\begin{cppspecific}
For class types, the copy assignment operator is invoked. The order in which copy
assignment operators for different variables of class type are called is unspecified.
\end{cppspecific}

\begin{fortranspecific}
The copy is done, as if by assignment, after the team is formed and prior to the start of
execution of the associated structured block.

On entry to any \code{parallel} region, each thread's copy of a variable that is affected by
a \code{copyin} clause for the \code{parallel} region will acquire the allocation, association, and
definition status of the master thread's copy, according to the following rules:

\begin{itemize}
\item If the original list item has the \code{POINTER} attribute, each copy receives the same
association status of the master thread's copy as if by pointer assignment.

\item If the original list item does not have the \code{POINTER} attribute, each copy becomes
defined with the value of the master thread's copy as if by intrinsic assignment,
unless it has the allocation status of unallocated, in which case each copy
will have the same status.
\end{itemize}
\end{fortranspecific}

\restrictions
The restrictions to the \code{copyin} clause are as follows:
\begin{ccppspecific}
\begin{itemize}
\item A list item that appears in a \code{copyin} clause must be threadprivate.

\item A variable of class type (or array thereof) that appears in a \code{copyin} clause requires
an accessible, unambiguous copy assignment operator for the class type.
\end{itemize}
\end{ccppspecific}

\begin{fortranspecific}
\begin{itemize}
\item A list item that appears in a \code{copyin} clause must be threadprivate. Named variables
appearing in a threadprivate common block may be specified: it is not necessary to
specify the whole common block.

\item A common block name that appears in a \code{copyin} clause must be declared to be a
common block in the same scoping unit in which the \code{copyin} clause appears.

\item If the list item is a polymorphic variable with the \code{ALLOCATABLE} attribute, the behavior is unspecified.
\end{itemize}
\end{fortranspecific}









\subsubsection{\hcode{copyprivate} Clause}
\index{copyprivate@{\code{copyprivate}}}
\index{clauses!copyprivate@{\code{copyprivate}}}
\label{subsubsec:copyprivate clause}
\summary
The \code{copyprivate} clause provides a mechanism to use a private variable to broadcast
a value from the data environment of one implicit task to the data environments of the
other implicit tasks belonging to the \code{parallel} region.

To avoid race conditions, concurrent reads or updates of the list item must be
synchronized with the update of the list item that occurs as a result of the
\code{copyprivate} clause.

\syntax
The syntax of the \code{copyprivate} clause is as follows:

\begin{ompSyntax}
copyprivate(\plc{list})
\end{ompSyntax}

\descr
The effect of the \code{copyprivate} clause on the specified list items occurs after the
execution of the structured block associated with the \code{single} construct (see
\specref{subsec:single Construct}),
and before any of the threads in the team have left the barrier
at the end of the construct.

\begin{ccppspecific}
In all other implicit tasks belonging to the \code{parallel} region, each specified list item
becomes defined with the value of the corresponding list item in the implicit task associated with the
thread that executed the structured block. For variables of non-array type, the definition
occurs by copy assignment. For an array of elements of non-array type, each element is
copied by copy assignment from an element of the array in the data environment of the
implicit task associated with the thread that executed the structured block to the
corresponding element of the array in the data environment of the other implicit tasks
\end{ccppspecific}

\begin{cppspecific}
For class types, a copy assignment operator is invoked. The order in which copy
assignment operators for different variables of class type are called is unspecified.
\end{cppspecific}

\begin{fortranspecific}
If a list item does not have the \code{POINTER} attribute, then in all other implicit tasks
belonging to the \code{parallel} region, the list item becomes defined as if by intrinsic
assignment with the value of the corresponding list item in the implicit task associated
with the thread that executed the structured block.

If the list item has the \code{POINTER} attribute, then, in all other implicit tasks belonging to
the \code{parallel} region, the list item receives, as if by pointer assignment, the same
association status of the corresponding list item in the implicit task associated with the
thread that executed the structured block.

The order in which any final subroutines for different variables of a finalizable type are called is unspecified.
\end{fortranspecific}

\begin{note}
The \code{copyprivate} clause is an alternative to using a shared variable for the
value when providing such a shared variable would be difficult (for example, in a
recursion requiring a different variable at each level).
\end{note}

\restrictions
The restrictions to the \code{copyprivate} clause are as follows:

\begin{itemize}
\item All list items that appear in the \code{copyprivate} clause must be either threadprivate
or private in the enclosing context.

\item A list item that appears in a \code{copyprivate} clause may not appear in a \code{private} or
\code{firstprivate} clause on the \code{single} construct.

\begin{cppspecific}
\item A variable of class type (or array thereof) that appears in a \code{copyprivate} clause
requires an accessible unambiguous copy assignment operator for the class type.
\end{cppspecific}

\begin{fortranspecific}
\item A common block that appears in a \code{copyprivate} clause must be threadprivate.

\item Pointers with the \code{INTENT(IN)} attribute may not appear in the \code{copyprivate}
clause.
\item The list item with the \code{ALLOCATABLE} attribute must have the allocation status of allocated when the intrinsic assignment is performed.

\item If the list item is a polymorphic variable with the \code{ALLOCATABLE} attribute, the behavior is unspecified.
\end{fortranspecific}
\end{itemize}







%%% map
\vspace{-12 pt} %% UGLY HACK
\subsection{Data-mapping Attribute Rules and Clauses}
\label{subsec:Data-mapping Attribute Rules and Clauses}
\index{data-mapping rules and clauses}
\index{attributes, data-mapping}

%% Do we need something about
%% ``explicit \code{declare}~\code{target} directives''?
This section describes how the data-mapping attributes of any variable
referenced in a \code{target} region are determined. When specified,
explicit \code{map} clauses on \code{target}
directives determine these attributes.  Otherwise, the following
data-mapping rules apply for variables referenced in a \code{target}
construct that are not declared in the construct and do not appear in
data-sharing attribute, \code{map} or \code{is_device_ptr} clauses:

Certain variables and objects have predetermined data-mapping attributes
as follows:

\begin{itemize}
\item If a variable appears in a \code{to} or \code{link} clause on a
  \code{declare}~\code{target} directive then it is treated as if it had
  appeared in a \code{map} clause with a \plc{map-type} of \code{tofrom}.

\begin{ccppspecific}

\item If a \code{defaultmap} clause that specifies an implicit behavior other
  than \code{default} for pointer variables does not appear on the \code{target}
  construct, a variable that is of type pointer is treated as if it is the named
  pointer of a zero-length array section that appeared as a list item in a
  \code{map} clause.  

\end{ccppspecific}

\begin{cppspecific}
\item A variable that is of type reference to pointer is treated as if it had appeared in a \code{map} clause as a zero-length array section.
\item A class member variable is treated as if the \plc{this[:1]} expression
  had appeared in a \code{map} clause.
\end{cppspecific}
\end{itemize}

For all variables and objects that do not have a predetermined data-mapping or
data-sharing attribute, the following implicit data-mapping attribute rules
apply:

\begin{itemize}
  \item If a \code{defaultmap} clause is present for the category of the
    variable, the data-mapping attribute is determined by that clause.

\item If a variable is not a scalar then it is treated as if it had appeared 
in a \code{map} clause with a \plc{map-type} of \code{tofrom}.

\item If a list item appears in a \code{lastprivate} or \code{linear} clause on a
    combined target construct then it is treated as if it also appears in a
    \code{map} clause with a \plc{map-type} of \code{tofrom}.

\item If a list item appears in a \code{reduction} clause on a
    \code{target} construct then it is treated as if it also appears in a
    \code{map} clause with a \plc{map-type} of \code{tofrom}.

\item If a list item appears in an \code{in_reduction} clause on a
    \code{target} construct then it is treated as if it also appears in a
    \code{map} clause with a \plc{map-type} of \code{tofrom} and a
    \plc{map-type-modifier} of \code{always}.

\begin{fortranspecific}
\item If a scalar variable has the \code{TARGET}, \code{ALLOCATABLE}
or \code{POINTER} attribute then
it is treated as if it has appeared in a \code{map} clause with a
\plc{map-type} of \code{tofrom}.
\end{fortranspecific}

\item If none of the above rules applies then a scalar variable is not
mapped, but instead has an implicit data-sharing attribute of
firstprivate (see \specref{subsubsec:Data-sharing Attribute Rules
for Variables Referenced in a Construct}).
\end{itemize}

\subsubsection{\hcode{map} Clause}
\label{subsec:map Clause}
\index{map@{\code{map}}}
\index{clauses!map@{\code{map}}}
\summary
The \code{map} clause specifies how an original list item is mapped from the current task's data environment to a corresponding list item in the device data environment of the device identified by the construct.

\syntax
The syntax of the map clause is as follows:

\begin{ompSyntax}
map(\plc{[ [map-type-modifier[,]] map-type} : \plc{] list})
\end{ompSyntax}

where \plc{map-type} is one of the following:

\begin{indentedcodelist}
to
from
tofrom
alloc
release
delete
\end{indentedcodelist}

and \plc{map-type-modifier} is one of the following:

\begin{indentedcodelist}
always
mapper(\plc{mapper-identifier})
\end{indentedcodelist}

\descr
The list items that appear in a \code{map} clause may include array sections and structure elements.



The \plc{map-type} and \plc{map-type-modifier} specify the effect of the \code{map} clause, as described below.

For a given construct, the effect of a \code{map} clause with the \code{to},
\code{from}, or \code{tofrom} \plc{map-type} is ordered before the effect of a
\code{map} clause with the \code{alloc}, \code{release}, or \code{delete}
\plc{map-type}. If a mapper is specified for the type being mapped, or
explicitly specified with the \verb`mapper` \plc{map-type-modifier}, then the
effective \code{map-type} of a list item will be determined according to the
rules of map-type decay.

If a mapper is specified for the type being mapped, or explicitly specified with
the \verb`mapper` \plc{map-type-modifier}, then all map clauses that appear on the
\verb`declare mapper` directive are treated as though they appeared on the
construct with the \verb`map` clause.  Array sections of a mapper type are
mapped as normal, then each element in the array section is mapped according to
the rules of the mapper.

\begin{ccppspecific}
If a list item in a \code{map} clause is a variable of structure type then it
is treated as if each structure element contained in the variable is a list
item in the clause.
\end{ccppspecific}

\begin{fortranspecific}
If a list item in a \code{map} clause is a derived type variable
then it is treated as if each nonpointer component is a list item in
the clause.
\end{fortranspecific}

If a list item in a \code{map} clause is a structure element then all other
structure elements (except pointer component, for Fortran) of the
containing structure variable form a \plc{structure sibling list}.  The \code{map} clause and the structure sibling list are
associated with the same construct.  If a corresponding list item of the
structure sibling list item is present in the device data environment when the
construct is encountered then:
    \begin{itemize}

    \item If the structure sibling list item does not appear in a \code{map}
    clause on the construct then:
    \begin{itemize}

        \item If the construct is a \code{target}, \code{target}~\code{data},
        or \code{target}~\code{enter}~\code{data} construct then the structure
        sibling list item is treated as if it is a list item in a \code{map}
        clause on the construct with a \plc{map-type} of \code{alloc}.

        \item If the construct is \code{target}~\code{exit}~\code{data}
        construct, then the structure sibling list item is treated as if it is
        a list item in a \code{map} clause on the construct with a
        \plc{map-type} of \code{release}.

    \end{itemize}

    \item If the \code{map} clause in which the structure element appears as a
    list item has a \plc{map-type} of \code{delete} and the structure sibling
    list item does not appear as a list item in a \code{map} clause on the
    construct with a \plc{map-type} of \code{delete} then the structure sibling
    list item is treated as if it is a list item in a \code{map} clause on the
    construct with a \plc{map-type} of \code{delete}.

\end{itemize}

\begin{fortranspecific}
If a list item in a \code{map} clause has the POINTER attribute and if
the association status of the list item is associated, then it is
treated as if the pointer target is a list item in the clause.
\end{fortranspecific}

\begin{ccppspecific}
If a list item in a \code{map} clause is a variable of pointer type, and the
variable is the named pointer of an array section that is a list item in a
\code{map} clause on the same construct, then the effect of a \code{map} clause
on the pointer variable and the effect of a \code{map} clause on the array section are
ordered as follows:

\begin{itemize}

\item If the \code{map} clauses appear on a \code{target},
\code{target}~\code{data}, or \code{target}~\code{enter}~\code{data} construct
then on entry to the region the effect of the \code{map} clause on the pointer
variable is ordered to occur before the effect of the \code{map} clause on
the array section.

\item If the \code{map} clauses appears on a \code{target},
\code{target}~\code{data}, or \code{target}~\code{exit}~\code{data} construct
then on exit from the region the effect of the \code{map} clause on the array
section is ordered to occur before the effect of the \code{map} clause on the
pointer variable.

\end{itemize}

If an array section with a named pointer is a list item in a \code{map} clause
and a pointer variable is present in the device data environment that
corresponds to the named pointer when the effect of the \code{map} clause
occurs, then if the corresponding array section is created in the device data
environment: \begin{enumerate}

\item The corresponding pointer variable is assigned the address of the
corresponding array section.

\item The corresponding pointer variable becomes an attached pointer
for the corresponding array section.

\end{enumerate}

\end{ccppspecific}

\begin{cppspecific}
If a \plc{lambda} is mapped explicitly or implicitly, variables
that are captured by the \plc{lambda} behave as follows:
\begin{itemize}
\item the variables that are of pointer type are treated as if they had
appeared in a \code{map} clause as zero-length array sections

\item the variables that are of reference type are treated as if they
had appeared in a \code{map} clause.
\end{itemize}

If a member variable is captured by a \plc{lambda} in class scope, and
the \plc{lambda} is later mapped explicitly or implicitly with its
full static type, the \plc{this} pointer is treated as if it had
appeared on a \code{map} clause.

\end{cppspecific}

The original and corresponding list items may share storage such that writes to either
item by one task followed by a read or write of the other item by another task without
intervening synchronization can result in data races.

If the \code{map} clause appears on a \code{target}, \code{target}~\code{data}, or \code{target}~\code{enter}~\code{data} construct then on entry to the region the following sequence of steps occurs as if performed as a single atomic operation:

% allocate storage
\begin{enumerate}
\item If a corresponding list item of the original list item is not present in the device data environment, then:
\begin{enumerate}
\item A new list item with language-specific attributes is derived from the original list item and created in the device data environment.
\item The new list item becomes the corresponding list item to the original list item in the device data environment.
\item The corresponding list item has a reference count that is initialized to zero.
\item The value of the corresponding list item is undefined.
\end{enumerate}

% incremenent the reference counter
\item If the corresponding list item's reference count was not already
incremented because of the effect of a \code{map} clause on the
construct then:
\begin{enumerate}
\item The corresponding list item's reference count is incremented by one
\end{enumerate}

% assign the corresponding variable
\item If the corresponding list item's reference count is one or the \code{always} \plc{map-type-modifier} is present, then:
\begin{enumerate}
\item If the \plc{map-type} is \code{to} or \code{tofrom}, then:
    \begin{itemize}

    \item For each part of the list item that is an attached pointer:  \begin{itemize}

        \item That part of the corresponding list item will
        have the value it had immediately prior to the effect of the \code{map} clause;

   \end{itemize}

    \item For each part of the list item that is not an attached pointer: \begin{itemize}

        \item The value of that part of the
        original list item is assigned to that part of the corresponding list item.

   \end{itemize}

    \item Concurrent reads or updates of any part
        of the corresponding list item must be synchronized with the update of the
        corresponding list item that occurs as a result of the \code{map} clause.

\end{itemize}
\end{enumerate}

\end{enumerate}

\begin{note}
If the effect of the \code{map} clauses on a construct would assign the
value of an original list item to a corresponding list item more than once,
then an implementation is allowed to ignore additional assignments of
the same value to the corresponding list item.
\end{note}

If the \code{map} clause appears on a \code{target}, \code{target}~\code{data}, or \code{target}~\code{exit}~\code{data} construct then on exit from the region the following sequence of steps occurs as if performed as a single atomic operation:
\begin{enumerate}
\item If a corresponding list item of the original list item is not present in the device data environment, then the list item is ignored.
\item If a corresponding list item of the original list item is present in the device data environment, then:
\begin{enumerate}

% decrement the ref count
\item If the corresponding list item's reference count is finite, then:
\begin{enumerate}
\item If the corresponding list item's reference count was not already
decremented because of the effect of a \code{map} clause on the
construct then:
\begin{enumerate}
\item If the \plc{map-type} is not \code{delete}, then the corresponding list
                item's reference count is decremented by one.
\end{enumerate}
\item If the \plc{map-type} is \code{delete}, then the corresponding list item's reference count is set to zero.
\end{enumerate}

% assign the original variable
\item If the corresponding list item's reference count is zero or the \code{always} \plc{map-type-modifier} is present, then:
\begin{enumerate}
\item If the \plc{map-type} is \code{from} or \code{tofrom} then:
\begin{itemize}

              \item For each part of the list item that is an attached pointer:  \begin{itemize}

                  \item That part of the original list item will
                    have the value it had immediately prior to the effect of the \code{map} clause;

                \end{itemize}

              \item For each part of the list item that is not an attached pointer: \begin{itemize}

                  \item The value of that part of the
                    corresponding list item is assigned to that part of the original list item;

                \end{itemize}

              \item To avoid race conditions: \begin{itemize}

                  \item Concurrent reads or updates of any part
                    of the original list item must be synchronized with the update of the
                    original list item that occurs as a result of the \code{map} clause;

                \end{itemize}


\end{itemize}
\end{enumerate}

% release the storage
\item If the corresponding list item's reference count is zero, then the corresponding list item is removed from the device data environment
\end{enumerate}
\end{enumerate}

\begin{note}
If the effect of the \code{map} clauses on a construct would assign the
value of a corresponding list item to an original list item more than once,
then an implementation is allowed to ignore additional assignments of
the same value to the original list item.
\end{note}

If a single contiguous part of the original storage of a list item with an
implicit data-mapping attribute has corresponding storage in the device data
environment prior to a task encountering the construct associated with the
\code{map} clause, only that part of the original storage will have
corresponding storage in the device data environment as a result of the \code{map}
clause.

\begin{ccppspecific}
If a new list item is created then a new list item of the same type, with automatic storage
duration, is allocated for the construct. The size and alignment of the new list
item are determined by the static type of the variable. This allocation occurs if the region
references the list item in any statement.
\end{ccppspecific}

\begin{fortranspecific}
If a new list item is created then a new list item of the same type, type parameter, and
rank is allocated.
\end{fortranspecific}

The \plc{map-type} determines how the new list item is initialized.

If a \plc{map-type} is not specified, the \plc{map-type} defaults to \code{tofrom}.

\events
The \plc{target-map} event occurs when a thread maps data to or from a target device.

The \plc{target-data-op} event occurs when a thread initiates a data operation on a target device.

\tools

A thread dispatches a registered \code{ompt_callback_target_map}
callback for each occurrence of a \plc{target-map} event in that thread.
The callback occurs in the context of the target task.  The callback has type signature
\code{ompt_callback_target_map_t}.

A thread dispatches a registered \code{ompt_callback_target_data_op}
callback for each occurrence of a \plc{target-data-op} event in that thread.
The callback occurs in the context of the target task.  The callback has type signature
\code{ompt_callback_target_data_op_t}.

\restrictions
\begin{itemize}

\item A list item cannot appear in both a \code{map} clause and a data-sharing
    attribute clause on the same construct, unless the data-sharing attribute
    clause is a \code{reduction} clause or the construct is a combined construct.

\begin{ccppspecific}

\item If a list item is an array section and the type of its base expression is
a pointer type, the base expression must be an lvalue expression.

\end{ccppspecific}

\item If a list item is an array section, it must specify contiguous storage.

\item If more than one list item of the \code{map} clauses on the same construct
are, or are part of, array items that have the same named array, they must indicate
identical original storage.

\item List items of the \code{map} clauses on the same construct must not share
original storage unless they are the same variable or array section.

\item If any part of the original storage of a list item with a predetermined or
explicit data-mapping attribute has corresponding storage in the device data
environment prior to a task encountering the construct associated with the map
clause, all of the original storage must have corresponding storage
in the device data environment prior to the task encountering the construct.

\item If a list item is an element of a structure, and a different element
of the structure has a corresponding list item in the device data environment
prior to a task encountering the construct associated with the \code{map}
clause, then the list item must also have a corresponding list item in the
device data environment prior to the task encountering the construct.

\item If a list item is an element of a structure, only the rightmost symbol of the variable reference can be an array section.

\item A list item must have a mappable type.

\item \code{threadprivate} variables cannot appear in a \code{map} clause.

\item If a \code{mapper} map-type-modifier is specified, its type must match the type of the list-items passed to that map clause.

\item Memory spaces and memory allocators cannot appear as a list item in a \code{map} clause.

\begin{cppspecific}
\item If the type of a list item is a reference to a type \plc{T} then the type will be considered to be \plc{T} for all purposes of this clause.

\item If the list item is a \plc{lambda}, any pointers and references captured by
the \plc{lambda} must have the corresponding list item in the device
data environment prior to the task encountering the construct.

\item In the class scope, if the \plc{lambda} is passed as a parameter to a
function in which it is specified in the \code{map} clause, the
behavior is unspecified.
\end{cppspecific}

\begin{ccppspecific}
\item Initialization and assignment are through bitwise copy.
\item A list item cannot be a variable that is a member of a structure with a union type.

\item A bit-field cannot appear in a \code{map} clause.

\item A pointer that has a corresponding attached pointer may not be modified
      for the duration of the lifetime of the array section to which the
      corresponding pointer is attached in the device data environment.
\end{ccppspecific}
\end{itemize}

\begin{fortranspecific}
\begin{itemize}
\item The value of the new list item becomes that of the original list item in the map
initialization and assignment.

\item A list item must not contain any components that have the
  \code{ALLOCATABLE} attribute.

\item If the allocation status of a list item with the
  \code{ALLOCATABLE} attribute is unallocated upon entry to a
  \code{target} region, the list item must be unallocated
  upon exit from the region.

\item If the allocation status of a list item with the
  \code{ALLOCATABLE} attribute is allocated upon entry to a
  \code{target} region, the allocation status of the corresponding
  list item must not be changed and must not be reshaped in the
  region.

\item If an array section is mapped and the size of the section is
  smaller than that of the whole array, the behavior of referencing
  the whole array in the \code{target} region is unspecified.

\item A list item must not be a whole array of an assumed-size array.

\item If the association status of a list item with the \code{POINTER}
  attribute is associated upon entry to a \code{target} region, the
  list item remains associated with the same pointer target upon exit
  from the region.

\item If the association status of a list item with the \code{POINTER}
  attribute is disassociated upon entry to a \code{target} region, the
  list item must be disassociated upon exit from the region.

\item If the association status of a list item with the \code{POINTER}
  attribute is undefined upon entry to a \code{target} region, the
  list item must be undefined upon exit from the region.

\item If the association status of a list item with the \code{POINTER}
  attribute is disassociated or undefined on entry and if the list
  item is associated with a pointer target inside a \code{target}
  region, then the pointer association status must become
  disassociated before the end of the region; otherwise the behavior
  is unspecified.

\item If the allocation status of the original list item with the
  \code{ALLOCATABLE} attribute is changed on the host device data
  environment and the corresponding list item is already present on
  the device data environment, the behavior is unspecified.

\end{itemize}
\end{fortranspecific}

\crossreferences
\begin{itemize}

\item \code{ompt_callback_target_map_t}, see
\specref{sec:ompt_callback_target_map_t}.

\item \code{ompt_callback_target_data_op_t}, see
\specref{sec:ompt_callback_target_data_op_t}.
\end{itemize}




\subsubsection{\hcode{defaultmap} Clause}
\label{subsubsec:defaultmap clause}
\index{defaultmap@{\code{defaultmap}}}
\index{clauses!defaultmap@{\code{defaultmap}}}

\summary

The \code{defaultmap} clause redefines the implicit data-mapping attributes of
variables that are referenced in a \code{target} construct and are implicitly
determined.

\syntax
The syntax of the \code{defaultmap} clause is as follows:

\begin{ompSyntax}
  defaultmap(\plc{implicit-behavior[:variable-category]})
\end{ompSyntax}

Where \plc{implicit-behavior} is one of:
\begin{indentedcodelist}
  alloc
  to
  from
  tofrom
  firstprivate
  none
  default
\end{indentedcodelist}

\begin{ccppspecific}
and \plc{variable-category} is one of:
\begin{indentedcodelist}
  scalar
  aggregate
  pointer
\end{indentedcodelist}
\end{ccppspecific}

\begin{fortranspecific}
and \plc{variable-category} is one of:
\begin{indentedcodelist}
  scalar
  aggregate
  allocatable
  pointer
\end{indentedcodelist}
\end{fortranspecific}

\descr
The \code{defaultmap} clause sets the implicit data-mapping attribute for all variables
referenced in the construct. If \plc{variable-category} is specified, the effect of
the \code{defaultmap} clause is as follows:

\begin{itemize}

  \item If \plc{variable-category} is \code{scalar}, all scalar variables of
    non-pointer type or all non-pointer non-allocatable scalar variables that
    have an implicitly determined data-mapping or data-sharing attribute will
    have a data-mapping or data-sharing attribute specified by
    \plc{implicit-behavior}.

  \item  If \plc{variable-category} is \code{aggregate} or \code{allocatable},
    all aggregate or allocatable variables that have an implicitly determined
    data-mapping or data-sharing attribute will have a data-mapping or
    data-sharing attribute specified by \plc{implicit-behavior}.

  \item If \plc{variable-category} is \code{pointer}, all variables of pointer
    type or with the POINTER attribute that have implicitly determined
    data-mapping or data-sharing attributes will have a data-mapping or
    data-sharing attribute specified by \plc{implicit-behavior}.  The
    zero-length array section and attachment an implicitly mapped pointer
    normally gets is only provided for the \code{default} behavior.

\end{itemize}
 
If no \plc{variable-category} is specified in the clause then
\plc{implicit-behavior} specifies the implicitly determined data-mapping or
data-sharing attribute for all variables referenced in the construct.  If
\plc{implicit-behavior} is \code{none}, each variable referenced in the
construct that does not have a predetermined data-sharing attribute must be
listed in a data-mapping attribute clause, a data-sharing attribute clause, or
an \code{is_device_ptr} clause. If \plc{implicit-behavior} is \code{default},
then the clause has no effect for the variables in the category specified by
\plc{variable-category}.


% This is an included file. See the master file for more information.
%
% When editing this file:
%
%    1. To change formatting, appearance, or style, please edit openmp.sty.
%
%    2. Custom commands and macros are defined in openmp.sty.
%
%    3. Be kind to other editors -- keep a consistent style by copying-and-pasting to
%       create new content.
%
%    4. We use semantic markup, e.g. (see openmp.sty for a full list):
%         \code{}     % for bold monospace keywords, code, operators, etc.
%         \plc{}      % for italic placeholder names, grammar, etc.
%
%    5. There are environments that provide special formatting, e.g. language bars.
%       Please use them whereever appropriate.  Examples are:
%
%         \begin{fortranspecific}
%         This is text that appears enclosed in blue language bars for Fortran.
%         \end{fortranspecific}
%
%         \begin{note}
%         This is a note.  The "Note -- " header appears automatically.
%         \end{note}
%
%    6. Other recommendations:
%         Use the convenience macros defined in openmp.sty for the minor headers
%         such as Comments, Syntax, etc.
%
%         To keep items together on the same page, prefer the use of 
%         \begin{samepage}.... Avoid \parbox for text blocks as it interrupts line numbering.
%         When possible, avoid \filbreak, \pagebreak, \newpage, \clearpage unless that's
%         what you mean. Use \needspace{} cautiously for troublesome paragraphs.
%
%         Avoid absolute lengths and measures in this file; use relative units when possible.
%         Vertical space can be relative to \baselineskip or ex units. Horizontal space
%         can be relative to \linewidth or em units.
%
%         Prefer \emph{} to italicize terminology, e.g.:
%             This is a \emph{definition}, not a placeholder.
%             This is a \plc{var-name}.
%


\section{\code{declare}~\code{reduction} Directive}
\index{declare reduction@{\code{declare}~\code{reduction}}}
\index{directives!declare reduction@{\code{declare}~\code{reduction}}}
\label{sec:declare reduction Directive}
\summary
The following section describes the directive for declaring user-defined reductions. The 
\code{declare}~\code{reduction} directive declares a \plc{reduction-identifier} that can be used in a 
\code{reduction} clause. The \code{declare}~\code{reduction} directive is a declarative directive.

\syntax
\begin{cspecific}
\begin{boxedcode}
\#pragma omp declare reduction(\plc{reduction-identifier }:\plc{ typename-list }: 
\plc{combiner })\plc{[initializer-clause] new-line}
\end{boxedcode}

where:

\begin{itemize}

\item \plc{reduction-identifier} is either a base language identifier or one of the following 
operators: 
\code{+}, 
\code{-}, 
\code{*}, 
\code{\&}, 
\code{|}, 
\code{\^}, 
\code{\&\&} and 
\code{||} 

\item \plc{typename-list} is a list of type names 

\item \plc{combiner} is an expression 

\item \plc{initializer-clause} is \code{initializer(}\plc{initializer-expr}\code{)}
where \plc{initializer-expr} is\linebreak
\code{omp\_priv = }\plc{initializer} or \plc{function-name}\code{(}\plc{argument-list}\code{)} 
\end{itemize}
\end{cspecific}


\begin{cppspecific}
\begin{boxedcode}
\#pragma omp declare reduction(\plc{reduction-identifier }:\plc{ typename-list }: 
\plc{combiner})\plc{ [initializer-clause] new-line}
\end{boxedcode}

where:

\begin{itemize}
\item \plc{reduction-identifier} is either an \plc{id-expression} or one of the following 
operators: 
\code{+}, 
\code{-}, 
\code{*}, 
\code{\&}, 
\code{|}, 
\code{\^}, 
\code{\&\&} and 
\code{||} 

\item \plc{typename-list} is a list of type names 

\item \plc{combiner} is an expression 

% An equal sign is intentionally missing for C++, so that
% initializer (omp_priv (4)) syntax is also valid in addition to
% initializer (omp_priv = 4).
\item \plc{initializer-clause} is \code{initializer(}\plc{initializer-expr}\code{)} 
where \plc{initializer-expr} is\linebreak
\code{omp\_priv} \plc{initializer} or \plc{function-name}\code{(}\plc{argument-list}\code{)} 
\end{itemize}
\end{cppspecific}


\begin{fortranspecific}
\begin{boxedcode}
!\$omp declare reduction(\plc{reduction-identifier }:\plc{ type-list }:\plc{ combiner}) 
\plc{[initializer-clause]}
\end{boxedcode}

where:

\begin{itemize}
\item \plc{reduction-identifier} is either a base language identifier, or a user-defined operator, or 
one of the following operators: 
\code{+}, 
\code{-}, 
\code{*}, 
\code{.and.}, 
\code{.or.}, 
\code{.eqv.}, 
\code{.neqv.}, or one of the following intrinsic procedure names: 
\code{max}, 
\code{min}, 
\code{iand}, 
\code{ior}, 
\code{ieor}. 

\item \plc{type-list} is a list of type specifiers 

\item \plc{combiner} is either an assignment statement or a subroutine name followed by an 
argument list 

\item \plc{initializer-clause} is \code{initializer(}\plc{initializer-expr}\code{)}, 
where \plc{initializer-expr} is\linebreak
\code{omp\_priv = }\plc{expression} or \plc{subroutine-name}\code{(}\plc{argument-list}\code{)}
\end{itemize}
\end{fortranspecific}

\descr
Custom reductions can be defined using the \code{declare}~\code{reduction} directive; the 
\plc{reduction-identifier} and the type identify the \code{declare}~\code{reduction} directive. The 
\plc{reduction-identifier} can later be used in a \code{reduction} clause using variables of the 
type or types specified in the \code{declare}~\code{reduction} directive. If the directive applies 
to several types then it is considered as if there were multiple \code{declare}~\code{reduction} 
directives, one for each type.

\begin{fortranspecific}
If a type with deferred or assumed length type parameter is specified in a \code{declare}~\code{reduction} directive, the \plc{reduction-identifier} of that directive can be used in a \code{reduction} clause with any variable of the same type and the same kind parameter, regardless of the length type Fortran parameters with which the variable is declared.
\end{fortranspecific}

The visibility and accessibility of this declaration are the same as those of a variable 
declared at the same point in the program. The enclosing context of the \plc{combiner} and of 
the \plc{initializer-expr} will be that of the \code{declare}~\code{reduction} directive. The \plc{combiner} 
and the \plc{initializer-expr} must be correct in the base language as if they were the body of 
a function defined at the same point in the program.

\begin{fortranspecific}
If the \plc{reduction-identifier} is the same as the name of a user-defined operator or an extended operator, or the same as a generic name that is one of the allowed intrinsic procedures, and if the operator or procedure name appears in an accessibility statement in the same module, the accessibility of the corresponding \code{declare}~\code{reduction} directive is determined by the accessibility attribute of the statement.

If the \plc{reduction-identifier} is the same as a generic name that is one of the allowed intrinsic procedures and is accessible, and if it has the same name as a derived type in the same module, the accessibility of the corresponding \code{declare}~\code{reduction} directive is determined by the accessibility of the generic name according to the base language.
\end{fortranspecific}

\newpage %% HACK

\begin{cppspecific}
The \code{declare}~\code{reduction} directive can also appear at points in the program at which 
a static data member could be declared. In this case, the visibility and accessibility of 
the declaration are the same as those of a static data member declared at the same point 
in the program.
\end{cppspecific}

The \plc{combiner} specifies how partial results can be combined into a single value. The 
\plc{combiner} can use the special variable identifiers \code{omp\_in} and \code{omp\_out} that are of the 
type of the variables being reduced with this \plc{reduction-identifier}. Each of them will 
denote one of the values to be combined before executing the \plc{combiner}. It is assumed 
that the special \code{omp\_out} identifier will refer to the storage that holds the resulting 
combined value after executing the \plc{combiner}.

The number of times the \plc{combiner} is executed, and the order of these executions, for 
any \code{reduction} clause is unspecified.

\begin{fortranspecific}
If the \plc{combiner} is a subroutine name with an argument list, the \plc{combiner} is evaluated by 
calling the subroutine with the specified argument list.

If the \plc{combiner} is an assignment statement, the \plc{combiner} is evaluated by executing the 
assignment statement.
\end{fortranspecific}

As the \plc{initializer-expr} value of a user-defined reduction is not known \emph{a priori} the 
\plc{initializer-clause} can be used to specify one. Then the contents of the \plc{initializer-clause} 
will be used as the initializer for private copies of reduction list items where the 
\code{omp\_priv} identifier will refer to the storage to be initialized. The special identifier 
\code{omp\_orig} can also appear in the \plc{initializer-clause} and it will refer to the storage of the 
original variable to be reduced.

The number of times that the \plc{initializer-expr} is evaluated, and the order of these 
evaluations, is unspecified.

\begin{ccppspecific}
If the \plc{initializer-expr} is a function name with an argument list, the \plc{initializer-expr} is 
evaluated by calling the function with the specified argument list. Otherwise, the 
\plc{initializer-expr} specifies how \code{omp\_priv} is declared and initialized.
\end{ccppspecific}
\bigskip

\begin{cspecific}
If no \plc{initializer-clause} is specified, the private variables will be initialized following the 
rules for initialization of objects with static storage duration.
\end{cspecific}

\begin{cppspecific}
If no \plc{initializer-expr} is specified, the private variables will be initialized following the 
rules for \plc{default-initialization}.
\end{cppspecific}
\bigskip

\begin{fortranspecific}
If the \plc{initializer-expr} is a subroutine name with an argument list, the \plc{initializer-expr} is 
evaluated by calling the subroutine with the specified argument list.

If the \plc{initializer-expr} is an assignment statement, the \plc{initializer-expr} is evaluated by 
executing the assignment statement.

If no \plc{initializer-clause} is specified, the private variables will be initialized as follows:
\begin{itemize}
\item For \code{complex}, \code{real}, or \code{integer} types, the value 0 will be used.
\item For \code{logical} types, the value \code{.false.} will be used.
\item For derived types for which default initialization is specified, default initialization 
will be used.
\item Otherwise, not specifying an \plc{initializer-clause} results in unspecified behavior.
\end{itemize}
\end{fortranspecific}
\bigskip

\begin{ccppspecific}
If \plc{reduction-identifier} is used in a \code{target} region then a \code{declare}~\code{target} construct 
must be specified for any function that can be accessed through the \plc{combiner} and 
\plc{initializer-expr}.
\end{ccppspecific}
\bigskip

\begin{fortranspecific}
If \plc{reduction-identifier} is used in a \code{target} region then a \code{declare}~\code{target} construct 
must be specified for any function or subroutine that can be accessed through the \plc{combiner} 
and \plc{initializer-expr}.
\end{fortranspecific}

\restrictions
\begin{itemize}
  
\item The only variables allowed in the \plc{combiner} are
  \code{omp\_in} and \code{omp\_out}.

\item The only variables allowed in the \plc{initializer-clause} are
  \code{omp\_priv} and \code{omp\_orig}.

\item If the variable \code{omp\_orig} is modified in the \plc{initializer-clause}, the behavior is 
unspecified. 

\item If execution of the \plc{combiner} or the \plc{initializer-expr} results in the execution of an 
OpenMP construct or an OpenMP API call, then the behavior is unspecified.

\item A \plc{reduction-identifier} may not be re-declared in the current scope for the same type 
or for a type that is compatible according to the base language rules.

\item At most one \plc{initializer-clause} can be specified. 

\begin{ccppspecific}
\item A type name in a \code{declare}~\code{reduction} directive cannot be a function type, an 
array type, a reference type, or a type qualified with \code{const}, \code{volatile} or 
\code{restrict}. 
\end{ccppspecific}
\bigskip

\begin{cspecific}
\item If the \plc{initializer-expr} is a function name with an argument list, then one of the 
arguments must be the address of \code{omp\_priv}. 
\end{cspecific}
\bigskip

\begin{cppspecific}
\item If the \plc{initializer-expr} is a function name with an argument list, then one of the 
arguments must be \code{omp\_priv} or the address of \code{omp\_priv}. 
\end{cppspecific}
\bigskip

\begin{fortranspecific}
\item If the \plc{initializer-expr} is a subroutine name with an argument list, then one of the 
arguments must be \code{omp\_priv}.

\item If the \code{declare}~\code{reduction} directive appears in the specification part of a module and the corresponding reduction clause does not appear in the same module, the \plc{reduction-identifier} must be the same as the name of a user-defined operator, one of the allowed operators that is extended or a generic name that is the same as the name of one of the allowed intrinsic procedures. 

\item If the \code{declare}~\code{reduction} directive appears in the specification of a module, if the corresponding \code{reduction} clause does not appear in the same module, and if the \plc{reduction-identifier} is the same as the name of a user-defined operator or an extended operator, or the same as a generic name that is the same as one of the allowed intrinsic procedures then the interface for that operator or the generic name must be defined in the specification of the same module, or must be accessible by use association. 

\item Any subroutine or function used in the \code{initializer} clause or \plc{combiner} expression must be an intrinsic function, or must have an accessible interface. 

\item Any user-defined operator or extended operator used in the \code{initializer} clause or \plc{combiner} expression must have an accessible interface. 

\item If any subroutine, function, user-defined operator, or extended operator is used in the \code{initializer} clause or \plc{combiner} expression, it must be accessible to the subprogram in which the corresponding \code{reduction} clause is specified. 

\item If the length type parameter is specified for a character type, it must be a constant, a colon or an~\code{*}. 

\item If a character type with deferred or assumed length parameter is specified in a \code{declare}~\code{reduction} directive, no other \code{declare}~\code{reduction} directive with Fortran character type of the same kind parameter and the same \plc{reduction-identifier} is allowed in the same scope.

\item Any subroutine used in the \code{initializer} clause or \plc{combiner} expression must not have any alternate returns appear in the argument list.
\end{fortranspecific}
\end{itemize}

\crossreferences
\begin{itemize}
\item \code{reduction} clause, 
\specref{subsubsec:reduction clause}.
\end{itemize}











% This is an included file. See the master file for more information.
%
% When editing this file:
%
%    1. To change formatting, appearance, or style, please edit openmp.sty.
%
%    2. Custom commands and macros are defined in openmp.sty.
%
%    3. Be kind to other editors -- keep a consistent style by copying-and-pasting to
%       create new content.
%
%    4. We use semantic markup, e.g. (see openmp.sty for a full list):
%         \code{}     % for bold monospace keywords, code, operators, etc.
%         \plc{}      % for italic placeholder names, grammar, etc.
%
%    5. There are environments that provide special formatting, e.g. language bars.
%       Please use them whereever appropriate.  Examples are:
%
%         \begin{fortranspecific}
%         This is text that appears enclosed in blue language bars for Fortran.
%         \end{fortranspecific}
%
%         \begin{note}
%         This is a note.  The "Note -- " header appears automatically.
%         \end{note}
%
%    6. Other recommendations:
%         Use the convenience macros defined in openmp.sty for the minor headers
%         such as Comments, Syntax, etc.
%
%         To keep items together on the same page, prefer the use of
%         \begin{samepage}.... Avoid \parbox for text blocks as it interrupts line numbering.
%         When possible, avoid \filbreak, \pagebreak, \newpage, \clearpage unless that's
%         what you mean. Use \needspace{} cautiously for troublesome paragraphs.
%
%         Avoid absolute lengths and measures in this file; use relative units when possible.
%         Vertical space can be relative to \baselineskip or ex units. Horizontal space
%         can be relative to \linewidth or em units.
%
%         Prefer \emph{} to italicize terminology, e.g.:
%             This is a \emph{definition}, not a placeholder.
%             This is a \plc{var-name}.
%


\section{Nesting of Regions}
\label{sec:Nesting of Regions}
\index{nesting of regions}
This section describes a set of restrictions on the nesting of regions. The restrictions on
nesting are as follows:

\begin{itemize}
\item A worksharing region may not be closely nested inside a worksharing, \code{task}, \code{taskloop},
\code{critical}, \code{ordered}, \code{atomic}, or \code{master} region.

\item A \code{barrier} region may not be closely nested inside a worksharing, \code{task}, \code{taskloop},
\code{critical}, \code{ordered}, \code{atomic}, or \code{master} region.

\item A \code{master} region may not be closely nested inside a worksharing,
\code{atomic}, \code{task}, or \code{taskloop} region.

\item An \code{ordered} region arising from an \code{ordered} construct without
any clause or with the \code{threads} or \code{depend} clause may not be closely
nested inside a \code{critical}, \code{ordered}, \code{atomic}, \code{task},
or \code{taskloop} region.

\item An \code{ordered} region arising from an \code{ordered} construct without
any clause or with the \code{threads} or \code{depend} clause must be closely nested
inside a loop region (or parallel loop region) with an \code{ordered} clause.

\item An \code{ordered} region arising from an \code{ordered} construct with the
\code{simd} clause must be closely nested inside a \code{simd} (or loop SIMD)
region.

\item An \code{ordered} region arising from an \code{ordered} construct with
  both the \code{simd} and \code{threads} clauses must be closely nested inside
  a loop SIMD region.

\item A \code{critical} region may not be nested (closely or otherwise) inside a \code{critical}
region with the same name. This restriction is not sufficient to prevent
deadlock.

\item OpenMP constructs may not be encountered during execution of an
\code{atomic} region.

\item The only OpenMP constructs that can be encountered during execution of a
  \code{simd} (or loop SIMD) region are the \code{atomic} construct,
  \code{concurrent} construct and an \code{ordered} construct with the \code{simd} clause.

\item If a \code{target}, \code{target}~\code{update},
\code{target}~\code{data}, \code{target}~\code{enter}~\code{data}, or
\code{target}~\code{exit}~\code{data} construct is encountered during
execution of a \code{target} region, the behavior is unspecified.

\item A \code{teams} region can only be nested within the implicit parallel region or a \code{target} region.
If a \code{teams} construct is nested within a \code{target} construct,
that \code{target} construct must contain no statements, declarations or
directives outside of the \code{teams} construct.

\item \code{distribute}, \code{distribute simd}, distribute parallel loop,
  distribute parallel loop SIMD, \code{concurrent}, and \code{parallel} regions, including any
\code{parallel} regions arising from combined constructs, are the only OpenMP regions
that may be strictly nested inside the \code{teams} region.

\item The region associated with the \code{distribute} construct must be
strictly nested inside a \code{teams} region.

\item If \plc{construct-type-clause} is \code{taskgroup}, the \code{cancel}
construct must be closely nested inside a \code{task} construct and the
\code{cancel} region must be closely nested inside a \code{taskgroup} region. If
\plc{construct-type-clause} is \code{sections}, the \code{cancel} construct
must be closely nested inside a \code{sections} or \code{section} construct.
Otherwise, the \code{cancel} construct must be closely
nested inside an OpenMP construct that matches the type specified in
\plc{construct-type-clause} of the \code{cancel} construct.

\item A \code{cancellation}~\code{point} construct for which
\plc{construct-type-clause} is \code{taskgroup} must be closely nested
inside a \code{task} construct, and the \code{cancellation}~\code{point}
region must be closely nested inside a \code{taskgroup} region. A
\code{cancellation}~\code{point} construct for which
\plc{construct-type-clause} is \code{sections} must be closely nested
inside a \code{sections} or \code{section} construct. Otherwise, a
\code{cancellation}~\code{point} construct must be closely nested inside
an OpenMP construct that matches the type specified in
\plc{construct-type-clause}.

\item A \code{concurrent} region must be closely nested within a \code{teams}
  construct, \code{distribute} construct, \code{parallel} construct, Loop
  construct, SIMD construct, \code{concurrent} construct, or regions resulting
  from the combined or composite constructs composed of these constructs.

\item Only the following directives may be nested within a \code{concurrent}
  construct: \code{parallel} construct, Loop construct, SIMD construct,
  \code{concurrent} construct, \code{single} construct, and any combined or
  compound construct composed of these constructs.
  %JLARKIN We'd like to go back and unrestrict atomic later.

\end{itemize}





\section{Combined Constructs}
\label{sec:Combined Constructs}
\index{combined constructs}
\index{constructs!combined constructs}
Combined constructs are shortcuts for specifying one construct immediately nested 
inside another construct. The semantics of the combined constructs are identical to that 
of explicitly specifying the first construct containing one instance of the second 
construct and no other statements.

Some combined constructs have clauses that are permitted on both constructs that were
combined. Where specified, the effect is as if applying the clauses to one or both
constructs. If not specified and applying the clause to one construct would result in
different program behavior than applying the clause to the other construct then the
program's behavior is unspecified.

For combined constructs, tool callbacks shall be invoked as if the constructs were
explicitly nested.







\subsection{Parallel Loop Construct}
\label{subsec:Parallel Loop Construct}
\index{parallel loop construct}
\index{constructs!parallel loop construct}
\index{constructs!parallel for@{\code{parallel}~\code{for} \emph{C/C++}}}
\index{constructs!parallel do@{\code{parallel}~\code{do} \emph{Fortran}}}
\index{combined constructs!parallel loop construct}
\index{worksharing!parallel}
\summary
The parallel loop construct is a shortcut for specifying a \code{parallel} construct
containing one loop constuct with one or more associated loops and no other statements.

\newpage %% HACK

\syntax
\ccppspecificstart
The syntax of the parallel loop construct is as follows:

\begin{boxedcode}
\#pragma omp parallel for \plc{[clause[ [},\plc{] clause] ... ] new-line}
   \plc{for-loops}
\end{boxedcode}

where \plc{clause} can be any of the clauses accepted by the \code{parallel} or \code{for} directives, 
except the \code{nowait} clause, with identical meanings and restrictions.
\ccppspecificend

\fortranspecificstart
The syntax of the parallel loop construct is as follows:

\begin{boxedcode}
!\$omp parallel do \plc{[clause[ [},\plc{] clause] ... ]}
   \plc{do-loops}
\plc{[}!\$omp end parallel do\plc{]} 
\end{boxedcode}

where \plc{clause} can be any of the clauses accepted by the \code{parallel} or \code{do} directives, 
with identical meanings and restrictions. 

If an \code{end}~\code{parallel}~\code{do} directive is not specified, an \code{end}~\code{parallel}~\code{do} directive is 
assumed at the end of the \plc{do-loops}. \code{nowait} may not be specified on an 
\code{end}~\code{parallel}~\code{do} directive.
\fortranspecificend

\descr
The semantics are identical to explicitly specifying a \code{parallel} directive immediately 
followed by a loop directive.

\restrictions
\begin{itemize}
\item The restrictions for the \code{parallel} construct and the loop construct apply.
\end{itemize}

\crossreferences
\begin{itemize}
\item \code{parallel} construct, see 
\specref{sec:parallel Construct}.

\item loop SIMD construct, see 
\specref{subsec:Loop SIMD Construct}.

\item Data attribute clauses, see 
\specref{subsec:Data-Sharing Attribute Clauses}.
\end{itemize}







\subsection{\code{parallel}~\code{sections} Construct}
\index{parallel sections@{\code{parallel}~\code{sections}}}
\index{constructs!parallel sections@{\code{parallel}~\code{sections}}}
\index{combined constructs!parallel sections@{\code{parallel}~\code{sections}}}
\label{subsec:parallel sections Construct}
\summary
The \code{parallel}~\code{sections} construct is a shortcut for specifying a \code{parallel} 
construct containing one \code{sections} construct and no other statements.

\syntax
\ccppspecificstart
The syntax of the \code{parallel}~\code{sections} construct is as follows:

\begin{boxedcode}
\#pragma omp parallel sections \plc{[clause[ [},\plc{] clause] ... ] new-line}
    \{
    \plc{[}\#pragma omp section \plc{new-line]}
        \plc{structured-block}
    \plc{[}\#pragma omp section \plc{new-line}
        \plc{structured-block]}
    \plc{...}
    \}
\end{boxedcode}

where \plc{clause} can be any of the clauses accepted by the \code{parallel} or \code{sections} 
directives, except the \code{nowait} clause, with identical meanings and restrictions.
\ccppspecificend

\fortranspecificstart
The syntax of the \code{parallel}~\code{sections} construct is as follows:

\begin{boxedcode}
!\$omp parallel sections \plc{[clause[ [},\plc{] clause] ... ]}
    \plc{[}!\$omp section\plc{]}
        \plc{structured-block}
    \plc{[}!\$omp section 
        \plc{structured-block]}
    \plc{...}
!\$omp end parallel sections
\end{boxedcode}

where \plc{clause} can be any of the clauses accepted by the \code{parallel} or \code{sections} 
directives, with identical meanings and restrictions. 

The last section ends at the \code{end}~\code{parallel}~\code{sections} directive. \code{nowait} cannot be 
specified on an \code{end}~\code{parallel}~\code{sections} directive.
\fortranspecificend

\descr
\ccppspecificstart
The semantics are identical to explicitly specifying a \code{parallel} directive immediately 
followed by a \code{sections} directive.
\ccppspecificend

\fortranspecificstart
The semantics are identical to explicitly specifying a \code{parallel} directive immediately 
followed by a \code{sections} directive, and an \code{end}~\code{sections} directive immediately 
followed by an \code{end}~\code{parallel} directive. 
\fortranspecificend

\restrictions
The restrictions for the \code{parallel} construct and the \code{sections} construct apply.

\crossreferences
\begin{itemize}
\item \code{parallel} construct, see 
\specref{sec:parallel Construct}. 

\item \code{sections} construct, see 
\specref{subsec:sections Construct}.

\item Data attribute clauses, see 
\specref{subsec:Data-Sharing Attribute Clauses}.
\end{itemize}









% Here we need to force the blue floater down lower and force the subsection
% header higher to reduce the space between the blue floater and the header,
% as per Richard:
\vspace{3\baselineskip}
\fortranspecificstart
\vspace{-3\baselineskip}
\subsection{\code{parallel}~\code{workshare} Construct}
\index{parallel workshare@{\code{parallel}~\code{workshare}}}
\index{constructs!parallel workshare@{\code{parallel}~\code{workshare}}}
\index{combined constructs!parallel workshare@{\code{parallel}~\code{workshare}}}
\label{subsec:parallel workshare Construct}
\summary
The \code{parallel}~\code{workshare} construct is a shortcut for specifying a \code{parallel} 
construct containing one \code{workshare} construct and no other statements.

\syntax
The syntax of the \code{parallel}~\code{workshare} construct is as follows:

\begin{boxedcode}
!\$omp parallel workshare \plc{[clause[ [},\plc{] clause] ... ]}
   \plc{structured-block }
!\$omp end parallel workshare
\end{boxedcode}

where \plc{clause} can be any of the clauses accepted by the \code{parallel} directive, with 
identical meanings and restrictions. \code{nowait} may not be specified on an 
\code{end}~\code{parallel}~\code{workshare} directive.

\descr
The semantics are identical to explicitly specifying a \code{parallel} directive immediately 
followed by a \code{workshare} directive, and an \code{end}~\code{workshare} directive immediately 
followed by an \code{end}~\code{parallel} directive. 

\restrictions
The restrictions for the \code{parallel} construct and the \code{workshare} construct apply.

\crossreferences
\begin{itemize}
\item \code{parallel} construct, see 
\specref{sec:parallel Construct}. 

\item \code{workshare} construct, see 
\specref{subsec:workshare Construct}.

\item Data attribute clauses, see 
\specref{subsec:Data-Sharing Attribute Clauses}.
\end{itemize}
\fortranspecificend










\subsection{Parallel Loop SIMD Construct}
\label{subsec:Parallel Loop SIMD Construct}
\index{parallel loop SIMD construct}
\index{constructs!parallel loop SIMD construct}
\index{combined constructs!parallel loop SIMD construct}
\summary
The parallel loop SIMD construct is a shortcut for specifying a \code{parallel} construct 
containing one loop SIMD construct and no other statement.

\begin{samepage}
\syntax
\ccppspecificstart
The syntax of the parallel loop SIMD construct is as follows:

\begin{boxedcode}
\#pragma omp parallel for simd \plc{[clause[ [},\plc{] clause] ... ] new-line}
    \plc{for-loops}
\end{boxedcode}

where \plc{clause} can be any of the clauses accepted by the \code{parallel}
or \code{for}~\code{simd} directives, except the \code{nowait} clause, with 
identical meanings and restrictions.
\ccppspecificend
\end{samepage}

\begin{samepage}
\fortranspecificstart
The syntax of the parallel loop SIMD construct is as follows:

\begin{boxedcode}
!\$omp parallel do simd \plc{[clause[ [},\plc{] clause] ... ]}
    \plc{do-loops}
\plc{[}!\$omp end parallel do simd\plc{]}
\end{boxedcode}
\end{samepage}

where \plc{clause} can be any of the clauses accepted by the \code{parallel}
or \code{do}~\code{simd} directives, with identical meanings and restrictions.

\begin{samepage}
If an \code{end}~\code{parallel}~\code{do}~\code{simd} directive is not specified, an 
\code{end}~\code{parallel}~\code{do}~\code{simd} directive is assumed at the end of the 
\plc{do-loops}. \code{nowait} may not be specified on 
an \code{end}~\code{parallel}~\code{do}~\code{simd} directive.
\fortranspecificend
\end{samepage}

\descr
The semantics of the parallel loop SIMD construct are identical to explicitly specifying 
a \code{parallel} directive immediately followed by a loop SIMD directive. The effect of 
any clause that applies to both constructs is as if it were applied to the loop SIMD 
construct and not to the \code{parallel} construct.

\restrictions
The restrictions for the \code{parallel} construct and the loop SIMD construct apply.

\crossreferences
\begin{itemize}
\item \code{parallel} construct, see 
\specref{sec:parallel Construct}.

\item loop SIMD construct, see 
\specref{subsec:Loop SIMD Construct}.

\item Data attribute clauses, see 
\specref{subsec:Data-Sharing Attribute Clauses}. 
\end{itemize}









\subsection{\code{target}~\code{parallel} Construct}
\label{subsec:target parallel Construct}
\index{target parallel@{\code{target}~\code{parallel}}}
\index{constructs!target parallel@{\code{target}~\code{parallel}}}
\index{combined constructs!target parallel@{\code{target}~\code{parallel}}}
\summary
The \code{target} \code{parallel} construct is a shortcut for specifying a \code{target} 
construct containing a \code{parallel} construct and no other statements.

\syntax
\ccppspecificstart
The syntax of the \code{target} \code{parallel} construct is as follows:

\begin{boxedcode}
\#pragma omp\plc{ }target\plc{ }parallel \plc{[clause[ [},\plc{] clause] ... ] new-line}
    \plc{structured-block}
\end{boxedcode}

where \plc{clause} can be any of the clauses accepted by the \code{target} or
\code{parallel} directives, except for \code{copyin}, with identical meanings and restrictions.
\ccppspecificend

\pagebreak

\begin{samepage}
\fortranspecificstart
The syntax of the \code{target} \code{parallel} construct is as follows:

\begin{boxedcode}
!\$omp target parallel \plc{[clause[ [},\plc{] clause] ... ]}
    \plc{structured-block}
!\$omp end target parallel
\end{boxedcode}
\end{samepage}

where \plc{clause} can be any of the clauses accepted by the \code{target} or 
\code{parallel} directives, except for \code{copyin}, with identical meanings and restrictions.
\fortranspecificend

\descr
The semantics are identical to explicitly specifying a \code{target} directive 
immediately followed by a \code{parallel} directive. 

\restrictions

The restrictions for the \code{target} and \code{parallel} constructs apply except for the following explicit modifications:

\begin{itemize}
\item If any \code{if} clause on the directive includes a
      \plc{directive-name-modifier} then all \code{if} clauses 
      on the directive must include a \plc{directive-name-modifier}.

\item At most one \code{if} clause without a 
      \plc{directive-name-modifier} can appear on the directive.

\item At most one \code{if} clause with the \code{parallel} 
      \plc{directive-name-modifier} can appear on the directive.


\item At most one \code{if} clause with the \code{target} 
      \plc{directive-name-modifier} can appear on the directive.
\end{itemize}

\crossreferences
\begin{itemize}
\item \code{parallel} construct, see 
\specref{sec:parallel Construct}.

\item \code{target} construct, see 
\specref{subsec:target Construct}.

\item \code{if} Clause, see \specref{sec:if Clause}.

\item Data attribute clauses, see 
\specref{subsec:Data-Sharing Attribute Clauses}.


%% \item Multi-if Clause, see \specref{subsec:Multi-if Clause}.
\end{itemize}









%Similar to Distribute Parallel Loop Construct

\subsection{Target Parallel Loop Construct}
\label{subsec:Target Parallel Loop Construct}
\index{target parallel loop construct}
\index{constructs!target parallel loop}
\index{constructs!target parallel for@{\code{target}~\code{parallel}~\code{for}}}
\index{constructs!target parallel do@{\code{target}~\code{parallel}~\code{do}}}
\index{combined constructs!target parallel loop}
\summary
The target parallel loop construct is a shortcut for specifying a \code{target} 
construct containing a parallel loop construct and no other statements.

\syntax
\ccppspecificstart
The syntax of the target parallel loop construct is as follows:

\begin{boxedcode}
\#pragma omp\plc{ }target\plc{ }parallel\plc{ }for \plc{[clause[ [},\plc{] clause] ... ] new-line}
    \plc{for-loops}
\end{boxedcode}

where \plc{clause} can be any of the clauses accepted by the \code{target} or
\code{parallel}~\code{for} directives, except for \code{copyin}, with identical meanings and restrictions.
\ccppspecificend

\needspace{6\baselineskip}\begin{samepage}
\fortranspecificstart
The syntax of the target parallel loop construct is as follows:

\begin{boxedcode}
!\$omp target parallel do \plc{[clause[ [},\plc{] clause] ... ]}
    \plc{do-loops}
\plc{[}!\$omp end target parallel do\plc{]}
\end{boxedcode}
\end{samepage}

where \plc{clause} can be any of the clauses accepted by the \code{target} or 
\code{parallel}~\code{do} directives, except for \code{copyin}, with identical meanings and restrictions.

If an \code{end}~\code{target}~\code{parallel}~\code{do} directive is not specified, an 
\code{end}~\code{target}~\code{parallel}~\code{do} directive is assumed at the end of 
the \plc{do-loops}.
\fortranspecificend

\descr
The semantics are identical to explicitly specifying a \code{target} directive
immediately followed by a parallel loop directive.


\restrictions
The restrictions for the \code{target} and parallel loop constructs apply except for the following explicit modifications:

\begin{itemize}
\item If any \code{if} clause on the directive includes a
      \plc{directive-name-modifier} then all \code{if} clauses 
      on the directive must include a \plc{directive-name-modifier}.

\item At most one \code{if} clause without a 
      \plc{directive-name-modifier} can appear on the directive.

\item At most one \code{if} clause with the \code{parallel} 
      \plc{directive-name-modifier} can appear on the directive.


\item At most one \code{if} clause with the \code{target} 
      \plc{directive-name-modifier} can appear on the directive.
\end{itemize}

\crossreferences
\begin{itemize}
\item \code{target} construct, see
\specref{subsec:target Construct}.

\item Parallel loop construct, see
\specref{subsec:Parallel Loop Construct}.

\item \code{if} Clause, see \specref{sec:if Clause}.

\item Data attribute clauses, see
\specref{subsec:Data-Sharing Attribute Clauses}.

%% \item Multi-if Clause, see \specref{subsec:Multi-if Clause}.
\end{itemize}









% Similar to Distribute Parallel Loop SIMD Construct

\subsection{Target Parallel Loop SIMD Construct}
\label{subsec:Target Parallel Loop SIMD Construct}
\index{target parallel loop SIMD construct}
\index{constructs!target parallel loop SIMD}
\index{constructs!target parallel for simd@{\code{target}~\code{parallel}~\code{for}~\code{simd}}}
\index{constructs!target parallel do simd@{\code{target}~\code{parallel}~\code{do}~\code{simd}}}
\index{combined constructs!target parallel loop SIMD}
\summary
The target parallel loop SIMD construct is a shortcut for specifying a \code{target} 
construct containing a parallel loop SIMD construct and no other statements.

\syntax
\ccppspecificstart
The syntax of the target parallel loop SIMD construct is as follows:

\begin{boxedcode}
\#pragma omp\plc{ }target\plc{ }parallel\plc{ }for\plc{ }simd \plc{[clause[
[},\plc{] clause] ... ] new-line}
    \plc{for-loops}
\end{boxedcode}

where \plc{clause} can be any of the clauses accepted by the \code{target} or
\code{parallel}~\code{for}~\code{simd} directives, except for \code{copyin}, with identical meanings and restrictions.
\ccppspecificend

\needspace{6\baselineskip}\begin{samepage}
\fortranspecificstart
The syntax of the target parallel loop SIMD construct is as follows:

\begin{boxedcode}
!\$omp target parallel do simd \plc{[clause[ [},\plc{] clause] ... ]}
    \plc{do-loops}
\plc{[}!\$omp end target parallel do simd\plc{]}
\end{boxedcode}
\end{samepage}

where \plc{clause} can be any of the clauses accepted by the \code{target} or 
\code{parallel}~\code{do}~\code{simd} directives, except for \code{copyin}, with identical meanings and restrictions.

If an \code{end}~\code{target}~\code{parallel}~\code{do}~\code{simd} directive is not specified, an 
\code{end}~\code{target}~\code{parallel}~\code{do}~\code{simd} directive is assumed at the end of 
the \plc{do-loops}.
\fortranspecificend

\descr
The semantics are identical to explicitly specifying a \code{target} directive
immediately followed by a parallel loop SIMD directive.


\restrictions
The restrictions for the \code{target} and parallel loop SIMD constructs apply except for the following explicit modifications:

\begin{itemize}
\item If any \code{if} clause on the directive includes a
      \plc{directive-name-modifier} then all \code{if} clauses 
      on the directive must include a \plc{directive-name-modifier}.

\item At most one \code{if} clause without a 
      \plc{directive-name-modifier} can appear on the directive.

\item At most one \code{if} clause with the \code{parallel} 
      \plc{directive-name-modifier} can appear on the directive.


\item At most one \code{if} clause with the \code{target} 
      \plc{directive-name-modifier} can appear on the directive.
\end{itemize}

\crossreferences
\begin{itemize}
\item \code{target} construct, see
\specref{subsec:target Construct}.

\item Parallel loop SIMD construct, see
\specref{subsec:Parallel Loop SIMD Construct}.

\item \code{if} Clause, see \specref{sec:if Clause}.

\item Data attribute clauses, see
\specref{subsec:Data-Sharing Attribute Clauses}.

%% \item Multi-if Clause, see \specref{subsec:Multi-if Clause}.
\end{itemize}










\subsection{\code{target}~\code{simd} Construct}
\index{target simd@{\code{target}~\code{simd}}}
\index{constructs!target simd@{\code{target}~\code{simd}}}
\index{combined constructs!target simd@{\code{target}~\code{simd}}}
\label{subsec:target simd Construct}

\summary
The \code{target} \code{simd} construct is a shortcut for specifying a \code{target} 
construct containing a \code{simd} construct and no other statements.

\syntax
\ccppspecificstart
The syntax of the \code{target} \code{simd} construct is as follows:

\begin{boxedcode}
\#pragma omp\plc{ }target\plc{ }simd \plc{[clause[ [},\plc{] clause] ... ] new-line}
    \plc{for-loops}
\end{boxedcode}

where \plc{clause} can be any of the clauses accepted by the \code{target} or
\code{simd} directives with identical meanings and restrictions.

\ccppspecificend

\needspace{6\baselineskip}\begin{samepage}
\fortranspecificstart
The syntax of the \code{target} \code{simd} construct is as follows:

\begin{boxedcode}
!\$omp target simd \plc{[clause[ [},\plc{] clause] ... ]}
    \plc{do-loops}
\plc{[}!\$omp end target simd\plc{]}
\end{boxedcode}
\end{samepage}

where \plc{clause} can be any of the clauses accepted by the \code{target} or 
\code{simd} directives with identical meanings and restrictions.

If an \code{end}~\code{target}~\code{simd} directive is not specified, an 
\code{end}~\code{target}~\code{simd} directive is assumed at the end of 
the \plc{do-loops}.
\fortranspecificend

\descr
The semantics are identical to explicitly specifying a \code{target} directive 
immediately followed by a \code{simd} directive. 

\restrictions

The restrictions for the \code{target} and \code{simd} constructs apply.

\crossreferences
\begin{itemize}
\item \code{simd} construct, see 
\specref{subsec:simd Construct}.

\item \code{target} construct, see 
\specref{subsec:target Construct}.

\item Data attribute clauses, see 
\specref{subsec:Data-Sharing Attribute Clauses}.
\end{itemize}









\subsection{\code{target}~\code{teams} Construct}
\label{subsec:target teams Construct}
\index{target teams@{\code{target}~\code{teams}}}
\index{constructs!target teams@{\code{target}~\code{teams}}}
\index{combined constructs!target teams@{\code{target}~\code{teams}}}
\summary
The \code{target}~\code{teams} construct is a shortcut for specifying a \code{target} construct 
containing a \code{teams} construct and no other statements.

\syntax
\ccppspecificstart
The syntax of the \code{target}~\code{teams} construct is as follows:

\begin{boxedcode}
\#pragma omp target teams \plc{[clause[ [},\plc{] clause] ... ] new-line}
   \plc{structured-block}
\end{boxedcode}

where \plc{clause} can be any of the clauses accepted by the \code{target} or \code{teams} directives 
with identical meanings and restrictions.
\ccppspecificend

\fortranspecificstart
The syntax of the \code{target}~\code{teams} construct is as follows:

\begin{boxedcode}
!\$omp target teams \plc{[clause[ [},\plc{] clause] ... ]}
    \plc{structured-block}
!\$omp end target teams
\end{boxedcode}

where \plc{clause} can be any of the clauses accepted by the \code{target} or \code{teams} directives 
with identical meanings and restrictions.
\fortranspecificend

\descr

The semantics are identical to explicitly specifying a \code{target} directive 
immediately followed by a \code{teams} directive. 

\restrictions
The restrictions for the \code{target} and \code{teams} constructs apply.

\crossreferences
\begin{itemize}
\item \code{target} construct, see 
\specref{subsec:target Construct}.

\item \code{teams} construct, see 
\specref{subsec:teams Construct}.

\item Data attribute clauses, see 
\specref{subsec:Data-Sharing Attribute Clauses}.
\end{itemize}









\subsection{\code{teams}~\code{distribute} Construct}
\index{teams distribute@{\code{teams}~\code{distribute}}}
\index{constructs!teams distribute@{\code{teams}~\code{distribute}}}
\index{combined constructs!teams distribute@{\code{teams}~\code{distribute}}}
\label{subsec:teams distribute Construct}
\summary
The \code{teams}~\code{distribute} construct is a shortcut for specifying a \code{teams} construct 
containing a \code{distribute} construct and no other statements.

\syntax
\ccppspecificstart
The syntax of the \code{teams}~\code{distribute} construct is as follows:

\begin{boxedcode}
\#pragma omp teams distribute \plc{[clause[ [},\plc{] clause] ... ] new-line}
    \plc{for-loops}
\end{boxedcode}

where \plc{clause} can be any of the clauses accepted by the \code{teams} or \code{distribute} 
directives with identical meanings and restrictions.
\ccppspecificend

\fortranspecificstart
The syntax of the \code{teams}~\code{distribute} construct is as follows:

\begin{boxedcode}
!\$omp teams distribute \plc{[clause[ [},\plc{] clause] ... ]}
    \plc{do-loops}
\plc{[}!\$omp end teams distribute\plc{]}
\end{boxedcode}

where \plc{clause} can be any of the clauses accepted by the \code{teams} or \code{distribute} 
directives with identical meanings and restrictions.

If an \code{end}~\code{teams}~\code{distribute} directive is not specified, an 
\code{end}~\code{teams}~\code{distribute} directive is assumed at the end of the \plc{do-loops}.
\fortranspecificend

\descr
The semantics are identical to explicitly specifying a \code{teams} directive immediately 
followed by a \code{distribute} directive. The effect of any clause that applies to both constructs is as if it were applied to both constructs separately.


\restrictions
The restrictions for the \code{teams} and \code{distribute} constructs apply.

\crossreferences
\begin{itemize}
\item \code{teams} construct, see 
\specref{subsec:teams Construct}.

\item \code{distribute} construct, see 
\specref{subsec:distribute Construct}.

\item Data attribute clauses, see 
\specref{subsec:Data-Sharing Attribute Clauses}.
\end{itemize}












\subsection{\code{teams}~\code{distribute}~\code{simd} Construct}
\index{teams distribute simd@{\code{teams}~\code{distribute}~\code{simd}}}
\index{constructs!teams distribute simd@{\code{teams}~\code{distribute}~\code{simd}}}
\index{combined constructs!teams distribute simd@{\code{teams}~\code{distribute}~\code{simd}}}
\label{subsec:teams distribute simd Construct}
\summary
The \code{teams}~\code{distribute}~\code{simd} construct is a shortcut for specifying a \code{teams} construct 
containing a \code{distribute}~\code{simd} construct and no other statements.

\syntax
\ccppspecificstart
The syntax of the \code{teams}~\code{distribute}~\code{simd} construct is as follows:

\begin{boxedcode}
\#pragma omp teams distribute simd \plc{[clause[ [},\plc{] clause] ... ] new-line}
    \plc{for-loops}
\end{boxedcode}

where \plc{clause} can be any of the clauses accepted by the \code{teams} or \code{distribute}~\code{simd} 
directives with identical meanings and restrictions.
\ccppspecificend

\fortranspecificstart
The syntax of the \code{teams}~\code{distribute}~\code{simd} construct is as follows:

\begin{boxedcode}
!\$omp teams distribute simd \plc{[clause[ [},\plc{] clause] ... ]}
    \plc{do-loops}
\plc{[}!\$omp end teams distribute simd\plc{]}
\end{boxedcode}

where \plc{clause} can be any of the clauses accepted by the \code{teams} or \code{distribute}~\code{simd} 
directives with identical meanings and restrictions.

If an \code{end}~\code{teams}~\code{distribute}~\code{simd} directive is 
not specified, an \code{end}~\code{teams}~\code{distribute}~\code{simd} 
directive is assumed at the end of the \plc{do-loops}.
\fortranspecificend

\descr
The semantics are identical to explicitly specifying a \code{teams} directive immediately 
followed by a \code{distribute}~\code{simd} directive. The effect of any clause that applies to both constructs is as if it were applied to both constructs separately.


\restrictions
The restrictions for the \code{teams} and \code{distribute}~\code{simd} constructs apply.

\crossreferences
\begin{itemize}
\item \code{teams} construct, see 
\specref{subsec:teams Construct}.

\item \code{distribute}~\code{simd} construct, see 
\specref{subsec:distribute simd Construct}.

\item Data attribute clauses, see 
\specref{subsec:Data-Sharing Attribute Clauses}.
\end{itemize}











\subsection{\code{target}~\code{teams}~\code{distribute} Construct}
\index{target teams distribute@{\code{target}~\code{teams}~\code{distribute}}}
\index{constructs!target teams distribute@{\code{target}~\code{teams}~\code{distribute}}}
\index{combined constructs!target teams distribute@{\code{target}~\code{teams}~\code{distribute}}}
\label{subsec:target teams distribute construct}
\summary
The \code{target}~\code{teams}~\code{distribute} construct is a shortcut for specifying a \code{target} construct 
containing a \code{teams}~\code{distribute} construct and no other statements.

\syntax
\ccppspecificstart
The syntax of the \code{target}~\code{teams}~\code{distribute} construct is as follows:

\begin{boxedcode}
\#pragma omp target teams distribute \plc{[clause[ [},\plc{] clause] ... ] new-line}
   \plc{for-loops}
\end{boxedcode}

where \plc{clause} can be any of the clauses accepted by the \code{target} or \code{teams}~\code{distribute} directives 
with identical meanings and restrictions.
\ccppspecificend

\fortranspecificstart
The syntax of the \code{target}~\code{teams}~\code{distribute} construct is as follows:

\begin{boxedcode}
!\$omp target teams distribute \plc{[clause[ [},\plc{] clause] ... ]}
    \plc{do-loops}
\plc{[}!\$omp end target teams distribute\plc{]}
\end{boxedcode}

where \plc{clause} can be any of the clauses accepted by the \code{target} or \code{teams}~\code{distribute} directives 
with identical meanings and restrictions.

If an \code{end}~\code{target}~\code{teams}~\code{distribute} directive is not specified, an 
\code{end}~\code{target}~\code{teams}~\code{distribute} directive is assumed at the end of the \plc{do-loops}.
\fortranspecificend

\descr
The semantics are identical to explicitly specifying a \code{target} directive immediately 
followed by a \code{teams}~\code{distribute} directive.

\restrictions
The restrictions for the \code{target} and \code{teams}~\code{distribute} constructs apply.

\crossreferences
\begin{itemize}
\item \code{target} construct, see 
\specref{subsec:target data Construct}.

\item \code{teams}~\code{distribute} construct, see 
\specref{subsec:teams distribute Construct}.

\item Data attribute clauses, see 
\specref{subsec:Data-Sharing Attribute Clauses}.
\end{itemize}










\subsection{\code{target}~\code{teams}~\code{distribute}~\code{simd} Construct}
\index{target teams distribute simd@{\code{target}~\code{teams}~\code{distribute}~\code{simd}}}
\index{constructs!target teams distribute simd@{\code{target}~\code{teams}~\code{distribute}~\code{simd}}}
\index{combined constructs!target teams distribute simd@{\code{target}~\code{teams}~\code{distribute}~\code{simd}}}
\label{subsec:target teams distribute simd construct}
\summary
The \code{target}~\code{teams}~\code{distribute}~\code{simd} construct is a shortcut for specifying a \code{target} construct 
containing a \code{teams}~\code{distribute}~\code{simd} construct and no other statements.

\syntax
\ccppspecificstart
The syntax of the \code{target}~\code{teams}~\code{distribute}~\code{simd} construct is as follows:

\begin{boxedcode}
\#pragma omp target teams distribute simd \plc{[clause[ [},\plc{] clause] ...  ] new-line}
   \plc{for-loops}
\end{boxedcode}

where \plc{clause} can be any of the clauses accepted by the \code{target} or 
\code{teams}~\code{distribute}~\code{simd} directives with identical meanings and restrictions.
\ccppspecificend

\fortranspecificstart
The syntax of the \code{target}~\code{teams}~\code{distribute}~\code{simd} construct is as follows:

\begin{boxedcode}
!\$omp target teams distribute simd \plc{[clause[ [},\plc{] clause] ... ]}
    \plc{do-loops}
\plc{[}!\$omp end target teams distribute simd\plc{]}
\end{boxedcode}

where \plc{clause} can be any of the clauses accepted by the \code{target} or 
\code{teams}~\code{distribute}~\code{simd} directives with identical meanings and restrictions.

If an \code{end}~\code{target}~\code{teams}~\code{distribute}~\code{simd} directive is not specified, an 
\code{end}~\code{target}~\code{teams}~\code{distribute}~\code{simd} directive is assumed at the end of the \plc{do-loops}.
\fortranspecificend

\descr
The semantics are identical to explicitly specifying a \code{target} directive immediately 
followed by a \code{teams}~\code{distribute}~\code{simd} directive.

\restrictions
The restrictions for the \code{target} and \code{teams}~\code{distribute}~\code{simd} constructs apply.

\crossreferences
\begin{itemize}
\item \code{target} construct, see 
\specref{subsec:target data Construct}.

\item \code{teams}~\code{distribute}~\code{simd} construct, see 
\specref{subsec:teams distribute simd Construct}.

\item Data attribute clauses, see 
\specref{subsec:Data-Sharing Attribute Clauses}.
\end{itemize}











\subsection{Teams Distribute Parallel Loop Construct}
\label{subsec:Teams Distribute Parallel Loop Construct}
\index{teams distribute parallel loop construct}
\index{constructs!teams distribute parallel loop construct}
\index{combined constructs!teams distribute parallel loop construct}
\summary
The teams distribute parallel loop construct is a shortcut for specifying a \code{teams} 
construct containing a distribute parallel loop construct and no other statements.

\syntax
\ccppspecificstart
The syntax of the teams distribute parallel loop construct is as follows:

\begin{boxedcode}[fontsize=\small]
\#pragma omp teams distribute parallel for \plc{[clause[ [},\plc{] clause] ...  ] new-line}
    \plc{for-loops}
\end{boxedcode}

where \plc{clause} can be any of the clauses accepted by the \code{teams} or 
\code{distribute}~\code{parallel}~\code{for} directives with identical meanings and restrictions.
\ccppspecificend

\fortranspecificstart
The syntax of the teams distribute parallel loop construct is as follows:

\begin{boxedcode}
!\$omp teams distribute parallel do \plc{[clause[ [},\plc{] clause] ... ]}
   \plc{do-loops}
\plc{[} !\$omp end teams distribute parallel do \plc{]}
\end{boxedcode}

where \plc{clause} can be any of the clauses accepted by the \code{teams} or 
\code{distribute}~\code{parallel}~\code{do} directives with identical meanings and restrictions.

If an \code{end}~\code{teams}~\code{distribute}~\code{parallel}~\code{do} directive is not specified, an 
\code{end}~\code{teams}~\code{distribute}~\code{parallel}~\code{do} directive is assumed at the end of the \plc{do-loops}.
\fortranspecificend

\descr
The semantics are identical to explicitly specifying a \code{teams} directive immediately 
followed by a distribute parallel loop directive. The effect of any clause that applies to 
both constructs is as if it were applied to both constructs separately.

\restrictions
The restrictions for the \code{teams} and distribute parallel loop constructs apply.

\crossreferences
\begin{itemize}
\item \code{teams} construct, see 
\specref{subsec:teams Construct}.

\item Distribute parallel loop construct, see 
\specref{subsec:Distribute Parallel Loop Construct}.

\item Data attribute clauses, see 
\specref{subsec:Data-Sharing Attribute Clauses}.
\end{itemize}









\subsection{Target Teams Distribute Parallel Loop Construct}
\label{subsec:Target Teams Distribute Parallel Loop Construct}
\index{target teams distribute parallel loop construct}
\index{constructs!target teams distribute parallel loop construct}
\index{combined constructs!target teams distribute parallel loop construct}
\summary
The target teams distribute parallel loop construct is a shortcut for specifying a \code{target} 
construct containing a teams distribute parallel loop construct and no other statements.

\syntax
\ccppspecificstart
The syntax of the target teams distribute parallel loop construct is as follows:

\begin{boxedcode}[fontsize=\small]
\#pragma omp\plc{ }target\plc{ }teams\plc{ }distribute\plc{ }parallel\plc{ }for \plc{[clause[ [},\plc{] clause] ... ] new-line}
    \plc{for-loops}
\end{boxedcode}

where \plc{clause} can be any of the clauses accepted by the \code{target} or
\code{teams}~\code{distribute}~\code{parallel}~\code{for} directives with identical 
meanings and restrictions.
\ccppspecificend

\needspace{6\baselineskip}\begin{samepage}
\fortranspecificstart
The syntax of the target teams distribute parallel loop construct is as follows:

\begin{boxedcode}
!\$omp target teams distribute parallel do \plc{[clause[ [},\plc{] clause] ... ]}
    \plc{do-loops}
\plc{[}!\$omp end target teams distribute parallel do\plc{]}
\end{boxedcode}
\end{samepage}

where \plc{clause} can be any of the clauses accepted by the \code{target} or 
\code{teams}~\code{distribute}~\code{parallel}~\code{do} directives with 
identical meanings and restrictions.

If an \code{end}~\code{target}~\code{teams}~\code{distribute}~\code{parallel}~\code{do} directive is not specified, an 
\code{end}~\code{target}~\code{teams}~\code{distribute}~\code{parallel}~\code{do} 
directive is assumed at the end of the \plc{do-loops}.
\fortranspecificend

\descr
The semantics are identical to explicitly specifying a \code{target} 
directive immediately followed by a teams distribute parallel loop directive.


\restrictions
The restrictions for the \code{target} and teams distribute parallel 
loop constructs apply except for the following explicit modifications:

\begin{itemize}
\item If any \code{if} clause on the directive includes a
      \plc{directive-name-modifier} then all \code{if} clauses 
      on the directive must include a \plc{directive-name-modifier}.

\item At most one \code{if} clause without a 
      \plc{directive-name-modifier} can appear on the directive.

\item At most one \code{if} clause with the \code{parallel} 
      \plc{directive-name-modifier} can appear on the directive.


\item At most one \code{if} clause with the \code{target} 
      \plc{directive-name-modifier} can appear on the directive.
\end{itemize}

\crossreferences
\begin{itemize}
\item \code{target} construct, see \specref{subsec:target Construct}.

\item Teams distribute parallel loop construct, see
      \specref{subsec:Teams Distribute Parallel Loop Construct}.

\item \code{if} Clause, see \specref{sec:if Clause}.

\item Data attribute clauses, see 
      \specref{subsec:Data-Sharing Attribute Clauses}.
\end{itemize}










\subsection{Teams Distribute Parallel Loop SIMD Construct}
\label{subsec:Teams Distribute Parallel Loop SIMD Construct}
\index{teams distribute parallel loop SIMD construct}
\index{constructs!teams distribute parallel loop SIMD construct}
\index{combined constructs!teams distribute parallel loop SIMD construct}
\summary
The teams distribute parallel loop SIMD construct is a shortcut for specifying a \code{teams} 
construct containing a distribute parallel loop SIMD construct and no other statements.

\syntax
\ccppspecificstart
The syntax of the teams distribute parallel loop construct is as follows:

\begin{boxedcode}[fontsize=\small]
\#pragma omp teams distribute parallel for simd \plc{[clause[ [},\plc{] clause] ... ] new-line}
    \plc{for-loops}
\end{boxedcode}

where \plc{clause} can be any of the clauses accepted by the \code{teams} or 
\code{distribute}~\code{parallel}~\code{for}~\code{simd}
directives with identical meanings and restrictions.
\ccppspecificend

\fortranspecificstart
The syntax of the teams distribute parallel loop construct is as follows:

\begin{boxedcode}
!\$omp teams distribute parallel do simd \plc{[clause[ [},\plc{] clause] ... ]}
    \plc{do-loops}
\plc{[}!\$omp end teams distribute parallel do simd\plc{]}
\end{boxedcode}

where \plc{clause} can be any of the clauses accepted by the \code{teams} or 
\code{distribute}~\code{parallel}~\code{do}~\code{simd}
directives with identical meanings and restrictions.

If an \code{end}~\code{teams}~\code{distribute}~\code{parallel}~\code{do}~\code{simd} directive is not specified, an 
\code{end}~\code{teams}~\code{distribute}~\code{parallel}~\code{do}~\code{simd} directive is assumed at the end of 
the \plc{do-loops}.
\fortranspecificend

\descr
The semantics are identical to explicitly specifying a \code{teams} directive immediately 
followed by a distribute parallel loop SIMD directive. The effect of any clause that 
applies to both constructs is as if it were applied to both constructs separately.

\restrictions
The restrictions for the \code{teams} and distribute parallel loop
SIMD constructs apply.

\crossreferences
\begin{itemize}
\item \code{teams} construct, see 
\specref{subsec:teams Construct}.

\item Distribute parallel loop SIMD construct, see 
\specref{subsec:Distribute Parallel Loop SIMD Construct}.

\item Data attribute clauses, see 
\specref{subsec:Data-Sharing Attribute Clauses}.
\end{itemize}










\subsection{Target Teams Distribute Parallel Loop SIMD Construct}
\label{subsec:Target Teams Distribute Parallel Loop SIMD Construct}
\index{target teams distribute parallel loop SIMD construct}
\index{constructs!target teams distribute parallel loop SIMD construct}
\index{combined constructs!target teams distribute parallel loop SIMD construct}
\summary
The target teams distribute parallel loop SIMD construct is a shortcut for specifying a \code{target} 
construct containing a teams distribute parallel loop SIMD construct and no other statements.

\syntax
\ccppspecificstart
The syntax of the target teams distribute parallel loop SIMD construct is as follows:

\begin{boxedcode}
\#pragma omp target teams distribute parallel for simd \plc{\textbackslash}
            \plc{[clause[ [},\plc{] clause] ... ] new-line}
    \plc{for-loops}
\end{boxedcode}

where \plc{clause} can be any of the clauses accepted by the \code{target} or 
\code{teams}~\code{distribute}~\code{parallel}~\code{for}~\code{simd}
directives with identical meanings and restrictions.
\ccppspecificend

\pagebreak

\fortranspecificstart
The syntax of the target teams distribute parallel loop SIMD construct is as follows:

\begin{boxedcode}
!\$omp target teams distribute parallel do simd \plc{[clause[ [},\plc{] clause] ... ]}
    \plc{do-loops}
\plc{[}!\$omp end target teams distribute parallel do simd\plc{]}
\end{boxedcode}

where \plc{clause} can be any of the clauses accepted by the 
\code{target} or \code{teams}~\code{distribute}~\code{parallel}~\code{do}~\code{simd}
directives with identical meanings and restrictions.

If an \code{end}~\code{target}~\code{teams}~\code{distribute}~\code{parallel}~\code{do}~\code{simd} 
directive is not specified, an 
\code{end}~\code{target}~\code{teams}~\code{distribute}~\code{parallel}~\code{do}~\code{simd} 
directive is assumed at the end of the \plc{do-loops}.
\fortranspecificend

\descr
The semantics are identical to explicitly specifying a \code{target} 
directive immediately followed by a teams distribute parallel loop 
SIMD directive.

\restrictions
The restrictions for the \code{target} and teams distribute parallel 
loop SIMD constructs apply except for the following explicit modifications:

\begin{itemize}
\item If any \code{if} clause on the directive includes a
      \plc{directive-name-modifier} then all \code{if} clauses 
      on the directive must include a \plc{directive-name-modifier}.

\item At most one \code{if} clause without a 
      \plc{directive-name-modifier} can appear on the directive.

\item At most one \code{if} clause with the \code{parallel}
      \plc{directive-name-modifier} can appear on the directive.

\item At most one \code{if} clause with the \code{target}
      \plc{directive-name-modifier} can appear on the directive.
\end{itemize}

\crossreferences
\begin{itemize}
\item \code{target} construct, see \specref{subsec:target Construct}.

\item Teams distribute parallel loop SIMD construct, see 
      \specref{subsec:Teams Distribute Parallel Loop SIMD Construct}.

\item \code{if} Clause, see \specref{sec:if Clause}.

\item Data attribute clauses, see 
      \specref{subsec:Data-Sharing Attribute Clauses}.
\end{itemize}




\section{\code{if} Clause}
\index{clauses!if Clause@{\code{if} Clause}}
\index{if Clause@{\code{if} Clause}}
\label{sec:if Clause}

\summary
The semantics of an \code{if} clause are described in the section on the
construct to which it applies. 
The \code{if} clause \plc{directive-name-modifier} names the associated
construct to which an expression applies, and is particularly useful for 
composite and combined constructs.

\syntax
\ccppspecificstart

The syntax of the \code{if} clause is as follows:

\begin{boxedcode}
if(\plc{[ directive-name-modifier }:\plc{] scalar-expression}) 
\end{boxedcode}
\ccppspecificend

\fortranspecificstart

The syntax of the \code{if} clause is as follows:

\begin{boxedcode}
if(\plc{[ directive-name-modifier }:\plc{] scalar-logical-expression}) 
\end{boxedcode}
\fortranspecificend

\descr
The effect of the \code{if} clause depends on the construct to 
which it is applied. For combined or composite constructs, the
\code{if} clause only applies to the semantics of the construct 
named in the \plc{directive-name-modifier} if one is specified. 
If no  \plc{directive-name-modifier} is specified for a combined 
or composite construct then the \code{if} clause applies to all 
constructs to which an \code{if} clause can apply.




\section{Master and Synchronization Constructs and Clauses}
\label{sec:Master and Synchronization Constructs and Clauses}
\index{master and synchronization constructs and clauses}
\index{synchronization constructs}
OpenMP provides the following synchronization constructs:
\begin{itemize}
\item the \code{master} construct;

\item the \code{critical} construct;

\item the \code{barrier} construct;

\item the \code{taskwait} construct;

\item the \code{taskgroup} construct;

\item the \code{atomic} construct;

\item the \code{flush} construct;

\item the \code{ordered} construct.
\end{itemize}










\subsection{\code{master} Construct}
\index{maste@{\code{master}}}
\index{constructs!master@{\code{master}}}
\label{subsec:master Construct}
\summary
The \code{master} construct specifies a structured block that is executed by the master thread 
of the team.

\syntax
\ccppspecificstart
The syntax of the \code{master} construct is as follows:

\begin{boxedcode}
\#pragma omp master \plc{new-line}
   \plc{structured-block}
\end{boxedcode}
\ccppspecificend

\fortranspecificstart
The syntax of the \code{master} construct is as follows:

\begin{boxedcode}
!\$omp master
   \plc{structured-block}
!\$omp end master
\end{boxedcode}
\fortranspecificend

\binding
The binding thread set for a \code{master} region is the current team. A \code{master} region 
binds to the innermost enclosing \code{parallel} region. Only the master thread of the team 
executing the binding \code{parallel} region participates in the execution of the structured 
block of the \code{master} region.

\descr
Other threads in the team do not execute the associated structured block. There is no 
implied barrier either on entry to, or exit from, the \code{master} construct.

\events

The \plc{master-begin} event occurs in the thread encountering the \code{master}
construct on entry to the master region, if it is the master thread of the team.

The \plc{master-end} event occurs in the thread encountering the \code{master}
construct on exit of the master region, if it is the master thread of the team.

\tools

A thread dispatches a registered \code{ompt\_callback\_master}
callback for each occurrence of a \plc{master-begin} and a
\plc{master-end} event in that thread.  

The callback occurs  in the context of the task executed by the master thread.
This callback has the type signature
\code{ompt\_callback\_master\_t}. The callback receives
\code{ompt\_scope\_begin} or \code{ompt\_scope\_end} 
as its \plc{endpoint} argument, as appropriate.

\restrictions
\cppspecificstart
\begin{itemize}
\item A throw executed inside a \code{master} region must cause execution to resume within the 
same \code{master} region, and the same thread that threw the exception must catch it
\end{itemize}
\cppspecificend

\crossreferences
\begin{itemize}

\item \code{ompt\_scope\_begin} and \code{ompt\_scope\_end}, see 
\specref{sec:ompt_scope_endpoint_t}.

\item \code{ompt\_callback\_master\_t}, see
\specref{sec:ompt_callback_master_t}.


\end{itemize}










\subsection{\code{critical} Construct}
\index{critical@{\code{critical}}}
\index{constructs!critical@{\code{critical}}}
\label{subsec:critical Construct}
\summary
The \code{critical} construct restricts execution of the associated structured block to a 
single thread at a time.

\syntax
\ccppspecificstart
The syntax of the \code{critical} construct is as follows:

\begin{boxedcode}
\#pragma omp critical \plc{[}(\plc{name}) \plc{[}hint(\plc{hint-expression})\plc{] ] new-line}
    \plc{structured-block}
\end{boxedcode}

where \plc{hint-expression} is an integer constant expression that
evaluates to a valid lock hint (as described 
in~\specref{subsec:omp_init_lock_with_hint and omp_init_nest_lock_with_hint}).
\ccppspecificend

\fortranspecificstart
The syntax of the \code{critical} construct is as follows:

\begin{boxedcode}
!\$omp critical \plc{[}(\plc{name}) \plc{[}hint(\plc{hint-expression})\plc{] ]}
    \plc{structured-block}
!\$omp end critical \plc{[}(\plc{name})\plc{]}  
\end{boxedcode}

where \plc{hint-expression} is a constant expression that evaluates to
a scalar value with kind \code{omp\_lock\_hint\_kind} and  a value
that is a valid lock hint (as described 
in~\specref{subsec:omp_init_lock_with_hint and omp_init_nest_lock_with_hint}).
\fortranspecificend

\binding
The binding thread set for a \code{critical} region is all threads in the contention group. 
The region is executed as if only a single thread at a time among all threads in the 
contention group is entering the region for execution, without regard to the team(s) to which the threads belong. 

\descr
An optional \plc{name} may be used to identify the \code{critical} construct. All \code{critical} 
constructs without a name are considered to have the same unspecified name. 

\ccppspecificstart
Identifiers used to identify a \code{critical} construct have external linkage and are in a 
name space that is separate from the name spaces used by labels, tags, members, and 
ordinary identifiers.
\ccppspecificend

\fortranspecificstart
The names of \code{critical} constructs are global entities of the program. If a name 
conflicts with any other entity, the behavior of the program is unspecified.
\fortranspecificend

The threads of a contention group execute the \code{critical} region as if only one thread of the contention group is executing the \code{critical} region at a time.
The \code{critical} construct enforces these execution semantics with respect to all \code{critical} constructs with the same name in all 
threads in the contention group, not just those threads in the current team.

The presence of a \code{hint} clause does not affect the isolation
guarantees provided by the \code{critical} construct. If no
\code{hint} clause is specified, the effect is as if \code{hint(omp\_lock\_hint\_none)}
had been specified.

% Replaced by the above paragraph
% If the \code{hint} clause is present, the effect of the \code{critical} 
% construct is implementation defined, but retains the mutual exclusion 
% semantics. If no
% \code{hint} clause is specified, the effect is as if \code{hint(none)}
% has been specified.

\def\omptMutex#1#2{
\events
The \plc{#1-acquire} event occurs in the thread encountering the
\code{#1} construct on entry to the #1 region before 
initiating synchronization for the region.

The \plc{#1-acquired} event occurs in the thread encountering the
\code{#1} construct after entering the region, but before executing the
structured block of the \code{#1} region.

The \plc{#1-release} event occurs in the thread encountering the
\code{#1} construct after completing any synchronization 
on exit from the \code{#1} region.

\tools
A thread dispatches a registered \code{ompt\_callback\_mutex\_acquire}
callback for each occurrence of #2 \plc{#1-acquire} event 
in that thread.  
This callback has the type signature \code{ompt\_callback\_mutex\_acquire\_t}.

A thread dispatches a registered \code{ompt\_callback\_mutex\_acquired}
callback for each occurrence of #2 \plc{#1-acquired} event 
in that thread.  This callback has the type signature \code{ompt\_callback\_mutex\_t}.

A thread dispatches a registered \code{ompt\_callback\_mutex\_released}
callback for each occurrence of #2 \plc{#1-release} event 
in that thread.  This callback has the type signature \code{ompt\_callback\_mutex\_t}.
The callbacks occur in the task encountering
the #1 construct.  The callbacks should receive \code{ompt\_mutex\_#1}
as their \plc{kind} argument if practical, but a less specific kind is acceptable.
}

\omptMutex{critical}{a}

\restrictions
\begin{itemize}
\item If the \code{hint} clause is specified, the \code{critical} 
      construct must have a \plc{name}.
\item If the \code{hint} clause is specified, each of the
  \code{critical} constructs with the same \plc{name} must have a
  \code{hint} clause for which the \plc{hint-expression} evaluates to the same
  value.

% \item All \code{critical} constructs with the same \plc{name} must 
%       have a \code{hint} clause for which the \plc{hint-expression} evaluates 
%       to the same value if any of them has a \code{hint} clause. 
\end{itemize}

\cppspecificstart
\begin{itemize}
\item A throw executed inside a \code{critical} region must cause execution to resume within 
the same \code{critical} region, and the same thread that threw the exception must catch 
it.
\end{itemize}
\cppspecificend

\fortranspecificstart
The following restrictions apply to the critical construct:

\begin{itemize}
\item If a \plc{name} is specified on a \code{critical} directive, the same \plc{name} must also be 
specified on the \code{end}~\code{critical} directive. 

\item If no \plc{name} appears on the \code{critical} directive, no \plc{name} can appear on the 
\code{end}~\code{critical} directive.
\end{itemize}
\fortranspecificend

\crossreferences
\begin{itemize}
\item \code{omp\_init\_lock\_with\_hint} and \code{omp\_init\_nest\_lock\_with\_hint}, see 
\specref{subsec:omp_init_lock_with_hint and omp_init_nest_lock_with_hint}.

\item \code{ompt\_mutex\_critical}, see
\specref{sec:ompt_mutex_kind_t}.

\item \code{ompt\_callback\_mutex\_acquire\_t}, see
\specref{sec:ompt_callback_mutex_acquire_t}.

\item \code{ompt\_callback\_mutex\_t}, see
\specref{sec:ompt_callback_mutex_t}.

\end{itemize}









\subsection{\code{barrier} Construct}
\index{barrier@{\code{barrier}}}
\index{constructs!barrier@{\code{barrier}}}
\label{subsec:barrier Construct}
\summary
The \code{barrier} construct specifies an explicit barrier at the point at which the construct 
appears. The \code{barrier} construct is a stand-alone directive.

\syntax
\ccppspecificstart
The syntax of the \code{barrier} construct is as follows:

\begin{boxedcode}
\#pragma omp barrier \plc{new-line}
\end{boxedcode}
\ccppspecificend

\fortranspecificstart
The syntax of the \code{barrier} construct is as follows:

\begin{boxedcode}
!\$omp barrier
\end{boxedcode}
\fortranspecificend

\binding
The binding thread set for a \code{barrier} region is the current team. A \code{barrier} region 
binds to the innermost enclosing \code{parallel} region. 

\descr
All threads of the team executing the binding \code{parallel} region must execute the 
\code{barrier} region and complete execution of all explicit tasks bound to this \code{parallel} 
region before any are allowed to continue execution beyond the barrier.

The \code{barrier} region includes an implicit task scheduling point in the current task 
region.

\def\omptSyncRegionEvents#1{
The \plc{#1-begin} event occurs in each thread encountering the
\code{#1} construct on entry to the \code{#1} region.

The \plc{#1-wait-begin} event occurs when a task begins an interval of active or passive waiting
in a \code{#1} region. 

The \plc{#1-wait-end} event occurs when a task ends an interval of active or passive waiting
and resumes execution in a \code{#1} region. 

The \plc{#1-end} event occurs in each thread encountering the
\code{#1} construct after the #1 synchronization on exit from the
\code{#1} region.
}

\def\omptSyncRegionTools#1{
A thread dispatches a registered \code{ompt\_callback\_sync\_region}
callback for each occurrence of a \plc{#1-begin} and \plc{#1-end} event 
in that thread.  The callback occurs in the task encountering
the #1 construct.  This callback has the type signature
\code{ompt\_callback\_sync\_region\_t}. 
The callback receives
\code{ompt\_sync\_region\_#1} as its \plc{kind} argument and
\code{ompt\_scope\_begin} or \code{ompt\_scope\_end} 
as its \plc{endpoint} argument, as appropriate.

A thread dispatches a registered
\code{ompt\_callback\_sync\_region\_wait} callback
for each occurrence of a \plc{#1-wait-begin} and \plc{#1-wait-end} event.
This callback has type signature \code{ompt\_callback\_sync\_region\_t}. 
This callback executes in the context of the task that encountered the
\code{#1} construct. The callback receives
\code{ompt\_sync\_region\_#1} as its \plc{kind} argument and
\code{ompt\_scope\_begin} or \code{ompt\_scope\_end} 
as its \plc{endpoint} argument, as appropriate.
}

\def\omptSyncRegion#1{
\events
\omptSyncRegionEvents{#1}
\tools
\omptSyncRegionTools{#1}
}

\events
\omptSyncRegionEvents{barrier}

A \plc{cancellation} event occurs if cancellation is activated at an implicit cancellation point in an barrier region.

\tools

\omptSyncRegionTools{barrier}

A thread dispatches a registered \code{ompt\_callback\_cancel}
callback for each occurrence of a \plc{cancellation} event in that thread. 
The callback occurs in the context of the encountering task.  The callback has type signature
\code{ompt\_callback\_cancel\_t}. 
The callback receives \code{ompt\_cancel\_detected} as its \plc{flags} argument. 

\restrictions
The following restrictions apply to the \code{barrier} construct:

\begin{itemize}
\item Each \code{barrier} region must be encountered by all threads in a team or by none at all, 
unless cancellation has been requested for the innermost enclosing parallel region.

\item The sequence of worksharing regions and \code{barrier} regions encountered must be the 
same for every thread in a team.
\end{itemize}

\crossreferences
\begin{itemize}
\item \code{ompt\_scope\_begin} and \code{ompt\_scope\_end}, see
\specref{sec:ompt_scope_endpoint_t}.

\item \code{ompt\_sync\_region\_barrier}, see  
\specref{sec:ompt_sync_region_kind_t}.

\item \code{ompt\_callback\_sync\_region\_t}, see
\specref{sec:ompt_callback_sync_region_t}.

\item \code{ompt\_callback\_cancel\_t}, see 
\specref{sec:ompt_callback_cancel_t}.

\end{itemize}






\subsection{Implicit Barriers}
\index{implicit barrier}
\index{barrier, implicit}
\label{subsec:implict-barrier}

Implicit tasks in a parallel region synchronize with one another using
implicit barriers at the end of worksharing constructs and at the end
of the \code{parallel} region. This section describes the OMPT events and
tool callbacks associated with implicit barriers.

Implicit barriers are task scheduling points. For a description of
task sheduling points, associated events, and tool callbacks, see
\specref{subsec:Task Scheduling}.

\events

A \plc{cancellation} event occurs if cancellation is activated at an
implicit cancellation point in an implicit barrier region.

The \plc{implicit-barrier-begin} event occurs in each implicit task
at the beginning of an implicit barrier.

The \plc{implicit-barrier-wait-begin} event occurs when a task begins an interval of
active or passive waiting while executing in an implicit barrier region.

The \plc{implicit-barrier-wait-end} event occurs when a task ends an interval of 
active or waiting and resumes execution of an implicit barrier region.

The \plc{implicit-barrier-end} event occurs in each implicit task
at the end of an implicit barrier.

\tools

A thread dispatches a registered \code{ompt\_callback\_sync\_region}
callback for each occurrence of a \plc{implicit-barrier-begin} and
\plc{implicit-barrier-end} event in that thread.  The callback occurs
in the implicit task executing in a parallel region.
This callback has the type signature
\code{ompt\_callback\_sync\_region\_t}.  The callback receives
\code{ompt\_sync\_region\_barrier} as its \plc{kind} argument and
\code{ompt\_scope\_begin} or \code{ompt\_scope\_end} as its
\plc{endpoint} argument, as appropriate.

A thread dispatches a registered \code{ompt\_callback\_cancel}
callback for each occurrence of a \plc{cancellation} event in that thread.
The callback occurs in the context of the encountering task.  The
callback has type signature \code{ompt\_callback\_cancel\_t}.  The
callback receives \code{ompt\_cancel\_detected} as its \plc{flags}
argument.

A thread dispatches a registered
\code{ompt\_callback\_sync\_region\_wait} callback for each occurrence
of a \plc{implicit-barrier-wait-begin} and
\plc{implicit-barrier-wait-end} event.  This callback has type
signature \code{ompt\_callback\_sync\_region\_t}.  The callback occurs
in each implicit task participating in an implicit barrier.  The
callback receives \code{ompt\_sync\_region\_barrier} as its \plc{kind}
argument and \code{ompt\_scope\_begin} or \code{ompt\_scope\_end} as
its \plc{endpoint} argument, as appropriate.

\restrictions
If a thread is in the state \code{omp\_state\_wait\_barrier\_implicit\_parallel},
a call to \code{ompt\_get\_parallel\_info}
may return a pointer to a copy of the current parallel region's \plc{parallel\_data}
rather than a pointer to the data word for the region itself. This convention enables the master thread
for a parallel region to free storage for the region immediately after the region ends, yet
avoid having some other thread in the region's team
potentially reference the region's \plc{parallel\_data} object after it has been freed. 

\crossreferences
\begin{itemize}

\item \code{ompt\_scope\_begin} and \code{ompt\_scope\_end}, see
  \specref{sec:ompt_scope_endpoint_t}.

\item \code{ompt\_sync\_region\_barrier}, see
  \specref{sec:ompt_sync_region_kind_t}

\item \code{ompt\_cancel\_detected}, see
  \specref{sec:ompt_cancel_flag_t}.

\item \code{ompt\_callback\_sync\_region\_t}, see
  \specref{sec:ompt_callback_sync_region_t}.

\item \code{ompt\_callback\_cancel\_t}, see
  \specref{sec:ompt_callback_cancel_t}.

\end{itemize}









\subsection{\code{taskwait} Construct}
\index{taskwait@{\code{taskwait}}}
\index{constructs!taskwait@{\code{taskwait}}}
\label{subsec:taskwait Construct}
\summary
The \code{taskwait} construct specifies a wait on the completion of child tasks
of the current task. The \code{taskwait} construct is a stand-alone directive.

\syntax
\ccppspecificstart
The syntax of the \code{taskwait} construct is as follows:

\begin{boxedcode}
\#pragma omp taskwait \plc{[clause[ [},\plc{] clause] ... ] new-line}
\end{boxedcode}

where \plc{clause} is one of the following:

\begin{indentedcodelist}
depend(\plc{dependence-type }:\plc{ locator-list})
\end{indentedcodelist}
\ccppspecificend

\fortranspecificstart
The syntax of the \code{taskwait} construct is as follows:

\begin{boxedcode}
!\$omp taskwait \plc{[clause[ [},\plc{] clause] ... ]}
\end{boxedcode}

where \plc{clause} is one of the following:

\begin{indentedcodelist}
depend(\plc{dependence-type }:\plc{ locator-list})
\end{indentedcodelist}

\fortranspecificend

\binding
The \code{taskwait} region binds to the current task region. The binding thread set of the 
\code{taskwait} region is the current team.

\descr

If no \code{depend} clause is present on the \code{taskwait} construct, the
current task region is suspended at an implicit task scheduling point
associated with the construct. The current task region remains suspended until
all child tasks that it generated before the \code{taskwait} region complete
execution.

Otherwise, if one or more \code{depend} clauses are present on the
\code{taskwait} construct, the behavior is as if these clauses were applied to
a \code{task} construct with an empty associated structured block that
generates a \emph{mergeable} and \emph{included task}. Thus, the current task region is
suspended until the \emph{predecessor tasks} of this task complete execution.

\omptSyncRegion{taskwait}

\crossreferences
\begin{itemize}
\item \code{task} construct, see \specref{subsec:task Construct}.

\item Task scheduling, see 
\specref{subsec:Task Scheduling}.

\item \code{depend} clause, see \specref{subsec:depend Clause}.

\item \code{ompt\_scope\_begin} and \code{ompt\_scope\_end}, see
\specref{sec:ompt_scope_endpoint_t}.

\item \code{ompt\_sync\_region\_taskwait}, see
\specref{sec:ompt_sync_region_kind_t}.

\item \code{ompt\_callback\_sync\_region\_t}, see
\specref{sec:ompt_callback_sync_region_t}.

\end{itemize}







\subsection{\code{taskgroup} Construct}
\index{taskgroup@{\code{taskgroup}}}
\index{constructs!taskgroup@{\code{taskgroup}}}
\label{subsec:taskgroup Construct}
\summary
The \code{taskgroup} construct specifies a wait on completion of child tasks of the current 
task and their descendent tasks.

\syntax
\ccppspecificstart
The syntax of the \code{taskgroup} construct is as follows:

\begin{boxedcode}
\#pragma omp taskgroup \plc{[clause[[,] clause] ...]} \plc{new-line}
    \plc{structured-block}
\end{boxedcode}
\ccppspecificend

where \plc{clause} is one of the following:

\begin{indentedcodelist}
task_reduction(\plc{reduction-identifier }:\plc{ list})
\end{indentedcodelist}

\fortranspecificstart
The syntax of the \code{taskgroup} construct is as follows:

\begin{boxedcode}
!\$omp taskgroup \plc{[clause [ [},\plc{] clause] ...]}
    \plc{structured-block}
!\$omp end taskgroup
\end{boxedcode}

where \plc{clause} is one of the following:

\begin{indentedcodelist}
task_reduction(\plc{reduction-identifier }:\plc{ list})
\end{indentedcodelist}

\fortranspecificend

\binding
A \code{taskgroup} region binds to the current task region. A \code{taskgroup} region binds to 
the innermost enclosing \code{parallel} region. 

\descr
When a thread encounters a \code{taskgroup} construct, it starts executing 
the region. All child tasks generated in the \code{taskgroup} region and all 
of their descendants that bind to the same \code{parallel} region as the 
\code{taskgroup} region are part of the \plc{taskgroup set} associated with 
the \code{taskgroup} region.

There is an implicit task scheduling point at the end of the \code{taskgroup} 
region. The current task is suspended at the task scheduling point until all 
tasks in the \plc{taskgroup set} complete execution.

\omptSyncRegion{taskgroup}

\crossreferences
\begin{itemize}
\item Task scheduling, see 
\specref{subsec:Task Scheduling}.
\item \code{task\_reduction} Clause, see \specref{subsubsec:task_reduction clause}.

\item \code{ompt\_scope\_begin} and \code{ompt\_scope\_end}, see
\specref{sec:ompt_scope_endpoint_t}.

\item \code{ompt\_sync\_region\_taskgroup}, see
\specref{sec:ompt_sync_region_kind_t}.

\item \code{ompt\_callback\_sync\_region\_t}, see
\specref{sec:ompt_callback_sync_region_t}.

\end{itemize}










\subsection{\code{atomic} Construct}
\index{atomic@{\code{atomic}}}
\index{constructs!atomic@{\code{atomic}}}
\index{constructs!atomic@{\code{atomic}}}
\index{write, atomic@{\code{write, atomic}}}
\index{read, atomic@{\code{read, atomic}}}
\index{update, atomic@{\code{update, atomic}}}
\label{subsec:atomic Construct}
\summary
The \code{atomic} construct ensures that a specific storage location is accessed atomically, 
rather than exposing it to the possibility of multiple, simultaneous reading and writing 
threads that may result in indeterminate values.

\syntax
In the following syntax, \plc{atomic-clause} is a clause that indicates
the semantics for which atomicity is enforced and is one of the following:

\begin{indentedcodelist}
read
write
update
capture
\end{indentedcodelist}

\ccppspecificstart
The syntax of the \code{atomic} construct takes one of the following forms:

\begin{boxedcode}
\#pragma omp atomic \plc{[}seq\_cst\plc{[},\plc{]]} \plc{atomic-clause} \plc{[[},\plc{]}seq\_cst\plc{]} \plc{new-line}
   \plc{expression-stmt}
\end{boxedcode}

%% where \plc{atomic-clause} is one of the following:
%%
%% \begin{indentedcodelist}
%% read
%% write
%% update
%% capture
%% \end{indentedcodelist}

or

\begin{boxedcode}
\#pragma omp atomic \plc{[}seq\_cst\plc{]} \plc{new-line} 
   \plc{expression-stmt}
\end{boxedcode}

or

\begin{boxedcode}
\#pragma omp atomic \plc{[}seq\_cst\plc{[},\plc{]]} capture \plc{[[},\plc{]}seq\_cst\plc{]} \plc{new-line}
    \plc{structured-block}
\end{boxedcode}

%% \needspace{6\baselineskip}
where \plc{expression-stmt} is an expression statement with one of the following forms:

\begin{itemize}
\item If \plc{atomic-clause} is \code{read}:\\
\code{\plc{v} = \plc{x};}

\item If \plc{atomic-clause} is \code{write}:\\
\code{\plc{x} = \plc{expr};}

\item If \plc{atomic-clause} is \code{update} or not present:\\
\code{\plc{x}++;}\\
\code{\plc{x}\--\--;}\\
\code{++\plc{x};}\\
\code{\--\--\plc{x};}\\
\code{\plc{x} \plc{binop}= \plc{expr};}\\
\code{\plc{x} = \plc{x} \plc{binop} \plc{expr};}\\
\code{\plc{x} = \plc{expr} \plc{binop} \plc{x};}

\item If \plc{atomic-clause} is \code{capture}:\\
\code{\plc{v} = \plc{x}++;}\\
\code{\plc{v} = \plc{x}\--\--;}\\
\code{\plc{v} = ++\plc{x};}\\
\code{\plc{v} = \--\--\plc{x};}\\
\code{\plc{v} = \plc{x} \plc{binop}= \plc{expr};}\\
\code{\plc{v} = \plc{x} = \plc{x} \plc{binop} \plc{expr};}\\
\code{\plc{v} = \plc{x} = \plc{expr} \plc{binop} \plc{x};}

and where \plc{structured-block} is a structured block with one of the following forms:

% blue line floater at top of this page for "C/C++, cont."
\begin{figure}[t!]
\linewitharrows{-1}{dashed}{C/C++ (cont.)}{8em}
\end{figure}

\code{\{\plc{v} = \plc{x}; \plc{x} \plc{binop}= \plc{expr};\}}\\
\code{\{\plc{x} \plc{binop}= \plc{expr}; \plc{v} = \plc{x};\}}\\
\code{\{\plc{v} = \plc{x}; \plc{x} = \plc{x} \plc{binop} \plc{expr};\}}\\
\code{\{\plc{v} = \plc{x}; \plc{x} = \plc{expr} \plc{binop} \plc{x};\}}\\
\code{\{\plc{x} = \plc{x} \plc{binop} \plc{expr}; \plc{v} = \plc{x};\}}\\
\code{\{\plc{x} = \plc{expr} \plc{binop} \plc{x}; \plc{v} = \plc{x};\}}\\
\code{\{\plc{v} = \plc{x}; \plc{x} = \plc{expr};\}}\\
\code{\{\plc{v} = \plc{x}; \plc{x}++;\}}\\
\code{\{\plc{v} = \plc{x}; ++\plc{x};\}}\\
\code{\{++\plc{x}; \plc{v} = \plc{x};\}}\\
\code{\{\plc{x}++\code{;} \plc{v} = \plc{x};\}}\\
\code{\{\plc{v} = \plc{x}; \plc{x}\--\--;\}}\\
\code{\{\plc{v} = \plc{x}; \--\--\plc{x};\}}\\
\code{\{\--\--\plc{x}; \plc{v} = \plc{x};\}}\\
\code{\{\plc{x}\--\--; \plc{v} = \plc{x};\}}
\end{itemize}

In the preceding expressions:

\begin{itemize}
\item \plc{x} and \plc{v} (as applicable) are both \plc{l-value} expressions with scalar type.

\item During the execution of an atomic region, multiple syntactic occurrences of \plc{x} must 
designate the same storage location.

\item Neither of \plc{v} and \plc{expr} (as applicable) may access the storage location designated by \plc{x}.

\item Neither of \plc{x} and \plc{expr} (as applicable) may access the storage location designated by \plc{v}.

\item \plc{expr} is an expression with scalar type. 

\item \plc{binop} is one of \code{+}, \code{*}, \code{-}, \code{/}, 
\code{\&}, \code{\^}, \code{|}, \code{\textless \hspace{0.05em}\textless}, or 
\code{\textgreater \hspace{0.05em}\textgreater}.

\item \plc{binop}, \plc{binop}\code{=}, \code{++}, and \code{\--\--} are not overloaded operators.

\item The expression \plc{x} \plc{binop} \plc{expr} must be numerically equivalent to 
\plc{x} \plc{binop} \plc{(expr)}. This 
requirement is satisfied if the operators in \plc{expr} have precedence greater than \plc{binop}, 
or by using parentheses around \plc{expr} or subexpressions of \plc{expr}.

\item The expression \plc{expr} \plc{binop} \plc{x} must be numerically equivalent to 
\plc{(expr)} \plc{binop} \plc{x}. This 
requirement is satisfied if the operators in \plc{expr} have precedence equal to or greater 
than \plc{binop}, or by using parentheses around \plc{expr} or subexpressions of \plc{expr}.

\item For forms that allow multiple occurrences of \plc{x}, the number of times that \plc{x} is 
evaluated is unspecified.
\end{itemize}
\ccppspecificend

\begin{samepage}
\fortranspecificstart
The syntax of the \code{atomic} construct takes any of the following forms: 

\begin{boxedcode}
!\$omp atomic \plc{[}seq\_cst\plc{[},\plc{]]} read \plc{[[},\plc{]}seq\_cst\plc{]}
    \plc{capture-statement }
\plc{[}!\$omp end atomic\plc{]}
\end{boxedcode}
\end{samepage}

or

\begin{boxedcode}
!\$omp atomic \plc{[}seq\_cst\plc{[},\plc{]]} write \plc{[[},\plc{]}seq\_cst\plc{]}
    \plc{write-statement }
\plc{[}!\$omp end atomic\plc{]}
\end{boxedcode}

or

\begin{boxedcode}
!\$omp atomic \plc{[}seq\_cst\plc{[},\plc{]]} update \plc{[[},\plc{]}seq\_cst\plc{]}
    \plc{update-statement }
\plc{[}!\$omp end atomic\plc{]}
\end{boxedcode}

or

\begin{boxedcode}
!\$omp atomic \plc{[}seq\_cst\plc{]} 
    \plc{update-statement }
\plc{[}!\$omp end atomic\plc{]}
\end{boxedcode}

or

\begin{boxedcode}
!\$omp atomic \plc{[}seq\_cst\plc{[},\plc{]]} capture \plc{[[},\plc{]}seq\_cst\plc{]}
    \plc{update-statement }
    \plc{capture-statement}
!\$omp end atomic
\end{boxedcode}

or

\begin{boxedcode}
!\$omp atomic \plc{[}seq\_cst\plc{[},\plc{]]} capture \plc{[[},\plc{]}seq\_cst\plc{]}
    \plc{capture-statement}
    \plc{update-statement}
!\$omp end atomic
\end{boxedcode}

or

\begin{boxedcode}
!\$omp atomic \plc{[}seq\_cst\plc{[},\plc{]]} capture \plc{[[},\plc{]}seq\_cst\plc{]}
    \plc{capture-statement}
    \plc{write-statement}
!\$omp end atomic
\end{boxedcode}

where \plc{write-statement} has the following form (if \plc{atomic-clause} 
is \code{capture} or \code{write}):

\begin{quote}
\code{\plc{x} = \plc{expr}}
\end{quote}

where \plc{capture-statement} has the following form (if \plc{atomic-clause} 
is \code{capture} or \code{read}):

\begin{quote}
\code{\plc{v} = \plc{x}}
\end{quote}

% blue line floater at top of this page for "Fortran, cont."
\begin{figure}[t!]
\linewitharrows{-1}{dashed}{Fortran (cont.)}{8em}
\end{figure}
and where \plc{update-statement} has one of the following forms (if \plc{atomic-clause} is \code{update}, 
\code{capture}, or not present):

\begin{quote}
\code{\plc{x} = \plc{x operator expr}}

\code{\plc{x} = \plc{expr operator x}}

\code{\plc{x} = \plc{intrinsic\_procedure\_name} (\plc{x}, \plc{expr\_list})}

\code{\plc{x} = \plc{intrinsic\_procedure\_name} (\plc{expr\_list}, \plc{x})}
\end{quote}

In the preceding statements:

\begin{itemize}
\item \plc{x} and \plc{v} (as applicable) are both scalar variables of intrinsic type.

\item \plc{x} must not have the \code{ALLOCATABLE} attribute.

\item During the execution of an atomic region, multiple syntactic occurrences of \plc{x} must 
designate the same storage location.

\item None of \plc{v}, \plc{expr}, and \plc{expr\_list} (as applicable) may access the same storage location as 
\plc{x}. 

\item None of \plc{x}, \plc{expr}, and \plc{expr\_list} (as applicable) may access the same storage location as 
\plc{v}.

\item \plc{expr} is a scalar expression.

\item \plc{expr\_list} is a comma-separated, non-empty list of scalar expressions. If 
\plc{intrinsic\_procedure\_name} refers to \code{IAND}, \code{IOR}, or \code{IEOR}, exactly one expression 
must appear in \plc{expr\_list}.

\item \plc{intrinsic\_procedure\_name} is one of \code{MAX}, \code{MIN}, \code{IAND}, \code{IOR}, or \code{IEOR}.

\item \plc{operator} is one of \code{+}, \code{*}, \code{-}, \code{/}, \code{.AND.}, \code{.OR.}, \code{.EQV.}, or \code{.NEQV.}.

\item The expression \plc{x operator expr} must be numerically equivalent to \plc{x operator (expr)}. 
This requirement is satisfied if the operators in \plc{expr} have precedence greater than 
\plc{operator}, or by using parentheses around \plc{expr} or subexpressions of \plc{expr}.

\item The expression \plc{expr operator x} must be numerically equivalent to \plc{(expr) operator 
x}. This requirement is satisfied if the operators in \plc{expr} have precedence equal to or 
greater than \plc{operator}, or by using parentheses around \plc{expr} or subexpressions of \plc{expr}.

\item \plc{intrinsic\_procedure\_name} must refer to the intrinsic procedure name and not to other 
program entities.

\item \plc{operator} must refer to the intrinsic operator and not to a user-defined operator.

\item All assignments must be intrinsic assignments.

\item For forms that allow multiple occurrences of \plc{x}, the number of times that \plc{x} is 
evaluated is unspecified.
\fortranspecificend

%% \item In all \code{atomic} construct forms, the \code{seq\_cst} clause and the clause that denotes the 
%% type of the atomic construct can appear in any order. In addition, an optional comma 
%% may be used to separate the clauses
\end{itemize}

\begin{samepage}

\binding
If the size of \plc{x} is 8, 16, 32, or 64 bits and \plc{x} is aligned to a
multiple of its size, the binding thread set for the \code{atomic} region is 
all threads on the device. Otherwise, the binding thread set for the 
\code{atomic} region is all threads in the contention group.  \code{atomic} 
regions enforce exclusive access with respect to other \code{atomic} regions 
that access the same storage location \plc{x} among all threads in the 
binding thread set without regard to the teams to which the threads belong. 

\descr
The \code{atomic} construct with the \code{read} clause forces an atomic read of the location 
designated by \plc{x} regardless of the native machine word size.
\end{samepage}

The \code{atomic} construct with the \code{write} clause forces an atomic write of the location 
designated by \plc{x} regardless of the native machine word size.

The \code{atomic} construct with the \code{update} clause forces an atomic update of the location 
designated by \plc{x} using the designated operator or intrinsic. Note that when no clause is 
present, the semantics are equivalent to atomic update. Only the read and write of the 
location designated by \plc{x} are performed mutually atomically. The evaluation of \plc{expr} or 
\plc{expr\_list} need not be atomic with respect to the read or write of the location designated 
by \plc{x}. No task scheduling points are allowed between the read and the write of the 
location designated by \plc{x}.

The \code{atomic} construct with the \code{capture} clause forces an atomic update of the 
location designated by \plc{x} using the designated operator or intrinsic while also capturing 
the original or final value of the location designated by \plc{x} with respect to the atomic 
update. The original or final value of the location designated by \plc{x} is written in the 
location designated by \plc{v} depending on the form of the \code{atomic} construct structured 
block or statements following the usual language semantics. Only the read and write of 
the location designated by \plc{x} are performed mutually atomically. Neither the evaluation 
of \plc{expr} or \plc{expr\_list}, nor the write to the location designated by \plc{v}, need be atomic with
respect to the read or write of the location designated by \plc{x}. No task scheduling points 
are allowed between the read and the write of the location designated by \plc{x}.

Any \code{atomic} construct with a \code{seq\_cst} clause forces the atomically performed 
operation to include an implicit flush operation without a list.

\notestart
\noteheader -- As with other implicit flush regions, 
\specref{subsec:OpenMP Memory Consistency} 
reduces the 
ordering that must be enforced. The intent is that, when the analogous operation exists 
in C++11 or C11, a sequentially consistent \code{atomic} construct has the same semantics as 
a \code{memory\_order\_seq\_cst} atomic operation in C++11/C11. Similarly, a 
non-sequentially consistent \code{atomic} construct has the same semantics as a 
\code{memory\_order\_relaxed} atomic operation in C++11/C11.

Unlike non-sequentially consistent \code{atomic} constructs, sequentially consistent \code{atomic} 
constructs preserve the interleaving (sequentially consistent) behavior of correct, 
data race free programs. However, they are not designed to replace the \code{flush} directive 
as a mechanism to enforce ordering for non-sequentially consistent \code{atomic} constructs, 
and attempts to do so require extreme caution. For example, a sequentially consistent 
\code{atomic}~\code{write} construct may appear to be reordered with a subsequent 
non-sequentially consistent \code{atomic}~\code{write} construct, since such reordering would not 
be observable by a correct program if the second write were outside an \code{atomic} 
directive.
\noteend

For all forms of the \code{atomic} construct, any combination of two or more
of these \code{atomic} constructs enforces mutually exclusive access to the
locations designated by \plc{x} among threads in the binding thread set.  To
avoid race conditions, all accesses of the locations designated by \plc{x}
that could potentially occur in parallel must be protected with an
\code{atomic} construct. 

\code{atomic} regions do not guarantee exclusive access with respect to any accesses outside 
of \code{atomic} regions to the same storage location \plc{x} even if those accesses occur during a 
\code{critical} or \code{ordered} region, while an OpenMP lock is owned by the executing 
task, or during the execution of a \code{reduction} clause.

However, other OpenMP synchronization can ensure the desired exclusive access. For 
example, a barrier following a series of atomic updates to \plc{x} guarantees that subsequent 
accesses do not form a race with the atomic accesses. 

A compliant implementation may enforce exclusive access between \code{atomic} regions 
that update different storage locations. The circumstances under which this occurs are 
implementation defined. 

If the storage location designated by \plc{x} is not size-aligned (that is, if the byte alignment 
of \plc{x} is not a multiple of the size of \plc{x}), then the behavior of the \code{atomic} region is 
implementation defined.

\needspace{16\baselineskip}\begin{samepage}

\omptMutex{atomic}{an}

\restrictions

The following restrictions apply to the \code{atomic} construct:

\begin{itemize}
\item At most one \code{seq\_cst} clause may appear on the construct.
\end{itemize}

\ccppspecificstart
\begin{itemize}
\item All atomic accesses to the storage locations designated by \plc{x} throughout the program 
are required to have a compatible type. 
\end{itemize}
\ccppspecificend
\end{samepage}

\fortranspecificstart
\begin{itemize}
\item All atomic accesses to the storage locations designated by \plc{x} throughout the program 
are required to have the same type and type parameters. 
\end{itemize}
\fortranspecificend

\begin{itemize}
\item OpenMP constructs may not be encountered during execution of an
\code{atomic} region.
\end{itemize}

\crossreferences
\begin{itemize}
\item \code{critical} construct, see 
\specref{subsec:critical Construct}. 

\item \code{barrier} construct, see 
\specref{subsec:barrier Construct}.

\item \code{flush} construct, see 
\specref{subsec:flush Construct}.

\item \code{ordered} construct, see 
\specref{subsec:ordered Construct}.

\item \code{reduction} clause, see 
\specref{subsubsec:reduction clause}.

\item lock routines, see 
\specref{sec:Lock Routines}.

\item \code{ompt\_mutex\_atomic}, see
\specref{sec:ompt_mutex_kind_t}.

\item \code{ompt\_callback\_mutex\_acquire\_t}, see
\specref{sec:ompt_callback_mutex_acquire_t}.

\item \code{ompt\_callback\_mutex\_t}, see
\specref{sec:ompt_callback_mutex_t}.

\end{itemize}









\subsection{\code{flush} Construct}
\index{flush@{\code{flush}}}
\index{constructs!flush@{\code{flush}}}
\label{subsec:flush Construct}
\summary
The \code{flush} construct executes the OpenMP flush operation. This operation makes a 
thread’s temporary view of memory consistent with memory and enforces an order on 
the memory operations of the variables explicitly specified or implied. See the memory 
model description in \specref{sec:Memory Model} for more details. The \code{flush} construct is a 
stand-alone directive.

\syntax
\ccppspecificstart
The syntax of the \code{flush} construct is as follows:

\begin{boxedcode}
\#pragma omp flush \plc{[}(\plc{list})\plc{] new-line}
\end{boxedcode}
\ccppspecificend

\fortranspecificstart
The syntax of the \code{flush} construct is as follows:

\begin{boxedcode}
!\$omp flush \plc{[}(\plc{list})\plc{]}
\end{boxedcode}
\fortranspecificend

\binding
The binding thread set for a \code{flush} region is the encountering thread. Execution of a 
\code{flush} region affects the memory and the temporary view of memory of only the thread 
that executes the region. It does not affect the temporary view of other threads. Other 
threads must themselves execute a flush operation in order to be guaranteed to observe 
the effects of the encountering thread’s flush operation

\descr
A \code{flush} construct without a list, executed on a given thread, operates as if the whole 
thread-visible data state of the program, as defined by the base language, is flushed. A 
\code{flush} construct with a list applies the flush operation to the items in the list, and does 
not return until the operation is complete for all specified list items. An implementation 
may implement a \code{flush} with a list by ignoring the list, and treating it the same as a 
\code{flush} without a list.

\ccppspecificstart
If a pointer is present in the list, the pointer itself is flushed, not the memory block to 
which the pointer refers.
\ccppspecificend

\fortranspecificstart
If the list item or a subobject of the list item has the \code{POINTER} attribute, the allocation 
or association status of the \code{POINTER} item is flushed, but the pointer target is not. If the 
list item is a Cray pointer, the pointer is flushed, but the object to which it points is not. 
If the list item is of type \code{C\_PTR}, the variable is flushed, but the storage that corresponds 
to that address is not flushed. If the list item or the subobject of the list item has the 
\code{ALLOCATABLE} attribute and has an allocation status of allocated, the
allocated variable is flushed; otherwise the allocation status is flushed.
\fortranspecificend

\begin{samepage}
\notestart
\noteheader -- Use of a \code{flush} construct with a list is extremely error prone and users are 
strongly discouraged from attempting it. The following examples illustrate the ordering 
properties of the flush operation. In the following incorrect pseudocode example, the 
programmer intends to prevent simultaneous execution of the protected section by the 
two threads, but the program does not work properly because it does not enforce the 
proper ordering of the operations on variables \code{a} and \code{b}. Any shared data accessed in the 
protected section is not guaranteed to be current or consistent during or after the 
protected section. The atomic notation in the pseudocode in the following two examples 
indicates that the accesses to \code{a} and \code{b} are \code{ATOMIC} writes and captures. Otherwise both 
examples would contain data races and automatically result in unspecified behavior. 
\end{samepage}

% Outlined mixed code and text:

\parbox{\linewidth}{%
\begin{spacing}{0.90}\begin{framed}
\emph{Incorrect example:}\\
\hspace{0.3\textwidth}\code{a = b = 0}
\begin{tabular}{ p{0.5\textwidth} p{0.5\textwidth}}\\
\hspace{0.1\textwidth}\plc{thread 1} & \hspace{0.1\textwidth}\plc{thread 2}\\[1.0em]
\code{atomic(b = 1)} & \code{atomic(a = 1)}\\
\code{\plc{flush}(b)} & \code{\plc{flush}(a)}\\
\code{\plc{flush}(a)} & \code{\plc{flush}(b)}\\
\code{atomic(tmp = a)} & \code{atomic(tmp = b)}\\
\code{if (tmp == 0) then} & \code{if (tmp == 0) then}\\
\hspace{1.25em}\plc{protected section} & \hspace{1.25em}\plc{protected section}\\
\code{end if} & \code{end if}\\
\end{tabular}
\end{framed}\end{spacing}} % use \parbox to keep the lines together for code only

The problem with this example is that operations on variables \code{a} and \code{b} are not ordered 
with respect to each other. For instance, nothing prevents the compiler from moving the 
flush of \code{b} on thread 1 or the flush of \code{a} on thread 2 to a position completely after the 
protected section (assuming that the protected section on thread 1 does not reference \code{b} and 
the protected section on thread 2 does not reference \code{a}). If either re-ordering happens, both 
threads can simultaneously execute the protected section.

The following pseudocode example correctly ensures that the protected section is executed 
by not more than one of the two threads at any one time. Execution of the 
protected section by neither thread is considered correct in this example. This occurs if 
both flushes complete prior to either thread executing its \code{if} statement.

\parbox{\linewidth}{%
\begin{spacing}{0.95}\begin{framed}
\emph{Correct example:}\\
\hspace{0.3\textwidth}\code{a = b = 0}
\begin{tabular}{ p{0.5\textwidth} p{0.5\textwidth}}\\
\hspace{0.1\textwidth}\plc{thread 1} & \hspace{0.1\textwidth}\plc{thread 2}\\[1.0em]
\code{atomic(b = 1)} & \code{atomic(a = 1)}\\
\code{\plc{flush}(a,b)} & \code{\plc{flush}(a,b)}\\
\code{atomic(tmp = a)} & \code{atomic(tmp = b)}\\
\code{if (tmp == 0) then} & \code{if (tmp == 0) then}\\
\hspace{1.25em}\plc{protected section} & \hspace{1.25em}\plc{protected section}\\
\code{end if} & \code{end if}\\
\end{tabular}
\end{framed}\end{spacing}} % use \parbox to keep lines together for code only
\bigskip

The compiler is prohibited from moving the flush at all for either thread, ensuring that the 
respective assignment is complete and the data is flushed before the \code{if} statement is 
executed.
\noteend
\bigskip

A \code{flush} region without a list is implied at the following locations:
\begin{itemize}
\item During a barrier region.
\item At entry to a \code{target update} region whose corresponding construct has a \code{to} clause.
\item At exit from a \code{target update} region whose corresponding construct has a \code{from} clause.
\item At entry to and exit from \code{parallel}, \code{critical}, \code{target} and \code{target data} regions.
\item At entry to and exit from an \code{ordered} region, if a \code{threads} clause or a \code{depend} clause is present, or if no clauses are present.
\item At entry to a \code{target enter data} region. 
\item At exit from a \code{target exit data} region. 
\item At exit from worksharing regions unless a \code{nowait} is present.
\item At entry to and exit from the \code{atomic} operation (read, write, update, or capture) 
performed in a sequentially consistent atomic region.
\item During \code{omp\_set\_lock} and \code{omp\_unset\_lock} regions.
\item During \code{omp\_test\_lock}, \code{omp\_set\_nest\_lock}, \code{omp\_unset\_nest\_lock} 
and \code{omp\_test\_nest\_lock} regions, if the region causes the lock to be set or unset.
\item Immediately before and immediately after every task scheduling point.
\item During a \code{cancel} or \code{cancellation point} region, if the \plc{cancel-var} ICV is \plc{true} and cancellation has been activated.
\end{itemize}

A \code{flush} region with a list is implied at the following locations:
\begin{itemize}
\item At entry to and exit from the \code{atomic} operation (read, write, update, or capture) 
performed in a non-sequentially consistent \code{atomic} region, where the list contains 
only the storage location designated as x according to the description of the syntax of 
the \code{atomic} construct in 
\specref{subsec:atomic Construct}.
\end{itemize}

\notestart
\noteheader -- A \code{flush} region is not implied at the following locations:

\begin{itemize}
\item At entry to worksharing regions.

\item At entry to or exit from a \code{master} region.
\end{itemize}
\noteend

\events

The \plc{flush} event occurs in a thread encountering the
\code{flush} construct.

\tools

A thread dispatches a registered \code{ompt\_callback\_flush} callback
for each occurrence of a \plc{flush} event in that thread. This
callback has the type signature \code{ompt\_callback\_flush\_t}.

\crossreferences
\begin{itemize}
\item \code{ompt\_callback\_flush\_t}, see
\specref{sec:ompt_callback_flush_t}.
\end{itemize}





\subsection{\code{ordered} Construct}
\index{ordered@{\code{ordered}}}
\index{constructs!ordered@{\code{ordered}}}
\label{subsec:ordered Construct}
\summary
The \code{ordered} construct either specifies a structured block in a loop,
\code{simd}, or loop SIMD region that will be executed in the order of the
loop iterations, or it is a stand-alone directive that specifies
cross-iteration dependences in a doacross loop nest. The \code{ordered}
construct sequentializes and orders the execution of \code{ordered} regions
while allowing code outside the region to run in parallel.

\begin{samepage}
\syntax
\ccppspecificstart
The syntax of the \code{ordered} construct is as follows:

\begin{boxedcode}
\#pragma omp ordered \plc{[clause[ [},\plc{] clause] ]} \plc{new-line}
   \plc{structured-block}
\end{boxedcode}

where \plc{clause} is one of the following:
\begin{indentedcodelist}
threads
simd
\end{indentedcodelist}

or

\begin{boxedcode}
\#pragma omp ordered \plc{clause [[[},\plc{] clause] ... ]} \plc{new-line}
\end{boxedcode}
where \plc{clause} is one of the following:
\begin{indentedcodelist}
depend(source)
depend(sink : \plc{vec})
\end{indentedcodelist}


\ccppspecificend
\end{samepage}

\fortranspecificstart
The syntax of the \code{ordered} construct is as follows:

\begin{boxedcode}
!\$omp ordered \plc{[clause[ [},\plc{] clause] ]}
    \plc{structured-block}
!\$omp end ordered
\end{boxedcode}

where \plc{clause} is one of the following:
\begin{indentedcodelist}
threads
simd
\end{indentedcodelist}

or

\begin{boxedcode}
!\$omp ordered \plc{clause [[[},\plc{] clause] ... ]}
\end{boxedcode}

where \plc{clause} is one of the following:
\begin{indentedcodelist}
depend(source)
depend(sink : \plc{vec})
\end{indentedcodelist}
\fortranspecificend

If the \code{depend} clause is specified, the \code{ordered} construct is a stand-alone directive.
 
\binding
The binding thread set for an \code{ordered} region is the current team. An \code{ordered} region 
binds to the innermost enclosing \code{simd} or loop SIMD region if the
\code{simd} clause is present, and otherwise it binds to the innermost
enclosing loop region. \code{ordered} regions that bind to different regions
execute independently of each other.

\descr
If no clause is specified, the \code{ordered} construct behaves as if the
\code{threads} clause had been specified. If the \code{threads} clause is
specified, the threads in the team executing the loop region execute
\code{ordered} regions sequentially in the order of the loop iterations.
If any \code{depend} clauses are specified then those clauses specify the 
order in which the threads in the team execute \code{ordered} regions. If
the \code{simd} clause is specified, the \code{ordered} regions encountered by
any thread will use only a single SIMD lane to execute the \code{ordered}
regions in the order of the loop iterations.

When the thread executing the first iteration of the loop encounters an
\code{ordered} construct, it can enter the \code{ordered} region without
waiting. When a thread executing any subsequent iteration encounters an
\code{ordered} construct without a \code{depend} clause, it waits at the
beginning of the \code{ordered} region until execution of all \code{ordered}
regions belonging to all previous iterations has completed. When a thread
executing any subsequent iteration encounters an \code{ordered} construct with
one or more \code{depend(sink:\plc{vec})} clauses, it waits until its dependences on 
all valid iterations specified by the \code{depend} clauses
are satisfied before it completes execution of the \code{ordered} region.
A specific dependence is satisfied when a thread executing the corresponding
iteration encounters an \code{ordered} construct with a \code{depend(source)} clause.

\omptMutex{ordered}{an}


\restrictions
Restrictions to the \code{ordered} construct are as follows:

\begin{itemize}
\item At most one \code{threads} clause can appear on an \code{ordered} construct.

\item At most one \code{simd} clause can appear on an \code{ordered} construct.

\item At most one \code{depend(source)} clause can appear on an \code{ordered} construct.

\item Either \code{depend(sink:\plc{vec})} clauses or \code{depend(source)}
clauses may appear on an \code{ordered} construct, but not both.

\item The loop or loop SIMD region to which an \code{ordered}
region arising from an \code{ordered} construct without a \code{depend}
clause binds must have an \code{ordered} clause without the parameter
specified on the corresponding loop or loop SIMD directive.

\item The loop region to which an \code{ordered} region arising from an
\code{ordered} construct with any \code{depend}
clauses binds must have an \code{ordered} clause with the parameter specified 
on the corresponding loop directive. 

\item An \code{ordered} construct with the \code{depend} clause specified must
be closely nested inside a loop (or parallel loop) construct. 

\item An \code{ordered} region arising from an \code{ordered} construct with
the \code{simd} clause specified must be closely nested inside a
\code{simd} or loop SIMD region.

\item An \code{ordered} region arising from an \code{ordered} construct with
  both the \code{simd} and \code{threads} clauses must be closely nested inside
  a loop SIMD region.
  
\item During execution of an iteration of a loop or a loop nest within a loop, \code{simd}, or loop SIMD
region, a thread must not execute more than one \code{ordered} region arising
from an \code{ordered} construct without a \code{depend} clause. 
\end{itemize}
\cppspecificstart
\begin{itemize}
\item A throw executed inside a \code{ordered} region must cause execution to 
resume within the same \code{ordered} region, and the same thread that threw 
the exception must catch it.
\end{itemize}
\cppspecificend



\crossreferences
\begin{itemize}
\item loop construct, see 
\specref{subsec:Loop Construct}.

\item \code{simd} construct, see
\specref{subsec:simd Construct}.

\item parallel loop construct, see 
\specref{subsec:Parallel Loop Construct}.

\item \code{depend} Clause, see
\specref{subsec:depend Clause}

\item \code{ompt\_mutex\_ordered}, see
\specref{sec:ompt_mutex_kind_t}.

\item \code{ompt\_callback\_mutex\_acquire\_t}, see
\specref{sec:ompt_callback_mutex_acquire_t}.

\item \code{ompt\_callback\_mutex\_t}, see
\specref{sec:ompt_callback_mutex_t}.

\end{itemize}


\subsection{\code{depend} Clause}
\index{depend@{\code{depend}}}
\index{clauses!depend@{\code{depend}}}
\label{subsec:depend Clause}
\summary
The \code{depend} clause enforces additional constraints on the scheduling of tasks or loop iterations.  These 
constraints establish dependences only between sibling tasks or between loop iterations. 

\syntax
The syntax of the \code{depend} clause is as follows:

\begin{boxedcode}
depend(\plc{dependence-type }:\plc{ locator-list})
\end{boxedcode}

where \plc{dependence-type} is one of the following:
\begin{indentedcodelist}
in
out
inout
\end{indentedcodelist}

or

\begin{boxedcode}
depend(\plc{dependence-type})
\end{boxedcode}

where \plc{dependence-type} is:
\begin{indentedcodelist}
source
\end{indentedcodelist}

or

\begin{boxedcode}
depend(\plc{dependence-type} : \plc{vec})
\end{boxedcode}

where \plc{dependence-type} is:
\begin{indentedcodelist}
sink
\end{indentedcodelist}

and where \plc{vec} is the iteration vector, which has the form:

x$_{1}$ [$\pm $ d$_{1}$], x$_{2}$ [$\pm $ d$_{2}$], \ldots, x$_{\plc{n}}$ [$\pm $ d$_{\plc{n}}$]

where \plc{n} is the value specified by the \code{ordered} clause in the loop
directive, x$_{\plc{i}}$ denotes the loop iteration variable of the \plc{i}-th
nested loop associated with the loop directive, and d$_{\plc{i}}$ is a
constant non-negative integer.

\descr
Task dependences are derived from the \plc{dependence-type} of a \code{depend} clause and its list 
items when \plc{dependence-type} is \code{in}, \code{out}, or \code{inout}.

For the \code{in} \plc{dependence-type}, if the storage location of at least one
of the list items is the same as the storage location of a list item appearing
in a \code{depend} clause with an \code{out} or \code{inout} \plc{dependence-type} on a construct
from which a sibling task was previously generated, then the generated task
will be a dependent task of that sibling task.

For the \code{out} and \code{inout} \plc{dependence-types}, if the storage location of at least one
of the list items is the same as the storage location of a list item appearing
in a \code{depend} clause with an \code{in}, \code{out}, or \code{inout} \plc{dependence-type} on
a construct from which a sibling task was previously generated, then the
generated task will be a dependent task of that sibling task.

\fortranspecificstart
If a list item has the \code{ALLOCATABLE} attribute and its allocation
status is unallocated, the behavior is unspecified. If a list item has
the \code{POINTER} attribute and its association status is
disassociated or undefined, the behavior is unspecified.

The list items that appear in the \code{depend} clause may include array sections.
\fortranspecificend

\notestart
\noteheader -- The enforced task dependence establishes a synchronization of memory 
accesses performed by a dependent task with respect to accesses performed by the 
predecessor tasks. However, it is the responsibility of the programmer to synchronize properly with respect to other concurrent accesses that occur outside of those tasks.
\noteend

The \code{source} \plc{dependence-type} specifies the satisfaction of
cross-iteration dependences that arise from the current iteration.

The \code{sink} \plc{dependence-type} specifies a cross-iteration dependence,
where the iteration vector \plc{vec} indicates the iteration that satisfies
the dependence.

If the iteration vector \plc{vec} does not occur in the iteration space,
the \code{depend} clause is ignored.  If all \code{depend} clauses on an
\code{ordered} construct are ignored then the construct is ignored.

\notestart
\noteheader -- If the iteration vector \plc{vec} does not indicate a lexicographically earlier iteration, it can cause a deadlock.
\noteend

\events

The \plc{task-dependences} event occurs in a thread encountering a
tasking construct with a \code{depend} clause immediately after the
\plc{task-create} event for the new task.  

The \plc{task-dependence} event indicates an unfulfilled dependence for the generated task.
This event occurs in a thread that observes the unfulfilled dependence before it is satisfied.
%unfulfilled dependence... already implies that if the dependence is satisfied (not unfulfilled)
% then there is no events... so can skip the sentence below.
%A dependence will not cause an event if the
%dependence producing task finishes before a dependence consuming task is created.

\tools

A thread dispatches the \code{ompt\_callback\_task\_dependences} callback
for each occurrence of the \plc{task-dependences} event to 
announce its dependences with respect to the list items in the \code{depend} clause.
This callback has type signature
\code{ompt\_callback\_task\_dependences\_t}.

A thread dispatches the \code{ompt\_callback\_task\_dependence}
callback for a \plc{task-dependence} event to report a
dependence between a predecessor task  (\plc{src\_task\_data}) and a dependent task
(\plc{sink\_task\_data}).  This callback has type signature
\code{ompt\_callback\_task\_dependence\_t}.

\restrictions
Restrictions to the \code{depend} clause are as follows:

\begin{itemize}
\item List items used in \code{depend} clauses of the same task or sibling tasks must indicate 
identical storage locations or disjoint storage locations. 

\item List items used in \code{depend} clauses cannot be zero-length array sections. 

\fortranspecificstart
\item A common block name cannot appear in a \code{depend} clause.
\fortranspecificend

\item For a \plc{vec} element of \code{sink} \plc{dependence-type} of the form
x$_{i}$ $+$ d$_{i}$ or x$_{i}$ $-$ d$_{i}$ if the loop iteration variable
x$_{i}$ has an integral or pointer type, the expression x$_{i}$ $+$ d$_{i}$ or
x$_{i}$ $-$ d$_{i}$ for any value of the loop iteration variable x$_{i}$ that
can encounter the \code{ordered} construct must be computable in the
loop iteration variable's type without overflow.

\cppspecificstart
\item For a \plc{vec} element of \code{sink} \plc{dependence-type} of the form
x$_{i}$ $+$ d$_{i}$ or x$_{i}$ $-$ d$_{i}$ if the loop iteration variable
x$_{i}$ is of a random access iterator type other than pointer type,
the expression $($ x$_{i}$ - lb$_{i}$ $)$ $+$ d$_{i}$ or
$($ x$_{i}$ - lb$_{i}$ $)$ $-$ d$_{i}$ for any value of the loop iteration variable
x$_{i}$ that can encounter the \code{ordered} construct must be computable in the
type that would be used by \plc{std::distance} applied to variables of the
type of x$_{i}$ without overflow.
\cppspecificend

\ccppspecificstart
\item A bit-field cannot appear in a \code{depend} clause.
\ccppspecificend

\end{itemize}

\crossreferences
\begin{itemize}
\item Array sections, see
\specref{sec:Array Sections}.

\item \code{task} construct, see 
\specref{subsec:task Construct}.

\item \code{target}~\code{enter}~\code{data} construct, see 
\specref{subsec:target enter data Construct}.

\item \code{target}~\code{exit}~\code{data} construct, see 
\specref{subsec:target exit data Construct}.

\item \code{target} construct, see 
\specref{subsec:target Construct}.

\item \code{target}~\code{update} construct, see 
\specref{subsec:target update Construct}.

\item Task scheduling constraints, see
\specref{subsec:Task Scheduling}. 

\item \code{ordered} construct, see
\specref{subsec:ordered Construct}. 

\item \code{ompt\_callback\_task\_dependences\_t}, see
\specref{sec:ompt_callback_task_dependences_t}.

\item \code{ompt\_callback\_task\_dependence\_t}, see
\specref{sec:ompt_callback_task_dependence_t}.
\end{itemize}








\section{Cancellation Constructs}
\label{sec:Cancellation Constructs}
\index{cancellation constructs}
\index{constructs!cancellation constructs}
\subsection{\code{cancel} Construct}
\index{cancel@{\code{cancel}}}
\index{constructs!cancel@{\code{cancel}}}
\index{cancellation constructs!cancel@{\code{cancel}}}
\label{subsec:cancel Construct}
\summary
The \code{cancel} construct activates cancellation of the innermost enclosing region of the 
type specified. The \code{cancel} construct is a stand-alone directive.

\syntax
\ccppspecificstart
The syntax of the \code{cancel} construct is as follows:

\begin{boxedcode}
\#pragma omp cancel \plc{construct-type-clause [ [},\plc{] if-clause] new-line}
\end{boxedcode}

\begin{samepage}
where \plc{construct-type-clause} is one of the following:

\begin{indentedcodelist}
parallel
sections
for
taskgroup
\end{indentedcodelist}
\end{samepage}

and \plc{if-clause} is
\begin{indentedcodelist}
if (\plc{[} cancel :\plc{] scalar-expression})
\end{indentedcodelist}
\ccppspecificend

\fortranspecificstart
The syntax of the \code{cancel} construct is as follows:

\begin{boxedcode}
!\$omp cancel \plc{construct-type-clause [ [},\plc{] if-clause]}
\end{boxedcode}
\vspace{-1.5ex} %% HACK

\begin{samepage}
where \plc{construct-type-clause} is one of the following:
\begin{indentedcodelist}
parallel
sections
do
taskgroup
\end{indentedcodelist}
\end{samepage}

and \plc{if-clause} is
\begin{indentedcodelist}
if (\plc{[} cancel :\plc{] scalar-logical-expression})
\end{indentedcodelist}
\fortranspecificend


\binding
The binding thread set of the \code{cancel} region is the current team.
The binding region of the \code{cancel} region 
is the innermost enclosing region of the type corresponding to the \plc{construct-type-clause} 
specified in the directive (that is, the innermost \code{parallel}, \code{sections}, loop, or 
\code{taskgroup} region).

\descr
The \code{cancel} construct activates cancellation of the binding region only if the \plc{cancel-var} ICV
is \plc{true}, in which case the \code{cancel} construct causes the encountering task to continue execution 
at the end of the binding region if \plc{construct-type-clause} is \code{parallel}, \code{for}, \code{do}, or \code{sections}. 
If the \plc{cancel-var} ICV is \plc{true} and \plc{construct-type-clause} is \code{taskgroup}, the encountering task continues execution at the end of the current task region.
If the \plc{cancel-var} ICV is \plc{false}, the \code{cancel} construct is 
ignored.

Threads check for active cancellation only at cancellation points that are 
implied at the following locations:

\begin{itemize}
\item \code{cancel} regions;
\item \code{cancellation}~\code{point} regions;
\item \code{barrier} regions;
\item implicit barriers regions.
\end{itemize}

When a thread reaches one of the above cancellation points and if the \plc{cancel-var} ICV is \plc{true},  then:
\begin{itemize}
\item If the thread is at a \code{cancel} or \code{cancellation point} region 
and \plc{construct-type-clause} is \code{parallel}, \code{for}, \code{do}, or 
\code{sections}, the thread continues execution at the end of the canceled 
region if cancellation has been activated for the innermost enclosing region 
of the type specified.

\item If the thread is at a \code{cancel} or \code{cancellation point} region 
and \plc{construct-type-clause} is \code{taskgroup}, the encountering task 
checks for active cancellation of all of the \plc{taskgroup sets} to which the 
encountering task belongs, and continues execution at the end of the current 
task region if cancellation has been activated for any of the \plc{taskgroup sets}.

\item If the encountering task is at a barrier region, the encountering task 
checks for active cancellation of the innermost enclosing \code{parallel} 
region. If cancellation has been activated, then the encountering task 
continues execution at the end of the canceled region.
\end{itemize}

\newpage %% HACK
\notestart
\noteheader -- If one thread activates cancellation and another thread encounters a cancellation 
point, the order of execution between the two threads is non-deterministic. 
Whether the thread that encounters a cancellation point detects the activated cancellation 
depends on the underlying hardware and operating system.
\noteend

When cancellation of tasks is activated through the
\code{cancel}~\code{taskgroup} construct, the tasks that belong to the
\plc{taskgroup set} of the innermost enclosing \code{taskgroup} region
will be canceled. The task that encountered the
\code{cancel}~\code{taskgroup} construct continues execution at the
end of its \code{task} region, which implies completion of that
task. Any task that belongs to the innermost enclosing
\code{taskgroup} and has already begun execution must run to
completion or until a cancellation point is reached. Upon reaching a
cancellation point and if cancellation is active, the task continues
execution at the end of its \code{task} region, which implies the
task's completion. Any task that belongs to the innermost enclosing
\code{taskgroup} and that has not begun execution may be discarded,
which implies its completion.

When cancellation is active for a \code{parallel}, \code{sections}, or loop region, each 
thread of the binding thread set resumes execution at the end of the canceled region if a 
cancellation point is encountered. If the canceled region is a \code{parallel} region, any 
tasks that have been created by a \code{task} construct and their descendent tasks are 
canceled according to the above \code{taskgroup} cancellation semantics. If the canceled 
region is a \code{sections}, or loop region, no task cancellation occurs.

\cppspecificstart
The usual C++ rules for object destruction are followed when cancellation is performed.
\cppspecificend

\fortranspecificstart
All private objects or subobjects with \code{ALLOCATABLE} attribute that are allocated inside 
the canceled construct are deallocated.
\fortranspecificend

If the canceled construct contains a \code{reduction} or \code{lastprivate} clause, the final 
value of the \code{reduction} or \code{lastprivate} variable is undefined.

When an \code{if} clause is present on a \code{cancel} construct and the \code{if} expression evaluates 
to \plc{false}, the \code{cancel} construct does not activate cancellation. The cancellation point 
associated with the \code{cancel} construct is always encountered regardless of the value of 
the \code{if} expression.

\notestart
\noteheader -- The programmer is responsible for releasing locks and
other synchronization data structures that might cause a deadlock when
a \code{cancel} construct is encountered and blocked threads cannot be
canceled. The programmer is also responsible for ensuring proper
synchronizations to avoid deadlocks that might arise from cancellation
of OpenMP regions that contain OpenMP synchronization constructs.
\noteend

\events

The \plc{cancel} event occurs after a task encounters a 
\code{cancel} construct if the \plc{cancel-var} ICV is \plc{true}.

\tools

A thread dispatches a registered \code{ompt\_callback\_cancel}
callback for each occurrence of a \plc{cancel} event in that thread. 
The callback occurs in the context of the encountering task.  The callback has type signature
\code{ompt\_callback\_cancel\_t}. 
The callback receives \code{ompt\_cancel\_activated} as its \plc{flags} argument. 

\restrictions
The restrictions to the \code{cancel} construct are as follows:

\begin{itemize}
\item The behavior for concurrent cancellation of a region and a region nested within it is 
unspecified.

\item If \plc{construct-type-clause} is \code{taskgroup}, the \code{cancel}
construct must be closely nested inside a \code{task} construct and the
\code{cancel} region must be closely nested inside a \code{taskgroup} region. If
\plc{construct-type-clause} is \code{sections}, the \code{cancel} construct
must be closely nested inside a \code{sections} or \code{section} construct.
Otherwise, the \code{cancel} construct must be closely
nested inside an OpenMP construct that matches the type specified in
\plc{construct-type-clause} of the \code{cancel} construct.

\item A worksharing construct that is canceled must not have a \code{nowait} clause.

\item A loop construct that is canceled must not have an \code{ordered} clause.

\item During execution of a construct that may be subject to cancellation, a
thread must not encounter an orphaned cancellation point. That is, a
cancellation point must only be encountered within that construct and must
not be encountered elsewhere in its region.
\end{itemize}

\crossreferences
\begin{itemize}
\item \plc{cancel-var} ICV, see
\specref{subsec:ICV Descriptions}.

\item \code{cancellation}~\code{point} construct, see 
\specref{subsec:cancellation point Construct}.

\item \code{if} Clause, see \specref{sec:if Clause}.

\item \code{omp\_get\_cancellation} routine, see 
\specref{subsec:omp_get_cancellation}.

\item \code{ompt\_callback\_cancel\_t}, see \specref{sec:ompt_callback_cancel_t}.
\end{itemize}









\subsection{\code{cancellation}~\code{point} Construct}
\index{cancellation point@{\code{cancellation}~\code{point}}}
\index{constructs!cancellation point@{\code{cancellation}~\code{point}}}
\label{subsec:cancellation point Construct}
\index{cancellation constructs!cancellation point@{\code{cancellation}~\code{point}}}
\summary
The \code{cancellation}~\code{point} construct introduces a user-defined cancellation point at 
which implicit or explicit tasks check if cancellation of the innermost enclosing region 
of the type specified has been activated. The \code{cancellation}~\code{point} construct is a 
stand-alone directive.

\syntax
\ccppspecificstart
The syntax of the \code{cancellation}~\code{point} construct is as follows:

\begin{boxedcode}
\#pragma omp cancellation point \plc{construct-type-clause new-line}
\end{boxedcode}

where \plc{construct-type-clause} is one of the following:

\begin{indentedcodelist}
parallel
sections
for
taskgroup
\end{indentedcodelist}
\ccppspecificend

\fortranspecificstart
The syntax of the \code{cancellation}~\code{point} construct is as follows:

\begin{boxedcode}
!\$omp cancellation point \plc{construct-type-clause}
\end{boxedcode}

where \plc{construct-type-clause} is one of the following:

\begin{indentedcodelist}
parallel
sections
do
taskgroup
\end{indentedcodelist}
\fortranspecificend

\binding
The binding thread set of the \code{cancellation point} construct is the current team.
The binding region of the \code{cancellation point} region is the innermost enclosing region of the type corresponding to the \plc{construct-type-clause} 
specified in the directive (that is, the innermost \code{parallel}, \code{sections}, loop, or 
\code{taskgroup} region).

\descr
This directive introduces a user-defined cancellation point at which an implicit or 
explicit task must check if cancellation of the innermost enclosing region of the type 
specified in the clause has been requested. This construct does not implement any 
synchronization between threads or tasks.

When an implicit or explicit task reaches a user-defined cancellation point and if 
the \plc{cancel-var} ICV is \plc{true}, then:
\begin{itemize}
\item If the \plc{construct-type-clause} of the encountered \code{cancellation
point} construct is \code{parallel}, \code{for}, \code{do}, or \code{sections},
the thread continues execution at the end of the canceled region if 
cancellation has been activated for the innermost enclosing region of 
the type specified.

\item If the \plc{construct-type-clause} of the encountered 
\code{cancellation point} construct is \code{taskgroup}, the encountering 
task checks for active cancellation of all  \plc{taskgroup sets} to which the 
encountering task belongs and continues execution at the end of the current 
task region if cancellation has been activated for any of them.
\end{itemize}


\events

The \plc{cancellation} event occurs if a task encounters a 
cancellation point and detected the activation of cancellation.

\tools

A thread dispatches a registered \code{ompt\_callback\_cancel}
callback for each occurrence of a \plc{cancellation} event in that thread. 
The callback occurs in the context of the encountering task.  The callback has type signature
\code{ompt\_callback\_cancel\_t}. 
The callback receives \code{ompt\_cancel\_detected} as its \plc{flags} argument. 

\restrictions
\begin{itemize}
\item A \code{cancellation}~\code{point} construct for which
\plc{construct-type-clause} is \code{taskgroup} must be closely nested
inside a \code{task} construct, and the \code{cancellation}~\code{point}
region must be closely nested inside a \code{taskgroup} region. A
\code{cancellation}~\code{point} construct for which
\plc{construct-type-clause} is \code{sections} must be closely nested
inside a \code{sections} or \code{section} construct. Otherwise, a
\code{cancellation}~\code{point} construct must be closely nested inside
an OpenMP construct that matches the type specified in
\plc{construct-type-clause}.
\end{itemize}

\begin{samepage}
\crossreferences
\begin{itemize}
\item \plc{cancel-var} ICV, see
\specref{subsec:ICV Descriptions}.

\item \code{cancel} construct, see 
\specref{subsec:cancel Construct}.

\item \code{omp\_get\_cancellation} routine, see 
\specref{subsec:omp_get_cancellation}.

\item \code{ompt\_callback\_cancel\_t}, see \specref{sec:ompt_callback_cancel_t}.

\end{itemize}
\end{samepage}









\section{Data Environment}
\label{sec:Data Environment}
\index{data environment}
This section presents a directive and several clauses for controlling the data environment 
during the execution of \code{teams}, \code{parallel}, \code{simd}, task generating, and worksharing regions.

\begin{itemize}
\item \specref{subsec:Data-sharing Attribute Rules} 
describes how the data-sharing attributes of variables
referenced in \code{teams}, \code{parallel}, \code{simd}, task generating, and worksharing regions are determined.

\item The \code{threadprivate} directive, which is provided to create threadprivate memory, 
is described in \specref{subsec:threadprivate Directive}.

\item Clauses that may be specified on directives to control the data-sharing attributes of 
variables referenced in \code{teams}, \code{parallel}, \code{simd}, task generating, or worksharing constructs are described in \specref{subsec:Data-Sharing Attribute Clauses}

\item Clauses that may be specified on directives to copy data values from private or 
threadprivate variables on one thread to the corresponding variables on other threads 
in the team are described in \specref{subsec:Data Copying Clauses}.

\item Clauses that may be specified on directives to control the data-mapping of variables to a device data environment are described in \specref{subsec:map Clause}.
\end{itemize}










\subsection{Data-sharing Attribute Rules}
\label{subsec:Data-sharing Attribute Rules}
\index{data-sharing attribute rules}
\index{attributes, data-sharing}
This section describes how the data-sharing attributes of variables referenced in 
\code{target}, \code{parallel}, \code{task}, \code{taskloop}, \code{simd}, and worksharing regions are determined. The following two cases are described separately:

\begin{itemize}
\item \specref{subsubsec:Data-sharing Attribute Rules for Variables Referenced in a Construct} 
describes the data-sharing attribute rules for variables 
referenced in a construct.

\item \specref{subsubsec:Data-sharing Attribute Rules for Variables Referenced in a Region but not in a Construct} describes the data-sharing attribute rules for variables 
referenced in a region, but outside any construct. 
\end{itemize}









\subsubsection{Data-sharing Attribute Rules for Variables Referenced in a Construct}
\label{subsubsec:Data-sharing Attribute Rules for Variables Referenced in a Construct}
The data-sharing attributes of variables that are referenced in a construct can be 
\emph{predetermined}, \emph{explicitly determined}, or \emph{implicitly determined}, according to the rules 
outlined in this section.

Specifying a variable on a \code{firstprivate}, \code{lastprivate}, \code{linear}, \code{reduction}, 
or \code{copyprivate} clause of an enclosed construct causes an implicit reference to the 
variable in the enclosing construct. Specifying a variable on a \code{map} clause of an enclosed 
construct may cause an implicit reference to the variable in the enclosing construct. 
Such implicit references are also subject to the data-sharing attribute rules outlined in 
this section.

Certain variables and objects have \emph{predetermined} data-sharing attributes as follows:

\ccppspecificstart
\begin{itemize}
\item Variables appearing in \code{threadprivate} directives are threadprivate.

\item Variables with automatic storage duration that are declared in a scope inside the 
construct are private. 

\item Objects with dynamic storage duration are shared.

\item Static data members are shared.

\item The loop iteration variable(s) in the associated \plc{for-loop(s)} of a
\code{for}, \code{parallel}~\code{for}, \code{taskloop}, or \code{distribute} construct is (are) private.

\item The loop iteration variable in the associated \plc{for-loop} of a \code{simd} construct with just 
one associated \plc{for-loop} is linear with a \plc{linear-step} that is the increment of
the associated \plc{for-loop}.

\item The loop iteration variables in the associated \plc{for-loops} of a \code{simd} construct with 
multiple associated \plc{for-loops} are lastprivate. 

\item Variables with static storage duration that are declared in a scope inside the construct 
are shared.

\item If an array section is a list item in a \code{map} clause on the \code{target} construct and the array section is derived from a variable for which the type is pointer then that variable is firstprivate. 
\end{itemize}
\ccppspecificend

\fortranspecificstart
\begin{itemize}
\item Variables and common blocks appearing in \code{threadprivate} directives are 
threadprivate. 

\item The loop iteration variable(s) in the associated \plc{do-loop(s)} of a \code{do}, \code{parallel}~\code{do},
\code{taskloop}, or \code{distribute} construct is (are) private.

\item The loop iteration variable in the associated \plc{do-loop} of a \code{simd} construct with just 
one associated \plc{do-loop} is linear with a \plc{linear-step} that is the increment of
the associated \plc{do-loop}.

\item The loop iteration variables in the associated \plc{do-loops} of a \code{simd} construct with 
multiple associated \plc{do-loops} are lastprivate. 

\item A loop iteration variable for a sequential loop in a \code{parallel} or task generating construct is 
private in the innermost such construct that encloses the loop.

\item Implied-do indices and \code{forall} indices are private. 

\item Cray pointees have the same the data-sharing attribute as the storage with which their Cray 
pointers are associated.

\item Assumed-size arrays are shared.
\nopagebreak
\item An associate name preserves the association with the selector established at the 
\code{ASSOCIATE} statement.
\end{itemize}
\fortranspecificend

Variables with predetermined data-sharing attributes may not be listed in data-sharing 
attribute clauses, except for the cases listed below. For these exceptions only, listing a 
predetermined variable in a data-sharing attribute clause is allowed and overrides the 
variable's predetermined data-sharing attributes.

\ccppspecificstart
\begin{itemize}
\item The loop iteration variable(s) in the associated \plc{for-loop(s)} of a \code{for}, 
\code{parallel}~\code{for}, \code{taskloop}, or \code{distribute} construct may be listed in a \code{private} or \code{lastprivate} clause.

\item The loop iteration variable in the associated \plc{for-loop} of a \code{simd} construct with just 
one associated \plc{for-loop} may be listed in a \code{linear} clause with a
\plc{linear-step}
that is the increment of the associated \plc{for-loop}.

\item The loop iteration variables in the associated \plc{for-loops} of a \code{simd} construct with 
multiple associated \plc{for-loops} may be listed in a \code{lastprivate} clause. 

\item Variables with \code{const}-qualified type having no mutable member may be listed in a
\code{firstprivate} clause, even if they are static data members.
\end{itemize}
\ccppspecificend

\fortranspecificstart
\begin{itemize}
\item The loop iteration variable(s) in the associated \plc{do-loop(s)} of a \code{do}, 
\code{parallel}~\code{do}, \code{taskloop}, or \code{distribute}
construct may be listed in a \code{private} or \code{lastprivate} clause. 

\item The loop iteration variable in the associated \plc{d}o-loop of a \code{simd} construct with just 
one associated \plc{do-loop} may be listed in a \code{linear} clause with a \plc{linear-step}
that is the increment of the associated loop.

\item The loop iteration variables in the associated \plc{do-loops} of a \code{simd} construct with 
multiple associated \plc{do-loops} may be listed in a \code{lastprivate} clause. 

\item Variables used as loop iteration variables in sequential loops in a \code{parallel} 
or task generating construct may be listed in data-sharing clauses on the construct itself, and on 
enclosed constructs, subject to other restrictions.

\item Assumed-size arrays may be listed in a \code{shared} clause.
\end{itemize}
\fortranspecificend

Additional restrictions on the variables that may appear in individual clauses are 
described with each clause in \specref{subsec:Data-Sharing Attribute Clauses}.

Variables with \emph{explicitly determined} data-sharing attributes are those that are referenced 
in a given construct and are listed in a data-sharing attribute clause on the construct.

Variables with \emph{implicitly determined} data-sharing attributes are those that are referenced 
in a given construct, do not have predetermined data-sharing attributes, and are not 
listed in a data-sharing attribute clause on the construct.

Rules for variables with \emph{implicitly determined} data-sharing attributes are as follows:

\begin{itemize}
\item In a \code{parallel}, \code{teams}, or task generating construct, the data-sharing attributes of these variables are
determined by the \code{default} clause, if present (see
\specref{subsubsec:default clause}).

\item In a \code{parallel} construct, if no \code{default} clause is present, these variables are 
shared.

\item For constructs other than task generating constructs, if no \code{default} clause is present, these variables reference the variables with the same names that exist in the enclosing context.

\item In a \code{target} construct, variables that are not mapped after applying data-mapping attribute rules (see \specref{subsec:Data-mapping Attribute Rules and Clauses}) are firstprivate.
\end{itemize}

\cppspecificstart
\begin{itemize}
\item In an orphaned task generating
construct, if no \code{default} clause is present, formal arguments passed by reference are firstprivate.
\end{itemize}
\cppspecificend

\fortranspecificstart
\begin{itemize}
\item In an orphaned task generating
construct, if no \code{default} clause is present, dummy arguments 
are firstprivate.
\end{itemize}
\fortranspecificend

\begin{itemize}
\item In a task generating construct, if no \code{default} clause is present, a variable
for which the data-sharing attribute is not determined by the rules above
and that in the enclosing context is determined to be shared by all implicit tasks bound
to the current team is shared.

\item In a task generating construct, if no
\code{default} clause is present, a variable for which the data-sharing 
attribute is not determined by the rules above is firstprivate.
\end{itemize}

Additional restrictions on the variables for which data-sharing attributes cannot be 
implicitly determined in a task generating construct are described in
\specref{subsubsec:firstprivate clause}.







\pagebreak

\subsubsection{Data-sharing Attribute Rules for Variables Referenced in a Region but not in a Construct}
\label{subsubsec:Data-sharing Attribute Rules for Variables Referenced in a Region but not in a Construct}
The data-sharing attributes of variables that are referenced in a region, but not in a 
construct, are determined as follows: 

\ccppspecificstart
\begin{itemize}
\item Variables with static storage duration that are declared in called routines in the region 
are shared.

\item File-scope or namespace-scope variables referenced in called routines in the region 
are shared unless they appear in a \code{threadprivate} directive.

\item Objects with dynamic storage duration are shared.

\item Static data members are shared unless they appear in a \code{threadprivate} directive.

\item In C++, formal arguments of called routines in the region that are passed by reference have the same data-sharing attributes as the associated actual arguments. 

\item Other variables declared in called routines in the region are private.
\end{itemize}
\ccppspecificend

\fortranspecificstart
\begin{itemize}
\item Local variables declared in called routines in the region and that have the \code{save} 
attribute, or that are data initialized, are shared unless they appear in a 
\code{threadprivate} directive.

\item Variables belonging to common blocks, or accessed by host or use association, and referenced in called routines in the region are shared unless they appear in a \code{threadprivate} directive. 

\item Dummy arguments of called routines in the region that have the
    \code{VALUE} attribute are private.

\item Dummy arguments of called routines in the region that do not have the
    \code{VALUE} attribute are private if the associated actual argument is not
    shared.

\item Dummy arguments of called routines in the region that do not have the
\code{VALUE} attribute are shared if the actual argument is shared and it
is a scalar variable, structure, an array that is not a pointer or
assumed-shape array, or a simply contiguous array section.  Otherwise, the
data-sharing attribute of the dummy argument is implementation-defined if
the associated actual argument is shared.

\item Cray pointees have the same data-sharing attribute as the storage with which their Cray pointers are associated.

\item Implied-do indices, \code{forall} indices, and other local variables declared in called 
routines in the region are private. 

\end{itemize}


\fortranspecificend









\subsection{\code{threadprivate} Directive}
\index{threadprivate@{\code{threadprivate}}}
\index{directives!threadprivate@{\code{threadprivate}}}
\label{subsec:threadprivate Directive}
\summary
The \code{threadprivate} directive specifies that variables are replicated, with each thread 
having its own copy. The \code{threadprivate} directive is a declarative directive.
\syntax
\ccppspecificstart
The syntax of the \code{threadprivate} directive is as follows:

\begin{boxedcode}
\#pragma omp threadprivate(\plc{list}) \plc{new-line}
\end{boxedcode}

where \plc{list} is a comma-separated list of file-scope, namespace-scope, or static 
block-scope variables that do not have incomplete types.
\ccppspecificend

\fortranspecificstart
The syntax of the \code{threadprivate} directive is as follows:

\begin{boxedcode}
!\$omp threadprivate(\plc{list})
\end{boxedcode}

where \plc{list} is a comma-separated list of named variables and named common blocks. 
Common block names must appear between slashes.
\fortranspecificend

\descr
Each copy of a threadprivate variable is initialized once, in the manner specified by the 
program, but at an unspecified point in the program prior to the first reference to that 
copy. The storage of all copies of a threadprivate variable is freed according to how 
static variables are handled in the base language, but at an unspecified point in the 
program.

A program in which a thread references another thread’s copy of a threadprivate variable 
is non-conforming.

The content of a threadprivate variable can change across a task scheduling point if the 
executing thread switches to another task that modifies the variable. For more details on 
task scheduling, see 
\specref{sec:Execution Model} and 
\specref{sec:Tasking Constructs}.

In \code{parallel} regions, references by the master thread will be to the copy of the 
variable in the thread that encountered the \code{parallel} region. 

During a sequential part references will be to the initial thread’s copy of the variable. 
The values of data in the initial thread’s copy of a threadprivate variable are guaranteed 
to persist between any two consecutive references to the variable in the program. 

The values of data in the threadprivate variables of non-initial threads 
are guaranteed to persist between two consecutive active \code{parallel} 
regions only if all of the following conditions hold:

\begin{itemize}  % L0 vvvvvvvvvvvvvvvvvvvvvvvvv
\item Neither \code{parallel} region is nested inside another explicit \code{parallel} region. 

\item The number of threads used to execute both \code{parallel} regions is the same. 

\item The thread affinity policies used to execute both \code{parallel} regions are the same. 

\item The value of the \plc{dyn-var} internal control variable in the enclosing task region is \plc{false} 
at entry to both \code{parallel} regions.
\end{itemize} % L0 ^^^^^^^^^^^^^^^^^^^^^^

If these conditions all hold, and if a threadprivate variable is referenced in both regions, 
then threads with the same thread number in their respective regions will reference the 
same copy of that variable.

\ccppspecificstart
If the above conditions hold, the storage duration, lifetime, and value of a thread’s copy 
of a threadprivate variable that does not appear in any \code{copyin} clause on the second 
region will be retained. Otherwise, the storage duration, lifetime, and value of a thread’s 
copy of the variable in the second region is unspecified.

If the value of a variable referenced in an explicit initializer of a threadprivate variable 
is modified prior to the first reference to any instance of the threadprivate variable, then 
the behavior is unspecified. 
\ccppspecificend

\cppspecificstart
The order in which any constructors for different threadprivate variables of class type 
are called is unspecified. The order in which any destructors for different threadprivate 
variables of class type are called is unspecified. 
\cppspecificend

\fortranspecificstart
A variable is affected by a \code{copyin} clause if the variable appears in the \code{copyin} clause 
or it is in a common block that appears in the \code{copyin} clause. 

If the above conditions hold, the definition, association, or allocation status of a thread’s 
copy of a threadprivate variable or a variable in a threadprivate common 
block, that is not affected by any \code{copyin} clause that appears on the second region, will 
be retained. Otherwise, the definition and association status of a thread’s copy of the 
variable in the second region are undefined, and the allocation status of an allocatable 
variable will be implementation defined. 

If a threadprivate variable or a variable in a threadprivate common block is 
not affected by any \code{copyin} clause that appears on the first \code{parallel} region in which 
it is referenced, the variable or any subobject of the variable is initially defined or 
undefined according to the following rules:

\begin{itemize} % L0 vvvvvvvvvvvvvvvvvvvvvv
\item If it has the \code{ALLOCATABLE} attribute, each copy created will have an initial 
allocation status of unallocated.

\item If it has the \code{POINTER} attribute:
\begin{itemize} % L1 vvvvvvvvvvvvvvvvvvvvvv
\item if it has an initial association status of disassociated, either through explicit 
initialization or default initialization, each copy created will have an association 
status of disassociated;
\item otherwise, each copy created will have an association status of undefined.
\end{itemize} % l1 ^^^^^^^^^^^^^^^^^^^^

\item If it does not have either the \code{POINTER} or the \code{ALLOCATABLE} attribute:

\begin{samepage}\begin{itemize} %L1 vvvvvvvvvvvvvvv
\item if it is initially defined, either through explicit initialization or default 
initialization, each copy created is so defined;

\item otherwise, each copy created is undefined.
\end{itemize} % L1 ^^^^^^^^^^^^^^^^^
\end{samepage}

\end{itemize} % L0 ^^^^^^^^^^^^^^^^^^^^
\fortranspecificend

\restrictions
The restrictions to the \code{threadprivate} directive are as follows:

\begin{itemize} % L0 vvvvvvvvvvvvvvv
\item A threadprivate variable must not appear in any clause except the \code{copyin}, 
\code{copyprivate}, \code{schedule}, \code{num\_threads}, \code{thread\_limit}, and \code{if} clauses.

\item A program in which an untied task accesses threadprivate storage is non-conforming.

\ccppspecificstart
\item A variable that is part of another variable (as an array or structure element) cannot 
appear in a \code{threadprivate} clause unless it is a static data member of a C++ 
class.

\item A \code{threadprivate} directive for file-scope variables must appear outside any 
definition or declaration, and must lexically precede all references to any of the 
variables in its list.

\item A \code{threadprivate} directive for namespace-scope variables must appear outside 
any definition or declaration other than the namespace definition itself, and must 
lexically precede all references to any of the variables in its list.

\item Each variable in the list of a \code{threadprivate} directive at file, namespace, or class 
scope must refer to a variable declaration at file, namespace, or class scope that 
lexically precedes the directive.

\item A \code{threadprivate} directive for static block-scope variables must appear in the 
scope of the variable and not in a nested scope. The directive must lexically precede 
all references to any of the variables in its list.

\item Each variable in the list of a \code{threadprivate} directive in block scope must refer to 
a variable declaration in the same scope that lexically precedes the directive. The 
variable declaration must use the static storage-class specifier.

\item If a variable is specified in a \code{threadprivate} directive in one translation unit, it 
must be specified in a \code{threadprivate} directive in every translation unit in which 
it is declared.

\item The address of a threadprivate variable is not an address constant.
\ccppspecificend

\cppspecificstart
\item A \code{threadprivate} directive for static class member variables must appear in the 
class definition, in the same scope in which the member variables are declared, and 
must lexically precede all references to any of the variables in its list.

\item A threadprivate variable must not have an incomplete type or a reference type.

\item A threadprivate variable with class type must have:

\begin{itemize} % L1 vvvvvvvvvvvvvv
\item an accessible, unambiguous default constructor in case of default initialization 
without a given initializer;

\item an accessible, unambiguous constructor accepting the given argument in case of 
direct initialization;

\item an accessible, unambiguous copy constructor in case of copy initialization with an 
explicit initializer
\end{itemize} % L1 ^^^^^^^^^^^^^
\cppspecificend

\end{itemize} % L0 ^^^^^^^^^^^^^^

\fortranspecificstart
\begin{itemize} % L0 vvvvvvvvvvvvvvv
\item A variable that is part of another variable (as an array or structure element) cannot 
appear in a \code{threadprivate} clause.

\item The \code{threadprivate} directive must appear in the declaration section of a scoping 
unit in which the common block or variable is declared. Although variables in 
common blocks can be accessed by use association or host association, common 
block names cannot. This means that a common block name specified in a 
\code{threadprivate} directive must be declared to be a common block in the same 
scoping unit in which the \code{threadprivate} directive appears. 

\item If a \code{threadprivate} directive specifying a common block name appears in one 
program unit, then such a directive must also appear in every other program unit that 
contains a \code{COMMON} statement specifying the same name. It must appear after the last 
such \code{COMMON} statement in the program unit.

\item If a threadprivate variable or a threadprivate common block is declared 
with the \code{BIND} attribute, the corresponding C entities must also be specified in a 
\code{threadprivate} directive in the C program.

\item A blank common block cannot appear in a \code{threadprivate} directive.

\item A variable can only appear in a \code{threadprivate} directive in the scope in which it 
is declared. It must not be an element of a common block or appear in an 
\code{EQUIVALENCE} statement.

\item A variable that appears in a \code{threadprivate} directive must be declared in the 
scope of a module or have the \code{SAVE} attribute, either explicitly or implicitly.
\end{itemize} % L0 ^^^^^^^^^^^^^^^^^^^
\fortranspecificend

\crossreferences
\begin{itemize}
\item \plc{dyn-var} ICV, see 
\specref{sec:Internal Control Variables}.

\item Number of threads used to execute a \code{parallel} region, see 
\specref{subsec:Determining the Number of Threads for a parallel Region}.

\item \code{copyin} clause, see 
\specref{subsubsec:copyin clause}.
\end{itemize}








\subsection{Data-Sharing Attribute Clauses}
\label{subsec:Data-Sharing Attribute Clauses}
\index{data-sharing attribute clauses}
\index{attribute clauses}
\index{clauses!data-sharing}
\index{clauses!attribute data-sharing}
Several constructs accept clauses that allow a user to control the data-sharing attributes 
of variables referenced in the construct. Data-sharing attribute clauses apply only to 
variables for which the names are visible in the construct on which the clause appears.

Not all of the clauses listed in this section are valid on all directives. The set of clauses 
that is valid on a particular directive is described with the directive.

Most of the clauses accept a comma-separated list of list items (see 
\specref{sec:Directive Format}). 
All list items appearing in a clause must be visible, according to the scoping rules 
of the base language. With the exception of the \code{default} clause, clauses may be 
repeated as needed. A list item that specifies a given variable may not appear in more 
than one clause on the same directive, except that a variable may be specified in both 
\code{firstprivate} and \code{lastprivate} clauses.

The reduction data-sharing clauses are explained in Section \ref{subsec:Reduction Clauses}.

\cppspecificstart
If a variable referenced in a data-sharing attribute clause has a type derived from a 
template, and there are no other references to that variable in the program, then any 
behavior related to that variable is unspecified. 
\cppspecificend

\fortranspecificstart
When a named common block appears in a \code{private}, \code{firstprivate}, 
\code{lastprivate}, or \code{shared} clause of a directive, none of its members may be declared 
in another data-sharing attribute clause in that directive. When individual members of a common block appear in a \code{private}, \code{firstprivate}, 
\code{lastprivate}, \code{reduction}, or \code{linear} clause of a directive, the storage of the specified variables is no longer Fortran associated with the storage of the common block itself.
\fortranspecificend










\subsubsection{\code{default} Clause}
\label{subsubsec:default clause}
\index{default@{\code{default}}}
\index{clauses!default@{\code{default}}}
\summary
The \code{default} clause explicitly determines the data-sharing attributes of variables that 
are referenced in a \code{parallel}, \code{teams}, or task generating construct
and would otherwise be implicitly determined (see
\specref{subsubsec:Data-sharing Attribute Rules for Variables Referenced in a Construct}).

\syntax
\ccppspecificstart
The syntax of the \code{default} clause is as follows:

\begin{boxedcode}
default(shared \textnormal{|} none)
\end{boxedcode}
\ccppspecificend

\fortranspecificstart
The syntax of the \code{default} clause is as follows:

\begin{boxedcode}
default(private \textnormal{|} firstprivate \textnormal{|} shared \textnormal{|} none)
\end{boxedcode}
\fortranspecificend

\descr
The \code{default(shared)} clause causes all variables referenced in the construct that 
have implicitly determined data-sharing attributes to be shared.

\fortranspecificstart
The \code{default(firstprivate)} clause causes all variables in the construct that have 
implicitly determined data-sharing attributes to be firstprivate.

The \code{default(private)} clause causes all variables referenced in the construct that 
have implicitly determined data-sharing attributes to be private.
\fortranspecificend

The \code{default(none)} clause requires that each variable that is referenced in the 
construct, and that does not have a predetermined data-sharing attribute, must have its 
data-sharing attribute explicitly determined by being listed in a data-sharing attribute 
clause. 

\restrictions
The restrictions to the \code{default} clause are as follows:

\begin{itemize}
\item Only a single \code{default} clause may be specified on a 
\code{parallel}, \code{task}, \code{taskloop} or \code{teams} directive.
\end{itemize}









\subsubsection{\code{shared} Clause}
\label{subsubsec:shared clause}
\index{shared@{\code{shared}}}
\index{clauses!shared@{\code{shared}}}
\summary
The \code{shared} clause declares one or more list items to be shared by tasks generated by 
a \code{parallel}, \code{teams}, or task generating construct.

\syntax
The syntax of the \code{shared} clause is as follows:

\begin{boxedcode}
shared(\plc{list})
\end{boxedcode}

\descr
All references to a list item within a task refer to the storage area of the original variable 
at the point the directive was encountered. 

The programmer must ensure, by adding proper synchronization, that 
storage shared by an explicit task region does not reach the end of its lifetime before 
the explicit task region completes its execution. 


\fortranspecificstart
The association status of a shared pointer becomes undefined upon entry to and on exit 
from the \code{parallel}, \code{teams}, or task generating construct if it
is associated with a target or a  subobject of a target that is in a \code{private},
\code{firstprivate}, \code{lastprivate}, or \code{reduction} clause in the construct.


\notestart
\noteheader -- Passing a shared variable to a procedure may result in the use of
temporary storage in place of the actual argument when the corresponding dummy
argument does not have the \code{VALUE} attribute and its data-sharing attribute
is implementation-defined as per the rules in
\specref{subsubsec:Data-sharing Attribute Rules for Variables Referenced in a Region but not in a Construct}.  
These conditions effectively result in references to, and definitions of, the
temporary storage during the procedure reference.  Furthermore, the value of
the shared variable is copied into the intervening temporary storage before the procedure
reference when the dummy argument does not have the \code{INTENT(OUT)}
attribute, and back out of the temporary storage into the shared variable when
the dummy argument does not have the \code{INTENT(IN)} attribute.  Any
references to (or definitions of) the shared storage that is associated with
the dummy argument by any other task must be synchronized with
the procedure reference to avoid possible race conditions.

\noteend
\medskip
\fortranspecificend


\restrictions
The restrictions for the \code{shared} clause are as follows:
\begin{itemize}

\cspecificstart
\item A variable that is part of another variable (as an array or structure element) cannot appear in a shared clause.
\cspecificend

\cppspecificstart
\item A variable that is part of another variable (as an array or structure
  element) cannot appear in a \code{shared} clause except if the \code{shared}
  clause is associated with a construct within a class non-static member
  function and the variable is an accessible data member of the object for
  which the non-static member function is invoked.
\cppspecificend

\fortranspecificstart
\item A variable that is part of another variable (as an array or structure element) cannot appear in a shared clause.
\fortranspecificend

\end{itemize}







\subsubsection{\code{private} Clause}
\index{private@{\code{private}}}
\index{clauses!private@{\code{private}}}
\label{subsubsec:private clause}
\summary
The \code{private} clause declares one or more list items to be private to a task or to a 
SIMD lane.

\syntax
The syntax of the private clause is as follows:

\begin{boxedcode}
private(\plc{list})
\end{boxedcode}

\descr
Each task that references a list item that appears in a \code{private} clause in any statement 
in the construct receives a new list item. Each SIMD lane used in a \code{simd} construct that 
references a list item that appears in a private clause in any statement in the construct 
receives a new list item. Language-specific attributes for new list items are derived from 
the corresponding original list item. Inside the construct, all references to the original 
list item are replaced by references to the new list item. In the rest of the region, it is 
unspecified whether references are to the new list item or the original list item. 

\cppspecificstart
If the construct is contained in a member function, it is unspecified anywhere in the region if accesses through the implicit \code{this} pointer refer to the new list item or the original list item.
\cppspecificend

Therefore, if an attempt is made to reference the original item, its value after the region 
is also unspecified. If a SIMD construct or a task does not reference a list item that 
appears in a \code{private} clause, it is unspecified whether SIMD lanes or the task receive 
a new list item. 

The value and/or allocation status of the original list item will change only: 

\begin{itemize}
\item if accessed and modified via pointer, 

\item if possibly accessed in the region but outside of the construct, 

\item as a side effect of directives or clauses, or

\newpage %% HACK
\fortranspecificstart
\item if accessed and modified via construct association.
\fortranspecificend
\end{itemize}

List items that appear in a \code{private}, \code{firstprivate}, or
\code{reduction} clause in a \code{parallel} construct may also appear
in a \code{private} clause in an enclosed \code{parallel},
worksharing, \code{task}, \code{taskloop}, \code{simd}, or
\code{target} construct.

List items that appear in a \code{private} or \code{firstprivate}
clause in a \code{task} or \code{taskloop} construct may also appear in a \code{private}
clause in an enclosed \code{parallel}, \code{task}, \code{taskloop}, \code{simd}, or
\code{target} construct.

List items that appear in a \code{private}, \code{firstprivate},
\code{lastprivate}, or \code{reduction} clause in a worksharing
construct may also appear in a \code{private} clause in an enclosed
\code{parallel}, \code{task}, \code{simd}, or \code{target} construct.

\ccppspecificstart
A new list item of the same type, with automatic storage duration, is allocated for the 
construct. The storage and thus lifetime of these list items lasts until the block in which 
they are created exits. The size and alignment of the new list item are determined by the 
type of the variable. This allocation occurs once for each task generated by the construct 
and once for each SIMD lane used by the construct.

The new list item is initialized, or has an undefined initial value, as if it had been locally 
declared without an initializer. 
\ccppspecificend

\cppspecificstart
If the type of a list item is a reference to a type \plc{T} then the type will be considered to be 
\plc{T} for all purposes of this clause.

The order in which any default constructors for different private variables of class type 
are called is unspecified. The order in which any destructors for different private 
variables of class type are called is unspecified.
\cppspecificend

\fortranspecificstart 
If any statement of the construct references a list item, a new list
item of the same type and type parameters is allocated. This
allocation occurs once for each task generated by the construct and
once for each SIMD lane used by the construct. The initial value of
the new list item is undefined. The initial status of a private
pointer is undefined.

For a list item or the subobject of a list item with the \code{ALLOCATABLE} attribute:

\begin{itemize}
\item if the allocation status is unallocated, the new list item or the subobject
of the new list item will have an initial allocation status of unallocated.

\item if the allocation status is allocated, the new list item or the subobject of
the new list item will have an initial allocation status of allocated.

\item If the new list item or the subobject of the new list item is an array, its bounds will be 
the same as those of the original list item or the subobject of the original list item.
\end{itemize}

A list item that appears in a \code{private} clause may be storage-associated with other 
variables when the \code{private} clause is encountered. Storage association may exist 
because of constructs such as \code{EQUIVALENCE} or \code{COMMON}. If \plc{A} is a variable appearing 
in a \code{private} clause on a construct and \plc{B} is a variable that is storage-associated with \plc{A}, then:

\begin{itemize}
\item The contents, allocation, and association status of \plc{B} are undefined on entry to the region.

\item Any definition of \plc{A}, or of its allocation or association status, causes the contents, 
allocation, and association status of \plc{B} to become undefined. 

\item Any definition of \plc{B}, or of its allocation or association status, causes the contents, 
allocation, and association status of \plc{A} to become undefined. 
\end{itemize}

A list item that appears in a \code{private} clause may be a selector of an \code{ASSOCIATE} 
construct. If the construct association is established prior to a \code{parallel} region, the 
association between the associate name and the original list item will be retained in the 
region.

Finalization of a list item of a finalizable type or subojects of a
list item of a finalizable type occurs at the end of the region. The
order in which any final subroutines for different variables of a
finalizable type are called is unspecified.
\fortranspecificend

\restrictions
The restrictions to the \code{private} clause are as follows:

\begin{itemize}
\cspecificstart
\item A variable that is part of another variable (as an array or structure element) cannot 
appear in a \code{private} clause.
\cspecificend

\cppspecificstart
\item A variable that is part of another variable (as an array or structure element) cannot 
appear in a \code{private} clause except if the \code{private} clause is associated with a construct within a class non-static member function and the variable is an accessible data member of the object for which the non-static member function is invoked.

\item A variable of class type (or array thereof) that appears in a \code{private} clause requires 
an accessible, unambiguous default constructor for the class type. 
\cppspecificend

\ccppspecificstart
\item A variable that appears in a \code{private} clause must not have a \code{const}-qualified type 
unless it is of class type with a \code{mutable} member. This restriction does not apply to 
the \code{firstprivate} clause.

\item A variable that appears in a \code{private} clause must not have an incomplete type or be a reference to an incomplete type.
\ccppspecificend


\fortranspecificstart
\item A variable that is part of another variable (as an array or structure element) cannot 
appear in a \code{private} clause.

\item A variable that appears in a \code{private} clause must either be definable, or an 
allocatable variable. This restriction does not apply to the \code{firstprivate} clause.

\item Variables that appear in namelist statements, in variable format expressions, and in 
expressions for statement function definitions, may not appear in a \code{private} clause.

\item Pointers with the \code{INTENT(IN)} attribute may not appear in a \code{private} clause. This 
restriction does not apply to the \code{firstprivate} clause. 
\fortranspecificend
\end{itemize}










\subsubsection{\code{firstprivate} Clause}
\label{subsubsec:firstprivate clause}
\index{firstprivate@{\code{firstprivate}}}
\index{clauses!firstprivate@{\code{firstprivate}}}
\summary
The \code{firstprivate} clause declares one or more list items to be private to a task, and 
initializes each of them with the value that the corresponding original item has when the 
construct is encountered. 

\syntax
The syntax of the \code{firstprivate} clause is as follows:

\begin{boxedcode}
firstprivate(\plc{list})
\end{boxedcode}

\descr
The \code{firstprivate} clause provides a superset of the functionality provided by the 
\code{private} clause. 

A list item that appears in a \code{firstprivate} clause is subject to the \code{private} clause 
semantics described in 
\specref{subsubsec:private clause}, 
except as noted. In addition, the 
new list item is initialized from the original list item existing before the construct. The 
initialization of the new list item is done once for each task that references the list item 
in any statement in the construct. The initialization is done prior to the execution of the 
construct.

For a \code{firstprivate} clause on a \code{parallel}, \code{task},
\code{taskloop}, \code{target}, or \code{teams} construct, the
initial value of the new list item is the value of the original list
item that exists immediately prior to the construct in the task region
where the construct is encountered. For a \code{firstprivate} clause
on a worksharing construct, the initial value of the new list item for
each implicit task of the threads that execute the worksharing
construct is the value of the original list item that exists in the
implicit task immediately prior to the point in time that the
worksharing construct is encountered.

To avoid race conditions, concurrent updates of the original list item must be 
synchronized with the read of the original list item that occurs as a result of the 
\code{firstprivate} clause.

If a list item appears in both \code{firstprivate} and \code{lastprivate} clauses, the update 
required for \code{lastprivate} occurs after all the initializations for \code{firstprivate}.

\ccppspecificstart
For variables of non-array type, the initialization occurs by copy assignment. For an 
array of elements of non-array type, each element is initialized as if by assignment from 
an element of the original array to the corresponding element of the new array. 
\ccppspecificend

\cppspecificstart
For variables of class type, a copy constructor is invoked to perform the initialization. 
The order in which copy constructors for different variables of class type are called is 
unspecified. 
\cppspecificend

\fortranspecificstart
If the original list item does not have the \code{POINTER} attribute, initialization of the new 
list items occurs as if by intrinsic assignment, unless the original list item has the 
allocation status of unallocated, in which case the new list items will have the
same status.

If the original list item has the \code{POINTER} attribute, the new list items receive the same 
association status of the original list item as if by pointer assignment.
\fortranspecificend

\restrictions
The restrictions to the \code{firstprivate} clause are as follows:

\begin{itemize}
\item A list item that is private within a \code{parallel} region must
not appear in a \code{firstprivate} clause on a worksharing construct
if any of the worksharing regions arising
from the worksharing construct ever bind to any of the
\code{parallel} regions arising from the \code{parallel} construct.

\item A list item that is private within a \code{teams} region must not appear in a 
\code{firstprivate} clause on a \code{distribute} construct if any of the \code{distribute} 
regions arising from the \code{distribute} construct ever bind to any of the \code{teams} 
regions arising from the \code{teams} construct.

\item A list item that appears in a \code{reduction} clause of a \code{parallel} 
construct must not appear in a \code{firstprivate} clause on a worksharing, \code{task}, 
or \code{taskloop} construct if any of the worksharing or task regions arising from 
the worksharing, \code{task}, or \code{taskloop} construct ever bind to any of the 
\code{parallel} regions arising from the \code{parallel} construct.

\item A list item that appears in a \code{reduction} clause of a \code{teams} construct must not 
appear in a \code{firstprivate} clause on a \code{distribute} construct if any of the 
\code{distribute} regions arising from the \code{distribute} construct ever bind to any of 
the \code{teams} regions arising from the \code{teams} construct. 

\item A list item that appears in a \code{reduction} clause of a worksharing construct must not 
appear in a \code{firstprivate} clause in a \code{task} construct encountered during execution 
of any of the worksharing regions arising from the worksharing construct.

\cppspecificstart
\item A variable of class type (or array thereof) that appears in a \code{firstprivate} clause 
requires an accessible, unambiguous copy constructor for the class type.
\cppspecificend

\ccppspecificstart
\item A variable that appears in a \code{firstprivate} clause must not have an incomplete C/C++ type or be a reference to an incomplete type.

\item If a list item in a \code{firstprivate} clause on a worksharing
construct has a reference type then it must bind to the same object for all threads of the team.
\ccppspecificend

\fortranspecificstart
\item Variables that appear in namelist statements, in variable format expressions, or in 
expressions for statement function definitions, may not appear in a \code{firstprivate} 
clause. 
\fortranspecificend
\end{itemize}










\subsubsection{\code{lastprivate} Clause}
\index{lastprivate@{\code{lastprivate}}}
\index{clauses!lastprivate@{\code{lastprivate}}}
\label{subsubsec:lastprivate clause}
\summary
The \code{lastprivate} clause declares one or more list items to be private to an implicit 
task or to a SIMD lane, and causes the corresponding original list item to be updated 
after the end of the region. 

\syntax
The syntax of the \code{lastprivate} clause is as follows:

\begin{boxedcode}
lastprivate(\plc{[ lastprivate-modifier}:\plc{] list})
\end{boxedcode}

where \plc{lastprivate-modifier} is:
\begin{indentedcodelist}
conditional
\end{indentedcodelist}

\descr
The \code{lastprivate} clause provides a superset of the functionality provided by the 
\code{private} clause.

A list item that appears in a \code{lastprivate} clause is subject to the \code{private} clause 
semantics described in 
\specref{subsubsec:private clause}. 
In addition, when a 
\code{lastprivate} clause without the \code{conditional} modifier appears on the directive that identifies a worksharing construct 
or a SIMD construct, the value of each new list item from the sequentially last iteration 
of the associated loops, or the lexically last \code{section} construct, is assigned to the 
original list item. When the \code{conditional} modifier appears on the clause,
if an assignment to a list item is encountered in the construct then the
original list item is assigned the value that is assigned to the new list item
in the sequentially last iteration or lexically last section in which such an
assignment is encountered.


\ccppspecificstart
For an array of elements of non-array type, each element is assigned to the 
corresponding element of the original array.
\ccppspecificend
\bigskip

\fortranspecificstart
If the original list item does not have the \code{POINTER} attribute, its update occurs as if by 
intrinsic assignment.

If the original list item has the \code{POINTER} attribute, its update occurs as if by pointer 
assignment.
\fortranspecificend

When the \code{conditional} modifier does not appear on the \code{lastprivate} clause, list items that are not
assigned a value by the sequentially last iteration of the loops, or by the
lexically last \code{section} construct, have unspecified values after the
construct.  Unassigned subcomponents also have unspecified values after the
construct.

The original list item becomes defined at the end of the construct if there is an implicit 
barrier at that point. To avoid race conditions, concurrent reads or updates of the original 
list item must be synchronized with the update of the original list item that occurs as a 
result of the \code{lastprivate} clause.

If the \code{lastprivate} clause is used on a construct that does not end with
an implicit barrier, accesses to the original list item may create a data race.
To avoid this, if an assignment to the original list item occurs
then synchronization must be inserted to ensure that the assignment completes and
the original list item is flushed to memory.

If a list item appears in both \code{firstprivate} and \code{lastprivate} clauses, the update 
required for \code{lastprivate} occurs after all initializations for \code{firstprivate}.

\restrictions
The restrictions to the \code{lastprivate} clause are as follows:

\begin{itemize}
\item A list item that is private within a \code{parallel} region, or
that appears in the \code{reduction} clause of a \code{parallel}
construct, must not appear in a \code{lastprivate} clause on a
worksharing construct if any of the corresponding
worksharing regions ever binds to any of the corresponding
\code{parallel} regions.

\item If a list item that appears in a \code{lastprivate} clause with the
\code{conditional} modifier is modified in the region by an assignment
outside the construct or not to the list item then the value assigned to
the original list item is unspecified.

\item A list item that appears in a \code{lastprivate} clause with the
\code{conditional} modifier must be a scalar variable.

\cppspecificstart
\item A variable of class type (or array thereof) that appears in a \code{lastprivate} clause 
requires an accessible, unambiguous default constructor for the class type, unless the 
list item is also specified in a \code{firstprivate} clause. 

\item A variable of class type (or array thereof) that appears in a \code{lastprivate} clause 
requires an accessible, unambiguous copy assignment operator for the class type. The 
order in which copy assignment operators for different variables of class type are 
called is unspecified.
\cppspecificend

\ccppspecificstart
\item A variable that appears in a \code{lastprivate} clause must not have a \code{const}-qualified 
type unless it is of class type with a \code{mutable} member. 

\item A variable that appears in a \code{lastprivate} clause must not have an incomplete C/C++ type or be a reference to an incomplete type.

\item If a list item in a \code{lastprivate} clause on a worksharing
construct has a reference type then it must bind to the same object for all threads of the team.
\ccppspecificend

\fortranspecificstart
\item A variable that appears in a \code{lastprivate} clause must be definable.

\item If the original list item has the \code{ALLOCATABLE} attribute, the
    corresponding list item whose value is assigned to the original list item must have an allocation status of allocated upon exit from
    the sequentially last iteration or lexically last \code{section} construct.

\item Variables that appear in namelist statements, in variable format expressions, or in 
expressions for statement function definitions, may not appear in a \code{lastprivate} 
clause.
\fortranspecificend
\end{itemize}










\subsubsection{\code{linear} Clause}
\index{linear@{\code{linear}}}
\index{clauses!linear@{\code{linear}}}
\label{subsubsec:linear clause}
\summary
The \code{linear} clause declares one or more list items to be private to a SIMD lane and to 
have a linear relationship with respect to the iteration space of a loop.

\syntax


\cspecificstart
The syntax of the \code{linear} clause is as follows:
\begin{boxedcode}
linear(\plc{linear-list[ }:\plc{ linear-step]})
\end{boxedcode}
where \plc{linear-list} is one of the following
\vspace{-2ex} %% HACK
\begin{indentedcodelist}
\plc{list}
\plc{modifier}(\plc{list})
\end{indentedcodelist}
where  \plc{modifier} is one of the following:
\vspace{-2ex} %% HACK
\begin{indentedcodelist}
val
\end{indentedcodelist}
\cspecificend

\cppspecificstart
The syntax of the \code{linear} clause is as follows:
\begin{boxedcode}
linear(\plc{linear-list[ }:\plc{ linear-step]})
\end{boxedcode}
where \plc{linear-list} is one of the following
\vspace{-2ex} %% HACK
\begin{indentedcodelist}
\plc{list}
\plc{modifier}(\plc{list})
\end{indentedcodelist}
where  \plc{modifier} is one of the following:
\vspace{-2ex} %% HACK
\begin{indentedcodelist}
ref
val
uval
\end{indentedcodelist}
\cppspecificend

\fortranspecificstart
The syntax of the \code{linear} clause is as follows:
\begin{boxedcode}
linear(\plc{linear-list[ }:\plc{ linear-step]})
\end{boxedcode}
where \plc{linear-list} is one of the following
\vspace{-2ex} %% HACK
\begin{indentedcodelist}
\plc{list}
\plc{modifier}(\plc{list})
\end{indentedcodelist}
where  \plc{modifier} is one of the following:
\vspace{-2ex} %% HACK
\begin{indentedcodelist}
ref
val
uval
\end{indentedcodelist}
\fortranspecificend


\descr
The \code{linear} clause provides a superset of the functionality provided by the \code{private} clause.
A list item that appears in a \code{linear} clause is subject to the \code{private} clause semantics described
in \specref{subsubsec:private clause} except as noted.
If \plc{linear-step} is not specified, it is assumed to be 1.

When a \code{linear} clause is specified on a construct, the value of the new list item on each iteration of the associated loop(s) corresponds to the value of the original list item before entering the construct plus the logical number of the iteration times \plc{linear-step}. 
The value corresponding to the sequentially last iteration of the associated loop(s) is assigned to the original list item.

When a \code{linear} clause is specified on a declarative directive, all list items must be formal parameters (or, in Fortran, dummy arguments) of a function that will be invoked concurrently on each SIMD lane.
If no \plc{modifier} is specified or the \code{val} or \code{uval} modifier is specified, the value of each list item on each lane corresponds to the value of the list item upon entry to the function plus the logical number of the lane times \plc{linear-step}.
If the \code{uval} modifier is specified, each invocation uses the same storage location for each SIMD lane; this storage location is updated with the final value of the logically last lane.
If the \code{ref} modifier is specified, the storage location of each list item on each lane corresponds to an array at the storage location upon entry to the function indexed by the logical number of the lane times \plc{linear-step}.


\restrictions
\begin{itemize}
\item The \plc{linear-step} expression must be invariant during the execution of the region 
associated with the construct. Otherwise, the execution results in unspecified 
behavior.

\item A \plc{list-item} cannot appear in more than one \code{linear} clause.

\item A \plc{list-item} that appears in a \code{linear} clause cannot appear in any other data-sharing 
attribute clause. 

\cspecificstart
\item A \plc{list-item} that appears in a \code{linear} clause must be of integral or pointer type.
\cspecificend

\cppspecificstart
\item A \plc{list-item} that appears in a \code{linear} clause without the \code{ref} modifier must be of integral or pointer type, or must be a reference to an integral or pointer type. 
\item The \code{ref} or \code{uval} modifier can only be used if the \plc{list-item} is of a reference type.
\item If a list item in a \code{linear} clause on a worksharing
construct has a reference type then it must bind to the same object for all threads of the team.
\item If the list item is of a reference type and the \code{ref} modifier is not specified and if any write to the list item occurs before any read of the list item then the result is unspecified.
\cppspecificend

\fortranspecificstart
\item A \plc{list-item} that appears in a \code{linear} clause without the \code{ref} modifier must be of type \code{integer}.
\item The \code{ref} or \code{uval} modifier can only be used if the \plc{list-item} is a dummy argument without the \code{VALUE} attribute.
\item Variables that have the \code{POINTER} attribute and Cray pointers may not appear in a linear clause. 
\item The list item with the \code{ALLOCATABLE} attribute in the sequentially last iteration must have an allocation status of allocated upon exit from that iteration. 
\item If the list item is a dummy argument without the \code{VALUE} attribute and the \code{ref} modifier is not specified and if any write to the list item occurs before any read of the list item then the result is unspecified.
\item A common block name cannot appear in a \code{linear} clause.
\fortranspecificend
\end{itemize}










\subsection{Reduction Clauses}
\index{reduction clauses}
\label{subsec:Reduction Clauses}
The reduction clauses can be used to perform some forms of recurrence
calculations (involving mathematically associative and commutative operators)
in parallel.

Reduction clauses include reduction scoping clauses and reduction participating
clauses. Reduction scoping clauses define the region in which a reduction is
computed. Reduction participating clauses define the participants in the
reduction.

Reduction clauses specify a \plc{reduction-identifier} and one or more list
items.










\subsubsection{Properties Common To All Reduction Clauses}
\label{subsubsec:Properties Common To All Reduction Clauses}

\syntax
The syntax of a \plc{reduction-identifier} is defined as follows:
\cspecificstart % L1 vvvvv
A \plc{reduction-identifier} is either an \plc{identifier} or one of the following operators:
\code{+},
\code{-},
\code{*},
\code{\&},
\code{|},
\code{\^},
\code{\&\&} and
\code{||}
\cspecificend % L1 ^^^^^

\cppspecificstart % L1 vvvvv
A \plc{reduction-identifier} is either an \plc{id-expression} or one of the following operators:
\code{+},
\code{-},
\code{*},
\code{\&},
\code{|},
\code{\^},
\code{\&\&} and
\code{||}
\cppspecificend % L1 ^^^^^

\fortranspecificstart
A \plc{reduction-identifier} is either a base language identifier, or a user-defined operator,
or one of the following operators:
\code{+},
\code{-},
\code{*},
\code{.and.},
\code{.or.},
\code{.eqv.},
\code{.neqv.},
 or one of the following intrinsic procedure names:
\code{max},
\code{min},
\code{iand},
\code{ior},
\code{ieor}.
\fortranspecificend


\ccppspecificstart % L0 vvvvv
Table~\ref{tab:Implicitly Declared C/C++ Reduction Identifiers} lists each
\plc{reduction-identifier} that is implicitly declared at every scope for
arithmetic types and its semantic initializer value. The actual initializer
value is that value as expressed in the data type of the reduction list item.


\newpage %% HACK 
% Table 
\nolinenumbers
\renewcommand{\arraystretch}{1.5}
%\begin{center}
\tablecaption{Implicitly Declared C/C++ \plc{reduction-identifiers}\label{tab:Implicitly Declared C/C++ Reduction Identifiers}}
%% \tablecaption*{}
\tablefirsthead{%
\hline
\textsf{\textbf{Identifier}} & \textsf{\textbf{Initializer}} & \textsf{\textbf{Combiner}}\\
\hline \\[-3ex]
}
\tablehead{%
\multicolumn{2}{l}{\small\slshape table continued from previous page}\\
\hline
\textsf{\textbf{Identifier}} & \textsf{\textbf{Initializer}} & \textsf{\textbf{Combiner}}\\
\hline \\[-3ex]
}
\tabletail{%
\hline\\[-4ex]
\multicolumn{2}{l}{\small\slshape table continued on next page}\\
}
\tablelasttail{\hline}
\begin{supertabular}{ p{0.1\textwidth} p{0.3\textwidth} p{0.5\textwidth}}
\code{+} & \code{omp\_priv = 0} & \code{omp\_out += omp\_in}\\
\code{*} & \code{omp\_priv = 1} & \code{omp\_out *= omp\_in}\\
\code{-} & \code{omp\_priv = 0} & \code{omp\_out += omp\_in}\\
\code{\&} & \code{omp\_priv =  \textasciitilde 0} & \code{omp\_out \&= omp\_in}\\
\code{|} & \code{omp\_priv = 0} & \code{omp\_out |= omp\_in}\\
\code{\^} & \code{omp\_priv = 0} & \code{omp\_out \^}\code{= omp\_in}\\
\code{\&\&} & \code{omp\_priv = 1} & \code{omp\_out = omp\_in \&\& omp\_out}\\
\code{||} & \code{omp\_priv = 0} & \code{omp\_out = omp\_in || omp\_out}\\
\code{max} & \code{omp\_priv = \plc{Least representable number in the reduction list item type}} & \code{omp\_out = omp\_in > omp\_out ? omp\_in : omp\_out}\\
\code{min} & \code{omp\_priv = \plc{Largest representable number in the reduction list item type}} & \code{omp\_out = omp\_in < omp\_out ? omp\_in : omp\_out}\\
\end{supertabular}
%\end{center}
\linenumbers

\ccppspecificend % L0 ^^^^^
\bigskip

\fortranspecificstart
Table~\ref{tab:Implicitly Declared Fortran Reduction Identifiers} lists each
\plc{reduction-identifier} that is implicitly declared for numeric and logical
types and its semantic initializer value. The actual initializer value is that
value as expressed in the data type of the reduction list item.

% Table
\nolinenumbers
\renewcommand{\arraystretch}{1.5}
\tablefirsthead{%
\hline
\textsf{\textbf{Identifier}} & \textsf{\textbf{Initializer}} & \textsf{\textbf{Combiner}}\\
\hline \\[-3ex]
}
\tablehead{%
\multicolumn{2}{l}{\small\slshape table continued from previous page}\\
\hline
\textsf{\textbf{Identifier}} & \textsf{\textbf{Initializer}} & \textsf{\textbf{Combiner}}\\
\hline \\[-3ex]
}
\tabletail{%
\hline\\[-4ex]
\multicolumn{2}{l}{\small\slshape table continued on next page}\\
}
\tablelasttail{\hline}
\tablecaption{Implicitly Declared Fortran \plc{reduction-identifiers}\label{tab:Implicitly Declared Fortran Reduction Identifiers}}
\begin{supertabular}{ p{0.1\textwidth} p{0.30\textwidth} p{0.5\textwidth}}
\code{+} & \code{omp\_priv = 0} & \code{omp\_out = omp\_in + omp\_out}\\
\code{*} & \code{omp\_priv = 1} & \code{omp\_out = omp\_in * omp\_out}\\
\code{-} & \code{omp\_priv = 0} & \code{omp\_out = omp\_in + omp\_out}\\
\code{.and.} & \code{omp\_priv = .true.} & \code{omp\_out = omp\_in .and.\ omp\_out}\\
\code{.or.} & \code{omp\_priv = .false.} & \code{omp\_out = omp\_in .or.\ omp\_out}\\
\code{.eqv.} & \code{omp\_priv = .true.} & \code{omp\_out = omp\_in .eqv.\ omp\_out}\\
\code{.neqv.} & \code{omp\_priv = .false.} & \code{omp\_out = omp\_in .neqv.\ omp\_out}\\
\code{max} & \code{omp\_priv = \plc{Least representable number in the reduction list item type}} & \code{omp\_out = max(omp\_in, omp\_out)}\\
\code{min} & \code{omp\_priv = \plc{Largest representable number in the reduction list item type}} & \code{omp\_out = min(omp\_in, omp\_out)}\\
\code{iand} & \code{omp\_priv = \plc{All bits on}} & \code{omp\_out = iand(omp\_in, omp\_out)}\\
\code{ior} & \code{omp\_priv = 0} & \code{omp\_out = ior(omp\_in, omp\_out)}\\
\code{ieor} & \code{omp\_priv = 0} & \code{omp\_out = ieor(omp\_in, omp\_out)}\\
\end{supertabular}

\linenumbers

\fortranspecificend
\vspace{\baselineskip}

In the above tables, \code{omp\_in} and \code{omp\_out} correspond to two
identifiers that refer to storage of the type of the list item. \code{omp\_out}
holds the final value of the combiner operation.

Any \plc{reduction-identifier} that is defined with the \code{declare}~\code{reduction}
directive is also valid. In that case, the initializer and combiner of the
\plc{reduction-identifier} are specified by the \plc{initializer-clause} and
the \plc{combiner} in the \code{declare}~\code{reduction} directive.




\descr
A reduction clause specifies a \plc{reduction-identifier} and one or more list
items.

The \plc{reduction-identifier} specified in a reduction clause must match a
previously declared \plc{reduction-identifier} of the same name and type for
each of the list items. This match is done by means of a name lookup in the
base language.

The list items that appear in a reduction clause may include array sections.


\cppspecificstart
If the type is a derived class, then any \plc{reduction-identifier} that
matches its base classes is also a match, if there is no specific match for the
type.

If the \plc{reduction-identifier} is not an \plc{id-expression}, then it is
implicitly converted to one by prepending the keyword operator (for example,
\code{+} becomes \code{\plc{operator}+}).

If the \plc{reduction-identifier} is qualified then a qualified name lookup is
used to find the declaration.

If the \plc{reduction-identifier} is unqualified then an \emph{argument-dependent name lookup}
must be performed using the type of each list item.
\cppspecificend

If the list item is an array or array section, it will be treated as
if a reduction clause would be applied to each separate element
of the array section. 



\restrictions
The restrictions common to reduction clauses are as follows:

\begin{itemize}
\item Any number of reduction clauses can be specified on the directive, but a
list item (or any array element in an array section) can appear only once in
reduction clauses for that directive.

\item For a \plc{reduction-identifier} declared with the \code{declare}~\code{reduction}
construct, the directive must appear before its use in a reduction clause.

\item If a list item is an array section, it must specify contiguous storage
and it cannot be a zero-length array section.

\item If a list item is an array section, accesses to the elements of the array
outside the specified array section result in unspecified behavior.

\ccppspecificstart
\item The type of a list item that appears in a reduction clause must be valid for the 
\plc{reduction-identifier}. For a \code{max} or \code{min} reduction in C, the type of the list item must 
be an allowed arithmetic data type: \code{char}, \code{int}, \code{float}, \code{double}, or \code{\_Bool}, 
possibly modified with \code{long}, \code{short}, \code{signed}, or \code{unsigned}. For a \code{max} or \code{min} 
reduction in C++, the type of the list item must be an allowed arithmetic data type: 
\code{char}, \code{wchar\_t}, \code{int}, \code{float}, \code{double}, or \code{bool}, possibly modified with \code{long}, 
\code{short}, \code{signed}, or \code{unsigned}.

\item A list item that appears in a reduction clause must not be \code{const}-qualified.

\item The \plc{reduction-identifier} for any list item must be unambiguous and accessible.
\ccppspecificend
\bigskip

\fortranspecificstart
\item The type and the rank of a list item that appears in a reduction clause
must be valid for the \plc{combiner} and \plc{initializer}.

\item A list item that appears in a reduction clause must be definable.

\item A procedure pointer may not appear in a reduction clause.

\item A pointer with the \code{INTENT(IN)} attribute may not appear in the
reduction clause.

\item An original list item with the \code{POINTER} attribute or any pointer
component of an original list item that is referenced in the \plc{combiner}
must be associated at entry to the construct that contains the reduction
clause. Additionally, the list item or the pointer component of the list item
must not be deallocated, allocated, or pointer assigned within the region.

\item An original list item with the \code{ALLOCATABLE} attribute or any
allocatable component of an original list item that is referenced in the
\plc{combiner} must be in the allocated state at entry to the construct that
contains the reduction clause. Additionally, the list item or the allocatable
component of the list item must be neither deallocated nor allocated within the
region.

\item If the \plc{reduction-identifier} is defined in a \code{declare}~\code{reduction}
directive, the \code{declare}~\code{reduction} directive must be in the same
subprogram, or accessible by host or use association.

\item If the \plc{reduction-identifier} is a user-defined operator, the same
explicit interface for that operator must be accessible as at the
\code{declare}~\code{reduction} directive.

\item If the \plc{reduction-identifier} is defined in a \code{declare}~\code{reduction}
directive, any subroutine or function referenced in the initializer clause or
combiner expression must be an intrinsic function, or must have an explicit
interface where the same explicit interface is accessible as at the
\code{declare}~\code{reduction} directive.
\fortranspecificend
\end{itemize}










\subsubsection{Reduction Scoping Clauses}
\label{subsubsec:Reduction Scoping Clauses}
Reduction scoping clauses define the region in which a reduction is computed by
tasks or SIMD lanes. All properties common to all reduction clauses,
which are defined in Section~\ref{subsubsec:Properties Common To All Reduction
Clauses}, apply to reduction scoping clauses.

The number of copies created for each list item and the time at which those
copies are initialized are determined by the particular reduction scoping clause
that appears on the construct. Any copies associated with the reduction are
initialized with the intializer value of the \plc{reduction-identifier}.


Any copies are combined using the combiner associated with the
\plc{reduction-identifier}. The time at which the original list item contains the result
of the reduction is determined by the particular reduction scoping clause.

\begin{samepage}
\fortranspecificstart
If the original list item has the \code{POINTER} attribute, copies of
the list item are associated with private targets.
\fortranspecificend
\end{samepage}

If the list item is an array section, the elements of any copy of the array section will
be allocated contiguously.

The location in the OpenMP program at which values are combined and the
order in which values are combined are unspecified. Therefore, when
comparing sequential and parallel runs, or when comparing one parallel run to
another (even if the number of threads used is the same), there is no guarantee
that bit-identical results will be obtained or that side effects (such as
floating-point exceptions) will be identical or take place at the same location
in the OpenMP program.

To avoid race conditions, concurrent reads or updates of the original list item
must be synchronized with the update of the original list item that occurs as a
result of the reduction computation.










\subsubsection{Reduction Participating Clauses}
\label{subsubsec:Reduction Participating Clauses}
A reduction participating clause specifies a task or a SIMD lane as a
participant in a reduction defined by a reduction scoping clause.
All properties common to all reduction clauses, which are defined in
Section~\ref{subsubsec:Properties Common To All Reduction Clauses}, apply to
reduction participating clauses.

Accesses to the original list item may be replaced by accesses to copies of the
original list item created by a region associated with a construct with a
reduction scoping clause.

In any case, the final value of the reduction must be determined as if all tasks
or SIMD lanes that participate in the reduction are executed sequentially in
some arbitrary order.










\subsubsection{\code{reduction} Clause}
\index{reduction@{\code{reduction}}}
\index{clauses!reduction@{\code{reduction}}}
\label{subsubsec:reduction clause}
\summary
The \code{reduction} clause specifies a \plc{reduction-identifier} and one or
more list items. For each list item, a private copy is created in each implicit
task or SIMD lane and is initialized with the initializer value of the
\plc{reduction-identifier}. After the end of the region, the original list item
is updated with the values of the private copies using the combiner associated
with the \plc{reduction-identifier}.

\syntax
\begin{boxedcode}
\code{reduction}(\plc{reduction-identifier }:\plc{ list})
\end{boxedcode}
Where \plc{reduction-identifier} is defined in Section
\ref{subsubsec:Properties Common To All Reduction Clauses}.

\descr
The \code{reduction} clause is a reduction scoping clause and a reduction
participating clause, as described in Sections \ref{subsubsec:Reduction Scoping
Clauses} and \ref{subsubsec:Reduction Participating Clauses}.

For \code{parallel} and worksharing constructs, a private copy of each list item is created, 
one for each implicit task, as if the \code{private} clause had been used. For the \code{simd} 
construct, a private copy of each list item is created, one for each SIMD lane as if the 
\code{private} clause had been used. 
For the \code{taskloop} construct, private copies are created according to the rules of the reduction scoping clauses. 
For the \code{target} construct, a private
copy of each list item is created and initialized for the initial task as if
the \code{private} clause had been used. For the \code{teams} construct, a
private copy of each list item is created and initialized, one for each team in
the league as if the \code{private} clause had been used. At the end of the
region for  which the \code{reduction} clause was specified, the original list
item is updated by combining its original value with the final value of each of
the private copies, using the combiner of the specified
\plc{reduction-identifier}.


If \code{nowait} is not used, the reduction computation will be complete at the end of the 
construct; however, if the reduction clause is used on a construct to which \code{nowait} is 
also applied, accesses to the original list item will create a race and, thus, have 
unspecified effect unless synchronization ensures that they occur after all threads have 
executed all of their iterations or \code{section} constructs, and the reduction computation 
has completed and stored the computed value of that list item. This can most simply be 
ensured through a barrier synchronization.


\restrictions
The restrictions to the \code{reduction} clause are as follows:

\begin{itemize}
\item All the common restrictions to all reduction clauses, which are listed in
Section \ref{subsubsec:Properties Common To All Reduction Clauses}, apply to
this clause.

\item A list item that appears in a \code{reduction} clause of a worksharing
construct must be shared in the \code{parallel} regions to which any of the
worksharing regions arising from the worksharing construct bind.

\item A list item that appears in a \code{reduction} clause of the innermost
enclosing worksharing or \code{parallel} construct may not be accessed in an
explicit task generated by a construct for which an \code{in\_reduction} clause
over the same list item does not appear.

\ccppspecificstart
\item If a list item in a \code{reduction} clause on a worksharing
construct has a reference type then it must bind to the same object for all threads of the team.
\ccppspecificend
\end{itemize}










\subsubsection{\code{task\_reduction} Clause}
\index{task\_reduction@{\code{task\_reduction}}}
\index{clauses!task\_reduction@{\code{task\_reduction}}}
\label{subsubsec:task_reduction clause}
\summary
The \code{task\_reduction} clause specifies a reduction among tasks.

\syntax
\begin{boxedcode}
\code{task\_reduction}(\plc{reduction-identifier }:\plc{ list})
\end{boxedcode}
Where \plc{reduction-identifier} is defined in Section
\ref{subsubsec:Properties Common To All Reduction Clauses}.

\descr
The \code{task\_reduction} clause is a reduction scoping clause, as described in
\ref{subsubsec:Reduction Scoping Clauses}.

For each list item, the number of copies is unspecified. Any copies associated
with the reduction are initialized before they are accessed by the tasks
participating in the reduction. After the end of the region, the original list
item contains the result of the reduction.

\restrictions
The restrictions to the \code{task\_reduction} clause are as follows:

\begin{itemize}
\item All the common restrictions to all reduction clauses, which are listed in
Section \ref{subsubsec:Properties Common To All Reduction Clauses}, apply to
this clause.
\end{itemize}










\newpage %% HACK

\subsubsection{\code{in\_reduction} Clause}
\index{in\_reduction@{\code{in\_reduction}}}
\index{clauses!in\_reduction@{\code{in\_reduction}}}
\label{subsubsec:in_reduction clause}
\summary
The \code{in\_reduction} clause specifies that a task participates in a reduction.

\syntax
\begin{boxedcode}
\code{in\_reduction}(\plc{reduction-identifier }:\plc{ list})
\end{boxedcode}
Where \plc{reduction-identifier} is defined in Section \ref{subsubsec:Properties Common To All Reduction Clauses}

\descr
The \code{in\_reduction} clause is a reduction participating clause, as described in
Section \ref{subsubsec:Reduction Participating Clauses}.

\restrictions
The restrictions to the \code{in\_reduction} clause are as follows:

\begin{itemize}
\item All the common restrictions to all reduction clauses, which are listed in
Section \ref{subsubsec:Properties Common To All Reduction Clauses}, apply to
this clause.


\item A list item that appears in an \code{in\_reduction} clause of a \code{task}
construct must appear in a \code{task\_reduction} clause of a construct
associated with a taskgroup region that includes the participating task in its
\plc{taskgroup set}. The construct associated with the innermost region that meets
this condition must specify the same \plc{reduction-identifier} as the
\code{in\_reduction} clause.
\end{itemize}










\subsection{Data Copying Clauses}
\label{subsec:Data Copying Clauses}
\index{data copying clauses}
\index{clauses!data copying}
This section describes the \code{copyin} clause (allowed on the \code{parallel} directive and 
combined parallel worksharing directives) and the \code{copyprivate} clause (allowed on 
the \code{single} directive).

These clauses support the copying of data values from private or threadprivate variables 
on one implicit task or thread to the corresponding variables on other implicit tasks or 
threads in the team.

The clauses accept a comma-separated list of list items (see \specref{sec:Directive Format}). 
All list items appearing in a clause must be visible, according to the scoping rules of the 
base language. Clauses may be repeated as needed, but a list item that specifies a given 
variable may not appear in more than one clause on the same directive.

\fortranspecificstart
An associate name preserves the association with the selector established at the \code{ASSOCIATE} statement. A list item that appears in a data copying clause may be a selector of an \code{ASSOCIATE} construct. If the construct association is established prior to a parallel region, the association between the associate name and the original list item will be retained in the region.
\fortranspecificend







\subsubsection{\code{copyin} Clause}
\index{copyin@{\code{copyin}}}
\index{clauses!copyin@{\code{copyin}}}
\label{subsubsec:copyin clause}
\summary
The \code{copyin} clause provides a mechanism to copy the value of the master thread’s 
threadprivate variable to the threadprivate variable of each other member of the team 
executing the \code{parallel} region. 

\syntax
The syntax of the \code{copyin} clause is as follows:

\begin{boxedcode}
copyin(\plc{list})
\end{boxedcode}

\descr
\ccppspecificstart
The copy is done after the team is formed and prior to the start of execution of the 
associated structured block. For variables of non-array type, the copy occurs by copy 
assignment. For an array of elements of non-array type, each element is copied as if by 
assignment from an element of the master thread’s array to the corresponding element of 
the other thread’s array. 
\ccppspecificend

\cppspecificstart
For class types, the copy assignment operator is invoked. The order in which copy 
assignment operators for different variables of class type are called is unspecified. 
\cppspecificend

\fortranspecificstart
The copy is done, as if by assignment, after the team is formed and prior to the start of 
execution of the associated structured block.

On entry to any \code{parallel} region, each thread’s copy of a variable that is affected by 
a \code{copyin} clause for the \code{parallel} region will acquire the allocation, association, and 
definition status of the master thread’s copy, according to the following rules:

\begin{itemize}
\item If the original list item has the \code{POINTER} attribute, each copy receives the same 
association status of the master thread’s copy as if by pointer assignment.

\item If the original list item does not have the \code{POINTER} attribute, each copy becomes 
defined with the value of the master thread's copy as if by intrinsic assignment, 
unless it has the allocation status of unallocated, in which case each copy
will have the same status.
\end{itemize}
\fortranspecificend

\restrictions
The restrictions to the \code{copyin} clause are as follows:
\ccppspecificstart
\begin{itemize}
\item A list item that appears in a \code{copyin} clause must be threadprivate.

\item A variable of class type (or array thereof) that appears in a \code{copyin} clause requires 
an accessible, unambiguous copy assignment operator for the class type.
\end{itemize}
\ccppspecificend

\fortranspecificstart
\begin{itemize}
\item A list item that appears in a \code{copyin} clause must be threadprivate. Named variables 
appearing in a threadprivate common block may be specified: it is not necessary to 
specify the whole common block. 

\item A common block name that appears in a \code{copyin} clause must be declared to be a 
common block in the same scoping unit in which the \code{copyin} clause appears.
\end{itemize}
\fortranspecificend









\subsubsection{\code{copyprivate} Clause}
\index{copyprivate@{\code{copyprivate}}}
\index{clauses!copyprivate@{\code{copyprivate}}}
\label{subsubsec:copyprivate clause}
\summary
The \code{copyprivate} clause provides a mechanism to use a private variable to broadcast 
a value from the data environment of one implicit task to the data environments of the 
other implicit tasks belonging to the \code{parallel} region.

To avoid race conditions, concurrent reads or updates of the list item must be 
synchronized with the update of the list item that occurs as a result of the 
\code{copyprivate} clause.

\syntax
The syntax of the \code{copyprivate} clause is as follows:

\begin{boxedcode}
copyprivate(\plc{list})
\end{boxedcode}

\descr
The effect of the \code{copyprivate} clause on the specified list items occurs after the 
execution of the structured block associated with the \code{single} construct (see 
\specref{subsec:single Construct}), 
and before any of the threads in the team have left the barrier 
at the end of the construct.

\ccppspecificstart
In all other implicit tasks belonging to the \code{parallel} region, each specified list item 
becomes defined with the value of the corresponding list item in the implicit task associated with the 
thread that executed the structured block. For variables of non-array type, the definition 
occurs by copy assignment. For an array of elements of non-array type, each element is
copied by copy assignment from an element of the array in the data environment of the 
implicit task associated with the thread that executed the structured block to the 
corresponding element of the array in the data environment of the other implicit tasks
\ccppspecificend

\cppspecificstart
For class types, a copy assignment operator is invoked. The order in which copy 
assignment operators for different variables of class type are called is unspecified. 
\cppspecificend

\fortranspecificstart
If a list item does not have the \code{POINTER} attribute, then in all other implicit tasks 
belonging to the \code{parallel} region, the list item becomes defined as if by intrinsic 
assignment with the value of the corresponding list item in the implicit task associated 
with the thread that executed the structured block. 

If the list item has the \code{POINTER} attribute, then, in all other implicit tasks belonging to 
the \code{parallel} region, the list item receives, as if by pointer assignment, the same 
association status of the corresponding list item in the implicit task associated with the 
thread that executed the structured block.

The order in which any final subroutines for different variables of a finalizable type are called is unspecified.
\fortranspecificend

\notestart
\noteheader -- The \code{copyprivate} clause is an alternative to using a shared variable for the 
value when providing such a shared variable would be difficult (for example, in a 
recursion requiring a different variable at each level). 
\noteend

\restrictions
The restrictions to the \code{copyprivate} clause are as follows:

\begin{itemize}
\item All list items that appear in the \code{copyprivate} clause must be either threadprivate 
or private in the enclosing context.

\item A list item that appears in a \code{copyprivate} clause may not appear in a \code{private} or
\code{firstprivate} clause on the \code{single} construct. 

\cppspecificstart
\item A variable of class type (or array thereof) that appears in a \code{copyprivate} clause 
requires an accessible unambiguous copy assignment operator for the class type.
\cppspecificend

\fortranspecificstart
\item A common block that appears in a \code{copyprivate} clause must be threadprivate. 

\item Pointers with the \code{INTENT(IN)} attribute may not appear in the \code{copyprivate} 
clause.
\item The list item with the \code{ALLOCATABLE} attribute must have the allocation status of allocated when the intrinsic assignment is performed. 
\fortranspecificend
\end{itemize}







%%% map

\subsection{Data-mapping Attribute Rules and Clauses}
\label{subsec:Data-mapping Attribute Rules and Clauses}
\index{data-mapping rules and clauses}
\index{attributes, data-mapping}

%% Do we need something about
%% ``explicit \code{declare}~\code{target} directives''?
This section describes how the data-mapping attributes of any variable 
referenced in a \code{target} region are determined. When specified, 
explicit \code{map} clauses on \code{target}~\code{data} and \code{target} 
directives determine these attributes.  Otherwise, the following 
data-mapping rules apply for variables referenced in a \code{target}
construct that are not declared in the construct and do not appear in 
data-sharing attribute or \code{map} clauses:

Certain variables and objects have predetermined data-mapping attributes 
as follows:

\begin{itemize}
\item If a variable appears in a \code{to} or \code{link} clause 
on a \code{declare}~\code{target} directive then it is treated as if it had appeared in a \code{map} clause with a \plc{map-type} of \code{tofrom}.

\ccppspecificstart
\item A variable that is of type pointer is treated as if it had appeared in a \code{map} clause as a zero-length array section.
\ccppspecificend

\cppspecificstart
\item A variable that is of type reference to pointer is treated as if it had appeared in a \code{map} clause as a zero-length array section.
\cppspecificend
\end{itemize}

Otherwise, the following implicit data-mapping attribute rules apply:

\begin{itemize}
\item If a \code{defaultmap(tofrom:scalar)} clause is not present then 
a scalar variable is not mapped, but instead has an implicit data-sharing 
attribute of firstprivate (see 
\specref{subsubsec:Data-sharing Attribute Rules for Variables Referenced in a Construct}).

\item If a \code{defaultmap(tofrom:scalar)} clause is present then a scalar 
variable is treated as if it had appeared in a \code{map} clause with a 
\plc{map-type} of \code{tofrom}.

\item If a variable is not a scalar then it is treated as if it had appeared 
in a \code{map} clause with a \plc{map-type} of \code{tofrom}.
\end{itemize}

\subsubsection{\code{map} Clause}
\label{subsec:map Clause}
\index{map@{\code{map}}}
\index{clauses!map@{\code{map}}}
\summary
The \code{map} clause specifies how an original list item is mapped from the current task's data environment to a corresponding list item in the device data environment of the device identified by the construct.

\syntax
The syntax of the map clause is as follows:

\begin{boxedcode}
map(\plc{[ [map-type-modifier[,]] map-type} : \plc{] list})
\end{boxedcode}

where \plc{map-type} is one of the following:

\begin{indentedcodelist}
to
from
tofrom
alloc
release
delete
\end{indentedcodelist}

and \plc{map-type-modifier} is \code{always}.

\descr
The list items that appear in a \code{map} clause may include array sections and structure elements.

The \plc{map-type} and \plc{map-type-modifier} specify the effect of the \code{map} clause, as described below.

The original and corresponding list items may share storage such that writes to either
item by one task followed by a read or write of the other item by another task without
intervening synchronization can result in data races.

If the \code{map} clause appears on a \code{target}, \code{target}~\code{data}, or \code{target}~\code{enter}~\code{data} construct then on entry to the region the following sequence of steps occurs as if performed as a single atomic operation:
\begin{enumerate}
\item If a corresponding list item of the original list item is not present in the device data environment, then:
\begin{enumerate}
\item A new list item with language-specific attributes is derived from the original list item and created in the device data environment.
\item The new list item becomes the corresponding list item to the original list item in the device data environment.
\item The corresponding list item has a reference count that is initialized to zero. 
\end{enumerate}
\item The corresponding list item's reference count is incremented by one.
\item If the corresponding list item's reference count is one or the \code{always} \plc{map-type-modifier} is present, then:
\begin{enumerate}
\item If the \plc{map-type} is \code{to} or \code{tofrom}, then the corresponding list item is assigned the value of the original list item. 
\end{enumerate}
\item If the corresponding list item's reference count is one, then:
\begin{enumerate}
\item If the \plc{map-type} is \code{from} or \code{alloc}, the value of the corresponding list item is undefined. 
\end{enumerate}
\end{enumerate}

If the \code{map} clause appears on a \code{target}, \code{target}~\code{data}, or \code{target}~\code{exit}~\code{data} construct then on exit from the region the following sequence of steps occurs as if performed as a single atomic operation:
\begin{enumerate}
\item If a corresponding list item of the original list item is not present in the device data environment, then the list item is ignored.
\item If a corresponding list item of the original list item is present in the device data environment, then:
\begin{enumerate}
\item If the corresponding list item's reference count is finite, then:
\begin{enumerate}
%\item the corresponding list item's reference count is decremented by one.
\item If the \plc{map-type} is not \code{delete}, then the corresponding list item's reference count is decremented by one.
\item If the \plc{map-type} is \code{delete}, then the corresponding list item's reference count is set to zero. 
\end{enumerate}
\item If the corresponding list item's reference count is zero or the \code{always} \plc{map-type-modifier} is present, then:
\begin{enumerate}
\item If the \plc{map-type} is \code{from} or \code{tofrom}, then the original list item is assigned the value of the corresponding list item. 
\end{enumerate}
\item If the corresponding list item's reference count is zero, then the corresponding list item is removed from the device data environment 
\end{enumerate}
\end{enumerate}

If a single contiguous part of the original storage of a list item with an
implicit data-mapping attribute has corresponding storage in the device data
environment prior to a task encountering the construct associated with the
\code{map} clause, only that part of the original storage will have
corresponding storage in the device data environment as a result of the \code{map}
clause. 

\ccppspecificstart
If a new list item is created then a new list item of the same type, with automatic storage
duration, is allocated for the construct. The size and alignment of the new list
item are determined by the static type of the variable. This allocation occurs if the region
references the list item in any statement.
\ccppspecificend

\fortranspecificstart
If a new list item is created then a new list item of the same type, type parameter, and
rank is allocated.
\fortranspecificend

The \plc{map-type} determines how the new list item is initialized.

If a \plc{map-type} is not specified, the \plc{map-type} defaults to \code{tofrom}.

\events
The \plc{target-map} event occurs when a thread maps data to or from a target device.

The \plc{target-transfer} event occurs when a thread initiates a data transfer to or from a target device.

\tools

A thread dispatches a registered \code{ompt\_callback\_target\_map}
callback for each occurrence of a \plc{target-map} event in that thread. 
The callback occurs in the context of the target task.  The callback has type signature
\code{ompt\_callback\_target\_map\_t}. 

A thread dispatches a registered \code{ompt\_callback\_target\_transfer}
callback for each occurrence of a \plc{target-transfer} event in that thread. 
The callback occurs in the context of the target task.  The callback has type signature
\code{ompt\_callback\_target\_transfer\_t}. 

\restrictions
\begin{itemize}

\item A list item cannot appear in both a \code{map} clause and a data-sharing attribute clause on the same construct.

\item If a list item is an array section, it must specify contiguous storage.

\item At most one list item can be an array item derived from a given variable in \code{map}
clauses of the same construct.

\item List items of \code{map} clauses in the same construct must not share original storage.

\item If any part of the original storage of a list item with a predetermined or
explicit data-mapping attribute has corresponding storage in the device data
environment prior to a task encountering the construct associated with the map
clause, all of the original storage must have corresponding storage
in the device data environment prior to the task encountering the construct.

\item If a list item is an element of a structure, and a different element 
of the structure has a corresponding list item in the device data environment 
prior to a task encountering the construct associated with the \code{map} 
clause, then the list item must also have a correspnding list item in the 
device data environment prior to the task encountering the construct.

\item If a list item is an element of a structure, only the rightmost symbol of the variable reference can be an array section.

\item If variables that share storage are mapped, the behavior is unspecified. 

\item A list item must have a mappable type.

\item \code{threadprivate} variables cannot appear in a \code{map} clause.

\cppspecificstart
\item If the type of a list item is a reference to a type \plc{T} then the type will be considered to be \plc{T} for all purposes of this clause.
\cppspecificend

\ccppspecificstart
\item Initialization and assignment are through bitwise copy.

\item A variable for which the type is pointer and an array section derived from that variable must not appear as list items of \code{map} clauses of the same construct. 

\item A list item cannot be a variable that is a member of a structure with a union type.

\item A bit-field cannot appear in a \code{map} clause.
\ccppspecificend
\end{itemize}

\begin{samepage}
\fortranspecificstart
\begin{itemize}
\item The value of the new list item becomes that of the original list item in the map 
initialization and assignment.

\item A list item must not contain any components that have the
  \code{ALLOCATABLE} attribute.

\item If the allocation status of a list item with the
  \code{ALLOCATABLE} attribute is unallocated upon entry to a
  \code{target} region, the list item must be unallocated
  upon exit from the region.

\item If the allocation status of a list item with the
  \code{ALLOCATABLE} attribute is allocated upon entry to a
  \code{target} region, the allocation status of the corresponding
  list item must not be changed and must not be reshaped in the
  region.

\item If an array section is mapped and the size of the section is
  smaller than that of the whole array, the behavior of referencing
  the whole array in the \code{target} region is unspecified.

\fortranspecificend
\end{itemize}
\end{samepage}





\subsubsection{\code{defaultmap} Clause}
\label{subsubsec:defaultmap clause}
\index{defaultmap@{\code{defaultmap}}}
\index{clauses!defaultmap@{\code{defaultmap}}}
\summary
The \code{defaultmap} clause explicitly determines the data-mapping attributes of variables that 
are referenced in a \code{target} construct and would otherwise be 
implicitly determined.

\syntax
\ccppspecificstart
The syntax of the \code{defaultmap} clause is as follows:

\begin{boxedcode}
defaultmap(tofrom:scalar)
\end{boxedcode}
\ccppspecificend

\fortranspecificstart
The syntax of the \code{defaultmap} clause is as follows:

\begin{boxedcode}
defaultmap(tofrom:scalar)
\end{boxedcode}
\fortranspecificend

\descr
The \code{defaultmap(tofrom:scalar)} clause causes all scalar variables
referenced in the construct that have implicitly determined data-mapping
attributes to have the \code{tofrom} \plc{map-type}.








\section{\code{declare}~\code{reduction} Directive}
\index{declare reduction@{\code{declare}~\code{reduction}}}
\index{directives!declare reduction@{\code{declare}~\code{reduction}}}
\label{sec:declare reduction Directive}
\summary
The following section describes the directive for declaring user-defined reductions. The 
\code{declare}~\code{reduction} directive declares a \plc{reduction-identifier} that can be used in a 
\code{reduction} clause. The \code{declare}~\code{reduction} directive is a declarative directive.

\syntax
\cspecificstart
\begin{boxedcode}
\#pragma omp declare reduction(\plc{reduction-identifier }:\plc{ typename-list }: 
\plc{combiner })\plc{[initializer-clause] new-line}
\end{boxedcode}

where:

\begin{itemize}

\item \plc{reduction-identifier} is either a base language identifier or one of the following 
operators: 
\code{+}, 
\code{-}, 
\code{*}, 
\code{\&}, 
\code{|}, 
\code{\^}, 
\code{\&\&} and 
\code{||} 

\item \plc{typename-list} is a list of type names 

\item \plc{combiner} is an expression 

\item \plc{initializer-clause} is \code{initializer(}\plc{initializer-expr}\code{)}
where \plc{initializer-expr} is\linebreak
\code{omp\_priv = }\plc{initializer} or \plc{function-name}\code{(}\plc{argument-list}\code{)} 
\end{itemize}
\cspecificend


\cppspecificstart
\begin{boxedcode}
\#pragma omp declare reduction(\plc{reduction-identifier }:\plc{ typename-list }: 
\plc{combiner})\plc{ [initializer-clause] new-line}
\end{boxedcode}

where:

\begin{itemize}
\item \plc{reduction-identifier} is either an \plc{id-expression} or one of the following 
operators: 
\code{+}, 
\code{-}, 
\code{*}, 
\code{\&}, 
\code{|}, 
\code{\^}, 
\code{\&\&} and 
\code{||} 

\item \plc{typename-list} is a list of type names 

\item \plc{combiner} is an expression 

% An equal sign is intentionally missing for C++, so that
% initializer (omp_priv (4)) syntax is also valid in addition to
% initializer (omp_priv = 4).
\item \plc{initializer-clause} is \code{initializer(}\plc{initializer-expr}\code{)} 
where \plc{initializer-expr} is\linebreak
\code{omp\_priv} \plc{initializer} or \plc{function-name}\code{(}\plc{argument-list}\code{)} 
\end{itemize}
\cppspecificend


\fortranspecificstart
\begin{boxedcode}
!\$omp declare reduction(\plc{reduction-identifier }:\plc{ type-list }:\plc{ combiner}) 
\plc{[initializer-clause]}
\end{boxedcode}

where:

\begin{itemize}
\item \plc{reduction-identifier} is either a base language identifier, or a user-defined operator, or 
one of the following operators: 
\code{+}, 
\code{-}, 
\code{*}, 
\code{.and.}, 
\code{.or.}, 
\code{.eqv.}, 
\code{.neqv.}, or one of the following intrinsic procedure names: 
\code{max}, 
\code{min}, 
\code{iand}, 
\code{ior}, 
\code{ieor}. 

\item \plc{type-list} is a list of type specifiers 

\item \plc{combiner} is either an assignment statement or a subroutine name followed by an 
argument list 

\item \plc{initializer-clause} is \code{initializer(}\plc{initializer-expr}\code{)}, 
where \plc{initializer-expr} is\linebreak
\code{omp\_priv = }\plc{expression} or \plc{subroutine-name}\code{(}\plc{argument-list}\code{)}
\end{itemize}
\fortranspecificend

\descr
Custom reductions can be defined using the \code{declare}~\code{reduction} directive; the 
\plc{reduction-identifier} and the type identify the \code{declare}~\code{reduction} directive. The 
\plc{reduction-identifier} can later be used in a \code{reduction} clause using variables of the 
type or types specified in the \code{declare}~\code{reduction} directive. If the directive applies 
to several types then it is considered as if there were multiple \code{declare}~\code{reduction} 
directives, one for each type.

\fortranspecificstart
If a type with deferred or assumed length type parameter is specified in a \code{declare}~\code{reduction} directive, the \plc{reduction-identifier} of that directive can be used in a \code{reduction} clause with any variable of the same type and the same kind parameter, regardless of the length type Fortran parameters with which the variable is declared.
\fortranspecificend

The visibility and accessibility of this declaration are the same as those of a variable 
declared at the same point in the program. The enclosing context of the \plc{combiner} and of 
the \plc{initializer-expr} will be that of the \code{declare}~\code{reduction} directive. The \plc{combiner} 
and the \plc{initializer-expr} must be correct in the base language as if they were the body of 
a function defined at the same point in the program.

\fortranspecificstart
If the \plc{reduction-identifier} is the same as the name of a user-defined operator or an extended operator, or the same as a generic name that is one of the allowed intrinsic procedures, and if the operator or procedure name appears in an accessibility statement in the same module, the accessibility of the corresponding \code{declare}~\code{reduction} directive is determined by the accessibility attribute of the statement.

If the \plc{reduction-identifier} is the same as a generic name that is one of the allowed intrinsic procedures and is accessible, and if it has the same name as a derived type in the same module, the accessibility of the corresponding \code{declare}~\code{reduction} directive is determined by the accessibility of the generic name according to the base language.
\fortranspecificend

\newpage %% HACK

\cppspecificstart
The \code{declare}~\code{reduction} directive can also appear at points in the program at which 
a static data member could be declared. In this case, the visibility and accessibility of 
the declaration are the same as those of a static data member declared at the same point 
in the program.
\cppspecificend

The \plc{combiner} specifies how partial results can be combined into a single value. The 
\plc{combiner} can use the special variable identifiers \code{omp\_in} and \code{omp\_out} that are of the 
type of the variables being reduced with this \plc{reduction-identifier}. Each of them will 
denote one of the values to be combined before executing the \plc{combiner}. It is assumed 
that the special \code{omp\_out} identifier will refer to the storage that holds the resulting 
combined value after executing the \plc{combiner}.

The number of times the \plc{combiner} is executed, and the order of these executions, for 
any \code{reduction} clause is unspecified.

\fortranspecificstart
If the \plc{combiner} is a subroutine name with an argument list, the \plc{combiner} is evaluated by 
calling the subroutine with the specified argument list.

If the \plc{combiner} is an assignment statement, the \plc{combiner} is evaluated by executing the 
assignment statement.
\fortranspecificend

As the \plc{initializer-expr} value of a user-defined reduction is not known \emph{a priori} the 
\plc{initializer-clause} can be used to specify one. Then the contents of the \plc{initializer-clause} 
will be used as the initializer for private copies of reduction list items where the 
\code{omp\_priv} identifier will refer to the storage to be initialized. The special identifier 
\code{omp\_orig} can also appear in the \plc{initializer-clause} and it will refer to the storage of the 
original variable to be reduced.

The number of times that the \plc{initializer-expr} is evaluated, and the order of these 
evaluations, is unspecified.

\ccppspecificstart
If the \plc{initializer-expr} is a function name with an argument list, the \plc{initializer-expr} is 
evaluated by calling the function with the specified argument list. Otherwise, the 
\plc{initializer-expr} specifies how \code{omp\_priv} is declared and initialized.
\ccppspecificend
\bigskip

\cspecificstart
If no \plc{initializer-clause} is specified, the private variables will be initialized following the 
rules for initialization of objects with static storage duration.
\cspecificend

\cppspecificstart
If no \plc{initializer-expr} is specified, the private variables will be initialized following the 
rules for \plc{default-initialization}.
\cppspecificend
\bigskip

\fortranspecificstart
If the \plc{initializer-expr} is a subroutine name with an argument list, the \plc{initializer-expr} is 
evaluated by calling the subroutine with the specified argument list.

If the \plc{initializer-expr} is an assignment statement, the \plc{initializer-expr} is evaluated by 
executing the assignment statement.

If no \plc{initializer-clause} is specified, the private variables will be initialized as follows:
\begin{itemize}
\item For \code{complex}, \code{real}, or \code{integer} types, the value 0 will be used.
\item For \code{logical} types, the value \code{.false.} will be used.
\item For derived types for which default initialization is specified, default initialization 
will be used.
\item Otherwise, not specifying an \plc{initializer-clause} results in unspecified behavior.
\end{itemize}
\fortranspecificend
\bigskip

\ccppspecificstart
If \plc{reduction-identifier} is used in a \code{target} region then a \code{declare}~\code{target} construct 
must be specified for any function that can be accessed through the \plc{combiner} and 
\plc{initializer-expr}.
\ccppspecificend
\bigskip

\fortranspecificstart
If \plc{reduction-identifier} is used in a \code{target} region then a \code{declare}~\code{target} construct 
must be specified for any function or subroutine that can be accessed through the \plc{combiner} 
and \plc{initializer-expr}.
\fortranspecificend

\restrictions
\begin{itemize}
\item Only the variables \code{omp\_in} and \code{omp\_out} are allowed in the \plc{combiner}.

\item Only the variables \code{omp\_priv} and \code{omp\_orig} are allowed in the \plc{initializer-clause}.

\item If the variable \code{omp\_orig} is modified in the \plc{initializer-clause}, the behavior is 
unspecified. 

\item If execution of the \plc{combiner} or the \plc{initializer-expr} results in the execution of an 
OpenMP construct or an OpenMP API call, then the behavior is unspecified.

\item A \plc{reduction-identifier} may not be re-declared in the current scope for the same type 
or for a type that is compatible according to the base language rules.

\item At most one \plc{initializer-clause} can be specified. 

\ccppspecificstart
\item A type name in a \code{declare}~\code{reduction} directive cannot be a function type, an 
array type, a reference type, or a type qualified with \code{const}, \code{volatile} or 
\code{restrict}. 
\ccppspecificend
\bigskip

\cspecificstart
\item If the \plc{initializer-expr} is a function name with an argument list, then one of the 
arguments must be the address of \code{omp\_priv}. 
\cspecificend
\bigskip

\cppspecificstart
\item If the \plc{initializer-expr} is a function name with an argument list, then one of the 
arguments must be \code{omp\_priv} or the address of \code{omp\_priv}. 
\cppspecificend
\bigskip

\fortranspecificstart
\item If the \plc{initializer-expr} is a subroutine name with an argument list, then one of the 
arguments must be \code{omp\_priv}.

\item If the \code{declare}~\code{reduction} directive appears in the specification part of a module and the corresponding reduction clause does not appear in the same module, the \plc{reduction-identifier} must be the same as the name of a user-defined operator, one of the allowed operators that is extended or a generic name that is the same as the name of one of the allowed intrinsic procedures. 

\item If the \code{declare}~\code{reduction} directive appears in the specification of a module, if the corresponding \code{reduction} clause does not appear in the same module, and if the \plc{reduction-identifier} is the same as the name of a user-defined operator or an extended operator, or the same as a generic name that is the same as one of the allowed intrinsic procedures then the interface for that operator or the generic name must be defined in the specification of the same module, or must be accessible by use association. 

\item Any subroutine or function used in the \code{initializer} clause or \plc{combiner} expression must be an intrinsic function, or must have an accessible interface. 

\item Any user-defined operator or extended operator used in the \code{initializer} clause or \plc{combiner} expression must have an accessible interface. 

\item If any subroutine, function, user-defined operator, or extended operator is used in the \code{initializer} clause or \plc{combiner} expression, it must be accessible to the subprogram in which the corresponding \code{reduction} clause is specified. 

\item If the length type parameter is specified for a character type, it must be a constant, a colon or an~\code{*}. 

\item If a character type with deferred or assumed length parameter is specified in a \code{declare}~\code{reduction} directive, no other \code{declare}~\code{reduction} directive with Fortran character type of the same kind parameter and the same \plc{reduction-identifier} is allowed in the same scope.

\item Any subroutine used in the \code{initializer} clause or \plc{combiner} expression must not have any alternate returns appear in the argument list.
\fortranspecificend
\end{itemize}

\crossreferences
\begin{itemize}
\item \code{reduction} clause, 
\specref{subsubsec:reduction clause}.
\end{itemize}










\section{Nesting of Regions}
\label{sec:Nesting of Regions}
\index{nesting of regions}
This section describes a set of restrictions on the nesting of regions. The restrictions on 
nesting are as follows:

\begin{itemize}
\item A worksharing region may not be closely nested inside a worksharing, \code{task}, \code{taskloop},
\code{critical}, \code{ordered}, \code{atomic}, or \code{master} region.

\item A \code{barrier} region may not be closely nested inside a worksharing, \code{task}, \code{taskloop},
\code{critical}, \code{ordered}, \code{atomic}, or \code{master} region.

\item A \code{master} region may not be closely nested inside a worksharing,
\code{atomic}, \code{task}, or \code{taskloop} region.

\item An \code{ordered} region arising from an \code{ordered} construct without
any clause or with the \code{threads} or \code{depend} clause may not be closely
nested inside a \code{critical}, \code{ordered}, \code{atomic}, \code{task},
or \code{taskloop} region. 

\item An \code{ordered} region arising from an \code{ordered} construct without
any clause or with the \code{threads} or \code{depend} clause must be closely nested
inside a loop region (or parallel loop region) with an \code{ordered} clause.

\item An \code{ordered} region arising from an \code{ordered} construct with the
\code{simd} clause must be closely nested inside a \code{simd} (or loop SIMD)
region.

\item An \code{ordered} region arising from an \code{ordered} construct with
  both the \code{simd} and \code{threads} clauses must be closely nested inside
  a loop SIMD region.

\item A \code{critical} region may not be nested (closely or otherwise) inside a \code{critical} 
region with the same name. This restriction is not sufficient to prevent 
deadlock.

\item OpenMP constructs may not be encountered during execution of an
\code{atomic} region.

\item An \code{ordered} construct with the \code{simd} clause is the only OpenMP
construct that can be encountered during execution of a \code{simd}
region.

\item If a \code{target}, \code{target}~\code{update}, 
\code{target}~\code{data}, \code{target}~\code{enter}~\code{data}, or 
\code{target}~\code{exit}~\code{data} construct is encountered during
execution of a \code{target} region, the behavior is unspecified.

\item If specified, a \code{teams} construct must be contained within a \code{target} construct. That 
\code{target} construct must not contain any statements or directives outside of the \code{teams} 
construct. 

\item \code{distribute}, \code{distribute simd}, distribute parallel loop,
distribute parallel loop SIMD, and \code{parallel} regions, including any
\code{parallel} regions arising from combined constructs, are the only OpenMP regions
that may be strictly nested inside the \code{teams} region.

\item The region associated with the \code{distribute} construct must be
strictly nested inside a \code{teams} region.

\item If \plc{construct-type-clause} is \code{taskgroup}, the \code{cancel}
construct must be closely nested inside a \code{task} construct and the
\code{cancel} region must be closely nested inside a \code{taskgroup} region. If
\plc{construct-type-clause} is \code{sections}, the \code{cancel} construct
must be closely nested inside a \code{sections} or \code{section} construct.
Otherwise, the \code{cancel} construct must be closely
nested inside an OpenMP construct that matches the type specified in
\plc{construct-type-clause} of the \code{cancel} construct.

\item A \code{cancellation}~\code{point} construct for which
\plc{construct-type-clause} is \code{taskgroup} must be closely nested
inside a \code{task} construct, and the \code{cancellation}~\code{point}
region must be closely nested inside a \code{taskgroup} region. A
\code{cancellation}~\code{point} construct for which
\plc{construct-type-clause} is \code{sections} must be closely nested
inside a \code{sections} or \code{section} construct. Otherwise, a
\code{cancellation}~\code{point} construct must be closely nested inside
an OpenMP construct that matches the type specified in
\plc{construct-type-clause}.

\end{itemize}

% This is the end of ch2-directives.tex of the OpenMP specification.


    % This is runtime_library.tex (Chapter 3) of the OpenMP specification.
% This is an included file. See the master file for more information.
%
% When editing this file:
%
%    1. To change formatting, appearance, or style, please edit openmp.sty.
%
%    2. Custom commands and macros are defined in openmp.sty.
%
%    3. Be kind to other editors -- keep a consistent style by copying-and-pasting to
%       create new content.
%
%    4. We use semantic markup, e.g. (see openmp.sty for a full list):
%         \code{}     % for bold monospace keywords, code, operators, etc.
%         \plc{}      % for italic placeholder names, grammar, etc.
%
%    5. There are environments that provide special formatting, e.g. language bars.
%       Please use them whereever appropriate.  Examples are:
%
%         \begin{fortranspecific}
%         This is text that appears enclosed in blue language bars for Fortran.
%         \end{fortranspecific}
%
%         \begin{note}
%         This is a note.  The "Note -- " header appears automatically.
%         \end{note}
%
%    6. Other recommendations:
%         Use the convenience macros defined in openmp.sty for the minor headers
%         such as Comments, Syntax, etc.
%
%         To keep items together on the same page, prefer the use of 
%         \begin{samepage}.... Avoid \parbox for text blocks as it interrupts line numbering.
%         When possible, avoid \filbreak, \pagebreak, \newpage, \clearpage unless that's
%         what you mean. Use \needspace{} cautiously for troublesome paragraphs.
%
%         Avoid absolute lengths and measures in this file; use relative units when possible.
%         Vertical space can be relative to \baselineskip or ex units. Horizontal space
%         can be relative to \linewidth or em units.
%
%         Prefer \emph{} to italicize terminology, e.g.:
%             This is a \emph{definition}, not a placeholder.
%             This is a \plc{var-name}.
%

\chapter{Runtime Library Routines}
\index{runtime library routines}
\label{chap:Runtime Library Routines}
This chapter describes the OpenMP API runtime library routines and queryable runtime states, and is divided into the 
following sections:

\begin{itemize}
\item Runtime library definitions 
(\specref{sec:runtime library definitions}).

\item Execution environment routines that can be used to control and to query the parallel 
execution environment 
(\specref{sec:Execution Environment Routines}).

\item Lock routines that can be used to synchronize access to data 
(\specref{sec:Lock Routines}). 

\item Portable timer routines 
(\specref{sec:Timing Routines}).

\item Device memory routines that can be used to allocate memory and 
to manage pointers on target devices (\specref{sec:Device Memory Routines}).

%\item Runtime states (\specref{sec:runtimeStates}).

% \item OMPT Query Functions (\specref{sec:omptFunctions}).

\item Execution routines to control the application monitoring
(\specref{sec:control_tool})
\end{itemize}

Throughout this chapter, \plc{true} and \plc{false} are used as generic terms to simplify the 
description of the routines. 

\begin{samepage}
\begin{ccppspecific}
\plc{true} means a nonzero integer value and \plc{false} means an integer value of zero. 
\end{ccppspecific}
\end{samepage}
\bigskip

\begin{samepage}
\begin{fortranspecific}
\plc{true} means a logical value of \code{.TRUE.} and \plc{false} means a logical value of \code{.FALSE.}.
\end{fortranspecific}
\end{samepage}
\bigskip

\begin{samepage}
\vspace{1\baselineskip}
\begin{fortranspecific}
\vspace{-1\baselineskip}
\restrictions

The following restriction applies to all OpenMP runtime library routines:

\begin{itemize}
\item OpenMP runtime library routines may not be called from \code{PURE} or \code{ELEMENTAL} 
procedures. 
\end{itemize}
\end{fortranspecific}
\end{samepage}











\section{Runtime Library Definitions}
\index{runtime library definitions}
\index{header files}
\index{include files}
\label{sec:runtime library definitions}
For each base language, a compliant implementation must supply a set of definitions for 
the OpenMP API runtime library routines and the special data types of their parameters. 
The set of definitions must contain a declaration for each OpenMP API runtime library 
routine and a declaration for the \emph{simple lock}, \emph{nestable lock}, \emph{schedule}, and \emph{thread affinity
policy} data types. In addition, each set of definitions may specify other implementation 
specific values.

\begin{ccppspecific}
The library routines are external functions with ``C'' linkage.

Prototypes for the C/C++ runtime library routines described in this chapter shall be 
provided in a header file named \code{omp.h}. This file defines the following: 

\begin{itemize}
\item The prototypes of all the routines in the chapter. 

\item The type \code{omp\_lock\_t}. 

\item The type \code{omp\_nest\_lock\_t}.

\item The type \code{omp\_lock\_hint\_t}. 

\item The type \code{omp\_sched\_t}.

\item The type \code{omp\_proc\_bind\_t}.

\item The type \code{omp\_control\_tool\_t}.

\item The type \code{omp\_control\_tool\_result\_t}.

\item The type \code{omp\_allocator\_t}.

\end{itemize}

% was: C.1 on page 288 
See \specref{sec:Example of the omp.h Header File} for an example of this file.


The \code{omp\_alloc.h} header file will enable memory routines of a system to be used in the \code{allocate} directive.
This file shall contain appropriate \code{declare alloc} directives for at least the following memory allocation routines:
\begin{itemize}
\item \code{malloc}
\item \code{calloc}
\item \code{realloc}
\item \code{posix\_memalign}
\item \code{aligned\_alloc}
\item \code{free}
\end{itemize} 
See \specref{sec:Example of the omp_alloc.h Header File} for an example of this file.
\end{ccppspecific}

\begin{fortranspecific}
The OpenMP Fortran API runtime library routines are external procedures. The return 
values of these routines are of default kind, unless otherwise specified.

Interface declarations for the OpenMP Fortran runtime library routines described in this 
chapter shall be provided in the form of a Fortran \code{include} file named \code{omp\_lib.h} or 
a Fortran~90 \code{module} named \code{omp\_lib}. It is implementation defined whether the 
\code{include} file or the \code{module} file (or both) is provided.

These files define the following:

\begin{itemize}
\item The interfaces of all of the routines in this chapter.

\item The \code{integer} \code{parameter} \code{omp\_lock\_kind}.

\item The \code{integer} \code{parameter} \code{omp\_nest\_lock\_kind}.

\item The \code{integer} \code{parameter} \code{omp\_lock\_hint\_kind}.

\item The \code{integer} \code{parameter} \code{omp\_sched\_kind}.

\item The \code{integer} \code{parameter} \code{omp\_proc\_bind\_kind}.

\item The \code{integer} \code{parameter} \code{omp\_allocator\_kind}.

\item The \code{integer} \code{parameter} \code{openmp\_version} with a value \plc{yyyymm} where \plc{yyyy} 
and \plc{mm} are the year and month designations of the version of the OpenMP Fortran 
API that the implementation supports. This value matches that of the C preprocessor 
macro \code{\_OPENMP}, when a macro preprocessor is supported (see 
\specref{sec:Conditional Compilation}).
\end{itemize}

See \specref{sec:Example of an Interface Declaration include File} 
and \specref{sec:Example of a Fortran Interface Declaration module} 
for examples of these files.

It is implementation defined whether any of the OpenMP runtime library routines that 
take an argument are extended with a generic interface so arguments of different \code{KIND} 
type can be accommodated. See Appendix~\ref{sec:Example of a Generic Interface for a Library Routine}
for an example of such an extension. 
\end{fortranspecific}

% This is an included file. See the master file for more information.
%
% When editing this file:
%
%    1. To change formatting, appearance, or style, please edit openmp.sty.
%
%    2. Custom commands and macros are defined in openmp.sty.
%
%    3. Be kind to other editors -- keep a consistent style by copying-and-pasting to
%       create new content.
%
%    4. We use semantic markup, e.g. (see openmp.sty for a full list):
%         \code{}     % for bold monospace keywords, code, operators, etc.
%         \plc{}      % for italic placeholder names, grammar, etc.
%
%    5. There are environments that provide special formatting, e.g. language bars.
%       Please use them whereever appropriate.  Examples are:
%
%         \begin{fortranspecific}
%         This is text that appears enclosed in blue language bars for Fortran.
%         \end{fortranspecific}
%
%         \begin{note}
%         This is a note.  The "Note -- " header appears automatically.
%         \end{note}
%
%    6. Other recommendations:
%         Use the convenience macros defined in openmp.sty for the minor headers
%         such as Comments, Syntax, etc.
%
%         To keep items together on the same page, prefer the use of 
%         \begin{samepage}.... Avoid \parbox for text blocks as it interrupts line numbering.
%         When possible, avoid \filbreak, \pagebreak, \newpage, \clearpage unless that's
%         what you mean. Use \needspace{} cautiously for troublesome paragraphs.
%
%         Avoid absolute lengths and measures in this file; use relative units when possible.
%         Vertical space can be relative to \baselineskip or ex units. Horizontal space
%         can be relative to \linewidth or em units.
%
%         Prefer \emph{} to italicize terminology, e.g.:
%             This is a \emph{definition}, not a placeholder.
%             This is a \plc{var-name}.
%


\section{Execution Environment Routines}
\index{execution environment routines}
\label{sec:Execution Environment Routines}
This section describes routines that affect and monitor threads, processors, and the 
parallel environment. 






\subsection{\code{omp\_set\_num\_threads}}
\index{omp\_set\_num\_threads@{\code{omp\_set\_num\_threads}}}
\label{subsec:omp_set_num_threads}
\summary
The \code{omp\_set\_num\_threads} routine affects the number of threads to be used for 
subsequent parallel regions that do not specify a \code{num\_threads} clause, by setting the 
value of the first element of the \plc{nthreads-var} ICV of the current task.

\format
\begin{ccppspecific}
\begin{boxedcode}
void omp\_set\_num\_threads(int \plc{num\_threads});
\end{boxedcode}
\end{ccppspecific}

\begin{fortranspecific}
\begin{boxedcode}
subroutine omp\_set\_num\_threads(\plc{num\_threads})
integer \plc{num\_threads}
\end{boxedcode}
\end{fortranspecific}

\constraints
The value of the argument passed to this routine must evaluate to a positive integer, or 
else the behavior of this routine is implementation defined.

\binding
The binding task set for an \code{omp\_set\_num\_threads} region is the generating task.

\effect
The effect of this routine is to set the value of the first element of the \plc{nthreads-var} ICV 
of the current task to the value specified in the argument. 

\crossreferences
\begin{itemize}
\item \plc{nthreads-var} ICV, see 
\specref{sec:Internal Control Variables}.

\item \code{parallel} construct and \code{num\_threads} clause, see 
\specref{sec:parallel Construct}.

\item Determining the number of threads for a \code{parallel} region, see
\specref{subsec:Determining the Number of Threads for a parallel Region}. 

\item \code{omp\_get\_max\_threads} routine, see 
\specref{subsec:omp_get_max_threads}.

\item \code{OMP\_NUM\_THREADS} environment variable, see 
\specref{sec:OMP_NUM_THREADS}.
\end{itemize}









\subsection{\code{omp\_get\_num\_threads}}
\index{omp\_get\_num\_threads@{\code{omp\_get\_num\_threads}}}
\label{subsec:omp_get_num_threads}
\summary
The \code{omp\_get\_num\_threads} routine returns the number of threads in the current 
team.
\format
\begin{ccppspecific}
\begin{boxedcode}
int omp\_get\_num\_threads(void); 
\end{boxedcode}
\end{ccppspecific}

\begin{fortranspecific}
\begin{boxedcode}
integer function omp\_get\_num\_threads()
\end{boxedcode}
\end{fortranspecific}

\binding
The binding region for an \code{omp\_get\_num\_threads} region is the innermost enclosing 
\code{parallel} region.

\effect
The \code{omp\_get\_num\_threads} routine returns the number of threads in the team 
executing the \code{parallel} region to which the routine region binds. If called from the 
sequential part of a program, this routine returns 1. 

\crossreferences
\begin{itemize}
\item \code{parallel} construct, see 
\specref{sec:parallel Construct}.

\item Determining the number of threads for a \code{parallel} region, see
\specref{subsec:Determining the Number of Threads for a parallel Region}. 

\item \code{omp\_set\_num\_threads} routine, see 
\specref{subsec:omp_set_num_threads}.

\item \code{OMP\_NUM\_THREADS} environment variable, see 
\specref{sec:OMP_NUM_THREADS}.
\end{itemize}










\subsection{\code{omp\_get\_max\_threads}}
\index{omp\_get\_max\_threads@{\code{omp\_get\_max\_threads}}}
\label{subsec:omp_get_max_threads}
\summary
The \code{omp\_get\_max\_threads} routine returns an upper bound on the number of 
threads that could be used to form a new team if a \code{parallel} construct without a 
\code{num\_threads} clause were encountered after execution returns from this routine.

\format
\begin{ccppspecific}
\begin{boxedcode}
int omp\_get\_max\_threads(void);
\end{boxedcode}
\end{ccppspecific}

\begin{fortranspecific}
\begin{boxedcode}
integer function omp\_get\_max\_threads()
\end{boxedcode}
\end{fortranspecific}

\binding
The binding task set for an \code{omp\_get\_max\_threads} region is the generating task. 

\effect
The value returned by \code{omp\_get\_max\_threads} is the value of the first element of 
the \plc{nthreads-var} ICV of the current task. This value is also an upper bound on the 
number of threads that could be used to form a new team if a parallel region without a 
\code{num\_threads} clause were encountered after execution returns from this routine.

\begin{note}
The return value of the \code{omp\_get\_max\_threads} routine can be used to 
dynamically allocate sufficient storage for all threads in the team formed at the 
subsequent active \code{parallel} region.
\end{note}

\crossreferences
\begin{itemize}
\item \plc{nthreads-var} ICV, see 
\specref{sec:Internal Control Variables}.

\item \code{parallel} construct, see 
\specref{sec:parallel Construct}.

\item \code{num\_threads} clause, see 
\specref{sec:parallel Construct}.

\item Determining the number of threads for a \code{parallel} region, see
\specref{subsec:Determining the Number of Threads for a parallel Region}. 

\item \code{omp\_set\_num\_threads} routine, see 
\specref{subsec:omp_set_num_threads}.

\item \code{OMP\_NUM\_THREADS} environment variable, see 
\specref{sec:OMP_NUM_THREADS}.
\end{itemize}









\newpage %% HACK

\subsection{\code{omp\_get\_thread\_num}}
\index{omp\_get\_thread\_num@{\code{omp\_get\_thread\_num}}}
\label{subsec:omp_get_thread_num}
\summary
The \code{omp\_get\_thread\_num} routine returns the thread number, within the current 
team, of the calling thread.

\format
\begin{ccppspecific}
\begin{boxedcode}
int omp\_get\_thread\_num(void); 
\end{boxedcode}
\end{ccppspecific}

\begin{fortranspecific}
\begin{boxedcode}
integer function omp\_get\_thread\_num() 
\end{boxedcode}
\end{fortranspecific}

\binding
The binding thread set for an \code{omp\_get\_thread\_num} region is the current team. The 
binding region for an \code{omp\_get\_thread\_num} region is the innermost enclosing 
\code{parallel} region. 

\effect
The \code{omp\_get\_thread\_num} routine returns the thread number of the calling thread, 
within the team executing the \code{parallel} region to which the routine region binds. The 
thread number is an integer between 0 and one less than the value returned by 
\code{omp\_get\_num\_threads}, inclusive. The thread number of the master thread of the 
team is 0. The routine returns 0 if it is called from the sequential part of a program.

\begin{note}
The thread number may change during the execution of an untied task. The 
value returned by \code{omp\_get\_thread\_num} is not generally useful during the execution 
of such a task region.
\end{note}

\crossreferences
\begin{itemize}
\item \code{omp\_get\_num\_threads} routine, see 
\specref{subsec:omp_get_num_threads}.
\end{itemize}








\subsection{\code{omp\_get\_num\_procs}}
\index{omp\_get\_num\_procs@{\code{omp\_get\_num\_procs}}}
\label{subsec:omp_get_num_procs}
\summary
The \code{omp\_get\_num\_procs} routine returns the number of processors available to the 
device.

\format
\begin{ccppspecific}
\begin{boxedcode}
int omp\_get\_num\_procs(void);
\end{boxedcode}
\end{ccppspecific}

\begin{fortranspecific}
\begin{boxedcode}
integer function omp\_get\_num\_procs()
\end{boxedcode}
\end{fortranspecific}

\binding
The binding thread set for an \code{omp\_get\_num\_procs} region is all threads on a device. 
The effect of executing this routine is not related to any specific region corresponding to 
any construct or API routine.

\effect
The \code{omp\_get\_num\_procs} routine returns the number of processors that are available 
to the device at the time the routine is called. This value may change between 
the time that it is determined by the \code{omp\_get\_num\_procs} routine and the time that it 
is read in the calling context due to system actions outside the control of the OpenMP 
implementation.

\crossreferences
None.







\subsection{\code{omp\_in\_parallel}}
\index{omp\_in\_parallel@{\code{omp\_in\_parallel}}}
\label{subsec:omp_in_parallel}
\summary
The \code{omp\_in\_parallel} routine returns \plc{true} if the \plc{active-levels-var} ICV is greater 
than zero; otherwise, it returns \plc{false}.

\pagebreak
\format
\begin{ccppspecific}
\begin{boxedcode}
int omp\_in\_parallel(void);
\end{boxedcode}
\end{ccppspecific}

\begin{fortranspecific}
\begin{boxedcode}
logical function omp\_in\_parallel()
\end{boxedcode}
\end{fortranspecific}

\binding
The binding task set for an \code{omp\_in\_parallel} region is the generating task.

\effect
The effect of the \code{omp\_in\_parallel} routine is to return \plc{true} if the current task is 
enclosed by an active \code{parallel} region, and the \code{parallel} region is enclosed by the 
outermost initial task region on the device; otherwise it returns \plc{false}.

\crossreferences
\begin{itemize}
\item \plc{active-levels-var}, see 
\specref{sec:Internal Control Variables}.

\item \code{parallel} construct, see 
\specref{sec:parallel Construct}.

\item \code{omp\_get\_active\_level} routine, see 
\specref{subsec:omp_get_active_level}.
\end{itemize}








\bigskip
\subsection{\code{omp\_set\_dynamic}}
\index{omp\_set\_dynamic@{\code{omp\_set\_dynamic}}}
\label{subsec:omp_set_dynamic}
\summary
The \code{omp\_set\_dynamic} routine enables or disables dynamic adjustment of the 
number of threads available for the execution of subsequent \code{parallel} regions by 
setting the value of the \plc{dyn-var} ICV.


\pagebreak
\format
\begin{ccppspecific}
\begin{boxedcode}
void omp\_set\_dynamic(int \plc{dynamic\_threads});
\end{boxedcode}
\end{ccppspecific}
\bigskip

\begin{samepage}
\begin{fortranspecific}
\begin{boxedcode}
subroutine omp\_set\_dynamic(\plc{dynamic\_threads})
logical \plc{dynamic\_threads}
\end{boxedcode}
\end{fortranspecific}
\end{samepage}

\binding
The binding task set for an \code{omp\_set\_dynamic} region is the generating task. 

\effect
For implementations that support dynamic adjustment of the number of threads, if the 
argument to \code{omp\_set\_dynamic} evaluates to \plc{true}, dynamic adjustment is enabled for 
the current task; otherwise, dynamic adjustment is disabled for the current task. For 
implementations that do not support dynamic adjustment of the number of threads this 
routine has no effect: the value of \plc{dyn-var} remains \plc{false}.

\crossreferences
\begin{itemize}
\item \plc{dyn-var} ICV, see 
\specref{sec:Internal Control Variables}.

\item Determining the number of threads for a \code{parallel} region, see
\specref{subsec:Determining the Number of Threads for a parallel Region}. 

\item \code{omp\_get\_num\_threads} routine, see 
\specref{subsec:omp_get_num_threads}.

\item \code{omp\_get\_dynamic} routine, see 
\specref{subsec:omp_get_dynamic}.

\item \code{OMP\_DYNAMIC} environment variable, see 
\specref{sec:OMP_DYNAMIC}.
\end{itemize}








\subsection{\code{omp\_get\_dynamic}}
\index{omp\_get\_dynamic@{\code{omp\_get\_dynamic}}}
\label{subsec:omp_get_dynamic}
\summary
The \code{omp\_get\_dynamic} routine returns the value of the \plc{dyn-var} ICV, which 
determines whether dynamic adjustment of the number of threads is enabled or disabled.

\format
\begin{ccppspecific}
\begin{boxedcode}
int omp\_get\_dynamic(void);
\end{boxedcode}
\end{ccppspecific}

\begin{fortranspecific}
\begin{boxedcode}
logical function omp\_get\_dynamic()
\end{boxedcode}
\end{fortranspecific}

\binding
The binding task set for an \code{omp\_get\_dynamic} region is the generating task. 

\effect
This routine returns \plc{true} if dynamic adjustment of the number of threads is enabled for 
the current task; it returns \plc{false}, otherwise. If an implementation does not support 
dynamic adjustment of the number of threads, then this routine always returns \plc{false}.

\crossreferences
\begin{itemize}
\item \plc{dyn-var} ICV, see 
\specref{sec:Internal Control Variables}.

\item Determining the number of threads for a \code{parallel} region, see
\specref{subsec:Determining the Number of Threads for a parallel Region}. 

\item \code{omp\_set\_dynamic} routine, see 
\specref{subsec:omp_set_dynamic}.

\item \code{OMP\_DYNAMIC} environment variable, see 
\specref{sec:OMP_DYNAMIC}.
\end{itemize}









\subsection{\code{omp\_get\_cancellation}}
\index{omp\_get\_cancellation@{\code{omp\_get\_cancellation}}}
\label{subsec:omp_get_cancellation}
\summary
The \code{omp\_get\_cancellation} routine returns the value of the \plc{cancel-var} ICV, which determines if cancellation is enabled or disabled.

\format
\begin{ccppspecific}
\begin{boxedcode}
int omp\_get\_cancellation(void);
\end{boxedcode}
\end{ccppspecific}

\begin{fortranspecific}
\begin{boxedcode}
logical function omp\_get\_cancellation()
\end{boxedcode}
\end{fortranspecific}

\binding
The binding task set for an \code{omp\_get\_cancellation} region is the whole program.

\effect
This routine returns \plc{true} if cancellation is enabled. It returns \plc{false} otherwise.

\crossreferences
\begin{itemize}
\item \plc{cancel-var} ICV, see 
\specref{subsec:ICV Descriptions}.

\item \code{cancel} construct, see \specref{subsec:cancel Construct}

\item \code{OMP\_CANCELLATION} environment variable, see 
\specref{sec:OMP_CANCELLATION}
\end{itemize}









\subsection{\code{omp\_set\_nested}}
\index{omp\_set\_nested@{\code{omp\_set\_nested}}}
\label{subsec:omp_set_nested}
\summary
The \code{omp\_set\_nested} routine enables or disables nested parallelism, by setting the 
\plc{nest-var} ICV. 

\format
\begin{ccppspecific}
\begin{boxedcode}
void omp\_set\_nested(int \plc{nested});
\end{boxedcode}
\end{ccppspecific}

\begin{fortranspecific}
\begin{boxedcode}
subroutine omp\_set\_nested(\plc{nested})
logical \plc{nested}
\end{boxedcode}
\end{fortranspecific}

\binding
The binding task set for an \code{omp\_set\_nested} region is the generating task. 

\effect
For implementations that support nested parallelism, if the argument to 
\code{omp\_set\_nested} evaluates to \plc{true}, nested parallelism is enabled for the current task; 
otherwise, nested parallelism is disabled for the current task. For implementations that 
do not support nested parallelism, this routine has no effect: the value of \plc{nest-var} 
remains \plc{false}.

\crossreferences
\begin{itemize}
\item \plc{nest-var} ICV, see 
\specref{sec:Internal Control Variables}.

\item Determining the number of threads for a \code{parallel} region, see
\specref{subsec:Determining the Number of Threads for a parallel Region}. 

\item \code{omp\_set\_max\_active\_levels} routine, see 
\specref{subsec:omp_set_max_active_levels}.

\item \code{omp\_get\_max\_active\_levels} routine, see 
\specref{subsec:omp_get_max_active_levels}.

\item \code{omp\_get\_nested} routine, see 
\specref{subsec:omp_get_nested}.

\item \code{OMP\_NESTED} environment variable, see 
\specref{sec:OMP_NESTED}.
\end{itemize}








\subsection{\code{omp\_get\_nested}}
\index{omp\_get\_nested@{\code{omp\_get\_nested}}}
\label{subsec:omp_get_nested}
\summary
The \code{omp\_get\_nested} routine returns the value of the \plc{nest-var} ICV, which 
determines if nested parallelism is enabled or disabled.

\format
\begin{ccppspecific}
\begin{boxedcode}
int omp\_get\_nested(void);
\end{boxedcode}
\end{ccppspecific}

\begin{fortranspecific}
\begin{boxedcode}
logical function omp\_get\_nested()
\end{boxedcode}
\end{fortranspecific}

\binding
The binding task set for an \code{omp\_get\_nested} region is the generating task. 

\effect
This routine returns \plc{true} if nested parallelism is enabled for the current task; it returns 
\plc{false}, otherwise. If an implementation does not support nested parallelism, this routine 
always returns \plc{false}.

\crossreferences
\begin{itemize}
\item \plc{nest-var} ICV, see 
\specref{sec:Internal Control Variables}.

\item Determining the number of threads for a \code{parallel} region, see
\specref{subsec:Determining the Number of Threads for a parallel Region}. 

\item \code{omp\_set\_nested} routine, see 
\specref{subsec:omp_set_nested}.

\item \code{OMP\_NESTED} environment variable, see 
\specref{sec:OMP_NESTED}.
\end{itemize}









\subsection{\code{omp\_set\_schedule}}
\index{omp\_set\_schedule@{\code{omp\_set\_schedule}}}
\label{subsec:omp_set_schedule}
\summary
The \code{omp\_set\_schedule} routine affects the schedule that is applied when \code{runtime} 
is used as schedule kind, by setting the value of the \plc{run-sched-var} ICV. 

\format
\begin{ccppspecific}
\begin{boxedcode}
void omp\_set\_schedule(omp\_sched\_t \plc{kind}, int \plc{chunk\_size});
\end{boxedcode}
\end{ccppspecific}

\begin{fortranspecific}
\begin{boxedcode}
subroutine omp\_set\_schedule(\plc{kind}, \plc{chunk\_size}) 
integer (kind=omp\_sched\_kind) \plc{kind}
integer \plc{chunk\_size}
\end{boxedcode}
\end{fortranspecific}

\constraints
The first argument passed to this routine can be one of the valid OpenMP schedule kinds 
(except for \code{runtime}) or any implementation specific schedule. The C/C++ header file 
(\code{omp.h}) and the Fortran include file (\code{omp\_lib.h}) and/or Fortran~90 module file 
(\code{omp\_lib}) define the valid constants. The valid constants must include the following, 
which can be extended with implementation specific values:

\pagebreak
\begin{ccppspecific}
\begin{boxedcode}
typedef enum omp\_sched\_t \{
    omp\_sched\_static = 1,
    omp\_sched\_dynamic = 2,
    omp\_sched\_guided = 3,
    omp\_sched\_auto = 4
\} omp\_sched\_t;
\end{boxedcode}
\end{ccppspecific}

\begin{samepage}
\begin{fortranspecific}
\begin{boxedcode}
integer(kind=omp\_sched\_kind), parameter :: omp\_sched\_static = 1
integer(kind=omp\_sched\_kind), parameter :: omp\_sched\_dynamic = 2
integer(kind=omp\_sched\_kind), parameter :: omp\_sched\_guided = 3
integer(kind=omp\_sched\_kind), parameter :: omp\_sched\_auto = 4
\end{boxedcode}
\end{fortranspecific}
\end{samepage}

\binding
The binding task set for an \code{omp\_set\_schedule} region is the generating task. 

\effect
The effect of this routine is to set the value of the \plc{run-sched-var} ICV of the current task 
to the values specified in the two arguments. The schedule is set to the schedule type 
specified by the first argument \plc{kind}. It can be any of the standard schedule types or 
any other implementation specific one. For the schedule types \code{static}, \code{dynamic}, and 
\code{guided} the \plc{chunk\_size} is set to the value of the second argument, or to the default 
\plc{chunk\_size} if the value of the second argument is less than 1; for the schedule type 
\code{auto} the second argument has no meaning; for implementation specific schedule types, 
the values and associated meanings of the second argument are implementation defined.

\crossreferences
\begin{itemize}
\item \plc{run-sched-var} ICV, see 
\specref{sec:Internal Control Variables}.

\item Determining the schedule of a worksharing loop, see 
\specref{subsubsec:Determining the Schedule of a Worksharing Loop}.

\item \code{omp\_get\_schedule} routine, see 
\specref{subsec:omp_get_schedule}.

\item \code{OMP\_SCHEDULE} environment variable, see 
\specref{sec:OMP_SCHEDULE}.
\end{itemize}









\subsection{\code{omp\_get\_schedule}}
\index{omp\_get\_schedule@{\code{omp\_get\_schedule}}}
\label{subsec:omp_get_schedule}
\summary
The \code{omp\_get\_schedule} routine returns the schedule that is applied when the 
runtime schedule is used. 
\format
\begin{ccppspecific}
\begin{boxedcode}
void omp\_get\_schedule(omp\_sched\_t * \plc{kind}, int * \plc{chunk\_size}); 
\end{boxedcode}
\end{ccppspecific}

\begin{fortranspecific}
\begin{boxedcode}
subroutine omp\_get\_schedule(\plc{kind}, \plc{chunk\_size}) 
integer (kind=omp\_sched\_kind) \plc{kind} 
integer \plc{chunk\_size}
\end{boxedcode}
\end{fortranspecific}

\binding
The binding task set for an \code{omp\_get\_schedule} region is the generating task. 

\effect
This routine returns the \plc{run-sched-var} ICV in the task to which the routine binds. The 
first argument \plc{kind} returns the schedule to be used. It can be any of the standard 
schedule types as defined in 
\specref{subsec:omp_set_schedule}, 
or any implementation specific 
schedule type. The second argument is interpreted as in the \code{omp\_set\_schedule} call, 
defined in 
\specref{subsec:omp_set_schedule}.

\crossreferences
\begin{itemize}
\item \plc{run-sched-var} ICV, see 
\specref{sec:Internal Control Variables}.

\item Determining the schedule of a worksharing loop, see 
\specref{subsubsec:Determining the Schedule of a Worksharing Loop}.

\item \code{omp\_set\_schedule} routine, see 
\specref{subsec:omp_set_schedule}.

\item \code{OMP\_SCHEDULE} environment variable, see 
\specref{sec:OMP_SCHEDULE}.
\end{itemize}









\subsection{\code{omp\_get\_thread\_limit}}
\index{omp\_get\_thread\_limit@{\code{omp\_get\_thread\_limit}}}
\label{subsec:omp_get_thread_limit}
\summary
The \code{omp\_get\_thread\_limit} routine returns the maximum number of OpenMP 
threads available to participate in the current contention group. 

\format
\begin{ccppspecific}
\begin{boxedcode}
int omp\_get\_thread\_limit(void);
\end{boxedcode}
\end{ccppspecific}

\begin{fortranspecific}
\begin{boxedcode}
integer function omp\_get\_thread\_limit()
\end{boxedcode}
\end{fortranspecific}

\binding
The binding thread set for an \code{omp\_get\_thread\_limit} region is all threads on the 
device. The effect of executing this routine is not related to any specific region 
corresponding to any construct or API routine. 

\effect
The \code{omp\_get\_thread\_limit} routine returns the value of the \plc{thread-limit-var} ICV.

\crossreferences
\begin{itemize}
\item \plc{thread-limit-var} ICV, see 
\specref{sec:Internal Control Variables}.

\item \code{OMP\_THREAD\_LIMIT} environment variable, see 
\specref{sec:OMP_THREAD_LIMIT}.
\end{itemize}









\subsection{\code{omp\_set\_max\_active\_levels}}
\index{omp\_set\_max\_active\_levels@{\code{omp\_set\_max\_active\_levels}}}
\label{subsec:omp_set_max_active_levels}
\summary
The \code{omp\_set\_max\_active\_levels} routine limits the number of nested active 
parallel regions on the device, by setting the \plc{max-active-levels-var} ICV

\format
\begin{ccppspecific}
\begin{boxedcode}
void omp\_set\_max\_active\_levels(int \plc{max\_levels});
\end{boxedcode}
\end{ccppspecific}

\begin{fortranspecific}
\begin{boxedcode}
subroutine omp\_set\_max\_active\_levels(\plc{max\_levels})
integer \plc{max\_levels}
\end{boxedcode}
\end{fortranspecific}

\constraints
The value of the argument passed to this routine must evaluate to a non-negative integer, 
otherwise the behavior of this routine is implementation defined.

\binding
When called from a sequential part of the program, the binding thread set for an 
\code{omp\_set\_max\_active\_levels} region is the encountering thread. When called 
from within any explicit parallel region, the binding thread set (and binding region, if 
required) for the \code{omp\_set\_max\_active\_levels} region is implementation defined. 

\effect
The effect of this routine is to set the value of the \plc{max-active-levels-var} ICV to the value 
specified in the argument. 

If the number of parallel levels requested exceeds the number of levels of parallelism 
supported by the implementation, the value of the \plc{max-active-levels-var} ICV will be set 
to the number of parallel levels supported by the implementation.

This routine has the described effect only when called from a sequential part of the 
program. When called from within an explicit \code{parallel} region, the effect of this 
routine is implementation defined.

\crossreferences
\begin{itemize}
\item \plc{max-active-levels-var} ICV, see 
\specref{sec:Internal Control Variables}.

\item \code{omp\_get\_max\_active\_levels} routine, see 
\specref{subsec:omp_get_max_active_levels}.

\item \code{OMP\_MAX\_ACTIVE\_LEVELS} environment variable, see 
\specref{sec:OMP_MAX_ACTIVE_LEVELS}.
\end{itemize}










\subsection{\code{omp\_get\_max\_active\_levels}}
\index{omp\_get\_max\_active\_levels@{\code{omp\_get\_max\_active\_levels}}}
\label{subsec:omp_get_max_active_levels}
\summary
The \code{omp\_get\_max\_active\_levels} routine returns the value of the 
\plc{max-active-levels-var} ICV, which determines the maximum number of nested active parallel regions 
on the device. 

\format
\begin{ccppspecific}
\begin{boxedcode}
int omp\_get\_max\_active\_levels(void);
\end{boxedcode}
\end{ccppspecific}

\begin{fortranspecific}
\begin{boxedcode}
integer function omp\_get\_max\_active\_levels()
\end{boxedcode}
\end{fortranspecific}

\binding
When called from a sequential part of the program, the binding thread set for an 
\code{omp\_get\_max\_active\_levels} region is the encountering thread. When called 
from within any explicit parallel region, the binding thread set (and binding region, if 
required) for the \code{omp\_get\_max\_active\_levels} region is implementation defined. 

\effect
The \code{omp\_get\_max\_active\_levels} routine returns the value of the \plc{max-active-levels-var} ICV,
which determines the maximum number of nested active parallel regions 
on the device. 

\crossreferences
\begin{itemize}
\item \plc{max-active-levels-var} ICV, see 
\specref{sec:Internal Control Variables}.

\item \code{omp\_set\_max\_active\_levels} routine, see 
\specref{subsec:omp_set_max_active_levels}.

\item \code{OMP\_MAX\_ACTIVE\_LEVELS} environment variable, see 
\specref{sec:OMP_MAX_ACTIVE_LEVELS}.
\end{itemize}








\subsection{\code{omp\_get\_level}}
\index{omp\_get\_level@{\code{omp\_get\_level}}}
\label{subsec:omp_get_level}
\summary
The \code{omp\_get\_level} routine returns the value of the \plc{levels-var} ICV. 

\format
\begin{ccppspecific}
\begin{boxedcode}
int omp\_get\_level(void);
\end{boxedcode}
\end{ccppspecific}

\begin{fortranspecific}
\begin{boxedcode}
integer function omp\_get\_level()
\end{boxedcode}
\end{fortranspecific}

\binding
The binding task set for an \code{omp\_get\_level} region is the generating task. 

\effect
The effect of the \code{omp\_get\_level} routine is to return the number of nested 
\code{parallel} regions (whether active or inactive) enclosing the current task such that all 
of the \code{parallel} regions are enclosed by the outermost initial task region on the 
current device.

\crossreferences
\begin{itemize}
\item \plc{levels-var} ICV, see 
\specref{sec:Internal Control Variables}.

\item \code{omp\_get\_active\_level} routine, see 
\specref{subsec:omp_get_active_level}.

\item \code{OMP\_MAX\_ACTIVE\_LEVELS} environment variable, see 
\specref{sec:OMP_MAX_ACTIVE_LEVELS}.
\end{itemize}










\subsection{\code{omp\_get\_ancestor\_thread\_num}}
\index{omp\_get\_ancestor\_thread\_num@{\code{omp\_get\_ancestor\_thread\_num}}}
\label{subsec:omp_get_ancestor_thread_num}
\summary
The \code{omp\_get\_ancestor\_thread\_num} routine returns, for a given nested level of 
the current thread, the thread number of the ancestor of the current thread.

\begin{samepage}
\format
\begin{ccppspecific}
\begin{boxedcode}
int omp\_get\_ancestor\_thread\_num(int \plc{level});
\end{boxedcode}
\end{ccppspecific}
\end{samepage}

\begin{fortranspecific}
\begin{boxedcode}
integer function omp\_get\_ancestor\_thread\_num(\plc{level})
integer \plc{level}
\end{boxedcode}
\end{fortranspecific}

\binding
The binding thread set for an \code{omp\_get\_ancestor\_thread\_num} region is the 
encountering thread. The binding region for an \code{omp\_get\_ancestor\_thread\_num} 
region is the innermost enclosing \code{parallel} region. 

\effect
The \code{omp\_get\_ancestor\_thread\_num} routine returns the thread number of the 
ancestor at a given nest level of the current thread or the thread number of the current 
thread. If the requested nest level is outside the range of 0 and the nest level of the 
current thread, as returned by the \code{omp\_get\_level} routine, the routine returns -1.

\begin{note}
When the \code{omp\_get\_ancestor\_thread\_num} routine is called with a value 
of \code{level}=0, the routine always returns 0. If \code{level}=\code{omp\_get\_level()}, the routine 
has the same effect as the \code{omp\_get\_thread\_num} routine. 
\end{note}

\crossreferences
\begin{itemize}
\item \code{omp\_get\_thread\_num} routine, see 
\specref{subsec:omp_get_thread_num}.

\item \code{omp\_get\_level} routine, see 
\specref{subsec:omp_get_level}.

\item \code{omp\_get\_team\_size} routine, see 
\specref{subsec:omp_get_team_size}.
\end{itemize}










\subsection{\code{omp\_get\_team\_size}}
\index{omp\_get\_team\_size@{\code{omp\_get\_team\_size}}}
\label{subsec:omp_get_team_size}
\summary
The \code{omp\_get\_team\_size} routine returns, for a given nested level of the current 
thread, the size of the thread team to which the ancestor or the current thread belongs. 

\format
\begin{ccppspecific}
\begin{boxedcode}
int omp\_get\_team\_size(int \plc{level});
\end{boxedcode}
\end{ccppspecific}

\begin{fortranspecific}
\begin{boxedcode}
integer function omp\_get\_team\_size(\plc{level})
integer \plc{level}
\end{boxedcode}
\end{fortranspecific}

\binding
The binding thread set for an \code{omp\_get\_team\_size} region is the encountering 
thread. The binding region for an \code{omp\_get\_team\_size} region is the innermost 
enclosing \code{parallel} region.

\effect
The \code{omp\_get\_team\_size} routine returns the size of the thread team to which the 
ancestor or the current thread belongs. If the requested nested level is outside the range 
of 0 and the nested level of the current thread, as returned by the \code{omp\_get\_level} 
routine, the routine returns -1. Inactive parallel regions are regarded like active parallel 
regions executed with one thread. 

\begin{note}
When the \code{omp\_get\_team\_size} routine is called with a value of \code{level}=0, 
the routine always returns 1. If \code{level}=\code{omp\_get\_level()}, the routine has the same 
effect as the\linebreak \code{omp\_get\_num\_threads} routine. 
\end{note}

\crossreferences
\begin{itemize}
\item \code{omp\_get\_num\_threads} routine, see 
\specref{subsec:omp_get_num_threads}.

\item \code{omp\_get\_level} routine, see 
\specref{subsec:omp_get_level}.

\item \code{omp\_get\_ancestor\_thread\_num} routine, see 
\specref{subsec:omp_get_ancestor_thread_num}.
\end{itemize}









\subsection{\code{omp\_get\_active\_level}}
\index{omp\_get\_active\_level@{\code{omp\_get\_active\_level}}}
\label{subsec:omp_get_active_level}
\summary
The \code{omp\_get\_active\_level} routine returns the value of the \plc{active-level-vars} ICV..

\format
\begin{ccppspecific}
\begin{boxedcode}
int \code{omp\_get\_active\_level}(void);
\end{boxedcode}
\end{ccppspecific}

\pagebreak
\begin{fortranspecific}
\begin{boxedcode}
integer function omp\_get\_active\_level()
\end{boxedcode}
\end{fortranspecific}

\binding
The binding task set for the an \code{omp\_get\_active\_level} region is the generating 
task. 

\effect
The effect of the \code{omp\_get\_active\_level} routine is to return the number of nested, 
active \code{parallel} regions enclosing the current task such that all of the \code{parallel} 
regions are enclosed by the outermost initial task region on the current device. 

\crossreferences
\begin{itemize}
\item \plc{active-levels-var} ICV, see 
\specref{sec:Internal Control Variables}.

\item \code{omp\_get\_level} routine, see 
\specref{subsec:omp_get_level}. 
\end{itemize}











\subsection{\code{omp\_in\_final}}
\index{omp\_in\_final@{\code{omp\_in\_final}}}
\label{subsec:omp_in_final}
\summary
The \code{omp\_in\_final} routine returns \plc{true} if the routine is executed in a final task 
region; otherwise, it returns \plc{false}.

\format
\begin{ccppspecific}
\begin{boxedcode}
int omp\_in\_final(void);
\end{boxedcode}
\end{ccppspecific}

\begin{fortranspecific}
\begin{boxedcode}
logical function omp\_in\_final()
\end{boxedcode}
\end{fortranspecific}

\binding
The binding task set for an \code{omp\_in\_final} region is the generating task.

\effect
\code{omp\_in\_final} returns \plc{true} if the enclosing task region is final. Otherwise, it returns 
\plc{false}.

\crossreferences
\begin{itemize}
\item \code{task} construct, see 
\specref{subsec:task Construct}. 
\end{itemize}









\subsection{\code{omp\_get\_proc\_bind}}
\index{omp\_get\_proc\_bind@{\code{omp\_get\_proc\_bind}}}
\label{subsec:omp_get_proc_bind}
\summary
The \code{omp\_get\_proc\_bind} routine returns the thread affinity policy to be used for the 
subsequent nested \code{parallel} regions that do not specify a \code{proc\_bind} clause.

\format
\begin{ccppspecific}
\begin{boxedcode}
omp\_proc\_bind\_t omp\_get\_proc\_bind(void);
\end{boxedcode}
\end{ccppspecific}

\begin{fortranspecific}
\begin{boxedcode}
integer (kind=omp\_proc\_bind\_kind) function omp\_get\_proc\_bind()
\end{boxedcode}
\end{fortranspecific}

\constraints
The value returned by this routine must be one of the valid affinity policy kinds. The C/
C++ header file (\code{omp.h}) and the Fortran include file (\code{omp\_lib.h}) and/or Fortran~90 
module file (\code{omp\_lib}) define the valid constants. The valid constants must include the 
following:

\begin{ccppspecific}
\begin{codepar}
typedef enum omp\_proc\_bind\_t \{
  omp\_proc\_bind\_false = 0,
  omp\_proc\_bind\_true = 1,
  omp\_proc\_bind\_master = 2,
  omp\_proc\_bind\_close = 3,
  omp\_proc\_bind\_spread = 4
\} omp\_proc\_bind\_t;
\end{codepar}
\end{ccppspecific}

\begin{fortranspecific}
\begin{codepar}
integer (kind=omp\_proc\_bind\_kind), &
                parameter :: omp\_proc\_bind\_false = 0
integer (kind=omp\_proc\_bind\_kind), &
                parameter :: omp\_proc\_bind\_true = 1
integer (kind=omp\_proc\_bind\_kind), &
                parameter :: omp\_proc\_bind\_master = 2
integer (kind=omp\_proc\_bind\_kind), &
                parameter :: omp\_proc\_bind\_close = 3
integer (kind=omp\_proc\_bind\_kind), &
                parameter :: omp\_proc\_bind\_spread = 4
\end{codepar}
\end{fortranspecific}

\binding
The binding task set for an \code{omp\_get\_proc\_bind} region is the generating task

\effect
The effect of this routine is to return the value of the first element of the \plc{bind-var} ICV 
of the current task. See \specref{subsec:Controlling OpenMP Thread Affinity} 
for the rules governing the thread affinity policy.

\crossreferences
\begin{itemize}
\item \plc{bind-var} ICV, see 
\specref{sec:Internal Control Variables}.

\item Controlling OpenMP thread affinity, see 
\specref{subsec:Controlling OpenMP Thread Affinity}. 

\item \code{OMP\_PROC\_BIND} environment variable, see 
\specref{sec:OMP_PROC_BIND}.
\end{itemize}



%%%%%%%%%% begin 392

\subsection{\code{omp\_get\_num\_places}}
\index{omp\_get\_num\_places@{\code{omp\_get\_num\_places}}}
\label{subsec:omp_get_num_places}
\summary
The \code{omp\_get\_num\_places} routine returns the number of places 
available to the execution environment in the place list.

\format
\begin{ccppspecific}
\begin{boxedcode}
int omp\_get\_num\_places(void);
\end{boxedcode}
\end{ccppspecific}

\begin{fortranspecific}
\begin{boxedcode}
integer function omp\_get\_num\_places()
\end{boxedcode}
\end{fortranspecific}

\binding
The binding thread set for an \code{omp\_get\_num\_places}  region is all threads on a device. The effect of executing this routine is not related to any specific region corresponding to any construct or API routine.

\effect

The \code{omp\_get\_num\_places} routine returns the number of places in the place list. This value is equivalent to the number of places in the  \plc{place-partition-var} ICV in the execution environment of the initial task.

\crossreferences
\begin{itemize}
\item \plc{place-partition-var} ICV, see 
\specref{sec:Internal Control Variables}.

\item \code{OMP\_PLACES} environment variable, see 
\specref{sec:OMP_PLACES}.
\end{itemize}





\subsection{\code{omp\_get\_place\_num\_procs}}
\index{omp\_get\_place\_num\_procs@{\code{omp\_get\_place\_num\_procs}}}
\label{subsec:omp_get_place_num_procs}

\summary
The \code{omp\_get\_place\_num\_procs}  routine returns the number of processors available to the execution environment in the specified place.

\format
\begin{ccppspecific}
\begin{boxedcode}
int omp\_get\_place\_num\_procs(int \plc{place\_num});
\end{boxedcode}
\end{ccppspecific}

\begin{fortranspecific}
\begin{boxedcode}
integer function omp\_get\_place\_num\_procs(\plc{place\_num})
integer \plc{place\_num}
\end{boxedcode}
\end{fortranspecific}

\binding
The binding thread set for an \code{omp\_get\_place\_num\_procs}  region is all threads on a device. The effect of executing this routine is not related to any specific region corresponding to any construct or API routine.

\effect
The \code{omp\_get\_place\_num\_procs} routine returns the number of 
processors associated with the place numbered \plc{place\_num}. The 
routine returns zero when \plc{place\_num} is negative, or is equal 
to or larger than the value returned by \code{omp\_get\_num\_places()}. 

\crossreferences
\begin{itemize}
\item \code{OMP\_PLACES} environment variable, see 
\specref{sec:OMP_PLACES}.
\end{itemize}




\subsection{\code{omp\_get\_place\_proc\_ids}}
\index{omp\_get\_place\_proc\_ids@{\code{omp\_get\_place\_proc\_ids}}}
\label{subsec:omp_get_place_proc_ids}

\summary
The \code{omp\_get\_place\_proc\_ids} routine returns the numerical identifiers of the processors available to the execution environment in the specified place.

\format
\begin{ccppspecific}
\begin{boxedcode}
void omp\_get\_place\_proc\_ids(int \plc{place\_num}, int *\plc{ids});
\end{boxedcode}
\end{ccppspecific}

\begin{fortranspecific}
\begin{boxedcode}
subroutine omp\_get\_place\_proc\_ids(\plc{place_num}, \plc{ids})
integer \plc{place\_num}
integer \plc{ids}(*)
\end{boxedcode}
\end{fortranspecific}

\binding
The binding thread set for an \code{omp\_get\_place\_proc\_ids} region is all 
threads on a device. The effect of executing this routine is not related to 
any specific region corresponding to any construct or API routine.

\effect
The \code{omp\_get\_place\_proc\_ids} routine returns the numerical 
identifiers of each processor associated with the place numbered 
\plc{place\_num}. The numerical identifiers are non-negative, and 
their meaning is implementation defined.  The numerical identifiers 
are returned in the array \plc{ids} and their order in the array is 
implementation defined. The array must be sufficiently large to contain 
\code{omp\_get\_place\_num\_procs(}\plc{place\_num}\code{)} integers; 
otherwise, the behavior is unspecified.  The routine has no effect when 
\plc{place\_num} has a negative value, or a value equal or larger 
than \code{omp\_get\_num\_places()}.

\crossreferences
\begin{itemize}
\item \code{omp\_get\_place\_num\_procs} routine, see 
\specref{subsec:omp_get_place_num_procs}.

\item \code{omp\_get\_num\_places} routine, see 
\specref{subsec:omp_get_num_places}.

\item \code{OMP\_PLACES} environment variable, see 
\specref{sec:OMP_PLACES}.
\end{itemize}




\subsection{\code{omp\_get\_place\_num}}
\index{omp\_get\_place\_num@{\code{omp\_get\_place\_num}}}
\label{subsec:omp_get_place_num}

\summary
The \code{omp\_get\_place\_num} routine returns the place number of the place to which the encountering thread is bound.

\format
\begin{ccppspecific}
\begin{boxedcode}
int omp\_get\_place\_num(void);
\end{boxedcode}
\end{ccppspecific}

\begin{fortranspecific}
\begin{boxedcode}
integer function omp\_get\_place\_num()
\end{boxedcode}
\end{fortranspecific}

\binding
The binding thread set for an \code{omp\_get\_place\_num} region is the encountering thread.

\effect
When the encountering thread is bound to a place, the 
\code{omp\_get\_place\_num} routine returns the place number associated 
with the thread. The returned value is between 0 and one less than the 
value returned by \code{omp\_get\_num\_places()}, inclusive. When the 
encountering thread is not bound to a place, the routine returns -1.

\crossreferences
\begin{itemize}
\item Controlling OpenMP thread affinity, see 
\specref{subsec:Controlling OpenMP Thread Affinity}. 

\item \code{omp\_get\_num\_places} routine, see 
\specref{subsec:omp_get_num_places}.

\item \code{OMP\_PLACES} environment variable, see 
\specref{sec:OMP_PLACES}.
\end{itemize}





\subsection{\code{omp\_get\_partition\_num\_places}}
\index{omp\_get\_partition\_num\_places@{\code{omp\_get\_partition\_num\_places}}}
\label{subsec:omp_get_partition_num_places}

\summary
The \code{omp\_get\_partition\_num\_places} routine returns the number of places in the place partition of the innermost implicit task.

\format
\begin{ccppspecific}
\begin{boxedcode}
int omp\_get\_partition\_num\_places(void);
\end{boxedcode}
\end{ccppspecific}

\begin{fortranspecific}
\begin{boxedcode}
integer function omp\_get\_partition\_num\_places()
\end{boxedcode}
\end{fortranspecific}

\binding
The binding task set for an  \code{omp\_get\_partition\_num\_places} region is the encountering implicit task.

\effect
The \code{omp\_get\_partition\_num\_places} routine returns the number of places in the \plc{place-partition-var} ICV.

\crossreferences
\begin{itemize}
\item \plc{place-partition-var} ICV, see 
\specref{sec:Internal Control Variables}.

\item Controlling OpenMP thread affinity, see 
\specref{subsec:Controlling OpenMP Thread Affinity}. 

\item \code{OMP\_PLACES} environment variable, see 
\specref{sec:OMP_PLACES}.
\end{itemize}





\subsection{\code{omp\_get\_partition\_place\_nums}}
\index{omp\_get\_partition\_place\_nums@{\code{omp\_get\_partition\_place\_nums}}}
\label{subsec:omp_get_partition_place_nums}

\summary
The \code{omp\_get\_partition\_place\_nums} routine returns the list of place numbers corresponding to the places in the \plc{place-partition-var} ICV of the innermost implicit task.

\format
\begin{ccppspecific}
\begin{boxedcode}
void omp\_get\_partition\_place\_nums(int *\plc{place\_nums});
\end{boxedcode}
\end{ccppspecific}

\begin{fortranspecific}
\begin{boxedcode}
subroutine omp\_get\_partition\_place\_nums(\plc{place\_nums})
integer \plc{place\_nums}(*)
\end{boxedcode}
\end{fortranspecific}

\binding
The binding task set for an \code{omp\_get\_partition\_place\_nums} region is the encountering implicit task.

\effect
The \code{omp\_get\_partition\_place\_nums} routine returns the list of 
place numbers corresponding to the places in the \plc{place-partition-var} 
ICV of the innermost implicit task. The array must be sufficiently large 
to contain \code{omp\_get\_partition\_num\_places()} integers; otherwise, 
the behavior is unspecified.

\crossreferences
\begin{itemize}
\item \plc{place-partition-var} ICV, see 
\specref{sec:Internal Control Variables}.

\item Controlling OpenMP thread affinity, see 
\specref{subsec:Controlling OpenMP Thread Affinity}. 

\item \code{omp\_get\_partition\_num\_places} routine, see 
\specref{subsec:omp_get_partition_num_places}.

\item \code{OMP\_PLACES} environment variable, see 
\specref{sec:OMP_PLACES}.
\end{itemize}



%%%%%%%%%%


\subsection{\code{omp\_set\_default\_device}}
\index{omp\_set\_default\_device@{\code{omp\_set\_default\_device}}}
\label{subsec:omp_set_default_device}

\summary

The \code{omp\_set\_default\_device} routine controls the default target device by 
assigning the value of the \plc{default-device-var} ICV.

\format
\begin{ccppspecific}
\begin{boxedcode}
void omp\_set\_default\_device(int \plc{device\_num});
\end{boxedcode}
\end{ccppspecific}

\begin{fortranspecific}
\begin{boxedcode}
subroutine omp\_set\_default\_device(\plc{device\_num})
integer \plc{device\_num}
\end{boxedcode}
\end{fortranspecific}

\binding
The binding task set for an \code{omp\_set\_default\_device} region is the generating 
task.

\effect
The effect of this routine is to set the value of the \plc{default-device-var} ICV of the current 
task to the value specified in the argument. When called from within a \code{target} region 
the effect of this routine is unspecified.

\crossreferences
\begin{itemize}
\item \plc{default-device-var}, see 
\specref{sec:Internal Control Variables}.

\item \code{omp\_get\_default\_device}, see 
\specref{subsec:omp_get_default_device}.

\item \code{OMP\_DEFAULT\_DEVICE} environment variable, see 
\specref{sec:OMP_DEFAULT_DEVICE}
\end{itemize}










\subsection{\code{omp\_get\_default\_device}}
\index{omp\_get\_default\_device@{\code{omp\_get\_default\_device}}}
\label{subsec:omp_get_default_device}
\summary
The \code{omp\_get\_default\_device} routine returns the default target device.

\format
\begin{ccppspecific}
\begin{boxedcode}
int omp\_get\_default\_device(void);
\end{boxedcode}
\end{ccppspecific}

\begin{fortranspecific}
\begin{boxedcode}
integer function omp\_get\_default\_device()
\end{boxedcode}
\end{fortranspecific}

\binding
The binding task set for an \code{omp\_get\_default\_device} region is the generating 
task. 

\effect
The \code{omp\_get\_default\_device} routine returns the value of the \plc{default-device-var} 
ICV of the current task. When called from within a \code{target} region the effect of this 
routine is unspecified.

\crossreferences
\begin{itemize}
\item \plc{default-device-var}, see 
\specref{sec:Internal Control Variables}.

\item \code{omp\_set\_default\_device}, see 
\specref{subsec:omp_set_default_device}.

\item \code{OMP\_DEFAULT\_DEVICE} environment variable, see 
\specref{sec:OMP_DEFAULT_DEVICE}. 
\end{itemize}











\subsection{\code{omp\_get\_num\_devices}}
\index{omp\_get\_num\_devices@{\code{omp\_get\_num\_devices}}}
\label{subsec:omp_get_num_devices}
\summary
The \code{omp\_get\_num\_devices} routine returns the number of target devices.

\format
\begin{ccppspecific}
\begin{boxedcode}
int omp\_get\_num\_devices(void);
\end{boxedcode}
\end{ccppspecific}

\begin{fortranspecific}
\begin{boxedcode}
integer function omp\_get\_num\_devices()
\end{boxedcode}
\end{fortranspecific}

\binding
The binding task set for an \code{omp\_get\_num\_devices} region is the generating task.

\effect
The \code{omp\_get\_num\_devices} routine returns the number of available target devices. 
When called from within a \code{target} region the effect of this routine is unspecified.

\crossreferences
None.










\subsection{\code{omp\_get\_num\_teams}}
\index{omp\@{\code{omp\_get\_num\_teams}}}
\label{subsec:omp_get_num_teams}
\summary
The \code{omp\_get\_num\_teams} routine returns the number of teams in the current \code{teams} 
region.

\pagebreak
\format
\begin{ccppspecific}
\begin{boxedcode}
int omp\_get\_num\_teams(void);
\end{boxedcode}
\end{ccppspecific}

\begin{fortranspecific}
\begin{boxedcode}
integer function omp\_get\_num\_teams()
\end{boxedcode}
\end{fortranspecific}

\binding
The binding task set for an \code{omp\_get\_num\_teams} region is the generating task

\effect
The effect of this routine is to return the number of teams in the current \code{teams} region. 
The routine returns 1 if it is called from outside of a \code{teams} region.

\crossreferences
\begin{itemize}
\item \code{teams} construct, see 
\specref{subsec:teams Construct}. 
\end{itemize}










\pagebreak
\subsection{\code{omp\_get\_team\_num}}
\index{omp\_get\_team\_num@{\code{omp\_get\_team\_num}}}
\label{subsec:omp_get_team_num}
\summary
The \code{omp\_get\_team\_num} routine returns the team number of the calling thread.

\format
\begin{ccppspecific}
\begin{boxedcode}
int omp\_get\_team\_num(void);
\end{boxedcode}
\end{ccppspecific}

\begin{fortranspecific}
\begin{boxedcode}
integer function omp\_get\_team\_num()
\end{boxedcode}
\end{fortranspecific}

\binding
The binding task set for an \code{omp\_get\_team\_num} region is the generating task.

\effect
The \code{omp\_get\_team\_num} routine returns the team number of the calling thread. The 
team number is an integer between 0 and one less than the value returned by 
\code{omp\_get\_num\_teams()}, inclusive. The routine returns 0 if it is called outside of a 
\code{teams} region.

\crossreferences
\begin{itemize}
\item \code{teams} construct, see 
\specref{subsec:teams Construct}.

\item \code{omp\_get\_num\_teams} routine, see 
\specref{subsec:omp_get_num_teams}. 
\end{itemize}









\subsection{\code{omp\_is\_initial\_device}}
\index{omp\_is\_initial\_device@{\code{omp\_is\_initial\_device}}}
\label{subsec:omp_is_initial_device}
\summary
The \code{omp\_is\_initial\_device} routine returns \plc{true} if the current task is executing 
on the host device; otherwise, it returns \plc{false}.

\begin{samepage}
\format
\begin{ccppspecific}
\begin{boxedcode}
int omp\_is\_initial\_device(void);
\end{boxedcode}
\end{ccppspecific}
\end{samepage}

\begin{fortranspecific}
\begin{boxedcode}
logical function omp\_is\_initial\_device()
\end{boxedcode}
\end{fortranspecific}

\binding
The binding task set for an \code{omp\_is\_initial\_device} region is the generating task.

\effect
The effect of this routine is to return \plc{true} if the current task is executing on the host 
device; otherwise, it returns \plc{false}.

\crossreferences
\begin{itemize}
\item \code{target} construct, see 
\specref{subsec:target Construct}
\end{itemize}





\subsection{\code{omp\_get\_initial\_device}}
\index{omp\_get\_initial\_device@{\code{omp\_get\_initial\_device}}}
\label{subsec:omp_get_initial_device}
\summary
The \code{omp\_get\_initial\_device} routine returns a device number representing
the host device.

\pagebreak

\begin{samepage}
\format
\begin{ccppspecific}
\begin{boxedcode}
int omp\_get\_initial\_device(void);
\end{boxedcode}
\end{ccppspecific}
\end{samepage}

\begin{fortranspecific}
\begin{boxedcode}
integer function omp\_get\_initial\_device()
\end{boxedcode}
\end{fortranspecific}

\binding
The binding task set for an \code{omp\_get\_initial\_device} region is the generating task.

\effect
The effect of this routine is to return the device number of the host device.
The value of the device number is implementation defined. If it is between 0 
and one less than \code{omp\_get\_num\_devices()} then it is valid for use 
with all device constructs and routines; if it is outside that range, then 
it is only valid for use with the device memory routines and not in the 
\code{device} clause. When called from within a \code{target} region 
the effect of this routine is unspecified.

\crossreferences
\begin{itemize}
\item \code{target} construct, see 
\specref{subsec:target Construct}

\item Device memory routines, see \specref{sec:Device Memory Routines}.
\end{itemize}




\subsection{\code{omp\_get\_max\_task\_priority}}
\index{omp\_get\_max\_task\_priority@{\code{omp\_get\_max\_task\_priority}}}
\label{subsec:omp_get_max_task_priority}
\summary

The \code{omp\_get\_max\_task\_priority} routine returns the maximum value that can be
specified in the \code{priority} clause.

\pagebreak

\begin{samepage}
\format
\begin{ccppspecific}
\begin{boxedcode}
int omp\_get\_max\_task\_priority(void);
\end{boxedcode}
\end{ccppspecific}
\end{samepage}

\begin{fortranspecific}
\begin{boxedcode}
integer function omp\_get\_max\_task\_priority()
\end{boxedcode}
\end{fortranspecific}

\binding

The binding thread set for an \code{omp\_get\_max\_task\_priority} region is all threads
on the device. The effect of executing this routine is not related to any specific region
corresponding to any construct or API routine.

\effect

The \code{omp\_get\_max\_task\_priority} routine returns the value of the \plc{max-task-priority-var}
ICV, which determines the maximum value that can be specified in the \code{priority} clause.

\crossreferences

\begin{itemize}
\item \plc{max-task-priority-var}, see 
\specref{sec:Internal Control Variables}.

\item \code{task} construct, see 
\specref{subsec:task Construct}. 
\end{itemize}










%% \newpage

% This is an included file. See the master file for more information.
%
% When editing this file:
%
%    1. To change formatting, appearance, or style, please edit openmp.sty.
%
%    2. Custom commands and macros are defined in openmp.sty.
%
%    3. Be kind to other editors -- keep a consistent style by copying-and-pasting to
%       create new content.
%
%    4. We use semantic markup, e.g. (see openmp.sty for a full list):
%         \code{}     % for bold monospace keywords, code, operators, etc.
%         \plc{}      % for italic placeholder names, grammar, etc.
%
%    5. There are environments that provide special formatting, e.g. language bars.
%       Please use them whereever appropriate.  Examples are:
%
%         \begin{fortranspecific}
%         This is text that appears enclosed in blue language bars for Fortran.
%         \end{fortranspecific}
%
%         \begin{note}
%         This is a note.  The "Note -- " header appears automatically.
%         \end{note}
%
%    6. Other recommendations:
%         Use the convenience macros defined in openmp.sty for the minor headers
%         such as Comments, Syntax, etc.
%
%         To keep items together on the same page, prefer the use of 
%         \begin{samepage}.... Avoid \parbox for text blocks as it interrupts line numbering.
%         When possible, avoid \filbreak, \pagebreak, \newpage, \clearpage unless that's
%         what you mean. Use \needspace{} cautiously for troublesome paragraphs.
%
%         Avoid absolute lengths and measures in this file; use relative units when possible.
%         Vertical space can be relative to \baselineskip or ex units. Horizontal space
%         can be relative to \linewidth or em units.
%
%         Prefer \emph{} to italicize terminology, e.g.:
%             This is a \emph{definition}, not a placeholder.
%             This is a \plc{var-name}.
%


\section{Lock Routines}
\index{lock routines}
\label{sec:Lock Routines}

\vspace{-8pt}

The OpenMP runtime library includes a set of general-purpose lock routines that can be 
used for synchronization. These general-purpose lock routines operate on OpenMP locks 
that are represented by OpenMP lock variables. OpenMP lock variables must be 
accessed only through the routines described in this section; programs that otherwise 
access OpenMP lock variables are non-conforming.

An OpenMP lock can be in one of the following states: \emph{uninitialized}, \emph{unlocked}, or 
\emph{locked}. If a lock is in the \emph{unlocked} state, a task can \emph{set} the lock, which changes its state 
to \emph{locked}. The task that sets the lock is then said to \emph{own} the lock. A task that owns a 
lock can \emph{unset} that lock, returning it to the \emph{unlocked} state. A program in which a task 
unsets a lock that is owned by another task is non-conforming.

Two types of locks are supported: \emph{simple locks} and \emph{nestable locks}. A \emph{nestable lock} can 
be set multiple times by the same task before being unset; a \emph{simple lock} cannot be set if 
it is already owned by the task trying to set it. \emph{Simple lock} variables are associated with 
\emph{simple locks} and can only be passed to \emph{simple lock} routines. \emph{Nestable lock} variables are 
associated with \emph{nestable locks} and can only be passed to \emph{nestable lock} routines.

Each type of lock can also have a \emph{synchronization hint} that contains information about the intended usage 
of the lock by the application code.  The effect of the hint is
implementation defined.  An OpenMP implementation can use this hint to select a
usage-specific lock, but hints do not change the mutual exclusion semantics of locks. 
A conforming implementation can safely ignore the hint.

Constraints on the state and ownership of the lock accessed by each of the lock routines 
are described with the routine. If these constraints are not met, the behavior of the 
routine is unspecified. 

The OpenMP lock routines access a lock variable such that they always read 
and update the most current value of the lock variable. It is not necessary for an 
OpenMP program to include explicit \code{flush} directives to ensure that the lock variable's 
value is consistent among different tasks. 

\vspace{-8pt}

\binding
The binding thread set for all lock routine regions is all threads in the contention group. 
As a consequence, for each OpenMP lock, the lock routine effects relate to all tasks that 
call the routines, without regard to which teams the threads in the contention group 
executing the tasks belong.


\vspace{-8pt}

\littleheader{Simple Lock Routines}
\index{Simple Lock Routines}
\begin{ccppspecific}
The type \code{omp\_lock\_t} represents a simple lock. For the 
following routines, a simple lock variable must be of \code{omp\_lock\_t} type. All simple 
lock routines require an argument that is a pointer to a variable of type \code{omp\_lock\_t}.
\end{ccppspecific}

\begin{fortranspecific}
For the following routines, a simple lock variable must be an integer variable of 
\code{kind=omp\_lock\_kind}.
\end{fortranspecific}

The simple lock routines are as follows:

\begin{itemize}
\item The \code{omp\_init\_lock} routine initializes a simple lock.

\item The \code{omp\_init\_lock\_with\_hint} routine initializes a simple lock and attaches a hint to it.

\item The \code{omp\_destroy\_lock} routine uninitializes a simple lock.

\item The \code{omp\_set\_lock} routine waits until a simple lock is available, and then sets it.

\item The \code{omp\_unset\_lock} routine unsets a simple lock.

\item The \code{omp\_test\_lock} routine tests a simple lock, and sets it if it is available.
\end{itemize}

\littleheader{Nestable Lock Routines}
\begin{ccppspecific}
The type \code{omp\_nest\_lock\_t} represents a nestable lock. 
For the following routines, a nestable lock variable must be of \code{omp\_nest\_lock\_t} type. 
All nestable lock routines require an argument that is a pointer to a variable of type 
\code{omp\_nest\_lock\_t}.
\end{ccppspecific}

\begin{fortranspecific}
For the following routines, a nestable lock variable must be an integer variable of 
\code{kind=omp\_nest\_lock\_kind}.
\end{fortranspecific}

The nestable lock routines are as follows:

\begin{itemize}
\item The \code{omp\_init\_nest\_lock} routine initializes a nestable lock.

\item The \code{omp\_init\_nest\_lock\_with\_hint} routine initializes a nestable lock and attaches a hint to it.

\item The \code{omp\_destroy\_nest\_lock} routine uninitializes a nestable lock.

\item The \code{omp\_set\_nest\_lock} routine waits until a nestable lock is available, and then 
sets it.

\item The \code{omp\_unset\_nest\_lock} routine unsets a nestable lock.

\item The \code{omp\_test\_nest\_lock} routine tests a nestable lock, and sets it if it is 
available
\end{itemize}

\restrictions
OpenMP lock routines have the following restrictions:

\begin{itemize}
\item The use of the same OpenMP lock in different contention groups results in 
unspecified behavior. 
\end{itemize}










\subsection{\code{omp\_init\_lock} and \code{omp\_init\_nest\_lock}}
\label{subsec:omp_init_lock and omp_init_nest_lock}
\index{omp\_init\_lock@{\code{omp\_init\_lock}}}
\index{omp\_init\_nest\_lock@{\code{omp\_init\_nest\_lock}}}
\summary
These routines initialize an OpenMP lock without a hint.

\format
\begin{ccppspecific}
\begin{boxedcode}
void omp\_init\_lock(omp\_lock\_t *\plc{lock});
void omp\_init\_nest\_lock(omp\_nest\_lock\_t *\plc{lock});
\end{boxedcode}
\end{ccppspecific}

\begin{fortranspecific}
\begin{boxedcode}
subroutine omp\_init\_lock(\plc{svar})
integer (kind=omp\_lock\_kind) \plc{svar}

subroutine omp\_init\_nest\_lock(\plc{nvar})
integer (kind=omp\_nest\_lock\_kind) \plc{nvar}
\end{boxedcode}
\end{fortranspecific}

\constraints
A program that accesses a lock that is not in the uninitialized state through either routine 
is non-conforming.

\effect
The effect of these routines is to initialize the lock to the unlocked state; that is, no task 
owns the lock. In addition, the nesting count for a nestable lock is set to zero.

\events

The \plc{lock-init} or \plc{nest-lock-init} event occurs in the thread 
executing a \code{omp\_init\_lock} or \code{omp\_init\_nest\_lock} region
after initialization of the lock, but before finishing the region.

\tools

A thread dispatches a registered \code{ompt\_callback\_lock\_init}
callback for each occurrence of a \plc{lock-init} or \plc{nest-lock-init} event 
in that thread.  This callback has the type signature \code{ompt\_callback\_mutex\_acquire\_t}.
The callbacks occur in the task encountering the routine.
The callback receives \code{omp\_sync\_hint\_none} as
\plc{hint} argument and
\code{ompt\_mutex\_lock} or 
 \code{ompt\_mutex\_nest\_lock} as \plc{kind} argument as appropriate.

\crossreferences
\begin{itemize}
\item \code{ompt\_callback\_mutex\_acquire\_t}, see
\specref{sec:ompt_callback_mutex_acquire_t}.
\end{itemize}











\subsection[\code{omp\_init\_lock\_with\_hint} and \\
\code{omp\_init\_nest\_lock\_with\_hint}]{\code{omp\_init\_lock\_with\_hint} and \\
\code{omp\_init\_nest\_lock\_with\_hint}}
\label{subsec:omp_init_lock_with_hint and omp_init_nest_lock_with_hint}
\index{omp\_init\_lock@{\code{omp\_init\_lock}}}
\index{omp\_init\_nest\_lock@{\code{omp\_init\_nest\_lock}}}
\summary
These routines initialize an OpenMP lock with a hint.  
The effect of the hint is implementation-defined. The OpenMP implementation
can ignore the hint without changing program semantics.


\format
\begin{ccppspecific}
\begin{boxedcode}
void omp\_init\_lock\_with\_hint(omp\_lock\_t *\plc{lock}, 
                             omp\_sync\_hint\_t \plc{hint});
void omp\_init\_nest\_lock\_with\_hint(omp\_nest\_lock\_t *\plc{lock}, 
                                  omp\_sync\_hint\_t \plc{hint});
\end{boxedcode}
\end{ccppspecific}

\newpage %% HACK
\begin{fortranspecific}
\begin{boxedcode}
subroutine omp\_init\_lock\_with\_hint(\plc{svar}, \plc{hint})
integer (kind=omp\_lock\_kind) \plc{svar}
integer (kind=omp\_sync\_hint\_kind) \plc{hint}

subroutine omp\_init\_nest\_lock\_with\_hint(\plc{nvar}, \plc{hint})
integer (kind=omp\_nest\_lock\_kind) \plc{nvar}
integer (kind=omp\_sync\_hint\_kind) \plc{hint}
\end{boxedcode}
\end{fortranspecific}

\constraints
A program that accesses a lock that is not in the uninitialized state through either routine 
is non-conforming.

The second argument passed to these routines (\plc{hint}) is a hint 
as described in \specref{subsec:Synchronization Hints}.

\effect
The effect of these routines is to initialize the lock to the unlocked state and, optionally, to choose a specific lock implementation based on the hint. 
After initialization no task owns the lock. In addition, the nesting count for a nestable lock is set to zero.

\events

The \plc{lock-init} or \plc{nest-lock-init} event occurs in the thread 
executing a \code{omp\_init\_lock\_with\_hint} or \code{omp\_init\_nest\_lock\_with\_hint} region
after initialization of the lock, but before finishing the region.

\tools

A thread dispatches a registered \code{ompt\_callback\_lock\_init}
callback for each occurrence of a \plc{lock-init} or \plc{nest-lock-init} event 
in that thread.  This callback has the type signature \code{ompt\_callback\_mutex\_acquire\_t}.
The callbacks occur in the task encountering the routine. 
The callback receives the function's \plc{hint} argument as
\plc{hint} argument and
\code{ompt\_mutex\_lock} or 
 \code{ompt\_mutex\_nest\_lock} as \plc{kind} argument as appropriate.

\crossreferences
\begin{itemize}
\item \code{ompt\_callback\_mutex\_acquire\_t}, see
\specref{sec:ompt_callback_mutex_acquire_t}.
\item Synchronization Hints, see \specref{subsec:Synchronization Hints}.
\end{itemize}



\subsection[\code{omp\_destroy\_lock} and \code{omp\_destroy\_nest\_lock}]{\code{omp\_destroy\_lock} and\\ \code{omp\_destroy\_nest\_lock}}
\label{subsec:omp_destroy_lock and omp_destroy_nest_lock}
\index{omp\_destroy\_lock@{\code{omp\_destroy\_lock}}}
\index{omp\_destroy\_nest\_lock@{\code{omp\_destroy\_nest\_lock}}}
\summary
These routines ensure that the OpenMP lock is uninitialized.

\format
\begin{ccppspecific}
\begin{boxedcode}
void omp\_destroy\_lock(omp\_lock\_t *\plc{lock});
void omp\_destroy\_nest\_lock(omp\_nest\_lock\_t *\plc{lock});
\end{boxedcode}
\end{ccppspecific}

\begin{fortranspecific}
\begin{boxedcode}
subroutine omp\_destroy\_lock(\plc{svar})
integer (kind=omp\_lock\_kind) \plc{svar}

subroutine omp\_destroy\_nest\_lock(\plc{nvar})
integer (kind=omp\_nest\_lock\_kind) \plc{nvar}
\end{boxedcode}
\end{fortranspecific}

\constraints
A program that accesses a lock that is not in the unlocked state through either routine is 
non-conforming.

\effect
The effect of these routines is to change the state of the lock to uninitialized.

\events

The \plc{lock-destroy} or \plc{nest-lock-destroy} event occurs in the thread 
executing a \code{omp\_destroy\_lock} or \code{omp\_destroy\_nest\_lock} region
before finishing the region.

\tools

A thread dispatches a registered \code{ompt\_callback\_lock\_destroy}
callback for each occurrence of a \plc{lock-destroy} or \plc{nest-lock-destroy} event 
in that thread.  This callback has the type signature \code{ompt\_callback\_mutex\_t}.
The callbacks occur in the task encountering the routine.
The callbacks receive \code{ompt\_mutex\_lock} or 
 \code{ompt\_mutex\_nest\_lock} as their \plc{kind} argument as appropriate.


\crossreferences
\begin{itemize}
\item \code{ompt\_callback\_mutex\_t}, see 
\specref{sec:ompt_callback_mutex_t}.
\end{itemize}









\subsection{\code{omp\_set\_lock} and \code{omp\_set\_nest\_lock}}
\label{subsec:omp_set_lock and omp_set_nest_lock}
\index{omp\_set\_lock@{\code{omp\_set\_lock}}}
\index{omp\_set\_nest\_lock@{\code{omp\_set\_nest\_lock}}}
\summary
These routines provide a means of setting an OpenMP lock. The calling
task region behaves as if it was suspended until the lock can be set
by this task.

\format
\begin{ccppspecific}
\begin{boxedcode}
void omp\_set\_lock(omp\_lock\_t *\plc{lock});
void omp\_set\_nest\_lock(omp\_nest\_lock\_t *\plc{lock});
\end{boxedcode}
\end{ccppspecific}

\begin{fortranspecific}
\begin{boxedcode}
subroutine omp\_set\_lock(\plc{svar})
integer (kind=omp\_lock\_kind) \plc{svar}

subroutine omp\_set\_nest\_lock(\plc{nvar})
integer (kind=omp\_nest\_lock\_kind) \plc{nvar}
\end{boxedcode}
\end{fortranspecific}

\constraints
A program that accesses a lock that is in the uninitialized state through either routine is 
non-conforming. A simple lock accessed by \code{omp\_set\_lock} that is in the locked state 
must not be owned by the task that contains the call or deadlock will result.

\effect
Each of these routines has an effect equivalent to suspension of the task
executing the routine until the specified lock is available. 

% However, the
% task region may continue to execute speculatively, provided that the 
% implementation ensures that execution is as if the lock were owned
% exclusively by one thread. 

% A note because it's only explanatory. The as-ifness already allows
% this via the ``equivalent to''
\begin{note} The semantics of these routines is specified
\emph{as if} they serialize execution of the region guarded by the
lock. However, implementations may implement them in other ways
provided that the isolation properties are respected so that the
actual execution delivers a result that could arise from some
serialization. 
\end{note}

A simple lock is available if it is unlocked. Ownership of the lock is 
granted to the task executing the routine.

A nestable lock is available if it is unlocked or if it is already owned by 
the task executing the routine. The task executing the routine is granted, 
or retains, ownership of the lock, and the nesting count for the lock is 
incremented.

\events

The \plc{lock-acquire} or \plc{nest-lock-acquire} event occurs in the thread 
executing a \code{omp\_set\_lock} or \code{omp\_set\_nest\_lock} region
before the associated lock is requested.

The \plc{lock-acquired} or \plc{nest-lock-acquired} event occurs in the thread 
executing a \code{omp\_set\_lock} or \code{omp\_set\_nest\_lock} region
after acquiring the associated lock, if the thread did not already own the lock, 
but before finishing the region.

The \plc{nest-lock-owned} event occurs in the thread 
executing a \code{omp\_set\_nest\_lock} region
when the thread already owned the lock, 
before finishing the region.


\tools

A thread dispatches a registered \code{ompt\_callback\_mutex\_acquire}
callback for each occurrence of a \plc{lock-acquire} or \plc{nest-lock-acquire} event 
in that thread.  This callback has the type signature \code{ompt\_callback\_mutex\_acquire\_t}.

A thread dispatches a registered \code{ompt\_callback\_mutex\_acquired}
callback for each occurrence of a \plc{lock-acquired} or \plc{nest-lock-acquired} event 
in that thread.  This callback has the type signature \code{ompt\_callback\_mutex\_t}.

A thread dispatches a registered \code{ompt\_callback\_nest\_lock}
callback for each occurrence of a \plc{nest-lock-owned} event 
in that thread. This callback has the type signature \code{ompt\_callback\_nest\_lock\_t}.
The callback receives \code{ompt\_scope\_begin} as its \plc{endpoint} argument.

The callbacks occur in the task encountering
the lock function.  The callbacks receive \code{ompt\_mutex\_lock} or 
 \code{ompt\_mutex\_nest\_lock} as their \plc{kind} argument, as appropriate.


\crossreferences
\begin{itemize}
\item \code{ompt\_callback\_mutex\_acquire\_t}, see 
\specref{sec:ompt_callback_mutex_acquire_t}.
\item \code{ompt\_callback\_mutex\_t}, see 
\specref{sec:ompt_callback_mutex_t}.
\item \code{ompt\_callback\_nest\_lock\_t}, see
\specref{sec:ompt_callback_nest_lock_t}.
\end{itemize}




\subsection{\code{omp\_unset\_lock} and \code{omp\_unset\_nest\_lock}}
\label{subsec:omp_unset_lock and omp_unset_nest_lock}
\index{omp\_unset\_lock@{\code{omp\_unset\_lock}}}
\index{omp\_unset\_nest\_lock@{\code{omp\_unset\_nest\_lock}}}
\summary
These routines provide the means of unsetting an OpenMP lock.

\format
\begin{ccppspecific}
\begin{boxedcode}
void omp\_unset\_lock(omp\_lock\_t *\plc{lock});
void omp\_unset\_nest\_lock(omp\_nest\_lock\_t *\plc{lock});
\end{boxedcode}
\end{ccppspecific}

\newpage %% HACK
\begin{fortranspecific}
\begin{boxedcode}
subroutine omp\_unset\_lock(\plc{svar})
integer (kind=omp\_lock\_kind) \plc{svar}

subroutine omp\_unset\_nest\_lock(\plc{nvar})
integer (kind=omp\_nest\_lock\_kind) \plc{nvar}
\end{boxedcode}
\end{fortranspecific}

\constraints
A program that accesses a lock that is not in the locked state or that is 
not owned by the task that contains the call through either routine is 
non-conforming.

% Jim Cownie: I don't believe we need anything more about speculation
% here. It has to behave ``as-if'' it held the lock, and released it
% here, and that's what we say happens.

\effect
For a simple lock, the \code{omp\_unset\_lock} routine causes the lock to become unlocked.

For a nestable lock, the \code{omp\_unset\_nest\_lock} routine decrements the nesting 
count, and causes the lock to become unlocked if the resulting nesting count is zero.

For either routine, if the lock becomes unlocked, and if one or more task 
regions were effectively suspended because the lock was unavailable, the 
effect is that one task is chosen and given ownership of the lock. 

\events

The \plc{lock-release} or \plc{nest-lock-release} event occurs in the thread 
executing a \code{omp\_unset\_lock} or \code{omp\_unset\_nest\_lock} region
after releasing the associated lock, but before finishing the region.

The \plc{nest-lock-held} event occurs in the thread 
executing a \code{omp\_unset\_nest\_lock} region
when the thread still owns the lock, 
before finishing the region.


\tools

A thread dispatches a registered \code{ompt\_callback\_mutex\_released}
callback for each occurrence of a \plc{lock-release} or \plc{nest-lock-release} event 
in that thread.  This callback has the type signature \code{ompt\_callback\_mutex\_t}.
The callback occurs in the task encountering
the routine. The callback receives \code{ompt\_mutex\_lock} or 
 \code{ompt\_mutex\_nest\_lock} as \plc{kind} argument as appropriate.

A thread dispatches a registered \code{ompt\_callback\_nest\_lock}
callback for each occurrence of a \plc{nest-lock-held} event 
in that thread. This callback has the type signature \code{ompt\_callback\_nest\_lock\_t}.
The callback receives \code{ompt\_scope\_end} as its \plc{endpoint} argument.

\crossreferences
\begin{itemize}
\item \code{ompt\_callback\_mutex\_t}, see
\specref{sec:ompt_callback_mutex_t}.
\item \code{ompt\_callback\_nest\_lock\_t}, see
\specref{sec:ompt_callback_nest_lock_t}.
\end{itemize}








\subsection{\code{omp\_test\_lock} and \code{omp\_test\_nest\_lock}}
\label{subsec:omp_test_lock and omp_test_nest_lock}
\index{omp\_test\_lock@{\code{omp\_test\_lock}}}
\index{omp\_test\_nest\_lock@{\code{omp\_test\_nest\_lock}}}
\summary
These routines attempt to set an OpenMP lock but do not suspend execution of the task 
executing the routine.

\format
\begin{ccppspecific}
\begin{boxedcode}
int omp\_test\_lock(omp\_lock\_t *\plc{lock});
int omp\_test\_nest\_lock(omp\_nest\_lock\_t *\plc{lock});
\end{boxedcode}
\end{ccppspecific}

\begin{fortranspecific}
\begin{boxedcode}
logical function omp\_test\_lock(\plc{svar})
integer (kind=omp\_lock\_kind) \plc{svar}
integer function omp\_test\_nest\_lock(\plc{nvar})
integer (kind=omp\_nest\_lock\_kind) \plc{nvar}
\end{boxedcode}
\end{fortranspecific}

\constraints
A program that accesses a lock that is in the uninitialized state through either routine is 
non-conforming. The behavior is unspecified if a simple lock accessed by 
\code{omp\_test\_lock} is in the locked state and is owned by the task that contains the call.

\effect 
These routines attempt to set a lock in the same manner as \code{omp\_set\_lock} and 
\code{omp\_set\_nest\_lock}, except that they do not suspend execution of the task 
executing the routine.

For a simple lock, the \code{omp\_test\_lock} routine returns \plc{true} if the lock is successfully 
set; otherwise, it returns \plc{false}.

For a nestable lock, the \code{omp\_test\_nest\_lock} routine returns the new nesting count 
if the lock is successfully set; otherwise, it returns zero.

\events

The \plc{lock-test} or \plc{nest-lock-test} event occurs in the thread 
executing a \code{omp\_test\_lock} or \code{omp\_test\_nest\_lock} region
before the associated lock is tested.

The \plc{lock-test-acquired} or \plc{nest-lock-test-acquired} event occurs in the thread 
executing a \code{omp\_test\_lock} or \code{omp\_test\_nest\_lock} region
before finishing the region if the associated lock was acquired and the thread did not already own the lock. 

The \plc{nest-lock-owned} event occurs in the thread 
executing a \code{omp\_test\_nest\_lock} region
if the thread already owned the lock, 
before finishing the region.


\tools

A thread dispatches a registered \code{ompt\_callback\_mutex\_acquire}
callback for each occurrence of a \plc{lock-test} or \plc{nest-lock-test} event 
in that thread.  This callback has the type signature \code{ompt\_callback\_mutex\_acquire\_t}.

A thread dispatches a registered \code{ompt\_callback\_mutex\_acquired}
callback for each occurrence of a \plc{lock-test-acquired} or \plc{nest-lock-test-acquired} event 
in that thread.  This callback has the type signature \code{ompt\_callback\_mutex\_t}.

A thread dispatches a registered \code{ompt\_callback\_nest\_lock}
callback for each occurrence of a \plc{nest-lock-owned} event 
in that thread. This callback has the type signature \code{ompt\_callback\_nest\_lock\_t}.
The callback receives \code{ompt\_scope\_begin} as its \plc{endpoint} argument.

The callbacks occur in the task encountering
the lock function.  The callbacks receive \code{ompt\_mutex\_lock} or 
 \code{ompt\_mutex\_nest\_lock} as their \plc{kind} argument, as appropriate.


\crossreferences
\begin{itemize}
\item \code{ompt\_callback\_mutex\_acquire\_t}, see 
\specref{sec:ompt_callback_mutex_acquire_t}.
\item \code{ompt\_callback\_mutex\_t}, see 
\specref{sec:ompt_callback_mutex_t}.
\item \code{ompt\_callback\_nest\_lock\_t}, see
\specref{sec:ompt_callback_nest_lock_t}.
\end{itemize}









\section{Timing Routines}
\index{timing routines}
\index{timer}
\index{wall clock timer}
\label{sec:Timing Routines}
This section describes routines that support a portable wall clock timer.

\vspace{-12pt}









\subsection{\code{omp\_get\_wtime}}
\index{omp\_get\_wtime@{\code{omp\_get\_wtime}}}
\label{subsec:omp_get_wtime}
\summary
The \code{omp\_get\_wtime} routine returns elapsed wall clock time in seconds.

\format
\begin{ccppspecific}
\begin{boxedcode}
double omp\_get\_wtime(void);
\end{boxedcode}
\end{ccppspecific}

\begin{fortranspecific}
\begin{boxedcode}
double precision function omp\_get\_wtime()
\end{boxedcode}
\end{fortranspecific}

\binding
The binding thread set for an \code{omp\_get\_wtime} region is the encountering thread. The 
routine's return value is not guaranteed to be consistent across any set of threads.

\effect
The \code{omp\_get\_wtime} routine returns a value equal to the elapsed wall clock time in 
seconds since some ``time in the past''. The actual ``time in the past'' is arbitrary, but it is 
guaranteed not to change during the execution of the application program. The time 
returned is a ``per-thread time'', so it is not required to be globally consistent across all 
threads participating in an application.

\begin{note}
It is anticipated that the routine will be used to measure elapsed times as shown 
in the following example:

\begin{ccppspecific}
\begin{boxedcode}
double start;
double end;
start = omp\_get\_wtime();
\plc{... work to be timed ...}
end = omp\_get\_wtime();
printf("Work took \%f seconds{\textbackslash}n", end - start);
\end{boxedcode}
\end{ccppspecific}

\begin{fortranspecific}
\begin{boxedcode}
DOUBLE PRECISION START, END
START = omp\_get\_wtime()
\plc{... work to be timed ...}
END = omp\_get\_wtime()
PRINT *, "Work took", END - START, "seconds"
\end{boxedcode}
\end{fortranspecific}
\end{note}









\subsection{\code{omp\_get\_wtick}}
\index{omp\_get\_wtick@{\code{omp\_get\_wtick}}}
\label{subsec:omp_get_wtick}
\summary
The \code{omp\_get\_wtick} routine returns the precision of the timer used by 
\code{omp\_get\_wtime}.

\newpage %% HACK
\format
\begin{ccppspecific}
\begin{boxedcode}
double omp\_get\_wtick(void);
\end{boxedcode}
\end{ccppspecific}

\begin{fortranspecific}
\begin{boxedcode}
double precision function omp\_get\_wtick()
\end{boxedcode}
\end{fortranspecific}

\binding
The binding thread set for an \code{omp\_get\_wtick} region is the encountering thread. The 
routine's return value is not guaranteed to be consistent across any set of threads.

\effect
The \code{omp\_get\_wtick} routine returns a value equal to the number of seconds between 
successive clock ticks of the timer used by \code{omp\_get\_wtime}.


\section{Memory Management Routines}
\index{memory management routines}
\label{sec:Memory Management Routines}
This section describes routines that support management of memory on the current device.

Instances of OpenMP memory management types must be accessed only through the routines described in this section; programs that otherwise access OpenMP instances of these types are non-conforming.

\subsection{Memory Management Types}
\label{subsec:Memory Management Types}

The following type definitions are used by the memory management routines:

\ccppspecificstart
omp_allocator_t;
enum \{ OMP_NULL_ALLOCATOR = NULL \};
\end{codepar}
\ccppspecificend

\fortranspecificstart
\begin{codepar}
integer parameter omp_allocator_kind
integer(kind=omp_allocator_kind), &
        parameter :: omp_null_allocator = 0
\end{codepar}
\fortranspecificend

\subsection{\code{omp\_set\_default\_allocator}}
\index{omp\_set\_default\_allocator@{\code{omp\_set\_default\_allocator}}}
\label{subsec:omp_set_default_allocator}

\summary
The \code{omp\_set\_default\_allocator} routine sets the default allocator to be used by allocation calls, directives and clauses that use default allocation.

\format
\ccppspecificstart
\begin{boxedcode}
void omp_set_default_allocator (omp_allocator_t *\plc{allocator});
\end{boxedcode}
\ccppspecificend
\fortranspecificstart
\begin{boxedcode}
subroutine omp_set_default_allocator ( \plc{allocator} )
integer(kind=omp_allocator_kind),intent(in) allocator
\end{boxedcode}
\fortranspecificend
\binding
The binding task set for an \code{omp\_set\_default\_allocator} region is the generating task.

\constraints

The \plc{allocator} argument must be a predefined allocator.

\effect

The effect of this routine is to set the value of the \plc{def-allocator-var} ICV of the current task to the value specified in the \plc{allocator} argument. 

\crossreferences

\begin{itemize}
\item \plc{def-allocator-var} ICV, see \specref{sec:Internal Control Variables}.
\item Memory Allocators, see \specref{sec:Memory Allocators}.
\item \code{omp\_alloc} routine, see \specref{subsec:omp_alloc}.
\end{itemize}

\subsection{\code{omp\_get\_default\_allocator}}
\index{omp\_get\_default\_allocator@{\code{omp\_get\_default\_allocator}}}
\label{subsec:omp_get_default_allocator}

\summary
The \code{omp\_get\_default\_allocator} routine returns the allocator to be used by allocation calls, directives and clauses that use default allocation.

\format
\ccppspecificstart
\begin{boxedcode}
omp_allocator_t * omp_get_default_allocator (void);
\end{boxedcode}
\ccppspecificend
\fortranspecificstart
\begin{boxedcode}
integer(kind=omp_allocator_kind) 
function omp_get_default_allocator ()
\end{boxedcode}
\fortranspecificend

\binding

The binding task set for an \code{omp\_get\_default\_allocator} region is the generating task.

\effect

The effect of this routine is to return the value of the \plc{def-allocator-var} ICV of the current task.

\crossreferences
\begin{itemize}
\item \plc{def-allocator-var} ICV, see \specref{sec:Internal Control Variables}.
\item Memory Allocators, see \specref{sec:Memory Allocators}.
\item \code{omp\_alloc} routine, see \specref{subsec:omp_alloc}.
\end{itemize}

\ccppspecificstart

\subsection{\code{omp\_set\_target\_default\_allocator}}
\index{omp\_set\_target\_default\_allocator@{\code{omp\_set\_target\_default\_allocator}}}
\label{subsec:omp_set_target_default_allocator}

\summary
The \code{omp\_set\_target\_default\_allocator} routine sets the default allocator to be used on a target device by allocation calls, directives and clauses that use default allocation.

\format
\ccppspecificstart
\begin{boxedcode}
void omp_set_target_default_allocator (omp_allocator_t *\plc{allocator});
\end{boxedcode}
\ccppspecificend
\fortranspecificstart
\begin{boxedcode}
subroutine omp_set_target_default_allocator ( \plc{allocator} )
integer(kind=omp_allocator_kind),intent(in) allocator
\end{boxedcode}
\fortranspecificend
\binding
The binding task set for an \code{omp\_set\_target\_default\_allocator} region is the generating task.

\constraints

The \plc{allocator} argument must be a predefined allocator.

\effect

The effect of this routine is to set the value of the \plc{target-def-allocator-var} ICV of the current task to the value specified in the \plc{allocator} argument. 

\crossreferences

\begin{itemize}
\item \plc{target-def-allocator-var} ICV, see \specref{sec:Internal Control Variables}.
\item Memory Allocators, see \specref{sec:Memory Allocators}.
\item \code{omp\_alloc} routine, see \specref{subsec:omp_alloc}.
\end{itemize}

\subsection{\code{omp\_get\_target\_default\_allocator}}
\index{omp\_get\_target\_default\_allocator@{\code{omp\_get\_target\_default\_allocator}}}
\label{subsec:omp_get_target_default_allocator}

\summary
The \code{omp\_get\_target\_default\_allocator} routine returns the allocator to be used on a target device by allocation calls, directives and clauses that use default allocation.

\format
\ccppspecificstart
\begin{boxedcode}
omp_allocator_t * omp_get_target_default_allocator (void);
\end{boxedcode}
\ccppspecificend
\fortranspecificstart
\begin{boxedcode}
integer(kind=omp_allocator_kind) 
function omp_get_target_default_allocator ()
\end{boxedcode}
\fortranspecificend

\binding

The binding task set for an \code{omp\_get\_target\_default\_allocator} region is the generating task.

\effect

The effect of this routine is to return the value of the \plc{target-def-allocator-var} ICV of the current task.

\crossreferences
\begin{itemize}
\item \plc{target-def-allocator-var} ICV, see \specref{sec:Internal Control Variables}.
\item Memory Allocators, see \specref{sec:Memory Allocators}.
\item \code{omp\_alloc} routine, see \specref{subsec:omp_alloc}.
\end{itemize}

\ccppspecificstart


\subsection{\code{omp\_alloc}}
\index{omp\_alloc@{\code{omp\_alloc}}}
\label{subsec:omp_alloc}

\summary
The \code{omp\_alloc} requests a memory allocation to an \plc{allocator}.

\format
\begin{boxedcode}
void * omp_alloc (size_t \plc{size}, omp_allocator_t *\plc{allocator});  (C)
void * omp_alloc (size_t \plc{size}, 
           omp_allocator_t *\plc{allocator}=OMP_NULL_ALLOCATOR); (C++)
\end{boxedcode}

\effect

The \code{omp\_alloc} routine requests a memory allocation of \plc{size} bytes from the specified \plc{allocator} without specifying an allocation alignment. If value of the \plc{allocator} argument is 
\code{OMP\_NULL\_ALLOCATOR} the allocator used by the routine will be the one specified by the \plc{def-allocator-var} ICV.
Upon success it returns a pointer to the allocated memory. Otherwise, the behavior of the call depends on the \code{fallback} trait of the allocator.

\crossreferences
\begin{itemize}
\item How Allocations Works, see \specref{subsec:How Allocation Works}.
\end{itemize}

\subsection{\code{omp\_free}}
\index{omp\_free@{\code{omp\_free}}}
\label{subsec:omp_free}

\summary
The \code{omp\_free} routine deallocates previously allocated memory. 

\format

\begin{boxedcode}
void omp_free ( void * \plc{ptr}, omp\_allocator\_t *\plc{allocator});        (C)
void omp_free ( void * \plc{ptr}, 
             omp\_allocator\_t *\plc{allocator} = OMP_NULL_ALLOCATOR); (C++)
\end{boxedcode}

\effect

The \code{omp\_free} routine deallocates the memory pointed by \plc{ptr}. The \plc{ptr} argument must point to memory previously allocated with an OpenMP allocator. If the \plc{allocator} is specified it must be the allocator to which the allocation request was made. If the \plc{allocator} argument is \code{OMP\_NULL\_ALLOCATOR} the implementation will find the allocator used to allocate the memory. Using \code{omp\_free} on memory that was already deallocated results in unspecified behavior.

\ccppspecificend



    % This is tool_support.tex (Chapter 4) of the OpenMP specification.
% This is an included file. See the master file for more information.
%
% When editing this file:
%
%    1. To change formatting, appearance, or style, please edit openmp.sty.
%
%    2. Custom commands and macros are defined in openmp.sty.
%
%    3. Be kind to other editors -- keep a consistent style by copying-and-pasting to
%       create new content.
%
%    4. We use semantic markup, e.g. (see openmp.sty for a full list):
%         \code{}     % for bold monospace keywords, code, operators, etc.
%         \plc{}      % for italic placeholder names, grammar, etc.
%
%    5. There are environments that provide special formatting, e.g. language bars.
%       Please use them whereever appropriate.  Examples are:
%
%         \begin{fortranspecific}
%         This is text that appears enclosed in blue language bars for Fortran.
%         \end{fortranspecific}
%
%         \begin{note}
%         This is a note.  The "Note -- " header appears automatically.
%         \end{note}
%
%    6. Other recommendations:
%         Use the convenience macros defined in openmp.sty for the minor headers
%         such as Comments, Syntax, etc.
%
%         To keep items together on the same page, prefer the use of
%         \begin{samepage}.... Avoid \parbox for text blocks as it interrupts line numbering.
%         When possible, avoid \filbreak, \pagebreak, \newpage, \clearpage unless that's
%         what you mean. Use \needspace{} cautiously for troublesome paragraphs.
%
%         Avoid absolute lengths and measures in this file; use relative units when possible.
%         Vertical space can be relative to \baselineskip or ex units. Horizontal space
%         can be relative to \linewidth or em units.
%
%         Prefer \emph{} to italicize terminology, e.g.:
%             This is a \emph{definition}, not a placeholder.
%             This is a \plc{var-name}.
%

\newcommand{\devicedesc}{
The argument \plc{device} is a pointer to an opaque object that
represents the target device instance. The pointer to the device
instance object is used by functions in the device tracing interface
to identify the device being addressed.
}

\newcommand{\epdesc}{
The argument \plc{endpoint} indicates whether the callback is
signalling the beginning or the end of a scope.
}

\newcommand\codeptrdesc{
The argument \plc{codeptr_ra} is used to relate the implementation
of an OpenMP region back to its source code.  In cases where a runtime
routine implements the region associated with this callback,
\plc{codeptr_ra} is expected to contain the return address of the
call to the runtime routine.  In cases where the implementation of
this feature is inlined, \plc{codeptr_ra} is expected to contain the
return address of the invocation of this callback.  In cases where
attribution to source code is impossible or inappropriate,
\plc{codeptr_ra} may be \code{NULL}.
}


\chapter{Tool Support}
\index{tool support}
\label{chap:ToolsSupport}

%To enable development of high-quality, portable, \emph{first-party} tools
%that support monitoring and performance analysis of OpenMP programs
%developed using any implementation of the OpenMP API, the OpenMP API
%includes a tool interface known as OMPT.

%OMPD-TODO: rewrite to add debugging interface here. Distrinction of 1st/3rd
%party

This chapter describes OMPT and OMPD, which are a pair of interfaces for first-party and third-party
tools, respectively.  Section~\ref{sec:ompt-overview} describes OMPT---an interface for first-party tools.
The section begins with a description of how to initialize (Section~\ref{sec:ompt-initialization})
and finalize (Section~\ref{sec:ompt-finalization}) a tool.
Subsequent sections describe details of the interface, including
data types shared between an OpenMP implementation and a tool
(Section~\ref{sec:ompt-data-types}),
an interface that enables an OpenMP implementation to determine that a
tool is available (Section~\ref{sec:ompt-check-tool}),
type signatures for tool callbacks
that an OpenMP implementation may dispatch for OpenMP events
(Section~\ref{sec:ompt-tool-callbacks}), and
\emph{runtime entry points}---function interfaces
provided by an OpenMP implementation for use by a tool
(Section~\ref{sec:entry-points}).

Section~\ref{sec:ompd-overview} describes
OMPD---an interface for  third-party tools such as debuggers.
Unlike OMPT, a third-party tool exists in a separate process from
the OpenMP program.
An OpenMP implementation need not maintain any extra information to support OMPD inquiries from third-party tools
\emph{unless} it is explicitly instructed to do so.
Section~\ref{subsec:activating} discusses the mechanisms for
activating support for OMPD in the OpenMP runtime.
Section~\ref{subsec:ompd-data-types}  describes the data types shared between the OMPD library and a third-party tool.
Section~\ref{sec:third-party-tool-callback-interface} describes the API provided by the OMPD library for use by a third-party tool.
An OMPD library will not interact directly with the OpenMP
runtime for which it is designed to operate.
Instead, the third-party tool must provide the OMPD library with a set of
callbacks that the OMPD library uses to access the OpenMP runtime.
This interface is given in
Section~\ref{sec:third-party-tool-callback-interface}.
In general, a third-party's tool's OpenMP-related activity will be
conducted through the OMPD interface.
However, there are a few instances where the third-party tool needs
to access the OpenMP runtime directly;
these cases are discussed in
Section~\ref{subsec:runtime-entry-points-for-ompd}.

\section{Tool Interfaces Definitions}
\index{tool interfaces definitions}
\index{tools header files}
\label{sec:tool_interfaces_definitions}

\begin{ccppspecific}

A compliant implementation must supply a set of definitions for the OMPT runtime entry 
points, OMPT callback signatures, OMPD runtime entry points, OMPD tool callback 
signatures, OMPD tool interface routines, and the special data types of their parameters 
and return values.

The set of definitions is provided in a header file named \code{omp-tools.h} and must 
contain a declaration for each of the types defined in 
Sections~\ref{sec:ompt-data-types} - \ref{sec:entry-points} and
\ref{subsec:ompd-data-types} - \ref{subsec:runtime-entry-points-for-ompd}. 

In addition, the set of definitions may specify other implementation specific values.

The \code{ompt_start_tool} function, the \code{ompd_dll_locations} function, all OMPD 
tool interface functions, and all OMPD runtime entry points are external functions with 
``C'' linkage.
	
			
\end{ccppspecific}



% OMPT
% This is an included file. See the master file for more information.
%
% When editing this file:
%
%    1. To change formatting, appearance, or style, please edit openmp.sty.
%
%    2. Custom commands and macros are defined in openmp.sty.
%
%    3. Be kind to other editors -- keep a consistent style by copying-and-pasting to
%       create new content.
%
%    4. We use semantic markup, e.g. (see openmp.sty for a full list):
%         \code{}     % for bold monospace keywords, code, operators, etc.
%         \plc{}      % for italic placeholder names, grammar, etc.
%
%    5. There are environments that provide special formatting, e.g. language bars.
%       Please use them whereever appropriate.  Examples are:
%
%         \begin{fortranspecific}
%         This is text that appears enclosed in blue language bars for Fortran.
%         \end{fortranspecific}
%
%         \begin{note}
%         This is a note.  The "Note -- " header appears automatically.
%         \end{note}
%
%    6. Other recommendations:
%         Use the convenience macros defined in openmp.sty for the minor headers
%         such as Comments, Syntax, etc.
%
%         To keep items together on the same page, prefer the use of 
%         \begin{samepage}.... Avoid \parbox for text blocks as it interrupts line numbering.
%         When possible, avoid \filbreak, \pagebreak, \newpage, \clearpage unless that's
%         what you mean. Use \needspace{} cautiously for troublesome paragraphs.
%
%         Avoid absolute lengths and measures in this file; use relative units when possible.
%         Vertical space can be relative to \baselineskip or ex units. Horizontal space
%         can be relative to \linewidth or em units.
%
%         Prefer \emph{} to italicize terminology, e.g.:
%             This is a \emph{definition}, not a placeholder.
%             This is a \plc{var-name}.
%


\section{Overview}
\label{sec:ompt-overview}

The OMPT interface defines mechanisms for initializing a tool,
exploring the details of an OpenMP implementation, examining OpenMP state
associated with an OpenMP thread, interpreting an OpenMP thread's call stack,
receiving notification about OpenMP \emph{events}, tracing activity on 
OpenMP target devices, and controlling a tool from an OpenMP application.

\section{Activating a Tool}
\label{sec:ompt-initialization}

There are three steps to activating a tool. First, an OpenMP
implementation determines whether a tool should be initialized.  If
so, the OpenMP implementation invokes the tool's initializer, enabling
the tool to prepare to monitor the execution on the host. Finally, a
tool may arrange to monitor computation that execute 
on target devices. This section explains how the tool and an 
OpenMP implementation interact to accomplish these tasks.

\subsection{Determining Whether a Tool Should be Initialized}
\label{sec:ompt-check-tool}

A tool indicates its interest in using the OMPT interface 
by providing a non-\code{NULL} pointer to an \code{ompt\_fns\_t}
structure to an OpenMP implementation as a return value from 
\code{ompt\_start\_tool}. There are three ways
that a tool can provide a definition of \code{ompt\_start\_tool} to an
OpenMP implementation:

\begin{itemize}
\item statically-linking the tool's definition of \code{ompt\_start\_tool}
  into an OpenMP application, 
\item introducing a dynamically-linked library that includes the tool's definition
  of \code{ompt\_start\_tool} into the application's address space, or 
\item providing the name of a dynamically-linked library appropriate
  for the architecture and operating system used by the application
  in the \plc{tool-libraries-var} ICV.
\end{itemize}

Immediately before an OpenMP implementation initializes itself, it
determines whether it should check for the presence of a tool
interested in using the OMPT interface by examining the \plc{tool-var}
ICV.  If value of \plc{tool-var} is \plc{disabled}, the OpenMP
implementation will initialize itself without even checking whether a
tool is present and the functionality of the OMPT interface will be
unavailable as the program executes.
 
If the value of \plc{tool-var} is \plc{enabled}, the OpenMP
implementation will check to see if a tool has provided an
implmentation of \code{ompt\_start\_tool}.  The OpenMP implementation first
checks if a tool-provided implementation of \code{ompt\_start\_tool} is
available in the address space, either statically-linked into the
application or in a dynamically-linked library loaded in the address
space. If multiple implementations of \code{ompt\_start\_tool} are available,
the OpenMP implementation will use the first tool-provided
implementation of \code{ompt\_start\_tool} found.

If no tool-provided implementation of \code{ompt\_start\_tool} is found in
the address space, the OpenMP implementation will consult the
\plc{tool-libraries-var} ICV, which contains a (possibly empty) list
of dynamically-linked libraries.  As described in detail in
Section~\ref{sec:OMP_TOOL_LIBRARIES}, the libraries in
\plc{tool-libraries-var}, will be searched for the first usable
implementation of \code{ompt\_start\_tool} provided by one of the libraries
in the list.

If a tool-provided definition of \code{ompt\_start\_tool} is found using
either method, the OpenMP implementation will invoke it; if it returns
a non-\code{NULL} pointer to an \code{ompt\_fns\_t} structure, 
the OpenMP implementation will know
that a tool is present that wants to use the OMPT interface.

Next, the OpenMP implementation will initialize itself. If a tool
provided a non-\code{NULL} pointer to an \code{ompt\_fns\_t} structure,
the OpenMP runtime will prepare itself for use of the OMPT interface by a tool. 

\crossreferences
\begin{itemize}
\item \plc{tool-var} ICV, see \specref{sec:Internal Control Variables}.
\item \plc{tool-libraries-var} ICV, see \specref{sec:Internal Control Variables}.
\item \code{ompt\_fns\_t}, see \specref{sec:ompt_fns_t}.
\item \code{ompt\_start\_tool}, see \specref{sec:ompt_start_tool}.
\end{itemize}

\subsection{Tool Initialization}
\index{tool initialization}
\label{sec:tool-initialize}

If a tool-provided implementation of \code{ompt\_start\_tool} returns a
non-\code{NULL} pointer to an \code{ompt\_fns\_t} structure,
the OpenMP implementation will invoke the tool initializer specified
in this structure prior to the occurrence of any OpenMP \emph{event}. 

A tool's initializer, described in \specref{sec:ompt_initialize_t}
uses its \callbackarg{} \plc{lookup} to look up pointers
to OMPT interface runtime entry points provided by the OpenMP
implementation; this process is described in \specref{sec:ompt-bind}.
After obtaining a pointer to the OpenMP runtime entry point known as
known as \code{ompt\_set\_callback} with type signature
\code{ompt\_set\_callback\_t}, the tool initializer should use it to
register tool callbacks for OpenMP events, as described in
\specref{sec:ompt-register-callbacks}.

A tool initializer may use the OMPT interface runtime
entry points known as \code{ompt\_enumerate\_states} and
\code{ompt\_enumerate\_mutex\_impls}, which have type signatures
\code{ompt\_enumerate\_states\_t} and
\code{ompt\_enumerate\_mutex\_impls\_t}, to determine what thread
states and implementations of mutual exclusion a particular OpenMP
implementation employs. The descriptions of the enumeration runtime entry point
type signatures show how to use them to determine what
thread states and mutual exclusion mechanisms an OpenMP implementation supports.

If a tool initializer returns a non-zero value, the tool will be
\emph{activated} for the execution; otherwise, the tool will be
inactive.

\crossreferences
\begin{itemize}
\item \code{ompt\_initialize\_t}, see \specref{sec:ompt_initialize_t}.
\item \code{ompt\_callback\_thread\_begin\_t}, see \specref{sec:ompt_callback_thread_begin_t}.
\item \code{ompt\_enumerate\_states\_t}, see \specref{sec:ompt_enumerate_states_t}.
\item \code{ompt\_enumerate\_mutex\_impls\_t}, see   \specref{sec:ompt_enumerate_mutex_impls_t}.
\item \code{ompt\_set\_callback\_t}, see \specref{sec:ompt_set_callback_t}.
\item \code{ompt\_function\_lookup\_t}, see \specref{sec:ompt_function_lookup_t}.
\end{itemize}


\subsubsection{Binding Entry Points in the OMPT Callback Interface}
\label{sec:ompt-bind}

Functions that an OpenMP implementation provides to support the OMPT interface
are not defined as global function symbols. Instead, they are defined as runtime entry points 
that a tool can only identify using the \plc{lookup} function provided as an
argument to the tool's initializer. This design avoids tool
implementations that
will fail in certain circumstances when functions defined as part of
the OpenMP runtime are not visible to a tool, even though the tool and
the OpenMP runtime are both present in the same address space.
It also prevents inadvertant use of a tool support routine by
applications.
 
A tool's initializer receives a function pointer to a \plc{lookup}
runtime entry point with type signature
\code{ompt\_function\_lookup\_t} as its first argument. Using this
function, a tool initializer may obtain a pointer to each of the
runtime entry points that an OpenMP implementation provides to support
the OMPT interface. Once a tool has obtained a 
\plc{lookup} function, it may employ it at any point in the future.

For each runtime entry point in the OMPT interface for the host device, 
Table~\ref{table:ompt-callback-interface-functions} provides the string
name by which it is known and its associated type signature. Implementations
can provide additional, implementation specific names and corresponding
entry points as long as they don't use names that start with the prefix
``\code{ompt\_}''. These are reserved for future extensions in the 
OpenMP specification.

During initialization, a tool should look up each runtime entry point in the
OMPT interface by name and bind a pointer maintained by the tool
that it can use later to invoke the entry point as needed. The entry points
described in Table~\ref{table:ompt-callback-interface-functions}
enable a tool to assess 
what thread states and mutual exclusion implementations that an OpenMP runtime supports,
register tool callbacks, inspect callbacks registered,
introspect OpenMP state associated with threads, and use tracing to monitor
computations that execute on target devices.

Detailed information about each runtime entry point listed in 
Table~\ref{table:ompt-callback-interface-functions} is included as
part of the description of its type signature.

\crossreferences
\begin{itemize}
\item \code{ompt\_enumerate\_states\_t}, see \specref{sec:ompt_enumerate_states_t}.
\item \code{ompt\_enumerate\_mutex\_impls\_t}, see  \specref{sec:ompt_enumerate_mutex_impls_t}.
\item \code{ompt\_set\_callback\_t}, see \specref{sec:ompt_set_callback_t}.
\item \code{ompt\_get\_callback\_t}, see \specref{sec:ompt_get_callback_t}.
\item \code{ompt\_get\_thread\_data\_t}, see \specref{sec:ompt_get_thread_data_t}.
\item \code{ompt\_get\_num\_places\_t}, see \specref{sec:ompt_get_num_places_t}.
\item \code{ompt\_get\_place\_proc\_ids\_t}, see \specref{sec:ompt_get_place_proc_ids_t}.
\item \code{ompt\_get\_place\_num\_t}, see \specref{sec:ompt_get_place_num_t}.
\item \code{ompt\_get\_partition\_place\_nums\_t}, see \specref{sec:ompt_get_partition_place_nums_t}.
\item \code{ompt\_get\_procid\_t}, see \specref{sec:ompt_get_proc_id_t}.
\item \code{ompt\_get\_state\_t}, see \specref{sec:ompt_get_state_t}.
\item \code{ompt\_get\_parallel\_info\_t}, see \specref{sec:ompt_get_parallel_info_t}.
\item \code{ompt\_get\_task\_info\_t}, see \specref{sec:ompt_get_task_info_t}.
\item \code{ompt\_get\_target\_info\_t}, see \specref{sec:ompt_get_target_info_t}.
\item \code{ompt\_get\_num\_devices\_t}, see \specref{sec:ompt_get_num_devices_t}.
\item \code{ompt\_get\_unique\_id\_t}, see \specref{sec:ompt_get_unique_id_t}.
\item \code{ompt\_function\_lookup\_t}, see \specref{sec:ompt_function_lookup_t}.
\end{itemize}

\begin{table}[p]
    \caption{OMPT callback interface runtime entry point names and their type signatures.\label{table:ompt-callback-interface-functions}}
    \begin{tabular}{ll}\hline
        {\small \textbf{\textsf{Entry Point String Name}}} & {\small \textbf{\textsf{Type signature}}}\\\hline
        ``\code{ompt\_enumerate\_states}'' & \code{ompt\_enumerate\_states\_t}\\
        ``\code{ompt\_enumerate\_mutex\_impls}'' & \code{ompt\_enumerate\_mutex\_impls\_t}\\
        ``\code{ompt\_set\_callback}'' & \code{ompt\_set\_callback\_t}\\
        ``\code{ompt\_get\_callback}'' & \code{ompt\_get\_callback\_t}\\
        ``\code{ompt\_get\_thread\_data}'' & \code{ompt\_get\_thread\_data\_t}\\
        ``\code{ompt\_get\_num\_places}'' & \code{ompt\_get\_num\_places\_t}\\
        ``\code{ompt\_get\_place\_proc\_ids}'' & \code{ompt\_get\_place\_proc\_ids\_t}\\
        ``\code{ompt\_get\_place\_num}'' & \code{ompt\_get\_place\_num\_t}\\
        ``\code{ompt\_get\_partition\_place\_nums}'' & \code{ompt\_get\_partition\_place\_nums\_t}\\
        ``\code{ompt\_get\_proc\_id}'' & \code{ompt\_get\_proc\_id\_t}\\
        ``\code{ompt\_get\_state}'' & \code{ompt\_get\_state\_t}\\
        ``\code{ompt\_get\_parallel\_info}'' & \code{ompt\_get\_parallel\_info\_t}\\
        ``\code{ompt\_get\_task\_info}'' & \code{ompt\_get\_task\_info\_t}\\
        ``\code{ompt\_get\_num\_devices}'' & \code{ompt\_get\_num\_devices\_t}\\
        ``\code{ompt\_get\_target\_info}'' & \code{ompt\_get\_target\_info\_t}\\
        ``\code{ompt\_get\_unique\_id}'' & \code{ompt\_get\_unique\_id\_t}\\\hline
        % ``\code{ompt\_callback\_device\_initialize}'' & \code{ompt\_callback\_device\_initialize\_t}\\\hline
    \end{tabular}
    \vskip 1ex
\end{table}

\subsection{Monitoring Activity on the Host}
\index{event callback registration}
\label{sec:ompt-register-callbacks}

To monitor execution of an OpenMP program on the host device, a tool's
initializer must register to receive notification
of events that occur as an OpenMP program executes.  
A tool can register callbacks for OpenMP events using
the runtime entry point known as 
\code{ompt\_set\_callback}.  The possible return codes for
\code{ompt\_set\_callback} and their meanings are shown in
Table~\ref{table:ToolsSupport_set_rc}.  
If the \code{ompt\_set\_callback} runtime entry point is
called outside a tool's initializer, registration of supported
callbacks may fail with a return code of \code{ompt\_set\_error}. 

All callbacks registered with \code{ompt\_set\_callback} or returned
by \code{ompt\_get\_callback} use the dummy type signature
\code{ompt\_callback\_t}.  While this is a compromise, it is better
than providing unique runtime entry points with a precise type signatures to
set and get the callback for each unique runtime entry point type signature.

Table~\ref{table:valid_rc} indicates the return codes permissible
when trying to register various callbacks. For callbacks where the only registration return code
allowed is \code{ompt\_set\_always}, an
OpenMP implementation must guarantee that the callback will be
invoked every time a runtime event associated with it occurs. Support
for such callbacks is required in a minimal implementation of the
OMPT interface. For other callbacks where registration is allowed to return values
other than \code{ompt\_set\_always}, its implementation-defined
whether an OpenMP implementation invokes a registered callback
never, sometimes, or always. If registration for a callback allows 
a return code of \code{omp\_set\_never}, support for invoking such 
a callback need not be present in a minimal implementation of the 
OMPT interface.  The return code when a callback is
registered enables a tool to know what to expect when the level
of support for the callback can be implementation defined.



\begin{table}
\renewcommand{\arraystretch}{1.2}
\caption{Valid return codes of \code{ompt\_set\_callback} for each callback.\label{table:valid_rc}}
\begin{tabular}{lp{3em}p{3em}p{3em}p{3em}}
                                & \rot{\code{ompt\_set\_never}}
                                & \rot{\vbox{\code{ompt\_set\_sometimes}
                                             \code{ompt\_set\_sometimes\_paired}}}
                                & \rot{\code{ompt\_set\_always}}\\
                                \midrule
\code{ompt\_callback\_thread\_begin}          &   &   & * \\
\code{ompt\_callback\_thread\_end}            &   &   & * \\
\code{ompt\_callback\_parallel\_begin}        &   &   & * \\
\code{ompt\_callback\_parallel\_end}          &   &   & * \\
\code{ompt\_callback\_task\_create}           &   &   & * \\
\code{ompt\_callback\_task\_schedule}         &   &   & * \\
\code{ompt\_callback\_implicit\_task}         &   &   & * \\
\code{ompt\_callback\_target}                 &   &   & * \\
\code{ompt\_callback\_target\_data\_op}       &   &   & * \\
\code{ompt\_callback\_target\_submit}         &   &   & * \\
\code{ompt\_callback\_control\_tool}          &   &   & * \\
\code{ompt\_callback\_device\_initialize}     &   &   & * \\
\code{ompt\_callback\_device\_finalize}       &   &   & * \\
\code{ompt\_callback\_sync\_region\_wait}     & * & * & * \\
\code{ompt\_callback\_mutex\_released}        & * & * & * \\
\code{ompt\_callback\_task\_dependences}      & * & * & * \\
\code{ompt\_callback\_task\_dependence}       & * & * & * \\
\code{ompt\_callback\_work}                   & * & * & * \\
\code{ompt\_callback\_master}                 & * & * & * \\
\code{ompt\_callback\_target\_map}            & * & * & * \\
\code{ompt\_callback\_sync\_region}           & * & * & * \\
\code{ompt\_callback\_lock\_init}             & * & * & * \\
\code{ompt\_callback\_lock\_destroy}          & * & * & * \\
\code{ompt\_callback\_mutex\_acquire}         & * & * & * \\
\code{ompt\_callback\_mutex\_acquired}        & * & * & * \\
\code{ompt\_callback\_nest\_lock}             & * & * & * \\
\code{ompt\_callback\_flush}                  & * & * & * \\
\code{ompt\_callback\_cancel}                 & * & * & * \\
\code{ompt\_callback\_idle}                   & * & * & * \\
\bottomrule
\end{tabular}
\vskip 1ex
\end{table}

To avoid a tool interface specification that enables a tool to
register unique callbacks for an overwhelming number of events,
the interface was collapsed in several ways.
First, in cases where events are naturally paired, e.g., the beginning and
end of a region, and the arguments needed by the callback at each
endpoint were identical, the pair of events was collapsed so that
a tool registers a single callback that will be invoked at both endpoints
with \code{ompt\_scope\_begin} or \code{ompt\_scope\_end} provided
as an argument to identify which endpoint the callback invocation reflects.
Second, when a whole class of events is amenable to uniform treatment, only a
single callback is provided for a family of events, e.g.,  a
\code{ompt\_callback\_sync\_region\_wait} callback is used for multiple
kinds of synchronization regions, i.e., barrier, taskwait, and taskgroup
regions. Some events involve both kinds of collapsing: the aforementioned
\code{ompt\_callback\_sync\_region\_wait} represents
a callback that will be invoked at each endpoint for different kinds
of synchronization regions.


\crossreferences
\begin{itemize}
\item \code{ompt\_set\_callback\_t}, see \specref{sec:ompt_set_callback_t}.
\item \code{ompt\_get\_callback\_t}, see \specref{sec:ompt_get_callback_t}.
\end{itemize}




\subsection{Tracing Activity on Target Devices}
\index{tracing device activity}
\label{sec:tracing-device-activity}

A target device may or may not initialize a full OpenMP runtime system.
Unless it does, it may not be possible to monitor activity 
on a device using a tool interface based on callbacks.
To accommodate such cases, the OMPT interface defines 
a performance monitoring interface for tracing activity on target
devices. Tracing activity on a target device involves the following
steps:

\begin{itemize}
\item To prepare to trace activity on a target device, when a tool
  initializer executes, it must register an 
  \code{ompt\_callback\_device\_initialize} callback. A tool may also optionally
  register an \code{ompt\_callback\_device\_finalize} callback.
\item When an OpenMP implementation initializes a target device, the
  OpenMP implementation will dispatch the tool's device initialization
  callback on the host device. If the OpenMP implementation or target device does not support tracing, 
  the OpenMP implementation will pass a \code{NULL} to the tool's device initializer for its
  \plc{lookup} argument; otherwise, the OpenMP implementation will pass 
  a pointer to a device-specific runtime entry point with type 
  signature \code{ompt\_function\_lookup\_t} to the tool's device initializer. 
\item If the device initializer for the tool receives a
  non-\code{NULL} \plc{lookup} pointer, the tool may use it to query
  which runtime entry points in the tracing interface are available for a target device
  and bind the function pointers returned to tool variables.
  Table~\ref{table:ompt-tracing-interface-functions} indicates the
  names of the runtime entry points that a target device may provide for use
  by a tool.  
  Implementations
can provide additional, implementation specific names and corresponding
entry points as long as they don't use names that start with the prefix
``\code{ompt\_}''. Theses are reserved for future extensions in the 
OpenMP specification.

  If \plc{lookup} is non-\code{NULL}, the driver for a device will
  provide runtime entry points that enable a tool to control the device's
  interface for collecting traces in its \emph{native} trace format,
  which may be device specific.  
  The kinds of trace records available for a device will typically be
  implementation-defined.
  Some devices may also allow a tool to
  collect traces of records in a standard format known as OMPT format,
  described in this document. If so, the \plc{lookup} function will
  return values for the runtime entry points 
  \code{ompt\_set\_trace\_ompt} and \code{ompt\_get\_record\_ompt}, which support
  collecting and decoding OMPT traces. 
  These runtime entry points are not required for all devices and will only be available for target devices that support 
  collection of standard traces in OMPT format.
  For some devices, their native
  tracing format may be OMPT format. In that case, tracing can be
  controlled using either the runtime entry points for native or OMPT
  tracing.

\begin{table}
{\small
\caption{OMPT tracing interface runtime entry point names and their type signatures.\label{table:ompt-tracing-interface-functions}}
\begin{tabular}{ll}\hline
\textbf{\textsf{Entry Point String Name}} & \textbf{\textsf{Type Signature}}\\\hline
``\code{ompt\_get\_device\_time}'' & \code{ompt\_get\_device\_time\_t}\\
``\code{ompt\_translate\_time}'' & \code{ompt\_translate\_time\_t}\\
``\code{ompt\_set\_trace\_ompt}'' & \code{ompt\_set\_trace\_ompt\_t}\\
``\code{ompt\_set\_trace\_native}'' & \code{ompt\_set\_trace\_native\_t}\\
``\code{ompt\_start\_trace}'' & \code{ompt\_start\_trace\_t}\\
``\code{ompt\_pause\_trace}'' & \code{ompt\_pause\_trace\_t}\\
``\code{ompt\_stop\_trace}'' & \code{ompt\_stop\_trace\_t}\\
``\code{ompt\_advance\_buffer\_cursor}'' & \code{ompt\_advance\_buffer\_cursor\_t}\\
``\code{ompt\_get\_record\_type}'' & \code{ompt\_get\_record\_type\_t}\\
``\code{ompt\_get\_record\_ompt}'' & \code{ompt\_get\_record\_ompt\_t}\\
``\code{ompt\_get\_record\_native}'' & \code{ompt\_get\_record\_native\_t}\\
``\code{ompt\_get\_record\_abstract}'' & \code{ompt\_get\_record\_abstract\_t}\\\hline
\end{tabular}
}
\vskip 1ex
\end{table}


\item The tool will use the \code{ompt\_set\_trace\_native}
  and/or the \code{ompt\_set\_trace\_ompt} runtime entry point to specify what
  types of events or activities to monitor on the target device.
\item The tool will initiate tracing on the target device by 
  invoking \code{ompt\_start\_trace}. Arguments to \code{ompt\_start\_trace}
  include two tool callbacks for use by the OpenMP implementation to manage
  traces associated with the target device: one to allocate
  a buffer where the target device can deposit trace events and a
  second to process a buffer of trace events from the target device. 
\item When the target device needs a trace buffer, the OpenMP implementation 
  will invoke the tool-supplied callback function on the host device to request a new buffer.
\item The OpenMP implementation will monitor execution of OpenMP constructs on the target device as
  directed and record a trace of events or activities into a trace
  buffer. If the device is capable, device trace records will be
  marked with a \plc{host\_op\_id}---an identifier used to associate
  device activities with the target operation initiated on the host
  that caused these activities.  To correlate activities on the host
  with activities on a device, a tool can register a
  \code{ompt\_callback\_target\_submit} callback. 
  Before the host initiates each distinct activity associated with a structured block for a \code{target} construct
  on a target device, the OpenMP implementation will dispatch the \code{ompt\_callback\_target\_submit} callback
  on the host in the thread executing the task that encounters the \code{target} construct. 
  Examples of activities that could cause an \code{ompt\_callback\_target\_submit} callback to be dispatched
  include an explicit data copy between a host and target device or execution of a computation.
  The callback provides the tool with a pair of identifiers: one that identifies the target region and a second
  that uniquely identifies an activity associated with that region.  
  These identifiers help the tool correlate activities on the target device with their target region.
\item When appropriate, e.g., when a trace buffer fills or needs to be
  flushed, the OpenMP implementation will invoke the tool-supplied buffer
  completion callback to process a non-empty sequence of
  records in a trace buffer associated with the target device.

\item The tool-supplied buffer completion callback may return
  immediately, ignoring records in the trace buffer, or it may iterate
  through them using the \code{ompt\_advance\_buffer\_cursor} entry
  point
  and inspect each one. A tool may inspect the type of the record at
  the current cursor position using the \code{ompt\_get\_record\_type}
  runtime entry point.  A tool may choose to inspect the contents of some or
  all records in a trace buffer using the \code{ompt\_get\_record\_ompt},
  \code{ompt\_get\_record\_native}, or
  \code{ompt\_get\_record\_abstract} runtime entry point.  Presumably, a tool that
  chooses to use the \code{ompt\_get\_record\_native} runtime entry point to
  inspect records will have some knowledge about a device's native
  trace format.  A tool may always use the
  \code{ompt\_get\_record\_abstract} runtime entry point to inspect a trace
  record; this runtime entry point will decode the contents of a native trace record
  and summarize them in a standard format, namely, a
  \code{ompt\_record\_abstract\_t} record.
  Only a record in OMPT format can be retrieved using the
  \code{ompt\_get\_record\_ompt} runtime entry point.
\item Once tracing has been started on a device, a tool may pause or resume
  tracing on the device at any time by invoking
  \code{ompt\_pause\_trace} with an appropriate flag value as an
  argument.  
\item A tool may start or stop tracing on a device at any time using the 
  \code{ompt\_start\_trace} or \code{ompt\_stop\_trace} runtime entry points,
  respectively. When tracing is stopped on a device, the OpenMP implementatin will eventually 
  gather all trace records already collected on the device and present to the tool using
  the buffer completion callback provided by the tool.
\item It is legal to shut down an OpenMP implementation while device tracing 
is in progress.  
\item When an OpenMP implementation begins to shut down, the OpenMP implementation will 
  finalize each target device.  Device finalization occurs in three steps.
  First, the OpenMP implementation halts any tracing in progress for the device. Second,
  the OpenMP implementation flushes all trace records collected for the device and presents them to
  the tool using the buffer completion callback associated with that device. 
  Finally, the OpenMP implementation dispatches 
  any \code{ompt\_callback\_device\_finalize} callback that was previously 
  registered by the tool.

\end{itemize}


\crossreferences
\begin{itemize}
\item \code{ompt\_callback\_device\_initialize\_t}, see \specref{sec:ompt_callback_device_initialize_t}.
\item \code{ompt\_callback\_device\_finalize\_t}, see \specref{sec:ompt_callback_device_finalize_t}.
\item \code{ompt\_get\_device\_time}, see \specref{sec:ompt_get_device_time_t}.
\item \code{ompt\_translate\_time}, see \specref{sec:ompt_translate_time_t}.
\item\code{ompt\_set\_trace\_ompt}, see \specref{sec:ompt_set_trace_ompt_t}.
\item \code{ompt\_set\_trace\_native}, see \specref{sec:ompt_set_trace_native_t}.
\item \code{ompt\_start\_trace}, see \specref{sec:ompt_start_trace_t}.
\item \code{ompt\_pause\_trace}, see \specref{sec:ompt_pause_trace_t}.
\item \code{ompt\_stop\_trace}, see \specref{sec:ompt_stop_trace_t}.
\item \code{ompt\_advance\_buffer\_cursor}, see \specref{sec:ompt_advance_buffer_cursor_t}.
\item \code{ompt\_get\_record\_type}, see \specref{sec:ompt_get_record_type_t}.
\item \code{ompt\_get\_record\_ompt}, see \specref{sec:ompt_get_record_ompt_t}.
\item \code{ompt\_get\_record\_native}, see \specref{sec:ompt_get_record_native_t}.
\item \code{ompt\_get\_record\_abstract}, see \specref{sec:ompt_get_record_abstract_t}.
\end{itemize}


\section{Finalizing a Tool}
\label{sec:ompt-finalization}

If \code{ompt\_start\_tool} returned a non-\code{NULL} pointer when an OpenMP
implementation was initialized, the tool finalizer, of type signature 
\code{ompt\_finalize\_t}, specified by the
\plc{finalize} field in this structure will be called as the OpenMP
implementation shuts down.

\crossreferences
\begin{itemize}
\item \code{ompt\_finalize\_t}, \specref{sec:ompt_finalize_t}
\end{itemize}

\section{Data Types}
\label{sec:ompt-data-types}

\subsection{Tool Initialization and Finalization}
\label{sec:ompt_fns_t}

\summary
A tool's implementation of \code{ompt\_start\_tool} returns a pointer to an
\code{ompt\_fns\_t} structure that contains pointers to the tool's 
initializer and finalizer functions.

\vbox{
\begin{ccppspecific}
\begin{boxedcode}
typedef struct ompt_fns_t \{
  ompt_initialize_t \plc{initialize};
  ompt_finalize_t \plc{finalize};
\} ompt_fns_t;
\end{boxedcode}
\end{ccppspecific}
}

\restrictions

Both the \plc{initialize} and \plc{finalize} function pointers in an
\code{ompt\_fns\_t} structure returned by \code{ompt\_start\_tool} must be
non-\code{NULL}.

\crossreferences
\begin{itemize}
\item \code{ompt\_start\_tool}, see \specref{sec:ompt_start_tool}.
\end{itemize}


\subsection{Thread States}
\label{sec:thread-states}

To enable a tool to understand the behavior of an executing program, 
an OpenMP implementation maintains a state for each thread. 
The state maintained for a thread is an
approximation of the thread's instantaneous state. 

\vbox{
\begin{ccppspecific}
A thread's state will be one of the values of the  
enumeration type \code{omp\_state\_t} or
an implementation-defined state value of 512 or higher. 
Thread states in the enumeration fall into several classes: 
work, barrier wait, task wait, mutex wait, target wait, 
and miscellaneous. 

\begin{boxedcode}
typedef enum omp_state_e \{
  omp_state_work_serial                      = 0x000,
  omp_state_work_parallel                    = 0x001,
  omp_state_work_reduction                   = 0x002,

  omp_state_wait_barrier                     = 0x010,
  omp_state_wait_barrier_implicit_parallel   = 0x011,
  omp_state_wait_barrier_implicit_workshare  = 0x012,
  omp_state_wait_barrier_implicit            = 0x013,
  omp_state_wait_barrier_explicit            = 0x014,

  omp_state_wait_taskwait                    = 0x020,
  omp_state_wait_taskgroup                   = 0x021,

  omp_state_wait_mutex                       = 0x040,
  omp_state_wait_lock                        = 0x041,
  omp_state_wait_critical                    = 0x042,
  omp_state_wait_atomic                      = 0x043,
  omp_state_wait_ordered                     = 0x044,

  omp_state_wait_target                      = 0x080,
  omp_state_wait_target_map                  = 0x081,
  omp_state_wait_target_update               = 0x082,

  omp_state_idle                             = 0x100, 
  omp_state_overhead                         = 0x101, 
  omp_state_undefined                        = 0x102
\} omp_state_t;
\end{boxedcode}
\end{ccppspecific}
}


A tool can query the OpenMP state of a thread at any time. 
If a tool queries the state of a thread that is not associated 
with OpenMP, the implementation reports the state as \code{omp\_state\_undefined}.



Some values of the enumeration type \code{omp\_state\_t} are used by all 
OpenMP implementations, 
e.g., \code{omp\_state\_work\_serial}, 
which indicates that a thread is executing in a serial region, and  
\code{omp\_state\_work\_parallel}, 
which indicates that a thread is executing in a parallel region.
Other values of the enumeration type describe a thread's state at 
different levels of specificity. 
For instance, an OpenMP implementation may use 
the state \code{omp\_state\_wait\_barrier}  to represent all 
waiting at barriers. It may differentiate between waiting at implicit or explicit barriers using
\code{omp\_state\_wait\_barrier\_implicit} and \code{omp\_state\_wait\_barrier\_explicit}. 
To provide full detail about the type of an implicit barrier, a runtime may report 
\code{omp\_state\_wait\_barrier\_implicit\_parallel} or 
\code{omp\_state\_wait\_barrier\_implicit\_workshare} as appropriate.

For states that represent waiting, an OpenMP implementation has the 
choice of transitioning a thread to such states early or late.
For instance, when an OpenMP thread is trying to acquire a lock,
there are several points at which an OpenMP implementation
transition the thread to the \code{omp\_state\_wait\_lock} state.
One implementation may transition the thread to the state 
early before the thread attempts to acquire a
lock. Another implementation may transition the thread to the state 
late, only if the thread begins to spin or
block to wait for an unavailable lock. A third implementation
may transition the thread to the state even later, e.g., only
after the thread waits for a significant amount of time. 

The following sections describe the classes of states and the states in each class.
\subsubsection{Work States}
An OpenMP implementation reports a thread in a work state 
when the thread is performing serial work, parallel work, or a reduction.

\begin{description}

\item \code{omp\_state\_work\_serial} 

  The thread is executing code outside all parallel regions. 

\item \code{omp\_state\_work\_parallel} 

  The thread is executing code within the scope of a parallel region construct.

\sloppy
\item \code{omp\_state\_work\_reduction} 
 
  The thread is combining partial reduction results from threads in its team. 
  An OpenMP implementation  
  might never report a thread in this state; a thread
  combining partial reduction results may have its state reported as
  \code{omp\_state\_work\_parallel} or \code{omp\_state\_overhead}.

\end{description}


\subsubsection{Barrier Wait States}

An OpenMP implementation reports that a thread is in a barrier wait state 
when the thread is awaiting completion of a barrier.


\begin{description}

  \item \code{omp\_state\_wait\_barrier} 
  
  \sloppy
  The thread is waiting at either an implicit or explicit barrier.
  A thread may enter this state
  early, when the thread encounters a barrier, or late, when the
  thread begins to wait at the barrier. An implementation may never report a thread in this state; instead, a thread may have its state reported
  as \code{omp\_state\_wait\_barrier\_implicit}  or \code{omp\_state\_wait\_barrier\_explicit}, as appropriate.
  
  \item \code{omp\_state\_wait\_barrier\_implicit} 
  
  \sloppy
  The thread is waiting at an implicit barrier in a parallel region. 
  A  thread may enter this state
  early, when the thread encounters a barrier, or late, when the
  thread begins to wait at the barrier.
  An OpenMP implementation may report \code{omp\_state\_wait\_barrier} 
  for implicit barriers.
  
  \item \code{omp\_state\_wait\_barrier\_explicit\_parallel} 

  The description of when a thread reports a state associated with an implicit barrier
  is described for state \code{omp\_state\_wait\_barrier\_implicit}.  
  An OpenMP implementation may report \code{omp\_state\_wait\_barrier\_explicit\_parallel} 
  for an implicit barrier that occurs at the end of a parallel region. 
  As explained in \specref{sec:ompt_callback_sync_region_t},
  reporting the state \code{omp\_state\_wait\_barrier\_implicit\_parallel} 
  permits a weaker contract between a runtime and a tool that 
  enables a simpler and faster implementation of parallel regions.

  \item \code{omp\_state\_wait\_barrier\_explicit\_workshare} 

  The description of when a thread reports a state associated with an implicit barrier
  is described for state \code{omp\_state\_wait\_barrier\_implicit}.  
  An OpenMP implementation may report \code{omp\_state\_wait\_barrier\_explicit\_parallel} 
  for an implicit barrier that occurs at the end of a worksharing construct.

  \item \code{omp\_state\_wait\_barrier\_explicit} 

  The thread is waiting at an explicit barrier  in a parallel region. 
  A thread may enter this state
  early, when the thread encounters a barrier, or late, when the
  thread begins to wait at the barrier.
  An implementation may report \code{omp\_state\_wait\_barrier} 
  for explicit barriers.

  
\end{description}
  
\subsubsection{Task Wait States}

\begin{description}

\item \code{omp\_state\_wait\_taskwait} 

  The thread is waiting at a taskwait construct. A 
  thread may enter this state early, when the
  thread encounters a taskwait construct, or late, when the thread
  begins to wait for an uncompleted task.

\item \code{omp\_state\_wait\_taskgroup} 

  The thread is waiting at the end of a taskgroup construct. A 
  thread may enter this state early, when the
  thread encounters the end of a taskgroup construct, or late, when the thread
  begins to wait for an uncompleted task.

\end{description}


\subsubsection{Mutex Wait States}

OpenMP provides several mechanisms that enforce mutual exclusion:
locks as well as critical, atomic, and ordered sections.  This
grouping contains all states used to indicate that a thread is
awaiting exclusive access to a lock, critical section, variable,
or ordered section.

An OpenMP implementation may report a thread waiting for any type
of mutual exclusion using either a state that precisely identifies
the type of mutual exclusion, or  a more generic state such as
\code{omp\_state\_wait\_mutex} or \code{omp\_state\_wait\_lock}.  This
flexibility may significantly simplify the maintenance of states
associated with mutual exclusion in the runtime when various
mechanisms for mutual exclusion rely on a common implementation,
e.g., locks.

% Section~\ref{sec:wait-identifier} describes how each thread maintains a wait identifier to identify what a thread is awaiting. Before a thread enters any state indicating that it is awaiting mutual exclusion, the OpenMP runtime will update the thread's wait identifier to indicate what the thread is awaiting. 

\begin{description}

\item \code{omp\_state\_wait\_mutex}

  The thread is waiting for a mutex of an unspecified type. A 
  thread may enter this state early, when a thread encounters a lock acquisition or a region that requires mutual exclusion, or late, when the thread begins to wait.

\item \code{omp\_state\_wait\_lock}

  The thread is waiting for a  lock  or nest lock. A 
  thread may enter this state early, when a thread
  encounters a lock \code{set} routine, or late, when the thread
  begins to wait for a lock.

\item \code{omp\_state\_wait\_critical} 

  The thread is waiting to enter a critical region. A 
  thread may enter this state early, when the
  thread encounters a critical construct, or late, when the thread
  begins to wait to enter the critical region. 


\item \code{omp\_state\_wait\_atomic} 

  The thread is waiting to enter an atomic region. A 
  thread may enter this state early, when the thread
  encounters an atomic construct, or late, when the thread begins
  to wait to enter the atomic region. 
  An implementation may opt not to report
  this state when using atomic hardware instructions that support non-blocking atomic implementations.
  

\item \code{omp\_state\_wait\_ordered} 

  The thread is waiting to enter an ordered region. A 
  thread may enter this state early, when the thread encounters
  an ordered construct, or late, when the thread begins
  to wait to enter the ordered region. 
  
\end{description}
  
\subsubsection{Target Wait States}

\begin{description}

\item \code{omp\_state\_wait\_target} 

  The thread is waiting for a target region to complete.
  
\item \code{omp\_state\_wait\_target\_map} 

  The thread is waiting for a target data mapping operation to complete. 
  An implementation may report \code{omp\_state\_wait\_target} 
  for target data constructs.

\item \code{omp\_state\_wait\_target\_update} 

  The thread is waiting for a target  update operation to complete. 
  An implementation may report \code{omp\_state\_wait\_target} 
  for target update constructs.

\end{description}


\subsubsection{Miscellaneous States}

\begin{description}
\item \code{omp\_state\_idle} 

  The thread is idle, waiting for work.

\item \code{omp\_state\_overhead} 

  A thread may be reported as being in the overhead state at any point while 
  executing within an OpenMP runtime, except while waiting indefinitely
  at a synchronization point.
%  e.g., while preparing to execute a parallel, task, or worksharing construct. 
  An OpenMP implementation report a thread's state as a work state for
  some or all of the time the thread spends in executing in the OpenMP runtime.

\item \code{omp\_state\_undefined} 

  This state is reserved for threads that are not user threads,
  initial threads, threads currently in an OpenMP team, or threads
  waiting to become part of an OpenMP team.

\end{description}

\subsection{Callbacks}
\label{sec:ompt_callbacks_t}

The following enumeration type indicates the integer codes used to identify 
OpenMP callbacks when registering or querying them.

\vbox{
\begin{ccppspecific}
\begin{boxedcode}
typedef enum ompt_callbacks_e \{
  ompt_callback_thread_begin             = 1,
  ompt_callback_thread_end               = 2,
  ompt_callback_parallel_begin           = 3,
  ompt_callback_parallel_end             = 4,
  ompt_callback_task_create              = 5,
  ompt_callback_task_schedule            = 6,
  ompt_callback_implicit_task            = 7,
  ompt_callback_target                   = 8,
  ompt_callback_target_data_op           = 9,
  ompt_callback_target_submit            = 10,
  ompt_callback_control_tool             = 11,
  ompt_callback_device_initialize        = 12,
  ompt_callback_device_finalize          = 13,
  ompt_callback_sync_region_wait         = 14,
  ompt_callback_mutex_released           = 15,
  ompt_callback_task_dependences         = 16,
  ompt_callback_task_dependence          = 17,
  ompt_callback_work                     = 18,
  ompt_callback_master                   = 19,
  ompt_callback_target_map               = 20,
  ompt_callback_sync_region              = 21,
  ompt_callback_lock_init                = 22,
  ompt_callback_lock_destroy             = 23,
  ompt_callback_mutex_acquire            = 24,
  ompt_callback_mutex_acquired           = 25,
  ompt_callback_nest_lock                = 26,
  ompt_callback_flush                    = 27,
  ompt_callback_cancel                   = 28,
  ompt_callback_idle                     = 29

\} ompt_callbacks_t;
\end{boxedcode}
\end{ccppspecific}
}


%\subsubsection{Triggers for Miscellaneous Events}
%Most events trigger during the execution of OpenMP directives. Other
%events trigger when an application calls certain runtime library
%routines, e.g., those for setting and unsetting locks.
%This section describes events triggered during initialization and
%finialization of an OpenMP implementation. 
%
%\ompteventswithoutdirectives{\code{ompt\_callback\_thread\_begin}}
%\label{sec:ompt_callback_thread_begin}
%
%An OpenMP implementation invokes this callback in the context of an
%initial thread just after it initializes the runtime, or in the
%context of a new thread created by the runtime just after the thread
%initializes itself. In either case, this callback must be the first
%callback for a thread and must occur before the thread executes any
%OpenMP tasks. This callback has type signature
%\code{ompt\_callback\_thread\_begin\_t}.  The callback argument
%\code{thread\_type} indicates the type of the thread: initial, worker,
%or other.
%
%\ompteventswithoutdirectives{\code{ompt\_callback\_thread\_end}}
%\label{sec:ompt_callback_thread_end}
%
%An OpenMP implementation invokes this callback after an OpenMP thread
%completes all of its tasks but before the thread is destroyed. The
%callback executes in the context of the OpenMP thread. This callback
%must be the last callback event for any worker thread; it is optional
%for other types of threads.  This callback has type signature
%\code{ompt\_callback\_thread\_end\_t}.
%
%\ompteventswithoutdirectives{\code{ompt\_callback\_idle}}
%\label{sec:ompt_callback_idle}
%
%An OpenMP implementation invokes this callback with
%\code{endpoint=}\code{ompt\_scope\_begin} when a thread waits for work
%outside a parallel region.  The OpenMP runtime invokes this callback
%with \code{endpoint=}\code{ompt\_scope\_end} before the thread begins
%to execute an implicit task for a parallel region or terminates. The
%callback executes in the environment of the waiting thread.  This
%callback has type signature \code{ompt\_callback\_idle\_t}.
%
%\ompteventswithoutdirectives{\code{ompt\_callback\_sync\_region\_wait}}
%\label{sec:ompt_callback_sync_region_wait}
%
%If the \code{ompt\_callback\_sync\_region\_wait} callback is registered,
%an OpenMP implementation will invoke this callback when a task starts
%and stops waiting in a barrier region, taskwait region, or taskgroup
%region.  This callback has type signature
%\code{ompt\_callback\_sync\_region\_t}.  One region may generate
%multiple pairs of start/stop callbacks if another task is scheduled on
%the thread while the task awaiting completion of the region is
%stalled.  This callback executes in the context of the task that
%encountered the barrier, taskwait, or taskgroup construct.
%
%\ompteventswithoutdirectives{\code{ompt\_callback\_runtime\_shutdown}}
%\label{sec:ompt_callback_runtime_shutdown}
%
%An OpenMP implementation invokes this callback before it shuts down
%the runtime system.  This callback enables a tool to clean up its
%state and record or report information gathered. A runtime may later
%restart and reinitialize the tool by calling the tool initializer
%function (described in Section~\ref{sec:tool-initialize}) again.  This
%callback has type signature \code{ompt\_callback\_t}.


\subsection{Frames}
\index{frames}
\label{sec:ompt_frame_t}

\vbox{
\begin{ccppspecific}
\begin{boxedcode}
typedef struct ompt_frame_s \{\\
  void *\plc{exit_frame};
  void *\plc{enter_frame};
\} ompt_frame_t;
\end{boxedcode}
\end{ccppspecific}
}

\descr

When executing an OpenMP program, at times, one or more procedure frames associated with
the OpenMP runtime may appear on a thread's stack between frames
associated with tasks. To help a tool determine whether a procedure
frame on the call stack belongs to a task or not,
for each task whose frames appear on the stack, the runtime
maintains an \code{ompt\_frame\_t} object 
that indicates a contiguous sequence of 
procedure frames associated with the task.
Each \code{ompt\_frame\_t} object is associated with the task to which the procedure frames belong.
Each non-merged initial, implicit, explicit, or target task with one or more frames on a thread's stack
will have an associated \code{ompt\_frame\_t} object.



An \code{ompt\_frame\_t} object associated with a task contains a pair
of pointers: \plc{exit\_frame} and \plc{enter\_frame}. The field names were
chosen, respectively, to reflect that they typically contain a pointer to a procedure frame on the stack when 
\emph{exiting} the OpenMP runtime into code for a task or \emph{entering} the OpenMP runtime from a task.

The \plc{exit\_frame} field of a task's \code{ompt\_frame\_t} object 
contains the canonical frame address for the procedure frame that
transfers control to the structured block for the task. 
The value of \plc{exit\_frame} is \code{NULL} until just prior to
beginning execution of the structured block for the task.
A task's \plc{exit\_frame} may point to a procedure frame that belongs
to the OpenMP runtime or one that belongs to another task.
The \plc{exit\_frame} for the \code{ompt\_frame\_t} object associated 
with an \emph{initial task} is \code{NULL}.

The \plc{enter\_frame} field of a task's \code{ompt\_frame\_t} object 
contains the canonical frame address of a task procedure frame that invoked the
OpenMP runtime causing the current task to suspend and another task to
execute.
If a task with frames on the stack has not suspended, the value of
\plc{enter\_frame} for the \code{ompt\_frame\_t} object 
associated with the task may contain \code{NULL}. 
The value of \plc{enter\_frame} in a task's \code{ompt\_frame\_t} is
reset to \code{NULL} just before a suspended task resumes execution.

An \code{ompt\_frame\_t}'s lifetime begins when a task is created
and ends when the task is destroyed. Tools should not assume that
a frame structure remains at a constant location in memory throughout
a task's lifetime. A pointer to a task's \code{ompt\_frame\_t} object is passed to
some callbacks; a pointer to a task's \code{ompt\_frame\_t} object 
can also be retrieved by a tool at any time, including in a signal
handler, by invoking the
\code{ompt\_get\_task\_info} runtime entry point (described in
Section~\ref{sec:ompt_get_task_info}).





\begin{table}
\begin{center}
\caption{Meaning of various states of an \code{ompt\_frame\_t}
    object.\label{tab:frame}}
\begin{tabular}{|p{1in}||p{2in}|p{2in}|}
\hline
\plc{exit\_frame} / \plc{enter\_frame} 	state & \plc{enter\_frame} is
\code{NULL}
& \plc{enter\_frame} is non-\code{NULL} \\
\hline
\hline
\plc{exit\_frame} is \code{NULL} & 
case 1)  initial task during execution\newline 
case 2) task that is created but not yet scheduled or already finished & 
initial task suspended while another task executes
\\\hline
\plc{exit\_frame} is non-\code{NULL} 	& non-initial task that has
been scheduled &
non-initial task 
suspended while another task executes
\\\hline
\end{tabular}
\vspace{1ex}
\end{center}
\end{table}

Table~\ref{tab:frame} describes various states in which 
an \code{ompt\_frame\_t} object may be observed and their meaning.
In the presence of nested parallelism, a tool may
observe a sequence of \code{ompt\_frame\_t} objects for a thread.
Appendix~\ref{chap:frames} illustrates 
use of \code{ompt\_frame\_t} objects with nested parallelism.

\needspace{6\baselineskip}\begin{note}
A monitoring tool using asynchronous sampling can observe values
of \plc{exit\_frame} and \plc{enter\_frame} at inconvenient times.
Tools must be prepared to observe and handle \code{ompt\_frame\_t}
objects observed just prior to when their field values should be set or reset.
\end{note}

\subsection{Tracing Support}
\label{sec:ompt-records}

\subsubsection{Record Kind}
\vbox{
\begin{ccppspecific}
\begin{boxedcode}
typedef enum ompt_record_kind_e \{
  ompt_record_ompt               = 1,
  ompt_record_native             = 2,
  ompt_record_invalid            = 3
\} ompt_record_kind_t; 
\end{boxedcode}
\end{ccppspecific}
}

\subsubsection{Native Record Kind}
\label{sec:ompt_record_native_kind_t}

\vbox{
\begin{ccppspecific}
\begin{boxedcode}
typedef enum ompt_record_native_kind_e \{
  ompt_record_native_info  = 1,
  ompt_record_native_event = 2
\} ompt_record_native_kind_t;
\end{boxedcode}
\end{ccppspecific}
}

\subsubsection{Native Record Abstract Type}
\label{sec:ompt_record_abstract_t}

\vbox{
\begin{ccppspecific}
\begin{boxedcode}
typedef struct ompt_record_abstract_s \{
  ompt_record_native_class_t \plc{rclass};
  const char *\plc{type};
  ompt_device_time_t \plc{start\_time};
  ompt_device_time_t \plc{end\_time};
  ompt_hwid_t \plc{hwid};
\} ompt_record_abstract_t;
\end{boxedcode}
\end{ccppspecific}
}

\descr

A \code{ompt\_record\_abstract\_t} record contains several
pieces of information that a tool can use to process a native record
that it may not fully understand. The \plc{rclass} field
indicates whether the record is informational
or represents an event; knowing this can help a tool determine
how to present the record. The record \plc{type} field points to
a statically-allocated, immutable character string that provides
a meaningful name that a tool might want to use to describe the event
to a user. The \plc{start\_time} and \plc{end\_time} fields are
used to place an event in time. The times are relative to the device
clock. If an event has no associated \plc{start\_time} and/or
\plc{end\_time}, its value will be 
\code{ompt\_time\_none}. The hardware id field,
\plc{hwid},  is used to indicate the location on the device where
the event occurred. A \plc{hwid} may represent a hardware abstraction
such as a core or a hardware thread id. The meaning of a \plc{hwid}
value for a device is defined by the implementer of the software
stack for the device. If there is no hardware abstraction associated 
with the record, the value of \plc{hwid} 
will be \code{ompt\_hwid\_none}.

\subsubsection{Record Type}
\label{sec:ompt_record_ompt_t}
\vbox{
\begin{ccppspecific}
\begin{boxedcode}
typedef struct ompt_record_ompt_s \{
  ompt_callbacks_t \plc{type};
  ompt_target_time\_t \plc{time};
  ompt_id_t \plc{thread_id};
  ompt_id_t \plc{target_id};
  union \{
    ompt_record_thread_begin_t \plc{thread_begin};
    ompt_record_idle_t \plc{idle};
    ompt_record_parallel_begin_t \plc{parallel_begin};
    ompt_record_parallel_end_t \plc{parallel_end};
    ompt_record_task_create_t \plc{task_create};
    ompt_record_task_dependence_t \plc{task_dep};
    ompt_record_task_schedule_t \plc{task_sched};
    ompt_record_implicit_t \plc{implicit};
    ompt_record_sync_region_t \plc{sync_region};
    ompt_record_target_t \plc{target_record};
    ompt_record_target_data_op_t \plc{target_data_op};
    ompt_record_target_map_t \plc{target_map};
    ompt_record_target_kernel_t \plc{kernel};
    ompt_record_lock_init_t \plc{lock_init};
    ompt_record_lock_destroy_t \plc{lock_destroy};
    ompt_record_mutex_acquire_t \plc{mutex_acquire};
    ompt_record_mutex_t \plc{mutex};
    ompt_record_nest_lock_t \plc{nest_lock};
    ompt_record_master_t \plc{master};
    ompt_record_work_t \plc{work};
    ompt_record_flush_t \plc{flush};
  \} \plc{record};
\} ompt_record_ompt_t;
\end{boxedcode}
\end{ccppspecific}
}

\subsection{Miscellaneous Type Definitions}
\label{sec:ompt-types:misc}
This section describes miscellaneous types and enumerations used by the tool interface.

\ompttype{\code{ompt\_callback\_t}}
\label{sec:ompt_callback_t}

Pointers to tool callback functions with many different type
signatures are passed to the \code{ompt\_set\_callback} runtime entry point and 
returned by the \code{ompt\_get\_callback} runtime entry point. For convenience,
these runtime entry points expect all type signatures to be cast to   
a dummy type \code{ompt\_callback\_t}.

\vbox{
\begin{ccppspecific}
\begin{boxedcode}
typedef void (*ompt_callback_t)(void);
\end{boxedcode}
\end{ccppspecific}
}

% ompt_id_t

\ompttype{\code{ompt\_id\_t}}
\label{sec:ompt_id_t} 
When tracing asynchronous activity on OpenMP devices, tools need identifiers to correlate target regions and operations initiated by the host with 
associated activities on a target device. In addition, tools need identifiers to refer to parallel regions and tasks that execute on a device.
OpenMP implementations use identifiers of type \code{ompt\_id\_t} type for each of these purposes. 
The value \code{ompt\_id\_none} is
reserved to indicate an invalid id.

\vbox{
\begin{ccppspecific}
\begin{boxedcode}
typedef \ulonglongint{} ompt_id_t;
#define ompt_id_none 0
\end{boxedcode}
\end{ccppspecific}
}

Identifiers created on each device must be unique from the time an OpenMP implementation is initialized until it is shut down.  
Specifically, this means that (1) identifiers for each target region and target operation instance initiated by the host device must be unique over time on the host,
and (2) identifiers for parallel and task region instances that execute on a device must be unique over time within that device.

Tools should not assume that \code{ompt\_id\_t} values are small or densely allocated. 

% ompt_data_t

\ompttype{\code{ompt\_data\_t}}
\label{sec:ompt_data_t} 
Threads, parallel regions, and task regions 
each have an associated data object of type \code{ompt\_data\_t} reserved for use by a tool.
When an OpenMP implementation creates a thread or an instance of a parallel or task region, 
it will initialize its associated \code{ompt\_data\_t} object with the value \code{ompt\_data\_none}. 

\vbox{
\begin{ccppspecific}
\begin{boxedcode}
typedef union ompt_data_u \{\\
  \ulonglongint{} \plc{value}; 
  void *\plc{ptr};                     
\} ompt_data_t;

const ompt_data_t ompt_data_none = \{.value=0\};         
\end{boxedcode}
\end{ccppspecific}
}



% ompt_wait_id_t

\ompttype{\code{ompt\_wait\_id\_t}}
\label{sec:ompt_wait_id_t} 
\index{wait identifier}

Each thread instance maintains a \emph{wait identifier} of type \code{ompt\_wait\_id\_t}.
When a task executing on a thread is waiting for mutual exclusion, the thread's wait identifer indicates what the thread is awaiting.
A wait identifier may represent a critical section {\em name}, a lock, a program variable accessed in an atomic region, or a synchronization object internal to an OpenMP implementation.
% A thread's wait identifier can be retrieved on demand by invoking the \code{ompt\_get\_state} function (described in Section~\ref{sec:ompt_get_state}).

\vbox{
\begin{ccppspecific}
\begin{boxedcode}
typedef \ulonglongint{} ompt_wait_id_t;
const ompt_wait_id_t ompt_wait_id_none = 0;
\end{boxedcode}
\end{ccppspecific}
}

When a thread is not in a wait state, the value of the thread's wait identifier is undefined.

% Tools should not assume that \code{ompt\_wait\_id\_t} values are small or densely allocated.



% ompt_device_t

\ompttype{\code{ompt\_device\_t}}
\label{sec:ompt_device_t} 
\code{ompt\_device\_t} is an opaque object representing a device.

\vbox{
\begin{ccppspecific}
\begin{boxedcode}
typedef void ompt_device_t;
\end{boxedcode}
\end{ccppspecific}
}



% ompt_device_time_t

\ompttype{\code{ompt\_device\_time\_t}}
\label{sec:ompt_device_time_t} 
\code{ompt\_device\_time\_t} is an opaque object representing a raw time value from a device.
\label{sec:ompt_time_none} 
\code{ompt\_time\_none} refers to an uknown or unspecified time.

\vbox{
\begin{ccppspecific}
\begin{boxedcode}
typedef \ulonglongint{} ompt_device_time_t;
#define ompt_time_none 0
\end{boxedcode}
\end{ccppspecific}
}



% ompt_buffer_t

\ompttype{\code{ompt\_buffer\_t}}
\label{sec:ompt_buffer_t} 
\code{ompt\_buffer\_t} is an opaque object handle for a target buffer.

\vbox{
\begin{ccppspecific}
\begin{boxedcode}
typedef void ompt_buffer_t; 
\end{boxedcode}
\end{ccppspecific}
}



% ompt_buffer_cursor_t

\ompttype{\code{ompt\_buffer\_cursor\_t}}
\label{sec:ompt_buffer_cursor_t} 
\code{ompt\_buffer\_cursor\_t} is an opaque handle for a position in a target buffer.

\vbox{
\begin{ccppspecific}
\begin{boxedcode}
typedef \ulonglongint{} ompt_buffer_cursor_t;
\end{boxedcode}
\end{ccppspecific}
}



% ompt_task_dependence_t

\ompttype{\code{ompt\_task\_dependence\_t}}
\label{sec:ompt_task_dependence_t} 
\code{ompt\_task\_dependence\_t} is a task dependence.

\vbox{
\begin{ccppspecific}
\begin{boxedcode}
typedef struct ompt_task_dependence_s \{\\
  void *\plc{variable_addr};
  unsigned int \plc{dependence_flags};
\} ompt_task_dependence_t;
\end{boxedcode}
\end{ccppspecific}
}







% ompt_thread_type_t

\ompttype{\code{ompt\_thread\_type\_t}}
\label{sec:ompt_thread_type_t} 
\code{ompt\_thread\_type\_t} is an enumeration that defines the valid thread type values.

\vbox{
\begin{ccppspecific}
\begin{boxedcode}
typedef enum ompt_thread_type_e \{
  ompt_thread_initial                 = 1,
  ompt_thread_worker                  = 2,
  ompt_thread_other                   = 3,
  ompt_thread_unknown                 = 4
\} ompt_thread_type_t;
\end{boxedcode}
\end{ccppspecific}
}



% ompt_scope_endpoint_t

\ompttype{\code{ompt\_scope\_endpoint\_t}}
\label{sec:ompt_scope_endpoint_t} 
\code{ompt\_scope\_endpoint\_t} is an enumeration that defines valid scope endpoint values.

\vbox{
\begin{ccppspecific}
\begin{boxedcode}
typedef enum ompt_scope_endpoint_e \{
  ompt_scope_begin                    = 1,
  ompt_scope_end                      = 2
\} ompt_scope_endpoint_t;
\end{boxedcode}
\end{ccppspecific}
}



% ompt_sync_region_kind_t

\ompttype{\code{ompt\_sync\_region\_kind\_t}}
\label{sec:ompt_sync_region_kind_t} 
\code{ompt\_sync\_region\_kind\_t} is an enumeration that defines the valid sync region kind values.

\vbox{
\begin{ccppspecific}
\begin{boxedcode}
typedef enum ompt_sync_region_kind_e \{
  ompt_sync_region_barrier            = 1, 
  ompt_sync_region_taskwait           = 2,
  ompt_sync_region_taskgroup          = 3
\} ompt_sync_region_kind_t;
\end{boxedcode}
\end{ccppspecific}
}



% ompt_target_data_op_t

\ompttype{\code{ompt\_target\_data\_op\_t}}
\label{sec:ompt_target_data_op_t} 
\code{ompt\_target\_data\_op\_t} is an enumeration that defines the valid target data operation values.

\vbox{
\begin{ccppspecific}
\begin{boxedcode}
typedef enum ompt_target_data_op_e \{
  ompt_target_data_alloc              = 1,
  ompt_target_data_transfer_to_dev    = 2,
  ompt_target_data_transfer_from_dev  = 3,
  ompt_target_data_delete             = 4
\} ompt_target_data_op_t;
\end{boxedcode}
\end{ccppspecific}
}



% ompt_work_type_t

\ompttype{\code{ompt\_work\_type\_t}}
\label{sec:ompt_work_type_t} 
\code{ompt\_work\_type\_t} is an enumeration that defines the valid work type values.

\vbox{
\begin{ccppspecific}
\begin{boxedcode}
typedef enum ompt_work_type_e \{
  ompt_work_loop               = 1, 
  ompt_work_sections           = 2,
  ompt_work_single_executor    = 3,
  ompt_work_single_other       = 4,
  ompt_work_workshare          = 5,
  ompt_work_distribute         = 6,
  ompt_work_taskloop           = 7
\} ompt_work_type_t;
\end{boxedcode}
\end{ccppspecific}
}



% ompt_mutex_kind_t

\ompttype{\code{ompt\_mutex\_kind\_t}}
\label{sec:ompt_mutex_kind_t} 
\code{ompt\_mutex\_kind\_t} is an enumeration that defines the valid mutex kind values.

\vbox{
\begin{ccppspecific}
\begin{boxedcode}
typedef enum ompt_mutex_kind_e \{
  ompt_mutex                          = 0x10,
  ompt_mutex_lock                     = 0x11,
  ompt_mutex_nest_lock                = 0x12,
  ompt_mutex_critical                 = 0x13,
  ompt_mutex_atomic                   = 0x14,
  ompt_mutex_ordered                  = 0x20
\} ompt_mutex_kind_t;
\end{boxedcode}
\end{ccppspecific}
}



% ompt_native_mon_flags_t

\ompttype{\code{ompt\_native\_mon\_flags\_t}}
\label{sec:ompt_native_mon_flags_t}
\code{ompt\_native\_mon\_flags\_t} is an enumeration that defines the valid native monitoring flag values.

\vbox{
\begin{ccppspecific}
\begin{boxedcode}
typedef enum ompt_native_mon_flags_e \{
  ompt_native_data_motion_explicit    = 1,
  ompt_native_data_motion_implicit    = 2,
  ompt_native_kernel_invocation       = 4,
  ompt_native_kernel_execution        = 8,
  ompt_native_driver                  = 16,
  ompt_native_runtime                 = 32,
  ompt_native_overhead                = 64,
  ompt_native_idleness                = 128
\} ompt_native_mon_flags_t;
\end{boxedcode}
\end{ccppspecific}
}



% ompt_task_type_t

\ompttype{\code{ompt\_task\_type\_t}}
\label{sec:ompt_task_type_t}
\code{ompt\_task\_type\_t} is an enumeration that defines the valid task type values.
The least significant byte provides information about the general classification of the task.
The other bits represent properties of the task. 
\vbox{
\begin{ccppspecific}
\begin{boxedcode}
typedef enum ompt_task_type_e \{
  ompt_task_initial                   = 0x1,
  ompt_task_implicit                  = 0x2,
  ompt_task_explicit                  = 0x4,
  ompt_task_target                    = 0x8,
  ompt_task_undeferred                = 0x8000000,
  ompt_task_untied                    = 0x10000000,
  ompt_task_final                     = 0x20000000,
  ompt_task_mergeable                 = 0x40000000,
  ompt_task_merged                    = 0x80000000
\} ompt_task_type_t;
\end{boxedcode}
\end{ccppspecific}
}



%xxx
% ompt_task_status

\ompttype{\code{ompt\_task\_status\_t}}
\label{sec:ompt_task_status_t}
\code{ompt\_task\_status\_t} is an enumeration that explains the
reasons for switching a task that reached  a task scheduling point. 

\vbox{
\begin{ccppspecific}
\begin{boxedcode}
typedef enum ompt_task_status_e \{
  ompt_task_complete  = 1,
  ompt_task_yield     = 2,
  ompt_task_cancel    = 3,
  ompt_task_others    = 4
\} ompt_task_status_t;
\end{boxedcode}
\end{ccppspecific}
}

The \code{ompt\_task\_complete} indicates the completion of task that
encountered the task scheduling point. The \code{ompt\_task\_yield} indicates
that the task encountered a \code{taskyield} construct. The \code{ompt\_task\_cancel} indicates
that the taks is canceled due to the encountering of an active cancellation point resulting in the
cancellation of that task.
The \code{ompt\_task\_others} is used in the remaining cases.

% ompt_target_type_t

\ompttype{\code{ompt\_target\_type\_t}}
\label{sec:ompt_target_type_t}
\code{ompt\_target\_type\_t} is an enumeration that defines the valid target type values.

\vbox{
\begin{ccppspecific}
\begin{boxedcode}
typedef enum ompt_target_type_e \{
  ompt_target                         = 1,
  ompt_target_enter_data              = 2,
  ompt_target_exit_data               = 3,
  ompt_target_update                  = 4
\} ompt_target_type_t;
\end{boxedcode}
\end{ccppspecific}
}



% ompt_invoker_t

\ompttype{\code{ompt\_invoker\_t}}
\label{sec:ompt_invoker_t}
\code{ompt\_invoker\_t} is an enumeration that defines the valid invoker values.

\vbox{
\begin{ccppspecific}
\begin{boxedcode}
typedef enum ompt_invoker_e \{
  ompt_invoker_program = 1, /* program invokes master task */
  ompt_invoker_runtime = 2  /* runtime invokes master task */
\} ompt_invoker_t;
\end{boxedcode}
\end{ccppspecific}
}



% ompt_target_map_flag_t

\ompttype{\code{ompt\_target\_map\_flag\_t}}
\label{sec:ompt_target_map_flag_t}
\code{ompt\_target\_map\_flag\_t} is an enumeration that defines the valid target map flag values.

\vbox{
\begin{ccppspecific}
\begin{boxedcode}
typedef enum ompt_target_map_flag_e \{
  ompt_target_map_flag_to             = 1,
  ompt_target_map_flag_from           = 2,
  ompt_target_map_flag_alloc          = 4,
  ompt_target_map_flag_release        = 8, 
  ompt_target_map_flag_delete         = 16,
  ompt_target_map_flag_implicit       = 32
\} ompt_target_map_flag_t;
\end{boxedcode}
\end{ccppspecific}
}



% ompt_task_dependence_flag_t

\ompttype{\code{ompt\_task\_dependence\_flag\_t}}
\label{sec:ompt_task_dependence_flag_t}
\code{ompt\_task\_dependence\_flag\_t} is an enumeration that defines the valid task dependence flag values.

\vbox{
\begin{ccppspecific}
\begin{boxedcode}
typedef enum ompt_task_dependence_flag_e \{
  ompt_task_dependence_type_out       = 1,
  ompt_task_dependence_type_in        = 2,
  ompt_task_dependence_type_inout     = 3
\} ompt_task_dependence_flag_t;
\end{boxedcode}
\end{ccppspecific}
}




% ompt_cancel_flag_t

\ompttype{\code{ompt\_cancel\_flag\_t}}
\label{sec:ompt_cancel_flag_t}
\code{ompt\_cancel\_flag\_t} is an enumeration that defines the valid cancel flag values.

\vbox{
\begin{ccppspecific}
\begin{boxedcode}
typedef enum ompt_cancel_flag_e \{
  ompt_cancel_parallel       = 0x1,
  ompt_cancel_sections       = 0x2,
  ompt_cancel_do             = 0x4,
  ompt_cancel_taskgroup      = 0x8,
  ompt_cancel_activated      = 0x10,
  ompt_cancel_detected       = 0x20,
  ompt_cancel_discarded_task = 0x40
\} ompt_cancel_flag_t;
\end{boxedcode}
\end{ccppspecific}
}

\crossreferences
\begin{itemize}
\item \code{ompt\_cancel\_t} data type, see \specref{sec:ompt_callback_cancel_t}.
\end{itemize}

% ompt_hwid_t

\ompttype{\code{ompt\_hwid\_t}}
\label{sec:ompt_hwid_t}
\code{ompt\_hwid\_t} is an opaque object representing a hardware identifier for a target device.
\label{sec:ompt_hwid_none} 
\code{ompt\_hwid\_none} refers to an uknown or unspecified hardware id.
If there is no \code{hwid} associated with a
\code{ompt\_record\_abstract\_t}, the value of \code{hwid} shall be
\code{ompt\_hwid\_none}.

\vbox{
\begin{ccppspecific}
\begin{boxedcode}
typedef \ulonglongint{} ompt_hwid_t;
#define ompt_hwid_none 0
\end{boxedcode}
\end{ccppspecific}
}



% end miscellaneous types


\section{Tool Interface Routine}
\label{sec:tool-interface}

\ompttoolsignature{\code{ompt\_start\_tool}}
\label{sec:ompt_start_tool}

\summary
If a tool wants to use the OMPT interface provided by an OpenMP implementation,
the tool must implement \code{ompt\_start\_tool} to announce its interest.

\format
\vbox{
    \begin{ccppspecific}
    \begin{boxedcode}
        ompt_fns_t *ompt_start_tool(
        unsigned int \plc{omp_version},
        const char *\plc{runtime_version}
        );
    \end{boxedcode}
    \end{ccppspecific}
}

\descr
For a tool to use the OMPT interface provided by an OpenMP implementation,
the tool must define a globally-visible implementation of the
function \code{ompt\_start\_tool}.

A tool may indicate its intent to use the OMPT interface provided
by an OpenMP implementation by having
\code{ompt\_start\_tool} return a non-\code{NULL} pointer to an
\code{ompt\_fns\_t} structure, which contains pointers to
a tool's initializer and finalizer functions.

A tool may use its \callbackarg{} \plc{omp\_version} to determine
whether it is compatible with the OMPT interface provided by an OpenMP
implementation.

If a tool implements \code{ompt\_start\_tool} but has no interest in using
the OMPT interface in a particular execution,
\code{ompt\_start\_tool} should return \code{NULL}. 

\argdesc

The \callbackarg{} \plc{omp\_version} 
is the value of the \code{\_OPENMP} version macro 
associated with the OpenMP API implementation. This value 
identifies the OpenMP API version supported by an OpenMP implementation,
which specifies the version of the OMPT interface that it supports.

The \callbackarg{} \plc{runtime\_version}
is a version string that unambiguously identifies the OpenMP implementation.

\constraints

The \callbackarg{} \plc{runtime\_version} must be
an immutable string that is defined for the lifetime of a program
execution.

\effect
If a tool returns a non-\code{NULL} pointer,
an OpenMP implementation will call the tool initializer specified by the
\plc{initialize} field in this structure before
beginning execution of any OpenMP construct
or completing execution of any environment routine invocation; the
OpenMP implementation will call the tool finializer specified by the
\plc{finalize} field in this structure when the OpenMP
implementation shuts down.



\crossreferences
\begin{itemize}
    \item \code{ompt\_fns\_t}, see \specref{sec:ompt_fns_t}.
\end{itemize}


% This is an included file. See the master file for more information.
%
% When editing this file:
%
%    1. To change formatting, appearance, or style, please edit openmp.sty.
%
%    2. Custom commands and macros are defined in openmp.sty.
%
%    3. Be kind to other editors -- keep a consistent style by copying-and-pasting to
%       create new content.
%
%    4. We use semantic markup, e.g. (see openmp.sty for a full list):
%         \code{}     % for bold monospace keywords, code, operators, etc.
%         \plc{}      % for italic placeholder names, grammar, etc.
%
%    5. There are environments that provide special formatting, e.g. language bars.
%       Please use them whereever appropriate.  Examples are:
%
%         \begin{fortranspecific}
%         This is text that appears enclosed in blue language bars for Fortran.
%         \end{fortranspecific}
%
%         \begin{note}
%         This is a note.  The "Note -- " header appears automatically.
%         \end{note}
%
%    6. Other recommendations:
%         Use the convenience macros defined in openmp.sty for the minor headers
%         such as Comments, Syntax, etc.
%
%         To keep items together on the same page, prefer the use of 
%         \begin{samepage}.... Avoid \parbox for text blocks as it interrupts line numbering.
%         When possible, avoid \filbreak, \pagebreak, \newpage, \clearpage unless that's
%         what you mean. Use \needspace{} cautiously for troublesome paragraphs.
%
%         Avoid absolute lengths and measures in this file; use relative units when possible.
%         Vertical space can be relative to \baselineskip or ex units. Horizontal space
%         can be relative to \linewidth or em units.
%
%         Prefer \emph{} to italicize terminology, e.g.:
%             This is a \emph{definition}, not a placeholder.
%             This is a \plc{var-name}.
%


\section{Tool Callback Signatures and Trace Records}
\label{sec:ompt-tool-callbacks}

\restrictions
Tool callbacks may not use OpenMP directives or call any runtime library routines
described in Section~\ref{chap:Runtime Library Routines}.

\subsection{Initialization and Finalization Callback Signature}

\omptcallbacksignature{\code{ompt\_initialize\_t}}
\label{sec:ompt_initialize_t}

\summary
A tool implements an initializer with the type signature
\code{ompt\_initialize\_t} to initialize the tool's use of
the OMPT interace.

\format
\vbox{
\begin{ccppspecific}
\begin{boxedcode}
typedef int (*ompt_initialize_t) (
  ompt_function_lookup_t \plc{lookup},
  struct ompt_fns_t *\plc{fns}
);
\end{boxedcode}
\end{ccppspecific}
}

\descr
For a tool to initialize the OMPT interface of an OpenMP implementation,
the tool's implementation of \code{ompt\_start\_tool} must return a 
pointer to a tool initializer with type signature \code{ompt\_initialize\_t}.
An OpenMP implementation will call the tool initializer returned by
\code{ompt\_start\_tool} after fully initializing itself but before 
beginning execution of any OpenMP construct
or completing execution of any environment routine invocation.  

The initializer returns a non-zero value if it succeeds.

\argdesc
The \callbackarg{} \plc{lookup} is a callback 
to an OpenMP runtime routine that a tool must use to 
obtain a pointer to each runtime entry point in the OMPT interface.
The \callbackarg{} \plc{fns} is the value returned by \code{ompt\_start\_tool}.
The actions of a tool initializer are described in \specref{sec:tool-initialize}.

\crossreferences
\begin{itemize}
\item \code{ompt\_function\_lookup\_t}, see
  \specref{sec:ompt_function_lookup_t}.
\end{itemize}


\omptcallbacksignature{\code{ompt\_finalize\_t}}
\label{sec:ompt_finalize_t}

\summary
A tool implements an finalizer with the type signature
\code{ompt\_finalize\_t} to finalize the tool's use of
the OMPT interface.

\format
\vbox{
\begin{ccppspecific}
\begin{boxedcode}
typedef void (*ompt_finalize_t) (
  struct ompt_fns_t *\plc{fns}
);
\end{boxedcode}
\end{ccppspecific}
}

\descr
The finalizer for an OpenMP implementation is invoked by an OpenMP
implementation as it shuts down.

\argdesc
The \callbackarg{} \plc{fns} is the value returned by \code{ompt\_start\_tool}.

\crossreferences
\begin{itemize}
\item \code{ompt\_fns\_t}, see
  \specref{sec:ompt_fns_t}.
\end{itemize}


\subsection{Event Callback Signatures and Trace Records}
\index{event callback signatures}
\label{sec:ToolsSupport_callback_signatures}

This section describes the signatures of tool callback functions that an OMPT
tool might register and that are called during runtime of an OpenMP program.

\omptcallbacksignature{\code{ompt\_callback\_thread\_begin\_t}}
\index{ompt\_callback\_thread\_begin\_t@{\code{ompt\_callback\_thread\_begin\_t}}}
\label{sec:ompt_callback_thread_begin_t}
\format
\vbox{
\begin{ccppspecific}
\begin{boxedcode}
typedef void (*ompt_callback_thread_begin_t) (
  ompt_thread_type_t \plc{thread_type},
  ompt_data_t *\plc{thread_data}
);
\end{boxedcode}
\end{ccppspecific}
}

\record
\vbox{
\begin{ccppspecific}
\begin{boxedcode}
typedef struct ompt_record_thread_begin_s \{
  ompt_thread_type_t \plc{thread_type};
\} ompt_record_thread_begin_t;
\end{boxedcode}
\end{ccppspecific}
}

\argdesc

The \callbackarg{} \plc{thread\_type}
indicates the type of the new thread: initial, worker, or other.

The binding of \callbackarg{} \plc{thread\_data} is the new thread.

\crossreferences
\begin{itemize}
\item \code{ompt\_data\_t} type, see
\specref{sec:ompt_data_t}.
\item \code{ompt\_thread\_type\_t} type, see
\specref{sec:ompt_thread_type_t}.
\end{itemize}



\omptcallbacksignature{\code{ompt\_callback\_thread\_end\_t}}
\index{ompt\_callback\_thread\_end\_t@{\code{ompt\_callback\_thread\_end\_t}}}
\label{sec:ompt_callback_thread_end_t}
\format
\vbox{
\begin{ccppspecific}
\begin{boxedcode}
typedef void (*ompt_callback_thread_end_t) (
  ompt_data_t *\plc{thread_data}
);
\end{boxedcode}
\end{ccppspecific}
}

\argdesc

The binding of \callbackarg{} \plc{thread\_data} is the thread that is
terminating.

\crossreferences
\begin{itemize}
\item \code{ompt\_data\_t} type, see
\specref{sec:ompt_data_t}.
\end{itemize}


\omptcallbacksignature{\code{ompt\_callback\_idle\_t}}
\index{ompt\_callback\_idle\_t@{\code{ompt\_callback\_idle\_t}}}
\label{sec:ompt_callback_idle_t}
\format
\vbox{
\begin{ccppspecific}
\begin{boxedcode}
typedef void (*ompt_callback\_idle_t) (
  ompt_scope_endpoint_t \plc{endpoint}
);
\end{boxedcode}
\end{ccppspecific}
}

\record
\vbox{
\begin{ccppspecific}
\begin{boxedcode}
typedef struct ompt_record_idle_s \{
  ompt_scope_endpoint_t \plc{endpoint};
\} ompt_record_idle_t;
\end{boxedcode}
\end{ccppspecific}
}

\argdesc

The \callbackarg{} \plc{endpoint} indicates whether the callback is
signalling the beginning or end of an idle interval.

\crossreferences
\begin{itemize}
\item \code{ompt\_scope\_endpoint\_t} type, see
\specref{sec:ompt_scope_endpoint_t}.
\end{itemize}



\omptcallbacksignature{\code{ompt\_callback\_parallel\_begin\_t}}
\index{ompt\_callback\_parallel\_begin\_t@{\code{ompt\_callback\_parallel\_begin\_t}}}
\label{sec:ompt_callback_parallel_begin_t}
\format
\vbox{
\begin{ccppspecific}
\begin{boxedcode}
typedef void (*ompt_callback_parallel_begin_t) (
  ompt_data_t *\plc{parent_task_data},
  const ompt_frame_t *\plc{parent_frame},
  ompt_data_t *\plc{parallel_data},
  unsigned int \plc{requested_team_size},
  ompt_invoker_t \plc{invoker},
  const void *\plc{codeptr_ra}
);
\end{boxedcode}
\end{ccppspecific}
}

\record
\vbox{
\begin{ccppspecific}
\begin{boxedcode}
typedef struct ompt_record_parallel_begin_s \{
  ompt_id_t \plc{parent_task_id};
  ompt_id_t \plc{parallel_id};
  unsigned int \plc{requested_team_size};
  ompt_invoker_t \plc{invoker};
  const void *\plc{codeptr_ra};
\} ompt_record_parallel_begin_t;
\end{boxedcode}
\end{ccppspecific}
}

\argdesc

The binding of \callbackarg{} \plc{parent\_task\_data} is the
encountering task.

The \callbackarg{} \plc{parent\_frame} points to the frame object
associated with the encountering task.

The binding of \callbackarg{} \plc{parallel\_data} is the parallel
region that is beginning.

The \callbackarg{} \plc{requested\_team\_size}
indicates the number of threads requested by the user. 

The \callbackarg{} \plc{invoker} indicates whether the code for the
parallel region is inlined into the application or invoked by the
runtime.

\codeptrdesc

\crossreferences
\begin{itemize}
\item \code{ompt\_data\_t} type, see \specref{sec:ompt_data_t}.
\item \code{ompt\_frame\_t} type, see \specref{sec:ompt_frame_t}.
\item \code{ompt\_invoker\_t} type, see \specref{sec:ompt_invoker_t}.
\end{itemize}



\omptcallbacksignature{\code{ompt\_callback\_parallel\_end\_t}}
\index{ompt\_callback\_parallel\_end\_t@{\code{ompt\_callback\_parallel\_end\_t}}}
\label{sec:ompt_callback_parallel_end_t}
\format
\vbox{
\begin{ccppspecific}
\begin{boxedcode}
typedef void (*ompt_callback_parallel_end_t) (
  ompt_data_t *\plc{parallel_data},
  ompt_data_t *\plc{task_data},
  ompt_invoker_t \plc{invoker},
  const void *\plc{codeptr_ra}
);
\end{boxedcode}
\end{ccppspecific}
}

\record
\vbox{
\begin{ccppspecific}
\begin{boxedcode}
typedef struct ompt_record_parallel_end_s \{
  ompt_id_t \plc{parallel_id}; 
  ompt_id_t \plc{task_id};
  ompt_invoker_t \plc{invoker};
  const void *\plc{codeptr_ra};
\} ompt_record_parallel_end_t;
\end{boxedcode}
\end{ccppspecific}
}

\argdesc

The binding of \callbackarg{} \plc{parallel\_data} is the parallel
region that is ending.

The binding of \callbackarg{} \plc{task\_data} is the encountering
task.

The \callbackarg{} \plc{invoker} explains whether the execution of the
parallel region code is inlined into the application code or started
by the runtime.

\codeptrdesc

\crossreferences
\begin{itemize}
\item \code{ompt\_data\_t} type signature, see
\specref{sec:ompt_data_t}.
\item \code{ompt\_invoker\_t} type signature, see
\specref{sec:ompt_invoker_t}.
\end{itemize}

\omptcallbacksignature{\code{ompt\_callback\_master\_t}}
\index{ompt\_callback\_master\_t@{\code{ompt\_callback\_master\_t}}}
\label{sec:ompt_callback_master_t}
\format
\vbox{
\begin{ccppspecific}
\begin{boxedcode}
typedef void (*ompt_callback_master_t) (
  ompt_scope_endpoint_t \plc{endpoint},
  ompt_data_t *\plc{parallel_data},
  ompt_data_t *\plc{task_data},
  const void *\plc{codeptr_ra}
);
\end{boxedcode}
\end{ccppspecific}
}

\record
\vbox{
\begin{ccppspecific}
\begin{boxedcode}
typedef struct ompt_record_master_s \{
  ompt_scope_endpoint_t \plc{endpoint};
  ompt_id_t \plc{parallel_id};
  ompt_id_t \plc{task_id};
  const void *\plc{codeptr_ra};
\} ompt_record_master_t;
\end{boxedcode}
\end{ccppspecific}
}

\argdesc

\epdesc

The binding of \callbackarg{} \plc{parallel\_data} is the current parallel region.

The binding of \callbackarg{} \plc{task\_data} is the encountering task.

\codeptrdesc

%\effect
% ompt events have no effect

\crossreferences
\begin{itemize}
\item \code{ompt\_data\_t} type signature, see \specref{sec:ompt_data_t}.
\item \code{ompt\_scope\_endpoint\_t} type, see \specref{sec:ompt_scope_endpoint_t}.
\end{itemize}





\omptcallbacksignature{\code{ompt\_callback\_task\_create\_t}}
\index{ompt\_callback\_task\_create\_t@{\code{ompt\_callback\_task\_create\_t}}}
\label{sec:ompt_callback_task_create_t}
\format
\vbox{
\begin{ccppspecific}
\begin{boxedcode}
typedef void (*ompt_callback_task_create_t) (
  ompt_data_t *\plc{parent_task_data},
  const ompt_frame_t *\plc{parent_frame},
  ompt_data_t *\plc{new_task_data},
  ompt_task_type_t \plc{type},
  int \plc{has_dependences},
  const void *\plc{codeptr_ra}
);
\end{boxedcode}
\end{ccppspecific}
}

\record
\vbox{
\begin{ccppspecific}
\begin{boxedcode}
typedef struct ompt_record_task_create_s \{
  ompt_id_t \plc{parent_task_id};
  ompt_id_t \plc{new_task_id};
  ompt_task_type_t \plc{type};
  int \plc{has_dependences};
  const void *\plc{codeptr_ra};
\} ompt_record_task_create_t;
\end{boxedcode}
\end{ccppspecific}
}

\argdesc

The binding of \callbackarg{} \plc{parent\_task\_data} is the
encountering task.  
This parameter is \code{NULL} for an initial task.

The \callbackarg{} \plc{parent\_frame} points to the frame object
associated with the encountering task. 
This parameter is \code{NULL} for an initial task.

The binding of \callbackarg{} \plc{new\_task\_data} is the created
task.

The \callbackarg{} \plc{type} indicates the kind of the task: initial,
explicit or target.

The \callbackarg{} \plc{has\_dependences} indicates whether created
task has dependences.

\codeptrdesc

\crossreferences
\begin{itemize}
\item \code{ompt\_data\_t} type, see
\specref{sec:ompt_data_t}.
\item \code{ompt\_frame\_t} type, see
\specref{sec:ompt_frame_t}.
\item \code{ompt\_task\_type\_t} type, see
\specref{sec:ompt_task_type_t}.
\end{itemize}



\omptcallbacksignature{\code{ompt\_callback\_task\_dependences\_t}}
\index{ompt\_callback\_task\_dependences\_t@{\code{ompt\_callback\_task\_dependences\_t}}}
\label{sec:ompt_callback_task_dependences_t}
\format
\vbox{
\begin{ccppspecific}
\begin{boxedcode}
typedef void (*ompt_callback_task_dependences_t) (
  ompt_data_t *\plc{task_data},
  const ompt_task_dependence_t *\plc{deps},
  int \plc{ndeps}
);
\end{boxedcode}
\end{ccppspecific}
}

\argdesc

The binding of \callbackarg{} \plc{task\_data} is the task being created.

The \callbackarg{} \plc{deps} lists all dependences of a new task.

The \callbackarg{} \plc{ndeps} specifies the length of the list.
The memory for \plc{deps} is owned by the caller; the tool cannot rely on
the data after the callback returns.

\crossreferences
\begin{itemize}
\item \code{ompt\_data\_t} type, see
\specref{sec:ompt_data_t}.
\item \code{ompt\_task\_dependence\_t} type, see
\specref{sec:ompt_task_dependence_t}.
\end{itemize}



\omptcallbacksignature{\code{ompt\_callback\_task\_dependence\_t}}
\index{ompt\_callback\_task\_dependence\_t@{\code{ompt\_callback\_task\_dependence\_t}}}
\label{sec:ompt_callback_task_dependence_t}
\format
\vbox{
\begin{ccppspecific}
\begin{boxedcode}
typedef void (*ompt_callback_task_dependence_t) (
  ompt_data_t *\plc{src_task_data},
  ompt_data_t *\plc{sink_task_data}
);
\end{boxedcode}
\end{ccppspecific}
}

\record
\vbox{
\begin{ccppspecific}
\begin{boxedcode}
typedef struct ompt_record_task_dependence_s \{
  ompt_id_t \plc{src_task_id};
  ompt_id_t \plc{sink_task_id};
\} ompt_record_task_dependence_t;
\end{boxedcode}
\end{ccppspecific}
}

\argdesc

The binding of \callbackarg{} \plc{src\_task\_data} is a running task
with an outgoing dependence.

The binding of \callbackarg{} \plc{sink\_task\_data} is a task with an
unsatisfied incoming dependence.


\crossreferences
\begin{itemize}
\item \code{ompt\_data\_t} type signature, see
\specref{sec:ompt_data_t}.
\end{itemize}



\omptcallbacksignature{\code{ompt\_callback\_task\_schedule\_t}}
\index{ompt\_callback\_task\_schedule\_t@{\code{ompt\_callback\_task\_schedule\_t}}}
\label{sec:ompt_callback_task_schedule_t}
\format
\vbox{
\begin{ccppspecific}
\begin{boxedcode}
typedef void (*ompt_callback_task_schedule_t) (
  ompt_data_t *\plc{prior_task_data},
  ompt_task_status_t \plc{prior_task_status},
  ompt_data_t *\plc{next_task_data}
);
\end{boxedcode}
\end{ccppspecific}
}

\record
\vbox{
\begin{ccppspecific}
\begin{boxedcode}
typedef struct ompt_record_task_schedule_s \{
  ompt_id_t \plc{prior_task_id};
  ompt_task_status_t \plc{prior_task_status},
  ompt_id_t \plc{next_task_id};
\} ompt_record_task_schedule_t;
\end{boxedcode}
\end{ccppspecific}
}

\argdesc

The \callbackarg{} \plc{prior\_task\_status} indicates the status of
the task that arrived at a task scheduling point.

The binding of \callbackarg{} \plc{prior\_task\_data} is the task that
arrived at the scheduling point.

The binding of \callbackarg{} \plc{next\_task\_data} is the task that
will resume at the scheduling point.

\crossreferences
\begin{itemize}
\item \code{ompt\_data\_t} type, see
\specref{sec:ompt_data_t}.

\item \code{ompt\_task\_status\_t} type, see
\specref{sec:ompt_task_status_t}.
\end{itemize}



\omptcallbacksignature{\code{ompt\_callback\_implicit\_task\_t}}
\index{ompt\_callback\_implicit\_task\_t@{\code{ompt\_callback\_implicit\_task\_t}}}
\label{sec:ompt_callback_implicit_task_t}
\format
\vbox{
\begin{ccppspecific}
\begin{boxedcode}
typedef void (*ompt_callback_implicit_task_t) (
  ompt_scope_endpoint_t \plc{endpoint},
  ompt_data_t *\plc{parallel_data},
  ompt_data_t *\plc{task_data},
  unsigned int \plc{team_size},
  unsigned int \plc{thread_num}
);
\end{boxedcode}
\end{ccppspecific}
}

\record
\vbox{
\begin{ccppspecific}
\begin{boxedcode}
typedef struct ompt_record_implicit_s \{
  ompt_scope_endpoint_t \plc{endpoint};
  ompt_id_t \plc{parallel_id};
  ompt_id_t \plc{task_id};
  unsigned int \plc{team_size},
  unsigned int \plc{thread_num};
\} ompt_record_implicit_t;
\end{boxedcode}
\end{ccppspecific}
}

\argdesc

\epdesc

The binding of \callbackarg{} \plc{parallel\_data} is the current
parallel region.

The binding of \callbackarg{} \plc{task\_data} is the implicit task
executing the parallel region's structured block.

The \callbackarg{} \plc{team\_size} indicates the number of
threads in the team.

The \callbackarg{} \plc{thread\_num} indicates
the thread number of the calling thread, within the team executing the parallel
region to which the implicit region binds.


\crossreferences
\begin{itemize}
\item \code{ompt\_data\_t} type, see
\specref{sec:ompt_data_t}.
\item \code{ompt\_scope\_endpoint\_t} enumeration type, see
\specref{sec:ompt_scope_endpoint_t}.
\end{itemize}



\omptcallbacksignature{\code{ompt\_callback\_sync\_region\_t}}
\index{ompt\_callback\_sync\_region\_t@{\code{ompt\_callback\_sync\_region\_t}}}
\label{sec:ompt_callback_sync_region_t}
\format
\vbox{
\begin{ccppspecific}
\begin{boxedcode}
typedef void (*ompt_callback_sync_region_t) (
  ompt_sync_region_kind_t \plc{kind},
  ompt_scope_endpoint_t \plc{endpoint},
  ompt_data_t *\plc{parallel_data},
  ompt_data_t *\plc{task_data},
  const void *\plc{codeptr_ra}
);
\end{boxedcode}
\end{ccppspecific}
}

\record
\vbox{
\begin{ccppspecific}
\begin{boxedcode}
typedef struct ompt_record_sync_region_s \{
  ompt_sync_region_kind_t \plc{kind};
  ompt_scope_endpoint_t \plc{endpoint};
  ompt_id_t \plc{parallel_id};
  ompt_id_t \plc{task_id};
  const void *\plc{codeptr_ra};
\} ompt_record_sync_region_t;
\end{boxedcode}
\end{ccppspecific}
}

% \descr
% A tool executes a callback with type signature
% \code{ompt\_callback\_sync\_region\_t} to receive notification of
% event \code{ompt\_callback\_sync\_region\_t} when an OpenMP
% implementation starts and stops waiting in a barrier region, taskwait
% region, or taskgroup region.

\argdesc

The \callbackarg{} \plc{kind} indicates the kind of
synchronization region.

\epdesc

The binding of \callbackarg{} \plc{parallel\_data} is the current
parallel region.

The binding of \callbackarg{} \plc{task\_data} is the current task.

\codeptrdesc

\crossreferences
\begin{itemize}
\item \code{ompt\_data\_t} type, see
\specref{sec:ompt_data_t}.
\item \code{ompt\_sync\_region\_kind\_t} type, see
\specref{sec:ompt_sync_region_kind_t}.
\item \code{ompt\_scope\_endpoint\_t} type, see
\specref{sec:ompt_scope_endpoint_t}.
\end{itemize}



\omptcallbacksignature{\code{ompt\_callback\_lock\_init\_t}}
\index{ompt\_callback\_lock\_init\_t@{\code{ompt\_callback\_lock\_init\_t}}}
\label{sec:ompt_callback_lock_init_t}
\format
\vbox{
\begin{ccppspecific}
\begin{boxedcode}
typedef void (*ompt_callback_lock_init_t) (
  ompt_mutex_kind_t \plc{kind},
  unsigned int \plc{hint},
  unsigned int \plc{impl},
  ompt_wait_id_t \plc{wait_id},
  const void *\plc{codeptr_ra}
);
\end{boxedcode}
\end{ccppspecific}
}

\record
\vbox{
\begin{ccppspecific}
\begin{boxedcode}
typedef struct ompt_record_lock_init_s \{
  ompt_mutex_kind_t \plc{kind};
  unsigned int \plc{hint};
  unsigned int \plc{impl};
  ompt_wait_id_t \plc{wait_id};
  const void *\plc{codeptr_ra};
\} ompt_record_lock_init_t;
\end{boxedcode}
\end{ccppspecific}
}

\argdesc

The \callbackarg{} \plc{kind} indicates the kind of the lock.

The \callbackarg{} \plc{hint} indicates the hint provided when initializing
an implementation of mutual exclusion.

The \callbackarg{} \plc{impl} indicates the mechanism chosen by the
runtime to implement the mutual exclusion.

The \callbackarg{} \plc{wait\_id} indicates the object being awaited. 

\codeptrdesc


\crossreferences
\begin{itemize}
\item \code{ompt\_wait\_id\_t} type, see
\specref{sec:ompt_wait_id_t}.
\end{itemize}



\omptcallbacksignature{\code{ompt\_callback\_lock\_destroy\_t}}
\index{ompt\_callback\_lock\_destroy\_t@{\code{ompt\_callback\_lock\_destroy\_t}}}
\label{sec:ompt_callback_lock_destroy_t}
\format
\vbox{
\begin{ccppspecific}
\begin{boxedcode}
typedef void (*ompt_callback_lock_destroy_t) (
  ompt_mutex_kind_t \plc{kind},
  ompt_wait_id_t \plc{wait_id},
  const void *\plc{codeptr_ra}
);
\end{boxedcode}
\end{ccppspecific}
}

\record
\vbox{
\begin{ccppspecific}
\begin{boxedcode}
typedef struct ompt_record_lock_destroy_s \{
  ompt_mutex_kind_t \plc{kind};
  ompt_wait_id_t \plc{wait_id};
  const void *\plc{codeptr_ra};
\} ompt_record_lock_destroy_t;
\end{boxedcode}
\end{ccppspecific}
}

\argdesc

The \callbackarg{} \plc{kind} indicates the kind of the lock.

The \callbackarg{} \plc{wait\_id} identifies the lock.

\codeptrdesc


\crossreferences
\begin{itemize}
\item \code{ompt\_wait\_id\_t} type, see
\specref{sec:ompt_wait_id_t}.
\end{itemize}



\omptcallbacksignature{\code{ompt\_callback\_mutex\_acquire\_t}}
\index{ompt\_callback\_mutex\_acquire\_t@{\code{ompt\_callback\_mutex\_acquire\_t}}}
\label{sec:ompt_callback_mutex_acquire_t}
\format
\vbox{
\begin{ccppspecific}
\begin{boxedcode}
typedef void (*ompt_callback_mutex_acquire_t) (
  ompt_mutex_kind_t \plc{kind},
  unsigned int \plc{hint},
  unsigned int \plc{impl},
  ompt_wait_id_t \plc{wait_id},
  const void *\plc{codeptr_ra}
);
\end{boxedcode}
\end{ccppspecific}
}

\record
\vbox{
\begin{ccppspecific}
\begin{boxedcode}
typedef struct ompt_record_mutex_acquire_s \{
  ompt_mutex_kind_t \plc{kind};
  unsigned int \plc{hint};
  unsigned int \plc{impl};
  ompt_wait_id_t \plc{wait_id};
  const void *\plc{codeptr_ra};
\} ompt_record_mutex_acquire_t;
\end{boxedcode}
\end{ccppspecific}
}

\argdesc

The \callbackarg{} \plc{kind} indicates the kind of the lock.

The \callbackarg{} \plc{hint} indicates the hint provided when initializing
an implementation of mutual exclusion.
If no hint is available when a thread initiates acquisition of mutual exclusion,
the runtime may supply \code{omp\_lock\_hint\_none} as the value for \plc{hint}.

The \callbackarg{} \plc{impl} indicates the mechanism chosen by the
runtime to implement the mutual exclusion.  

The \callbackarg{} \plc{wait\_id} indicates the object being awaited. 

\codeptrdesc

\crossreferences
\begin{itemize}
\item \code{ompt\_wait\_id\_t} type, see
\specref{sec:ompt_wait_id_t}.
\item \code{ompt\_mutex\_kind\_t} type, see
\specref{sec:ompt_mutex_kind_t}.
\end{itemize}



\omptcallbacksignature{\code{ompt\_callback\_mutex\_t}}
\index{ompt\_callback\_mutex\_t@{\code{ompt\_callback\_mutex\_t}}}
\label{sec:ompt_callback_mutex_t}
\format
\vbox{
\begin{ccppspecific}
\begin{boxedcode}
typedef void (*ompt_callback_mutex_t) (
  ompt_mutex_kind_t \plc{kind},
  ompt_wait_id_t \plc{wait_id},
  const void *\plc{codeptr_ra}
);
\end{boxedcode}
\end{ccppspecific}
}

\record
\vbox{
\begin{ccppspecific}
\begin{boxedcode}
typedef struct ompt_record_mutex_s \{
  ompt_mutex_kind_t \plc{kind};
  ompt_wait_id_t \plc{wait_id};
  const void *\plc{codeptr_ra};
\} ompt_record_mutex_t;
\end{boxedcode}
\end{ccppspecific}
}

\argdesc

The \callbackarg{} \plc{kind} indicates the kind of mutual exclusion event.

The \callbackarg{} \plc{wait\_id} indicates the object being awaited. 

\codeptrdesc

\crossreferences
\begin{itemize}
\item \code{ompt\_wait\_id\_t} type signature, see
\specref{sec:ompt_wait_id_t}.
\item \code{ompt\_mutex\_kind\_t} type signature, see
\specref{sec:ompt_mutex_kind_t}.
\end{itemize}



\omptcallbacksignature{\code{ompt\_callback\_nest\_lock\_t}}
\index{ompt\_callback\_nest\_lock\_t@{\code{ompt\_callback\_nest\_lock\_t}}}
\label{sec:ompt_callback_nest_lock_t}
\format
\vbox{
\begin{ccppspecific}
\begin{boxedcode}
typedef void (*ompt_callback_nest_lock_t) (
  ompt_scope_endpoint_t \plc{endpoint},
  ompt_wait_id_t \plc{wait_id},
  const void *\plc{codeptr_ra}
);
\end{boxedcode}
\end{ccppspecific}
}

\record
\vbox{
\begin{ccppspecific}
\begin{boxedcode}
typedef struct ompt_record_nest_lock_s \{
  ompt_scope_endpoint_t \plc{endpoint};
  ompt_wait_id_t \plc{wait_id};
  const void *\plc{codeptr_ra};
\} ompt_record_nest_lock_t;
\end{boxedcode}
\end{ccppspecific}
}

\argdesc

\epdesc

The \callbackarg{} \plc{wait\_id} indicates the object being awaited. 

\codeptrdesc

\crossreferences
\begin{itemize}
\item \code{ompt\_wait\_id\_t} type signature, see
\specref{sec:ompt_wait_id_t}.
\item \code{ompt\_scope\_endpoint\_t} type signature, see
\specref{sec:ompt_scope_endpoint_t}.
\end{itemize}



\omptcallbacksignature{\code{ompt\_callback\_work\_t}}
\index{ompt\_callback\_work\_t@{\code{ompt\_callback\_work\_t}}}
\label{sec:ompt_callback_work_t}
\format
\vbox{
\begin{ccppspecific}
\begin{boxedcode}
typedef void (*ompt_callback_work_t) (
  ompt_work_type_t \plc{wstype},
  ompt_scope_endpoint_t \plc{endpoint},
  ompt_data_t *\plc{parallel_data},
  ompt_data_t *\plc{task_data},
  \longlongint{} \plc{count},
  const void *\plc{codeptr_ra}
);
\end{boxedcode}
\end{ccppspecific}
}

\record
\vbox{
\begin{ccppspecific}
\begin{boxedcode}
typedef struct ompt_record_work_s \{
  ompt_work_type_t \plc{wstype};
  ompt_scope_endpoint_t \plc{endpoint};
  ompt_id_t \plc{parallel_id};
  ompt_id_t \plc{task_id};
  \longlongint{} \plc{count};
  const void *\plc{codeptr_ra};
\} ompt_record_work_t;
\end{boxedcode}
\end{ccppspecific}
}

\argdesc

The \callbackarg{} \plc{wstype} indicates the kind of worksharing
region.

\epdesc

The binding of \callbackarg{} \plc{parallel\_data} is the current
parallel region.

The binding of \callbackarg{} \plc{task\_data} is the current task.

The \callbackarg{} \plc{count} is a measure of the quantity of work involved in the worksharing construct. 
For a loop construct, \plc{count} represents the number of iterations of the loop.
For a \code{taskloop} construct, \plc{count} represents the number of iterations in the iteration space, 
which may be the result of collapsing several associated loops.
For a \code{sections} construct, \plc{count} represents the number of sections. 
For a \code{workshare} construct, \plc{count} represents the units of work, as defined by the \code{workshare} construct.
For a \code{single} construct, \plc{count} is always 1.

\codeptrdesc


\crossreferences
\begin{itemize}
\item worksharing constructs, see \specref{sec:Worksharing Constructs}.
\item \code{ompt\_data\_t} type signature, see
\specref{sec:ompt_data_t}.
\item \code{ompt\_scope\_endpoint\_t} type signature, see
\specref{sec:ompt_scope_endpoint_t}.
\item \code{ompt\_work\_type\_t} type signature, see
\specref{sec:ompt_work_type_t}.
\end{itemize}



\omptcallbacksignature{\code{ompt\_callback\_flush\_t}}
\index{ompt\_callback\_flush\_t@{\code{ompt\_callback\_flush\_t}}}
\label{sec:ompt_callback_flush_t}
\format
\vbox{
\begin{ccppspecific}
\begin{boxedcode}
typedef void (*ompt_callback_flush_t) (
  ompt_data_t *\plc{thread_data},
  const void *\plc{codeptr_ra}
);
\end{boxedcode}
\end{ccppspecific}
}

\record
\vbox{
\begin{ccppspecific}
\begin{boxedcode}
typedef struct ompt_record_flush_s \{
  void *\plc{codeptr_ra};
\} ompt_record_flush_t;
\end{boxedcode}
\end{ccppspecific}
}

\argdesc

\codeptrdesc

\crossreferences
\begin{itemize}
\item \code{ompt\_data\_t} type signature, see
\specref{sec:ompt_data_t}.
\end{itemize}



\omptcallbacksignature{\code{ompt\_callback\_target\_t}}
\index{ompt\_callback\_target\_t@{\code{ompt\_callback\_target\_t}}}
\label{sec:ompt_callback_target_t}
\format
\vbox{
\begin{ccppspecific}
\begin{boxedcode}
typedef void (*ompt_callback_target_t) (
  ompt_target_type_t \plc{kind},
  ompt_scope_endpoint_t \plc{endpoint},
  uint64_t \plc{device_num},
  ompt_data_t *\plc{task_data},
  ompt_id_t \plc{target_id},
  const void *\plc{codeptr_ra}
);
\end{boxedcode}
\end{ccppspecific}
}

\record
\vbox{
\begin{ccppspecific}
\begin{boxedcode}
typedef struct ompt_record_target_s \{
  ompt_target_type_t \plc{kind};
  ompt_scope_endpoint_t \plc{endpoint};
  uint64_t \plc{device_num};
  ompt_data_t *\plc{task_data};
  ompt_id_t \plc{target_id};
  const void *\plc{codeptr_ra};
\} ompt_record_target_t;
\end{boxedcode}
\end{ccppspecific}
}

\argdesc

The \callbackarg{} \plc{kind} indicates the kind of target region.

\epdesc

The \callbackarg{} \plc{device\_num} indicates the id of the device
which will execute the target region.

The binding of \callbackarg{} \plc{task\_data} is the target task.

The binding of \callbackarg{} \plc{target\_id} is the target region.

\codeptrdesc

\crossreferences
\begin{itemize}
\item \code{ompt\_id\_t} type, see
\specref{sec:ompt_id_t}.
\item \code{ompt\_data\_t} type signature, see
\specref{sec:ompt_data_t}.
\item \code{ompt\_scope\_endpoint\_t} type signature, see
\specref{sec:ompt_scope_endpoint_t}.
\item \code{ompt\_target\_type\_t} type signature, see
\specref{sec:ompt_target_type_t}.
\end{itemize}



\omptcallbacksignature{\code{ompt\_callback\_target\_data\_op\_t}}
\index{ompt\_callback\_target\_data\_op\_t@{\code{ompt\_callback\_target\_data\_op\_t}}}
\label{sec:ompt_callback_target_data_op_t}
\format

\vbox{
\begin{ccppspecific}
\begin{boxedcode}
typedef void (*ompt_callback_target_data_op_t) (
  ompt_id_t \plc{target_id},
  ompt_id_t \plc{host_op_id},
  ompt_target_data_op_t \plc{optype},
  void *\plc{host_addr},
  void *\plc{device_addr},
  size_t \plc{bytes}
);
\end{boxedcode}
\end{ccppspecific}
}

\record
\vbox{
\begin{ccppspecific}
\begin{boxedcode}
typedef struct ompt_record_target_data_op_s \{
  ompt_id_t \plc{host\_op\_id};
  ompt_target_data_op_t \plc{optype};
  void *\plc{host\_addr};
  void *\plc{device\_addr};
  size_t \plc{bytes};
  ompt_device_time_t \plc{end\_time};
\} ompt_record_target_data_op_t;
\end{boxedcode}
\end{ccppspecific}
}

\argdesc

The \callbackarg{} \plc{host\_op\_id} is a unique identifer for a data
operations on a target device.

The \callbackarg{} \plc{optype} indicates the kind of data mapping.

The \callbackarg{} \plc{host\_addr} indicates the address of data on
host side.

The \callbackarg{} \plc{device\_addr} indicates the address of data on
device side.

The \callbackarg{} \plc{bytes} indicates the size of data.

\crossreferences
\begin{itemize}
\item \code{ompt\_id\_t} type, see
\specref{sec:ompt_id_t}.
\item \code{ompt\_target\_data\_op\_t} type signature, see
\specref{sec:ompt_target_data_op_t}.
\end{itemize}



\omptcallbacksignature{\code{ompt\_callback\_target\_map\_t}}
\index{ompt\_callback\_target\_map\_t@{\code{ompt\_callback\_target\_map\_t}}}
\label{sec:ompt_callback_target_map_t}
\format
\vbox{
\begin{ccppspecific}
\begin{boxedcode}
typedef void (*ompt_callback_target_map_t) (
  ompt_id_t \plc{target_id},
  unsigned int \plc{nitems},
  void **\plc{host_addr},
  void **\plc{device_addr},
  size_t *\plc{bytes},
  unsigned int *\plc{mapping_flags}
);
\end{boxedcode}
\end{ccppspecific}
}

\record
\vbox{
\begin{ccppspecific}
\begin{boxedcode}
typedef struct ompt_record_target_map_s \{
  ompt_id_t \plc{target_id};
  unsigned int \plc{nitems};
  void **\plc{host_addr};
  void **\plc{device_addr};
  size_t *\plc{bytes};
  unsigned int *\plc{mapping_flags};
\} ompt_record_target_map_t;
\end{boxedcode}
\end{ccppspecific}
}

\argdesc

The binding of \callbackarg{} \plc{target\_id} is the target region.

The \callbackarg{} \plc{nitems} indicates the number of data mappings.

The \callbackarg{} \plc{host\_addr} indicates an array of addresses of
data on host side.

The \callbackarg{} \plc{device\_addr} indicates an array of addresses
of data on device side.

The \callbackarg{} \plc{bytes} indicates an array of size of data.

The \callbackarg{} \plc{mapping\_flags} indicates the kind of data
mapping.


\crossreferences
\begin{itemize}
\item \code{ompt\_id\_t} type, see
\specref{sec:ompt_id_t}.
\end{itemize}



\omptcallbacksignature{\code{ompt\_callback\_target\_submit\_t}}
\index{ompt\_callback\_target\_submit\_t@{\code{ompt\_callback\_target\_submit\_t}}}
\label{sec:ompt_callback_target_submit_t}
\format
\vbox{
\begin{ccppspecific}
\begin{boxedcode}
typedef void (*ompt_callback_target_submit_t) (
  ompt_id_t \plc{target_id},
  ompt_id_t \plc{host_op_id}
);
\end{boxedcode}
\end{ccppspecific}
}

\descr
This callback is invoked when a target task creates an initial task on a
target device. 

\argdesc

The \callbackarg{} \plc{target\_id} is a unique identifier for the
associated target region.

The \callbackarg{} \plc{host\_op\_id} is a unique identifer for the
initial task on the target device.

\constraints

The \callbackarg{} \plc{target\_id} indicates the instance of the
target construct to which the computation belongs.

The \callbackarg{} \plc{host\_op\_id} provides a unique host-side
identifier that represents the computation on the device.

\record
\vbox{
\begin{ccppspecific}
\begin{boxedcode}
typedef struct ompt_record_target_kernel_s \{
  ompt_id_t \plc{host_op_id};
  unsigned int \plc{requested_num_teams};
  unsigned int \plc{granted_num_teams};
  ompt_device_time_t \plc{end_time};
\} ompt_record_target_kernel_t;
\end{boxedcode}
\end{ccppspecific}
}

\crossreferences
\begin{itemize}
\item \code{ompt\_id\_t} type, see
\specref{sec:ompt_id_t}.
\end{itemize}


\omptcallbacksignature{\code{ompt\_callback\_buffer\_request\_t}}
\index{ompt\_callback\_buffer\_request\_t@{\code{ompt\_callback\_buffer\_request\_t}}}
\label{sec:ompt_callback_buffer_request_t}

\summary
The OpenMP runtime will invoke a callback with type signature  
\code{ompt\_callback\_buffer\_request\_t} to request a
buffer to store event records for a device.

\format
\vbox{
\begin{ccppspecific}
\begin{boxedcode}
typedef void (*ompt_callback_buffer_request_t) (
  uint64_t \plc{device_num},
  ompt_buffer_t **\plc{buffer},
  size_t *\plc{bytes}
);
\end{boxedcode}
\end{ccppspecific}
}

\descr
The callback requests a buffer to store trace records for the
specified device.

A buffer request callback may set \plc{*bytes} to 0 if it does not
want to provide a buffer for any reason. If a callback sets
\plc{*bytes} to 0, further recording of events for the device will be
disabled until the next invocation of \code{ompt\_start\_trace}.  This
will cause the device to drop future trace records until recording is
restarted.

The buffer request callback is not required to be \emph{async signal safe}.

\argdesc

The \callbackarg{} \plc{device\_num} specifies the device. 

A tool should set \plc{*buffer} to point to a buffer where device events
may be recorded and \plc{*bytes} to the length of that buffer.  

\crossreferences
\begin{itemize}
\item \code{ompt\_buffer\_t} type, see
\specref{sec:ompt_buffer_t}.
\end{itemize}

\omptcallbacksignature{\code{ompt\_callback\_buffer\_complete\_t}}
\index{ompt\_callback\_buffer\_complete\_t@{\code{ompt\_callback\_buffer\_complete\_t}}}
\label{sec:ompt_callback_buffer_complete_t}
\summary
A device triggers a call to \code{ompt\_callback\_buffer\_complete\_t} when no further records will be recorded in an event buffer and all records written to the buffer are valid. 

\format
\vbox{
\begin{ccppspecific}
\begin{boxedcode}
typedef void (*ompt_callback_buffer_complete_t) (
  uint64_t \plc{device_num},
  const ompt_buffer_t *\plc{buf},
  size_t \plc{bytes},
  ompt_buffer_cursor_t \plc{begin},
  int \plc{buffer_owned}
);
\end{boxedcode}
\end{ccppspecific}
}

\descr 
The callback provides a tool with a buffer containing trace records for the
specified device. Typically, a tool will iterate through the records
in the buffer and process them.

The OpenMP implementation will make these callbacks
on a thread that is not an OpenMP master or worker. 

The callee may delete the buffer if the \callbackarg{}
\plc{buffer\_owned}=0.

The buffer completion callback is not 
required to be \emph{async signal safe}.

\argdesc

The \callbackarg{} \plc{device\_num} indicates the device whose events
the buffer contains.

The \callbackarg{} \plc{buffer} is the address of a buffer previously
allocated by a \emph{buffer request} callback.

The \callbackarg{} \plc{bytes} indicates the full size of the buffer.

The \callbackarg{} \plc{begin} is an opaque cursor that indicates the
position at the beginning of the first record in the buffer.

The \callbackarg{} \plc{buffer\_owned} is 1 if the data pointed to by
buffer can be deleted by the callback and 0 otherwise. If multiple
devices accumulate trace events into a single buffer, this callback
might be invoked with a pointer to one or more trace records in a
shared buffer with \plc{buffer\_owned} = 0. In this case, the callback
may not delete the buffer.

\crossreferences
\begin{itemize}
\item \code{ompt\_buffer\_t} type, see
\specref{sec:ompt_buffer_t}.
\item \code{ompt\_buffer\_cursor\_t} type, see
\specref{sec:ompt_buffer_cursor_t}.
\end{itemize}

\omptcallbacksignature{\code{ompt\_callback\_control\_tool\_t}}
\index{ompt\_callback\_control\_tool\_t@{\code{ompt\_callback\_control\_tool\_t}}}
\label{sec:ompt_callback_control_tool_t}
\format
\vbox{
\begin{ccppspecific}
\begin{boxedcode}
typedef int (*ompt_callback_control\_tool_t) (
  \longlongint{} \plc{command},
  \longlongint{} \plc{modifier},
  void *\plc{arg}
);
\end{boxedcode}
\end{ccppspecific}
}

\descr

The tool control callback may return any non-negative value, which will be returned to the 
application by the OpenMP implementation as the return value of the 
\code{omp\_control\_tool} call that triggered the callback.

\argdesc 

The \callbackarg{} \plc{command} passes a command from an application
to a tool.  Standard values for \plc{command} are defined by
\code{omp\_control\_tool\_t}.  defined in \specref{sec:control_tool}.

The \callbackarg{} \plc{modifier} passes a command modifier from an
application to a tool.

The callback allows tool-specific values for \plc{command} and
\plc{modifier}.  Tools must ignore \plc{command} values that they are
not explicitly designed to handle.

The \callbackarg{} \plc{arg} is a void pointer that enables a tool and
an application to pass arbitrary state back and forth. 
The \callbackarg{} \plc{arg} may be \code{NULL}.



\constraints
Tool-specific values for \plc{command} must be $\geq$ 64.

\crossreferences
\begin{itemize}
\item \code{omp\_control\_tool\_t} enumeration type, see \specref{sec:control_tool}.
\end{itemize}

\omptcallbacksignature{\code{ompt\_callback\_cancel\_t}}
\index{ompt\_callback\_cancel\_t@{\code{ompt\_callback\_cancel\_t}}}
\label{sec:ompt_callback_cancel_t}
\format
\vbox{
\begin{ccppspecific}
\begin{boxedcode}
typedef void (*ompt_callback_cancel_t) (
  ompt_data_t *\plc{task_data},
  int \plc{flags},
  const void *\plc{codeptr_ra}
  );
\end{boxedcode}
\end{ccppspecific}
}

\argdesc 

The \callbackarg{} \plc{task\_data} corresponds to the task
encountering a \code{cancel} construct, a \code{cancellation point}
construct, or a construct defined as having an implicit cancellation
point.

The \callbackarg{} \plc{flags}, defined by the enumeration
\code{ompt\_cancel\_flag\_t}, indicates whether the cancel is
activated by the current task, or detected as being activated by
another task.  The construct being canceled is also described in the
\plc{flags}. When several constructs are detected as being
concurrently canceled, each corresponding bit in the flags will be
set.

\codeptrdesc

\crossreferences
\begin{itemize}
\item \code{omp\_cancel\_flag\_t} enumeration type, see \specref{sec:ompt_cancel_flag_t}.
\end{itemize}


\omptcallbacksignature{\code{ompt\_callback\_device\_initialize\_t}}
\index{ompt\_callback\_flush\_t@{\code{ompt\_callback\_device\_initialize\_t}}}
\label{sec:ompt_callback_device_initialize_t}

\summary The tool callback with type signature
\code{ompt\_callback\_device\_initialize\_t} initializes a
tool's tracing interface for a device.

\format
\vbox{
\begin{ccppspecific}
\begin{boxedcode}
typedef void (*ompt_callback_device_initialize_t) (
  uint64_t \plc{device_num},
  const char *\plc{type},
  ompt_device_t *\plc{device},
  ompt_function_lookup_t *\plc{lookup},
  const char *\plc{documentation}
);
\end{boxedcode}
\end{ccppspecific}
}

\descr 

A tool that wants to asynchronously collect a trace of
activities on a device should register a callback with type signature
\code{ompt\_callback\_device\_initialize\_t} for the
\code{ompt\_callback\_device\_initialize} OpenMP event. An OpenMP
implementation will invoke this callback for a device after OpenMP is
initialized for the device but before beginning execution of any
OpenMP construct on the device.

\argdesc

The \callbackarg{} \plc{device\_num} identifies the logical device
being initialized.

The \callbackarg{} \plc{type} is a character string indicating the
type of the device. A device type string is a semicolon separated
character string that includes at a minimum the vendor and model name
of the device. This may be followed by a semicolon-separated sequence
of properties that describe a device's hardware or software.

\devicedesc

The \callbackarg{} \plc{lookup} is a pointer to a runtime callback
that a tool must use to obtain pointers to runtime entry points in the
device's OMPT tracing interface. If a device does not support tracing,
it should provide \code{NULL} for \plc{lookup}.

The \callbackarg{} \plc{documentation} is a string that describes
how to use any device-specific runtime
entry points that can be obtained using \plc{lookup}. This
documentation string could simply be a pointer to external
documentation, or it could be inline descriptions 
that includes names and type signatures for any
device-specific interfaces that are available through \plc{lookup}
along with descriptions of how to use these interface functions to
control monitoring and analysis of device traces.

\constraints
The \callbackarg{}s \plc{type} and \plc{documentation} must be
immutable strings that are defined for the lifetime of a program
execution.

\effect

A tool's device initializer has several duties.  First, it should use
\plc{type} to determine whether the tool has any special knowledge
about a device's hardware and/or software.  Second, it should use
\plc{lookup} to look up pointers to runtime entry points in the OMPT tracing
interface for the device.  Finally, using these runtime entry points, it can
then set up tracing for a device.

Initializing tracing for a target device is described in section
\specref{sec:tracing-device-activity}. 

\crossreferences
\begin{itemize}
\item \code{ompt\_function\_lookup\_t}, see
  \specref{sec:ompt_function_lookup_t}.
\end{itemize}


% This is an included file. See the master file for more information.
%
% When editing this file:
%
%    1. To change formatting, appearance, or style, please edit openmp.sty.
%
%    2. Custom commands and macros are defined in openmp.sty.
%
%    3. Be kind to other editors -- keep a consistent style by copying-and-pasting to
%       create new content.
%
%    4. We use semantic markup, e.g. (see openmp.sty for a full list):
%         \code{}     % for bold monospace keywords, code, operators, etc.
%         \plc{}      % for italic placeholder names, grammar, etc.
%
%    5. There are environments that provide special formatting, e.g. language bars.
%       Please use them whereever appropriate.  Examples are:
%
%         \begin{fortranspecific}
%         This is text that appears enclosed in blue language bars for Fortran.
%         \end{fortranspecific}
%
%         \begin{note}
%         This is a note.  The "Note -- " header appears automatically.
%         \end{note}
%
%    6. Other recommendations:
%         Use the convenience macros defined in openmp.sty for the minor headers
%         such as Comments, Syntax, etc.
%
%         To keep items together on the same page, prefer the use of
%         \begin{samepage}.... Avoid \parbox for text blocks as it interrupts line numbering.
%         When possible, avoid \filbreak, \pagebreak, \newpage, \clearpage unless that's
%         what you mean. Use \needspace{} cautiously for troublesome paragraphs.
%
%         Avoid absolute lengths and measures in this file; use relative units when possible.
%         Vertical space can be relative to \baselineskip or ex units. Horizontal space
%         can be relative to \linewidth or em units.
%
%         Prefer \emph{} to italicize terminology, e.g.:
%             This is a \emph{definition}, not a placeholder.
%             This is a \plc{var-name}.
%


\subsection{OMPT Runtime Entry Points for Tools}
\label{sec:entry-points}

OMPT supports two principal sets of runtime entry points for tools. One set of 
runtime entry points enables a tool to register callbacks for OpenMP events and 
to inspect the state of an OpenMP thread while executing in a tool callback or 
a signal handler. The second set of runtime entry points enables a tool to trace 
activities on a device. When directed by the tracing interface, an OpenMP 
implementation will trace activities on a device, collect buffers of trace 
records, and invoke callbacks on the host to process these records. OMPT runtime 
entry points should not be global symbols since tools cannot rely on the 
visibility of such symbols.

OMPT also supports runtime entry points for two classes of lookup routines. The 
first class of lookup routines contains a single member: a routine that returns 
runtime entry points in the OMPT callback interface. The second class of lookup 
routines includes a unique lookup routine for each kind of device that can return 
runtime entry points in a device's OMPT tracing interface.

The C/C++ header file (omp-tools.h) provides the definitions of
the types that are specified throughout this subsection.

\restrictions

OMPT runtime entry points have the following restrictions:

\begin{itemize}
\item OMPT runtime entry points must not be called from a signal handler
      on a native thread before a \plc{native-thread-begin} or after a 
      \plc{native-thread-end} event.
\item OMPT device runtime entry points must not be called after a 
      \plc{device-finalize} event for that device.
\end{itemize}



\subsubsection{Entry Points in the OMPT Callback Interface}
\label{sec:ompt-callback-entry-points}

Entry points in the OMPT callback interface enable a tool to register callbacks 
for OpenMP events and to inspect the state of an OpenMP thread while executing 
in a tool callback or a signal handler. Pointers to these runtime entry points 
are obtained through the lookup function that is provided through the OMPT initializer.

\subsubsubsection{\hcode{ompt_enumerate_states_t}}
\label{sec:ompt_enumerate_states_t}
\label{sec:ompt_enumerate_states}

\summary
The \code{ompt_enumerate_states_t} type is the type signature of 
the \code{ompt_enumerate_states} runtime entry point, which
enumerates the thread states that an OpenMP implementation supports.

\format
\begin{ccppspecific}
\begin{omptInquiry}
typedef int (*ompt_enumerate_states_t) (
  int \plc{current_state},
  int *\plc{next_state},
  const char **\plc{next_state_name}
);
\end{omptInquiry}
\end{ccppspecific}

\descr
An OpenMP implementation may support only a subset of the states defined by
the \code{omp_state_t} enumeration type. An OpenMP implementation may also
support implementation-specific states. The \code{ompt_enumerate_states}
runtime entry point, which has type signature \code{ompt_enumerate_states_t}, 
 enables a tool to enumerate the supported thread states.

When a supported thread state is passed as \plc{current_state}, the runtime 
entry point assigns the next thread state in the enumeration to the variable 
passed by reference in \plc{next_state} and assigns the name associated with 
that state to the character pointer passed by reference in \plc{next_state_name}.

Whenever one or more states are left in the enumeration, the 
\code{ompt_enumerate_states} runtime entry point returns $1$. When 
the last state in the enumeration is passed as \plc{current_state}, 
\code{ompt_enumerate_states} returns $0$, which indicates that the 
enumeration is complete.

\argdesc
The \plc{current_state} argument must be a thread state that the OpenMP 
implementation supports. To begin enumerating the supported states, a tool 
should pass \code{omp_state_undefined} as \plc{current_state}. Subsequent
invocations of \code{ompt_enumerate_states} should pass the value assigned 
to the variable passed by reference in \plc{next_state} to the previous call.

The value \code{ompt_state_undefined} is reserved to indicate an invalid thread 
state. \code{ompt_state_undefined} is defined as an integer with the value 0.

The \plc{next_state} argument is a pointer to an integer in which 
\code{ompt_enumerate_states} returns the value of the next state in the enumeration.

The \plc{next_state_name} argument is a pointer to a character string 
pointer through which \code{ompt_enumerate_states} returns a string 
that describes the next state.

\constraints
Any string returned through the \plc{next_state_name} argument must be 
immutable and defined for the lifetime of a program execution.

\crossreferences
\begin{itemize}
\item \code{omp_state_t}, see \specref{sec:thread-states}.
\end{itemize}



\subsubsubsection{\hcode{ompt_enumerate_mutex_impls_t}}
\label{sec:ompt_enumerate_mutex_impls_t}
\label{sec:ompt_enumerate_mutex_impls}
\label{sec:ompt_mutex_impl_none}

\summary
The \code{ompt_enumerate_mutex_impls_t} type is the type signature of 
the \code{ompt_enumerate_mutex_impls} runtime entry point, which
enumerates the kinds of mutual exclusion implementations that an OpenMP 
implementation employs.

\format
\begin{ccppspecific}
\begin{omptInquiry}
typedef int (*ompt_enumerate_mutex_impls_t) (
  int \plc{current_impl},
  int *\plc{next_impl},
  const char **\plc{next_impl_name}
);
\end{omptInquiry}
\end{ccppspecific}

\descr
Mutual exclusion for locks, \code{critical} sections, and \code{atomic} 
regions may be implemented in several ways. The \code{ompt_enumerate_mutex_impls} 
runtime entry point, which has type signature \code{ompt_enumerate_mutex_impls_t},
enables a tool to enumerate the supported mutual exclusion implementations.

When a supported mutex implementation is passed as \plc{current_impl}, the 
runtime entry point assigns the next mutex implementation in the enumeration 
to the variable passed by reference in \plc{next_impl} and assigns the name 
associated with that mutex implementation to the character pointer passed by 
reference in \plc{next_impl_name}.

Whenever one or more mutex implementations are left in the enumeration, the 
\code{ompt_enumerate_mutex_impls} runtime entry point returns $1$. When the 
last mutex implementation in the enumeration is passed as \plc{current_impl}, the 
runtime entry point returns $0$, which indicates that the enumeration is complete.

\argdesc
The \plc{current_impl} argument must be a mutex implementation that an OpenMP 
implementation supports. To begin enumerating the supported mutex implementations, 
a tool should pass \code{ompt_mutex_impl_none} as \plc{current_impl}. 
Subsequent invocations of \code{ompt_enumerate_mutex_impls} should pass the
value assigned to the variable passed in \plc{next_impl} to the previous call.

The value \code{ompt_mutex_impl_none} is reserved to indicate an invalid 
mutex implementation. \code{ompt_mutex_impl_none} is defined as an integer 
with the value 0.

The \plc{next_impl} argument is a pointer to an integer in which 
\code{ompt_enumerate_mutex_impls} returns the value of the next mutex 
implementation in the enumeration.

The \plc{next_impl_name} argument is a pointer to a character string pointer
in which \code{ompt_enumerate_mutex_impls} returns a string that describes
the next mutex implementation.

\constraints
Any string returned through the \plc{next_impl_name} argument must be 
immutable and defined for the lifetime of a program execution.

\crossreferences
\begin{itemize}
\item \code{ompt_mutex_t}, see \specref{sec:ompt_mutex_t}.
\end{itemize}



\subsubsubsection{\hcode{ompt_set_callback_t}}
\label{sec:ompt_set_callback_t}
\label{sec:ompt_set_callback}

\summary
The \code{ompt_set_callback_t} type is the type signature of the 
\code{ompt_set_callback} runtime entry point, which registers a
pointer to a tool callback that an OpenMP implementation invokes 
when a host OpenMP event occurs.

\format
\begin{ccppspecific}
\begin{omptCallback}
typedef int (*ompt_set_callback_t) (
  ompt_callbacks_t \plc{event},
  ompt_callback_t \plc{callback}
);
\end{omptCallback}
\end{ccppspecific}

\descr
OpenMP implementations can use callbacks to indicate the occurrence of events 
during the execution of an OpenMP program. The \code{ompt_set_callback} runtime
entry point, which has type signature \code{ompt_set_callback_t}, registers a 
callback for an OpenMP event on the current device,

The return value of \code{ompt_set_callback} may indicate several possible 
outcomes. Callback registration may fail if it is called outside the initializer 
for the callback interface,which is indicated by the \code{omp_set_error} return 
value. Otherwise, the return value indicates if \emph{dispatching} a callback 
event leads to the invocation of the callback. The \code{ompt_set_never} return 
value indicates that the callback will never be invoked at runtime. The 
\code{ompt_set_sometimes} return value indicates that the callback will be 
invoked at runtime for an implementation-defined subset of associated event 
occurrences. The \code{ompt_set_sometimes_paired} return value indicates the 
same result as \code{ompt_set_sometimes}, and, in addition, that a callback
with an \plc{endpoint} value of \code{ompt_scope_begin} will be invoked if 
and only if the same callback with an \plc{endpoint} value of \code{ompt_scope_end} 
will also be invoked sometime in the future. The \code{ompt_set_always} return 
value indicates that the callback will always be invoked when an associated 
event occurs.

\argdesc
The \plc{event} argument indicates the event for which the callback is 
being registered.

The \plc{callback} argument is a tool callback function. If \plc{callback}
is \code{NULL} then callbacks associated with \plc{event} are disabled. If
callbacks are successfully disabled then \code{ompt_set_always} is returned.

\constraints
When a tool registers a callback for an event, the type signature for the 
callback must match the type signature appropriate for the event.

\begin{table}
\caption{Return Codes for \code{ompt_set_callback} and
    \code{ompt_set_trace_ompt}\label{table:ToolsSupport_set_rc}}
\begin{omptEnum}
typedef enum ompt_set_result_t {
  ompt_set_error            = 0,
  ompt_set_never            = 1,
  ompt_set_sometimes        = 2,
  ompt_set_sometimes_paired = 3,
  ompt_set_always           = 4
} ompt_set_result_t;
\end{omptEnum}
\end{table}

\crossreferences
\begin{itemize}
\item \code{ompt_callbacks_t} enumeration type, see \specref{sec:ompt_callbacks_t}.

\item \code{ompt_callback_t} type, see \specref{sec:ompt_callback_t}.

\item \code{ompt_get_callback_t} host callback type signature,
see \specref{sec:ompt_get_callback_t}.
\end{itemize}



\subsubsubsection{\hcode{ompt_get_callback_t}}
\label{sec:ompt_get_callback_t}
\label{sec:ompt_get_callback}

\summary
The \code{ompt_get_callback_t} type is the type signature of the 
\code{ompt_get_callback} runtime entry point, which retrieves a
pointer to a registered tool callback routine (if any) that an 
OpenMP implementation invokes when a host OpenMP event occurs.

\format
\begin{ccppspecific}
\begin{omptCallback}
typedef int (*ompt_get_callback_t) (
  ompt_callbacks_t \plc{event},
  ompt_callback_t *\plc{callback}
);
\end{omptCallback}
\end{ccppspecific}

\descr
The \code{ompt_get_callback} runtime entry point, which has type signature 
\code{ompt_get_callback_t}, retrieves a pointer to the tool callback that
an OpenMP implementation may invoke when a host OpenMP event occurs. If a 
non-null tool callback is registered for the specified event, the pointer 
to the tool callback is assigned to the variable passed by reference in
\plc{callback} and \code{ompt_get_callback} returns 1; otherwise, it returns 0. 
If \code{ompt_get_callback} returns 0, the value of the variable passed by 
reference as \plc{callback} is undefined.

\argdesc
The \plc{event} argument indicates the event for which the callback would 
be invoked.

The \plc{callback} argument returns a pointer to the callback associated 
with \plc{event}.

\constraints
The \plc{callback} argument must be a reference to a variable of specified type.

\crossreferences
\begin{itemize}
\item \code{ompt_callbacks_t} enumeration type, see \specref{sec:ompt_callbacks_t}.

\item \code{ompt_callback_t} type, see \specref{sec:ompt_callback_t}.

\item \code{ompt_set_callback_t} type signature,
see \specref{sec:ompt_set_callback_t}.
\end{itemize}


\subsubsubsection{\hcode{ompt_get_thread_data_t}}
\label{sec:ompt_get_thread_data_t}
\label{sec:ompt_get_thread_data}

\summary
The \code{ompt_get_thread_data_t} type is the type signature of the 
\code{ompt_get_thread_data} runtime entry point, which returns the
address of the thread data object for the current thread.

\format
\begin{ccppspecific}
\begin{omptInquiry}
typedef ompt_data_t *(*ompt_get_thread_data_t) (void);
\end{omptInquiry}
\end{ccppspecific}

\binding
The binding thread for the \code{ompt_get_thread_data} runtime entry 
point is the current thread.

\descr
Each OpenMP thread can have an associated thread data object of type 
\code{ompt_data_t}. The \code{ompt_get_thread_data} runtime entry point, 
which has type signature \code{ompt_get_thread_data_t}, retrieves a pointer 
to the thread data object, if any, that is associated with the current thread. 
A tool may use a pointer to an OpenMP thread's data object that 
\code{ompt_get_thread_data} retrieves to inspect or to modify the value of 
the data object. When an OpenMP thread is created, its data object is
initialized with value \code{ompt_data_none}.

This runtime entry point is \emph{async signal safe}.

\crossreferences
\begin{itemize}
\item \code{ompt_data_t} type, see \specref{sec:ompt_data_t}.
\end{itemize}



\subsubsubsection{\hcode{ompt_get_num_procs_t}}
\label{sec:ompt_get_num_procs_t}

\summary
The \code{ompt_get_num_procs_t} type is the type signature of the 
\code{ompt_get_num_procs} runtime entry point, which returns the
number of processors currently available to the execution environment 
on the host device.

\format
\begin{ccppspecific}
\begin{omptInquiry}
typedef int (*ompt_get_num_procs_t) (void);
\end{omptInquiry}
\end{ccppspecific}

\binding
The binding thread set for the \code{ompt_get_num_procs} runtime entry 
point is all threads on the host device.

\descr
The \code{ompt_get_num_procs} runtime entry point, which has type 
signature  \code{ompt_get_num_procs_t}, returns the number of processors 
that are available on the host device at the time the routine is called. 
This value may change between the time that it is determined and the 
time that it is read in the calling context due to system actions 
outside the control of the OpenMP implementation.

This runtime entry point is \emph{async signal safe}.



\subsubsubsection{\hcode{ompt_get_num_places_t}}
\label{sec:ompt_get_num_places_t}
\label{sec:ompt_get_num_places}

\summary
The \code{ompt_get_num_places_t} type is the type signature of the 
\code{ompt_get_num_places} runtime entry point, which returns the
number of places currently available to the execution environment 
in the place list.

\format
\begin{ccppspecific}
\begin{omptInquiry}
typedef int (*ompt_get_num_places_t) (void);
\end{omptInquiry}
\end{ccppspecific}

\binding
The binding thread set for the \code{ompt_get_num_places} runtime entry 
point is all threads on a device.

\descr
The \code{ompt_get_num_places} runtime entry point, which has type signature 
\code{ompt_get_num_places_t}, returns the number of places in the place list. 
This value is equivalent to the number of places in the \plc{place-partition-var} 
ICV in the execution environment of the initial task.

This runtime entry point is \emph{async signal safe}.

\crossreferences
\begin{itemize}
\item \plc{place-partition-var} ICV, see
\specref{sec:Internal Control Variables}.

\item \code{OMP_PLACES} environment variable, see
\specref{sec:OMP_PLACES}.
\end{itemize}




\subsubsubsection{\hcode{ompt_get_place_proc_ids_t}}
\label{sec:ompt_get_place_proc_ids_t}
\label{sec:ompt_get_place_proc_ids}

\summary
The \code{ompt_get_place_procs_ids_t} type is the type signature of the 
\code{ompt_get_num_place_procs_ids} runtime entry point, which returns the
the numerical identifiers of the processors that are available to the execution 
environment in the specified place.

\format
\begin{ccppspecific}
\begin{omptInquiry}
typedef int (*ompt_get_place_proc_ids_t) (
  int \plc{place_num},
  int \plc{ids_size},
  int *\plc{ids}
);
\end{omptInquiry}
\end{ccppspecific}

\binding
The binding thread set for the \code{ompt_get_place_proc_ids} runtime 
entry point is all threads on a device.

\descr
The \code{ompt_get_place_proc_ids} runtime entry point, which has type 
signature \code{ompt_get_place_proc_ids_t}, returns the numerical 
identifiers of each processor that is associated with the specified 
place. These numerical identifiers are non-negative and their meaning 
is implementation defined.

\argdesc
The \plc{place_num} argument specifies the place that is being queried.

The \plc{ids} argument is an array in which the routine can return
a vector of processor identifiers in the specified place.

The \plc{ids_size} argument indicates the size of the result array that
is specified by \plc{ids}.

\effect
If the \plc{ids} array of size \plc{ids_size} is large enough to contain 
all identifiers then they are returned in \plc{ids} and their order in the 
array is implementation defined. Otherwise, if the \plc{ids} array is too 
small the the values in \plc{ids} when the function returns are unspecified.
The routine always returns the number of numerical identifiers of the 
processors that are available to the execution environment in the specified place.



\subsubsubsection{\hcode{ompt_get_place_num_t}}
\label{sec:ompt_get_place_num_t}
\label{sec:ompt_get_place_num}

\summary
The \code{ompt_get_place_num_t} type is the type signature of the 
\code{ompt_get_place_num} runtime entry point, which returns the
place number of the place to which the current thread is bound.

\format
\begin{ccppspecific}
\begin{omptInquiry}
typedef int (*ompt_get_place_num_t) (void);
\end{omptInquiry}
\end{ccppspecific}

\binding
The binding thread set of the \code{ompt_get_place_num} runtime entry point
is the current thread.

\descr
When the current thread is bound to a place, \code{ompt_get_place_num}
returns the place number associated with the thread. The returned value 
is between 0 and one less than the value returned by \code{ompt_get_num_places}, 
inclusive. When the current thread is not bound to a place, the routine returns -1.

This runtime entry point is \emph{async signal safe}.



\subsubsubsection{\hcode{ompt_get_partition_place_nums_t}}
\label{sec:ompt_get_partition_place_nums_t}
\label{sec:ompt_get_partition_place_nums}

\summary
The \code{ompt_get_partition_place_nums_t} type is the type signature 
of the \code{ompt_get_partition_place_nums} runtime entry point, which 
returns a list of place numbers that correspond to the places in the 
\plc{place-partition-var} ICV of the innermost implicit task.

\format
\begin{ccppspecific}
\begin{omptInquiry}
typedef int (*ompt_get_partition_place_nums_t) (
  int \plc{place_nums_size},
  int *\plc{place_nums}
);
\end{omptInquiry}
\end{ccppspecific}

\binding
The binding task set for the \code{ompt_get_partition_place_nums} runtime 
entry point is the current implicit task.

\descr
The \code{ompt_get_partition_place_nums} runtime entry point, which has
type signature \code{ompt_get_partition_place_nums_t}, returns a list of 
place numbers that correspond to the places in the \plc{place-partition-var}
ICV of the innermost implicit task.

This runtime entry point is \emph{async signal safe}.

\argdesc
The \plc{place_nums} argument is an array in which the routine can 
return a vector of place identifiers.

The \plc{place_nums_size} argument indicates the size of the result
array that the \plc{place_nums} argument specifies.

\effect
If the \plc{place_nums} array of size \plc{place_nums_size} is large 
enough to contain all identifiers then they are returned in \plc{place_nums} 
and their order in the array is implementation defined. Otherwise, if the 
\plc{place_nums} array is too small, the values in \plc{place_nums} when 
the function returns are unspecified. The routine always returns the number 
of places in the \plc{place-partition-var} ICV of the innermost implicit task.

\crossreferences
\begin{itemize}
\item \plc{place-partition-var} ICV, see
\specref{sec:Internal Control Variables}.

\item \code{OMP_PLACES} environment variable, see
\specref{sec:OMP_PLACES}.
\end{itemize}



\subsubsubsection{\hcode{ompt_get_proc_id_t}}
\label{sec:ompt_get_proc_id_t}
\label{sec:ompt_get_proc_id}

\summary
The \code{ompt_get_proc_id_t} type is the type signature of the 
\code{ompt_get_proc_id} runtime entry point, which returns the
numerical identifier of the processor of the current thread.

\format
\begin{ccppspecific}
\begin{omptInquiry}
typedef int (*ompt_get_proc_id_t) (void);
\end{omptInquiry}
\end{ccppspecific}

\binding
The binding thread set for the \code{ompt_get_proc_id} runtime entry point
is the current thread.

\descr
The \code{ompt_get_proc_id} runtime entry point, which has type signature
\code{ompt_get_proc_id_t}, returns the numerical identifier of the processor 
of the current thread. A defined numerical identifier is non-negative and
its meaning is implementation defined. A negative number indicates a failure 
to retrieve the numerical identifier.

This runtime entry point is \emph{async signal safe}.



\subsubsubsection{\hcode{ompt_get_state_t}}
\label{sec:ompt_get_state_t}
\label{sec:ompt_get_state}

\summary
The \code{ompt_get_state_t} type is the type signature of the 
\code{ompt_get_state} runtime entry point, which returns the
the state and the wait identifier of the current thread.

\format
\begin{ccppspecific}
\begin{omptInquiry}
typedef int (*ompt_get_state_t) (
  omp_wait_id_t *\plc{wait_id}
);
\end{omptInquiry}
\end{ccppspecific}

\binding
The binding thread for the \code{ompt_get_state} runtime entry point 
is the current thread.

\descr
Each OpenMP thread has an associated state and a wait identifier. If
a thread's state indicates that the thread is waiting for mutual exclusion
then its wait identifier contains an opaque handle that indicates the 
data object upon which the thread is waiting. The \code{ompt_get_state} 
runtime entry point, which has type signature \code{ompt_get_state_t},
retrieves the state and wait identifier of the current thread.
The returned value may be any one of the states predefined by 
\code{omp_state_t} or a value that represents any implementation
specific state. The tool may obtain a string representation for each 
state with the \code{ompt_enumerate_states} function.

If the returned state indicates that the thread is waiting for a
lock, nest lock, critical section, atomic region, or ordered region
then the value of the thread's wait identifier is assigned to a
non-null wait identifier passed as the \plc{wait_id} argument.

This runtime entry point is \emph{async signal safe}.

\argdesc
The \plc{wait_id} argument is a pointer to an opaque handle that is
available to receive the value of the thread's wait identifier. If
\plc{wait_id} is not \code{NULL} then the entry point assigns the 
value of the thread's wait identifier to the object to which 
\plc{wait_id} points. If the returned state is not one of the specified
wait states then the value of opaque object to which \plc{wait_id} points
is undefined after the call.

\constraints
The argument passed to the entry point must be a reference
to a variable of the specified type or \code{NULL}.

\crossreferences
\begin{itemize}
\item \code{ompt_enumerate_states_t} type, see \specref{sec:ompt_enumerate_states_t}.

\item \code{omp_state_t} type, see \specref{sec:omp_state_t}.

\item \code{omp_wait_id_t} type, see \specref{sec:omp_wait_id_t}.
\end{itemize}



\subsubsubsection{\hcode{ompt_get_parallel_info_t}}
\label{sec:ompt_get_parallel_info_t}
\label{sec:ompt_get_parallel_info}

\summary
The \code{ompt_get_parallel_info_t} type is the type signature of the 
\code{ompt_get_parallel_info} runtime entry point, which returns information 
about the parallel region, if any, at the specified ancestor level for 
the current execution context.

\format
\begin{ccppspecific}
\begin{omptInquiry}
typedef int (*ompt_get_parallel_info_t) (
  int \plc{ancestor_level},
  ompt_data_t **\plc{parallel_data},
  int *\plc{team_size}
);
\end{omptInquiry}
\end{ccppspecific}

\descr
During execution, an OpenMP program may employ nested parallel regions.
The \code{ompt_get_parallel_info} runtime entry point known, which has
type signature \code{ompt_get_parallel_info_t}, retrieves information,
about the current parallel region and any enclosing parallel regions 
for the current execution context. The entry point returns 2 if there 
is a parallel region at the specified ancestor level and the information 
is available, 1 if there is a parallel region at the specified ancestor 
level but the information is currently unavailable, and 0 otherwise.

A tool may use the pointer to a parallel region's data object that it
obtains from this runtime entry point to inspect or to modify the value
of the data object. When a parallel region is created, its data
object will be initialized with the value \code{ompt_data_none}.

This runtime entry point is \emph{async signal safe}.

Between a \emph{parallel-begin} event and an \emph{implicit-task-begin} event, 
a call to \code{ompt_get_parallel_info(0,...)} may return information about 
the outer parallel team, the new parallel team or an inconsistent state.

If a thread is in the state \code{omp_state_wait_barrier_implicit_parallel}
then a call to \code{ompt_get_parallel_info} may return a pointer to a copy 
of the specified parallel region's \plc{parallel_data} rather than a pointer 
to the data word for the region itself. This convention enables the master 
thread for a parallel region to free storage for the region immediately after 
the region ends, yet avoid having some other thread in the region's team potentially 
reference the region's \plc{parallel_data} object after it has been freed.

\argdesc
The \plc{ancestor_level} argument specifies the parallel region of 
interest by its ancestor level. Ancestor level 0 refers to the innermost 
parallel region; information about enclosing parallel regions may be 
obtained using larger values for \plc{ancestor_level}.

The \plc{parallel_data} argument returns the parallel data if the 
argument is not \code{NULL}.

The \plc{team_size} argument returns the team size if the argument 
is not \code{NULL}.

\effect
If the runtime entry point returns 0 or 1, no argument is modified. 
Otherwise, \code{ompt_get_parallel_info} has the following effects:

\begin{itemize}
\item If a non-null value was passed for \plc{parallel_data}, the value 
      returned in \plc{parallel_data} is a pointer to a data word that is
      associated with the parallel region at the specified level; and
\item If a non-null value was passed for \plc{team_size}, the value
      returned in the integer to which \plc{team_size} point is the 
      number of threads in the team that is associated with the parallel region.
\end{itemize}

\constraints
While argument \plc{ancestor_level} is passed by value, all other arguments 
to the entry point must be pointers to variables of the specified types or \code{NULL}.

\crossreferences
\begin{itemize}
\item \code{ompt_data_t} type, see \specref{sec:ompt_data_t}.
\end{itemize}



\subsubsubsection{\hcode{ompt_get_task_info_t}}
\label{sec:ompt_get_task_info_t}
\label{sec:ompt_get_task_info}

\summary
The \code{ompt_get_task_info_t} type is the type signature of the 
\code{ompt_get_task_info} runtime entry point, which returns information 
about the task, if any, at the specified ancestor level in the current 
execution context.

\format
\begin{ccppspecific}
\begin{omptInquiry}
typedef int (*ompt_get_task_info_t) (
  int \plc{ancestor_level},
  int *\plc{flag},
  ompt_data_t **\plc{task_data},
  omp_frame_t **\plc{task_frame},
  ompt_data_t **\plc{parallel_data},
  int *\plc{thread_num}
);
\end{omptInquiry}
\end{ccppspecific}

\descr
During execution, an OpenMP thread may be executing an OpenMP task.
Additionally, the thread's stack may contain procedure frames that 
are associated with suspended OpenMP tasks or OpenMP runtime system 
routines. To obtain information about any task on the current thread's 
stack, a tool uses the \code{ompt_get_task_info} runtime entry point,
which has type signature \code{ompt_get_task_info_t}.

Ancestor level 0 refers to the active task; information about other 
tasks with associated frames present on the stack in the current execution 
context may be queried at higher ancestor levels.

The \code{ompt_get_task_info} runtime entry point returns 2 if there 
is a task region at the specified ancestor level and the information 
is available, 1 if there is a task region at the specified ancestor 
level but the information is currently unavailable, and 0 otherwise.

If a task exists at the specified ancestor level and the information is 
available then information is returned in the variables passed by reference 
to the entry point. If no task region exists at the specified ancestor level 
or the information is unavailable then the values of variables passed by 
reference to the entry point are undefined when \code{ompt_get_task_info} returns.

A tool may use a pointer to a data object for a task or parallel region 
that it obtains from \code{ompt_get_task_info} to inspect or to modify the
value of the data object. When either a parallel region or a task region 
is created, its data object will be initialized with the value \code{ompt_data_none}.

This runtime entry point is \emph{async signal safe}.

\argdesc
The \plc{ancestor_level} argument specifies the task region of interest by 
its ancestor level. Ancestor level 0 refers to the active task; information 
about ancestor tasks found in the current execution context may be queried 
at higher ancestor levels.

The \plc{flag} argument returns the task type if the argument is not \code{NULL}.

The \plc{task_data} argument returns the task data if the argument is not \code{NULL}.

The \plc{task_frame} argument returns the task frame pointer
if the argument is not \code{NULL}.

The \plc{parallel_data} argument returns the parallel data
if the argument is not \code{NULL}.

The \plc{thread_num} argument returns the thread number
if the argument is not \code{NULL}.

\effect
If the runtime entry point returns 0 or 1, no argument is modified.
Otherwise, \code{ompt_get_task_info} has the following effects:

\begin{itemize}
\item If a non-null value was passed for \plc{flag} then the value 
      returned in the integer to which \plc{flag} points represents 
      the type of the task at the specified level; possible task types 
      include initial, implicit, explicit, and target tasks;
\item If a non-null value was passed for \plc{task_data} then the value that
      is returned in the object to which it points is a pointer to a data word
      that is associated with the task at the specified level;
\item If a non-null value was passed for \plc{task_frame} then the value that
      is returned in the object to which \plc{task_frame} points is a pointer 
      to the \code{omp_frame_t} structure that is associated with the task at 
      the specified level;
\item If a non-null value was passed for \plc{parallel_data} then the value that
      is returned in the object to which \plc{parallel_data} points is a pointer 
      to a data word that is associated with the parallel region that contains 
      the task at the specified level or, if the task at the specified level is 
      an initial task, \code{NULL}; and
\item If a non-null value was passed for \plc{thread_num} then the value that 
      is returned in the object to which \plc{thread_num} points indicates the 
      number of the thread in the parallel region that is executing the task
      at the specified level.
\end{itemize}

\constraints
While argument \plc{ancestor_level} is passed by value, all other arguments to 
\code{ompt_get_task_info} must be pointers to variables of the specified types 
or \code{NULL}.

\crossreferences
\begin{itemize}
\item \code{ompt_data_t} type, see \specref{sec:ompt_data_t}.

\item \code{ompt_task_flag_t} type, see \specref{sec:ompt_task_flag_t}.

\item \code{omp_frame_t} type, see \specref{sec:omp_frame_t}.

\item Example of the use of \code{omp_frame_t} structures with multiple
      threads and nested parallelism, see Appendix~\ref{chap:frames}.
\end{itemize}



\subsubsubsection{\hcode{ompt_get_task_memory_t}}
\label{sec:ompt_get_task_memory_t}
\label{sec:ompt_get_task_memory}

\summary
The \code{ompt_get_task_memory_t} type is the type signature of the 
\code{ompt_get_task_memory} runtime entry point, which returns information 
about memory ranges that are associated with the task.

\format
\begin{ccppspecific}
\begin{omptInquiry}
typedef int (*ompt_get_task_memory_t)(
  void **\plc{addr},
  size_t *\plc{size},
  int \plc{block}
);
\end{omptInquiry}
\end{ccppspecific}

\descr
During execution, an OpenMP thread may be executing an OpenMP task. The 
OpenMP implementation must preserve the data environment from the creation 
of the task for the execution of the task. The \code{ompt_get_task_memory} 
runtime entry point, which has type signature \code{ompt_get_task_memory_t},
provides information about the  memory ranges used to store the data 
environment for the current task.

Multiple memory ranges may be used to store these data. The \plc{block} 
argument supports iteration over these memory ranges.

The \code{ompt_get_task_memory} runtime entry point returns 1 if there 
are more memory ranges available, and 0 otherwise. If no memory is used 
for a task, \plc{size} is set to 0. In this case, addr is unspecified.

This runtime entry point is \emph{async signal safe}.

\argdesc
The \plc{addr} argument is a pointer to a void pointer return value
to provide the start address of a memory block.  

The \plc{size} argument is a pointer to a size type return
value to provide the size of the memory block.

The \plc{block} argument is an integer value to specify the
memory block of interest.



\subsubsubsection{\hcode{ompt_get_target_info_t}}
\label{sec:ompt_get_target_info_t}
\label{sec:ompt_get_target_info}

\summary
The \code{ompt_get_target_info_t} type is the type signature of the 
\code{ompt_get_target_info} runtime entry point, which returns identifiers 
that specify a thread's current \code{target} region and target operation ID, if any.

\format
\begin{ccppspecific}
\begin{omptInquiry}
typedef int (*ompt_get_target_info_t) (
  uint64_t *\plc{device_num},
  ompt_id_t *\plc{target_id},
  ompt_id_t *\plc{host_op_id}
);
\end{omptInquiry}
\end{ccppspecific}

\descr
The \code{ompt_get_target_info} entry point, which has type signature 
\code{ompt_get_target_info_t}, returns 1 if the current thread is in a 
\code{target} region and 0 otherwise. If the entry point returns 0 then
the values of the variables passed by reference as its arguments are undefined.

If the current thread is in a \code{target} region then \code{ompt_get_target_info}
returns information about the current device, active \code{target} region, and
active host operation, if any.

This runtime entry point is \emph{async signal safe}.

\argdesc
The \plc{device_num} argument returns the device number if the current 
thread is in a \code{target} region.

Th \plc{target_id} argument returns the \code{target} region identifier 
if the current thread is in a \code{target} region.

If the current thread is in the process of initiating an operation on a 
target device (for example, copying data to or from an accelerator or 
launching a kernel) then \plc{host_op_id} returns the identifier for the 
operation; otherwise, \plc{host_op_id} returns \code{ompt_id_none}.

\constraints
Arguments passed to the entry point must be valid references to variables 
of the specified types.

\crossreferences
\begin{itemize}
\item \code{ompt_id_t} type, see \specref{sec:ompt_id_t}.
\end{itemize}



\subsubsubsection{\hcode{ompt_get_num_devices_t}}
\label{sec:ompt_get_num_devices_t}
\label{sec:ompt_get_num_devices}

\summary
The \code{ompt_get_num_devices_t} type is the type signature of the 
\code{ompt_get_num_devices} runtime entry point, which returns the
number of available devices.

\format
\begin{ccppspecific}
\begin{omptInquiry}
typedef int (*ompt_get_num_devices_t) (void);
\end{omptInquiry}
\end{ccppspecific}

\descr
The \code{ompt_get_num_devices} runtime entry point, which has 
type signature \code{ompt_get_num_devices_t}, returns the number 
of devices available to an OpenMP program.

This runtime entry point is \emph{async signal safe}.



\subsubsubsection{\hcode{ompt_get_unique_id_t}}
\label{sec:ompt_get_unique_id_t}
\label{sec:ompt_get_unique_id}

\summary
The \code{ompt_get_unique_id_t} type is the type signature of the 
\code{ompt_get_unique_id} runtime entry point, which returns a unique number.

\format
\begin{ccppspecific}
\begin{omptInquiry}
typedef uint64_t (*ompt_get_unique_id_t) (void);
\end{omptInquiry}
\end{ccppspecific}

\descr
The \code{ompt_get_unique_id} runtime entry point, which has type 
signature \code{ompt_get_unique_id_t}, returns a number that is unique 
for the duration of an OpenMP program. Successive invocations may not 
result in consecutive or even increasing numbers.

This runtime entry point is \emph{async signal safe}.



\subsubsubsection{\hcode{ompt_finalize_tool_t}}
\label{sec:ompt_finalize_tool_t}
\label{sec:ompt_finalize_tool}

\summary
The \code{ompt_finalize_tool_t} type is the type signature of the 
\code{ompt_finalize_tool} runtime entry point, which enables a tool 
to finalize itself.

\format
\begin{ccppspecific}
\begin{omptInquiry}
typedef void (*ompt_finalize_tool_t) (void);
\end{omptInquiry}
\end{ccppspecific}

\descr
A tool may detect that the execution of an OpenMP program is ending 
before the OpenMP implementation does. To facilitate clean termination 
of the tool, the tool may invoke the \code{ompt_finalize_tool} runtime 
entry point, which has type signature \code{ompt_finalize_tool_t}. Upon 
completion of \code{ompt_finalize_tool}, no OMPT callbacks are dispatched.

\effect
The \code{ompt_finalize_tool} routine detaches the tool from the runtime, 
unregisters all callbacks and invalidates all OMPT entry points passed to 
the tool in the \plc{lookup-function}. Upon completion of \code{ompt_finalize_tool},
no further callbacks will be issued on any thread.

Before the callbacks are unregistered, the OpenMP runtime should attempt
to dispatch all outstanding registered callbacks as well as the callbacks 
that would be encountered during shutdown of the runtime, if possible in 
the current execution context.



\subsubsection{Entry Points in the OMPT Device Tracing Interface}
\label{sec:ompt-tracing-entry-points}

The runtime entry points with type signatures of the types that are 
specified in this section enable a tool to trace activities on a device.



\subsubsubsection{\hcode{ompt_get_device_num_procs_t}}
\label{sec:ompt_get_device_num_procs_t}

\summary
The \code{ompt_get_device_num_procs_t} type is the type signature of the 
\code{ompt_get_device_num_procs} runtime entry point, which returns the 
number of processors currently available to the execution environment on 
the specified device.

\format
\begin{ccppspecific}
\begin{omptInquiry}
typedef int (*ompt_get_device_num_procs_t) (
  ompt_device_t *\plc{device}
);
\end{omptInquiry}
\end{ccppspecific}

\descr
The \code{ompt_get_device_num_procs} runtime entry point, which has type 
signature \code{ompt_get_device_num_procs_t}, returns the number of processors 
that are available on the device at the time the routine is called. This value 
may change between the time that it is determined and the time that it is read 
in the calling context due to system actions outside the control of the OpenMP 
implementation.

\argdesc
The \plc{device} argument is a pointer to an opaque object that
represents the target device instance. The pointer to the device
instance object is used by functions in the device tracing interface
to identify the device being addressed.

\crossreferences
\begin{itemize}
\item \code{ompt_device_t}, see \specref{sec:ompt_device_t}.
\end{itemize}



\subsubsubsection{\hcode{ompt_get_device_time_t}}
\label{sec:ompt_get_device_time_t}

\summary
The \code{ompt_get_device_time_t} type is the type signature of the 
\code{ompt_get_device_time} runtime entry point, which returns the 
current time on the specified device.

\format
\begin{ccppspecific}
\begin{omptInquiry}
typedef ompt_device_time_t (*ompt_get_device_time_t) (
  ompt_device_t *\plc{device}
);
\end{omptInquiry}
\end{ccppspecific}

\descr
Host and target devices are typically distinct and run independently.
If host and target devices are different hardware components, they
may use different clock generators. For this reason, a common time 
base for ordering host-side and device-side events may not be available.

The \code{ompt_get_device_time} runtime entry point, which has type 
signature \code{ompt_get_device_time_t}, returns the current time on 
the specified device. A tool can use this information to align time 
stamps from different devices.

\argdesc
The \plc{device} argument is a pointer to an opaque object that
represents the target device instance. The pointer to the device
instance object is used by functions in the device tracing interface
to identify the device being addressed.

\crossreferences
\begin{itemize}
\item \code{ompt_device_t}, see \specref{sec:ompt_device_t}.

\item \code{ompt_device_time_t}, see \specref{sec:ompt_device_time_t}.
\end{itemize}



\subsubsubsection{\hcode{ompt_translate_time_t}}
\label{sec:ompt_translate_time_t}

\summary
The \code{ompt_translate_time_t} type is the type signature of the 
\code{ompt_translate_time} runtime entry point, which translates a 
time value that is obtained from the specified device to a corresponding 
time value on the host device.

\format
\begin{ccppspecific}
\begin{omptInquiry}
typedef double (*ompt_translate_time_t) (
  ompt_device_t *\plc{device},
  ompt_device_time_t \plc{time}
);
\end{omptInquiry}
\end{ccppspecific}

\descr
The \code{ompt_translate_time} runtime entry point, which has type signature 
\code{ompt_translate_time_t}, translates a time value obtained from the specified 
device to a corresponding time value on the host device. The returned value for 
the host time has the same meaning as the value returned from \code{omp_get_wtime}.

\begin{note}
The accuracy of time translations may degrade if they are not performed promptly 
after a device time value is received and if either the host or device vary their 
clock speeds. Prompt translation of device times to host times is recommended.
\end{note}

\argdesc
The \plc{device} argument is a pointer to an opaque object that
represents the target device instance. The pointer to the device
instance object is used by functions in the device tracing interface
to identify the device being addressed.

The \plc{time} argument is a time from the specified device.

\crossreferences
\begin{itemize}
\item \code{ompt_device_t}, see \specref{sec:ompt_device_t}.

\item \code{ompt_device_time_t}, see \specref{sec:ompt_device_time_t}.
\end{itemize}



\subsubsubsection{\hcode{ompt_set_trace_ompt_t}}
\label{sec:ompt_set_trace_ompt_t}

\summary
The \code{ompt_set_trace_ompt_t} type is the type signature of the 
\code{ompt_set_trace_ompt} runtime entry point, which enables or disables 
the recording of trace records for one or more types of OMPT events.

\format
\begin{ccppspecific}
\begin{omptInquiry}
typedef int (*ompt_set_trace_ompt_t) (
  ompt_device_t *\plc{device},
  unsigned int \plc{enable},
  unsigned int \plc{etype}
);
\end{omptInquiry}
\end{ccppspecific}

\argdesc
The \plc{device} argument points to an opaque object that represents
the target device instance. Functions in the device tracing interface
use this pointer to identify the device that is being addressed.

The \plc{enable} argument indicates whether tracing should be enabled 
or disabled for the event or events that the \plc{etype} argument specifies. 
A positive value for \plc{enable} indicates that recording of the one or 
more events that \plc{etype} specifies should be enabled; a value of 0 for 
\plc{enable} indicates that recording of events should be disabled by this invocation.

An \plc{etype} argument value of 0 indicates that traces for all event 
types will be enabled or disabled.  Passing a positive value for
\plc{etype} inidicates that recording should be enabled or disabled
for the event in \code{ompt_callbacks_t} that matches \plc{etype}.


\effect
Table~\ref{table:record_set} shows the possible return codes for 
\code{ompt_set_trace_ompt}. If a single invocation of \code{ompt_set_trace_ompt} 
is used to enable or disable more than one event, the return code is 3 if tracing 
is possible for one or more events but not for others.

\nolinenumbers
\renewcommand{\arraystretch}{1.5}
\tablefirsthead{%
\hline
\textsf{\textbf{Return Code}} & \textsf{\textbf{Meaning}}\\
\hline\\[-3ex]
}
\tablehead{%
\multicolumn{2}{l}{\small\slshape table continued from previous page}\\
\hline
\textsf{\textbf{Return Code}} & \textsf{\textbf{Meaning}}\\
\hline\\[-3ex]
}
\tabletail{%
\hline\\[-4ex]
\multicolumn{2}{l}{\small\slshape table continued on next page}\\
}
\tablelasttail{\hline}
\tablecaption{Return Codes for \code{ompt_set_trace_ompt} and
    \code{ompt_set_trace_native}\label{table:record_set}}
\begin{supertabular}{p{2.0in} p{3.0in}}
0 & error\\
1 & event will never occur\\
2 & event may occur but no tracing is possible\\
3 & event may occur and will be traced when convenient\\
4 & event may occur and will always be traced if event occurs\\
\end{supertabular}

\linenumbers

\crossreferences
\begin{itemize}
\item \code{ompt_callbacks_t},
see \specref{sec:ompt_callbacks_t}.
\item \code{ompt_device_t},
see \specref{sec:ompt_device_t}.
\end{itemize}

\subsubsubsection{\hcode{ompt_set_trace_native_t}}
\label{sec:ompt_set_trace_native_t}

\summary
A runtime entry point for a device known as \code{ompt_set_trace_native}
with type signature \code{ompt_set_trace_native_t}
enables or disables the recording of native trace records for a device.


\format
\begin{ccppspecific}
\begin{omptInquiry}
typedef int (*ompt_set_trace_native_t) (
  ompt_device_t *\plc{device},
  int \plc{enable},
  int \plc{flags}
);
\end{omptInquiry}
\end{ccppspecific}

\descr
This interface is designed for use by a tool with no knowledge about
an attached device. If a tool knows how to program a particular
attached device, it may opt to invoke native control functions
directly using pointers obtained through the \plc{lookup} function
associated with the device and described in the \plc{documentation}
string that is provided to the device initializer callback.

\argdesc
The \plc{device} argument points to an opaque object that represents
the target device instance. Functions in the device tracing interface
use this pointer to identify the device that is being addressed.

The argument \plc{enable} indicates whether recording of events
should be enabled or disabled by this invocation.

The argument \plc{flags} specifies the kinds of native device
monitoring to enable or disable.
Each kind of monitoring is specified by a flag bit.
Flags can be composed by using logical {\ttfamily or}  to combine enumeration
values from type \code{ompt_native_mon_flag_t}.
Table~\ref{table:record_set} shows the possible return codes for
\code{ompt_set_trace_native}.  If a single invocation of
\code{ompt_set_trace_native} is used to enable/disable more
than one kind of monitoring, the return code will be 3 if tracing
is possible for one or more kinds of monitoring but not for others.

To start, pause, flush, or stop tracing for a specific target device
associated with the handle \plc{device}, a tool invokes the 
\code{ompt_start_trace}, \code{ompt_pause_trace}, \code{ompt_flush_trace}, or
\code{ompt_stop_trace} runtime entry point for the device.


\crossreferences
\begin{itemize}
\item \code{ompt_device_t},
see \specref{sec:ompt_device_t}.
\end{itemize}

\subsubsubsection{\hcode{ompt_start_trace_t}}
\label{sec:ompt_start_trace_t}

\summary
A runtime entry point for a device known as \code{ompt_start_trace}
with type signature \code{ompt_start_trace_t}
starts tracing of activity on a specific device.

\format
\begin{ccppspecific}
\begin{omptInquiry}
typedef int (*ompt_start_trace_t) (
  ompt_device_t *\plc{device},
  ompt_callback_buffer_request_t \plc{request},
  ompt_callback_buffer_complete_t \plc{complete}
);
\end{omptInquiry}
\end{ccppspecific}

\descr
A tool may initiate tracing on a device by invoking the device's \code{ompt_start_trace}
runtime entry point.

Under normal operating conditions, every event buffer provided to
a device by a tool callback will be returned to the tool
before the OpenMP runtime shuts down.
If an exceptional condition terminates  execution of an OpenMP
program, the OpenMP runtime may not return buffers provided to the
device.

An invocation of \code{ompt_start_trace} returns 1 if the command
succeeds and 0 otherwise.

\argdesc
The \plc{device} argument points to an opaque object that represents
the target device instance. Functions in the device tracing interface
use this pointer to identify the device that is being addressed.

The argument \emph{buffer request} specifies a tool callback
that will supply a device with a buffer to deposit events.

The argument \emph{buffer complete} specifies a tool callback
that will be invoked by the OpenMP implementation to empty a buffer
containing event records.

\crossreferences
\begin{itemize}
\item \code{ompt_device_t},
see \specref{sec:ompt_device_t}.
\item \code{ompt_callback_buffer_request_t},
see \specref{sec:ompt_callback_buffer_request_t}.
\item \code{ompt_callback_buffer_complete_t},
see \specref{sec:ompt_callback_buffer_complete_t}.
\end{itemize}

\subsubsubsection{\hcode{ompt_pause_trace_t}}
\label{sec:ompt_pause_trace_t}

\summary
A runtime entry point for a device known as \code{ompt_pause_trace}
with type signature \code{ompt_pause_trace_t}
pauses or restarts activity tracing on a specific device.

\begin{ccppspecific}
\begin{omptInquiry}
typedef int (*ompt_pause_trace_t) (
  ompt_device_t *\plc{device},
  int \plc{begin_pause}
);
\end{omptInquiry}
\end{ccppspecific}

\descr

A tool may pause or resume tracing on a device by invoking the device's
\code{ompt_pause_trace} runtime entry point.
An invocation of \code{ompt_pause_trace} returns 1 if the command
succeeds and 0 otherwise.

Redundant pause or resume commands are idempotent and will
return 1 indicating success.

\argdesc
The \plc{device} argument points to an opaque object that represents
the target device instance. Functions in the device tracing interface
use this pointer to identify the device that is being addressed.

The argument \plc{begin_pause} indicates whether to pause or
resume tracing.
To resume tracing, zero should be supplied for \plc{begin_pause}.

\crossreferences
\begin{itemize}
\item \code{ompt_device_t},
see \specref{sec:ompt_device_t}.
\end{itemize}

\subsubsubsection{\hcode{ompt_flush_trace_t}}
\label{sec:ompt_flush_trace_t}

\summary
A runtime entry point for a device known as \code{ompt_flush_trace}
with type signature \code{ompt_flush_trace_t} causes all pending trace 
records for the specified device to be delivered to the tool.

\begin{ccppspecific}
\begin{omptInquiry}
typedef int (*ompt_flush_trace_t) (
  ompt_device_t *\plc{device}
);
\end{omptInquiry}
\end{ccppspecific}

\descr

A tool may request that a device flush any pending trace records
by invoking the \code{ompt_flush_trace} runtime entry point for the device.
Invoking \code{ompt_flush_trace} causes the OpenMP implementation to issue a sequence 
of zero or more buffer completion callbacks to deliver to the tool all trace records 
that have been collected prior to the flush.
An invocation of \code{ompt_flush_trace} returns 1 if the command
succeeds and 0 otherwise.

\argdesc
The \plc{device} argument points to an opaque object that represents
the target device instance. Functions in the device tracing interface
use this pointer to identify the device that is being addressed.

\crossreferences
\begin{itemize}
\item \code{ompt_device_t},
see \specref{sec:ompt_device_t}.
\end{itemize}

\subsubsubsection{\hcode{ompt_stop_trace_t}}
\label{sec:ompt_stop_trace_t}

\summary
A runtime entry point for a device known as \code{ompt_stop_trace}
with type signature \code{ompt_stop_trace_t}
stops tracing for a device.

\begin{ccppspecific}
\begin{omptInquiry}
typedef int (*ompt_stop_trace_t) (
  ompt_device_t *\plc{device}
);
\end{omptInquiry}
\end{ccppspecific}

\descr

A tool may halt tracing on a device and request that the device flush any pending trace records
by invoking the \code{ompt_stop_trace} runtime entry point for the device.
An invocation of \code{ompt_stop_trace} returns 1 if the command
succeeds and 0 otherwise.

\argdesc
The \plc{device} argument points to an opaque object that represents
the target device instance. Functions in the device tracing interface
use this pointer to identify the device that is being addressed.


%%? johnmc says: we should export one more function from a target device: ompt_target_scope(begin/end, host_id)
%% this function will be used by the OpenMP runtime when a target device is being monitored to signal the target device when
%% (1) entering and leaving a target region
%% (2) before and after launching a kernel
%% (3) before and after performing a data operation: copy, allocation, release, ...
%% this function will enable the target device to associate device_activity_ids with some host_id that either represents a
%% target region, target data operation, or target kernel submission

\crossreferences
\begin{itemize}
\item \code{ompt_device_t},
see \specref{sec:ompt_device_t}.
\end{itemize}

\subsubsubsection{\hcode{ompt_advance_buffer_cursor_t}}
\label{sec:ompt_advance_buffer_cursor_t}

%There are several routines that need to be used together to process %target event records deposited in a buffer by a device.

\summary
A runtime entry point for a device known as \code{ompt_advance_buffer_cursor}
with type signature \code{ompt_advance_buffer_cursor_t}
advances a trace buffer cursor to the next record.

\format
\begin{ccppspecific}
\begin{omptInquiry}
typedef int (*ompt_advance_buffer_cursor_t) (
  ompt_buffer_t *\plc{buffer},
  size_t \plc{size},
  ompt_buffer_cursor_t \plc{current},
  ompt_buffer_cursor_t *\plc{next}
);
\end{omptInquiry}
\end{ccppspecific}

\descr
It returns \plc{true} if the advance is successful and the next
position in the buffer is valid.

\argdesc
The \plc{device} argument points to an opaque object that represents
the target device instance. Functions in the device tracing interface
use this pointer to identify the device that is being addressed.

The argument \plc{buffer} indicates a trace buffer associated
with the cursors.

The argument \plc{size} indicates the size of \plc{buffer} in
bytes.

The argument \plc{current} is an opaque buffer cursor.

The argument \plc{next} returns the next value of a opaque buffer cursor.


\crossreferences
\begin{itemize}
\item \code{ompt_device_t},
see \specref{sec:ompt_device_t}.
\item \code{ompt_buffer_cursor_t},
see \specref{sec:ompt_buffer_cursor_t}.
\end{itemize}

\subsubsubsection{\hcode{ompt_get_record_type_t}}
\label{sec:ompt_get_record_type_t}

\summary
A runtime entry point for a device known as
\code{ompt_get_record_type} with type signature
\code{ompt_get_record_type_t} inspects the type
of a trace record for a device.

\format
\begin{ccppspecific}
\begin{omptInquiry}
typedef ompt_record_t (*ompt_get_record_type_t) (
  ompt_buffer_t *\plc{buffer},
  ompt_buffer_cursor_t \plc{current}
);
\end{omptInquiry}
\end{ccppspecific}

\descr

Trace records for a device may be in one of two forms: a
\emph{native} record format, which may be device-specific,
or an \emph{OMPT} record format, where each trace record
corresponds to an OpenMP \emph{event} and fields in the record
structure are mostly the arguments that would be passed to the
OMPT callback for the event.

A runtime entry point for a device known as
\code{ompt_get_record_type} with type signature
\code{ompt_get_record_type_t} inspects the type
of a trace record and indicates whether the record at the current
position in the provided trace buffer is an OMPT record,
a native record, or an invalid record. An invalid record type
is returned if the cursor is out of bounds.

\argdesc
The argument \plc{buffer} indicates a trace buffer.

The argument \plc{current} is an opaque buffer cursor.




\crossreferences
\begin{itemize}
\item \code{ompt_record_t},
see \specref{sec:ompt_record_t}.
\item \code{ompt_buffer_t},
see \specref{sec:ompt_buffer_t}.
\item \code{ompt_buffer_cursor_t},
see \specref{sec:ompt_buffer_cursor_t}.
\end{itemize}

\subsubsubsection{\hcode{ompt_get_record_ompt_t}}
\label{sec:ompt_get_record_ompt_t}

\summary
A runtime entry point for a device known as \code{ompt_get_record_ompt}
with type signature \code{ompt_get_record_ompt_t}
obtains a pointer to an OMPT trace record from a trace buffer associated with a device.

\format
\begin{ccppspecific}
\begin{omptInquiry}
typedef ompt_record_ompt_t *(*ompt_get_record_ompt_t) (
  ompt_buffer_t *\plc{buffer},
  ompt_buffer_cursor_t \plc{current}
);
\end{omptInquiry}
\end{ccppspecific}

\descr

This function returns a pointer that may point to a record in the
trace buffer, or it may point to a record in thread local storage
where the information extracted from a record was assembled. The
information available for an event depends upon its type.

The return value of type \code{ompt_record_ompt_t}
includes a field of a union type that can represent
information for any OMPT event record type.
Another call to the runtime entry point may overwrite the
contents of the fields in a record returned by a prior invocation.

\argdesc
The argument \plc{buffer} indicates a trace buffer.

The argument \plc{current} is an opaque buffer cursor.

\crossreferences
\begin{itemize}
\item \code{ompt_record_ompt_t},
see \specref{sec:ompt_record_ompt_t}.
\item \code{ompt_device_t},
see \specref{sec:ompt_device_t}.
\item \code{ompt_buffer_cursor_t},
see \specref{sec:ompt_buffer_cursor_t}.
\end{itemize}

\subsubsubsection{\hcode{ompt_get_record_native_t}}
\label{sec:ompt_get_record_native_t}

\summary

A runtime entry point for a device known as
\code{ompt_get_record_native} with type signature
\code{ompt_get_record_native_t}
obtains a pointer to a native trace record from a trace buffer associated with a device.

\format
\begin{ccppspecific}
\begin{omptInquiry}
typedef void *(*ompt_get_record_native_t) (
  ompt_buffer_t *\plc{buffer},
  ompt_buffer_cursor_t \plc{current},
  ompt_id_t *\plc{host_op_id}
);
\end{omptInquiry}
\end{ccppspecific}

\descr

The pointer returned  may point into the specified trace buffer, or into
thread local storage where the information extracted from a trace
record was assembled. The information available for a native event
depends upon its type. If the function returns a non-null result,
it will also set \code{*host_op_id} to identify host-side identifier
for the operation associated with the record.  A subsequent call
to \code{ompt_get_record_native} may overwrite the
contents of the fields in a record returned by a prior invocation.

\argdesc
The argument \plc{buffer} indicates a trace buffer.

The argument \plc{current} is an opaque buffer cursor.

The argument \plc{host_op_id} is a pointer to an identifier
that will be returned by the function. The entry point will set
*\plc{host_op_id} to the value of a host-side identifier for an operation on
a target device that was created when the operation was initiated by
the host.


\crossreferences
\begin{itemize}
\item \code{ompt_id_t},
see \specref{sec:ompt_id_t}.
\item \code{ompt_buffer_t},
see \specref{sec:ompt_buffer_t}.
\item \code{ompt_buffer_cursor_t},
see \specref{sec:ompt_buffer_cursor_t}.
\end{itemize}

\subsubsubsection{\hcode{ompt_get_record_abstract_t}}
\label{sec:ompt_get_record_abstract_t}

\summary
A runtime entry point for a device known as
\code{ompt_get_record_abstract}
with type signature
\code{ompt_get_record_abstract_t}
summarizes the context of a native (device-specific) trace record.

\format
\begin{ccppspecific}
\begin{omptOther}
typedef ompt_record_abstract_t *
(*ompt_get_record_abstract_t) (
  void *\plc{native_record}
);
\end{omptOther}
\end{ccppspecific}

\descr
An OpenMP implementation may execute on a device that logs
trace records in a native (device-specific) format unknown to a tool.
A tool can use the \code{ompt_get_record_abstract}
runtime entry point for the device with type signature
\code{ompt_get_record_abstract_t}
to decode a native trace record that it
does not understand into a standard form that it can interpret.

\argdesc

The argument \plc{native_record} is a pointer to a native trace
record.

\crossreferences
\begin{itemize}
\item \code{ompt_record_abstract_t},
see \specref{sec:ompt_record_abstract_t}.
\end{itemize}

\subsubsection{Lookup Entry Point}

\subsubsubsection{\hcode{ompt_function_lookup_t}}
\label{sec:ompt_function_lookup_t}
\label{sec:ompt_function_lookup}

\summary
A tool uses a lookup routine with type signature
\code{ompt_function_lookup_t}
to obtain pointers to runtime entry points that are
part of the OMPT interface.

\format

\begin{ccppspecific}
\begin{omptInquiry}
typedef void (*ompt_interface_fn_t) (void);

typedef ompt_interface_fn_t (*ompt_function_lookup_t) (
  const char *\plc{interface_function_name}
);
\end{omptInquiry}
\end{ccppspecific}


\descr

An OpenMP implementation provides a pointer to a lookup routine as an
argument to tool callbacks used to initialize tool support for
monitoring an OpenMP device using either tracing or callbacks.



When an OpenMP implementation invokes a tool initializer to configure
the OMPT callback interface, the OpenMP implementation will pass the
initializer a lookup function that the tool can use to obtain
pointers to runtime entry points that implement routines that are part of
the OMPT callback interface.

When an OpenMP implementation invokes a tool initializer to configure
the OMPT tracing interface for a device, the Open implementation will
pass the device tracing initializer a lookup function that the tool
can use to obtain pointers to runtime entry points that implement
tracing control routines appropriate for that device.

A tool can call the lookup function to obtain a pointer to a runtime
entry point.

\argdesc
The argument \plc{interface_function_name} is a C string
that represents the name of a runtime entry point.

\crossreferences
\begin{itemize}
\item Tool initializer for a device's OMPT tracing interface, 
\specref{sec:tracing-device-activity}.
\item Tool initializer for the OMPT callback interface, \specref{sec:ompt_initialize_t}
\item Entry points in the OMPT callback interface, see
  \tabref{table:ompt-callback-interface-functions} for a list and
  \specref{sec:ompt-callback-entry-points} for detailed definitions.
\item Entry points in the OMPT tracing interface, see
  \tabref{table:ompt-tracing-interface-functions} for a list and
  \specref{sec:ompt-tracing-entry-points} for detailed definitions.
\end{itemize}
% This is the end of ch4-toolsSupport.tex


% OMPD
%%%% \ompdsection{Overview on OMPD Tool Interface}
%%%% \label{sec:ompd-overview}
\label{ompd:third-party-tool-callback-interface}

\section{OMPD}
\label{sec:ompd-overview}
\label{sec:third-party-tool-callback-interface}

OMPD allows \emph{third-party tools} such as a debuggers to inspect the 
OpenMP state of a live program or core file in an implementation-agnostic 
manner. That is, a tool that uses OMPD should work with any conforming 
OpenMP implementation. An OpenMP implementor provides a library for OMPD 
that a third-party tool can dynamically load. Using the interface exported 
by the OMPD library, the external tool can inspect the OpenMP state of a 
program. In order to satisfy requests from the third-party tool, the OMPD 
library may need to read data from, or to find the addresses of symbols in 
the OpenMP program. The OMPD library provides this functionality through a 
callback interface that the third-party tool must instantiate for the OMPD library.

To use OMPD, the third-party tool loads the OMPD library. The OMPD library exports 
the API that is defined throughout this section and that the tool uses to 
determine OpenMP information about the OpenMP program. The OMPD library must
look up the symbols and read data out of the program. It does not perform
these operations directly, but instead it use the callback interface that the
tool exports to cause the tool to perform them.

This architectural layout insulates the tool from the details
of the internal structure of the OpenMP runtime.
Similarly, the OMPD library does not need to be concerned about
how to access the OpenMP program.
Decoupling the library and tool in this way allows for
great flexibility in how the OpenMP program and tool are deployed,
so that, for example, there is no requirement that tool
and OpenMP program execute on the same machine.

Generally the tool does not interact directly with the OpenMP
runtime in the OpenMP program, and instead uses the OMPD library
for this purpose.
However, there are a few instances where the tool does need
to access the OpenMP runtime directly.
These cases fall into two broad categories.
The first is during initialization, where the tool needs
to be able to look up symbols and read variables in OpenMP runtime
in order to identify the OMPD library it should use.
This is discussed in Sections~\ref{subsubsec:ompd_dll_locations}
and~\ref{subsubsec:ompd_dll_locations_valid}.
The second category relates to arranging for the tool to be notified
when certain events occur during the execution of the OpenMP program.
The model used for this purpose is that the OpenMP implementation
is required to define certain symbols in the runtime code.
This is discussed in Section~\ref{subsec:runtime-entry-points-for-ompd}.
Each of these symbols corresponds to an event type.
The runtime must ensure that control passes through the appropriate
named location when events occur.
If the tool wants to get notification of an event, it can plant
a breakpoint at the matching location.

The code locations can, but do not need to, be functions.
They can, for example, simply be labels.
However, the names must have external \texttt{C} linkage.



% %%%%%%%%%%%%%%%%%%%%%%%%%%%%%%%%%%%%%%%%%%%%%%%%%%%%%%%%%%%%%%%%%%%%%%%%%%%

\subsection{Activating an OMPD Tool}
\label{sec:ompd:activating}

The tool and the OpenMP program the tool controls
exist as separate processes.; thus coordination is required between the OpenMP runtime
%resident in the OpenMP program 
and the external tool for OMPD to be used successfully.

\subsubsection{Enabling the Runtime for OMPD}
\label{sec:ompd:enabling-ompd}

In order to support external tools, the OpenMP runtime may need to collect
and maintain information that it might otherwise not do, perhaps
for performance reasons, or because it is not otherwise needed.
The OpenMP runtime collects whatever information is necessary
to support OMPD if at least one of the following conditions is true:
\begin{enumerate}
\item
  the environment variable \code{OMP\_DEBUG} is set to \code{enabled}
\item
  the OpenMP program defines the function
  \code{ompd\_debug\_enable}
  (See \specref{sec:ompd:omp_debug_enable})
  and it returns logical true (i.e., non-zero) when
  the OpenMP runtime calls this function.
  Immediately before the OpenMP implementation initializes itself
  it will check to see whether the \code{ompd\_debug\_enable} function
  is defined in the OpenMP program or one of its dynamically-linked libraries.
  If a definition is found, the OpenMP will call the function it finds to determine what it returns.
%  The OpenMP runtime will enable data collection for the external tool
%  if \code{omp\_debug\_enable} returns logical true (i.e., non-zero).
\end{enumerate}
%These methods are equivalent, and either can be used.
In some cases it may not be possible to control an OpenMP program's
environment.
\code{omp\_debug\_enable} allows an OpenMP program itself to turn on
data collection for OMPD.
The function can be positioned in an otherwise empty DLL the
programmer can link with the OpenMP program.
This leaves the OpenMP program code unmodified.
A tool implementor may choose to distribute a DLL defining
\code{omp\_debug\_enable} as a convenience to programmers.

\crossreferences
\begin{itemize}
\item
  \code{OMP\_DEBUG}, \specref{sec:OMP_DEBUG}
\item
  \code{omp\_debug\_enable}, \specref{sec:ompd:omp_debug_enable}
\item
  Activating an OMPT Tool, \specref{sec:ompt-initialization}
\end{itemize}

\subsubsection{Finding the OMPD plugin}
\label{sec:ompd:finding-the-ompd}

An OpenMP runtime may have more than one matching OMPD plugins for 
tools to use.
The tool must be able to locate the right plugin to use
for the OpenMP program it is examining.

As part of the OpenMP interface, OMPD requires that the OpenMP
runtime system provides a public variable \code{ompd\_dll\_locations},
which is an \code{argv}-style vector of filename string pointers that
provides the pathnames(s) of any compatible OMPD plugin implementations.
\code{ompd\_dll\_locations} must have \code{C} linkage.
The tool uses the name of the variable verbatim,
and in particular, will not apply any name mangling before
performing the look up.
The pathnames may be relative or absolute.

\code{ompd\_dll\_locations} points to a NULL-terminated
vector of zero or more NULL-terminated pathname strings.
There are no filename conventions for pathname strings.
The last entry in the vector is NULL.
The vector of string pointers must be fully initialized \emph{before}
\code{ompd\_dll\_locations} is set to a non-NULL value,
such that if the tool, such as a debugger,
stops execution of the OpenMP program at any point where
\code{ompd\_dll\_locations} is non-NULL,
then the vector of strings it points to is valid and complete.

The programming model or architecture of the tool (and hence
that of the required OMPD) does not have to match that of the OpenMP program
being examined.
It is the responsibility of the tool to interpret the contents
of \code{ompd\_dll\_locations} to find a suitable OMPD that matches
its own architectural characteristics.
On platforms that support different programming models
(\textit{e.g.}, 32-bit vs 64-bit), OpenMP implementers are encouraged
to provide OMPD library for all models, and which can handle
OpenMP programs of any model.
Thus, for example, a 32-bit debugger using OMPD should be able
to debug a 64-bit OpenMP program
by loading a 32-bit OMPD that can manage a 64-bit OpenMP runtime.

\crossreferences
\begin{itemize}
	\item Identifying the Matching OMPD, \specref{sec:ompd:ompd_dll_locations}
	%\item \code{omp\_debug\_enable}, \specref{sec:ompd:omp_debug_enable}
	%\item Activating an OMPT Tool, \specref{sec:ompt-initialization}
\end{itemize}

\begin{comment}
Depending on how the OpenMP runtime is implemented, \code{ompd\_dll\_locations}
might not be a static variable, and therefore needs to be initialized
at runtime.
The OpenMP runtime notifies tools that \code{ompd\_dll\_locations}
is valid by allowing execution to pass through a location identified
by the symbol \code{ompd\_dll\_locations\_valid}.
This symbol has external, \code{C}, linkage.
Conceptually, \code{ompd\_dll\_locations\_valid} has the signature
\code{void ompd\_dll\_locations\_valid~(~void~);}.
If \code{ompd\_dll\_locations} is NULL, a tool like a debugger
can place a breakpoint at \code{ompd\_dll\_locations\_valid} to be notified
when the vector of name strings has been setup, and is valid.

\code{ompd\_dll\_locations\_valid} does not need not be a function,
and can instead be a labeled location in the code through which
execution passes once the vector is valid.

\crossreferences
\begin{itemize}
\item
  \code{ompd\_dll\_locations}, \specref{sec:ompd:ompd_dll_locations}
\item
  \code{ompd\_dll\_locations\_valid}, \specref{sec:ompd:ompd_dll_locations_valid}
\end{itemize}
\end{comment}




%%%% \ompdsection{Activating a OMPD Tool}
%%%% \label{sec:ompd-initialization}

% 
%%%%%%%%%%%%%%%%%%%%%%%%%%%%%%%%%%%%%%%%%%%%%%%%%%%%%%%%%%%%%%%%%%%%%%%%%%%%%%%%%%%%%%%%%%%%%%%%%%%%%%%%%%%%%%%%%%%%%%%%%%%%%%%%%%%%%%%%%%%%%%%%%%%%%%%

\subsection{OMPD Data Types}
\label{ompd:ompd-data-types}

In this section, we define the types, structures, and functions for the OMPD API.

\ompdsubsection{Basic Types}
%\label{ompd:ompd_addr_t}
%\label{ompd:ompd_word_t}
%\label{ompd:ompd_wait_id_t}
%\label{ompd:ompd_address_t}
The following describes the basic OMPD API types.

\subsubsubsection{Size type}
\label{ompd:ompd_size_t}

This type is used to specify the number of bytes in opaque data objects passed across the OMPD API.

\format
\vbox{
	\ccppspecificstart
	\begin{boxedcode}
  typedef uint64_t ompd_size_t;  
	\end{boxedcode}
	\ccppspecificend
}

\subsubsubsection{Wait id type}
\label{ompd:ompd_wait_id_t}

This type identifies what a thread is waiting for.

\format
\vbox{
	\ccppspecificstart
	\begin{boxedcode}
  typedef uint64_t ompd_wait_id_t;  
	\end{boxedcode}
	\ccppspecificend
}

\subsubsubsection{Basic value types}
\label{ompd:ompd_addr_t}
\label{ompd:ompd_word_t}
\label{ompd:ompd_seg_t}

These definitions represent a word, address, and segment value types.

\format
\vbox{
\ccppspecificstart
\begin{boxedcode}
  typedef uint64_t ompd_addr_t;
  typedef int64_t  ompd_word_t;
  typedef uint64_t ompd_seg_t;
\end{boxedcode}
\ccppspecificend
}

The \plc{ompd\_addr\_t} type represents an unsigned integer large enough to hold an address in an 
OpenMP process.
The \plc{ompd\_word\_t} type represents a signed version of  \plc{ompd\_addr\_t} to hold a signed 
integer of the OpenMP process.
The \plc{ompd\_seg\_t} type represents an unsigned integer large enough to hold a segment value.

\subsubsubsection{Address type}
\label{ompd:ompd_address_t}

This type is a structure that OMPD uses to specify device addresses, 
which may or may not be segmented.

\format
\vbox{
\ccppspecificstart
\begin{boxedcode}
typedef struct \{
 ompd_seg_t \plc{segment};
 ompd_addr_t \plc{address};
\} ompd_address_t;

#define OMPD_SEGMENT_UNSPECIFIED 0
\end{boxedcode}
\ccppspecificend
}

For non segmented architectures, use OMPD\_SEGMENT\_UNSPECIFIED in the \plc{segment} 
field of \code{ompd\_address\_t}.

%\format
%\vbox{
%\ccppspecificstart
%\begin{boxedcode}
%/* unsigned integer large enough to hold a target device */
%/* address or a target device segment value */
%typedef uint64_t ompd_addr_t;
%/* signed version of ompd_addr_t */
%typedef int64_t  ompd_word_t;
%/* identifies what a thread is waiting for */
%typedef uint64_t ompd_wait_id_t;
%typedef struct \{
%  /* target device specific segment value */
%  ompd_addr_t segment;
%  /* target device address in the segment */ 
%  ompd_addr_t address;
%\} ompd_address_t;
%\end{boxedcode}
%\ccppspecificend
%}

%The \code{ompd\_address\_t} is a structure that OMPD uses to specify target device addresses, 
%which 
%may or may not be segmented.  
  
\ompdsubsection{System Device Identifiers}

Different OpenMP runtimes may utilize different underlying devices.
The type used to hold an device identifier can vary in size and format, and 
therefore is not explicitly represented in the OMPD API. Device identifiers are 
passed across the interface using a device-identifier `kind', a pointer to where
the device identifier is stored, and the size of the device identifier in bytes.
The OMPD library and tool using it must agree on the format
of what is being passed.
Each different kind of device identifier uses a unique
unsigned 64-bit integer value.

Recommended values of \code{ompd\_device\_kind\_t} are defined in the \code{ompd\_types.h} 
header file, which is available on \url{http://www.openmp.org/}. 

\label{ompd:ompd_device_kind_t}
\format
\vbox{
	\ccppspecificstart
	\begin{boxedcode}
  typedef uint64_t ompd_device_kind_t;
	\end{boxedcode}
	\ccppspecificend
}

\ompdsubsection{Thread Identifiers}

Different OpenMP runtimes may use different underlying native
thread implementations.
The type used to hold a thread identifier can vary in size and format, and 
therefore is not explicitly represented in the OMPD API. Thread identifiers are 
passed across the interface using a thread-identifier `kind', a pointer to where
the thread identifier is stored, and the size of the thread identifier in bytes.
The OMPD library and tool using it must agree on the format
of what is being passed.
Each different kind of thread identifier uses a unique
unsigned 64-bit integer value.

Recommended values of \code{ompd\_thread\_id\_kind\_t} are defined in the \code{ompd\_types.h} 
header file, which is available on \url{http://www.openmp.org/}. 

\label{ompd:ompd_thread_id_kind_t}
\format
\vbox{
\ccppspecificstart
\begin{boxedcode}
typedef uint64_t ompd_thread_id_kind_t;
\end{boxedcode}
\ccppspecificend
}

\ompdsubsection{OMPD Handle Types}
\label{ompd:ompd_address_space_handle_t}
\label{ompd:ompd_thread_handle_t}
\label{ompd:ompd_parallel_handle_t}
\label{ompd:ompd_task_handle_t}

Each operation of the OMPD interface that applies to a particular address space, thread, parallel 
region, or task must explicitly specify
% ilaguna: device can be confused by a GPU device
%the device for the operation using a \emph{handle}.
a \emph{handle} for the operation.
OMPD employs handles for address spaces (for a host or target device), threads, parallel regions, 
and tasks. A handle for an entity is constant while the entity itself is alive. Handles are defined by 
the OMPD plugin, and are opaque to the tool. The following defines the OMPD 
handle types:

\format
\vbox{
\ccppspecificstart
\begin{boxedcode}
typedef struct _ompd_aspace_handle_s ompd_address_space_handle_t;
typedef struct _ompd_thread_handle_s ompd_thread_handle_t;
typedef struct _ompd_parallel_handle_s ompd_parallel_handle_t;
typedef struct _ompd_task_handle_s ompd_task_handle_t;
\end{boxedcode}
\ccppspecificend
}

Defining externally visible type names in this way introduces type safety to the interface, and helps 
to catch instances where incorrect handles are passed by the tool to the OMPD 
library. The \code{struct}s do not need to be defined at all. The OMPD library 
must cast incoming (pointers to) handles to the appropriate internal, private types.

\ompdsubsection{Tool Context Types}

A third-party tool uses contexts to uniquely  identifies abstractions. These contexts are opaque to 
the OMPD library and are defined as follows:
%The debugger uses The debugger contexts are opaque to the OMPD and are defined as follows:

\format
\vbox{
\ccppspecificstart
\begin{boxedcode}
typedef struct _ompd_aspace_cont_s ompd_address_space_context_t;
typedef struct _ompd_thread_cont_s ompd_thread_context_t;
\end{boxedcode}
\ccppspecificend
}

\ompdsubsection{Return Code Types}
\label{ompd:ompd_rc_t}

Each OMPD operation has a return code. The return code types and their semantics are defined as 
follows:

\format
\vbox{
\ccppspecificstart
\begin{boxedcode}
typedef enum \{
  ompd_rc_ok = 0,
  ompd_rc_unavailable = 1, 
  ompd_rc_stale_handle = 2,
  ompd_rc_bad_input = 3,
  ompd_rc_error = 4,
  ompd_rc_unsupported = 5,
  ompd_rc_needs_state_tracking = 6,
  ompd_rc_incompatible = 7,
  ompd_rc_device_read_error = 8,
  ompd_rc_device_write_error = 9,
  ompd_rc_nomem = 10, 
\} ompd_rc_t;	
\end{boxedcode}
\ccppspecificend
}

\descr
\label{ompd:ompd_rc_ok}
\textbf{ompd\_rc\_ok} is returned when the operation is successful.

\label{ompd:ompd_rc_unavailable}
\textbf{ompd\_rc\_unavailable} is returned when 
information is not available for the specified context.

\label{ompd:ompd_rc_stale_handle}
\textbf{ompd\_rc\_stale\_handle} is returned when
the specified handle is no longer valid.

\label{ompd:ompd_rc_bad_input}
\textbf{ompd\_rc\_bad\_input} is returned when
the input parameters (other than handle) are invalid. 

\label{ompd:ompd_rc_error}
\textbf{ompd\_rc\_error} is returned when
a fatal error occurred.

\label{ompd:ompd_rc_unsupported}
\textbf{ompd\_rc\_unsupported} is returned when
the requested operation is not supported.

\label{ompd:ompd_rc_needs_state_tracking}
\textbf{ompd\_rc\_needs\_state\_tracking} is returned when
the state tracking operation failed because state tracking is not currently enabled.

\label{ompd:ompd_rc_incompatible}
\textbf{ompd\_rc\_incompatible} is returned when
this OMPD is incompatible with the OpenMP program.

\label{ompd:ompd_rc_device_read_error}
\textbf{ompd\_rc\_device\_read\_error} is returned when
a read operation failed on the device

\label{ompd:ompd_rc_device_write_error}
\textbf{ompd\_rc\_device\_write\_error} is returned when
a write operation failed to the device.

\label{ompd:ompd_rc_nomem}
\textbf{ompd\_rc\_nomem} is returned when
unable to allocate memory.

\ompdsubsection{OpenMP Scheduling Types}
\label{ompd:ompd_sched_t}

The following enumeration defines \code{ompd\_sched\_t}, which is the OMPD API definition 
corresponding to the OpenMP enumeration type \code{omp\_sched\_t} (see 

\specref{subsec:omp_set_schedule}).
\code{ompd\_sched\_t} also defines \code{ompd\_sched\_vendor\_lo} and
\code{ompd\_sched\_vendor\_hi} to define the range of implementation-specific 
\code{omp\_sched\_t} values than can be handle by the OMPD API.

\begin{quote}
	\begin{lstlisting}

	\end{lstlisting}
\end{quote}

\format
\vbox{
\ccppspecificstart
\begin{boxedcode}
typedef enum \{
  ompd_sched_static = 1,
  ompd_sched_dynamic = 2,
  ompd_sched_guided = 3,
  ompd_sched_auto = 4,
  ompd_sched_vendor_lo = 5,
  ompd_sched_vendor_hi = 0x7fffffff
\} ompd_sched_t;
\end{boxedcode}
\ccppspecificend
}

\ompdsubsection{OpenMP Proc Binding Types}
\label{ompd:ompd_proc_bind_t}

The following enumeration defines \code{ompd\_proc\_bind\_t}, which is the OMPD
API definition corresponding to the OpenMP enumeration type
\code{omp\_proc\_bind\_t} (\specref{subsec:omp_get_proc_bind}).

\format
\vbox{
\ccppspecificstart
\begin{boxedcode}
typedef enum \{
  ompd_proc_bind_false = 0,
  ompd_proc_bind_true = 1,
  ompd_proc_bind_master = 2,
  ompd_proc_bind_close = 3,
  ompd_proc_bind_spread = 4
\} ompd_proc_bind_t;
\end{boxedcode}
\ccppspecificend
}

\ompdsubsection{Primitive Types}
\label{ompd:ompd_device_type_sizes_t}
The following structure contains members that the OMPD library can use
to interrogate the tool about the ``sizeof'' of primitive types in the OpenMP architecture 
address space.

\format
\vbox{
\ccppspecificstart
\begin{boxedcode}
typedef struct \{
  int \plc{sizeof_char};
  int \plc{sizeof_short};
  int \plc{sizeof_int};
  int \plc{sizeof_long};
  int \plc{sizeof_long_long};
  int \plc{sizeof_pointer};
\} ompd_device_type_sizes_t;
\end{boxedcode}
\ccppspecificend
}

\descr
The fields of \code{ompd\_device\_type\_sizes\_t} give the sizes of
the eponymous basic types used by the OpenMP runtime.
As the tool and the OMPD plugin, by definition, have the same
architecture and programming model, the size of the fields can be given
as \code{int}.

\crossreferences
\begin{itemize}
	\item
	\code{ompd\_sizeof\_fn\_t}, \specref{ompd:ompd_sizeof_fn_t}
\end{itemize}

The following enumeration of primitive types are used by OMPD to express the primitive
type of data for OpenMP device to OMPD host conversion.

\format
\vbox{
\ccppspecificstart
\begin{boxedcode}
typedef enum \{
  ompd_type_char = 0,
  ompd_type_short = 1,
  ompd_type_int = 2,
  ompd_type_long = 3,
  ompd_type_long_long = 4,
  ompd_type_pointer = 5
\} ompd_device_prim_types_t;
\end{boxedcode}
\ccppspecificend
}

\ompdsubsection{Runtime State Types}

The OMPD runtime states mirror those in OMPT (\specref{sec:ompt_get_state_t}). Note that there is no guarantee that 
the numeric values of the corresponding members of the enumerations are identical.

\format
\vbox{
\ccppspecificstart
\begin{boxedcode}
typedef enum \{
  ompd_state_work_serial = 0x00,
  ompd_state_work_parallel = 0x01,
  ompd_state_work_reduction = 0x02,
  ompd_state_idle = 0x10,
  ompd_state_overhead = 0x20,
  ompd_state_wait_barrier = 0x40,
  ompd_state_wait_barrier_implicit = 0x41,
  ompd_state_wait_barrier_explicit = 0x42,
  ompd_state_wait_taskwait = 0x50,
  ompd_state_wait_taskgroup = 0x51,
  ompd_state_wait_mutex = 0x60,
  ompd_state_wait_lock = 0x61,  
  ompd_state_wait_critical = 0x62,
  ompd_state_wait_atomic = 0x63,
  ompd_state_wait_ordered = 0x64,
  ompd_state_undefined = 0x70,
  ompd_state_first = 0x71,
\} ompd_state_t;
\end{boxedcode}
\ccppspecificend
}

\descr
\label{ompd_state_work_serial}
\textbf{ompd\_state\_work\_serial} - 
working outside parallel
 
\label{ompd_state_work_parallel}
\textbf{ompd\_state\_work\_parallel} - 
working within parallel

\label{ompd_state_work_reduction}
\textbf{ompd\_state\_work\_reduction} - 
performing a reduction

\label{ompd_state_idle}
\textbf{ompd\_state\_idle} - 
waiting for work
 
\label{ompd_state_overhead}
\textbf{ompd\_state\_overhead} - 
non-wait overhead

\label{ompd_state_wait_barrier}
\textbf{ompd\_state\_wait\_barrier} - 
generic barrier

\label{ompd_state_wait_barrier_implicit}
\textbf{ompd\_state\_wait\_barrier\_implicit} - 
implicit barrier

\label{ompd_state_wait_barrier_explicit}
\textbf{ompd\_state\_wait\_barrier\_explicit} - 
explicit barrier

\label{ompd_state_wait_taskwait}
\textbf{ompd\_state\_wait\_taskwait} - 
waiting at a taskwait

\label{ompd_state_wait_taskgroup}
\textbf{ompd\_state\_wait\_taskgroup} - 
waiting at a taskgroup

\label{ompd_state_wait_mutex}
\textbf{ompd\_state\_wait\_mutex} - 
waiting for any mutex kind

\label{ompd_state_wait_lock}
\textbf{ompd\_state\_wait\_lock} - 
waiting for lock

\label{ompd_state_wait_critical}
\textbf{ompd\_state\_wait\_critical} - 
waiting for critical

\label{ompd_state_wait_atomic}
\textbf{ompd\_state\_wait\_atomic} - 
waiting for atomic

\label{ompd_state_wait_ordered}
\textbf{ompd\_state\_wait\_ordered} - 
waiting for ordered

\label{ompd_state_undefined}
\textbf{ompd\_state\_undefined} - 
undefined thread state

\label{ompd_state_first}
\textbf{ompd\_state\_first} - 
initial enumeration state


\section{Third-party Tool Callback Interface}
\label{sec:ompd:third-party-tool-callback-interface}

For the OMPD plugin to provide information about the internal state
of the OpenMP runtime system in an OpenMP process or core file,
it must have a means to extract information from
the tool's target.
The tool's target ``process'' may be a ``live'' process or a core file,
and a target thread may be a ``live'' thread in a process,
or a thread in a core file.
To enable OMPD to extract state information from a target process or core file,
the third-party tool must supply the OMPD with callback functions to inquire
about the size of primitive types in the target,
look up the addresses of symbols,
as well as read and write memory in the target.
OMPD then uses these callbacks to implement its interface operations.
Signatures for the third-party tool callbacks used by OMPD are given below.

\subsection{Memory Management}

The OMPD plugin \emph{must} obtain and release heap memory \emph{only}
using the callbacks provided to it by the third-party tool.
It must \emph{not} call the heap manager directly using \code{malloc}.
For C++ implementations of OMPD this means that \code{new},
\code{new(throw)}, \code{new[]}, \code{delete}, \code{delete(throw)},
and \code{delete[]} in \emph{all} their variants \emph{must} be overloaded,
and implemented in terms of the callback functions
provided to it by the third-party tool.

\subsubsection{Memory Allocation Callback}
\label{sec:ompd:ompd_dmemory_alloc_fn_t}
\index{ompd\_dmemory\_alloc\_fn\_t@{\code{ompd\_dmemory\_alloc\_fn\_t}}}

\summary
The type signature of the callback routine provided by the third-tool
to be used by the OMPD plugin to allocate memory.

\vbox{
% the odd-looking spacing between type and argument name ensures
% they line up in the pdf
\cspecificstart
\begin{boxedcode}
typedef ompd\_rc\_t (*ompd\_dmemory\_alloc\_fn\_t) (
  ompd\_size\_t   \plc{nbytes},
  void        **\plc{ptr}
);
\end{boxedcode}
\cspecificend
}

\argdesc
The argument \plc{nbytes} gives the size in bytes of the block of memory the
OMPD wants allocated.

On success, the third-party tool returns \code{ompd\_rc\_ok} as the
return value of the callback, and the address of the newly allocated
block in \plc{*ptr}.
The newly allocated block is suitably aligned for any type of variable,
and is not guaranteed to be zeroed.

On failure an error code from \code{ompd\_rc\_t} is returned by the callback.

\crossreferences
\begin{itemize}
\item
  \code{ompd\_size\_t}, \specref{ompd:ompd_size_t}
\item
  \code{ompd\_rc\_t}, \specref{ompd:ompd_rc_t}
\item
  \code{ompd\_callbacks\_t}, \specref{ompd:ompd_callbacks_t}
\end{itemize}

\subsubsection{Memory Deallocation Callback}
\label{sec:ompd:ompd_dmemory_free_fn_t}
\index{ompd\_dmemory\_free\_fn\_t@{\code{ompd\_dmemory\_free\_fn\_t}}}

\summary
The type signature of the callback routine provided by the third-party
tool to be used by the OMPD plugin to deallocate memory.

\vbox{
\cspecificstart
\begin{boxedcode}
typedef ompd\_rc\_t (*ompd\_dmemory\_free\_fn\_t) (
  void    *\plc{ptr}
);
\end{boxedcode}
\cspecificend
}

\argdesc
\plc{ptr} is the adderss of the block to be released.
The block must have been previously allocated using the allocation callback
provided by the third-party tool to the OMPD plugin.

On success, \code{ompd\_rc\_ok} is returned by the callback to the OMPD plugin.
Otherwise, on failure an error code from \code{ompd\_rc\_t} is returned.

\crossreferences
\begin{itemize}
\item
  \code{ompd\_rc\_t}, \specref{ompd:ompd_rc_t}
\item
  \code{ompd\_callbacks\_t}, \specref{ompd:ompd_callbacks_t}
\end{itemize}

\subsection{Context Management and Navigation}

The third-party tool provides the OMPD plugin with callbacks
to manage and navigate context relationships.

\subsubsection{Context Management}
\label{sec:ompd:ompd_get_thread_context_for_osthread_fn_t}
% the use of \discretionary below overrides the character used for
% hyphenation when an optional linebreak is used.
% if \- were used, we'd get a "-" at the end of the line being
% broken.  We probably don't wantthat, but instead want just a plain
% linebreak without the "-".
\index{ompd\_get\_thread\_context\_for\_osthread\_fn\_t@{\code{ompd\_get\_thread\_context\_for\_\discretionary{}{}{}osthread\_fn\_t}}}

\summary
The type signature of the callback routine provided by the third-party
tool the OMPD plugin can use to map an operating system thread handle
to a tool \plc{thread context}.

\vbox{
% the odd-looking spacing between type and argument name ensures
% they line up in the pdf
\cspecificstart
\begin{boxedcode}
typedef ompd\_rc\_t (*ompd\_get\_thread\_context\_for\_osthread\_fn\_t) (
  \code{ompd\_address\_space\_context\_t}  *\plc{address\_space\_context},
  \code{ompd\_osthread\_kind\_t}           \plc{kind},
  \code{ompd\_size\_t}                    \plc{sizeof\_osthread},
  const void                    *\plc{osthread},
  \code{ompd\_thread\_context\_t}        **\plc{thread\_context}
);
\end{boxedcode}
\cspecificend
}

\argdesc
This callback maps an operating system thread handle within the address
space identified by \plc{address\_space\_context} to a tool thread context.
The operating system thread is identified by the combination
\plc{kind}, \plc{sizeof\_osthread}, and \plc{osthread} arguments.

On success the tool \plc{thread context} is returned in \plc{*thread\_context}.
The \plc{thread context} is created, and remains owned, by the tool.
The OMPD plugin can assume that the \plc{thread context} is valid for
as long as the tool is holding any references to \plc{thread handles}
that may contain the \plc{thread context}.
The OMPD implementation can use the \plc{thread context} to access
thread local storage (TLS).

On failure, the callback returns an error code from \code{ompd\_rc\_t}.

\crossreferences
\begin{itemize}
\item
  \code{ompd\_address\_space\_context\_t}, \specref{ompd:ompd_address_space_context_t}
\item
  \code{ompd\_osthread\_kind\_t}, \specref{ompd:ompd_osthread_kind_t}
\item
  \code{ompd\_size\_t}, \specref{ompd:ompd_size_t}
\item
  \code{ompd\_thread\_context\_t}, \specref{ompd:ompd_thread_context_t}
\item
  \code{ompd\_rc\_t}, \specref{ompd:ompd_rc_t}
\item
  \code{ompd\_callbacks\_t}, \specref{ompd:ompd_callbacks_t}
\item
  \code{ompd\_tmemory\_read\_fn\_t}, \specref{sec:ompd:ompd_tmemory_read_fn_t}
\item
  \code{ompd\_tmemory\_write\_fn\_t}, \specref{sec:ompd:ompd_tmemory_write_fn_t}
\end{itemize}

\subsubsection{Context Navigation}
\label{sec:ompd:ompd_get_address_space_context_for_thread_fn_t}
% the use of \discretionary below overrides the character used for
% hyphenation when an optional linebreak is used.
% if \- were used, we'd get a "-" at the end of the line being
% broken.  We probably don't wantthat, but instead want just a plain
% linebreak without the "-".
\index{ompd\_get\_address\_space\_context\_for\_thread\_fn\_t@{\code{ompd\_get\_address\_space\_context\_\discretionary{}{}{}for\_thread\_fn\_t}}}

\summary
The type signature of the callback routine provided by the third-party
tool the OMPD plugin can use to find the address space context
for a thread identified by a thread context.

\vbox{
% the odd-looking spacing between type and argument name ensures
% they line up in the pdf
\cspecificstart
\begin{boxedcode}
typedef ompd\_rc\_t
  (*ompd\_get\_address\_space\_context\_for\_thread\_fn\_t) (
    \code{ompd\_thread\_context\_t}          *\plc{thread\_context},
    \code{ompd\_address\_space\_context\_t}  **\plc{address\_space\_context}
);
\end{boxedcode}
\cspecificend
}

\argdesc

This callback finds the tool address space context for the thread
identified by the tool thread context \plc{thread\_context}.
On success, the callbacks returns \code{ompd\_rc\_ok},
and the context in \plc{*address\_space\_context}.
If \plc{thread\_context} refers to a host device thread,
this function returns the context for the host address space.
If \plc{thread\_context} referes to a target device thread,
this function returns the target device's address space.

On failure, the callback returns an error code from \code{ompd\_rc\_t}.

\crossreferences
\begin{itemize}
\item
  \code{ompd\_address\_space\_context\_t}, \specref{ompd:ompd_address_space_context_t}
\item
  \code{ompd\_thread\_context\_t}, \specref{ompd:ompd_thread_context_t}
\item
  \code{ompd\_rc\_t}, \specref{ompd:ompd_rc_t}
\item
  \code{ompd\_callbacks\_t}, \specref{ompd:ompd_callbacks_t}
\end{itemize}

\subsection{Sizes of Primitive Types}

The third-party tool provides the OMPD plugin with callbacks
to discover information about the sizes of the basic primitive types
in an address space.
This information is important to an OMPD plugin because the architecture
or programming model of the OpenMP runtime it is examining may be different
from the its own and that of the third-party tool which loaded it.
On platforms with multiple programming models, an OMPD plugin may need
to be able to handle OpenMP runtimes using different sizes for the
basic types.

\crossreferences
\begin{itemize}
\item
  OMPD --- Overivew, \specref{sec:ompd:ompd-overview}
\end{itemize}

\subsubsection{Representation of the Sizes of the Basic Types}
\label{sec:ompd:ompd_target_type_sizes_t}
\index{ompd\_target\_sizes\_t@{\code{ompd\_target\_sizes\_t}}}

\summary
The defintion of the type that represents the sizes of the basic types
used by an OpenMP program.

\vbox{
% the odd-looking spacing between type and argument name ensures
% they line up in the pdf
\cspecificstart
\begin{boxedcode}
typedef struct \{
  int  \plc{sizeof\_char};
  int  \plc{sizeof\_short};
  int  \plc{sizeof\_int};
  int  \plc{sizeof\_long};
  int  \plc{sizeof\_long\_long};
  int  \plc{sizeof\_pointer};
\} ompd\_target\_type\_sizes\_t;
\end{boxedcode}
\cspecificend
}

\descr
The fields of \code{ompd\_target\_type\_sizes\_t} give the sizes of
the eponymous basic types used by the OpenMP runtime.
As the third-party tool and the OMPD plugin, by definition, have the same
architecture and programming model, the size of the fields can be given
as \code{int}.

\crossreferences
\begin{itemize}
\item
  \code{ompd\_tsizeof\_fn\_t}, \specref{sec:ompd:ompd_tsizeof_fn_t}
\end{itemize}

\subsubsection{Callback to get the Sizes of the Primitive Types}
\label{sec:ompd:ompd_tsizeof_fn_t}
\index{ompd\_tsizeof\_fn\_t@{\code{ompd\_tsizeof\_fn\_t}}}

\summary
The type signature of the callback routine provided by the third-party
tool the OMPD plugin can use to find the sizes of the primitive types
in an address space.

\vbox{
% the odd-looking spacing between type and argument name ensures
% they line up in the pdf
\cspecificstart
\begin{boxedcode}
typedef ompd\_rc\_t (*ompd\_tsizeof\_fn\_t) (
    \code{ompd\_address\_space\_context\_t}  *\plc{address\_space\_context},
    \code{ompd\_target\_type\_sizes\_t}      *\plc{sizes}
);
\end{boxedcode}
\cspecificend
}

\argdesc

On success, the callback returns \code{ompd\_rc\_ok}, and the sizes of
the basic primitive types used by the \plc{address\_space\_context}
are returned through \plc{*sizes}.
Otherwise, an error code from \code{ompd\_rc\_t} is returned.

\crossreferences
\begin{itemize}
\item
  \code{ompd\_address\_space\_context\_t}, \specref{ompd:ompd_address_space_context_t}
\item
  \code{ompd\_target\_type\_sizes\_t}, \specref{sec:ompd:ompd_target_type_sizes_t}
\item
  \code{ompd\_rc\_t}, \specref{ompd:ompd_rc_t}
\item
  \code{ompd\_callbacks\_t}, \specref{ompd:ompd_callbacks_t}
\end{itemize}

\subsection{Symbol lookup}
\label{sec:ompd:ompd_tsymbol_addr_fn_t}
\index{ompd\_tsymbol\_addr\_fn\_t@{\code{ompd\_tsymbol\_addr\_fn\_t}}}

\summary
The type signature of the callback provided by the third-party tool the
OMPD plugin can use to look up the addresses of symbols in an OpenMP program.


\vbox{
% the odd-looking spacing between type and argument name ensures
% they line up in the pdf
\cspecificstart
\begin{boxedcode}
typedef ompd\_rc\_t (*ompd\_tsymbol\_addr\_fn\_t) (
    ompd\_address\_space\_context\_t  *\plc{address\_space\_context},
    ompd\_thread\_context\_t         *\plc{thread\_context},
    const char                    *\plc{symbol\_name},
    ompd\_address\_t                *\plc{symbol\_addr}
);
\end{boxedcode}
\cspecificend
}

\argdesc
The OMPD plugin may want to look up the addresses of symbols in
the OpenMP program in order to locate information it needs to satisfy
requests from the third-party tool.
This callback looks up the symbol \plc{symbol\_name}.
The \plc{thread\_context} is optional for global memory access,
and should be NULL in this case.
If \plc{thread\_context} is not NULL, it gives the thread specific
context for the symbol lookup, for the purpose of calculating thread
local storage~(TLS) addresses.

The \plc{symbol\_name} supplied by the OMPD plugin is used verbatim by
the third-party tool, and in particular, no name mangling, demangling
or other transformations are performed prior to the lookup.

On success, the callback returns \code{ompd\_rc\_ok} and the address
of the symbol in \plc{*symbol\_addr}.
Otherwise, on failure an error code from \code{ompd\_rc\_t} is returned.

\crossreferences
\begin{itemize}
\item
  \code{ompd\_address\_space\_context\_t}, \specref{ompd:ompd-address-space-context-t}
\item
  \code{ompd\_thread\_context\_t}, \specref{sec:ompd:ompd_thread_context_t}
\item
  \code{ompd\_address\_t}, \specref{ompd:ompd_address_t}
\item
  \code{ompd\_rc\_t}, \specref{ompd:ompd_rc_t}
\item
  \code{ompd\_callbacks\_t}, \specref{ompd:ompd_callbacks_t}
\end{itemize}

\subsection{Accessing Memory in the OpenMP Program or Runtime}

The OMPD plugin may need to read from, or write to, the OpenMP program.
It cannot do this directly, but instead must use the callbacks provided
to it by the third-party tool, which will perform the operation
on its behalf.

\subsubsection{Reading Memory}
\label{sec:ompd:ompd_tmemory_read_fn_t}
\index{ompd\_tmemory\_read\_fn\_t@{\code{ompd\_tmemory\_read\_fn\_t}}}

\summary

The type signature of the callback provided by the third-party tool the
OMPD plugin can use to read data out of an OpenMP program.


\vbox{
% the odd-looking spacing between type and argument name ensures
% they line up in the pdf
\cspecificstart
\begin{boxedcode}
typedef ompd\_rc\_t (*ompd\_tmemory\_read\_fn\_t) (
    ompd\_address\_space\_context\_t  *\plc{address\_space\_context},
    ompd\_thread\_context\_t         *\plc{thread\_context},
    const ompd\_address\_t          *\plc{addr},
    ompd\_word\_t                    \plc{nbytes},
    void                          *\plc{buffer}
);
\end{boxedcode}
\cspecificend
}

\argdesc
The address from which the data are to be read out of the OpenMP program
specified by \plc{address\_space\_context} is given by \plc{addr}.
\plc{nbytes} gives the number of bytes to be transfered.
The \plc{thread\_context} argument is optional for global memory access,
and in this case should be NULL.
If it is not NULL, \plc{thread\_context} identifies the thread
specific context for the memory access for the purpose of accessing
thread local storage (TLS).

The data are returned through \plc{buffer}, which is allocated and
owned by the OMPD plugin.
The contents of the buffer are unstructured, raw bytes.
It is the responsibility of the OMPD plugin to arrange for
any transformations such as byte-swapping that may be necessary
(see~\specref{sec:ompd:ompd_target_host_fn_t}) to interpret the
data returned.

On success, \code{ompd\_rc\_ok} is returned.
Otherwise, on failure an error code from \code{ompd\_rc\_t} is returned.

\crossreferences
\begin{itemize}
\item
  \code{ompd\_address\_space\_context\_t}, \specref{ompd:ompd_address_space_context_t}
\item
  \code{ompd\_thread\_context\_t}, \specref{sec:ompd:ompd_thread_context_t}
\item
  \code{ompd\_address\_t}, \specref{ompd:ompd_address_t}
\item
  \code{ompd\_tword\_t}, \specref{ompd:ompd_tword_t}
\item
  \code{ompd\_rc\_t}, \specref{ompd:ompd_rc_t}
\item
  \code{ompd\_target\_host\_fn\_t}, \specref{sec:ompd:ompd_target_host_fn_t}
\item
  \code{ompd\_callbacks\_t}, \specref{ompd:ompd_callbacks_t}
\end{itemize}

\subsubsection{Writing Memory}
\label{sec:ompd:ompd_tmemory_write_fn_t}
\index{ompd\_tmemory\_write\_fn\_t@{\code{ompd\_tmemory\_write\_fn\_t}}}

\summary

The type signature of the callback provided by the third-party tool the
OMPD plugin can use to write data to an OpenMP program.

\vbox{
% the odd-looking spacing between type and argument name ensures
% they line up in the pdf
\cspecificstart
\begin{boxedcode}
typedef ompd\_rc\_t (*ompd\_tmemory\_write\_fn\_t) (
    ompd\_address\_space\_context\_t  *\plc{address\_space\_context},
    ompd\_thread\_context\_t         *\plc{thread\_context},
    const ompd\_address\_t          *\plc{addr},
    ompd\_word\_t                    \plc{nbytes},
    const void                    *\plc{buffer}
);
\end{boxedcode}
\cspecificend
}

\argdesc
The address to which the data are to be writen in the OpenMP program
specified by \plc{address\_space\_context} is given by \plc{addr}.
\plc{nbytes} gives the number of bytes to be transfered.
The \plc{thread\_context} argument is optional for global memory access,
and in this case should be NULL.
If it is not NULL, \plc{thread\_context} identifies the thread
specific context for the memory access for the purpose of accessing
thread local storage (TLS).

The data to be written are passed through \plc{buffer}, which is allocated and
owned by the OMPD plugin.
The contents of the buffer are unstructured, raw bytes.
It is the responsibility of the OMPD plugin to arrange for
any transformations such as byte-swapping that may be necessary
(see~\specref{sec:ompd:ompd_target_host_fn_t})
to render the data into a form compatible with the OpenMP runtime.

On success, \code{ompd\_rc\_ok} is returned.
Otherwise, on failure an error code from \code{ompd\_rc\_t} is returned.

\crossreferences
\begin{itemize}
\item
  \code{ompd\_address\_space\_context\_t} \specref{ompd:ompd_address_space_context_t}
\item
  \code{ompd\_thread\_context\_t}, \specref{sec:ompd:ompd_thread_context_t}
\item
  \code{ompd\_address\_t}, \specref{ompd:ompd_address_t}
\item
  \code{ompd\_tword\_t}, \specref{ompd:ompd_tword_t}
\item
  \code{ompd\_rc\_t}, \specref{ompd:ompd_rc_t}
\item
  \code{ompd\_target\_host\_fn\_t}, \specref{sec:ompd:ompd_target_host_fn_t}
\item
  \code{ompd\_callbacks\_t}, \specref{ompd:ompd_callbacks_t}
\end{itemize}

\subsection{Data Format Conversion}
\label{sec:ompd:data-format-conversion}

The architecuture and/or programming-model of third-party tool and
OMPD plugin may be different from that of the OpenMP program being
examined.
Consequently, the conventions for representing data will differ.
The callback interface includes operations for converting between
the conventions, such as byte order (`endianness'),
used by the third-party tool and OMPD plugin on the
one hand, and the OpenMP program on the other.

\subsubsection{Primitive Data Types}
\label{sec:ompd:primitive-data-types}
\index{ompd\_target\_prim\_types\_t@{\code{ompd\_target\_prim\_types\_t}}}

\summary
The enumeration that identifies the primitive types for data format
conversion.

\textsl{\Large Do we need this?  Where is this used?}

\vbox{
% the odd-looking spacing between type and argument name ensures
% they line up in the pdf
\cspecificstart
\begin{boxedcode}
typedef enum \{
  ompd\_type\_char          = 0,
  ompd\_type\_short         = 1,
  ompd\_type\_int           = 2,
  ompd\_type\_long          = 3,
  ompd\_type\_long_long     = 4,
  ompd\_type\_pointer       = 5
\} ompd\_target\_prim\_types\_t;
\end{boxedcode}
\cspecificend
}

\descr
The members of this enumeration identify the basic types known
to the OMPD interface.

\crossreferences
\begin{itemize}
\item
  \code{ompd\_target\_host\_fn\_t}, \specref{sec:ompd:ompd_target_host_fn_t}
\end{itemize}

\subsubsection{Data Conversion}
\label{sec:ompd:ompd_target_host_fn_t}
\index{ompd\_target\_host\_fn\_t@{\code{ompd\_target\_host\_fn\_t}}}

\summary

The type signature of the callback provided by the third-party tool the
OMPD plugin can use to convert data between the formats used by the
third-part tool and OMPD plugin, and the OpenMP program.

\vbox{
% the odd-looking spacing between type and argument name ensures
% they line up in the pdf
\cspecificstart
\begin{boxedcode}
typedef ompd\_rc\_t (*ompd\_target\_host\_fn\_t) (
    ompd\_address\_space\_context\_t  *\plc{address\_space\_context},
    const void                    *\plc{input},
    int                            \plc{unit\_size},
    int                            \plc{count},
    void                          *\plc{output}
);
\end{boxedcode}
\cspecificend
}

\argdesc
The OpenMP address space associated with the data is given by
\plc{address\_space\_context}.
The source and destination buffers are given by \plc{input}
and \plc{output}, respectively.
\plc{unit\_size} gives the size of each of the elements to be converted.
\plc{count} is the number of elements to be transformed.

The input and output buffers are allocated and owned by the OMPD plugin,
and it is its responsibility to ensure that the buffers are the correct
size, and eventually deallocated when they are no longer needed.

On success, \code{ompd\_rc\_ok} is returned.
Otherwise, on failure an error code from \code{ompd\_rc\_t} is returned.

\crossreferences
\begin{itemize}
\item
  \code{ompd\_address\_space\_context\_t}, \specref{ompd:ompd_address_space_context_t}
\item
  \code{ompd\_rc\_t}, \specref{ompd:ompd_rc_t}
\item
  \code{ompd\_callbacks\_t}, \specref{ompd:ompd_callbacks_t}
\end{itemize}

\subsection{Emitting Output}
\label{sec:ompd:ompd_print_string_fn_t}
\index{ompd\_print\_string\_fn\_t@{\code{ompd\_print\_string\_fn\_t}}}

\summary

The type signature of the callback provided by the third-party tool the
OMPD plugin can use to emit output.

\vbox{
% the odd-looking spacing between type and argument name ensures
% they line up in the pdf
\cspecificstart
\begin{boxedcode}
typedef ompd\_rc\_t (*ompd\_print\_string\_fn\_t) (
    const char        *\plc{string}
);
\end{boxedcode}
\cspecificend
}

\argdesc
The OMPD plugin emits output, such as logging or debug information,
using a callback supplied to it by the third-party tool.
It should not emit output directly.

\plc{string} is the string to be printed.
No conversion or formating is performed on the string.

On success, \code{ompd\_rc\_ok} is returned.
Otherwise, on failure an error code from \code{ompd\_rc\_t} is returned.

\crossreferences
\begin{itemize}
\item
  \code{ompd\_rc\_t}, \specref{ompd:ompd_rc_t}
\item
  \code{ompd\_callbacks\_t}, \specref{ompd:ompd_callbacks_t}
\end{itemize}

\subsection{The Callback Interface}
\label{ompd:ompd_callbacks_t}
\index{ompd\_callbacks\_t@{\code{ompd\_callbacks\_t}}}

\summary

All the OMPD plugin's interactions with the OpenMP program
must be through a set of callbacks provided to it by the
third-party tool which loaded it.
These callbacks must also be used for allocating or releasing resources,
such as memory, that the plugin needs.

\vbox{
% the odd-looking spacing between type and argument name ensures
% they line up in the pdf
\cspecificstart
\begin{boxedcode}
typedef struct \{
  ompd\_dmemory\_alloc\_fn\_t   \plc{d\_alloc\_memory};
  ompd\_dmemory\_free\_fn\_t    \plc{d\_free\_memory};
  ompd\_print\_string\_fn\_t    \plc{print\_string};
  ompd\_tsizeof\_prim\_fn\_t    \plc{t\_sizeof\_prim\_type};
  ompd\_tsymbol\_addr\_fn\_t    \plc{t\_symbol\_addr\_lookup};
  ompd\_tmemory\_read\_fn\_t    \plc{t\_read\_memory};
  ompd\_tmemory\_write\_fn\_t   \plc{t\_write\_memory};
  ompd\_target\_host\_fn\_t     \plc{target\_to\_host};
  ompd\_target\_host\_fn\_t     \plc{host\_to\_target};
  ompd\_get\_thread\_context\_for\_osthread\_fn\_t
                            \plc{get\_thread\_context\_for\_osthread}:
  ompd\_get\_address\_space\_context\_for\_thread\_fn\_t
                            \plc{get\_address\_space\_context\_for\_thread};
\} ompd\_callbacks\_t;
\end{boxedcode}
\cspecificend
}

\descr
The set of callbacks the OMPD plugin should use is collected
in the \code{ompd\_callbacks\_t} record structure.
An instance of this type is passed to the OMPD plugin
as a parameter to \code{ompd\_initialize}.
Each field points to a function that the OMPD plugin should use
for interacting with the OpenMP program, or getting memory from
the third-party tool.

The \plc{d\_alloc\_memory} and \plc{d\_free\_memory} fields are
pointers to functions the OMPD plugin uses to allocate and release
dynamic memory.

\plc{print\_string} points to a function that prints a string.

The architectures or programming models of the OMPD plugin and
third party tool may be different from that of the OpenMP
program being examined.
\plc{t\_sizeof\_prim\_type} points to function that allows
the OMPD plugin to determine the sizes of the basic integer
and pointer types used by the OpenMP program.
Because of the differences in architecture or programming model,
the conventions for representing data in the OMPD plugin and
the OpenMP program may be different.
The \plc{target\_to\_host} field points to a function which translates
data from the conventions used by the OpenMP program to that used
by the third-party tool and OMPD plugin.
The reverse operation is performed by the function pointed
to by the \plc{host\_to\_target} field.

The OMPD plugin may need to access memory in the OpenMP program.
The \plc{t\_symbol\_addr\_lookup} field points to a callback the
OMPD plugin can use to find the address of a global or thread
local storage (TLS) symbol.
The \plc{t\_read\_memory} and \plc{t\_write\_memory} fields are
pointers to functions for reading from, and writing to, global or TLS
memory in the OpenMP program, respectively.

\plc{get\_thread\_context\_for\_osthread} is a pointer to a function
the OMPD plugin can use to obtain a tool context that corresponds to
an operating system id.
\plc{get\_address\_space\_context\_for\_thread} points to a callback
the OMPD plugin can use to get the tool context that `owns' the
thread represented by a tool thread context.

\crossreferences
\begin{itemize}
\item
  \code{ompd\_dmemory\_alloc\_fn\_t}, \specref{sec:ompd:ompd_dmemory_alloc_fn_t}
\item
  \code{ompd\_dmemory\_free\_fn\_t}, \specref{sec:ompd:ompd_dmemory_free_fn_t}
\item
  \code{ompd\_print\_string\_fn\_t}, \specref{sec:ompd:ompd_print_string_fn_t}
\item
  \code{ompd\_tsizeof\_prim\_fn\_t}, \specref{sec:ompd:ompd_tsizeof_fn_t}
\item
  \code{ompd\_tsymbol\_addr\_fn\_t}, \specref{sec:ompd:ompd_tsymbol_addr_fn_t}
\item
  \code{ompd\_tmemory\_read\_fn\_t}, \specref{sec:ompd:ompd_tmemory_read_fn_t}
\item
  \code{ompd\_tmemory\_write\_fn\_t}, \specref{sec:ompd:ompd_tmemory_write_fn_t}
\item
  \code{ompd\_target\_host\_fn\_t}, \specref{sec:ompd:ompd_target_host_fn_t}
\item
  \code{ompd\_get\_thread\_context\_for\_osthread\_fn\_t}, \specref{sec:ompd:ompd_get_thread_context_for_osthread_fn_t}
\item
  \code{ompd\_get\_address\_space\_context\_for\_thread\_fn\_t}, \specref{sec:ompd:ompd_get_address_space_context_for_thread_fn_t}
\end{itemize}




\subsection{OMPD Tool Interface Routines}
\label{subsec:ompd-api}

\subsubsection{Per OMPD Library Initialization and Finalization}

The OMPD library must be initialized exactly once after it is loaded, 
and finalized exactly once before it is unloaded. Per OpenMP process 
or core file initialization and finalization are also required.

Once loaded, the tool can determine the version of the OMPD API that 
the library supports by calling \code{ompd_get_api_version} (see
\specref{subsubsubsec:ompd_get_api_version}). If the tool supports the 
version that \code{ompd_get_api_version} returns, the tool starts the 
initialization by calling \code{ompd_initialize} (see
\specref{subsubsubsec:ompd_initialize}) using the version of the OMPD API 
that the library supports. If the tool does not support the version that
\code{ompd_get_api_version} returns, it may attempt to call 
\code{ompd_initialize} with a different version.



\subsubsubsection{\hcode{ompd_initialize}}
\label{subsubsubsec:ompd_initialize}

\summary
The \code{ompd_initialize} function initializes the OMPD library.

\format
\begin{cspecific}
\begin{ompSyntax}
ompd_rc_t ompd_initialize(
  ompd_word_t \plc{api_version},
  const ompd_callbacks_t *\plc{callbacks}
);
\end{ompSyntax}
\end{cspecific}

\descr
A tool that uses OMPD calls \code{ompd_initialize} to initialize each OMPD 
library that it loads. More than one library may be present in a third-party 
tool, such as a debugger, because the tool may control multiple devices, which
may use different runtime systems that require different OMPD libraries. This 
initialization must be performed exactly once before the tool can begin to 
operate on an OpenMP process or core file.

\argdesc
The \plc{api_version} argument is the OMPD API version that the tool requests
to use. The tool may call \code{ompd_get_api_version} to obtain the latest 
version that the OMPD library supports.

The tool provides the OMPD library with a set of callback functions in the 
\plc{callbacks} input argument which enables the OMPD library to allocate and 
to deallocate memory in the tool's address space, to lookup the sizes of basic 
primitive types in the device, to lookup symbols in the device, and to read and 
to write memory in the device.

\crossreferences
\begin{itemize}
\item \code{ompd_rc_t} type, see \specref{subsubsec:ompd_rc_t}.

\item \code{ompd_callbacks_t} type, see \specref{subsubsec:ompd_callbacks_t}.

\item \code{ompd_get_api_version} call, 
see \specref{subsubsubsec:ompd_get_api_version}.
\end{itemize}



\subsubsubsection{\hcode{ompd_get_api_version}}
\label{subsubsubsec:ompd_get_api_version}

\summary
The \code{ompd_get_api_version} function returns the OMPD API version.

\format
\begin{cspecific}
\begin{ompSyntax}
ompd_rc_t ompd_get_api_version(ompd_word_t *\plc{version});
\end{ompSyntax}
\end{cspecific}

\descr
The tool may call the \code{ompd_get_api_version} function to obtain the 
latest OMPD API version number of the OMPD library.

\argdesc
The latest version number is returned in to the location to which the 
\plc{version} argument points.

\crossreferences
\begin{itemize}
\item \code{ompd_rc_t} type, see \specref{subsubsec:ompd_rc_t}.
\end{itemize}



\subsubsubsection{\hcode{ompd_get_version_string}}
\label{subsubsubsec:ompd_get_version_string}

\summary
The \code{ompd_get_version_string} function returns a descriptive 
string for the OMPD API version.

\format
\begin{cspecific}
\begin{ompSyntax}
ompd_rc_t ompd_get_version_string(const char **\plc{string});
\end{ompSyntax}
\end{cspecific}


\descr
The tool may call this function to obtain a pointer to a descriptive 
version string of the OMPD API  version.

\argdesc
A pointer to a descriptive version string is placed into the location
to which \plc{string} output argument points. The OMPD library owns the string 
that the OMPD library returns; the tool must not modify or release this string.
The string remains valid for as long as the library is loaded. The
\code{ompd_get_version_string} function may be called before 
\code{ompd_initialize} (see \specref{subsubsubsec:ompd_initialize}).
Accordingly, the OMPD library must not use heap or stack memory for the string.

The signatures of \code{ompd_get_api_version} (see
\specref{subsubsubsec:ompd_get_api_version}) and \code{ompd_get_version_string} 
are guaranteed not to change in future versions of the API. In contrast, the 
type definitions and prototypes in the rest of the API do not carry the same 
guarantee. Therefore a tool that uses OMPD should check the version of the API 
of the loaded OMPD library before it calls any other function of the API.

\crossreferences
\begin{itemize}
\item \code{ompd_rc_t} type, see \specref{subsubsec:ompd_rc_t}.
\end{itemize}



\subsubsubsection{\hcode{ompd_finalize}}
\label{subsubsubsec:ompd_finalize}

\summary
When the tool is finished with the OMPD library it should call 
\code{ompd_finalize} before it unloads the library.

\format
\begin{cspecific}
\begin{ompSyntax}
ompd_rc_t ompd_finalize(void);
\end{ompSyntax}
\end{cspecific}

\descr
The call to \code{ompd_finalize}is must be the last OMPD call that the tool 
makes before it unloads the library. This call allows the OMPD library to
free any resources that it may be holding.

The OMPD library may implement a \emph{finalizer} section, which executes
as the library is unloaded, and therefore after the call to \code{ompd_finalize}. 
During finalization, the OMPD library may use the callbacks that tool earlier 
provided after the call to \code{ompd_initialize}.

\crossreferences
\begin{itemize}
\item \code{ompd_rc_t} type, see \specref{subsubsec:ompd_rc_t}.
\end{itemize}



\subsubsection{Per OpenMP Process Initialization and Finalization}



\subsubsubsection{\hcode{ompd_process_initialize}}
\label{subsubsubsec:ompd_process_initialize}
\summary
A tool calls \code{ompd_process_initialize} to obtain an address space 
handle when it initializes a session on a live process or core file.

\format
\begin{cspecific}
\begin{ompSyntax}
ompd_rc_t ompd_process_initialize(
  ompd_address_space_context_t *\plc{context},
  ompd_address_space_handle_t **\plc{handle}
);
\end{ompSyntax}
\end{cspecific}

\descr
A tool calls \code{ompd_process_initialize} to obtain an address space 
handle when it initializes a session on a live process or core file.
On return from \code{ompd_process_initialize}, the tool owns the address
space handle, which it must release with \code{ompd_rel_address_space_handle}.
The initialization function must be called before any OMPD operations
are performed on the OpenMP process. This call allows the OMPD library 
to confirm that it can handle the OpenMP process or core file that the 
\plc{context} identifies. Incompatibility is signaled by a 
return value of \code{ompd_rc_incompatible}. 

\argdesc
The \plc{context} argument is an opaque handle that the tool provides to 
address an address space. On return, the \plc{handle} argument provides 
an opaque handle to the tool for this address space, which the tool must
release when it is no longer needed.

\crossreferences
\begin{itemize}
\item \code{ompd_address_space_handle_t} type, 
see \specref{subsubsec:ompd_address_space_handle_t}.

\item \code{ompd_address_space_context_t} type, 
see \specref{subsubsec:ompd_address_space_context_t}.

\item \code{ompd_rc_t} type, see \specref{subsubsec:ompd_rc_t}.

\item \code{ompd_rel_address_space_handle} type, 
see \specref{subsubsubsec:ompd_rel_address_space_handle}.
\end{itemize}



\subsubsubsection{\hcode{ompd_device_initialize}}
\label{subsubsubsec:ompd_device_initialize}

\summary
A tool calls \code{ompd_device_initialize} to obtain an address space 
handle for a device that has at least one  active target region.

\format
\begin{cspecific}
\begin{ompSyntax}
ompd_rc_t ompd_device_initialize(
  ompd_address_space_handle_t *\plc{process_handle},
  ompd_address_space_context_t *\plc{device_context},
  omp_device_t \plc{kind},
  ompd_size_t \plc{sizeof_id},
  void *\plc{id},
  ompd_address_space_handle_t **\plc{device_handle}
);
\end{ompSyntax}
\end{cspecific}

\descr
A tool calls \code{ompd_device_initialize} to obtain an address space 
handle for a device that has at least one  active target region. On return 
from \code{ompd_device_initialize}, the tool owns the address space handle.

\argdesc
The \plc{process_handle} argument is an opaque handle that the tool provides
to reference the address space of the OpenMP process. The \plc{device_context} 
argument is an opaque handle that the tool provides to reference a device 
address space. The \plc{kind}, \plc{sizeof_id}, and \plc{id} arguments represent 
a device identifier. On return the \plc{device_handle} argument provides an 
opaque handle to the tool for this address space.

\crossreferences
\begin{itemize}
\item \code{ompd_size_t} type, see \specref{subsubsubsec:ompd_size_t}.

\item \code{omp_device_t} type, see \specref{ompd:omp_device_t}.

\item \code{ompd_address_space_handle_t} type, 
see \specref{subsubsec:ompd_address_space_handle_t}.

\item \code{ompd_address_space_context_t} type, 
see \specref{subsubsec:ompd_address_space_context_t}.

\item \code{ompd_rc_t} type, see \specref{subsubsec:ompd_rc_t}.
\end{itemize}



\subsubsubsection{\hcode{ompd_rel_address_space_handle}}
\label{subsubsubsec:ompd_rel_address_space_handle}

\summary
A tool calls \code{ompd_rel_address_space_handle} to release an address space
handle.

\format
\begin{cspecific}
\begin{ompSyntax}
ompd_rc_t ompd_rel_address_space_handle(
  ompd_address_space_handle_t *\plc{handle}
);
\end{ompSyntax}
\end{cspecific}

\descr
When the tool is finished with the OpenMP process address space handle it
should call \code{ompd_rel_address_space_handle} to release the handle, 
which allows the OMPD library to release any resources that it has related 
to the address space.

\argdesc
The \plc{handle} argument is an opaque handle for the address space to be released.

\restrictions
The \code{ompd_rel_address_space_handle} has the following restriction:

\begin{itemize}
\item An address space context must not be used after the corresponding 
      address space handle is released.
\end{itemize}

\crossreferences
\begin{itemize}
\item \code{ompd_address_space_handle_t} type, 
see \specref{subsubsec:ompd_address_space_handle_t}.

\item \code{ompd_rc_t} type, see \specref{subsubsec:ompd_rc_t}.
\end{itemize}



\subsubsection{Thread and Signal Safety}

The OMPD library does not need to be reentrant. The tool must ensure that only 
one thread enters the OMPD library at a time. The OMPD library must not install 
signal handlers or otherwise interfere with the tool's signal configuration.



\subsubsection{Address Space Information}

\subsubsubsection{\hcode{ompd_get_omp_version}}
\label{subsubsubsec:ompd_get_omp_version}

\summary
The tool may call the \code{ompd_get_omp_version} function to obtain the version 
of the OpenMP API that is associated with an address space.

\format
\begin{cspecific}
\begin{ompSyntax}
ompd_rc_t ompd_get_omp_version(
  ompd_address_space_handle_t *\plc{address_space},
  ompd_word_t *\plc{omp_version}
);
\end{ompSyntax}
\end{cspecific}

\descr
The tool may call the \code{ompd_get_omp_version} function to obtain the 
version of the OpenMP API that is associated with the address space.

\argdesc
The \plc{address_space} argument is an opaque handle that the tool provides
to reference the address space of the OpenMP process or device.

Upon return, the \plc{omp_version} argument contains the version of the OpenMP 
runtime in the \code{_OPENMP} version macro format.

\crossreferences
\begin{itemize}
\item \code{ompd_address_space_handle_t} type, 
see \specref{subsubsec:ompd_address_space_handle_t}.

\item \code{ompd_rc_t} type, see \specref{subsubsec:ompd_rc_t}.
\end{itemize}



\subsubsubsection{\hcode{ompd_get_omp_version_string}}
\label{subsubsubsec:ompd_get_omp_version_string}

\summary
The \code{ompd_get_omp_version_string} function returns a descriptive 
string for the OpenMP API version that is associated with an address space.

\format
\begin{cspecific}
\begin{ompSyntax}
ompd_rc_t ompd_get_omp_version_string(
  ompd_address_space_handle_t *\plc{address_space},
  const char **\plc{string}
);
\end{ompSyntax}
\end{cspecific}

\descr
After initialization, the tool may call the \code{ompd_get_omp_version_string} 
function to obtain the version of the OpenMP API that is associated with an 
address space.

\argdesc
The \plc{address_space} argument is an opaque handle that the tool provides
to reference the address space of the OpenMP process or device. A pointer to 
a descriptive version string is placed into the location to which the 
\plc{string} output argument points. After returning from the call, the tool 
owns the string. The OMPD library must use the memory allocation callback that 
the tool provides to allocate the string storage. The tool is responsible 
for releasing the memory.

\crossreferences
\begin{itemize}
\item \code{ompd_address_space_handle_t} type, 
see \specref{subsubsec:ompd_address_space_handle_t}.

\item \code{ompd_rc_t} type, see \specref{subsubsec:ompd_rc_t}.
\end{itemize}



\subsubsection{Thread Handles}



\subsubsubsection{\hcode{ompd_get_thread_in_parallel}}
\label{subsubsubsec:ompd_get_thread_in_parallel}

\summary
The  \code{ompd_get_thread_in_parallel} function enables a tool to obtain 
handles for OpenMP threads that are associated with a parallel region.

\format
\begin{cspecific}
\begin{ompSyntax}
ompd_rc_t ompd_get_thread_in_parallel(
  ompd_parallel_handle_t *\plc{parallel_handle},
  int \plc{thread_num},
  ompd_thread_handle_t **\plc{thread_handle}
);
\end{ompSyntax}
\end{cspecific}


\descr
A successful invocation of  \code{ompd_get_thread_in_parallel} returns a 
pointer to a thread handle in the location to which \code{thread_handle}
points. This call yields  meaningful results only if all OpenMP threads 
in the parallel region are stopped.

\argdesc
The \plc{parallel_handle} argument is an opaque handle for a parallel region
and selects the parallel region on which to operate. The \plc{thread_num} 
argument selects the thread of the team to be returned. On return, the 
\plc{thread_handle} argument is an opaque handle for the selected thread.

\restrictions
The \code{ompd_get_thread_in_parallel} function has the following restriction:

\begin{itemize}
\item The value of \plc{thread_num} must be a non-negative integer smaller 
      than the team size that was provided as the \plc{ompd-team-size-var} 
      from \code{ompd_get_icv_from_scope}.
\end{itemize}

\crossreferences
\begin{itemize}
\item \code{ompd_parallel_handle_t} type, 
see \specref{subsubsec:ompd_parallel_handle_t}.

\item \code{ompd_thread_handle_t} type, 
see \specref{subsubsec:ompd_thread_handle_t}.

\item \code{ompd_rc_t} type, see \specref{subsubsec:ompd_rc_t}.

\item \code{ompd_get_icv_from_scope} call, 
see \specref{subsubsubsec:ompd_get_icv_from_scope}.
\end{itemize}



\subsubsubsection{\hcode{ompd_get_thread_handle}}
\label{subsubsubsec:ompd_get_thread_handle}

\summary
The \code{ompd_get_thread_handle} function maps a native thread 
to an OMPD thread handle.

\format
\begin{cspecific}
\begin{ompSyntax}
ompd_rc_t ompd_get_thread_handle(
  ompd_address_space_handle_t *\plc{handle},
  ompd_thread_id_t \plc{kind},
  ompd_size_t \plc{sizeof_thread_id},
  const void *\plc{thread_id},
  ompd_thread_handle_t **\plc{thread_handle}
);
\end{ompSyntax}
\end{cspecific}

\descr
The \code{ompd_get_thread_handle} function determines if a native thread 
that the native thread identifier to which  \plc{thread_id} points represents 
an OpenMP thread. If so, the function returns \code{ompd_rc_ok} and the 
location to which \plc{thread_handle} points is set to the thread handle 
for the OpenMP thread.

\argdesc
The \plc{handle} argument is an opaque handle that the tool provides
to reference an address space. The \plc{kind}, \plc{sizeof_thread_id}, 
and \plc{thread_id} arguments represent a native thread identifier.
On return, the \plc{thread_handle} argument provides an opaque handle 
to the thread within the provided address space.

The native thread identifier to which \plc{thread_id} points is guaranteed 
to be valid  for the duration of the call. If the OMPD library must retain 
the native thread identifier, it must copy it.

\crossreferences
\begin{itemize}
\item \code{ompd_size_t} type, see \specref{subsubsubsec:ompd_size_t}.

\item \code{ompd_thread_id_t} type, see \specref{ompd:ompd_thread_id_t}.

\item \code{ompd_address_space_handle_t} type, 
see \specref{subsubsec:ompd_address_space_handle_t}.

\item \code{ompd_thread_handle_t} type, 
see \specref{subsubsec:ompd_thread_handle_t}.

\item \code{ompd_rc_t} type, see \specref{subsubsec:ompd_rc_t}.
\end{itemize}



\subsubsubsection{\hcode{ompd_rel_thread_handle}}
\label{subsubsubsec:ompd_rel_thread_handle}
\summary
The \code{ompd_rel_thread_handle} function releases a thread handle.

\format
\begin{cspecific}
\begin{ompSyntax}
ompd_rc_t ompd_rel_thread_handle(
  ompd_thread_handle_t *\plc{thread_handle}
);
\end{ompSyntax}
\end{cspecific}

\descr
Thread handles are opaque to tools, which therefore cannot release them 
directly. Instead, when the tool is finished with a thread handle it must 
pass it to \code{ompd_rel_thread_handle} for disposal.

\argdesc
The \plc{thread_handle} argument is an opaque handle for a thread to be released.

\crossreferences
\begin{itemize}
\item \code{ompd_thread_handle_t} type, 
see \specref{subsubsec:ompd_thread_handle_t}.

\item \code{ompd_rc_t} type, see \specref{subsubsec:ompd_rc_t}.
\end{itemize}



\subsubsubsection{\hcode{ompd_thread_handle_compare}}
\label{subsubsubsec:ompd_thread_handle_compare}
\summary
The \code{ompd_thread_handle_compare} function allows tools to compare
two thread handles.

\format
\begin{cspecific}
\begin{ompSyntax}
ompd_rc_t ompd_thread_handle_compare(
  ompd_thread_handle_t *\plc{thread_handle_1},
  ompd_thread_handle_t *\plc{thread_handle_2},
  int *\plc{cmp_value}
);
\end{ompSyntax}
\end{cspecific}

\descr
The internal structure of thread handles is opaque to a tool. While the 
tool can easily compare pointers to thread handles, it cannot determine 
whether handles of two different addresses refer to the same underlying 
thread. The \code{ompd_thread_handle_compare} function compares thread handles.

On success, \code{ompd_thread_handle_compare} returns in the location to 
which \plc{cmp_value} points a signed integer value that indicates how the 
underlying threads compare: a value less than, equal to, or greater than 0 
indicates that the thread corresponding to \plc{thread_handle_1} is, respectively,
less than, equal to, or greater than that corresponding to \plc{thread_handle_2}.

\argdesc
The \plc{thread_handle_1} and \plc{thread_handle_2} arguments are opaque 
handles for threads. On return the \plc{cmp_value} argument is set to a 
signed integer value.

\crossreferences
\begin{itemize}
\item \code{ompd_thread_handle_t} type, 
see \specref{subsubsec:ompd_thread_handle_t}.

\item \code{ompd_rc_t} type, see \specref{subsubsec:ompd_rc_t}.
\end{itemize}



\subsubsubsection{\hcode{ompd_get_thread_id}}
\label{subsubsubsec:ompd_get_thread_id}
\summary
The \code{ompd_get_thread_id} maps an OMPD thread handle to a native thread.

\format
\begin{cspecific}
\begin{ompSyntax}
ompd_rc_t ompd_get_thread_id(
  ompd_thread_handle_t *\plc{thread_handle},
  ompd_thread_id_t \plc{kind},
  ompd_size_t \plc{sizeof_thread_id},
  void *\plc{thread_id}
);
\end{ompSyntax}
\end{cspecific}

\descr
The \code{ompd_get_thread_id} function maps an OMPD thread handle to a 
native thread identifier.

\argdesc
The \plc{thread_handle} argument is an opaque thread handle. The \plc{kind} 
argument represents the native thread identifier. The \plc{sizeof_thread_id}
argument represents the size of the native thread identifier. On return, the 
\plc{thread_id} argument is a buffer that represents a native thread identifier.

\crossreferences
\begin{itemize}
\item \code{ompd_size_t} type, see \specref{subsubsubsec:ompd_size_t}.

\item \code{ompd_thread_id_t} type, see \specref{ompd:ompd_thread_id_t}.

\item \code{ompd_thread_handle_t} type, 
see \specref{subsubsec:ompd_thread_handle_t}.

\item \code{ompd_rc_t} type, see \specref{subsubsec:ompd_rc_t}.
\end{itemize}



\subsubsection{Parallel Region Handles}

\subsubsubsection{\hcode{ompd_get_curr_parallel_handle}}
\label{subsubsubsec:ompd_get_curr_parallel_handle}
\summary
The  \code{ompd_get_curr_parallel_handle} function obtains a pointer 
to the parallel handle for an OpenMP thread's current parallel region.

\format
\begin{cspecific}
\begin{ompSyntax}
ompd_rc_t ompd_get_curr_parallel_handle(
  ompd_thread_handle_t *\plc{thread_handle},
  ompd_parallel_handle_t **\plc{parallel_handle}
);
\end{ompSyntax}
\end{cspecific}

\descr
The  \code{ompd_get_curr_parallel_handle} function enables the tool 
to obtain a pointer to the parallel handle for the current parallel region 
that is associated with an OpenMP thread. This call is meaningful only if 
the associated thread is stopped. The parallel handle must be released by 
calling \code{ompd_rel_parallel_handle}.

\argdesc
The \plc{thread_handle} argument is an opaque handle for a thread and selects 
the thread on which to operate. On return, the \plc{parallel_handle} 
argument is set to a handle for the parallel region that the associated thread 
is currently executing, if any.

\crossreferences
\begin{itemize}
\item \code{ompd_thread_handle_t} type, 
see \specref{subsubsec:ompd_thread_handle_t}.

\item \code{ompd_parallel_handle_t} type, 
see \specref{subsubsec:ompd_parallel_handle_t}.

\item \code{ompd_rc_t} type, see \specref{subsubsec:ompd_rc_t}.

\item \code{ompd_rel_parallel_handle} call, 
see \specref{subsubsubsec:ompd_rel_parallel_handle}.
\end{itemize}



\subsubsubsection{\hcode{ompd_get_enclosing_parallel_handle}}
\label{subsubsubsec:ompd_get_enclosing_parallel_handle}

\summary
The  \code{ompd_get_enclosing_parallel_handle} function obtains a pointer 
to the parallel handle for an enclosing parallel region.

\format
\begin{cspecific}
\begin{ompSyntax}
ompd_rc_t ompd_get_enclosing_parallel_handle(
  ompd_parallel_handle_t *\plc{parallel_handle},
  ompd_parallel_handle_t **\plc{enclosing_parallel_handle}
);
\end{ompSyntax}
\end{cspecific}

\descr
The  \code{ompd_get_enclosing_parallel_handle} function enables a tool 
to obtain a pointer to the parallel handle for the parallel region that
encloses the parallel region that \code{parallel_handle} specifies. This 
call is meaningful only if at least one thread in the parallel region 
is stopped. A pointer to the parallel handle for the enclosing region 
is returned in the location to which \plc{enclosing_parallel_handle}
points. After the call, the tool owns the handle; the tool must release the 
handle with \code{ompd_rel_parallel_handle} when it is no longer required.

\argdesc
The \plc{parallel_handle} argument is an opaque handle for a parallel 
region that selects the parallel region on which to operate. On return, 
the \plc{enclosing_parallel_handle} argument is set to a handle for the 
parallel region that encloses the selected parallel region.

\crossreferences
\begin{itemize}
\item \code{ompd_parallel_handle_t} type, 
see \specref{subsubsec:ompd_parallel_handle_t}.

\item \code{ompd_rc_t} type, see \specref{subsubsec:ompd_rc_t}.

\item \code{ompd_rel_parallel_handle} call, 
see \specref{subsubsubsec:ompd_rel_parallel_handle}.
\end{itemize}



\subsubsubsection{\hcode{ompd_get_task_parallel_handle}}
\label{subsubsubsec:ompd_get_task_parallel_handle}
\summary
The  \code{ompd_get_task_parallel_handle} function obtains a pointer to 
the parallel handle for the parallel region that encloses a task region.

\format
\begin{cspecific}
\begin{ompSyntax}
ompd_rc_t ompd_get_task_parallel_handle(
  ompd_task_handle_t *\plc{task_handle},
  ompd_parallel_handle_t **\plc{task_parallel_handle}
);
\end{ompSyntax}
\end{cspecific}

\descr
The  \code{ompd_get_task_parallel_handle} function enables a tool to obtain a
pointer to the parallel handle for the parallel region that encloses the task 
region that \plc{task_handle} specifies. This call is meaningful only if at 
least one thread in the parallel region is stopped. A pointer to the parallel 
regions handle is returned in the location to which \plc{task_parallel_handle}
points. The tool owns that parallel handle, which it must release with 
\code{ompd_rel_parallel_handle}.

\argdesc
The \plc{task_handle} argument is an opaque handle that selects the task on 
which to operate. On return, the \plc{parallel_handle} argument is set 
to a handle for the parallel region that encloses the selected task.

\crossreferences
\begin{itemize}
\item \code{ompd_task_handle_t} type, see \specref{subsubsec:ompd_task_handle_t}.

\item \code{ompd_parallel_handle_t} type, 
see \specref{subsubsec:ompd_parallel_handle_t}.

\item \code{ompd_rc_t} type, see \specref{subsubsec:ompd_rc_t}.

\item \code{ompd_rel_parallel_handle} call, 
see \specref{subsubsubsec:ompd_rel_parallel_handle}.
\end{itemize}



\subsubsubsection{\hcode{ompd_rel_parallel_handle}}
\label{subsubsubsec:ompd_rel_parallel_handle}
\summary
The \code{ompd_rel_parallel_handle} function releases a parallel region handle.

\format
\begin{cspecific}
\begin{ompSyntax}
ompd_rc_t ompd_rel_parallel_handle(
  ompd_parallel_handle_t *\plc{parallel_handle}
);
\end{ompSyntax}
\end{cspecific}

\descr
Parallel region handles are opaque so tools cannot release them directly. 
Instead, a tool must pass a parallel region handle to the 
\code{ompd_rel_parallel_handle} function for disposal when finished with it.

\argdesc
The \plc{parallel_handle} argument is an opaque handle to be released.

\crossreferences
\begin{itemize}
\item \code{ompd_parallel_handle_t} type, 
see \specref{subsubsec:ompd_parallel_handle_t}.

\item \code{ompd_rc_t} type, see \specref{subsubsec:ompd_rc_t}.
\end{itemize}



\subsubsubsection{\hcode{ompd_parallel_handle_compare}}
\label{subsubsubsec:ompd_parallel_handle_compare}

\summary
The \code{ompd_parallel_handle_compare} function compares two parallel 
region handles.

\format
\begin{cspecific}
\begin{ompSyntax}
ompd_rc_t ompd_parallel_handle_compare(
  ompd_parallel_handle_t *\plc{parallel_handle_1},
  ompd_parallel_handle_t *\plc{parallel_handle_2},
  int *\plc{cmp_value}
);
\end{ompSyntax}
\end{cspecific}

\descr
The internal structure of parallel region handles is opaque to tools. While 
tools can easily compare pointers to parallel region handles, they cannot 
determine whether handles at two different addresses refer to the same 
underlying parallel region and, instead must use the 
\code{ompd_parallel_handle_compare} function.

On success, \code{ompd_parallel_handle_compare} returns a signed integer value 
in the location to which \plc{cmp_value} points that indicates how the underlying 
parallel regions compare. A value less than, equal to, or greater than 0 indicates
that the region corresponding to \plc{parallel_handle_1} is, respectively, less 
than, equal to, or greater than that corresponding to \plc{parallel_handle_2}.
This function is provided since the means by which parallel region handles are 
ordered is implementation defined.

\argdesc
The \plc{parallel_handle_1} and \plc{parallel_handle_2} arguments are 
opaque handles that correspond to parallel regions. On return the \plc{cmp_value}
argument points to a signed integer value that indicates how the underlying 
parallel regions compare.

\crossreferences
\begin{itemize}
\item \code{ompd_parallel_handle_t} type, 
see \specref{subsubsec:ompd_parallel_handle_t}.

\item \code{ompd_rc_t} type, see \specref{subsubsec:ompd_rc_t}.
\end{itemize}



\subsubsection{Task Handles}



\subsubsubsection{\hcode{ompd_get_curr_task_handle}}
\label{subsubsubsec:ompd_get_curr_task_handle}

\summary
The \code{ompd_get_curr_task_handle} function obtains a pointer to the task 
handle for the current task region that is associated with an OpenMP thread.

\format
\begin{cspecific}
\begin{ompSyntax}
ompd_rc_t ompd_get_curr_task_handle(
  ompd_thread_handle_t *\plc{thread_handle},
  ompd_task_handle_t **\plc{task_handle}
);
\end{ompSyntax}
\end{cspecific}

\descr
The \code{ompd_get_curr_task_handle} function obtains a pointer to the task 
handle for the current task region that is associated with an OpenMP thread.
This call is meaningful only if the thread for which the handle is provided 
is stopped. The task handle must be released with \code{ompd_rel_task_handle}.

\argdesc
The \plc{thread_handle} argument is an opaque handle that selects the thread 
on which to operate. On return, the \plc{task_handle} argument points to a 
location that points to a handle for the task that the thread is currently 
executing.

\crossreferences
\begin{itemize}
\item \code{ompd_thread_handle_t} type, 
see \specref{subsubsec:ompd_thread_handle_t}.

\item \code{ompd_task_handle_t} type, see \specref{subsubsec:ompd_task_handle_t}.

\item \code{ompd_rc_t} type, see \specref{subsubsec:ompd_rc_t}.

\item \code{ompd_rel_task_handle} call, 
see \specref{subsubsubsec:ompd_rel_task_handle}.
\end{itemize}



\subsubsubsection{\hcode{ompd_get_generating_task_handle}}
\label{subsubsubsec:ompd_get_generating_task_handle}

\summary
The \code{ompd_get_generating_task_handle} function obtains a pointer 
to the task handle of a generating task region.

\format
\begin{cspecific}
\begin{ompSyntax}
ompd_rc_t ompd_get_generating_task_handle(
  ompd_task_handle_t *\plc{task_handle},
  ompd_task_handle_t **\plc{generating_task_handle}
);
\end{ompSyntax}
\end{cspecific}


\descr
The \code{ompd_get_generating_task_handle} function obtains a pointer to 
the task handle for the task that encountered the OpenMP task construct 
that generated the task represented by \plc{task_handle}. The generating 
task is the OpenMP task that was active when the task specified by 
\plc{task_handle} was created. This call is meaningful only if the thread 
that is executing the task that \plc{task_handle} specifies is stopped. 
The generating task handle must be released with \code{ompd_rel_task_handle}.

\argdesc
The \plc{task_handle} argument is an opaque handle that selects the task on
which to operate. On return, the \plc{generating_task_handle} argument points
to a location that points to a handle for the generating task.

\crossreferences
\begin{itemize}
\item \code{ompd_task_handle_t} type, see \specref{subsubsec:ompd_task_handle_t}.

\item \code{ompd_rc_t} type, see \specref{subsubsec:ompd_rc_t}.

\item \code{ompd_rel_task_handle} call, 
see \specref{subsubsubsec:ompd_rel_task_handle}.
\end{itemize}



\subsubsubsection{\hcode{ompd_get_scheduling_task_handle}}
\label{subsubsubsec:ompd_get_scheduling_task_handle}
\summary
The \code{ompd_get_scheduling_task_handle} function obtains a task handle 
for the task that was active at a task scheduling point.

\format
\begin{cspecific}
\begin{ompSyntax}
ompd_rc_t ompd_get_scheduling_task_handle(
  ompd_task_handle_t *\plc{task_handle},
  ompd_task_handle_t **\plc{scheduling_task_handle}
);
\end{ompSyntax}
\end{cspecific}

\descr
The \code{ompd_get_scheduling_task_handle} function obtains a task handle 
for the task that was active when the task that \plc{task_handle} represents
was scheduled. This call is meaningful only if the thread that is executing 
the task that \plc{task_handle} specifies is stopped. The scheduling task 
handle must be released with \code{ompd_rel_task_handle}.

\argdesc
The \plc{task_handle} argument is an opaque handle for a task and selects 
the task on which to operate. On return, the \plc{scheduling_task_handle} 
argument points to a location that points to a handle for the task that is
still on the stack of execution on the same thread and was deferred in favor 
of executing the selected task.

\crossreferences
\begin{itemize}
\item \code{ompd_task_handle_t} type, see \specref{subsubsec:ompd_task_handle_t}.

\item \code{ompd_rc_t} type, see \specref{subsubsec:ompd_rc_t}.

\item \code{ompd_rel_task_handle} call, see 
\specref{subsubsubsec:ompd_rel_task_handle}.
\end{itemize}



\subsubsubsection{\hcode{ompd_get_task_in_parallel}}
\label{subsubsubsec:ompd_get_task_in_parallel}
\summary
The  \code{ompd_get_task_in_parallel} function obtains handles for
the implicit tasks that are associated with a parallel region.

\format
\begin{cspecific}
\begin{ompSyntax}
ompd_rc_t ompd_get_task_in_parallel(
  ompd_parallel_handle_t *\plc{parallel_handle},
  int \plc{thread_num},
  ompd_task_handle_t **\plc{task_handle}
);
\end{ompSyntax}
\end{cspecific}

\descr
The \code{ompd_get_task_in_parallel} function obtains handles for
the implicit tasks that are associated with a parallel region. A 
successful invocation of  \code{ompd_get_task_in_parallel} returns 
a pointer to a task handle in the location to which \plc{task_handle}
points. This call yields meaningful results only if all OpenMP threads 
in the parallel region are stopped.

\argdesc
The \plc{parallel_handle} argument is an opaque handle that selects the 
parallel region on which to operate. The \plc{thread_num} argument selects 
the implicit task of the team that is returned. The selected implicit task 
would return \plc{thread_num} from a call of the \code{omp_get_thread_num()} 
routine. On return, the \plc{task_handle} argument points to a location 
that points to an opaque handle for the selected implicit task.

\restrictions

The following restriction applies to the \code{ompd_get_task_in_parallel} function:

\begin{itemize}
\item The value of \plc{thread_num} must be a non-negative integer that is 
      smaller than the size of the team size that is the value of the 
      \plc{ompd-team-size-var} that \code{ompd_get_icv_from_scope} returns.
\end{itemize}

\crossreferences
\begin{itemize}
\item \code{ompd_parallel_handle_t} type, 
see \specref{subsubsec:ompd_parallel_handle_t}.

\item \code{ompd_task_handle_t} type, see \specref{subsubsec:ompd_task_handle_t}.

\item \code{ompd_rc_t} type, see \specref{subsubsec:ompd_rc_t}.

\item \code{ompd_get_icv_from_scope} call, 
see \specref{subsubsubsec:ompd_get_icv_from_scope}.
\end{itemize}



\subsubsubsection{\hcode{ompd_rel_task_handle}}
\label{subsubsubsec:ompd_rel_task_handle}
\summary
This \code{ompd_rel_task_handle} function releases a task handle.

\format
\begin{cspecific}
\begin{ompSyntax}
ompd_rc_t ompd_rel_task_handle(
  ompd_task_handle_t *\plc{task_handle}
);
\end{ompSyntax}
\end{cspecific}

\descr
Task handles are opaque so tools cannot release them directly. Instead, 
when a tool is finished with a task handle it must use the 
\code{ompd_rel_task_handle} function to release it.

\argdesc
The \plc{task_handle} argument is an opaque task handle to be released.

\crossreferences
\begin{itemize}
\item \code{ompd_task_handle_t} type, see \specref{subsubsec:ompd_task_handle_t}.

\item \code{ompd_rc_t} type, see \specref{subsubsec:ompd_rc_t}.
\end{itemize}



\subsubsubsection{\hcode{ompd_task_handle_compare}}
\label{subsubsubsec:ompd_task_handle_compare}
\summary
The \code{ompd_task_handle_compare} function compares task handles.

\format
\begin{cspecific}
\begin{ompSyntax}
ompd_rc_t ompd_task_handle_compare(
  ompd_task_handle_t *\plc{task_handle_1},
  ompd_task_handle_t *\plc{task_handle_2},
  int *\plc{cmp_value}
);
\end{ompSyntax}
\end{cspecific}


\descr
The internal structure of task handles is opaque so tools  cannot directly
determine if handles at two different addresses refer to the same underlying 
task. The \code{ompd_task_handle_compare} function compares task handles.
After a successful call to \code{ompd_task_handle_compare}, the value of 
the location to which \plc{cmp_value} points is a signed integer that 
indicates how the underlying tasks compare: a value less than, equal to, 
or greater than 0 indicates that the task that corresponds to \plc{task_handle_1} 
is, respectively, less than, equal to, or greater than the task that 
corresponds to \plc{task_handle_2}. The means by which task handles are 
ordered is implementation defined.

\argdesc
The \plc{task_handle_1} and \plc{task_handle_2} arguments are opaque handles
that correspond to tasks. On return, the \plc{cmp_value} argument points to
a location in which a signed integer value indicates how the underlying tasks 
compare.

\crossreferences
\begin{itemize}
\item \code{ompd_task_handle_t} type, see \specref{subsubsec:ompd_task_handle_t}.

\item \code{ompd_rc_t} type, see \specref{subsubsec:ompd_rc_t}.
\end{itemize}



\subsubsubsection{\hcode{ompd_get_task_function}}
\label{subsubsubsec:ompd_get_task_function}

\summary
This \code{ompd_get_task_function} function returns the entry point of 
the code that corresponds to the body of a task.

\format
\begin{cspecific}
\begin{ompSyntax}
ompd_rc_t ompd_get_task_function (
  ompd_task_handle_t *\plc{task_handle},
  ompd_address_t *\plc{entry_point}
);
\end{ompSyntax}
\end{cspecific}

\descr
The \code{ompd_get_task_function} function returns the entry point of the code
that corresponds to the body of code that the task executes.

\argdesc
The \plc{task_handle} argument is an opaque handle that selects the task 
on which to operate. On return, the \plc{entry_point} argument is set to
an address that describes the beginning of application code that executes 
the task region.

\crossreferences
\begin{itemize}
\item \code{ompd_address_t} type, see \specref{subsubsubsec:ompd_address_t}.

\item \code{ompd_task_handle_t} type, see \specref{subsubsec:ompd_task_handle_t}.

\item \code{ompd_rc_t} type, see \specref{subsubsec:ompd_rc_t}.
\end{itemize}



\subsubsubsection{\hcode{ompd_get_task_frame}}
\label{subsubsubsec:ompd_get_task_frame}
\summary
The \code{ompd_get_task_frame} function extracts the frame pointers of a task.

\format
\begin{cspecific}
\begin{ompSyntax}
ompd_rc_t ompd_get_task_frame (
  ompd_task_handle_t *\plc{task_handle},
  ompd_frame_info_t *\plc{exit_frame},
  ompd_frame_info_t *\plc{enter_frame}
);
\end{ompSyntax}
\end{cspecific}

\descr
An OpenMP implementation  maintains an \code{ompt_frame_t} object for every 
implicit or explicit task. The \code{ompd_get_task_frame} function extracts 
the \plc{enter_frame} and \plc{exit_frame} fields of the \code{ompt_frame_t} 
object of the task that \plc{task_handle} identifies.

\argdesc
The \plc{task_handle} argument specifies an OpenMP task. On return, the 
\plc{exit_frame} argument points to an \code{ompd_frame_info_t} object  
that has the frame information with the same semantics as the \plc{exit_frame} 
field in the \code{ompt_frame_t} object that is associated with the specified 
task. On return, the \plc{enter_frame} argument points to an 
\code{ompd_frame_info_t} object that has the frame information with the same 
semantics as the \plc{enter_frame} field in the \code{ompt_frame_t} object 
that is associated with the specified task. The \plc{frame_flags} argument 
is a conjunction of enum items in \code{ompt_frame_flag_t}. This argument 
specifies the meaning of the \plc{exit_frame} and \plc{enter_frame} argument.

\crossreferences
\begin{itemize}
\item \code{ompd_address_t} type, see \specref{subsubsubsec:ompd_address_t}.

\item \code{ompd_task_handle_t} type, see \specref{subsubsec:ompd_task_handle_t}.

\item \code{ompd_rc_t} type, see \specref{subsubsec:ompd_rc_t}.

\item \code{ompt_frame_t} type, see \specref{sec:ompt_frame_t}.

\item \code{ompd_frame_info_t} type, see \specref{subsubsubsec:ompd_frame_info_t}.
\end{itemize}



\subsubsubsection{\hcode{ompd_enumerate_states}}
\label{subsubsubsec:ompd_enumerate_states}

\summary
The \code{ompd_enumerate_states} function enumerates thread states 
that an OpenMP implementation supports.

\format
\begin{cspecific}
\begin{ompSyntax}
ompd_rc_t ompd_enumerate_states (
  ompd_address_space_handle_t *\plc{address_space_handle},
  ompd_word_t \plc{current_state},
  ompd_word_t *\plc{next_state},
  const char **\plc{next_state_name},
  ompd_word_t *\plc{more_enums}
);
\end{ompSyntax}
\end{cspecific}

\descr
An OpenMP implementation may support only a subset of the states that 
the \code{omp_states_t} enumeration type defines. In addition, an
OpenMP implementation may support implementation-specific states.
The \code{ompd_enumerate_states} call enables a tool to enumerate 
the thread states that an OpenMP implementation supports.

When the \plc{current_state} argument is a thread state that an OpenMP 
implementation supports, the call assigns the value and string name of 
the next thread state in the enumeration to the locations to which the 
\plc{next_state} and \plc{next_state_name} arguments point.

On return, the third-party tool owns the \plc{next_state_name} string.
The OMPD library allocates storage for the string with the memory allocation 
callback that the tool provides. The tool is responsible for releasing the memory.

On return, the location to which the \plc{more_enums} argument points has
the value $1$ whenever one or more states are left in the enumeration. On
return, the location to which the \plc{more_enums} argument points
has the value $0$ when \plc{current_state} is the last state in the enumeration.

\argdesc
The \plc{address_space_handle} argument identifies the address space. 
The \plc{current_state} argument must be a thread state that the OpenMP 
implementation supports. To begin enumerating the supported states, a 
tool should pass \code{ompt_state_undefined} as the value of \plc{current_state}.
Subsequent calls to \code{ompd_enumerate_states} by the tool should pass the
value that the call returned in the \plc{next_state} argument. On return, the 
\plc{next_state} argument points to an integer with the value of the next state 
in the enumeration. On return, the \plc{next_state_name} argument points to a 
character string that describes the next state. On return, the \plc{more_enums} 
argument points to an integer with a value of $1$ when more states are left to 
enumerate and a value of $0$ when no more states are left.

\constraints
Any string that is returned through the \plc{next_state_name} argument 
must be immutable and defined for the lifetime of program execution.

\crossreferences
\begin{itemize}
\item \code{ompd_address_space_handle_t} type, 
see \specref{subsubsec:ompd_address_space_handle_t}.

\item \code{ompd_rc_t} type, see \specref{subsubsec:ompd_rc_t}.

\item \code{ompt_state_t} type, see \specref{sec:thread-states}.
\end{itemize}



\subsubsubsection{\hcode{ompd_get_state}}
\label{subsubsubsec:ompd_get_state}
\summary
The \code{ompd_get_state} function obtains the state of a thread.

\format
\begin{cspecific}
\begin{ompSyntax}
ompd_rc_t ompd_get_state (
  ompd_thread_handle_t *\plc{thread_handle},
  ompd_word_t *\plc{state},
  ompt_wait_id_t *\plc{wait_id}
);
\end{ompSyntax}
\end{cspecific}

\descr
The \code{ompd_get_state} function returns the state of an OpenMP thread.

\argdesc
The \plc{thread_handle} argument identifies the thread. The \plc{state} 
argument represents the state of that thread as represented by a value 
that \code{ompd_enumerate_states} returns. On return, the \plc{wait_id} 
argument points to a handle that corresponds to the \plc{wait_id} wait 
identifier of the thread if it is non-null. If the thread state is not 
one of the specified wait states, the value to which \plc{wait_id} 
points is undefined.

\crossreferences
\begin{itemize}
\item \code{ompd_wait_id_t} type, see \specref{subsubsubsec:ompd_wait_id_t}.

\item \code{ompd_thread_handle_t} type, 
see \specref{subsubsec:ompd_thread_handle_t}.

\item \code{ompd_rc_t} type, see \specref{subsubsec:ompd_rc_t}.

\item \code{ompd_enumerate_states} call, 
see \specref{subsubsubsec:ompd_enumerate_states}.
\end{itemize}



\subsubsection{Display Control Variables}



\subsubsubsection{\hcode{ompd_get_display_control_vars}}
\label{subsubsubsec:ompd_get_display_control_vars}

\summary
The \code{ompd_get_display_control_vars} function returns a list of 
name/value pairs for the OpenMP control variables that are user-controllable 
and important to the operation or performance of OpenMP programs.

\format
\begin{cspecific}
\begin{ompSyntax}
ompd_rc_t ompd_get_display_control_vars (
  ompd_address_space_handle_t *\plc{address_space_handle},
  const char * const **\plc{control_vars}
);
\end{ompSyntax}
\end{cspecific}

\descr
The \code{ompd_get_display_control_vars} function returns a
NULL-terminated vector of strings of name/value pairs of control
variables that have user controllable settings and are important 
to the operation or performance of an OpenMP runtime system. The 
control variables that this interface exposes include all OpenMP 
environment variables, settings that may come from vendor or 
platform-specific environment variables, and other settings that 
affect the operation or functioning of an OpenMP runtime.

The format of the strings is:
\begin{center}
\code{name=a string}
\end{center}

The third-party tool must not modify the vector or the strings.
The strings are NULL terminated.
The vector is NULL terminated.

After returning from the call, the vector and strings are `owned'
by the third-party tool.
Providing the termination constraints are satisfied, the OMPD library
is free to use static or dynamic memory for the
vector and/or the strings, and to arrange them in memory
as it pleases.
If dynamic memory is used, then the OMPD library must use
the allocate callback it received in the call to \code{ompd_initialize}.
As the third-party tool cannot make any assumptions about the origin or
layout of the memory used for the vector or strings, it cannot release
the display control variables directly when they are no longer
needed; instead it must use \code{ompd_rel_display_control_vars()}.

\argdesc
The address space is identified by the input argument \plc{address_space_handle}.
The vector of display control variables is returned through
the output argument \plc{control_vars}.

\crossreferences
\begin{itemize}
	\item \code{ompd_address_space_handle_t} type, see 
	\specref{subsubsec:ompd_address_space_handle_t}.
	\item \code{ompd_rc_t} type, see \specref{subsubsec:ompd_rc_t}.
	\item \code{ompd_initialize} call, see \specref{subsubsubsec:ompd_initialize}.
	\item \code{ompd_rel_display_control_vars} type, see 
	\specref{subsubsubsec:ompd_rel_display_control_vars}.
\end{itemize}


\subsubsubsection{\hcode{ompd_rel_display_control_vars}}
\label{subsubsubsec:ompd_rel_display_control_vars}
\summary

Releases a list of name/value pairs of OpenMP control variables
previously acquired using \code{ompd_get_display_control_vars}.
\format
\begin{cspecific}
\begin{ompSyntax}
ompd_rc_t ompd_rel_display_control_vars (
  const char * const **\plc{control_vars}
);
\end{ompSyntax}
\end{cspecific}

\descr
The vector and strings returned from \code{ompd_get_display_control_vars}
are `owned' by the third-party tool, but allocated by the OMPD library.
Because the third-party tool doesn't know how the memory for the vector
and strings was allocated, it cannot deallocate the memory itself.
Instead, the third-party tool must call
\code{ompd_rel_display_control_vars} to release the vector
and strings.

\argdesc
The input parameter \plc{control_vars} is the vector of display control variables to be released.

\crossreferences
\begin{itemize}
  \item \code{ompd_rc_t} type, see \specref{subsubsec:ompd_rc_t}.
	\item \code{ompd_get_display_control_vars} call, see \specref{subsubsubsec:ompd_get_display_control_vars}.
\end{itemize}


\subsubsection{Accessing Scope Specific Information}

\subsubsubsection{\hcode{ompd_enumerate_icvs}}
\label{subsubsubsec:ompd_enumerate_icvs}

\summary
Enumerate ICVs supported by an OpenMP implementation.

\format
\begin{cspecific}
\begin{ompSyntax}
ompd_rc_t ompd_enumerate_icvs (
  ompd_address_space_handle *\plc{handle}, 
  ompd_icv_id_t \plc{current},
  ompd_icv_id_t *\plc{next_id},
  const char **\plc{next_icv_name},
  ompd_scope_t *\plc{next_scope},
  int *\plc{more}
);
\end{ompSyntax}
\end{cspecific}

\descr
In addition to the ICVs listed in Table~\ref{tab:ICV Initial Values}, an OpenMP 
implementation must support the OMPD specific ICVs listed in Table~\ref{tab:OMPD internal varibales} in 
the OMPD interface.
An OpenMP implementation might support additional implementation specific variables.

Also an implementation might decide to store ICVs in a different scope than suggested in 
Table~\ref{tab:Scopes of ICVs}.
The \code{ompd_enumerate_icvs} call enables a tool to enumerate the ICVs supported by an 
OpenMP implementation and the related scopes.

When the \plc{current} input argument is set to a value supported by an OpenMP 
implementation,
the call will assign the value, string name, and scope of the next ICV in the enumeration
to the locations pointed to by the 
\plc{next_id}, \plc{next_icv_name}, and \plc{next_scope} output arguments, respectively.

After returning from the call, the string \plc{next_icv_name} is `owned' by the 
third-party tool.
The string storage must be allocated by the OMPD library using the memory allocation 
callback provided by the tool.
The tool is responsible for releasing the memory.


Whenever one or more ICV are left in the enumeration,
the call will set the location pointed to by the \plc{more} 
output argument to $1$.
When the last ICV in the enumeration is passed in \plc{current}, 
the call will set the location pointed to by the \plc{more} output
argument to $0$


\argdesc

The address space is identified by the input argument \plc{address_space_handle}.

The input argument \plc{current} must be an ICV supported by the OpenMP implementation.  
To begin enumerating the ICVs that an OpenMP implementation supports, a tool should pass
\code{ompd_icv_undefined} as the value of the input argument \plc{current}.  
Subsequent calls to \code{ompd_enumerate_icvs} by the tool should pass the
value returned by the call in the \plc{next_id} output argument.

The output argument \plc{next_id} is a pointer to an integer where
the call will return the id of the next ICV in the
enumeration.

The output argument \plc{next_icv} is a pointer to a
character string pointer, where the call will return a string
providing the name of the next ICV.

The output argument \plc{next_scope} is a pointer to a
scope enum value, where the call will return the scope for the next ICV.

The output argument \plc{more_enums} is a pointer to an integer where
the call will return a value of $1$ when there are more ICV left to enumerate
or a value of $0$ when there are not.

\constraints
Any string returned through the argument
\plc{next_icv} must be immutable and defined
for the lifetime of a program execution.

\begin{table}[h!]
\caption{OMPD-specific ICVs\label{tab:OMPD internal varibales}}
\begin{tabular}{p{1.5in} p{0.5in} p{2.7in}}
\hline
\textsf{\textbf{Variable}} & \textsf{\textbf{Scope}} & \textsf{\textbf{Meaning}}\\
\hline
{\splc{ompd-num-procs-var}} & device & return value of \scode{omp_get_num_procs()} when 
executed on this device \\
{\splc{ompd-thread-num-var}} & task & return value of \scode{omp_get_thread_num()} when 
executed in this task \\
{\splc{ompd-final-var}} & task &  return value of \scode{omp_in_final()} when 
executed in this task \\
{\splc{ompd-implicit-var}} & task & the task is an implicit task\\
{\splc{ompd-team-size-var}} & team & return value of \scode{omp_get_num_threads()} 
when executed in this team \\
\hline
\end{tabular}
\end{table}


\crossreferences
\begin{itemize}
	\item \code{ompd_address_space_handle_t} type, see 
	\specref{subsubsec:ompd_address_space_handle_t}.
	\item \code{ompd_scope_t} type, see \specref{subsubsec:ompd_scope_t}.
	\item \code{ompd_icv_id_t} type, see \specref{subsubsec:ompd_icv_id_t}.
	\item \code{ompd_rc_t} type, see \specref{subsubsec:ompd_rc_t}.
\end{itemize}



\subsubsubsection{\hcode{ompd_get_icv_from_scope}}
\label{subsubsubsec:ompd_get_icv_from_scope}
\summary
Returns the value of an ICV as present in the provided scope.
\format
\begin{cspecific}
\begin{ompSyntax}
ompd_rc_t ompd_get_icv_from_scope (
  void *\plc{handle}, 
  ompd_scope_t \plc{scope},
  ompd_icv_id_t \plc{icv_id},
  ompd_word_t *\plc{icv_value}
); 
\end{ompSyntax}
\end{cspecific}

\descr
The function \code{ompd_get_icv_from_scope} provides access to the internal control 
variables as defined in Tables~\ref{tab:ICV Initial Values} and~\ref{tab:OMPD internal varibales}.

\argdesc

The argument \plc{handle} provides an OpenMP scope handle.
The argument \plc{scope} specifies the kind of scope provided in \plc{handle}.
The argument \plc{icv_name} specifies the name of the requested ICV.
On success, the output argument \plc{icv_value} is set to the value of the 
requested ICV.

\constraints

If the ICV cannot be represented by an integer type value, the function returns 
\code{ompd_rc_incompatible}. 

The provided \plc{handle} must match the \plc{scope} as defined in 
\specref{subsubsec:ompd_icv_id_t}. 

The provided \plc{scope} must match the scope for \plc{icv_id} as requested by 
\code{ompd_enumerate_icvs}. 

\crossreferences
\begin{itemize}
	\item \code{ompd_address_space_handle_t} type, see 
	   \specref{subsubsec:ompd_address_space_handle_t}.
	\item \code{ompd_thread_handle_t} type, see \specref{subsubsec:ompd_thread_handle_t}.
	\item \code{ompd_parallel_handle_t} type, see 
       \specref{subsubsec:ompd_parallel_handle_t}.
	\item \code{ompd_task_handle_t} type, see \specref{subsubsec:ompd_task_handle_t}.
	\item \code{ompd_scope_t} type, see \specref{subsubsec:ompd_scope_t}.
	\item \code{ompd_icv_id_t} type, see \specref{subsubsec:ompd_icv_id_t}.
	\item \code{ompd_rc_t} type, see \specref{subsubsec:ompd_rc_t}.
\end{itemize}

\subsubsubsection{\hcode{ompd_get_icv_string_from_scope}}
\label{subsubsubsec:ompd_get_icv_string_from_scope}
\summary
Returns the value of an ICV as present in the provided scope.
\format
\begin{cspecific}
\begin{ompSyntax}
ompd_rc_t ompd_get_icv_string_from_scope (
  void *\plc{handle},
  ompd_scope_t \plc{scope},
  ompd_icv_id_t \plc{icv_id},
  const char **\plc{icv_string}
); 
\end{ompSyntax}
\end{cspecific}

\descr
The function \code{ompd_get_icv_string_from_scope} provides access to the internal 
control variables as defined in Table~\ref{tab:ICV Initial Values}.

\argdesc

The argument \plc{handle} provides an OpenMP scope handle.
The argument \plc{scope} specifies the kind of scope provided in \plc{handle}.
The argument \plc{icv_id} specifies the id of the requested ICV.
On successful return, the output argument \plc{icv_string} points to a string 
representation of the requested ICV.

After returning from the call, the string \plc{icv_string} is `owned' by the third-party 
tool.
The string storage must be allocated by the OMPD library using the memory allocation 
callback provided by the tool.
The tool is responsible for releasing the memory.


\constraints
Any string passed through the argument \plc{icv_string} must be allocated by the OMPD 
library with the memory alloc callback \code{ompd_callback_memory_alloc_fn_t} and freed 
by the tool.

The provided \plc{handle} must match the \plc{scope} as defined in 
\specref{subsubsec:ompd_icv_id_t}. 

The provided \plc{scope} must match the scope for \plc{icv_id} as requested by 
\code{ompd_enumerate_icvs}. 


\crossreferences
\begin{itemize}
	\item \code{ompd_address_space_handle_t} type, see 
	\specref{subsubsec:ompd_address_space_handle_t}.
	\item \code{ompd_thread_handle_t} type, see \specref{subsubsec:ompd_thread_handle_t}.
	\item \code{ompd_parallel_handle_t} type, see 
	\specref{subsubsec:ompd_parallel_handle_t}.
	\item \code{ompd_task_handle_t} type, see \specref{subsubsec:ompd_task_handle_t}.
	\item \code{ompd_scope_t} type, see \specref{subsubsec:ompd_scope_t}.
	\item \code{ompd_icv_id_t} type, see \specref{subsubsec:ompd_icv_id_t}.
	\item \code{ompd_rc_t} type, see \specref{subsubsec:ompd_rc_t}.
\end{itemize}

\subsubsubsection{\hcode{ompd_get_tool_data}}
\label{subsubsubsec:ompd_get_tool_data}
\summary
The \code{ompd_get_tool_data} function provides access to the OMPT data variable 
stored for each OpenMP scope.

\format

\begin{cspecific}
\begin{ompSyntax}
ompd_rc_t ompd_get_tool_data(
  void* \plc{handle}, 
  ompd_scope_t \plc{scope},
  ompd_word_t *\plc{value},
  ompd_address_t *\plc{ptr}
);
\end{ompSyntax}
\end{cspecific}

\descr
The function \code{ompd_get_tool_data} provides access to the OMPT tool data
stored for each scope.

If the runtime library has no support for OMPT, the function returns 
\code{ompd_rc_unsupported}.


\argdesc
The argument \plc{handle} provides an OpenMP scope handle.
The argument \plc{scope} specifies the kind of scope provided in \plc{handle}.
On return, the output argument \plc{value} is set to the \plc{value} field of the 
\code{ompt_data_t} union stored for the selected scope.
On return, the output argument \plc{ptr} is set to the  \plc{ptr} field of the 
\code{ompt_data_t} union stored for the selected scope.


\crossreferences
\begin{itemize}
    \item \code{ompt_data_t} type, see \specref{sec:ompt_data_t}.
	\item \code{ompd_address_space_handle_t} type, see 
        \specref{subsubsec:ompd_address_space_handle_t}.
    \item \code{ompd_thread_handle_t} type, see \specref{subsubsec:ompd_thread_handle_t}.
    \item \code{ompd_parallel_handle_t} type, see 
        \specref{subsubsec:ompd_parallel_handle_t}.
    \item \code{ompd_task_handle_t} type, see \specref{subsubsec:ompd_task_handle_t}.
    \item \code{ompd_scope_t} type, see \specref{subsubsec:ompd_scope_t}.
    \item \code{ompd_rc_t} type, see \specref{subsubsec:ompd_rc_t}.
\end{itemize}



% %%%%%%%%%%%%%%%%%%%%%%%%%%%%%%%%%%%%%%%%%%%%%%%%%%%%%%%%%%%%%%%%%%%%%%%%%%%

%%%% \ompdsection{OMPD Tool Callback Interface}
%%%% \label{sec:ompd-callbacks}

% %%%%%%%%%%%%%%%%%%%%%%%%%%%%%%%%%%%%%%%%%%%%%%%%%%%%%%%%%%%%%%%%%%%%%%%%%%%

\subsection{Runtime Entry Points for OMPD}
\label{ompd:runtime-entry-points-for-ompd}

Most of the tool's OpenMP-related activity on an OpenMP
program will be performed through the OMPD interface.
However, supporting OMPD introduced some requirements of the OpenMP
runtime.
These fall into three categories: entrypoints the user's code in
OpenMP program can call; locations in the OpenMP runtime through
which control must pass when specific events occur; and data that must
be accessible to the tool.

\subsubsection{Identifying the Matching OMPD}
\label{ompd:ompd_dll_locations}
\index{ompd\_dll\_locations@{\code{ompd\_dll\_locations}}}

\summary
Names the OMPD plugin(s) that are compatible with the runtime.

\vbox{
\cspecificstart
\begin{boxedcode}
const char **ompd\_dll\_locations;
\end{boxedcode}
\cspecificend
}

\descr
\code{ompd\_dll\_locations} is an \code{argv}-style vector of filename
strings that provide the names of any OMPD plugin implementations
that are compatible with the OpenMP runtime.
The vector is NULL-terminated.

The programming model or architecture of the tool, and
hence that of the required OMPD plugin, might not match that of
the OpenMP program to be examined.
On platforms that support multiple programming models (\textit{e.g.},
32- v. 64-bit), or in heterogenous  environments where the architectures
of the OpenMP program and tool may be be different,
OpenMP implementors are encourgaed to provide OMPD plugins for all models.
The vector, therefore, may name plugins that are not compatible
with the tool.
This is legal, and it is up to the tool to check that
a plugin is compatible.
(Typically, a tool might iterate over the vector until a compatible
plugin is found.)

\restrictions
\code{ompd\_dll\_locations} has external \code{C} linkage,
no demangling or other transformations are required by a 
tool before looking up its address in the OpenMP program.

The vector and its members must be fully initialized before
\code{ompd\_dll\_locations} is set to a non-NULL value.
That is, if \code{ompd\_dll\_locations} is not NULL, the vector
and its contents are valid.

\crossreferences
\begin{itemize}
\item
  \code{ompd\_dll\_locations\_valid}, \specref{ompd:ompd_dll_locations_valid}
\item
  Finding the OMPD plugin, \specref{ompd:finding-the-ompd}
\end{itemize}

\subsubsection{Indicating that \code{ompd\_dll\_locations} is Valid}
\label{ompd:ompd_dll_locations_valid}
\index{ompd\_dll\_locations\_valid@{\code{ompd\_dll\_locations}}}

\summary
The OpenMP indicates that the vector of names of matching OMPD plugins
is valid by allowing execution to pass through the location
identified by \code{ompd\_dll\_locations\_valid}.

\vbox{
\cspecificstart
\begin{boxedcode}
void ompd\_dll\_locations\_valid ( void );
\end{boxedcode}
\cspecificend
}

\descr
In some implementations of the OpenMP runtime the vector of names
of matching OMPD plugins may need to be constructed at runtime.
Before the vector is valid, \code{ompd\_dll\_locations} must be NULL.
To avoid the tool having to poll \code{ompd\_dll\_locations}
to become non-NULL, the OpenMP runtime must arrange for execution
to pass through the location named by \code{ompd\_dll\_locations\_valid}
once the vector has been initialized.
By planting a breakpoint at \code{ompd\_dll\_locations\_valid},
the tool can be notified when the vector of names is
ready to be read.

If \code{ompd\_dll\_locations} vector is statically allocated
and initialized, the tool should not need to use
\code{ompd\_dll\_locations\_valid} since \code{ompd\_dll\_locations}
will be non-NULL.

\restrictions

\code{ompd\_dll\_locations\_valid} has external \code{C} linkage,
and no demangling or other transformations are required by a 
tool before looking up its address in the OpenMP program.

Conceptually \code{ompd\_dll\_locations\_valid} has the signature
given above.
However, it does not need to be a function, but can be a labeled
location in the runtime code.

\crossreferences
\begin{itemize}
\item
  \code{ompd\_dll\_locations}, \specref{ompd:ompd_dll_locations}
\item
  Finding the OMPD plugin, \specref{ompd:finding-the-ompd}
\end{itemize}

\subsubsection{Event notification}
\label{ompd:event-notification}

Neither a tool nor an OpenMP runtime system know what
application code a program will launch as parallel regions
or tasks until the program invokes the runtime system and
provides a code address as an argument.
To help a tool control the execution of an OpenMP program
launching parallel regions or tasks,
the OpenMP runtime must define a number of symbols
through which execution must pass when particular events occur
\emph{and} data collection for OMPD is enabled.
These locations may, but do not have to, be subroutines,
They may, for example, be labeled locations.
The symbols must all have external, \code{C}, linkage.

A tool can gain notification of the event by planting a breakpoint
at the corresponding named location.

\subsubsubsection{Beginning Parallel Regions}
\label{ompd:ompd_bp_parallel_begin}
\index{ompd\_bp\_parallel\_begin@{\code{ompd\_bp\_parallel\_begin}}}

\summary
The OpenMP runtime must execute through
\code{ompd\_bp\_parallel\_begin} when a new parallel region is launched.

\vbox{
\cspecificstart
\begin{boxedcode}
void ompd\_bp\_parallel\_begin ( void );
\end{boxedcode}
\cspecificend
}

\descr

When starting a new parallel region, the runtime must allow execution
to flow through \code{ompd\_bp\_parallel\_begin}.
This should occur after a task encounters a parallel construct,
but before any implicit task starts to execute the parallel
region's work.

Control passes through \code{ompd\_bp\_parallel\_begin}
once per region, and not once for each thread per region.

At the point where the runtime reaches \code{ompd\_bp\_parallel\_begin},
a tool can map the encountering native thread to an OpenMP
thread handle using
\code{ompd\_get\_thread\_handle}.
At this point the handle returned by \code{ompd\_get\_current\_parallel\_handle}
is that of the new parallel region.
The tool can find the entry point of the user code that
the new parallel region will execute by passing the parallel handle region
to \code{ompd\_get\_parallel\_function}.

The actual number of threads, rather than the requested number of threads,
in the team is returned by
\code{ompd\_get\_num\_threads}.

The task handle returned by
\code{ompd\_get\_current\_task\_handle}
will be that of the task encountering the parallel construct.

The `reenter runtime' address in the information returned by
\code{ompd\_get\_task\_frame}
will be that of the stack frame where the thread called the OpenMP
runtime to handle the parallel construct.
The `exit runtime' address will be for the stack frame where the thread
left the OpenMP runtime to execute the user code that encountered
the parallel construct.

\restrictions

\code{ompd\_bp\_parallel\_begin} has external \code{C} linkage, and no
demangling or other transformations are required by a tool
to look up its address in the OpenMP program.

Conceptually \code{ompd\_bp\_parallel\_begin} has the type signature
given above.
However, it does not need to be a function, but can be a labeled location
in the runtime code.

\crossreferences

\begin{itemize}
\item
  \code{ompd\_get\_thread\_handle}, \specref{ompd:ompd_get_thread_handle}
\item
  \code{ompd\_get\_current\_parallel\_handle}, \specref{ompd:ompd_get_current_parallel_handle}
\item
  \code{ompd\_get\_parallel\_function}, \specref{ompd:ompd_get_parallel_function}
\item
  \code{ompd\_get\_num\_threads}, \specref{ompd:ompd_get_num_threads}
\item
  \code{ompd\_get\_current\_task\_handle}, \specref{ompd:ompd_get_current_task_handle}
\item
  \code{ompd\_get\_task\_frame}, \specref{ompd:ompd_get_task_frame}
\end{itemize}

\subsubsubsection{Ending Parallel Regions}
\label{ompd:ompd_bp_parallel_end}
\index{ompd\_bp\_parallel\_end@{\code{ompd\_bp\_parallel\_end}}}

The OpenMP runtime must execute through \code{ompd\_bp\_parallel\_end}
when a parallel region ends.

\vbox{
\cspecificstart
\begin{boxedcode}
void ompd\_bp\_parallel\_end ( void );
\end{boxedcode}
\cspecificend
}

\descr

When a parallel region finishes, the OpenMP runtime must allow execution
to flow through \code{ompd\_bp\_parallel\_end}.

Control passes through \code{ompd\_bp\_parallel\_end}
once per region, and not once for each thread per region.

At the point the runtime reaches \code{ompd\_bp\_parallel\_end}
the tool can map the encountering native thread
to an OpenMP thread handle using \code{ompd\_get\_thread\_handle}.
\code{ompd\_get\_current\_parallel\_handle}
returns the handle of the terminating parallel region.

\code{ompd\_get\_current\_task\_handle}
returns the handle of the task that encountered the
parallel construct that initiated the parallel region just
terminating.
The `reenter runtime' address in the frame information returned by
\code{ompd\_get\_task\_frame}
will be that for the stack frame in which the thread called the
OpenMP runtime to start the parallel construct just terminating.
The `exit runtime' address will refer to the stack frame where the
thread left the OpenMP runtime to execute the user code that
invoked the parallel construct just terminating.

\restrictions

\code{ompd\_bp\_parallel\_end} has external \code{C} linkage, and no
demangling or other transformations are required by a tool
to look up its address in the OpenMP program.

Conceptually \code{ompd\_bp\_parallel\_end} has the type signature
given above.
However, it does not need to be a function, but can be a labeled location
in the runtime code.

\crossreferences
\begin{itemize}
\item
  \code{ompd\_get\_thread\_handle}, \specref{ompd:ompd_get_thread_handle}
\item
  \code{ompd\_get\_current\_parallel\_handle}, \specref{ompd:ompd_get_current_parallel_handle}
\item
  \code{ompd\_get\_current\_task\_handle}, \specref{ompd:ompd_get_current_task_handle}
\item
  \code{ompd\_get\_task\_frame}, \specref{ompd:ompd_get_task_frame}
\end{itemize}


\subsubsubsection{Beginning Task Regions}
\label{ompd:ompd_bp_task_begin}
\index{ompd\_bp\_task\_begin@{\code{ompd\_bp\_task\_begin}}}

The OpenMP runtime must execute through \code{ompd\_bp\_task\_begin}
when a new task is started.

\vbox{
\cspecificstart
\begin{boxedcode}
void ompd\_bp\_task\_begin ( void );
\end{boxedcode}
\cspecificend
}

\descr

When starting a new task region, the OpenMP runtime system
must allow control to pass through \code{ompd\_bp\_task\_begin}.

The OpenMP runtime system will execute through this location after the task
construct is encountered, but before the new explicit task starts.
At the point where the runtime reaches \code{ompd\_bp\_task\_begin}
the tool can map the native thread to an OpenMP handle using
\code{ompd\_get\_thread\_handle}.

\code{ompd\_get\_current\_task\_handle} returns the handle of the new task region.
The entry point of the user code to be executed by the new task
is returned from
\code{ompd\_get\_task\_function}.

\restrictions

\code{ompd\_bp\_task\_begin} has external \code{C} linkage, and no
demangling or other transformations are required by a tool
to look up its address in the OpenMP program.

Conceptually \code{ompd\_bp\_task\_begin} has the type signature
given above.
However, it does not need to be a function, but can be a labeled location
in the runtime code.

\crossreferences
\begin{itemize}
\item
  \code{ompd\_get\_thread\_handle}, \specref{ompd:ompd_get_thread_handle}
\item
  \code{ompd\_get\_current\_task\_handle}, \specref{ompd:ompd_get_current_task_handle}
\item
  \code{ompd\_get\_task\_function}, \specref{ompd:ompd_get_task_function}
\end{itemize}

\subsubsubsection{Ending Task Regions}
\label{ompd:ompd_bp_task_end}
\index{ompd\_bp\_task\_end@{\code{ompd\_bp\_task\_end}}}

\summary
The OpenMP runtime must execute through \code{ompd\_bp\_task\_end}
when a task region ends.

\vbox{
\cspecificstart
\begin{boxedcode}
void ompd\_bp\_task\_end ( void );
\end{boxedcode}
\cspecificend
}

\descr

When a task region completes, the OpenMP runtime system
must allow execution to flow through the location \code{ompd\_bp\_task\_end}.

At the point where the runtime reaches \code{ompd\_bp\_task\_end}
the tool can use
\code{ompd\_get\_thread\_handle}
to map the encountering native thread to the corresponding
OpenMP thread handle.
At this point \code{ompd\_get\_current\_task\_handle}
returns the handle for the terminating task.

\restrictions

\code{ompd\_bp\_task\_end} has external \code{C} linkage, and no
demangling or other transformations are required by a tool
to look up its address in the OpenMP program.

Conceptually \code{ompd\_bp\_task\_end} has the type signature
given above.
However, it does not need to be a function, but can be a labeled location
in the runtime code.

\crossreferences
\begin{itemize}
\item
  \code{ompd\_get\_thread\_handle}, \specref{ompd:ompd_get_thread_handle}
\item
  \code{ompd\_get\_current\_task\_handle}, \specref{ompd:ompd_get_current_task_handle}
\end{itemize}


\subsection{Entry Points for OMPD defined in the OpenMP program}
%REDEFINITION \label{ompd:runtime-entry-points-for-ompd}

There is a small group of OpenMP entrypoints that are defined in
OpenMP program, rather than in the OpenMP runtime.
Unless otherwise stated, these entrypoints need not be defined
in a program, in which case default behavior will apply.

\subsubsection{Enabling Support for OMPD at Runtime}
\label{ompd:ompd_enable}
\index{ompd\_enable@{\code{ompd\_enable}}}

\summary
Instructs the OpenMP runtime to collect whatever information
is necessary for supporting access by tools using OMPD.

\vbox{
\cspecificstart
\begin{boxedcode}
int ompd_enable ( void );
\end{boxedcode}
\cspecificend
}

\descr

In some cases it may not be possible to control an OpenMP program's
environment to set the \code{OMPD\_ENABLED} variable so as to enable
data collection by the runtime for OMPD.
\code{ompd\_enable} allows an OpenMP process itself to turn on data collection.
Upon starting, the OpenMP runtime will check to see if the function
\code{ompd\_enable} is defined in the OpenMP program or any of
its dynamically-linked libraries.
If it is defined, the OpenMP runtime will call the function,
and if logical true (non-zero) is returned, it will enable data
collection to support external tools.
The function may be positioned in an otherwise empty DLL that the
programmer can link with the OpenMP program, thereby leaving
the program code unmodified.

\crossreferences
\begin{itemize}
\item
  \code{OMPD\_ENABLED}, \specref{sec:OMPD_ENABLED}
\item
  Enabling the Runtime for OMPD, \specref{ompd:enabling-ompd}
\end{itemize}




\section{Tool Foundation}
\subsection{Data Types}
\subsubsection{Thread States}
\label{sec:thread-states}
\label{sec:omp_state_t}
\summary
If the OMPT interface is in state \plc{active}, an OpenMP implementation
must maintain \plc{thread state} information for each thread.
The thread state maintained is an approximation of the instantaneous
state of a thread.

\format
\begin{ccppspecific}
A thread state must be one of the values of the
enumeration type \code{omp_state_t} or
an implementation-defined state value of 512 or higher.

\begin{ompcEnum}
typedef enum omp_state_t {
  omp_state_work_serial                      = 0x000,
  omp_state_work_parallel                    = 0x001,
  omp_state_work_reduction                   = 0x002,

  omp_state_wait_barrier                     = 0x010,
  omp_state_wait_barrier_implicit_parallel   = 0x011,
  omp_state_wait_barrier_implicit_workshare  = 0x012,
  omp_state_wait_barrier_implicit            = 0x013,
  omp_state_wait_barrier_explicit            = 0x014,

  omp_state_wait_taskwait                    = 0x020,
  omp_state_wait_taskgroup                   = 0x021,

  omp_state_wait_mutex                       = 0x040,
  omp_state_wait_lock                        = 0x041,
  omp_state_wait_critical                    = 0x042,
  omp_state_wait_atomic                      = 0x043,
  omp_state_wait_ordered                     = 0x044,

  omp_state_wait_target                      = 0x080,
  omp_state_wait_target_map                  = 0x081,
  omp_state_wait_target_update               = 0x082,

  omp_state_idle                             = 0x100,
  omp_state_overhead                         = 0x101,
  omp_state_undefined                        = 0x102
} omp_state_t;
\end{ompcEnum}
\end{ccppspecific}

\descr

A tool can query the OpenMP state of a thread at any time.
If a tool queries the state of a thread that is not associated
with OpenMP, the implementation reports the state as \code{omp_state_undefined}.


The value \code{omp_state_work_serial} indicates that the thread is executing code 
outside all parallel regions.

The value \code{omp_state_work_parallel} indicates that the thread is executing code 
within the scope of a parallel region construct.

The value \code{omp_state_work_reduction} indicates that the thread is combining partial 
reduction results from threads in its team.
An OpenMP implementation might never report a thread in this state; a thread combining 
partial reduction results may have its state reported as \code{omp_state_work_parallel} 
or \code{omp_state_overhead}.

The value \code{omp_state_wait_barrier} indicates that the thread is waiting at either an 
implicit or explicit barrier.
An implementation may never report a thread in this state; instead, a thread may have its 
state reported as \code{omp_state_wait_barrier_implicit}  or 
\code{omp_state_wait_barrier_explicit}, as appropriate.

The value \code{omp_state_wait_barrier_implicit} indicates that the thread is waiting at 
an implicit barrier in a parallel region. An OpenMP implementation may report 
\code{omp_state_wait_barrier} for implicit barriers.

The value \code{omp_state_wait_barrier_implicit_parallel}  indicates that the thread is 
waiting at an implicit barrier at the end of a parallel region. An OpenMP implementation 
may report \code{omp_state_wait_barrier} or \code{omp_state_wait_barrier_implicit} for 
these barriers.

The value \code{omp_state_wait_barrier_implicit_workshare}  indicates that the thread is 
waiting at an implicit barrier at the end of a workshare-construct. An OpenMP 
implementation may report \code{omp_state_wait_barrier} or 
\code{omp_state_wait_barrier_implicit} for these barriers.

The value \code{omp_state_wait_barrier_explicit} indicates that the thread is waiting in 
a \code{barrier} region. An OpenMP implementation may report 
\code{omp_state_wait_barrier} for these barriers.

The value \code{omp_state_wait_taskwait} indicates that the thread is waiting at a 
taskwait construct. 

The value \code{omp_state_wait_taskgroup} indicates that the thread is waiting at the end 
of a taskgroup construct. 

The value \code{omp_state_wait_mutex} indicates that the thread is waiting for a mutex of 
an unspecified type. 

The value \code{omp_state_wait_lock} indicates that the thread is waiting for a  lock  or 
nest lock. 

The value \code{omp_state_wait_critical} indicates that the thread is waiting to enter a 
critical region. 

The value \code{omp_state_wait_atomic} indicates that the thread is waiting to enter an 
atomic region. 

The value \code{omp_state_wait_ordered} indicates that the thread is waiting to enter an 
ordered region. 


The value \code{omp_state_wait_target} indicates that the thread is waiting for a target 
region to complete.

The value \code{omp_state_wait_target_map} indicates that the  thread is waiting for a 
target data mapping operation to complete.
An implementation may report \code{omp_state_wait_target} for target data constructs.

The value \code{omp_state_wait_target_update} indicates that the thread is waiting for a 
target  update operation to complete.
An implementation may report \code{omp_state_wait_target} for target update constructs.

The value \code{omp_state_idle} indicates that the thread is idle, that is not part of an 
OpenMP team.

The value \code{omp_state_overhead} indicates that the thread is in the overhead state at 
any point while executing within an OpenMP runtime, except while waiting indefinitely at 
a synchronization point.

The value \code{omp_state_undefined} indicates that the native thread is not created by 
the OpenMP implementation.


\subsubsection{Frames}
\index{frames}
\label{sec:omp_frame_t}

\vbox{
\begin{ccppspecific}
\begin{boxedcode}
typedef struct omp_frame_t \{\\
  void *\plc{exit_frame};
  void *\plc{enter_frame};
\} omp_frame_t;
\end{boxedcode}
\end{ccppspecific}
}

\descr

When executing an OpenMP program, at times, one or more procedure frames associated with
the OpenMP runtime may appear on a thread's stack between frames
associated with tasks. To help a tool determine whether a procedure
frame on the call stack belongs to a task or not,
for each task whose frames appear on the stack, the runtime
maintains an \code{omp\_frame\_t} object 
that indicates a contiguous sequence of 
procedure frames associated with the task.
Each \code{omp\_frame\_t} object is associated with the task to which the procedure frames belong.
Each non-merged initial, implicit, explicit, or target task with one or more frames on a thread's stack
will have an associated \code{omp\_frame\_t} object.



An \code{omp\_frame\_t} object associated with a task contains a pair
of pointers: \plc{exit\_frame} and \plc{enter\_frame}. The field names were
chosen, respectively, to reflect that they typically contain a pointer to a procedure frame on the stack when 
\emph{exiting} the OpenMP runtime into code for a task or \emph{entering} the OpenMP runtime from a task.

The \plc{exit\_frame} field of a task's \code{omp\_frame\_t} object 
contains the canonical frame address for the procedure frame that
transfers control to the structured block for the task. 
The value of \plc{exit\_frame} is \code{NULL} until just prior to
beginning execution of the structured block for the task.
A task's \plc{exit\_frame} may point to a procedure frame that belongs
to the OpenMP runtime or one that belongs to another task.
The \plc{exit\_frame} for the \code{omp\_frame\_t} object associated 
with an \emph{initial task} is \code{NULL}.

The \plc{enter\_frame} field of a task's \code{omp\_frame\_t} object 
contains the canonical frame address of a task procedure frame that invoked the
OpenMP runtime causing the current task to suspend and another task to
execute.
If a task with frames on the stack has not suspended, the value of
\plc{enter\_frame} for the \code{omp\_frame\_t} object 
associated with the task may contain \code{NULL}. 
The value of \plc{enter\_frame} in a task's \code{omp\_frame\_t} is
reset to \code{NULL} just before a suspended task resumes execution.

An \code{omp\_frame\_t}'s lifetime begins when a task is created
and ends when the task is destroyed. Tools should not assume that
a frame structure remains at a constant location in memory throughout
a task's lifetime. A pointer to a task's \code{omp\_frame\_t} object is passed to
some callbacks; a pointer to a task's \code{omp\_frame\_t} object 
can also be retrieved by a tool at any time, including in a signal
handler, by invoking the
\code{ompt\_get\_task\_info} runtime entry point (described in
Section~\ref{sec:ompt_get_task_info}).





\begin{table}
\begin{center}
\caption{Meaning of various states of an \code{omp\_frame\_t}
    object.\label{tab:frame}}
\begin{tabular}{|p{1in}||p{2in}|p{2in}|}
\hline
\plc{exit\_frame} / \plc{enter\_frame} 	state & \plc{enter\_frame} is
\code{NULL}
& \plc{enter\_frame} is non-\code{NULL} \\
\hline
\hline
\plc{exit\_frame} is \code{NULL} & 
case 1)  initial task during execution\newline 
case 2) task that is created but not yet scheduled or already finished & 
initial task suspended while another task executes
\\\hline
\plc{exit\_frame} is non-\code{NULL} 	& non-initial task that has
been scheduled &
non-initial task 
suspended while another task executes
\\\hline
\end{tabular}
\vspace{1ex}
\end{center}
\end{table}

Table~\ref{tab:frame} describes various states in which 
an \code{omp\_frame\_t} object may be observed and their meaning.
In the presence of nested parallelism, a tool may
observe a sequence of \code{omp\_frame\_t} objects for a thread.
Appendix~\ref{chap:frames} illustrates 
use of \code{omp\_frame\_t} objects with nested parallelism.

\needspace{6\baselineskip}\begin{note}
A monitoring tool using asynchronous sampling can observe values
of \plc{exit\_frame} and \plc{enter\_frame} at inconvenient times.
Tools must be prepared to observe and handle \code{omp\_frame\_t}
objects observed just prior to when their field values will be set or
cleared.
\end{note}

\subsection{\code{ompt_wait_id_t}}
\label{sec:ompt_wait_id_t}
\index{wait identifier}

\summary
The \code{ompt_wait_id_t} type describes wait identifier for an OpenMP thread.

\format
\begin{ccppspecific}
\begin{omptOther}
typedef uint64_t ompt_wait_id_t;
\end{omptOther}
\end{ccppspecific}

\descr

Each thread instance maintains a \emph{wait identifier} of type 
\code{ompt_wait_id_t}. When a task executing on a thread is waiting 
for mutual exclusion, the thread's wait identifier indicates what the 
thread is awaiting. A wait identifier may represent a critical section 
{\em name}, a lock, a program variable accessed in an atomic region, or 
a synchronization object internal to an OpenMP implementation.

\code{ompt_wait_id_none} is defined as an instance of type 
\code{ompt_wait_id_t} with the value 0.

When a thread is not in a wait state, the value of the thread's wait 
identifier is undefined.


\subsection{Global Symbols}
Many of the interfaces between tools and an OpenMP implementation are invisible to users. This section describes
a few global symbols used by OMPT and OMPD tools to coordinate with an OpenMP implementation.
\ompttoolsignature{\code{ompt\_start\_tool}}
\label{sec:ompt_start_tool}

\summary
If a tool wants to use the OMPT interface provided by an OpenMP implementation,
the tool must implement the function \code{ompt\_start\_tool} to announce its interest.

\format
\vbox{
\begin{cspecific}
\begin{boxedcode}
ompt_start_tool_result_t *ompt_start_tool(
  unsigned int \plc{omp_version},
  const char *\plc{runtime_version}
);
\end{boxedcode}
\end{cspecific}
}

\descr
For a tool to use the OMPT interface provided by an OpenMP implementation,
the tool must define a globally-visible implementation of the
function \code{ompt\_start\_tool}.

A tool may indicate its intent to use the OMPT interface provided
by an OpenMP implementation by having
\code{ompt\_start\_tool} return a non-\code{NULL} pointer to an
\code{ompt\_start\_tool\_result\_t} structure, which contains pointers to
tool initialization and finalization callbacks along with
a tool data word that an OpenMP implementation must pass by reference
to these callbacks.

A tool may use its \callbackarg{} \plc{omp\_version} to determine
whether it is compatible with the OMPT interface provided by an OpenMP
implementation.

If a tool implements \code{ompt\_start\_tool} but has no interest in using
the OMPT interface in a particular execution,
\code{ompt\_start\_tool} should return \code{NULL}. 

\argdesc

The \callbackarg{} \plc{omp\_version} 
is the value of the \code{\_OPENMP} version macro 
associated with the OpenMP API implementation. This value 
identifies the OpenMP API version supported by an OpenMP implementation,
which specifies the version of the OMPT interface that it supports.

The \callbackarg{} \plc{runtime\_version}
is a version string that unambiguously identifies the OpenMP implementation.

\constraints

The \callbackarg{} \plc{runtime\_version} must be
an immutable string that is defined for the lifetime of a program
execution.

\effect
If a tool returns a non-\code{NULL} pointer to an
\code{ompt\_start\_tool\_result\_t} structure,
an OpenMP implementation will call the tool initializer specified by the
\plc{initialize} field in this structure before
beginning execution of any OpenMP construct
or completing execution of any environment routine invocation; the
OpenMP implementation will call the tool finalizer specified by the
\plc{finalize} field in this structure when the OpenMP
implementation shuts down.



\crossreferences
\begin{itemize}
    \item \code{ompt\_start\_tool\_result\_t}, see 
     \specref{sec:ompt_start_tool_result_t}.
\end{itemize}


\subsubsection{\hcode{ompd_dll_locations}}
\label{subsubsec:ompd_dll_locations}
\index{ompd_dll_locations@{\code{ompd_dll_locations}}}

\summary
The \code{ompd_dll_locations} global variable indicates the location
of OMPD libraries that are compatible with the OpenMP implementation.

\format
\begin{cspecific}
\begin{ompSyntax}
const char **ompd_dll_locations;
\end{ompSyntax}
\end{cspecific}

\descr
An OpenMP runtime may have more than one OMPD libary. The tool must be able 
to locate the right library to use for the OpenMP program that it is examining.
The OpenMP runtime system must provide a public variable \code{ompd_dll_locations},
which is an \code{argv}-style vector of filename string pointers that provides 
the name(s) of any compatible OMPD library. This variable must have \code{C} 
linkage. The tool uses the name of the variable verbatim and, in particular, 
does not apply any name mangling before performing the look up.

The programming model or architecture of the tool and, thus, that of OMPD does 
not have to match that of the OpenMP program that is being examined. The tool
must interpret the contents of \code{ompd_dll_locations} to find a suitable OMPD 
that matches its own architectural characteristics. On platforms that support 
different programming models (for example, 32-bit vs 64-bit), OpenMP 
implementations are encouraged to provide OMPD libraries for all models, and 
that can handle OpenMP programs of any model. Thus, for example, a 32-bit 
debugger that uses OMPD should be able to debug a 64-bit OpenMP program by 
loading a 32-bit OMPD implementation that can manage a 64-bit OpenMP runtime.

\code{ompd_dll_locations} points to a NULL-terminated vector of zero or more 
NULL-terminated pathname strings that do not have any filename conventions. 
This vector must be fully initialized \emph{before} \code{ompd_dll_locations} 
is set to a non-null value, such that if a tool, such as a debugger, stops 
execution of the OpenMP program at any point at which \code{ompd_dll_locations} 
is non-null, then the vector of strings to which it points is valid and complete.

\crossreferences
\begin{itemize}
\item   \code{ompd_dll_locations_valid}, 
see \specref{subsubsec:ompd_dll_locations_valid}
\end{itemize}



\subsubsection{\hcode{ompd_dll_locations_valid}}
\label{subsubsec:ompd_dll_locations_valid}
\index{ompd_dll_locations@{\code{ompd_dll_locations_valid}}}

\summary
The OpenMP runtime notifies third-party tools that \code{ompd_dll_locations}
is valid by allowing execution to pass through a location identified
by the symbol \code{ompd_dll_locations_valid}.


\begin{cspecific}
\begin{ompSyntax}
void ompd_dll_locations_valid(void);
\end{ompSyntax}
\end{cspecific}


\descr
Depending on how the OpenMP runtime is
implemented, \code{ompd_dll_locations} might not be a static
variable, and therefore needs to be initialized at runtime.  The
OpenMP runtime notifies third-party tools
that \code{ompd_dll_locations} is valid by having execution pass
through a location identified by the
symbol \code{ompd_dll_locations_valid}.
If \code{ompd_dll_locations} is NULL, a third-party tool, e.g., a
debugger can place a breakpoint at \code{ompd_dll_locations_valid}
to be notified when \code{ompd_dll_locations} has been initialized.
In practice, the symbol \code{ompd_dll_locations_valid} need not be
a function; instead, it may be a labeled machine instruction through
which execution passes once the vector is valid.




    % This is environment_variables.tex (Chapter 5) of the OpenMP specification.
% This is an included file. See the master file for more information.
%
% When editing this file:
%
%    1. To change formatting, appearance, or style, please edit openmp.sty.
%
%    2. Custom commands and macros are defined in openmp.sty.
%
%    3. Be kind to other editors -- keep a consistent style by copying-and-pasting to
%       create new content.
%
%    4. We use semantic markup, e.g. (see openmp.sty for a full list):
%         \code{}     % for bold monospace keywords, code, operators, etc.
%         \plc{}      % for italic placeholder names, grammar, etc.
%
%    5. There are environments that provide special formatting, e.g. language bars.
%       Please use them whereever appropriate.  Examples are:
%
%         \begin{fortranspecific}
%         This is text that appears enclosed in blue language bars for Fortran.
%         \end{fortranspecific}
%
%         \begin{note}
%         This is a note.  The "Note -- " header appears automatically.
%         \end{note}
%
%    6. Other recommendations:
%         Use the convenience macros defined in openmp.sty for the minor headers
%         such as Comments, Syntax, etc.
%
%         To keep items together on the same page, prefer the use of
%         \begin{samepage}.... Avoid \parbox for text blocks as it interrupts line numbering.
%         When possible, avoid \filbreak, \pagebreak, \newpage, \clearpage unless that's
%         what you mean. Use \needspace{} cautiously for troublesome paragraphs.
%
%         Avoid absolute lengths and measures in this file; use relative units when possible.
%         Vertical space can be relative to \baselineskip or ex units. Horizontal space
%         can be relative to \linewidth or em units.
%
%         Prefer \emph{} to italicize terminology, e.g.:
%             This is a \emph{definition}, not a placeholder.
%             This is a \plc{var-name}.
%

\chapter{Environment Variables}
\index{environment variables}
\index{variables, environment}
\label{chap:Environment Variables}
This chapter describes the OpenMP environment variables that specify the settings of
the ICVs that affect the execution of OpenMP programs (see \specref{sec:Internal Control Variables}).
The names of the environment variables must be upper case. The values assigned to the
environment variables are case insensitive and may have leading and trailing white
space. Modifications to the environment variables after the program has started, even if
modified by the program itself, are ignored by the OpenMP implementation. However,
the settings of some of the ICVs can be modified during the execution of the OpenMP
program by the use of the appropriate directive clauses or OpenMP API routines.

The following examples demonstrate how the OpenMP environment variables can
be set in different environments:

\begin{itemize}
\item csh-like shells:
\end{itemize}

\begin{ompEnv}
setenv OMP_SCHEDULE "dynamic"
\end{ompEnv}

\begin{itemize}
\item bash-like shells:
\end{itemize}

\begin{ompEnv}
export OMP_SCHEDULE="dynamic"
\end{ompEnv}

\begin{itemize}
\item Windows Command Line:
\end{itemize}

\begin{ompEnv}
set OMP_SCHEDULE=dynamic
\end{ompEnv}


\section{\hcode{OMP_SCHEDULE}}
\index{OMP_SCHEDULE@{\code{OMP_SCHEDULE}}}
\index{environment variables!OMP_SCHEDULE@{\code{OMP_SCHEDULE}}}
\label{sec:OMP_SCHEDULE}
The \code{OMP_SCHEDULE} environment variable controls the schedule kind and chunk size
of all loop directives that have the schedule kind \code{runtime}, by setting the value of the
\plc{run-sched-var} ICV.

The value of this environment variable takes the form:

\plc{[}\plc{modifier}:\plc{]}\plc{kind}\plc{[},\plc{ chunk}\plc{]}

where

\begin{itemize}
\item \plc{modifier} is one of \code{monotonic} or \code{nonmonotonic};

\item \plc{kind} is one of \code{static}, \code{dynamic}, \code{guided}, or \code{auto};

\item \plc{chunk} is an optional positive integer that specifies the chunk size.
\end{itemize}

If the \plc{modifier} is not present, the \plc{modifier} is set to \code{monotonic} if \plc{kind} is \code{static}; for any other \plc{kind} it is set to \code{nonmonotonic}.

If \plc{chunk} is present, there may be white space on either side of the ``\code{,}''. See
\specref{subsec:Worksharing-Loop Construct} for a detailed description of the schedule kinds.

The behavior of the program is implementation defined if the value of \code{OMP_SCHEDULE}
does not conform to the above format.

Implementation specific schedules cannot be specified in \code{OMP_SCHEDULE}. They can
only be specified by calling \code{omp_set_schedule}, described in \specref{subsec:omp_set_schedule}.

Examples:

\begin{ompEnv}
setenv OMP_SCHEDULE "guided,4"
setenv OMP_SCHEDULE "dynamic"
setenv OMP_SCHEDULE "nonmonotonic:dynamic,4"
\end{ompEnv}

\crossreferences
\begin{itemize}
\item \plc{run-sched-var} ICV, see \specref{sec:Internal Control Variables}.

\item Worksharing-Loop construct, see \specref{subsec:Worksharing-Loop Construct}.

\item Parallel worksharing-loop construct, see \specref{subsec:Parallel Worksharing-Loop Construct}.

\item \code{omp_set_schedule} routine, see \specref{subsec:omp_set_schedule}.

\item \code{omp_get_schedule} routine, see \specref{subsec:omp_get_schedule}.
\end{itemize}









\section{\hcode{OMP_NUM_THREADS}}
\index{OMP_NUM_THREADS@{\code{OMP_NUM_THREADS}}}
\index{environment variables!OMP_NUM_THREADS@{\code{OMP_NUM_THREADS}}}
\label{sec:OMP_NUM_THREADS}
The \code{OMP_NUM_THREADS} environment variable sets the number of threads to use for
\code{parallel} regions by setting the initial value of the \plc{nthreads-var} ICV. See
\specref{sec:Internal Control Variables} for a comprehensive set of rules about the interaction between the
\code{OMP_NUM_THREADS} environment variable, the \code{num_threads} clause, the
\code{omp_set_num_threads} library routine and dynamic adjustment of threads, and
\specref{subsec:Determining the Number of Threads for a parallel Region}
for a complete algorithm that describes how the number of
threads for a \code{parallel} region is determined.

The value of this environment variable must be a list of positive integer values. The
values of the list set the number of threads to use for \code{parallel} regions at the
corresponding nested levels.

The behavior of the program is implementation defined if any value of the list specified
in the \code{OMP_NUM_THREADS} environment variable leads to a number of threads which is
greater than an implementation can support, or if any value is not a positive integer.

Example:
\begin{ompEnv}
setenv OMP_NUM_THREADS 4,3,2
\end{ompEnv}

\crossreferences
\begin{itemize}
\item \plc{nthreads-var} ICV, see \specref{sec:Internal Control Variables}.

\item \code{num_threads} clause, \specref{sec:parallel Construct}.

\item \code{omp_set_num_threads} routine, see \specref{subsec:omp_set_num_threads}.

\item \code{omp_get_num_threads} routine, see \specref{subsec:omp_get_num_threads}.

\item \code{omp_get_max_threads} routine, see \specref{subsec:omp_get_max_threads}.

\item \code{omp_get_team_size} routine, see \specref{subsec:omp_get_team_size}.
\end{itemize}







\section{\hcode{OMP_DYNAMIC}}
\index{OMP_DYNAMIC@{\code{OMP_DYNAMIC}}}
\index{environment variables!OMP_DYNAMIC@{\code{OMP_DYNAMIC}}}
\label{sec:OMP_DYNAMIC}
The \code{OMP_DYNAMIC} environment variable controls dynamic adjustment of the number
of threads to use for executing \code{parallel} regions by setting the initial value of the
\plc{dyn-var} ICV. The value of this environment variable must be \code{true} or \code{false}. If the
environment variable is set to \code{true}, the OpenMP implementation may adjust the
number of threads to use for executing \code{parallel} regions in order to optimize the use
of system resources. If the environment variable is set to \code{false}, the dynamic
adjustment of the number of threads is disabled. The behavior of the program is
implementation defined if the value of \code{OMP_DYNAMIC} is neither \code{true} nor \code{false}.

Example:
\begin{ompEnv}
setenv OMP_DYNAMIC true
\end{ompEnv}

\crossreferences
\begin{itemize}
\item \plc{dyn-var} ICV, see \specref{sec:Internal Control Variables}.

\item \code{omp_set_dynamic} routine, see \specref{subsec:omp_set_dynamic}.

\item \code{omp_get_dynamic} routine, see \specref{subsec:omp_get_dynamic}.
\end{itemize}









\section{\hcode{OMP_PROC_BIND}}
\index{OMP_PROC_BIND@{\code{OMP_PROC_BIND}}}
\index{environment variables!OMP_PROC_BIND@{\code{OMP_PROC_BIND}}}
\label{sec:OMP_PROC_BIND}
The \code{OMP_PROC_BIND} environment variable sets the initial value of the \plc{bind-var} ICV.
The value of this environment variable is either \code{true}, \code{false}, or a comma separated
list of \code{master}, \code{close}, or \code{spread}. The values of the list set the thread affinity policy
to be used for parallel regions at the corresponding nested level.

If the environment variable is set to \code{false}, the execution environment may move
OpenMP threads between OpenMP places, thread affinity is disabled, and \code{proc_bind}
clauses on \code{parallel} constructs are ignored.

Otherwise, the execution environment should not move OpenMP threads between
OpenMP places, thread affinity is enabled, and the initial thread is bound to the first
place in the OpenMP place list prior to the first active parallel region.

The behavior of the program is implementation defined if the value in the
\code{OMP_PROC_BIND} environment variable is not \code{true}, \code{false}, or a comma separated
list of \code{master}, \code{close}, or \code{spread}. The behavior is also implementation defined if an
initial thread cannot be bound to the first place in the OpenMP place list.

\pagebreak

Examples:
\begin{ompEnv}
setenv OMP_PROC_BIND false
setenv OMP_PROC_BIND "spread, spread, close"
\end{ompEnv}


\crossreferences
\begin{itemize}
\item \plc{bind-var} ICV, see \specref{sec:Internal Control Variables}.

\item \code{proc_bind} clause, see \specref{subsec:Controlling OpenMP Thread Affinity}.

\item \code{omp_get_proc_bind} routine, see \specref{subsec:omp_get_proc_bind}.
\end{itemize}









\section{\hcode{OMP_PLACES}}
\index{OMP_PLACES@{\code{OMP_PLACES}}}
\index{environment variables!OMP_PLACES@{\code{OMP_PLACES}}}
\label{sec:OMP_PLACES}
A list of places can be specified in the \code{OMP_PLACES} environment variable. The
\plc{place-partition-var} ICV obtains its initial value from the \code{OMP_PLACES} value, and makes the
list available to the execution environment. The value of \code{OMP_PLACES} can be one of
two types of values: either an abstract name describing a set of places or an explicit list
of places described by non-negative numbers.

The \code{OMP_PLACES} environment variable can be defined using an explicit ordered list of
comma-separated places. A place is defined by an unordered set of comma-separated
non-negative numbers enclosed by braces. The meaning of the numbers and how the
numbering is done are implementation defined. Generally, the numbers represent the
smallest unit of execution exposed by the execution environment, typically a hardware
thread.

Intervals may also be used to define places. Intervals can be specified using the
<\plc{lower-bound}> : <\plc{length}> : <\plc{stride}> notation to represent the following list of numbers:
``<\plc{lower-bound}>, <\plc{lower-bound}> + <\plc{stride}>, ..., <\plc{lower-bound}> + (<\plc{length}>-
1)*<\plc{stride}>.'' When <\plc{stride}> is omitted, a unit stride is assumed. Intervals can specify
numbers within a place as well as sequences of places.

An exclusion operator ``\code{!}'' can also be used to exclude the number or place immediately
following the operator.

Alternatively, the abstract names listed in
Table~\ref{tab:Defined Abstract Names for OMP PLACES} should be understood
by the execution and runtime environment. The precise definitions of the
abstract names are implementation defined. An implementation may also add
abstract names as appropriate for the target platform.

The abstract name may be appended by a positive number in parentheses to denote the
length of the place list to be created, that is \plc{abstract_name(num-places)}. When
requesting fewer places than available on the system, the determination of which
resources of type \plc{abstract_name} are to be included in the place list is implementation
defined. When requesting more resources than available, the length of the place list is
implementation defined.

% Table

\medskip
\nolinenumbers
\renewcommand{\arraystretch}{1.5}
\tablefirsthead{%
\hline
\textsf{\textbf{Abstract Name}} & \textsf{\textbf{Meaning}}\\
\hline
}
\tablehead{%
\multicolumn{2}{l}{\small\slshape table continued from previous page}\\
\hline\\
\textsf{\textbf{Abstract Name}} & \textsf{\textbf{Meaning}}\\
\hline
}
\tabletail{%
\hline\\
\multicolumn{2}{l}{\small\slshape table continued on next page}\\
}
\tablelasttail{\hline}
\tablecaption{Defined Abstract Names for \code{OMP_PLACES}\label{tab:Defined Abstract Names for OMP PLACES}}
\begin{supertabular}{p{1.5in} p{3.5in}}
{\scode{threads}} & Each place corresponds to a single hardware thread on the target machine.\\
{\scode{cores}} & Each place corresponds to a single core (having one or more hardware
threads) on the target machine.\\
{\scode{sockets}} & Each place corresponds to a single socket (consisting of one or more cores)
on the target machine.\\
\end{supertabular}

\linenumbers

The behavior of the program is implementation defined when the execution environment
cannot map a numerical value (either explicitly defined or implicitly derived from an
interval) within the \code{OMP_PLACES} list to a processor on the target platform, or if it maps
to an unavailable processor. The behavior is also implementation defined when the
\code{OMP_PLACES} environment variable is defined using an abstract name.

The following grammar describes the values accepted for the \code{OMP_PLACES} environment
variable.

\begin{bnf*}
   \bnfprod{list}{
      \bnfpn{p-list} \bnfor
      \bnfpn{aname}
      } \\
   \bnfprod{p-list}{
      \bnfpn{p-interval} \bnfor \bnfpn{p-list} \bnfts{,} \bnfpn{p-interval}
      } \\
   \bnfprod{p-interval}{
      \bnfpn{place} \bnfts{:} \bnfpn{len} \bnfts{:} \bnfpn{stride} \bnfor
      \bnfpn{place} \bnfts{:} \bnfpn{len} \bnfor
      \bnfpn{place} \bnfor
      \bnfts{!} \bnfpn{place}
      } \\
   \bnfprod{place}{
      \bnfts{\{} \bnfpn{res-list} \bnfts{\}}
      } \\
   \bnfprod{res-list}{
      \bnfpn{res-interval} \bnfor
      \bnfpn{res-list} \bnfts{,} \bnfpn{res-interval}
      } \\
   \bnfprod{res-interval}{
      \bnfpn{res} \bnfts{:} \bnfpn{num-places} \bnfts{:} \bnfpn{stride} \bnfor
      \bnfpn{res} \bnfts{:} \bnfpn{num-places} \bnfor
      \bnfpn{res} \bnfor
      \bnfts{!} \bnfpn{res}
   } \\
   \bnfprod{aname}{
      \bnfpn{word} \bnfts{(} \bnfpn{num-places} \bnfts{)} \bnfor
      \bnfpn{word}
   } \\
   \bnfprod{word}{
      \bnfts{sockets} \bnfor
      \bnfts{cores} \bnfor
      \bnfts{threads} \bnfor
      \bnfts{<implementation-defined abstract name>}
   } \\
   \bnfprod{res}{
      \bnftd{non-negative integer}
   } \\
   \bnfprod{num-places}{
      \bnftd{positive integer}
   } \\
   \bnfprod{stride}{
      \bnftd{integer}
   } \\
   \bnfprod{len}{
      \bnftd{positive integer}
   } \\
\end{bnf*}

\pagebreak

\begin{samepage}
Examples:
\begin{ompEnv}
setenv OMP_PLACES threads
setenv OMP_PLACES "threads(4)"
setenv OMP_PLACES "{0,1,2,3},{4,5,6,7},{8,9,10,11},{12,13,14,15}"
setenv OMP_PLACES "{0:4},{4:4},{8:4},{12:4}"
setenv OMP_PLACES "{0:4}:4:4"
\end{ompEnv}
\end{samepage}

where each of the last three definitions corresponds to the same 4 places including the
smallest units of execution exposed by the execution environment numbered, in turn, 0
to 3, 4 to 7, 8 to 11, and 12 to 15.

\crossreferences
\begin{itemize}
\item \plc{place-partition-var}, \specref{sec:Internal Control Variables}.

\item Controlling OpenMP thread affinity, \specref{subsec:Controlling OpenMP Thread Affinity}.

\item \code{omp_get_num_places} routine, see \specref{subsec:omp_get_num_places}.

\item \code{omp_get_place_num_procs} routine, see \specref{subsec:omp_get_place_num_procs}.

\item \code{omp_get_place_proc_ids} routine, see \specref{subsec:omp_get_place_proc_ids}.

\item \code{omp_get_place_num} routine, see \specref{subsec:omp_get_place_num}.

\item \code{omp_get_partition_num_places} routine, see \specref{subsec:omp_get_partition_num_places}.

\item \code{omp_get_partition_place_nums} routine, see \specref{subsec:omp_get_partition_place_nums}.
\end{itemize}










\section{\hcode{OMP_NESTED}}
\index{OMP_NESTED@{\code{OMP_NESTED}}}
\index{environment variables!OMP_NESTED@{\code{OMP_NESTED}}}
\label{sec:OMP_NESTED}
The deprecated \code{OMP_NESTED} environment variable controls nested parallelism by setting the
initial value of the \plc{nest-var} ICV. The value of this environment variable must be \code{true}
or \code{false}. If the environment variable is set to \code{true}, nested parallelism is enabled; if
set to \code{false}, nested parallelism is disabled. The behavior of the program is
implementation defined if the value of \code{OMP_NESTED} is neither \code{true} nor \code{false}.

Example:
\begin{ompEnv}
setenv OMP_NESTED false
\end{ompEnv}

\crossreferences
\begin{itemize}
\item \plc{nest-var} ICV, see \specref{sec:Internal Control Variables}.

\item \code{omp_set_nested} routine, see \specref{subsec:omp_set_nested}.

\item \code{omp_get_team_size} routine, see \specref{subsec:omp_get_team_size}.
\end{itemize}











\section{\hcode{OMP_STACKSIZE}}
\index{OMP_STACKSIZE@{\code{OMP_STACKSIZE}}}
\index{environment variables!OMP_STACKSIZE@{\code{OMP_STACKSIZE}}}
\label{sec:OMP_STACKSIZE}
The \code{OMP_STACKSIZE} environment variable controls the size of the stack for threads
created by the OpenMP implementation, by setting the value of the \plc{stacksize-var} ICV.
The environment variable does not control the size of the stack for an initial thread.

The value of this environment variable takes the form:

\plc{size} | \plc{size}\code{B} | \plc{size}\code{K} | \plc{size}\code{M} | \plc{size}\code{G}

where:

\begin{itemize}
\item \plc{size} is a positive integer that specifies the size of the stack for threads that are created
by the OpenMP implementation.

\item \code{B}, \code{K}, \code{M}, and \code{G} are letters that specify
whether the given size is in Bytes, Kilobytes
(1024 Bytes), Megabytes (1024 Kilobytes), or Gigabytes (1024 Megabytes),
respectively. If one of these letters is present, there may be white space between
\plc{size} and the letter.
\end{itemize}

If only \plc{size} is specified and none of \code{B}, \code{K}, \code{M}, or \code{G}
is specified, then \plc{size} is assumed to be in Kilobytes.

The behavior of the program is implementation defined if \code{OMP_STACKSIZE} does not
conform to the above format, or if the implementation cannot provide a stack with the
requested size.

Examples:
\begin{ompEnv}
setenv OMP_STACKSIZE 2000500B
setenv OMP_STACKSIZE "3000 k "
setenv OMP_STACKSIZE 10M
setenv OMP_STACKSIZE " 10 M "
setenv OMP_STACKSIZE "20 m "
setenv OMP_STACKSIZE " 1G"
setenv OMP_STACKSIZE 20000
\end{ompEnv}

\crossreferences
\begin{itemize}
\item \plc{stacksize-var} ICV, see \specref{sec:Internal Control Variables}.
\end{itemize}









\section{\hcode{OMP_WAIT_POLICY}}
\index{OMP_WAIT_POLICY@{\code{OMP_WAIT_POLICY}}}
\index{environment variables!OMP_WAIT_POLICY@{\code{OMP_WAIT_POLICY}}}
\label{sec:OMP_WAIT_POLICY}
The \code{OMP_WAIT_POLICY} environment variable provides a hint to an OpenMP
implementation about the desired behavior of waiting threads by setting the
\plc{wait-policy-var} ICV. A compliant OpenMP implementation may or may not abide by the setting of
the environment variable.

The value of this environment variable takes the form:

{\code{ACTIVE }|\code{ PASSIVE}}

The \code{ACTIVE} value specifies that waiting threads should mostly be active, consuming
processor cycles, while waiting. An OpenMP implementation may, for example, make
waiting threads spin.

The \code{PASSIVE} value specifies that waiting threads should mostly be passive, not
consuming processor cycles, while waiting. For example, an OpenMP implementation
may make waiting threads yield the processor to other threads or go to sleep.

The details of the \code{ACTIVE} and \code{PASSIVE} behaviors are implementation defined.

Examples:
\begin{ompEnv}
setenv OMP_WAIT_POLICY ACTIVE
setenv OMP_WAIT_POLICY active
setenv OMP_WAIT_POLICY PASSIVE
setenv OMP_WAIT_POLICY passive
\end{ompEnv}

\crossreferences
\begin{itemize}
\item \plc{wait-policy-var} ICV, see \specref{sec:Internal Control Variables}.
\end{itemize}










\section{\hcode{OMP_MAX_ACTIVE_LEVELS}}
\index{OMP_MAX_ACTIVE_LEVELS@{\code{OMP_MAX_ACTIVE_LEVELS}}}
\index{environment variables!OMP_MAX_ACTIVE_LEVELS@{\code{OMP_MAX_ACTIVE_LEVELS}}}
\label{sec:OMP_MAX_ACTIVE_LEVELS}
The \code{OMP_MAX_ACTIVE_LEVELS} environment variable controls the maximum number
of nested active \code{parallel} regions by setting the initial value of the \plc{max-active-levels-var} ICV.

The value of this environment variable must be a non-negative integer. The behavior of
the program is implementation defined if the requested value of
\code{OMP_MAX_ACTIVE_LEVELS} is greater than the maximum number of nested active
parallel levels an implementation can support, or if the value is not a non-negative
integer.

\crossreferences
\begin{itemize}
\item \plc{max-active-levels-var} ICV, see \specref{sec:Internal Control Variables}.

\item \code{omp_set_max_active_levels} routine, see \specref{subsec:omp_set_max_active_levels}.

\item \code{omp_get_max_active_levels} routine, see \specref{subsec:omp_get_max_active_levels}.
\end{itemize}










\section{\hcode{OMP_THREAD_LIMIT}}
\index{OMP_THREAD_LIMIT@{\code{OMP_THREAD_LIMIT}}}
\index{environment variables!OMP_THREAD_LIMIT@{\code{OMP_THREAD_LIMIT}}}
\label{sec:OMP_THREAD_LIMIT}
The \code{OMP_THREAD_LIMIT} environment variable sets the maximum number of OpenMP threads to use in a contention group by setting the \plc{thread-limit-var} ICV.

The value of this environment variable must be a positive integer. The behavior of the
program is implementation defined if the requested value of \code{OMP_THREAD_LIMIT} is
greater than the number of threads an implementation can support, or if the value is not
a positive integer.

\crossreferences
\begin{itemize}
\item \plc{thread-limit-var} ICV, see \specref{sec:Internal Control Variables}.

\item \code{omp_get_thread_limit} routine, see \specref{subsec:omp_get_thread_limit}.
\end{itemize}









\section{\hcode{OMP_CANCELLATION}}
\index{OMP_CANCELLATION@{\code{OMP_CANCELLATION}}}
\index{environment variables!OMP_CANCELLATION@{\code{OMP_CANCELLATION}}}
\label{sec:OMP_CANCELLATION}
The \code{OMP_CANCELLATION} environment variable sets the initial value of the \plc{cancel-var}
ICV.

The value of this environment variable must be \code{true} or \code{false}. If set to \code{true}, the
effects of the \code{cancel} construct and of cancellation points are enabled and cancellation
is activated. If set to \code{false}, cancellation is disabled and the \code{cancel} construct and
cancellation points are effectively ignored.

\crossreferences
\begin{itemize}
\item \plc{cancel-var}, see \specref{subsec:ICV Descriptions}.

\item \code{cancel} construct, see \specref{subsec:cancel Construct}.

\item \code{cancellation point} construct, see \specref{subsec:cancellation point Construct}.

\item \code{omp_get_cancellation} routine, see \specref{subsec:omp_get_cancellation}.
\end{itemize}









\section{\hcode{OMP_DISPLAY_ENV}}
\index{OMP_DISPLAY_ENV@{\code{OMP_DISPLAY_ENV}}}
\index{environment variables!OMP_DISPLAY_ENV@{\code{OMP_DISPLAY_ENV}}}
\index{_OPENMP@{\code{_OPENMP} macro}}
\label{sec:OMP_DISPLAY_ENV}
The \code{OMP_DISPLAY_ENV} environment variable instructs the runtime to display the
OpenMP version number and the value of the ICVs associated with the environment
variables described in Chapter \ref{chap:Environment Variables},
as \plc{name} = \plc{value} pairs. The runtime displays this
information once, after processing the environment variables and before any user calls
to change the ICV values by runtime routines defined in Chapter \ref{chap:Runtime Library Routines}.

The value of the \code{OMP_DISPLAY_ENV} environment variable may be set to one of these
values:

{\code{TRUE }|\code{ FALSE }|\code{ VERBOSE}}

The \code{TRUE} value instructs the runtime to display the OpenMP version number defined by
the \code{_OPENMP} version macro (or the \code{openmp_version} Fortran parameter) value and
the initial ICV values for the environment variables listed in
Chapter \ref{chap:Environment Variables}. The \code{VERBOSE}
value indicates that the runtime may also display the values
of runtime variables that may be modified by vendor-specific
environment variables. The runtime does not display any information
when the \code{OMP_DISPLAY_ENV} environment variable is
\code{FALSE} or undefined. For all values of the environment
variable other than \code{TRUE}, \code{FALSE}, and \code{VERBOSE},
the displayed information is unspecified.

The display begins with \texttt{"OPENMP DISPLAY ENVIRONMENT BEGIN"}, followed by
the \code{_OPENMP} version macro (or the \code{openmp_version} Fortran parameter) value and
ICV values, in the format \plc{NAME} '=' \plc{VALUE}. \plc{NAME} corresponds to the macro or
environment variable name, optionally prepended by a bracketed \plc{device-type}. \plc{VALUE}
corresponds to the value of the macro or ICV associated with this environment variable.
Values should be enclosed in single quotes. The display is terminated with
\texttt{"OPENMP DISPLAY ENVIRONMENT END"}.

Example:
\begin{ompEnv}
% setenv OMP_DISPLAY_ENV TRUE
\end{ompEnv}

The above example causes an OpenMP implementation to generate output of the
following form:

\begin{ompEnv}
OPENMP DISPLAY ENVIRONMENT BEGIN
  _OPENMP='201811'
  [host] OMP_SCHEDULE='GUIDED,4'
  [host] OMP_NUM_THREADS='4,3,2'
  [device] OMP_NUM_THREADS='2'
  [host,device] OMP_DYNAMIC='TRUE'
  [host] OMP_PLACES='{0:4},{4:4},{8:4},{12:4}'
  ...
OPENMP DISPLAY ENVIRONMENT END
\end{ompEnv}


\section{\hcode{OMP_DISPLAY_AFFINITY}}
\index{OMP_DISPLAY_AFFINITY@{\code{OMP_DISPLAY_AFFINITY}}}
\index{environment variables!OMP_DISPLAY_AFFINITY@{\code{OMP_DISPLAY_AFFINITY}}}
\index{_OPENMP@{\code{_OPENMP} macro}}
\label{sec:OMP_DISPLAY_AFFINITY}
The \code{OMP_DISPLAY_AFFINITY} environment variable instructs the runtime to
display formatted affinity information for all OpenMP threads in the parallel
region upon entering the first parallel region and when there is any change in
the information accessible by the format specifiers listed in table
\ref{tab:Available Field Types for Formatting OpenMP Thread Affinity Information}.
If there is a change of affinity of any thread in a parallel region, thread
affinity information for all threads in that region will be displayed.
There is no specific order in displaying thread affinity information for all
threads in the same parallel region.

The value of the \code{OMP_DISPLAY_AFFINITY} environment variable may be set to one of these
values:

{\code{TRUE }|\code{ FALSE}}

%The default is set to \code{FALSE}.

The \code{TRUE} value instructs the runtime to display the OpenMP thread affinity information, and uses the
format setting defined in the \plc{affinity-format-var} ICV.

The runtime does not display the OpenMP thread affinity information when the value of the \code{OMP_DISPLAY_AFFINITY}
environment variable is \code{FALSE} or undefined. For all values of the environment
variable other than \code{TRUE} or \code{FALSE}, the display action is implementation defined.

Example:
\begin{ompEnv}
setenv OMP_DISPLAY_AFFINITY TRUE
\end{ompEnv}

The above example causes an OpenMP implementation to display OpenMP thread affinity information during execution of
the program, in a format given by the \plc{affinity-format-var} ICV.  The following is a sample output:
\begin{ompSyntax}
thread_level=   1,   thread_id=   0,   thread_affinity=    0,1
thread_level=   1,   thread_id=   1,   thread_affinity=    2,3
\end{ompSyntax}

\crossreferences
\begin{itemize}

\item Controlling OpenMP thread affinity, see
\specref{subsec:Controlling OpenMP Thread Affinity}.
\item \code{omp_set_affinity_format} routine, see \specref{subsec:omp_set_affinity_format}.
\item \code{omp_get_affinity_format} routine, see \specref{subsec:omp_get_affinity_format}.
\item \code{omp_display_affinity} routine, see \specref{subsec:omp_display_affinity}.
\item \code{omp_capture_affinity} routine, see \specref{subsec:omp_capture_affinity}.
\item \code{OMP_AFFINITY_FORMAT} environment variable, see
\specref{sec:OMP_AFFINITY_FORMAT}.
\end{itemize}


\section{\hcode{OMP_AFFINITY_FORMAT}}
\index{OMP_AFFINITY_FORMAT@{\code{OMP_AFFINITY_FORMAT}}}
\index{environment variables!OMP_AFFINITY_FORMAT@{\code{OMP_AFFINITY_FORMAT}}}
\index{_OPENMP@{\code{_OPENMP} macro}}
\label{sec:OMP_AFFINITY_FORMAT}

The \code{OMP_AFFINITY_FORMAT} environment variable sets the initial value of the
\plc{affinity-format-var} ICV which defines the format when displaying OpenMP
thread affinity information.

%The default value for \code{OMP_AFFINITY_FORMAT} is implementation defined.

The value of this environment variable is a character string that may contain as
substrings one or more field specifiers, in addition to other characters.
The format of each field specifier is

\begin{ompSyntax}
%\plc{[[[}0\plc{]}.\plc{] size ] type}
\end{ompSyntax}

where an individual field specifier must contain the percent symbol ({\pcode{\%}}) and a type.
The type can be a single character short name or its corresponding long name delimited with curly braces,
 such as {\pcode{\%n}} or {\pcode{\%\{thread\_num\}}}.
%(The "0", "." and \plc{size} are optional.)
A literal percent is specified as {\pcode{\%\%}}.  Field specifiers can be provided in any order.

The \code{0} modifier indicates whether or not to add leading zeros to the output,
following any indication of sign or base.
The \code{.} modifier indicates the output should be right justified when \plc{size} is specified.
By default, output is left justified.
The minimum field length is \plc{size}, which is decimal digit string
with non-zero first digit.
If no \plc{size} is specified, the actual length needed to print the field will be used.
If the \code{0} modifier is used with \plc{type} of \code{a}, {\pcode{\{thread\_affinity\}}},
\code{h}, {\pcode{\{host\}}}, or a type that is not printed as a number, the
result is unspecified.

Any other characters in the format string that are not part of a field specifier will be included literally
in the output.

Available field types are:

% Table
\nolinenumbers
\renewcommand{\arraystretch}{1.5}
\tablefirsthead{%
\hline
\textsf{\textbf{Short Name}} & \textsf{\textbf{Long Name}} & \textsf{\textbf{Meaning}}\\
\hline \\[-3ex]
}
\tablehead{%
\multicolumn{3}{l}{\small\slshape table continued from previous page}\\
\hline
\textsf{\textbf{Short Name}} & \textsf{textbf{Long Name}} & \textsf{\textbf{Meaning}}\\
\hline \\[-3ex]
}
\tabletail{%
\hline\\[-4ex]
\multicolumn{3}{l}{\small\slshape table continued on next page}\\
}
\tablelasttail{\hline}
\tablecaption{Available Field Types for Formatting OpenMP Thread Affinity Information \label{tab:Available Field Types for Formatting OpenMP Thread Affinity Information}}
\begin{supertabular}{ p{0.06\textwidth} p{0.24\textwidth} p{0.6\textwidth}}
{\scode{L}} & {\scode{thread_level}} & The value returned by {\scode{omp_get_level()}}. \\
{\scode{n}} & {\scode{thread_num}} &  The value returned by {\scode{omp_get_thread_num()}}. \\
{\scode{h}} & {\scode{host}} & The name for the host machine on which the OpenMP program is running. \\
{\scode{P}} & {\scode{process_id}} & The process identifier used by the implementation. \\
{\scode{T}} & {\scode{thread_identifier}} & The thread identifier for a native thread defined by the implementation. \\
{\scode{N}} & {\scode{num_threads}} & The value returned by {\scode{omp_get_num_threads()}}. \\
{\scode{A}} & {\scode{ancestor_tnum}} & The value returned by {\scode{omp_get_ancestor_thread_num(}\splc{level}\scode{)}}, where {\splc{level}} is
{\scode{omp_get_level()}} minus 1. \\
{\scode{a}} & {\scode{thread_affinity}} & The list of numerical identifiers, in the format of a comma-separated list of integers or integer ranges, representing processors on which a thread may execute, subject to OpenMP thread affinity control and/or other external affinity mechanisms. \\

\end{supertabular}

\linenumbers

%Often a descriptive string is used to prefix a field, as in the format  
%"thread\_level=\%.5L",
%and a comma or space is often used to separate the output of each field, as in the format
%"thread\_num=\%0.4n affinity=\%a".
% A literal percent is specified as "\%\%".

Implementations may define additional field types.  If an implementation does not have information
for a field type, "undefined" is printed for this field when displaying the OpenMP thread 
affinity information.

%As an example, some compilers may wish to define field names corresponding to MPI rank information.


Example:
\begin{ompEnv}
setenv OMP_AFFINITY_FORMAT
       "Thread Affinity: %0.3L %.8n %.15{thread_affinity} %.12h"
\end{ompEnv}

The above example causes an OpenMP implementation to display OpenMP thread affinity information in the following form:
\begin{ompSyntax}
Thread Affinity: 001        0      0-1,16-17      nid003
Thread Affinity: 001        1      2-3,18-19      nid003
\end{ompSyntax}

\crossreferences
\begin{itemize}
\item Controlling OpenMP thread affinity, see
\specref{subsec:Controlling OpenMP Thread Affinity}.
\item \code{omp_set_affinity_format} routine, see \specref{subsec:omp_set_affinity_format}.
\item \code{omp_get_affinity_format} routine, see \specref{subsec:omp_get_affinity_format}.
\item \code{omp_display_affinity} routine, see \specref{subsec:omp_display_affinity}.
\item \code{omp_capture_affinity} routine, see \specref{subsec:omp_capture_affinity}.
\item \code{OMP_DISPLAY_AFFINITY} environment variable, see
\specref{sec:OMP_DISPLAY_AFFINITY}.
\end{itemize}


\section{\hcode{OMP_DEFAULT_DEVICE}}
\index{OMP_DEFAULT_DEVICE@{\code{OMP_DEFAULT_DEVICE}}}
\index{environment variables!OMP_DEFAULT_DEVICE@{\code{OMP_DEFAULT_DEVICE}}}
\label{sec:OMP_DEFAULT_DEVICE}
The \code{OMP_DEFAULT_DEVICE} environment variable sets the device number to use in
device constructs by setting the initial value of the \plc{default-device-var} ICV.

The value of this environment variable must be a non-negative integer value.

\crossreferences
\begin{itemize}
\item \plc{default-device-var} ICV, see \specref{sec:Internal Control Variables}.

\item device directives, \specref{sec:Device Directives}.
\end{itemize}


\section{\hcode{OMP_MAX_TASK_PRIORITY}}
\index{OMP_MAX_TASK_PRIORITY_DEVICE@{\code{OMP_MAX_TASK_PRIORITY}}}
\index{environment variables!OMP_MAX_TASK_PRIORITY@{\code{OMP_MAX_TASK_PRIORITY}}}
\label{sec:OMP_MAX_TASK_PRIORITY}

The \code{OMP_MAX_TASK_PRIORITY} environment variable controls the use of task
priorities by setting the initial value of the \plc{max-task-priority-var} ICV. The
value of this environment variable must be a non-negative integer.

Example:
\begin{ompEnv}
% setenv OMP_MAX_TASK_PRIORITY 20
\end{ompEnv}

\crossreferences
\begin{itemize}
\item \plc{max-task-priority-var} ICV, see \specref{sec:Internal Control Variables}.
\item Tasking Constructs, see \specref{sec:Tasking Constructs}.
\item \code{omp_get_max_task_priority} routine, see \specref{subsec:omp_get_max_task_priority}.
\end{itemize}




\section{\hcode{OMP_TARGET_OFFLOAD}}
\index{OMP_TARGET_OFFLOAD@{\code{OMP_TARGET_OFFLOAD}}}
\index{environment variables!OMP_TARGET_OFFLOAD@{\code{OMP_TARGET_OFFLOAD}}}
\label{sec:OMP_TARGET_OFFLOAD}
%
The \code{OMP_TARGET_OFFLOAD} environment variable sets the initial value of the \plc{target-offload-var} ICV.
The value of the \code{OMP_TARGET_OFFLOAD} environment variable may be set to one of these
values:

{\code{MANDATORY }|\code{ DISABLED }|\code{ DEFAULT}}

The \code{MANDATORY} value specifies that a device construct or a device memory routine must execute on a target device. If the device construct cannot execute on its target device, or if a device memory routine fails to execute, a warning is issued and the program execution aborts. Device constructs are exempt from this behavior when an if-clause is present and the if-clause expression evaluates to false.

The support of \code{DISABLED}  is implementation defined. If an implementation supports it, the behavior should be that a device construct must execute on the host.  The behavior with this environment value is equivalent to an if clause present on all device constructs, where each of these if clause expressions evaluate to false. Device memory routines behave as if all device number parameters are set to the value returned by \code{omp_get_initial_device()}. The \code{omp_get_initial_device()} routine returns that no target device is available

The \code{DEFAULT} value specifies that when one or more target devices are available, the runtime behaves as if this environment variable is set to \code{MANDATORY}; otherwise, the runtime behaves as if this environment variable is set to \code{DISABLED}.

Example:
\begin{ompEnv}
% setenv OMP_TARGET_OFFLOAD MANDATORY
\end{ompEnv}

\crossreferences
\begin{itemize}
\item \plc{target-offload-icv} ICV, see \specref{sec:Internal Control Variables}.
\item device directives, \specref{sec:Device Directives}.
\end{itemize}




\section{\hcode{OMP_TOOL}}
\index{OMP_TOOL@{\code{OMP_TOOL}}}
\index{environment variables!OMP_TOOL@{\code{OMP_TOOL}}}
\label{sec:OMP_TOOL}

The \code{OMP_TOOL} environment variable sets the \plc{tool-var} ICV which controls whether an OpenMP runtime will try to register a
first party tool.
The value of this environment variable must be \code{enabled} or \code{disabled}.
If \code{OMP_TOOL} is set to any value other than \code{enabled} or \code{disabled}, the behavior is unspecified.
If \code{OMP_TOOL} is not defined, the default value for \plc{tool-var} is \code{enabled}.

Example:
\begin{ompEnv}
% setenv OMP_TOOL enabled
\end{ompEnv}

\crossreferences
\begin{itemize}
\item \plc{tool-var} ICV, see \specref{sec:Internal Control Variables}.
\item Tool Interface, see \specref{chap:ToolsSupport}.
\end{itemize}




\section{\hcode{OMP_TOOL_LIBRARIES}}
\index{OMP_TOOL_LIBRARIES@{\code{OMP_TOOL_LIBRARIES}}}
\index{environment variables!OMP_TOOL_LIBRARIES@{\code{OMP_TOOL_LIBRARIES}}}
\label{sec:OMP_TOOL_LIBRARIES}

The \code{OMP_TOOL_LIBRARIES} environment variable sets the
\plc{tool-libraries-var} ICV to a list of tool libraries that will
be considered for use on a device where an OpenMP implementation
is being initialized.  The value of this environment variable must
be a colon-separated list of dynamically-linked libraries, each
specified by an absolute path.

If the \plc{tool-var} ICV is not enabled,
the value of \plc{tool-libraries-var} will be ignored.
Otherwise, if \code{ompt_start_tool}, a global function symbol for a tool initializer,
isn't visible in the address space on a device where OpenMP is being
initialized or if \code{ompt_start_tool} returns \code{NULL}, an OpenMP implementation
will consider libraries in the \plc{tool-libraries-var} list
in a left to right order.  The OpenMP implementation will search the list for
a library that meets two criteria: it can be dynamically
loaded on the current device and it defines the symbol \code{ompt_start_tool}.
If an OpenMP implementation finds a suitable library,
no further libraries in the list will be considered.

\crossreferences
\begin{itemize}
\item \plc{tool-libraries-var} ICV, see \specref{sec:Internal Control Variables}.
\item Tool Interface, see \specref{chap:ToolsSupport}.
\item \code{ompt_start_tool} routine, see \specref{sec:ompt_start_tool}.
\end{itemize}

\section{\hcode{OMP_DEBUG}}
\index{OMP_DEBUG@{\code{OMP_DEBUG}}}
\index{environment variables!OMP_DEBUG@{\code{OMP_DEBUG}}}
\label{sec:OMP_DEBUG}

The \code{OMP_DEBUG} environment variable sets the
\plc{debug-var} ICV which controls whether an OpenMP runtime
will collect information that an OMPD library may need to support
a tool.
The value of this environment variable must be \code{enabled}
or \code{disabled}.
If \code{OMP_DEBUG} is set to any value other than \code{enabled}
or \code{disabled}, the behavior is implementation defined.

Example:
\begin{ompEnv}
% setenv OMP_DEBUG enabled
\end{ompEnv}

\crossreferences
\begin{itemize}
\item \plc{debug-var} ICV,
see \specref{sec:Internal Control Variables}.
\item Tool Interface, see \specref{chap:ToolsSupport}.
\item Enabling the Runtime for OMPD, see \specref{subsubsec:enabling-ompd}.
\end{itemize}


\section{\hcode{OMP_ALLOCATOR}}
\index{OMP_ALLOCATOR@{\code{OMP_ALLOCATOR}}}
\index{environment variables!OMP_ALLOCATOR@{\code{OMP_ALLOCATOR}}}
\label{sec:OMP_ALLOCATOR}

\code{OMP_ALLOCATOR} sets the \plc{def-allocator-var} ICV that specifies the default
allocator for allocation calls, directives and clauses that do not specify an allocator.
The value of this environment variable is a predefined allocator from \tabref{tab:Predefined Allocators}. The value of this environment variable is not case sensitive.

\crossreferences
\begin{itemize}
\item \plc{def-allocator-var} ICV, see \specref{sec:Internal Control Variables}.

\item Memory allocators, see \specref{subsec:Memory Allocators}.

\item \code{omp_set_default_allocator} routine, see \specref{subsec:omp_set_default_allocator}.

\item \code{omp_get_default_allocator} routine, see \specref{subsec:omp_get_default_allocator}.
\end{itemize}


% This is the end of ch5-environmentVariables.tex


    \setcounter{chapter}{0}  % restart chapter numbering with "letter A"
    \renewcommand{\thechapter}{\Alph{chapter}}%
    \appendix

    \chapter{Stubs for Runtime Library Routines}
\label{chap:Stubs for Runtime Library Routines}
\label{chap:Appendix A}
This section provides stubs for the runtime library routines defined in the OpenMP API.
The stubs are provided to enable portability to platforms that do not support the
OpenMP API. On these platforms, OpenMP programs must be linked with a library
containing these stub routines. The stub routines assume that the directives in the
OpenMP program are ignored. As such, they emulate serial semantics
executing on the host.

Note that the lock variable that appears in the lock routines must be accessed
exclusively through these routines. It should not be initialized or otherwise modified in
the user program.

In an actual implementation the lock variable might be used to hold the address of an
allocated memory block, but here it is used to hold an integer value. Users should not
make assumptions about mechanisms used by OpenMP implementations to implement
locks based on the scheme used by the stub procedures.

\begin{fortranspecific}

In order to be able to compile the Fortran stubs file, the include file
\code{omp_lib.h} was split into two files: \code{omp_lib_kinds.h} and \code{omp_lib.h} and the
\code{omp_lib_kinds.h} file included where needed. There is no requirement for the
implementation to provide separate files.

\end{fortranspecific}


\pagebreak
\section{C/C++ Stub Routines}
\index{C/C++ stub routines}
\index{stub routines}
\label{sec:C/C++ Stub Routines}
\cstub{}{stubs.c}



%\pagebreak
% This is stubs_fortran.tex (Appendix A) of the OpenMP specification.
% This is an included file. See the master file for more information.
%
% When editing this file:
%
%    1. To change formatting, appearance, or style, please edit openmp.sty.
%
%    2. Custom commands and macros are defined in openmp.sty.
%
%    3. Be kind to other editors -- keep a consistent style by copying-and-pasting to
%       create new content.
%
%    4. We use semantic markup, e.g. (see openmp.sty for a full list):
%         \code{}     % for bold monospace keywords, code, operators, etc.
%         \plc{}      % for italic placeholder names, grammar, etc.
%
%    5. There are environments that provide special formatting, e.g. language bars.
%       Please use them whereever appropriate.  Examples are:
%
%         \begin{fortranspecific}
%         This is text that appears enclosed in blue language bars for Fortran.
%         \end{fortranspecific}
%
%         \begin{note}
%         This is a note.  The "Note -- " header appears automatically.
%         \end{note}
%
%    6. Other recommendations:
%         Use the convenience macros defined in openmp.sty for the minor headers
%         such as Comments, Syntax, etc.
%
%         To keep items together on the same page, prefer the use of
%         \begin{samepage}.... Avoid \parbox for text blocks as it interrupts line numbering.
%         When possible, avoid \filbreak, \pagebreak, \newpage, \clearpage unless that's
%         what you mean. Use \needspace{} cautiously for troublesome paragraphs.
%
%         Avoid absolute lengths and measures in this file; use relative units when possible.
%         Vertical space can be relative to \baselineskip or ex units. Horizontal space
%         can be relative to \linewidth or em units.
%
%         Prefer \emph{} to italicize terminology, e.g.:
%             This is a \emph{definition}, not a placeholder.
%             This is a \plc{var-name}.
%


\section{Fortran Stub Routines}
\label{sec:Fortran Stub Routines}
{\small \begin{codepar}
subroutine omp\_set\_num\_threads(num\_threads)
  integer num\_threads
  return
end subroutine

integer function omp\_get\_num\_threads()
  omp\_get\_num\_threads = 1
  return
end function

integer function omp\_get\_max\_threads()
  omp\_get\_max\_threads = 1
  return
end function

integer function omp\_get\_thread\_num()
  omp\_get\_thread\_num = 0
  return
end function

integer function omp\_get\_num\_procs()
  omp\_get\_num\_procs = 1
  return
end function

logical function omp\_in\_parallel()
  omp\_in\_parallel = .false.
  return
end function

subroutine omp\_set\_dynamic(dynamic\_threads)
  logical dynamic\_threads
  return
end subroutine

logical function omp\_get\_dynamic()
  omp\_get\_dynamic = .false.
  return
end function

logical function omp\_get\_cancellation()
  omp\_get\_cancellation = .false.
  return
end function

subroutine omp\_set\_nested(nested)
  logical nested
  return
end subroutine

logical function omp\_get\_nested()
  omp\_get\_nested = .false.
  return
end function

subroutine omp\_set\_schedule(kind, chunk\_size)
  include 'omp\_lib\_kinds.h'
  integer (kind=omp\_sched\_kind) kind
  integer chunk\_size
  return
end subroutine

subroutine omp\_get\_schedule(kind, chunk\_size)
  include 'omp\_lib\_kinds.h'
  integer (kind=omp\_sched\_kind) kind
  integer chunk\_size
  kind = omp\_sched\_static
  chunk\_size = 0
  return
end subroutine

integer function omp\_get\_thread\_limit()
  omp\_get\_thread\_limit = 1
  return
end function

subroutine omp\_set\_max\_active\_levels(max\_level)
  integer max\_level
end subroutine

integer function omp\_get\_max\_active\_levels()
  omp\_get\_max\_active\_levels = 0
  return
end function

integer function omp\_get\_level()
  omp\_get\_level = 0
  return
end function

integer function omp\_get\_ancestor\_thread\_num(level)
  integer level
  if ( level .eq. 0 ) then
     omp\_get\_ancestor\_thread\_num = 0
  else
     omp\_get\_ancestor\_thread\_num = -1
  end if
  return
end function

integer function omp\_get\_team\_size(level)
  integer level
  if ( level .eq. 0 ) then
     omp\_get\_team\_size = 1
  else
     omp\_get\_team\_size = -1
  end if
  return
end function

integer function omp\_get\_active\_level()
  omp\_get\_active\_level = 0
  return
end function

logical function omp\_in\_final()
  omp\_in\_final = .true.
  return
end function

function omp\_get\_proc\_bind()
  include 'omp\_lib\_kinds.h'
  integer (kind=omp\_proc\_bind\_kind) omp\_get\_proc\_bind
  omp\_get\_proc\_bind = omp\_proc\_bind\_false
end function

integer function omp\_get\_num\_places()
  return 0
end function

integer function omp\_get\_place\_num\_procs(place\_num)
  integer place\_num
  return 0
end function

subroutine omp\_get\_place\_proc\_ids(place\_num, ids)
  integer place\_num
  integer ids(*)
  return
end subroutine

integer function omp\_get\_place\_num()
  return -1
end function

integer function omp\_get\_partition\_num\_places()
  return 0
end function

subroutine omp\_get\_partition\_place\_nums(place\_nums)
  integer place\_nums(*)
  return
end subroutine


subroutine omp\_set\_affinity\_format(\plc{format})
   character(len=*),intent(in)::format
   return
end subroutine

integer function omp\_get\_affinity\_format(buffer)
   character(len=*),intent(out)::buffer
   return 0
end function

subroutine omp\_display\_affinity(format)
   character(len=*),intent(in)::format
   return
end subroutine

integer function omp\_capture\_affinity(buffer,format)
   character(len=*),intent(out)::buffer
   character(len=*),intent(in)::format
   return 0
end function

subroutine omp\_set\_default\_device(device\_num)
  integer device\_num
  return
end subroutine

integer function omp\_get\_default\_device()
  omp\_get\_default\_device = 0
  return
end function

integer function omp\_get\_num\_devices()
  omp\_get\_num\_devices = 0
  return
end function

integer function omp\_get\_device\_num()
  omp\_get\_device\_num = -10
  return
end function

integer function omp\_get\_num\_teams()
  omp\_get\_num\_teams = 1
  return
end function

integer function omp\_get\_team\_num()
  omp\_get\_team\_num = 0
  return
end function

logical function omp\_is\_initial\_device()
  omp\_is\_initial\_device = .true.
  return
end function

integer function omp\_get\_initial\_device()
  omp\_get\_initial\_device = -10
  return
end function

integer function omp\_get\_max\_task\_priority()
  omp\_get\_max\_task\_priority = 0
  return
end function

subroutine omp\_init\_lock(lock)
  ! lock is 0 if the simple lock is not initialized
  !        -1 if the simple lock is initialized but not set
  !         1 if the simple lock is set
  include 'omp\_lib\_kinds.h'
  integer(kind=omp\_lock\_kind) lock

  lock = -1
  return
end subroutine

subroutine omp\_init\_lock\_with\_hint(lock, hint)
  include 'omp\_lib\_kinds.h'
  integer(kind=omp\_lock\_kind) lock
  integer(kind=omp\_lock\_hint\_kind) hint

  call omp\_init\_lock(lock)
  return
end subroutine

subroutine omp\_destroy\_lock(lock)
  include 'omp\_lib\_kinds.h'
  integer(kind=omp\_lock\_kind) lock

  lock = 0
  return
end subroutine

subroutine omp\_set\_lock(lock)
  include 'omp\_lib\_kinds.h'
  integer(kind=omp\_lock\_kind) lock

  if (lock .eq. -1) then
    lock = 1
  elseif (lock .eq. 1) then
    print *, 'error: deadlock in using lock variable'
    stop
  else
    print *, 'error: lock not initialized'
    stop
  endif
  return
end subroutine

subroutine omp\_unset\_lock(lock)
  include 'omp\_lib\_kinds.h'
  integer(kind=omp\_lock\_kind) lock

  if (lock .eq. 1) then
    lock = -1
  elseif (lock .eq. -1) then
    print *, 'error: lock not set'
    stop
  else
    print *, 'error: lock not initialized'
    stop
  endif
  return
end subroutine

logical function omp\_test\_lock(lock)
  include 'omp\_lib\_kinds.h'
  integer(kind=omp\_lock\_kind) lock

  if (lock .eq. -1) then
    lock = 1
    omp\_test\_lock = .true.
  elseif (lock .eq. 1) then
    omp\_test\_lock = .false.
  else
    print *, 'error: lock not initialized'
    stop
  endif

  return
end function

subroutine omp\_init\_nest\_lock(nlock)
  ! nlock is
  ! 0 if the nestable lock is not initialized
  ! -1 if the nestable lock is initialized but not set
  ! 1 if the nestable lock is set
  ! no use count is maintained
  include 'omp\_lib\_kinds.h'
  integer(kind=omp\_nest\_lock\_kind) nlock

  nlock = -1

  return
end subroutine

subroutine omp\_init\_nest\_lock\_with\_hint(nlock, hint)
  include 'omp\_lib\_kinds.h'
  integer(kind=omp\_nest\_lock\_kind) nlock
  integer(kind=omp\_lock\_hint\_kind) hint

  call omp\_init\_nest\_lock(nlock)
  return
end subroutine

subroutine omp\_destroy\_nest\_lock(nlock)
  include 'omp\_lib\_kinds.h'
  integer(kind=omp\_nest\_lock\_kind) nlock

  nlock = 0

  return
end subroutine

subroutine omp\_set\_nest\_lock(nlock)
  include 'omp\_lib\_kinds.h'
  integer(kind=omp\_nest\_lock\_kind) nlock

  if (nlock .eq. -1) then
    nlock = 1
  elseif (nlock .eq. 0) then
    print *, 'error: nested lock not initialized'
    stop
  else
    print *, 'error: deadlock using nested lock variable'
    stop
  endif

  return
end subroutine

subroutine omp\_unset\_nest\_lock(nlock)
  include 'omp\_lib\_kinds.h'
  integer(kind=omp\_nest\_lock\_kind) nlock

  if (nlock .eq. 1) then
    nlock = -1
  elseif (nlock .eq. 0) then
    print *, 'error: nested lock not initialized'
    stop
  else
    print *, 'error: nested lock not set'
    stop
  endif

  return
end subroutine

integer function omp\_test\_nest\_lock(nlock)
  include 'omp\_lib\_kinds.h'
  integer(kind=omp\_nest\_lock\_kind) nlock

  if (nlock .eq. -1) then
    nlock = 1
    omp\_test\_nest\_lock = 1
  elseif (nlock .eq. 1) then
    omp\_test\_nest\_lock = 0
  else
    print *, 'error: nested lock not initialized'
    stop
  endif

  return
end function

double precision function omp\_get\_wtime()
  ! this function does not provide a working
  ! wall clock timer. replace it with a version
  ! customized for the target machine.

  omp\_get\_wtime = 0.0d0

  return
end function

double precision function omp\_get\_wtick()
  ! this function does not provide a working
  ! clock tick function. replace it with
  ! a version customized for the target machine.
  double precision one\_year
  parameter (one\_year=365.d0*86400.d0)

  omp\_get\_wtick = one\_year

  return
end function

int function omp\_control\_tool(command, modifier)
  include 'omp\_lib\_kinds.h'
  integer (kind=omp\_control\_tool\_kind) command
  integer (kind=omp\_control\_tool\_kind) modifier

  return omp\_control\_tool\_notool
end function

subroutine omp\_set\_default\_allocator(allocator)
  include 'omp\_lib\_kinds.h'
  integer (kind=omp\_allocator\_kind) allocator
  return
end subroutine

function omp\_get\_default\_allocator
  include 'omp\_lib\_kinds.h'
  integer (kind=omp\_allocator\_kind) omp\_get\_default\_allocator
  omp\_get\_default\_allocator = omp\_null\_allocator
end function

\end{codepar}} % end \small block


    % This is interface_declarations.tex (Appendix B) of the OpenMP specification.
% This is an included file. See the master file for more information.
%
% When editing this file:
%
%    1. To change formatting, appearance, or style, please edit openmp.sty.
%
%    2. Custom commands and macros are defined in openmp.sty.
%
%    3. Be kind to other editors -- keep a consistent style by copying-and-pasting to
%       create new content.
%
%    4. We use semantic markup, e.g. (see openmp.sty for a full list):
%         \code{}     % for bold monospace keywords, code, operators, etc.
%         \plc{}      % for italic placeholder names, grammar, etc.
%
%    5. There are environments that provide special formatting, e.g. language bars.
%       Please use them whereever appropriate.  Examples are:
%
%         \begin{fortranspecific}
%         This is text that appears enclosed in blue language bars for Fortran.
%         \end{fortranspecific}
%
%         \begin{note}
%         This is a note.  The "Note -- " header appears automatically.
%         \end{note}
%
%    6. Other recommendations:
%         Use the convenience macros defined in openmp.sty for the minor headers
%         such as Comments, Syntax, etc.
%
%         To keep items together on the same page, prefer the use of 
%         \begin{samepage}.... Avoid \parbox for text blocks as it interrupts line numbering.
%         When possible, avoid \filbreak, \pagebreak, \newpage, \clearpage unless that's
%         what you mean. Use \needspace{} cautiously for troublesome paragraphs.
%
%         Avoid absolute lengths and measures in this file; use relative units when possible.
%         Vertical space can be relative to \baselineskip or ex units. Horizontal space
%         can be relative to \linewidth or em units.
%
%         Prefer \emph{} to italicize terminology, e.g.:
%             This is a \emph{definition}, not a placeholder.
%             This is a \plc{var-name}.
%



\chapter{Interface Declarations}
\index{interface declarations}
\index{header files}
\index{include files}
\label{chap:Interface Declarations}
This appendix gives examples of the C/C++ header file, the Fortran \code{include} file and 
Fortran \code{module} that shall be provided by implementations as specified in Chapter 3. It 
also includes an example of a Fortran 90 generic interface for a library routine. This is a 
non-normative section, implementation files may differ.




\pagebreak
\section{Example of the \code{omp.h} Header File}
\label{sec:Example of the omp.h Header File}
{\small \begin{codepar}
\#ifndef \_OMP\_H\_DEF
\#define \_OMP\_H\_DEF

/*
 * define the lock data types
 */
typedef void *omp\_lock\_t;

typedef void *omp\_nest\_lock\_t;

/*
 * define the synchronization hints
 */
typedef enum omp\_sync\_hint\_t \{
  omp\_sync\_hint\_none = 0,
  omp\_lock\_hint\_none = omp\_sync\_hint\_none,
  omp\_sync\_hint\_uncontended = 1,
  omp\_lock\_hint\_uncontended = omp\_sync\_hint\_uncontended,
  omp\_sync\_hint\_contended = 2,
  omp\_lock\_hint\_contended = omp\_sync\_hint\_contended,
  omp\_sync\_hint\_nonspeculative = 4,
  omp\_lock\_hint\_nonspeculative = omp\_sync\_hint\_nonspeculative,
  omp\_sync\_hint\_speculative = 8,
  omp\_lock\_hint\_speculative = omp\_sync\_hint\_speculative
/* , Add vendor specific constants for lock hints here,
   starting from the most-significant bit. */
\} omp\_sync\_hint\_t;

/* omp\_lock\_hint\_t has been deprecated */
typedef omp\_sync\_hint\_t omp\_lock\_hint\_t;

/*
 * define the schedule kinds
 */
typedef enum omp\_sched\_t
\{
 omp\_sched\_static = 1,
 omp\_sched\_dynamic = 2,
 omp\_sched\_guided = 3,
 omp\_sched\_auto = 4
/* , Add vendor specific schedule constants here */
\} omp\_sched\_t;

/*
* define the proc bind values 
*/ 
typedef enum omp\_proc\_bind\_t
\{
 omp\_proc\_bind\_false = 0,
 omp\_proc\_bind\_true = 1,
 omp\_proc\_bind\_master = 2,
 omp\_proc\_bind\_close = 3,
 omp\_proc\_bind\_spread = 4
\} omp\_proc\_bind\_t; 
\begin{samepage}
/*
 * define the tool control commands 
 */
typedef omp\_control\_tool\_t 
\{
  omp\_control\_tool\_start = 1,
  omp\_control\_tool\_pause = 2,
  omp\_control\_tool\_flush = 3,
  omp\_control\_tool\_end = 4,
\} omp\_control\_tool\_t;
\end{samepage}

/*
 * define memory management types
 */
typedef void *omp\_allocator\_t;
enum \{ OMP\_NULL\_ALLOCATOR = NULL \};

/*
 * exported OpenMP functions
 */
\#ifdef _\_cplusplus
extern "C"
\{
\#endif

extern void omp\_set\_num\_threads(int num\_threads);
extern int omp\_get\_num\_threads(void);
extern int omp\_get\_max\_threads(void);
extern int omp\_get\_thread\_num(void);
extern int omp\_get\_num\_procs(void);
extern int omp\_in\_parallel(void);
extern void omp\_set\_dynamic(int dynamic\_threads);
extern int omp\_get\_dynamic(void);
extern int omp\_get\_cancellation(void);
extern void omp\_set\_nested(int nested);
extern int omp\_get\_nested(void);
extern void omp\_set\_schedule(omp\_sched\_t kind, int chunk\_size);
extern void omp\_get\_schedule(omp\_sched\_t *kind, int *chunk\_size);
extern int omp\_get\_thread\_limit(void);
extern void omp\_set\_max\_active\_levels(int max\_active\_levels);
extern int omp\_get\_max\_active\_levels(void);
extern int omp\_get\_level(void);
extern int omp\_get\_ancestor\_thread\_num(int level);
extern int omp\_get\_team\_size(int level);
extern int omp\_get\_active\_level(void);
extern int omp\_in\_final(void);
extern omp\_proc\_bind\_t omp\_get\_proc\_bind(void);
extern int omp\_get\_num\_places(void);
extern int omp\_get\_place\_num\_procs(int place\_num);
extern void omp\_get\_place\_proc\_ids(int place\_num, int *ids);
extern int omp\_get\_place\_num(void);
extern int omp\_get\_partition\_num\_places(void);
extern void omp\_get\_partition\_place\_nums(int *place\_nums);

extern void omp\_set\_affinity\_format(char const *format);
extern size_t omp\_get\_affinity\_format(char* buffer, size_t size);
extern void omp\_display\_affinity(char const *format);
extern size_t omp\_capture\_affinity(char *buffer, size_t size, char const *format);

extern void omp\_set\_default\_device(int device\_num);
extern int omp\_get\_default\_device(void);

extern int omp\_get\_num\_devices(void);
extern int omp\_get\_device\_num(void);
extern int omp\_get\_num\_teams(void);
extern int omp\_get\_team\_num(void);
extern int omp\_is\_initial\_device(void);
extern int omp\_get\_initial\_device(void);
extern int omp\_get\_max\_task\_priority(void);

extern void omp\_init\_lock(omp\_lock\_t *lock);
extern void omp\_init\_lock\_with\_hint(omp\_lock\_t *lock, 
                                   omp\_lock\_hint\_t hint);
extern void omp\_destroy\_lock(omp\_lock\_t *lock);
extern void omp\_set\_lock(omp\_lock\_t *lock);
extern void omp\_unset\_lock(omp\_lock\_t *lock);
extern int omp\_test\_lock(omp\_lock\_t *lock);

extern void omp\_init\_nest\_lock(omp\_nest\_lock\_t *lock);
extern void omp\_init\_nest\_lock\_with\_hint(omp\_nest\_lock\_t *lock, 
                                          omp\_lock\_hint\_t hint);
extern void omp\_destroy\_nest\_lock(omp\_nest\_lock\_t *lock);
extern void omp\_set\_nest\_lock(omp\_nest\_lock\_t *lock);
extern void omp\_unset\_nest\_lock(omp\_nest\_lock\_t *lock);
extern int omp\_test\_nest\_lock(omp\_nest\_lock\_t *lock);

extern double omp\_get\_wtime(void);
extern double omp\_get\_wtick(void);

extern void * omp\_target\_alloc(size\_t size, int device\_num);
extern void omp\_target\_free(void * device\_ptr, int device\_num);
extern int omp\_target\_is\_present(void * ptr, int device\_num);
extern int omp\_target\_memcpy(void *dst, void *src, size\_t length,
                              size\_t dst\_offset, size\_t src\_offset,
                              int dst\_device\_num, int src\_device\_num);
extern int omp\_target\_memcpy\_rect(
    void *dst, void *src,
    size\_t element\_size,
    int num\_dims,
    const size\_t *volume,
    const size\_t *dst\_offsets,
    const size\_t *src\_offsets,
    const size\_t *dst\_dimensions,
    const size\_t *src\_dimensions,
    int dst\_device\_num, int src\_device\_num);
extern int omp\_target\_associate\_ptr(void * host\_ptr,
                                       void * device\_ptr,
                                       size\_t size,
                                       size\_t device\_offset,
                                       int device\_num);
extern int omp\_target\_disassociate\_ptr(void * ptr,
                                          int device\_num);

extern void omp\_control\_tool(int command, int modifier, void *arg);

extern void omp\_set\_default allocator(const omp\_allocator\_t *allocator);
extern void omp\_set\_target\_default allocator(const omp\_allocator\_t *allocator);
extern const omp\_allocator\_t * omp\_get\_default\_allocator (void);
extern const omp\_allocator\_t * omp\_get\_target\_default\_allocator (void);


extern const omp_allocator_t * omp_default_mem_alloc;
extern const omp_allocator_t * omp_large_cap_mem_alloc;
extern const omp_allocator_t * omp_const_mem_alloc;
extern const omp_allocator_t * omp_high_bw_mem_alloc;
extern const omp_allocator_t * omp_low_lat_mem_alloc;
extern const omp_allocator_t * omp_cgroup_mem_alloc;
extern const omp_allocator_t * omp_pteam_mem_alloc;
extern const omp_allocator_t * omp_thread_mem_alloc;


\#ifdef _\_cplusplus
extern void * omp\_alloc (size\_t size, 
                          const omp\_allocator\_t *allocator = OMP\_NULL\_ALLOCATOR);
extern void omp\_free (void * ptr, 
                       const omp\_allocator\_t *allocator = OMP\_NULL\_ALLOCATOR);
\#else
extern void * omp\_alloc (size\_t size, const omp\_allocator\_t *allocator);
extern void omp\_free (void * ptr, const omp\_allocator\_t *allocator);
\#endif


\#ifdef _\_cplusplus
\}
\#endif

\#endif
\end{codepar}} % end of \small block

\pagebreak

{\hyphenpenalty=10000\section{Example of an Interface Declaration \code{include} File}}
\label{sec:Example of an Interface Declaration include File}
\emph{omp\_lib\_kinds.h:}
{\small \begin{codepar}

     integer omp\_lock\_kind
     integer omp\_nest\_lock\_kind
     integer omp\_control\_tool\_kind
     integer omp\_control\_tool\_result\_kind
! this selects an integer that is large enough to hold a 64 bit integer
     parameter ( omp\_lock\_kind = selected\_int\_kind( 10 ) )
     parameter ( omp\_nest\_lock\_kind = selected\_int\_kind( 10 ) )
     
     integer omp\_sched\_kind
! this selects an integer that is large enough to hold a 32 bit integer
     parameter ( omp\_sched\_kind = selected\_int\_kind( 8 ) )
     integer ( omp\_sched\_kind ) omp\_sched\_static
     parameter ( omp\_sched\_static = 1 )
     integer ( omp\_sched\_kind ) omp\_sched\_dynamic
     parameter ( omp\_sched\_dynamic = 2 )
     integer ( omp\_sched\_kind ) omp\_sched\_guided
     parameter ( omp\_sched\_guided = 3 )
     integer ( omp\_sched\_kind ) omp\_sched\_auto
     parameter ( omp\_sched\_auto = 4 )
     
     integer omp\_proc\_bind\_kind
     parameter ( omp\_proc\_bind\_kind = selected\_int\_kind( 8 ) )
     integer ( omp\_proc\_bind\_kind ) omp\_proc\_bind\_false
     parameter ( omp\_proc\_bind\_false = 0 )
     integer ( omp\_proc\_bind\_kind ) omp\_proc\_bind\_true
     parameter ( omp\_proc\_bind\_true = 1 )
     integer ( omp\_proc\_bind\_kind ) omp\_proc\_bind\_master
     parameter ( omp\_proc\_bind\_master = 2 )
     integer ( omp\_proc\_bind\_kind ) omp\_proc\_bind\_close
     parameter ( omp\_proc\_bind\_close = 3 )
     integer ( omp\_proc\_bind\_kind ) omp\_proc\_bind\_spread
     parameter ( omp\_proc\_bind\_spread = 4 )
\begin{samepage}
     integer omp\_sync\_hint\_kind
     parameter ( omp\_sync\_hint\_kind = selected\_int\_kind( 10 ) )
     integer omp\_lock\_hint\_kind
     parameter ( omp\_lock\_hint\_kind = omp\_sync\_hint\_kind )
     integer ( omp\_sync\_hint\_kind) omp\_sync\_hint\_none
     parameter ( omp\_sync\_hint\_none = 0 )
     integer ( omp\_lock\_hint\_kind) omp\_lock\_hint\_none 
     parameter ( omp\_lock\_hint\_none = omp\_sync\_hint\_none )
     integer ( omp\_sync\_hint\_kind) omp\_sync\_hint\_uncontended 
     parameter ( omp\_sync\_hint\_uncontended = 1 )
     integer ( omp\_lock\_hint\_kind) omp\_lock\_hint\_uncontended 
     parameter ( omp\_lock\_hint\_uncontended = omp\_sync\_hint\_uncontended )
     integer ( omp\_sync\_hint\_kind) omp\_sync\_hint\_contended 
     parameter ( omp\_sync\_hint\_contended = 2 )
     integer ( omp\_lock\_hint\_kind) omp\_lock\_hint\_contended 
     parameter ( omp\_lock\_hint\_contended = omp\_sync\_hint\_contended )
     integer ( omp\_sync\_hint\_kind) omp\_sync\_hint\_nonspeculative 
     parameter ( omp\_sync\_hint\_nonspeculative = 4 )
     integer ( omp\_lock\_hint\_kind) omp\_lock\_hint\_nonspeculative 
     parameter ( omp\_lock\_hint\_nonspeculative = omp\_sync\_hint\_nonspeculative )
     integer ( omp\_sync\_hint\_kind) omp\_sync\_hint\_speculative 
     parameter ( omp\_sync\_hint\_speculative = 8 )
     integer ( omp\_lock\_hint\_kind) omp\_lock\_hint\_speculative 
     parameter ( omp\_lock\_hint\_speculative = omp\_sync\_hint\_speculative )
\end{samepage}
\begin{samepage}
     parameter ( omp\_control\_tool\_kind = selected\_int\_kind( 8 ) )
     integer ( omp\_control\_tool\_kind ) omp\_control\_tool\_start 
     parameter ( omp\_control\_tool\_start = 1 )
     integer ( omp\_control\_tool\_kind ) omp\_control\_tool\_pause 
     parameter ( omp\_control\_tool\_pause = 2 )
     integer ( omp\_control\_tool\_kind ) omp\_control\_tool\_flush 
     parameter ( omp\_control\_tool\_flush = 3 )
     integer ( omp\_control\_tool\_kind ) omp\_control\_tool\_end 
     parameter ( omp\_control\_tool\_end = 4 )
\end{samepage}
\begin{samepage}
     parameter ( omp\_control\_tool\_result\_kind = selected\_int\_kind( 8 ) )
     integer ( omp\_control\_tool\_result\_kind ) omp_control_tool_notool 
     parameter ( omp_control_tool_notool = -2 )
     integer ( omp\_control\_tool\_result\_kind ) omp_control_tool_nocallback
     parameter ( omp_control_tool_nocallback = -1 )
     integer ( omp\_control\_tool\_result\_kind ) omp_control_tool_success
     parameter ( omp_control_tool_success = 0 )
     integer ( omp\_control\_tool\_result\_kind ) omp_control_tool_ignored 
     parameter ( omp_control_tool_ignored = 1 )
\end{samepage}
\begin{samepage}
     integer omp\_allocator\_kind
     parameter ( omp\_allocator\_kind = selected\_int\_kind( 8 ) )
     integer ( omp\_allocator\_kind ) omp\_null\_allocator
     parameter ( omp\_null\_allocator = 0 )
     integer ( omp\_allocator\_kind ) omp\_default\_mem\_alloc
     parameter ( omp\_default\_mem\_alloc = 1 )
     integer ( omp\_allocator\_kind ) omp\_large\_cap\_mem\_alloc
     parameter ( omp\_large\_cap\_mem\_alloc = 2 )
     integer ( omp\_allocator\_kind ) omp\_const\_mem\_alloc
     parameter ( omp\_const\_mem\_alloc = 3 )
     integer ( omp\_allocator\_kind ) omp\_high\_bw\_mem\_alloc
     parameter ( omp\_high\_bw\_mem\_alloc = 4 )
     integer ( omp\_allocator\_kind ) omp\_low\_lat\_mem\_alloc
     parameter ( omp\_low\_lat\_mem\_alloc = 5 )
     integer ( omp\_allocator\_kind ) omp\_cgroup\_mem\_alloc
     parameter ( omp\_cgroup\_mem\_alloc = 6 )
     integer ( omp\_allocator\_kind ) omp\_pteam\_mem\_alloc
     parameter ( omp\_pteam\_mem\_alloc = 7 )
     integer ( omp\_allocator\_kind ) omp\_thread\_mem\_alloc
     parameter ( omp\_thread\_mem\_alloc = 8 )
\end{samepage}
\end{codepar}}

\emph{omp\_lib.h:}

{\small \begin{codepar}
! default integer type assumed below
! default logical type assumed below
! OpenMP API v5.0 Preview 1 (TR4)

     include 'omp\_lib\_kinds.h'
     integer openmp\_version
     parameter ( openmp\_version = 201611 )

     external omp\_set\_num\_threads
     external omp\_get\_num\_threads
     integer omp\_get\_num\_threads
     external omp\_get\_max\_threads
     integer omp\_get\_max\_threads
     external omp\_get\_thread\_num
     integer omp\_get\_thread\_num
     external omp\_get\_num\_procs
     integer omp\_get\_num\_procs
     external omp\_in\_parallel
     logical omp\_in\_parallel
     external omp\_set\_dynamic
     external omp\_get\_dynamic
     logical omp\_get\_dynamic
     external omp\_get\_cancellation
     logical omp\_get\_cancellation
     external omp\_set\_nested
     external omp\_get\_nested
     logical omp\_get\_nested
     external omp\_set\_schedule
     external omp\_get\_schedule
     external omp\_get\_thread\_limit
     integer omp\_get\_thread\_limit
     external omp\_set\_max\_active\_levels
     external omp\_get\_max\_active\_levels
     integer omp\_get\_max\_active\_levels
     external omp\_get\_level
     integer omp\_get\_level
     external omp\_get\_ancestor\_thread\_num
     integer omp\_get\_ancestor\_thread\_num
     external omp\_get\_team\_size
     integer omp\_get\_team\_size
     external omp\_get\_active\_level
     integer omp\_get\_active\_level
     external omp\_set\_affinity\_format
     external omp\_get\_affinity\_format
     integer omp\_get\_affinity\_format
     external omp\_display\_affinity
     external omp\_capture\_affinity
     integer omp\_capture\_affinity
     external omp\_set\_default\_device
     external omp\_get\_default\_device
     integer omp\_get\_default\_device
     external omp\_get\_num\_devices
     integer omp\_get\_num\_devices
     external omp\_get\_device\_num
     integer omp\_get\_device\_num
     external omp\_get\_num\_teams
     integer omp\_get\_num\_teams
     external omp\_get\_team\_num
     integer omp\_get\_team\_num
     external omp\_is\_initial\_device
     logical omp\_is\_initial\_device
     external omp\_get\_initial\_device
     integer omp\_get\_initial\_device
     external omp\_get\_max\_task\_priority
     integer omp\_get\_max\_task\_priority

     external omp\_in\_final
     logical omp\_in\_final

     integer ( omp\_proc\_bind\_kind ) omp\_get\_proc\_bind
     external omp\_get\_proc\_bind
     integer omp\_get\_num\_places
     external omp\_get\_num\_places
     integer omp\_get\_place\_num\_procs
     external omp\_get\_place\_num\_procs
     external omp\_get\_place\_proc\_ids
     integer omp\_get\_place\_num
     external omp\_get\_place\_num
     integer omp\_get\_partition\_num\_places
     external omp\_get\_partition\_num\_places
     external omp\_get\_partition\_place\_nums
     
     external omp\_init\_lock
     external omp\_init\_lock\_with\_hint
     external omp\_destroy\_lock
     external omp\_set\_lock
     external omp\_unset\_lock
     external omp\_test\_lock
     logical omp\_test\_lock

     external omp\_init\_nest\_lock
     external omp\_init\_nest\_lock\_with\_hint
     external omp\_destroy\_nest\_lock
     external omp\_set\_nest\_lock
     external omp\_unset\_nest\_lock
     external omp\_test\_nest\_lock
     integer omp\_test\_nest\_lock

     external omp\_get\_wtick
     double precision omp\_get\_wtick
     external omp\_get\_wtime
     double precision omp\_get\_wtime

     integer  omp\_control\_tool
     external omp\_control\_tool

     external omp\_set\_default\_allocator
     external omp\_get\_default\_allocator
     integer ( omp\_allocator\_kind ) omp\_get\_default\_allocator
     external omp\_set\_target\_default\_allocator
     external omp\_get\_target\_default\_allocator
     integer ( omp\_allocator\_kind ) omp\_get\_target\_default\_allocator

\end{codepar}}





\pagebreak
\section{Example of a Fortran Interface Declaration \code{module}}
\label{sec:Example of a Fortran Interface Declaration module}
{\small \begin{codepar}
!      the "!" of this comment starts in column 1
!23456

        module omp\_lib\_kinds
        integer, parameter :: omp\_lock\_kind = selected\_int\_kind( 10 )
        integer, parameter :: omp\_nest\_lock\_kind = selected\_int\_kind( 10 )
        integer, parameter :: omp\_lock\_hint\_kind = selected\_int\_kind( 10 )
        integer (kind=omp\_lock\_hint\_kind), parameter :: 
      \&   omp\_lock\_hint\_none = 0
        integer (kind=omp\_lock\_hint\_kind), parameter :: 
      \&   omp\_lock\_hint\_uncontended = 1
        integer (kind=omp\_lock\_hint\_kind), parameter :: 
      \&   omp\_lock\_hint\_contended = 2
        integer (kind=omp\_lock\_hint\_kind), parameter :: 
      \&   omp\_lock\_hint\_nonspeculative = 4 
        integer (kind=omp\_lock\_hint\_kind), parameter :: 
      \&   omp\_lock\_hint\_speculative = 8

        integer, parameter :: omp\_sched\_kind = selected\_int\_kind( 8 ) 
        integer(kind=omp\_sched\_kind), parameter ::
      \&   omp\_sched\_static = 1
        integer(kind=omp\_sched\_kind), parameter ::
      \&   omp\_sched\_dynamic = 2
        integer(kind=omp\_sched\_kind), parameter ::
      \&   omp\_sched\_guided = 3
        integer(kind=omp\_sched\_kind), parameter ::
      \&   omp\_sched\_auto = 4
      
        integer, parameter :: omp\_proc\_bind\_kind = selected\_int\_kind( 8 )
        integer (kind=omp\_proc\_bind\_kind), parameter ::
      \&   omp\_proc\_bind\_false = 0
        integer (kind=omp\_proc\_bind\_kind), parameter ::
      \&   omp\_proc\_bind\_true = 1
        integer (kind=omp\_proc\_bind\_kind), parameter ::
      \&   omp\_proc\_bind\_master = 2
        integer (kind=omp\_proc\_bind\_kind), parameter ::
      \&   omp\_proc\_bind\_close = 3
        integer (kind=omp\_proc\_bind\_kind), parameter ::
      \&   omp\_proc\_bind\_spread = 4
\begin{samepage}
        integer, parameter :: omp\_control\_tool\_kind = selected\_int\_kind( 8 )
        integer (kind=omp\_control\_tool\_kind), parameter :: 
      \&   omp\_control\_tool\_start = 1
        integer (kind=omp\_control\_tool\_kind), parameter :: 
      \&   omp\_control\_tool\_pause = 2
        integer (kind=omp\_control\_tool\_kind), parameter :: 
      \&   omp\_control\_tool\_flush = 3
        integer (kind=omp\_control\_tool\_kind), parameter :: 
      \&   omp\_control\_tool\_end = 4
        end module omp\_lib\_kinds
\end{samepage}
\begin{samepage}
        integer, parameter :: omp\_control\_tool\_result\_kind = 
      \&   selected\_int\_kind( 8 )
        integer ( omp\_control\_tool\_result\_kind ), parameter :: 
      \&   omp_control_tool_notool = -2 
        integer ( omp\_control\_tool\_result\_kind ), parameter ::
      \&   omp_control_tool_nocallback = -1
        integer ( omp\_control\_tool\_result\_kind ), parameter ::
      \&   omp_control_tool_success = 0
        integer ( omp\_control\_tool\_result\_kind ), parameter ::
      \&   omp_control_tool_ignored = 1
\end{samepage}
\begin{samepage}
        integer, pamameter :: omp\_allocator\_kind =
      \&  selected\_int\_kind( 8 )
        integer ( omp\_allocator\_kind ), parameter ::  
      \& omp\_null\_allocator = 0
        integer ( omp\_allocator\_kind ), parameter ::
      \& omp\_default\_mem\_alloc = 1
        integer ( omp\_allocator\_kind ), parameter ::
      \& omp\_large\_cap\_mem\_alloc = 2
        integer ( omp\_allocator\_kind ), parameter ::
      \& omp\_const\_mem\_alloc = 3
        integer ( omp\_allocator\_kind ), parameter :: 
      \& omp\_high\_bw\_mem\_alloc = 4
        integer ( omp\_allocator\_kind ), parameter ::
      \& omp\_low\_lat\_mem\_alloc = 5
        integer ( omp\_allocator\_kind ), parameter ::
      \& omp\_cgroup\_mem\_alloc = 6
        integer ( omp\_allocator\_kind ), parameter ::
      \& omp\_pteam\_mem\_alloc = 7
        integer ( omp\_allocator\_kind ), parameter ::
      \& omp\_thread\_mem\_alloc = 8 
\end{samepage}

        module omp\_lib

          use omp\_lib\_kinds

!                                     OpenMP API v5.0 Preview 1 (TR4)
          integer, parameter :: openmp\_version = 201611

         interface

          subroutine omp\_set\_num\_threads (num\_threads)
           integer, intent(in) :: num\_threads
          end subroutine omp\_set\_num\_threads

          function omp\_get\_num\_threads ()
           integer :: omp\_get\_num\_threads
          end function omp\_get\_num\_threads

          function omp\_get\_max\_threads ()
           integer :: omp\_get\_max\_threads
          end function omp\_get\_max\_threads

          function omp\_get\_thread\_num ()
           integer :: omp\_get\_thread\_num
          end function omp\_get\_thread\_num

          function omp\_get\_num\_procs ()
           integer :: omp\_get\_num\_procs
          end function omp\_get\_num\_procs

          function omp\_in\_parallel ()
           logical :: omp\_in\_parallel
          end function omp\_in\_parallel

          subroutine omp\_set\_dynamic (dynamic\_threads)
           logical, intent(in) ::dynamic\_threads
          end subroutine omp\_set\_dynamic

          function omp\_get\_dynamic ()
           logical :: omp\_get\_dynamic
          end function omp\_get\_dynamic

          function omp\_get\_cancellation ()
           logical :: omp\_get\_cancellation
          end function omp\_get\_cancellation

          subroutine omp\_set\_nested (nested)
           logical, intent(in) :: nested
          end subroutine omp\_set\_nested

          function omp\_get\_nested ()
           logical :: omp\_get\_nested
          end function omp\_get\_nested

          subroutine omp\_set\_schedule (kind, chunk\_size)
           use omp\_lib\_kinds
           integer(kind=omp\_sched\_kind), intent(in) :: kind
           integer, intent(in) :: chunk\_size
          end subroutine omp\_set\_schedule

          subroutine omp\_get\_schedule (kind, chunk\_size)
           use omp\_lib\_kinds
           integer(kind=omp\_sched\_kind), intent(out) :: kind
           integer, intent(out)::chunk\_size
          end subroutine omp\_get\_schedule

          function omp\_get\_thread\_limit ()
           integer :: omp\_get\_thread\_limit
          end function omp\_get\_thread\_limit

          subroutine omp\_set\_max\_active\_levels (max\_levels)
           integer, intent(in) :: max\_levels
          end subroutine omp\_set\_max\_active\_levels

          function omp\_get\_max\_active\_levels ()
           integer :: omp\_get\_max\_active\_levels
          end function omp\_get\_max\_active\_levels

          function omp\_get\_level()
           integer :: omp\_get\_level
          end function omp\_get\_level

          function omp\_get\_ancestor\_thread\_num (level)
           integer, intent(in) :: level
           integer :: omp\_get\_ancestor\_thread\_num
          end function omp\_get\_ancestor\_thread\_num

          function omp\_get\_team\_size (level)
           integer, intent(in) :: level
           integer :: omp\_get\_team\_size
          end function omp\_get\_team\_size

          function omp\_get\_active\_level ()
           integer :: omp\_get\_active\_level
          end function omp\_get\_active\_level

          function omp\_in\_final ()
           logical :: omp\_in\_final
          end function omp\_in\_final

          function omp\_get\_proc\_bind ()
           use omp\_lib\_kinds
           integer(kind=omp\_proc\_bind\_kind) :: omp\_get\_proc\_bind
           omp\_get\_proc\_bind = omp\_proc\_bind\_false
          end function omp\_get\_proc\_bind

          function omp\_get\_num\_places ()
           integer :: omp\_get\_num\_places
          end function omp\_get\_num\_places

          function omp\_get\_place\_num\_procs (place\_num)
           integer, intent(in) :: place\_num
           integer :: omp\_get\_place\_num\_procs
          end function omp\_get\_place\_num\_procs

          subroutine omp\_get\_place\_proc\_ids (place\_num, ids)
           integer, intent(in) :: place\_num
           integer, intent(out) :: ids(*)
          end subroutine omp\_get\_place\_proc\_ids

          function omp\_get\_place\_num ()
           integer :: omp\_get\_place\_num
          end function omp\_get\_place\_num

          function omp\_get\_partition\_num\_places ()
           integer :: omp\_get\_partition\_num\_places
          end function omp\_get\_partition\_num\_places

          subroutine omp\_get\_partition\_place\_nums (place\_nums)
           integer, intent(out) :: place\_nums(*)
          end subroutine omp\_get\_partition\_place\_nums

          subroutine omp\_set\_affinity\_format(format)
           character(len=*),intent(in)::format
          end subroutine omp\_set\_affinity\_format

          function omp\_get\_affinity\_format(buffer)
           character(len=*),intent(out)::buffer
           integer :: omp\_get\_affinity\_format
          end function omp\_get\_affinity\_format

          subroutine omp\_display\_affinity(format)
           character(len=*),intent(in)::format
          end subroutine omp\_display\_affinity

          function omp\_capture\_affinity(buffer,format)
           character(len=*),intent(out)::buffer
           character(len=*),intent(in)::format
           integer :: omp\_capture\_affinity
          end function omp\_capture\_affinity

          subroutine omp\_set\_default\_device (device\_num)
           integer :: device\_num
          end subroutine omp\_set\_default\_device

          function omp\_get\_default\_device ()
           integer :: omp\_get\_default\_device
          end function omp\_get\_default\_device

          function omp\_get\_num\_devices ()
           integer :: omp\_get\_num\_devices
          end function omp\_get\_num\_devices

          function omp\_get\_device\_num ()
           integer :: omp\_get\_device\_num
          end function omp\_get\_device\_num

          function omp\_get\_num\_teams ()
           integer :: omp\_get\_num\_teams
          end function omp\_get\_num\_teams

          function omp\_get\_team\_num ()
           integer :: omp\_get\_team\_num
          end function omp\_get\_team\_num

          function omp\_is\_initial\_device ()
           logical :: omp\_is\_initial\_device
          end function omp\_is\_initial\_device

          function omp\_get\_initial\_device ()
           integer :: omp\_get\_initial\_device
          end function omp\_get\_initial\_device

          function omp\_get\_max\_task\_priority ()
           integer :: omp\_get\_max\_task\_priority
          end function omp\_get\_max\_task\_priority

          subroutine omp\_init\_lock (svar)
           use omp\_lib\_kinds
           integer(kind=omp\_lock\_kind), intent(out) :: svar
          end subroutine omp\_init\_lock

          subroutine omp\_init\_lock\_with\_hint (svar, hint)
           use omp\_lib\_kinds
           integer(kind=omp\_lock\_kind), intent(out) :: svar
           integer(kind=omp\_lock\_hint\_kind), intent(in) :: hint
          end subroutine omp\_init\_lock\_with\_hint

          subroutine omp\_destroy\_lock (svar)
           use omp\_lib\_kinds
           integer(kind=omp\_lock\_kind), intent(inout) :: svar
          end subroutine omp\_destroy\_lock

          subroutine omp\_set\_lock (svar)
           use omp\_lib\_kinds
           integer(kind=omp\_lock\_kind), intent(inout) :: svar
          end subroutine omp\_set\_lock

          subroutine omp\_unset\_lock (svar)
           use omp\_lib\_kinds
           integer(kind=omp\_lock\_kind), intent(inout) :: svar
          end subroutine omp\_unset\_lock

          function omp\_test\_lock (svar)
           use omp\_lib\_kinds
           logical :: omp\_test\_lock
           integer(kind=omp\_lock\_kind), intent(inout) :: svar
          end function omp\_test\_lock

          subroutine omp\_init\_nest\_lock (nvar)
           use omp\_lib\_kinds
           integer(kind=omp\_nest\_lock\_kind), intent(out) :: nvar
          end subroutine omp\_init\_nest\_lock

          subroutine omp\_init\_nest\_lock\_with\_hint (nvar, hint)
           use omp\_lib\_kinds
           integer(kind=omp\_nest\_lock\_kind), intent(out) :: nvar
           integer(kind=omp\_lock\_hint\_kind), intent(in) :: hint
          end subroutine omp\_init\_nest\_lock\_with\_hint

          subroutine omp\_destroy\_nest\_lock (nvar)
           use omp\_lib\_kinds
           integer(kind=omp\_nest\_lock\_kind), intent(inout) :: nvar
          end subroutine omp\_destroy\_nest\_lock

          subroutine omp\_set\_nest\_lock (nvar)
           use omp\_lib\_kinds
           integer(kind=omp\_nest\_lock\_kind), intent(inout) :: nvar
          end subroutine omp\_set\_nest\_lock

          subroutine omp\_unset\_nest\_lock (nvar)
           use omp\_lib\_kinds
           integer(kind=omp\_nest\_lock\_kind), intent(inout) :: nvar
          end subroutine omp\_unset\_nest\_lock

          function omp\_test\_nest\_lock (nvar)
           use omp\_lib\_kinds
           integer :: omp\_test\_nest\_lock
           integer(kind=omp\_nest\_lock\_kind), intent(inout) :: nvar
          end function omp\_test\_nest\_lock

          function omp\_get\_wtick ()
           double precision :: omp\_get\_wtick
          end function omp\_get\_wtick

          function omp\_get\_wtime ()
           double precision :: omp\_get\_wtime
          end function omp\_get\_wtime
\begin{samepage}
          function omp\_control\_tool (command, modifier)
           use omp\_lib\_kinds
           integer :: omp\_control\_tool 
           integer(kind=omp\_control\_tool\_kind), intent(in) :: command
           integer(kind=omp\_control\_tool\_kind), intent(in) :: modifier
          end function omp\_control\_tool
\end{samepage}

          subroutine omp\_set\_default\_allocator (svar)
           use omp\_lib\_kinds
           integer(kind=omp\_allocator\_kind), intent(in) :: svar
          end subroutine omp\_set\_default\_allocator

          function omp\_get\_default\_allocator ()
           use omp\_lib\_kinds
           integer(kind=omp\_allocator\_kind) :: omp\_get\_default\_allocator
          end function omp\_get\_default\_allocator

          subroutine omp\_set\_target\_default\_allocator (svar)
           use omp\_lib\_kinds
           integer(kind=omp\_allocator\_kind), intent(in) :: svar
          end subroutine omp\_set\_target\_default\_allocator

          function omp\_get\_target\_default\_allocator ()
           use omp\_lib\_kinds
           integer(kind=omp\_allocator\_kind) :: omp\_get\_target\_default\_allocator
          end function omp\_get\_target\_default\_allocator

          end interface

        end module omp\_lib
\end{codepar}} % end of \small block






\pagebreak
\section{Example of a Generic Interface for a Library Routine}
\label{sec:Example of a Generic Interface for a Library Routine}
Any of the OpenMP runtime library routines that take an argument may be extended 
with a generic interface so arguments of different \code{KIND} type can be accommodated.

The \code{OMP\_SET\_NUM\_THREADS} interface could be specified in the \code{omp\_lib} module 
as follows:

\begin{boxedcode}
interface omp\_set\_num\_threads

    subroutine omp\_set\_num\_threads_4(num\_threads)
      use omp\_lib\_kinds
      integer(4), intent(in) :: num\_threads
    end subroutine omp\_set\_num\_threads_4

    subroutine omp\_set\_num\_threads_8(num\_threads)
      use omp\_lib\_kinds
      integer(8), intent(in) :: num\_threads
    end subroutine omp\_set\_num\_threads_8

end interface omp\_set\_num\_threads
\end{boxedcode}

% This is the end of appendix-C-InterfaceDeclarations.tex


    % This is implementation_defined.tex (Appendix C) of the OpenMP specification.
% This is an included file. See the master file for more information.
%
% When editing this file:
%
%    1. To change formatting, appearance, or style, please edit openmp.sty.
%
%    2. Custom commands and macros are defined in openmp.sty.
%
%    3. Be kind to other editors -- keep a consistent style by copying-and-pasting to
%       create new content.
%
%    4. We use semantic markup, e.g. (see openmp.sty for a full list):
%         \code{}     % for bold monospace keywords, code, operators, etc.
%         \plc{}      % for italic placeholder names, grammar, etc.
%
%    5. There are environments that provide special formatting, e.g. language bars.
%       Please use them whereever appropriate.  Examples are:
%
%         \begin{fortranspecific}
%         This is text that appears enclosed in blue language bars for Fortran.
%         \end{fortranspecific}
%
%         \begin{note}
%         This is a note.  The "Note -- " header appears automatically.
%         \end{note}
%
%    6. Other recommendations:
%         Use the convenience macros defined in openmp.sty for the minor headers
%         such as Comments, Syntax, etc.
%
%         To keep items together on the same page, prefer the use of
%         \begin{samepage}.... Avoid \parbox for text blocks as it interrupts line numbering.
%         When possible, avoid \filbreak, \pagebreak, \newpage, \clearpage unless that's
%         what you mean. Use \needspace{} cautiously for troublesome paragraphs.
%
%         Avoid absolute lengths and measures in this file; use relative units when possible.
%         Vertical space can be relative to \baselineskip or ex units. Horizontal space
%         can be relative to \linewidth or em units.
%
%         Prefer \emph{} to italicize terminology, e.g.:
%             This is a \emph{definition}, not a placeholder.
%             This is a \plc{var-name}.
%

\chapter{OpenMP Implementation-Defined Behaviors}
\label{chap:OpenMP Implementation-Defined Behaviors}
\index{implementation}
This appendix summarizes the behaviors that are described as implementation defined in
this API. Each behavior is cross-referenced back to its description in the main
specification. An implementation is required to define and document its behavior in
these cases.

\begin{itemize}
\item \textbf{Processor}: a hardware unit that is implementation defined (see
\specref{subsec:Threading Concepts}).

\item \textbf{Device}: an implementation defined logical execution engine (see
\specref{subsec:Threading Concepts}).

\item \textbf{Device address}: reference to an address in a \emph{device data environment} (see \specref{subsec:Data Terminology}).

\item \textbf{Memory model}: the minimum size at which a memory update may also read and
write back adjacent variables that are part of another variable (as array or structure
elements) is implementation defined but is no larger than required by the base
language (see \specref{subsec:Structure of the OpenMP Memory Model}).

\item \code{requires} \textbf{directive}: support of requirements is implementation defined. All implementation-defined requirements should begin with \code{ext_} (see \specref{sec:requires Directive}).

\item \textbf{Requires directive}: Support for any feature specified by a
requirement clause on a \code{requires} directive is implementation
defined (see \specref{sec:requires Directive}).

\index{internal control variables}
\item \textbf{Internal control variables}: the initial values of
\plc{dyn-var}, \plc{nest-var}, \plc{nthreads-var}, \plc{run-sched-var},
\plc{def-sched-var}, \plc{bind-var}, \plc{stacksize-var},
\plc{wait-policy-var}, \plc{thread-limit-var},
\plc{max-active-levels-var}, \plc{place-partition-var}, \plc{affinity-format-var},
\plc{default-device-var} and \plc{def-allocator-var} are implementation defined.  The method for
initializing a target device's internal control variable is
implementation defined (see \specref{subsec:ICV Initialization}).

\item \textbf{OpenMP context}: the accepted \plc{isa-name} values for
  the \plc{isa} trait, the accepted \plc{arch-name} values for the
  \plc{arch} trait, and the accepted \plc{extension-name} values for
  the \plc{extension} trait are implementation defined (see
  \specref{subsec:OpenMP Context}).

\item \code{declare}~\code{variant} \textbf{directive}: whether, for some specific
  OpenMP context, the prototype of the variant should differ from that of
  the base function, and if so how it should differ, is implementation defined
  (see \specref{subsec:declare variant Directive}).

\index{dynamic thread adjustment}
\item \textbf{Dynamic adjustment of threads}: providing the ability to dynamically adjust the
number of threads is implementation defined. Implementations are allowed to deliver
fewer threads (but at least one) than indicated in Algorithm 2.1 even if dynamic
adjustment is disabled (see \specref{subsec:Determining the Number of Threads for a parallel Region}).

\item \textbf{Thread affinity}: For the \code{close} thread affinity
  policy, if $T > P$ and $P$ does not divide $T$ evenly, the exact
  number of threads in a particular place is implementation defined.
  For the \code{spread} thread affinity, if $T > P$ and $P$ does not
  divide $T$ evenly, the exact number of threads in a particular
  subpartition is implementation defined.  The determination of
  whether the affinity request can be fulfilled is implementation
  defined.  If not, the mapping of threads in the team 
  to places is implementation defined (see
  \specref{subsec:Controlling OpenMP Thread Affinity}).

\item \code{teams} \textbf{construct}: the number of teams that are created is implementation defined but
less than or equal to the value of the \code{num_teams} clause if specified. The maximum
number of threads participating in the contention group that each team initiates is
implementation defined but less than or equal to the value of the \code{thread_limit}
clause if specified.  The assignment of the initial threads to places and
the values of the \plc{place-partition-var} and
\plc{default-device-var} ICVs for each initial thread are
implementation defined (see \specref{sec:teams Construct}).

\item \code{sections} \textbf{construct}: the method of scheduling the structured blocks among threads
in the team is implementation defined (see \specref{subsec:sections Construct}).

\item \code{single} \textbf{construct}: the method of choosing a thread to execute the structured block
is implementation defined (see \specref{subsec:single Construct})

\item \textbf{Worksharing-Loop directive}: the integer type (or kind, for Fortran) used to compute the iteration
count of a collapsed loop is implementation defined. The effect of the
\code{schedule(runtime)} clause when the \plc{run-sched-var} ICV is set to \code{auto} is
implementation defined.
The value of \plc{simd_width} for the \code{simd} schedule modifier is
implementation defined (see \specref{subsec:Worksharing-Loop
  Construct}).

\item \code{simd} \textbf{construct}: the integer type (or kind, for
  Fortran) used to compute the iteration count for the collapsed loop
  is implementation defined. The number of iterations that are
  executed concurrently at any given time is implementation
  defined. If the \plc{alignment} parameter is not specified in the
  \code{aligned} clause, the default alignments for the SIMD
  instructions are implementation defined (see \specref{subsubsec:simd Construct}).

\item \code{declare simd} \textbf{directive}: if the parameter of the
  \code{simdlen} clause is not a constant positive integer expression,
  the number of concurrent arguments for the function is
  implementation defined. If the \plc{alignment} parameter of the
  \code{aligned} clause is not specified, the default alignments for
  SIMD instructions are implementation defined (see
  \specref{subsubsec:declare simd Directive}).

\item \code{distribute} \textbf{construct}: the integer type (or kind, for
    Fortran) used to compute the iteration count for the collapsed loop is
    implementation defined.  If no \code{dist_schedule} clause is specified then the schedule for the \code{distribute}
construct is implementation defined (see \specref{subsec:distribute Construct}).

\item \code{taskloop} \textbf{construct}: The number of loop
  iterations assigned to a task created from a \code{taskloop}
  construct is implementation defined, unless the \code{grainsize} or
  \code{num_tasks} clause is specified. The integer type (or kind,
  for Fortran) used to compute the iteration count for the collapsed
  loop is implementation defined (see \specref{subsec:taskloop Construct}).

\begin{cppspecific}
\item \code{taskloop} \textbf{construct}: For \code{firstprivate} variables of class type, the number of invocations of copy constructors to perform the initialization is implementation defined 
(see \specref{subsec:taskloop Construct}).
\end{cppspecific}

\item \textbf{Memory spaces}: The actual storage resource that each memory space defined in \tabref{tab:Predefined Memory Spaces} represents is implementation defined.

\item \textbf{Memory allocators}: The minimum partitioning size for partitioning of allocated memory over the storage resources is implementation defined (see \specref{subsec:Memory Allocators}).
The default value for the \code{pool_size} allocator trait is implementation defined (see \tabref{tab:Allocator traits}).
The associated memory space for each of the predefined \code{omp_cgroup_mem_alloc}, \code{omp_pteam_mem_alloc} and \code{omp_thread_mem_alloc} allocators is implementation defined (see  \tabref{tab:Predefined Allocators}).

\item \code{allocate} \textbf{directive}: The effect of not being able to fulfill an allocation request specified in \code{allocate} directive is implementation defined (see \specref{subsec:allocate Directive}).

\item \code{allocate} \textbf{clause}: The effect of not being able to fulfill an allocation request specified in \code{allocate} clause is implementation defined (see \specref{subsec:allocate Clause}).

\item \code{is_device_ptr} \textbf{clause}:
  Support for pointers created outside of the OpenMP device data management
  routines is implementation defined (see \specref{subsec:target Construct}).

\item \code{target} \textbf{construct}: the effect of invoking a virtual member
function of an object on a device other than the device on which the object was
constructed is implementation defined (see \specref{subsec:target Construct}).

\item \code{critical} \textbf{construct}: the effect of using a \code{hint}
clause is implementation defined (see \specref{subsec:critical Construct} and \specref{subsec:Synchronization Hints}).

\item \code{atomic} \textbf{construct}: a compliant implementation may enforce exclusive access
\index{atomic construct@{\code{atomic} construct}}
between \code{atomic} regions that update different storage locations. The circumstances
under which this occurs are implementation defined. If the storage location
designated by $x$ is not size-aligned (that is, if the byte alignment of $x$ is not a multiple
of the size of $x$), then the behavior of the atomic region is implementation defined (see \specref{subsec:atomic Construct}).
The effect of using a \code{hint} clause is implementation defined (see \specref{subsec:atomic Construct} and \specref{subsec:Synchronization Hints}).

\item \textbf{Synchronization hints}: The effect of the combined hint
  is implementation defined (see \specref{subsec:Synchronization Hints}).

\begin{fortranspecific}

\item \textbf{Data-sharing attributes}: The data-sharing attributes of dummy arguments without the \code{VALUE} attribute are implementation-defined if the associated actual argument is shared, except for the conditions specified (see
\specref{subsubsec:Data-sharing Attribute Rules for Variables Referenced in a Region but not in a Construct}).

\item \code{threadprivate} \textbf{directive}: if the conditions for values of data in the threadprivate
objects of threads (other than an initial thread) to persist between two consecutive
active parallel regions do not all hold, the allocation status of an allocatable variable
in the second region is implementation defined (see \specref{subsec:threadprivate Directive}).

\item \textbf{Runtime library definitions}: it is implementation defined whether the include file
\code{omp_lib.h} or the module \code{omp_lib} (or both) is provided. It is implementation
defined whether any of the OpenMP runtime library routines that take an argument
are extended with a generic interface so arguments of different \code{KIND} type can be
accommodated (see \specref{sec:runtime library definitions}).

\end{fortranspecific}

\item \code{omp_set_num_threads} \textbf{routine}: if the argument is not a positive integer the
behavior is implementation defined (see \specref{subsec:omp_set_num_threads}).

\item \code{omp_set_schedule} \textbf{routine}: for implementation specific schedule types, the
values and associated meanings of the second argument are implementation defined.
(see \specref{subsec:omp_set_schedule}).

\item \code{omp_set_max_active_levels} \textbf{routine}: when called from within any explicit
\code{parallel} region the binding thread set (and binding region, if required) for the
\code{omp_set_max_active_levels} region is implementation defined and the
behavior is implementation defined. If the argument is not a non-negative integer
then the behavior is implementation defined (see \specref{subsec:omp_set_max_active_levels}).

\item \code{omp_get_max_active_levels} \textbf{routine}: when called from within any explicit
\code{parallel} region the binding thread set (and binding region, if required) for the
\code{omp_get_max_active_levels} region is implementation defined (see
\specref{subsec:omp_get_max_active_levels}).

\item \code{omp_get_place_proc_ids} \textbf{routine}: the meaning of the
non-negative numerical identifiers returned by the
\code{omp_get_place_proc_ids} routine is implementation defined. The
order of the numerical identifiers returned in the array \plc{ids} is
implementation defined (see
\specref{subsec:omp_get_place_proc_ids}).

\item \code{omp_set_affinity_format} \textbf{routine}: when called from within any explicit
\code{parallel} region, the binding thread set (and binding region, if required) for the
\code{omp_set_affinity_format} region is implementation defined and the
behavior is implementation defined. If the argument does not
conform to the specified format then the result is implementation defined (see
\specref{subsec:omp_set_affinity_format}).

\item \code{omp_get_affinity_format} \textbf{routine}: when called from within any explicit
\code{parallel} region the binding thread set (and binding region, if required) for the
\code{omp_get_affinity_format} region is implementation defined (see
\specref{subsec:omp_get_affinity_format}).

\item \code{omp_display_affinity} \textbf{routine}:  if the argument does not
conform to the specified format then the result is implementation defined (see
\specref{subsec:omp_display_affinity}).

\item \code{omp_capture_affinity} \textbf{routine}:  if the \plc{format} argument does not
conform to the specified format then the result is implementation defined (see
\specref{subsec:omp_capture_affinity}).

\item \code{omp_get_initial_device} \textbf{routine}: the value of
  the device number of the host device is implementation defined (see \specref{subsec:omp_get_initial_device}).

\item \code{omp_init_lock_with_hint} and \code{omp_init_nest_lock_with_hint} \textbf{routines}:
if hints are stored with a lock variable, the effect of the hints on the locks are implementation defined
(see \specref{subsec:Synchronization Hints} and \specref{subsec:omp_init_lock_with_hint and omp_init_nest_lock_with_hint}).

\item \code{omp_target_memcpy_rect} \textbf{routine}:
  the maximum number of dimensions supported is implementation defined, but
  must be at least three (see \specref{subsec:omp_target_memcpy_rect}).

\item
\code{ompt_callback_sync_region_wait},
\code{ompt_callback_mutex_released},
\code{ompt_callback_task_dependences},
\code{ompt_callback_task_dependence},
\code{ompt_callback_work},
\code{ompt_callback_master},
\code{ompt_callback_target_map},
\code{ompt_callback_sync_region},
\code{ompt_callback_lock_init},
\code{ompt_callback_lock_destroy},
\code{ompt_callback_mutex_acquire},
\code{ompt_callback_mutex_acquired},
\code{ompt_callback_nest_lock},
\code{ompt_callback_flush},
\code{ompt_callback_cancel} and
\code{ompt_callback_dispatch}
\textbf{tool callbacks}: if a tool attempts to register a callback with the string name using the runtime entry point \code{ompt_set_callback}, it is implementation defined whether the registered callback may never or sometimes invoke this callback for the associated events (see \tabref{table:valid_rc})

\item \textbf{Device tracing}: Whether a target device supports tracing or not is implementation defined;
if a target device does not support tracing, a \code{NULL} may be supplied for the \plc{lookup} function
to a tool's device initializer (see \specref{sec:tracing-device-activity}).

\newcommand{\ompttrace}[1]{
\item \code{#1} \textbf{runtime entry point}: it is implementation defined whether a device-specific tracing interface will define this
runtime entry point, indicating that it can collect traces in OMPT format (see \specref{sec:tracing-device-activity}).
}

\ompttrace{ompt_set_trace_ompt}
\ompttrace{ompt_buffer_get_record_ompt}

\item \code{ompt_callback_target_data_op_t} \textbf{callback type}:
it is implementation defined whether in some operations \plc{src_addr} or \plc{dest_addr} might point to an intermediate buffer
(see \specref{sec:ompt_callback_target_data_op_t}).

\item \code{ompt_get_place_proc_ids_t} \textbf{entry point type}:
the meaning of the numerical identifiers returned is implementation defined.  The order of \plc{ids} returned in the array is implementation defined (see \specref{sec:ompt_get_place_proc_ids_t}).

\item \code{ompt_get_partition_place_nums_t} \textbf{entry point type}:
  the order of the identifiers returned in the array \plc{place_nums}
  is implementation defined (see
  \specref{sec:ompt_get_partition_place_nums_t}).

\item \code{ompt_get_proc_id_t} \textbf{entry point type}:
the meaning of the numerical identifier returned is implementation defined (see \specref{sec:ompt_get_proc_id_t}).

\item \code{ompd_callback_print_string_fn_t} \textbf{callback function}:
the value of \plc{catergory} is implementation defined (see
\specref{subsubsubsec:ompd_callback_print_string_fn_t}).

\item \code{ompd_parallel_handle_compare} \textbf{operation}:
the means by which parallel region handles are ordered is implementation defined (see \specref{subsubsubsec:ompd_parallel_handle_compare}).

\item \code{ompd_task_handle_compare} \textbf{operation}:
the means by which task handles are ordered is implementation defined (see
\specref{subsubsubsec:ompd_task_handle_compare}).

\item \textbf{OMPT thread states}: The set of OMPT thread states supported is implementation defined (see \specref{sec:thread-states}).

\item \code{OMP_SCHEDULE} \textbf{environment variable}: if the value does not
conform to the specified format then the result is implementation defined (see
\specref{sec:OMP_SCHEDULE}).

\item \code{OMP_NUM_THREADS} \textbf{environment variable}: if any value of the list specified
leads to a number of threads that is
greater than the implementation can support, or if any value is not a positive integer,
then the result is implementation defined (see \specref{sec:OMP_NUM_THREADS}).

\item \code{OMP_DYNAMIC} \textbf{environment variable}: if the value is neither
\code{true} nor \code{false} the behavior is implementation defined (see
\specref{sec:OMP_DYNAMIC}).

\item \code{OMP_PROC_BIND} \textbf{environment variable}: if the value is not \code{true}, \code{false}, or a
comma separated list of \code{master}, \code{close}, or \code{spread}, the behavior is
implementation defined. The behavior is also implementation defined if an initial
thread cannot be bound to the first place in the OpenMP place list (see
\specref{sec:OMP_PROC_BIND}).

\item \code{OMP_PLACES} \textbf{environment variable}: the meaning of the numbers specified in the
environment variable and how the numbering is done are implementation defined.
The precise definitions of the abstract names are implementation defined. An
implementation may add implementation-defined abstract names as appropriate for
the target platform. When creating a place list of n elements by appending the
number $n$ to an abstract name, the determination of which resources to include in the
place list is implementation defined. When requesting more resources than available,
the length of the place list is also implementation defined. The behavior of the
program is implementation defined when the execution environment cannot map a
numerical value (either explicitly defined or implicitly derived from an interval)
within the \code{OMP_PLACES} list to a processor on the target platform, or if it maps to an
unavailable processor. The behavior is also implementation defined when the
\code{OMP_PLACES} environment variable is defined using an abstract name (see \specref{sec:OMP_PLACES}).

\item \code{OMP_NESTED} \textbf{environment variable}: if the value is neither
\code{true} nor \code{false} the behavior is implementation defined (see
\specref{sec:OMP_NESTED}).

\item \code{OMP_STACKSIZE} \textbf{environment variable}: if the value does not conform to the
specified format or the implementation cannot provide a stack of the specified size
then the behavior is implementation defined (see \specref{sec:OMP_STACKSIZE}).

\item \code{OMP_WAIT_POLICY} \textbf{environment variable}: the details of the \code{ACTIVE} and
\code{PASSIVE} behaviors are implementation defined (see \specref{sec:OMP_WAIT_POLICY}).

\item \code{OMP_MAX_ACTIVE_LEVELS} \textbf{environment variable}: if the value is not a
non-negative integer or is greater than the number of parallel levels an implementation
can support then the behavior is implementation defined (see \specref{sec:OMP_MAX_ACTIVE_LEVELS}).

\item \code{OMP_THREAD_LIMIT} \textbf{environment variable}: if the requested value is greater than
the number of threads an implementation can support, or if the value is not a positive
integer, the behavior of the program is implementation defined (see \specref{sec:OMP_THREAD_LIMIT}).

\item \code{OMP_DISPLAY_AFFINITY} \textbf{environment variable}: for all values of the environment variables other than \code{TRUE} or \code{FALSE}, the display action is implementation defined (see
\specref{sec:OMP_DISPLAY_AFFINITY}).

\item \code{OMP_AFFINITY_FORMAT} \textbf{environment variable}: if the value does not
conform to the specified format then the result is implementation defined (see
\specref{sec:OMP_AFFINITY_FORMAT}).

\item \code{OMP_TARGET_OFFLOAD} \textbf{environment variable}: the support of \code{disabled} is implementation defined (see
\specref{sec:OMP_TARGET_OFFLOAD}).

\item \code{OMP_DEBUG} \textbf{environment variable}: if the value is neither
\code{disabled} nor \code{enabled} the behavior is implementation defined (see
\specref{sec:OMP_DEBUG}).

\end{itemize}


% This is the end of appendix-D-ImplementationDefined.tex of the OpenMP specification.


    % This is tool_support_frames.tex (Appendix D) of the OpenMP specification.
% This is an included file. See the master file for more information.
%
% When editing this file:
%
%    1. To change formatting, appearance, or style, please edit openmp.sty.
%
%    2. Custom commands and macros are defined in openmp.sty.
%
%    3. Be kind to other editors -- keep a consistent style by copying-and-pasting to
%       create new content.
%
%    4. We use semantic markup, e.g. (see openmp.sty for a full list):
%         \code{}     % for bold monospace keywords, code, operators, etc.
%         \plc{}      % for italic placeholder names, grammar, etc.
%
%    5. There are environments that provide special formatting, e.g. language bars.
%       Please use them whereever appropriate.  Examples are:
%
%         \begin{fortranspecific}
%         This is text that appears enclosed in blue language bars for Fortran.
%         \end{fortranspecific}
%
%         \begin{note}
%         This is a note.  The "Note -- " header appears automatically.
%         \end{note}
%
%    6. Other recommendations:
%         Use the convenience macros defined in openmp.sty for the minor headers
%         such as Comments, Syntax, etc.
%
%         To keep items together on the same page, prefer the use of
%         \begin{samepage}.... Avoid \parbox for text blocks as it interrupts line numbering.
%         When possible, avoid \filbreak, \pagebreak, \newpage, \clearpage unless that's
%         what you mean. Use \needspace{} cautiously for troublesome paragraphs.
%
%         Avoid absolute lengths and measures in this file; use relative units when possible.
%         Vertical space can be relative to \baselineskip or ex units. Horizontal space
%         can be relative to \linewidth or em units.
%
%         Prefer \emph{} to italicize terminology, e.g.:
%             This is a \emph{definition}, not a placeholder.
%             This is a \plc{var-name}.
%


\chapter{Task Frame Management for the Tool Interface}
\index{task frame management}
\label{chap:frames}


   \begin{figure}[h]
    \centering
        \includegraphics[width=4in]{appendices/callstack-cropped.pdf}
    \caption{Thread call stacks implementing nested parallelism
      annotated with frame information for the OMPT tool interface.}
    \label{fig:frame}
\end{figure}

The top half of Figure~\ref{fig:frame} illustrates a
conceptualization of a program executing a nested
parallel region, where code A, B, and C represent, respectively, one
or more procedure frames of code
associated with an initial task, an outer parallel region, and an inner parallel
region. The bottom half of Figure~\ref{fig:frame} illustrates the stacks of two
threads executing the nested parallel region.
In the illustration, stacks grow downward---a call to a function adds
a new frame to the stack below the frame of its caller.
When thread 1 encounters the outer-parallel
region ``b", it calls a routine in the OpenMP runtime to
create a new parallel region. The OpenMP runtime sets the
\plc{enter_frame} field in the \code{omp_frame_t} for the initial
task executing code A to the canonical frame address of frame f1---the user frame in the initial task
that calls the runtime. The \code{omp_frame_t} for the initial task
is labeled \plc{r1} in Figure~\ref{fig:frame}. In this figure, three
consecutive runtime system frames, labeled ``par'' with frame
identifiers f2--f4, are on the stack.  Before starting the implicit
task for parallel region ``b" in thread 1, the runtime sets the
\plc{exit_frame} in the implicit task's \code{omp_frame_t} (labeled
\plc{r2}) to the canonical frame address of frame f4. Execution of application code for parallel region
``b'' begins on thread 1 when the runtime system invokes application
code B (frame f5) from frame f4.

Let us focus now on thread 2, an OpenMP thread. Figure~\ref{fig:frame}
shows this worker executing work for the outer-parallel region ``b."
On the OpenMP thread's stack is a runtime frame labeled ``idle,''
where the OpenMP thread waits for work.  When work becomes available,
the runtime system invokes a function to dispatch it. While
dispatching parallel work might involve a chain of several calls, here
we assume that the length of this chain is 1 (frame f7).  Before
thread 2 exits the runtime to execute an implicit task for parallel
region ``b,'' the runtime sets the \plc{exit_frame} field of the
implicit task's \code{omp_frame_t} (labeled \plc{r3}) to the canonical frame address of frame f7.
When thread 2 later encounters the inner-parallel region ``c," as
execution returns to the runtime, the runtime fills in the
\plc{enter_frame} field of the current task's \code{omp_frame_t}
(labeled \plc{r3}) to the canonical frame address of frame f8---the frame that invoked the
runtime. Before the task for parallel region ``c'' is invoked on
thread 2, the runtime system sets the \plc{exit_frame} field of the
\code{omp_frame_t} (labeled \plc{r4}) for the implicit task for
``c'' to the canonical frame address of frame f11. Execution of application code for parallel region
``c'' begins on thread 2 when the runtime system invokes application
code C (frame f12) from frame f11.


Below the stack for each thread in Figure~\ref{fig:frame}, the figure
shows the \code{omp_frame_t} information obtained by calls to
\code{ompt_get_task_info} made on each thread for the stack state
shown. We show the ID of the \code{omp_frame_t} object returned at
each ancestor level. Note that thread 2 has task frame information for
three levels of tasks, whereas thread 1 has only two.

\crossreferences
\begin{itemize}
\item \code{ompt_get_task_info_t}, see \specref{sec:ompt_get_task_info_t}.\
\item \code{omp_frame_t}, see \specref{sec:omp_frame_t}.
\end{itemize}


% This is the end of appendix-frames.tex


    % This is features_history.tex (Appendix E) of the OpenMP specification.
% This is an included file. See the master file for more information.
%
% When editing this file:
%
%    1. To change formatting, appearance, or style, please edit openmp.sty.
%
%    2. Custom commands and macros are defined in openmp.sty.
%
%    3. Be kind to other editors -- keep a consistent style by copying-and-pasting to
%       create new content.
%
%    4. We use semantic markup, e.g. (see openmp.sty for a full list):
%         \code{}     % for bold monospace keywords, code, operators, etc.
%         \plc{}      % for italic placeholder names, grammar, etc.
%
%    5. There are environments that provide special formatting, e.g. language bars.
%       Please use them whereever appropriate.  Examples are:
%
%         \begin{fortranspecific}
%         This is text that appears enclosed in blue language bars for Fortran.
%         \end{fortranspecific}
%
%         \begin{note}
%         This is a note.  The "Note -- " header appears automatically.
%         \end{note}
%
%    6. Other recommendations:
%         Use the convenience macros defined in openmp.sty for the minor headers
%         such as Comments, Syntax, etc.
%
%         To keep items together on the same page, prefer the use of
%         \begin{samepage}.... Avoid \parbox for text blocks as it interrupts line numbering.
%         When possible, avoid \filbreak, \pagebreak, \newpage, \clearpage unless that's
%         what you mean. Use \needspace{} cautiously for troublesome paragraphs.
%
%         Avoid absolute lengths and measures in this file; use relative units when possible.
%         Vertical space can be relative to \baselineskip or ex units. Horizontal space
%         can be relative to \linewidth or em units.
%
%         Prefer \emph{} to italicize terminology, e.g.:
%             This is a \emph{definition}, not a placeholder.
%             This is a \plc{var-name}.%


\chapter{Features History}
\index{features history}
\index{history of features}
\label{chap:Features History}
This appendix summarizes the major changes between recent versions of the OpenMP
API since version 2.5.

\section{Deprecated Features}
\index{deprecated features}
\label{chap:Deprecated Features}

The following features have been deprecated in Version 5.0.

\begin{itemize}
\item The \plc{nest-var} ICV, the \code{OMP_NESTED} environment variable,
   and the \code{omp_set_nested} and \code{omp_get_nested} routines were deprecated.

\item Lock hints were renamed to synchronization hints. The following lock 
   hint type and constants were deprecated:
   \begin{itemize}
   \item the C/C++ type \code{omp_lock_hint_t} and the Fortran kind
      \code{omp_lock_hint_kind};
   \item the constants \code{omp_lock_hint_none},
      \code{omp_lock_hint_uncontended}, \code{omp_lock_hint_contended},
      \code{omp_lock_hint_nonspeculative}, and \code{omp_lock_hint_speculative}.
   \end{itemize}

\end{itemize}



\section{Version 4.5 to 5.0 Differences}
\label{sec:Version 4.5 to 5.0 Differences}
\begin{itemize}
\item The memory model was extended to distinguish different types of flush
      operations according to specified flush properties (see
      \specref{subsec:The Flush Operation}) and to define a happens
      before order based on synchronizing flush operations
      (see \specref{subsec:happens-before}).

\item Various changes throughout the specification were made to provide
      initial support of C11, C++11, C++14, C++17 and Fortran 2008 (see
      \specref{sec:normative references}).

\item Fortran 2003 is now fully supported (see
      \specref{sec:normative references}).

\item The \code{requires} directive (see \specref{sec:requires Directive}) was
      added to support applications that require implementation-specific
      features.

\item The \plc{target-offload-var} internal control variable (see
      \specref{sec:Internal Control Variables}) and the
      \code{OMP_TARGET_OFFLOAD} environment variable (see
      \specref{sec:OMP_TARGET_OFFLOAD}) were added to support runtime
      control of the execution of device constructs.

\item The default value of the \plc{nest-var} ICV was changed from \plc{false}
      to implementation defined (see \specref{subsec:ICV Initialization}).
      The \plc{nest-var} ICV (see \specref{subsec:ICV Descriptions}), the
      \code{OMP_NESTED} environment variable (see \specref{sec:OMP_NESTED}),
      and the \code{omp_set_nested} and \code{omp_get_nested} routines
      were deprecated (see \specref{subsec:omp_set_nested} and
      \specref{subsec:omp_get_nested}).

\item Support for array shaping (see \specref{sec:Array Shaping}) and 
      for array sections with non-unit strides  in C and C++ (see 
      \specref{sec:Array Sections}) were added to facilitate specification 
      of discontiguous storage and the \code{target}~\code{update} construct (see 
      \specref{subsec:target update Construct}) and the \code{depend} clause 
      (see \specref{subsec:depend Clause}) were extended to allow the use 
      of shape-operators (see \specref{sec:Array Shaping}).

\item Iterators (see \specref{sec:iterators}) were added to express that an
      expression in a list may expand to multiple expressions.

\item The \code{declare}~\code{variant} directive (see
      \specref{subsec:declare variant Directive}) and the \code{metadirective}
      directive (see \specref{subsec:Metadirective Meta-Directive}) were
      added to support selection of declared function variants at a callsite
      or directive variants, respectively, based on compile-time traits for
      the enclosing context.

\item The \code{teams} construct (see \specref{sec:teams Construct}) was
      extended to support execution on the host device without surrounding
      \code{target} construct (see \specref{subsec:target Construct}).

\item The canonical loop form was defined for Fortran and, for all base
    languages, extended to permit non-rectangular loop nests (see
    \specref{subsec:Canonical Loop Form}).

\item The \plc{relational-op} in the canonical loop form for C/C++ was
      extended to include \code{!=} (see \specref{subsec:Canonical Loop Form}).

\item The default loop schedule modifier for worksharing-loop constructs 
      without the \code{static} schedule and the \code{ordered} clause 
      was changed to \code{nonmonotonic} (see 
      \specref{subsec:Worksharing-Loop Construct}).

\item The collapse of associated loops that are imperfectly nested loops
      was defined for the worksharing-loop (see \specref{subsec:Worksharing-Loop Construct}),
      \code{simd} (see \specref{subsubsec:simd Construct}), \code{taskloop}
      (see \specref{subsec:taskloop Construct}) and \code{distribute} (see
      \specref{subsec:distribute simd Construct}) constructs.

\item The \code{simd} construct (see \specref{subsubsec:simd Construct}) was extended
      to allow the use of \code{atomic} constructs within it.

\item The \code{if} and \code{nontemporal} clauses were added to the 
      \code{simd} construct (see \specref{subsubsec:simd Construct}).

\item The \code{loop} construct and the \code{order(concurrent)} clause were added to support compiler
      optimization and parallelization of loops for which iterations may
      execute in any order, including concurrently (see \specref{subsec:loop Construct}).

\item The \code{scan} directive (see \specref{subsec:scan Directive}) and the
      \code{inscan} modifier for the \code{reduction} clause (see
      \specref{subsubsec:reduction clause}) were added to support
      inclusive and exclusive scan computations.

\item To support task reductions, the \code{task} (see
      \specref{subsec:task Construct}) and \code{target} (see
      \specref{subsec:target Construct}) constructs were extended to
      accept the the \code{in_reduction} clause (see
      \specref{subsubsec:in_reduction clause}), the \code{taskgroup}
      construct (see \specref{subsec:taskgroup Construct}) was extended
      to accept the \code{task_reduction} clause
      \specref{subsubsec:task_reduction clause}), and the \code{task}
      modifier was added to the \code{reduction} clause (see
      \specref{subsubsec:reduction clause}).

\item The \code{affinity} clause was added to the \code{task} construct
      (see \specref{subsec:task Construct}) to support hints that indicate 
      data affinity of explicit tasks.

\item The \code{detach} clause for the \code{task} construct (see 
      \specref{subsec:task Construct}) and the \code{omp_fulfill_event}
      runtime routine (see \specref{subsec:omp_fulfill_event}) were added 
      to support execution of detachable tasks.

\item To support taskloop reductions, the \code{taskloop} (see
      \specref{subsec:taskloop Construct}) and \code{taskloop simd} (see
      \specref{subsec:taskloop simd Construct}) constructs were extended
      to accept the \code{reduction} (see \specref{subsubsec:reduction clause})
      and \code{in_reduction} (see \specref{subsubsec:in_reduction clause})
      clauses.

\item To support mutually exclusive inout sets, a \code{mutexinoutset}
      \plc{dependence-type} was added to the \code{depend} clause (see
      \specref{subsec:Task Scheduling} and \specref{subsec:depend Clause}).

\item Predefined memory spaces (see \specref{subsec:Memory Spaces}), 
      predefined memory allocators and allocator traits (see 
      \specref{subsec:Memory Allocators}) and directives, clauses (see 
      \specref{sec:Memory Management Directives} and API routines (see 
      \specref{sec:Memory Management Routines}) to use them were added 
      to support different kinds of memories.

\item The semantics of the \code{use_device_ptr} clause for pointer variables
      was clarified and the \code{use_device_addr} clause for using the 
      device address of non-pointer variables inside the 
      \code{target}~\code{data} construct was added 
      (see \specref{subsec:target data Construct}).

\item To support reverse offload, the \code{ancestor} modifier was
      added to the \code{device} clause for \code{target} constructs (see
      \specref{subsec:target Construct}).

\item To reduce programmer effort implicit declare target directives for
      some functions (C, C++, Fortran) and subroutines (Fortran) were added
      (see \specref{subsec:target Construct} and
      \specref{subsec:declare target Directive}).

\item The \code{target}~\code{update} construct (see \specref{subsec:target update 
      Construct}) was modified to allow array sections that specify 
      discontiguous storage.

\item The \code{to} and \code{from} clauses on the \code{target}~\code{update}
    construct (see \specref{subsec:target update Construct}), the
    \code{depend} clause on task generating constructs (see
    \specref{subsec:depend Clause}), and the \code{map} clause (see
    \specref{subsec:map Clause}) were extended to allow any lvalue expression
    as a list item for C/C++.

\item Support for nested \code{declare}~\code{target} directives was added
      (see \specref{subsec:declare target Directive}).

\item The \code{implements} clause was added to the
      \code{declare}~\code{target} directive to support the use of
      device-specific function implementations (see
      \specref{subsec:declare target Directive}).

\item The \code{depend} clause was added to the \code{taskwait} construct
      (see \specref{subsec:taskwait Construct}).

\item To support acquire and release semantics with weak memory ordering, the
      \code{acq_rel}, \code{acquire}, and \code{release} clauses were added to
      the \code{atomic} construct (see \specref{subsec:atomic Construct}) and
      \code{flush} construct (see \specref{subsec:flush Construct}).
      Memory ordering semantics by implicit flushes on various constructs and
      runtime routines were clarified (see \specref{subsec:implicit flushes}).

\item The \code{atomic} construct was extended with the \code{hint} clause
      (see \specref{subsec:atomic Construct}).

\item The \code{depend} clause (see \specref{subsec:depend Clause}) was
      extended to support iterators and to support depend objects that can be created with the new
      \code{depobj} construct.

\item Lock hints were renamed to synchronization hints, and the
      old names were deprecated (see \specref{subsec:Synchronization Hints}).

\item To support conditional assignment to lastprivate variables, the
      \code{conditional} modifier was added to the \code{lastprivate}
      clause (see \specref{subsubsec:lastprivate clause}).

\item The description of the \code{map} clause was modified to clarify 
      the mapping order when multiple \plc{map-types} are specified for
      a variable or structure members of a variable on the same
      construct. The \code{close} \plc{map-type-modifier} was added as a hint
      for the runtime to allocate memory close to the target device 
      (see \specref{subsec:map Clause}).

\item The capability to map C/C++ pointer variables and assign the
      address of device memory that is mapped by an array section to them
      was added.
      Support for mapping of Fortran pointer and allocatable
      variables, including pointer and allocatable components of variables, 
      was added (see \specref{subsec:map Clause}).

\item The \code{defaultmap} clause (see \specref{subsubsec:defaultmap clause})
      was extended to allow selecting the data-mapping or data-sharing attributes for
      any of the scalar, aggregate, pointer or allocatable classes on a
      per-region basis. Additionally it accepts the \code{none} parameter to support the
      requirement that all variables referenced in the
      construct must be explicitly mapped or privatized.

\item The \code{declare}~\code{mapper} directive was added to support
      mapping of data types with direct and indirect members (see
      \specref{subsubsec:declare mapper Directive}).

\item Runtime routines (see \specref{subsec:omp_set_affinity_format},
      \specref{subsec:omp_get_affinity_format},
      \specref{subsec:omp_display_affinity}, and
      \specref{subsec:omp_capture_affinity}) and environment variables
      (see \specref{sec:OMP_DISPLAY_AFFINITY} and
      \specref{sec:OMP_AFFINITY_FORMAT}) were added to provide 
      and display OpenMP thread affinity information.

\item The \code{omp_get_device_num} runtime routine
      (see \specref{subsec:omp_get_device_num}) was added to support
      determination of the device on which a thread is executing.

\item The \code{omp_pause_resource} and \code{omp_pause_resource_all} 
      runtime routines were added to allow the runtime to relinquish 
      resources used by OpenMP (see \specref{subsec:omp_pause_resource}
      and \specref{subsec:omp_pause_resource_all}).

\item Support for a first-party tool interface (see
      \specref{sec:ompt-overview}) was added.

\item Support for a third-party tool interface (see
      \specref{sec:ompd-overview}) was added.
      
\item Stubs for Runtime Library Routines(previously Appendix A) were moved to a separate document.
\item Interface Declarations (previously Appendix B) were moved to a separate document.

\end{itemize}


\section{Version 4.0 to 4.5 Differences}
\label{sec:Version 4.0 to 4.5 Differences}
\begin{itemize}
\item Support for several features of Fortran 2003 was added (see
      \specref{sec:normative references} for features that are still
      not supported).

\item A parameter was added to the \code{ordered} clause of the loop construct
      (see \specref{subsec:Worksharing-Loop Construct}) and clauses were added to the
      \code{ordered} construct (see \specref{subsec:ordered Construct}) to
      support doacross loop nests and use of the \code{simd} construct on
      loops with loop-carried backward dependences.

\item The \code{linear} clause was added to the loop construct
      (see \specref{subsec:Worksharing-Loop Construct}).

\item The \code{simdlen} clause was added to the \code{simd} construct
      (see \specref{subsubsec:simd Construct}) to support specification of
      the exact number of iterations desired per SIMD chunk.

\item The \code{priority} clause was added to the \code{task} construct
      (see \specref{subsec:task Construct}) to support hints that specify
      the relative execution priority of explicit tasks. The
      \code{omp_get_max_task_priority} routine was added to return
      the maximum supported priority value (see
      \specref{subsec:omp_get_max_task_priority}) and the
      \code{OMP_MAX_TASK_PRIORITY} environment variable was added to
      control the maximum priority value allowed (see
      \specref{sec:OMP_MAX_TASK_PRIORITY}).

\item Taskloop constructs (see \specref{subsec:taskloop Construct} and
      \specref{subsec:taskloop simd Construct}) were added to support
      nestable parallel loops that create OpenMP tasks.

\item To support interaction with native device implementations, the
      \code{use_device_ptr} clause was added to the \code{target data}
      construct (see \specref{subsec:target data Construct}) and the
      \code{is_device_ptr} clause was added to the \code{target} construct
      (see \specref{subsec:target Construct}).

\item The \code{nowait} and \code{depend} clauses were added to the
      \code{target} construct (see \specref{subsec:target Construct})
      to improve support for asynchronous execution of \code{target} regions.

\item The \code{private}, \code{firstprivate} and \code{defaultmap} clauses
      were added to the \code{target} construct (see \specref{subsec:target
      Construct}).

\item The \code{declare}~\code{target} directive was extended to allow
      mapping of global variables to be deferred to specific device
      executions and to allow an \plc{extended-list}
      to be specified in C/C++ (see \specref{subsec:declare target Directive}).

\item To support unstructured data mapping for devices, the
      \code{target enter data} (see \specref{subsec:target enter data
      Construct}) and \code{target exit data} (see \specref{subsec:target
      exit data Construct}) constructs were added and the \code{map} clause
      (see \specref{subsec:map Clause}) was updated.

\item To support a more complete set of device construct shortcuts, the
      \code{target}~\code{parallel}
      (see \specref{subsec:target parallel Construct}),
      target parallel loop
      (see \specref{subsec:Target Parallel Worksharing-Loop Construct}),
      target parallel worksharing-loop SIMD
      (see \specref{subsec:Target Parallel Worksharing-Loop SIMD Construct}),
      and \code{target}~\code{simd}
      (see \specref{subsec:target simd Construct}),
      combined constructs were added.

\item The \code{if} clause was extended to take a
      \plc{directive-name-modifier} that allows it to apply
      to combined constructs (see \specref{sec:if Clause}).

\item The \code{hint} clause was addded to the \code{critical} construct
      (see \specref{subsec:critical Construct}).

\item The \code{source} and \code{sink} dependence types were added to the
      \code{depend} clause (see \specref{subsec:depend Clause}) to support
      doacross loop nests.

\item The implicit data-sharing attribute for scalar variables in
      \code{target} regions was changed to \code{firstprivate} (see
      \specref{subsubsec:Data-sharing Attribute Rules for Variables
      Referenced in a Construct}).

\item Use of some C++ reference types was allowed in some data sharing
      attribute clauses (see \specref{subsec:Data-Sharing Attribute Clauses}).

\item Semantics for reductions on C/C++ array sections were added and
      restrictions on the use of arrays and pointers in reductions were
      removed (see \specref{subsubsec:reduction clause}).

\item The \code{ref}, \code{val}, and \code{uval} modifiers were added to the
      \code{linear} clause (see \specref{subsubsec:linear clause}).

\item Support was added to the map clauses to handle structure elements
	(see \specref{subsec:map Clause}).

\item Query functions for OpenMP thread affinity were added (see
      \specref{subsec:omp_get_num_places} to \specref{subsec:omp_get_partition_place_nums}).

\item The lock API was extended with lock routines that support storing a hint
      with a lock to select a desired lock implementation for a lock's
      intended usage by the application code (see
      \specref{subsec:omp_init_lock_with_hint and omp_init_nest_lock_with_hint}).

\item Device memory routines were added to allow explicit allocation,
      deallocation, memory transfers and memory associations (see
      \specref{sec:Device Memory Routines}).

\item C/C++ Grammar (previously Appendix B) was moved to a separate document.
\end{itemize}



\section{Version 3.1 to 4.0 Differences}
\label{sec:Version 3.1 to 4.0 Differences}
\begin{itemize}
\item Various changes throughout the specification were made to provide initial support of
Fortran 2003 (see
\specref{sec:normative references}).

\item C/C++ array syntax was extended to support array sections (see
\specref{sec:Array Sections}).

\item The \code{proc_bind} clause (see
\specref{subsec:Controlling OpenMP Thread Affinity}),
the \code{OMP_PLACES}
environment variable (see
\specref{sec:OMP_PLACES}), and the \code{omp_get_proc_bind}
runtime routine (see
\specref{subsec:omp_get_proc_bind})
were added to support thread
affinity policies.

\item SIMD directives were added to support SIMD parallelism (see
\specref{subsec:SIMD Directives}).

\item Implementation defined task scheduling points for untied tasks were removed (see
\specref{subsec:Task Scheduling}).

\item Device directives (see
\specref{sec:Device Directives}),
the \code{OMP_DEFAULT_DEVICE}
environment variable (see
\specref{sec:OMP_DEFAULT_DEVICE}), the
\code{omp_set_default_device}, \code{omp_get_default_device},
\code{omp_get_num_devices}, \code{omp_get_num_teams}, \code{omp_get_team_num}, and
\code{omp_is_initial_device} routines were added to support execution on devices.

\item The \code{depend} clause (see
\specref{subsec:depend Clause})
was added to support task dependences.

\item The \code{taskgroup} construct (see
\specref{subsec:taskgroup Construct}) was added to support
more flexible deep task synchronization.

\item The \code{atomic} construct (see
\specref{subsec:atomic Construct}) was extended to support
atomic swap with the \code{capture} clause, to allow new atomic update and capture
forms, and to support sequentially consistent atomic operations with a new \code{seq_cst}
clause.

\item The \code{cancel} construct (see
\specref{subsec:cancel Construct}), the \code{cancellation}~\code{point} construct (see
\specref{subsec:cancellation point Construct}),
the \code{omp_get_cancellation}
runtime routine (see
\specref{subsec:omp_get_cancellation})
and the \code{OMP_CANCELLATION}
environment variable (see
\specref{sec:OMP_CANCELLATION}) were added to support the
concept of cancellation.

\item The \code{reduction} clause (see
\specref{subsubsec:reduction clause}) was extended and the
\code{declare}~\code{reduction} construct (see
\specref{subsubsec:declare reduction Directive}) was added to
support user defined reductions.

\item The \code{OMP_DISPLAY_ENV} environment variable (see
\specref{sec:OMP_DISPLAY_ENV}) was
added to display the value of ICVs associated with the OpenMP environment
variables.

\item Examples (previously Appendix A) were moved to a separate document.
\end{itemize}






\section{Version 3.0 to 3.1 Differences}
\label{sec:Version 3.0 to 3.1 Differences}
\begin{itemize}
\item The \plc{bind-var} ICV has been added, which controls whether or not threads are bound
to processors (see
\specref{subsec:ICV Descriptions}).
The value of this ICV can be set with
the \code{OMP_PROC_BIND} environment variable (see
\specref{sec:OMP_PROC_BIND}).

\item The \code{final} and \code{mergeable} clauses (see
\specref{subsec:task Construct}) were added to
the \code{task} construct to support optimization of task data environments.

\item The \code{taskyield} construct (see
\specref{subsec:taskyield Construct}) was added to allow
user-defined task scheduling points.

\item The \code{atomic} construct (see
\specref{subsec:atomic Construct}) was extended to include
\code{read}, \code{write}, and \code{capture} forms, and an \code{update} clause was added to apply
the already existing form of the \code{atomic} construct.

\item Data environment restrictions were changed to allow \code{intent(in)} and
\code{const}-qualified types for the \code{firstprivate} clause (see
\specref{subsubsec:firstprivate clause}).

\item Data environment restrictions were changed to allow Fortran pointers in
\code{firstprivate} (see
\specref{subsubsec:firstprivate clause})
and \code{lastprivate} (see
\specref{subsubsec:lastprivate clause}).

\item New reduction operators \code{min} and \code{max} were added for C and C++ (see \specref{subsec:Reduction Clauses and Directives}).

\item The nesting restrictions in
\specref{sec:Nesting of Regions} were clarified to disallow
closely-nested OpenMP regions within an \code{atomic} region. This allows an \code{atomic}
region to be consistently defined with other OpenMP regions so that they include all
code in the atomic construct.

\item The \code{omp_in_final} runtime library routine (see
\specref{subsec:omp_in_final}) was
added to support specialization of final task regions.

\item The \plc{nthreads-var} ICV has been modified to be a list of the number of threads to use
at each nested parallel region level. The value of this ICV is still set with the
\code{OMP_NUM_THREADS} environment variable (see
\specref{sec:OMP_NUM_THREADS}), but the
algorithm for determining the number of threads used in a parallel region has been
modified to handle a list (see
\specref{subsec:Determining the Number of Threads for a parallel Region}).

\item Descriptions of examples (previously Appendix A) were expanded and
clarified.

\item Replaced incorrect use of \code{omp_integer_kind} in Fortran interfaces with
\code{selected_int_kind(8)}.
\end{itemize}







\section{Version 2.5 to 3.0 Differences}
\label{sec:Version 2.5 to 3.0 Differences}

\begin{itemize}
\item The definition of active \code{parallel} region has been changed: in Version 3.0 a
\code{parallel} region is active if it is executed by a team consisting of more than one
thread (see
\specref{subsec:OpenMP Language Terminology}).

\item The concept of tasks has been added to the OpenMP execution model (see
\specref{subsec:Tasking Terminology} and
\specref{sec:Execution Model}).

\item The OpenMP memory model now covers atomicity of memory accesses (see
\specref{subsec:Structure of the OpenMP Memory Model}).
The description of the behavior of \code{volatile} in terms of
\code{flush} was removed.

\item In Version 2.5, there was a single copy of the \plc{nest-var}, \plc{dyn-var}, \plc{nthreads-var} and
\plc{run-sched-var} internal control variables (ICVs) for the whole program. In Version
3.0, there is one copy of these ICVs per task (see
\specref{sec:Internal Control Variables}). As a result,
the \code{omp_set_num_threads}, \code{omp_set_nested} and \code{omp_set_dynamic}
runtime library routines now have specified effects when called from inside a
\code{parallel} region (see
\specref{subsec:omp_set_num_threads},
\specref{subsec:omp_set_dynamic} and
\specref{subsec:omp_set_nested}).

\item The \plc{thread-limit-var} ICV has been added, which controls the maximum number of
threads participating in the OpenMP program. The value of this ICV can be set with
the \code{OMP_THREAD_LIMIT} environment variable and retrieved with the
\code{omp_get_thread_limit} runtime library routine (see
\specref{subsec:ICV Descriptions},
\specref{subsec:omp_get_thread_limit} and
\specref{sec:OMP_THREAD_LIMIT}).

\item The \plc{max-active-levels-var} ICV has been added, which controls the number of nested
active \code{parallel} regions. The value of this ICV can be set with the
\code{OMP_MAX_ACTIVE_LEVELS} environment variable and the
\code{omp_set_max_active_levels} runtime library routine, and it can be retrieved
with the \code{omp_get_max_active_levels} runtime library routine (see
\specref{subsec:ICV Descriptions},
\specref{subsec:omp_set_max_active_levels},
\specref{subsec:omp_get_max_active_levels} and
\specref{sec:OMP_MAX_ACTIVE_LEVELS}).

\item The \plc{stacksize-var} ICV has been added, which controls the stack size for threads that
the OpenMP implementation creates. The value of this ICV can be set with the
\code{OMP_STACKSIZE} environment variable (see
\specref{subsec:ICV Descriptions} and
\specref{sec:OMP_STACKSIZE}).

\item The \plc{wait-policy-var} ICV has been added, which controls the desired behavior of
waiting threads. The value of this ICV can be set with the \code{OMP_WAIT_POLICY}
environment variable (see
\specref{subsec:ICV Descriptions} and
\specref{sec:OMP_WAIT_POLICY}).

\item The rules for determining the number of threads used in a \code{parallel} region have
been modified (see
\specref{subsec:Determining the Number of Threads for a parallel Region}).

\item In Version 3.0, the assignment of iterations to threads in a loop construct with a
\code{static} schedule kind is deterministic (see
\specref{subsec:Worksharing-Loop Construct}).

\item In Version 3.0, a loop construct may be associated with more than one perfectly
nested loop. The number of associated loops may be controlled by the \code{collapse}
clause (see
\specref{subsec:Worksharing-Loop Construct}).

\item Random access iterators, and variables of unsigned integer type, may now be used as
loop iterators in loops associated with a loop construct (see
\specref{subsec:Worksharing-Loop Construct}).

\item The schedule kind \code{auto} has been added, which gives the implementation the
freedom to choose any possible mapping of iterations in a loop construct to threads in
the team (see \specref{subsec:Worksharing-Loop Construct}).

\item The \code{task} construct (see \specref{sec:Tasking Constructs}) has been
      added, which provides a mechanism for creating tasks explicitly.

\item The \code{taskwait} construct (see
\specref{subsec:taskwait Construct}) has been added, which
causes a task to wait for all its child tasks to complete.

\item Fortran assumed-size arrays now have predetermined data-sharing attributes (see
\specref{subsubsec:Data-sharing Attribute Rules for Variables Referenced in a Construct}).

\item In Version 3.0, static class members variables may appear in a \code{threadprivate}
directive (see
\specref{subsec:threadprivate Directive}).

\item Version 3.0 makes clear where, and with which arguments, constructors and
destructors of private and threadprivate class type variables are called (see
\specref{subsec:threadprivate Directive},
\specref{subsubsec:private clause},
\specref{subsubsec:firstprivate clause},
\specref{subsubsec:copyin clause} and
\specref{subsubsec:copyprivate clause}).

\item In Version 3.0, Fortran allocatable arrays may appear in \code{private},
\code{firstprivate}, \code{lastprivate}, \code{reduction}, \code{copyin} and \code{copyprivate}
clauses (see
\specref{subsec:threadprivate Directive},
\specref{subsubsec:private clause},
\specref{subsubsec:firstprivate clause},
\specref{subsubsec:lastprivate clause},
\specref{subsubsec:reduction clause},
\specref{subsubsec:copyin clause} and
\specref{subsubsec:copyprivate clause}).

\item In Fortran, \code{firstprivate} is now permitted as an argument to the \code{default}
clause (see
\specref{subsubsec:default clause}).

\item For list items in the \code{private} clause, implementations are no longer permitted to use
the storage of the original list item to hold the new list item on the master thread. If
no attempt is made to reference the original list item inside the \code{parallel} region, its
value is well defined on exit from the \code{parallel} region (see
\specref{subsubsec:private clause}).

\item The runtime library routines \code{omp_set_schedule} and \code{omp_get_schedule}
have been added; these routines respectively set and retrieve the value of the
\plc{run-sched-var} ICV (see\\
\specref{subsec:omp_set_schedule} and
\specref{subsec:omp_get_schedule}).

\item The \code{omp_get_level} runtime library routine has been added, which returns the
number of nested \code{parallel} regions enclosing the task that contains the call (see
\specref{subsec:omp_get_level}).

\item The \code{omp_get_ancestor_thread_num} runtime library routine has been added,
which returns, for a given nested level of the current thread, the thread number of the
ancestor (see
\specref{subsec:omp_get_ancestor_thread_num}).

\item The \code{omp_get_team_size} runtime library routine has been added, which returns,
for a given nested level of the current thread, the size of the thread team to which the
ancestor belongs (see
\specref{subsec:omp_get_team_size}).

\item The \code{omp_get_active_level} runtime library routine has been added, which
returns the number of nested, active \code{parallel} regions enclosing the task that
contains the call (see\linebreak \specref{subsec:omp_get_active_level}).

\item In Version 3.0, locks are owned by tasks, not by threads (see
\specref{sec:Lock Routines}).
\end{itemize}


% This is the end of appendix-E-FeaturesHistory.tex



    \nolinenumbers
    \clearpage
    \phantomsection
    \addcontentsline{toc}{chapter}{Index}
    \printindex
\end{document}

