\chapter{Runtime Library Routines}
\index{runtime library routines}
\label{chap:Runtime Library Routines}
This chapter describes the OpenMP API runtime library routines and 
queryable runtime states. In this chapter, \plc{true} and \plc{false} 
are used as generic terms to simplify the description of the routines.

\begin{samepage}
\begin{ccppspecific}
\plc{true} means a nonzero integer value and \plc{false} means 
an integer value of zero.
\end{ccppspecific}
\end{samepage}
\bigskip

\begin{samepage}
\begin{fortranspecific}
\plc{true} means a logical value of \code{.TRUE.} and \plc{false} 
means a logical value of \code{.FALSE.}.
\end{fortranspecific}
\end{samepage}
\bigskip

\begin{samepage}

\begin{fortranspecific}

\restrictions
The following restriction applies to all OpenMP runtime library routines:

\begin{itemize}
\item OpenMP runtime library routines may not be called 
      from \code{PURE} or \code{ELEMENTAL} procedures.
\end{itemize}
\end{fortranspecific}
\end{samepage}



\section{Runtime Library Definitions}
\index{runtime library definitions}
\index{header files}
\index{include files}
\label{sec:runtime library definitions}
For each base language, a compliant implementation must supply a set of 
definitions for the OpenMP API runtime library routines and the special 
data types of their parameters. The set of definitions must contain a 
declaration for each OpenMP API runtime library routine and variable and 
a definition of each required data type listed below. In addition, each 
set of definitions may specify other implementation specific values.

\begin{ccppspecific}
The library routines are external functions with ``C'' linkage.

Prototypes for the C/C++ runtime library routines described in this 
chapter shall be provided in a header file named \code{omp.h}. This 
file also defines the following:

\begin{itemize}
\item The type \code{omp_lock_t};
\item The type \code{omp_nest_lock_t};
\item The type \code{omp_sync_hint_t};
\item The type \code{omp_lock_hint_t} (deprecated);
\item The type \code{omp_sched_t};
\item The type \code{omp_proc_bind_t};
\item The type \code{omp_control_tool_t};
\item The type \code{omp_control_tool_result_t};
\item The type \code{omp_depend_t};
\item The type \code{omp_memspace_handle_t}, which must be an 
      implementation-defined enum type with an enumerator for 
      at least each predefined memory space in 
      \tabref{tab:Predefined Memory Spaces};
\item The type \code{omp_allocator_handle_t}, which must be an
      implementation-defined enum type with at least the 
      \code{omp_null_allocator} enumerator with the value 
      zero and an enumerator for each predefined memory allocator 
      in \tabref{tab:Predefined Allocators};
\item The type \code{omp_uintptr_t}, which is an unsigned integer type 
      capable of holding a pointer on any device;
\item The type \code{omp_pause_resource_t}; and
\item The type \code{omp_event_handle_t}, which must be an 
      implementation-defined enum type.
\end{itemize}
\end{ccppspecific}

\begin{cppspecific}
The \code{omp.h} header file also defines a class template that
models the \code{Allocator} concept in the \code{omp::allocator}
namespace for each predefined memory allocator in
\tabref{tab:Predefined Allocators} for which the name includes
neither the \code{omp_} prefix nor the \code{_alloc} suffix.
\end{cppspecific}

\begin{fortranspecific}
The OpenMP Fortran API runtime library routines are external procedures. The 
return values of these routines are of default kind, unless otherwise specified.

Interface declarations for the OpenMP Fortran runtime library routines 
described in this chapter shall be provided in the form of a Fortran 
\code{include} file named \code{omp_lib.h} or a Fortran~90 \code{module} 
named \code{omp_lib}. It is implementation defined whether the
\code{include} file or the \code{module} file (or both) is provided.

These files also define the following:

\begin{itemize}
\item The \code{integer} \code{parameter} \code{omp_lock_kind};
\item The \code{integer} \code{parameter} \code{omp_nest_lock_kind};
\item The \code{integer} \code{parameter} \code{omp_sync_hint_kind};
\item The \code{integer} \code{parameter} \code{omp_lock_hint_kind} (deprecated);
\item The \code{integer} \code{parameter} \code{omp_sched_kind};
\item The \code{integer} \code{parameter} \code{omp_proc_bind_kind};
\item The \code{integer} \code{parameter} \code{omp_control_tool_kind};
\item The \code{integer} \code{parameter} \code{omp_control_tool_result_kind};
\item The \code{integer} \code{parameter} \code{omp_depend_kind};
\item The \code{integer} \code{parameter} \code{omp_memspace_handle_kind};
\item The \code{integer} \code{parameter} \code{omp_allocator_handle_kind};
\item The \code{integer} \code{parameter} \code{omp_alloctrait_key_kind};
\item The \code{integer} \code{parameter} \code{omp_alloctrait_val_kind};
\item An \code{integer} \code{parameter} of kind \code{omp_memspace_handle_kind} 
      for each predefined memory space in \tabref{tab:Predefined Memory Spaces};
\item An \code{integer} \code{parameter} of kind \code{omp_allocator_handle_kind} 
      for each predefined memory allocator in \tabref{tab:Predefined Allocators};
\item The \code{integer} \code{parameter} \code{omp_pause_resource_kind};
\item The \code{integer} \code{parameter} \code{omp_event_handle_kind}; and
\item The \code{integer} \code{parameter} \code{openmp_version} with a value 
      \plc{yyyymm} where \plc{yyyy} and \plc{mm} are the year and month designations 
      of the version of the OpenMP Fortran API that the implementation supports; 
      this value matches that of the C preprocessor macro \code{_OPENMP}, when a 
      macro preprocessor is supported (see \specref{sec:Conditional Compilation}).
\end{itemize}

It is implementation defined whether any of the OpenMP runtime library 
routines that take an argument are extended with a generic interface so 
arguments of different \code{KIND} type can be accommodated. 
\end{fortranspecific}

% This is an included file. See the master file for more information.
%
% When editing this file:
%
%    1. To change formatting, appearance, or style, please edit openmp.sty.
%
%    2. Custom commands and macros are defined in openmp.sty.
%
%    3. Be kind to other editors -- keep a consistent style by copying-and-pasting to
%       create new content.
%
%    4. We use semantic markup, e.g. (see openmp.sty for a full list):
%         \code{}     % for bold monospace keywords, code, operators, etc.
%         \plc{}      % for italic placeholder names, grammar, etc.
%
%    5. There are environments that provide special formatting, e.g. language bars.
%       Please use them whereever appropriate.  Examples are:
%
%         \begin{fortranspecific}
%         This is text that appears enclosed in blue language bars for Fortran.
%         \end{fortranspecific}
%
%         \begin{note}
%         This is a note.  The "Note -- " header appears automatically.
%         \end{note}
%
%    6. Other recommendations:
%         Use the convenience macros defined in openmp.sty for the minor headers
%         such as Comments, Syntax, etc.
%
%         To keep items together on the same page, prefer the use of
%         \begin{samepage}.... Avoid \parbox for text blocks as it interrupts line numbering.
%         When possible, avoid \filbreak, \pagebreak, \newpage, \clearpage unless that's
%         what you mean. Use \needspace{} cautiously for troublesome paragraphs.
%
%         Avoid absolute lengths and measures in this file; use relative units when possible.
%         Vertical space can be relative to \baselineskip or ex units. Horizontal space
%         can be relative to \linewidth or em units.
%
%         Prefer \emph{} to italicize terminology, e.g.:
%             This is a \emph{definition}, not a placeholder.
%             This is a \plc{var-name}.
%


\section{Execution Environment Routines}
\index{execution environment routines}
\label{sec:Execution Environment Routines}
This section describes routines that affect and monitor threads, processors, and the
parallel environment.






\subsection{\hcode{omp_set_num_threads}}
\index{omp_set_num_threads@{\code{omp_set_num_threads}}}
\label{subsec:omp_set_num_threads}
\summary
The \code{omp_set_num_threads} routine affects the number of threads to be used for
subsequent parallel regions that do not specify a \code{num_threads} clause, by setting the
value of the first element of the \plc{nthreads-var} ICV of the current task.

\format
\begin{ccppspecific}
\begin{ompcFunction}
void omp_set_num_threads(int \plc{num_threads});
\end{ompcFunction}
\end{ccppspecific}

\begin{fortranspecific}
\begin{ompfSubroutine}
subroutine omp_set_num_threads(\plc{num_threads})
integer \plc{num_threads}
\end{ompfSubroutine}
\end{fortranspecific}

\constraints
The value of the argument passed to this routine must evaluate to a positive integer, or
else the behavior of this routine is implementation defined.

\binding
The binding task set for an \code{omp_set_num_threads} region is the generating task.

\effect
The effect of this routine is to set the value of the first element of the \plc{nthreads-var} ICV
of the current task to the value specified in the argument.

\crossreferences
\begin{itemize}
\item \plc{nthreads-var} ICV, see
\specref{sec:Internal Control Variables}.

\item \code{parallel} construct and \code{num_threads} clause, see
\specref{sec:parallel Construct}.

\item Determining the number of threads for a \code{parallel} region, see
\specref{subsec:Determining the Number of Threads for a parallel Region}.

\item \code{omp_get_max_threads} routine, see
\specref{subsec:omp_get_max_threads}.

\item \code{OMP_NUM_THREADS} environment variable, see
\specref{sec:OMP_NUM_THREADS}.
\end{itemize}









\subsection{\hcode{omp_get_num_threads}}
\index{omp_get_num_threads@{\code{omp_get_num_threads}}}
\label{subsec:omp_get_num_threads}
\summary
The \code{omp_get_num_threads} routine returns the number of threads in the current
team.
\format
\begin{ccppspecific}
\begin{ompcFunction}
int omp_get_num_threads(void);
\end{ompcFunction}
\end{ccppspecific}

\begin{fortranspecific}
\begin{ompfFunction}
integer function omp_get_num_threads()
\end{ompfFunction}
\end{fortranspecific}

\binding
The binding region for an \code{omp_get_num_threads} region is the innermost enclosing
\code{parallel} region.

\effect
The \code{omp_get_num_threads} routine returns the number of threads in the team
executing the \code{parallel} region to which the routine region binds. If called from the
sequential part of a program, this routine returns 1.

\crossreferences
\begin{itemize}
\item \code{parallel} construct, see
\specref{sec:parallel Construct}.

\item Determining the number of threads for a \code{parallel} region, see
\specref{subsec:Determining the Number of Threads for a parallel Region}.

\item \code{omp_set_num_threads} routine, see
\specref{subsec:omp_set_num_threads}.

\item \code{OMP_NUM_THREADS} environment variable, see
\specref{sec:OMP_NUM_THREADS}.
\end{itemize}










\subsection{\hcode{omp_get_max_threads}}
\index{omp_get_max_threads@{\code{omp_get_max_threads}}}
\label{subsec:omp_get_max_threads}
\summary
The \code{omp_get_max_threads} routine returns an upper bound on the number of
threads that could be used to form a new team if a \code{parallel} construct without a
\code{num_threads} clause were encountered after execution returns from this routine.

\format
\begin{ccppspecific}
\begin{ompcFunction}
int omp_get_max_threads(void);
\end{ompcFunction}
\end{ccppspecific}

\begin{fortranspecific}
\begin{ompfFunction}
integer function omp_get_max_threads()
\end{ompfFunction}
\end{fortranspecific}

\binding
The binding task set for an \code{omp_get_max_threads} region is the generating task.

\effect
The value returned by \code{omp_get_max_threads} is the value of the first element of
the \plc{nthreads-var} ICV of the current task. This value is also an upper bound on the
number of threads that could be used to form a new team if a parallel region without a
\code{num_threads} clause were encountered after execution returns from this routine.

\begin{note}
The return value of the \code{omp_get_max_threads} routine can be used to
dynamically allocate sufficient storage for all threads in the team formed at the
subsequent active \code{parallel} region.
\end{note}

\crossreferences
\begin{itemize}
\item \plc{nthreads-var} ICV, see
\specref{sec:Internal Control Variables}.

\item \code{parallel} construct, see
\specref{sec:parallel Construct}.

\item \code{num_threads} clause, see
\specref{sec:parallel Construct}.

\item Determining the number of threads for a \code{parallel} region, see
\specref{subsec:Determining the Number of Threads for a parallel Region}.

\item \code{omp_set_num_threads} routine, see
\specref{subsec:omp_set_num_threads}.

\item \code{OMP_NUM_THREADS} environment variable, see
\specref{sec:OMP_NUM_THREADS}.
\end{itemize}









%\newpage %% HACK

\subsection{\hcode{omp_get_thread_num}}
\index{omp_get_thread_num@{\code{omp_get_thread_num}}}
\label{subsec:omp_get_thread_num}
\summary
The \code{omp_get_thread_num} routine returns the thread number, within the current
team, of the calling thread.

\format
\begin{ccppspecific}
\begin{ompcFunction}
int omp_get_thread_num(void);
\end{ompcFunction}
\end{ccppspecific}

\begin{fortranspecific}
\begin{ompfFunction}
integer function omp_get_thread_num()
\end{ompfFunction}
\end{fortranspecific}

\binding
The binding thread set for an \code{omp_get_thread_num} region is the current team. The
binding region for an \code{omp_get_thread_num} region is the innermost enclosing
\code{parallel} region.

\effect
The \code{omp_get_thread_num} routine returns the thread number of the calling thread,
within the team executing the \code{parallel} region to which the routine region binds. The
thread number is an integer between 0 and one less than the value returned by
\code{omp_get_num_threads}, inclusive. The thread number of the master thread of the
team is 0. The routine returns 0 if it is called from the sequential part of a program.

\begin{note}
The thread number may change during the execution of an untied task. The
value returned by \code{omp_get_thread_num} is not generally useful during the execution
of such a task region.
\end{note}

\crossreferences
\begin{itemize}
\item \code{omp_get_num_threads} routine, see
\specref{subsec:omp_get_num_threads}.
\end{itemize}








\subsection{\hcode{omp_get_num_procs}}
\index{omp_get_num_procs@{\code{omp_get_num_procs}}}
\label{subsec:omp_get_num_procs}
\summary
The \code{omp_get_num_procs} routine returns the number of processors available to the
device.

\format
\begin{ccppspecific}
\begin{ompcFunction}
int omp_get_num_procs(void);
\end{ompcFunction}
\end{ccppspecific}

\begin{fortranspecific}
\begin{ompfFunction}
integer function omp_get_num_procs()
\end{ompfFunction}
\end{fortranspecific}

\binding
The binding thread set for an \code{omp_get_num_procs} region is all threads on a device.
The effect of executing this routine is not related to any specific region corresponding to
any construct or API routine.

\effect
The \code{omp_get_num_procs} routine returns the number of processors that are available
to the device at the time the routine is called. This value may change between
the time that it is determined by the \code{omp_get_num_procs} routine and the time that it
is read in the calling context due to system actions outside the control of the OpenMP
implementation.

\crossreferences
None.







\subsection{\hcode{omp_in_parallel}}
\index{omp_in_parallel@{\code{omp_in_parallel}}}
\label{subsec:omp_in_parallel}
\summary
The \code{omp_in_parallel} routine returns \plc{true} if the \plc{active-levels-var} ICV is greater
than zero; otherwise, it returns \plc{false}.

%\pagebreak
\format
\begin{ccppspecific}
\begin{ompcFunction}
int omp_in_parallel(void);
\end{ompcFunction}
\end{ccppspecific}

\begin{fortranspecific}
\begin{ompfFunction}
logical function omp_in_parallel()
\end{ompfFunction}
\end{fortranspecific}

\binding
The binding task set for an \code{omp_in_parallel} region is the generating task.

\effect
The effect of the \code{omp_in_parallel} routine is to return \plc{true} if the current task is
enclosed by an active \code{parallel} region, and the \code{parallel} region is enclosed by the
outermost initial task region on the device; otherwise it returns \plc{false}.

\crossreferences
\begin{itemize}
\item \plc{active-levels-var}, see
\specref{sec:Internal Control Variables}.

\item \code{parallel} construct, see
\specref{sec:parallel Construct}.

\item \code{omp_get_active_level} routine, see
\specref{subsec:omp_get_active_level}.
\end{itemize}








\bigskip
\subsection{\hcode{omp_set_dynamic}}
\index{omp_set_dynamic@{\code{omp_set_dynamic}}}
\label{subsec:omp_set_dynamic}
\summary
The \code{omp_set_dynamic} routine enables or disables dynamic adjustment of the
number of threads available for the execution of subsequent \code{parallel} regions by
setting the value of the \plc{dyn-var} ICV.


%\pagebreak
\format
\begin{ccppspecific}
\begin{ompcFunction}
void omp_set_dynamic(int \plc{dynamic_threads});
\end{ompcFunction}
\end{ccppspecific}
\bigskip

\begin{samepage}
\begin{fortranspecific}
\begin{ompfSubroutine}
subroutine omp_set_dynamic(\plc{dynamic_threads})
logical \plc{dynamic_threads}
\end{ompfSubroutine}
\end{fortranspecific}
\end{samepage}

\binding
The binding task set for an \code{omp_set_dynamic} region is the generating task.

\effect
For implementations that support dynamic adjustment of the number of threads, if the
argument to \code{omp_set_dynamic} evaluates to \plc{true}, dynamic adjustment is enabled for
the current task; otherwise, dynamic adjustment is disabled for the current task. For
implementations that do not support dynamic adjustment of the number of threads this
routine has no effect: the value of \plc{dyn-var} remains \plc{false}.

\crossreferences
\begin{itemize}
\item \plc{dyn-var} ICV, see
\specref{sec:Internal Control Variables}.

\item Determining the number of threads for a \code{parallel} region, see
\specref{subsec:Determining the Number of Threads for a parallel Region}.

\item \code{omp_get_num_threads} routine, see
\specref{subsec:omp_get_num_threads}.

\item \code{omp_get_dynamic} routine, see
\specref{subsec:omp_get_dynamic}.

\item \code{OMP_DYNAMIC} environment variable, see
\specref{sec:OMP_DYNAMIC}.
\end{itemize}








\subsection{\hcode{omp_get_dynamic}}
\index{omp_get_dynamic@{\code{omp_get_dynamic}}}
\label{subsec:omp_get_dynamic}
\summary
The \code{omp_get_dynamic} routine returns the value of the \plc{dyn-var} ICV, which
determines whether dynamic adjustment of the number of threads is enabled or disabled.

\format
\begin{ccppspecific}
\begin{ompcFunction}
int omp_get_dynamic(void);
\end{ompcFunction}
\end{ccppspecific}

\begin{fortranspecific}
\begin{ompfFunction}
logical function omp_get_dynamic()
\end{ompfFunction}
\end{fortranspecific}

\binding
The binding task set for an \code{omp_get_dynamic} region is the generating task.

\effect
This routine returns \plc{true} if dynamic adjustment of the number of threads is enabled for
the current task; it returns \plc{false}, otherwise. If an implementation does not support
dynamic adjustment of the number of threads, then this routine always returns \plc{false}.

\crossreferences
\begin{itemize}
\item \plc{dyn-var} ICV, see
\specref{sec:Internal Control Variables}.

\item Determining the number of threads for a \code{parallel} region, see
\specref{subsec:Determining the Number of Threads for a parallel Region}.

\item \code{omp_set_dynamic} routine, see
\specref{subsec:omp_set_dynamic}.

\item \code{OMP_DYNAMIC} environment variable, see
\specref{sec:OMP_DYNAMIC}.
\end{itemize}








\subsection{\hcode{omp_get_cancellation}}
\index{omp_get_cancellation@{\code{omp_get_cancellation}}}
\label{subsec:omp_get_cancellation}
\summary
The \code{omp_get_cancellation} routine returns the value of the \plc{cancel-var} ICV, which determines if cancellation is enabled or disabled.

%\newpage %% HACK
\format
\begin{ccppspecific}
\begin{ompcFunction}
int omp_get_cancellation(void);
\end{ompcFunction}
\end{ccppspecific}

\begin{fortranspecific}
\begin{ompfFunction}
logical function omp_get_cancellation()
\end{ompfFunction}
\end{fortranspecific}

\binding
The binding task set for an \code{omp_get_cancellation} region is the whole program.

\effect
This routine returns \plc{true} if cancellation is enabled. It returns \plc{false} otherwise.

\crossreferences
\begin{itemize}
\item \plc{cancel-var} ICV, see
\specref{subsec:ICV Descriptions}.

\item \code{cancel} construct, see \specref{subsec:cancel Construct}

\item \code{OMP_CANCELLATION} environment variable, see
\specref{sec:OMP_CANCELLATION}
\end{itemize}









\subsection{\hcode{omp_set_nested}}
\index{omp_set_nested@{\code{omp_set_nested}}}
\label{subsec:omp_set_nested}
\summary
The deprecated \code{omp_set_nested} routine enables or disables nested parallelism, by setting the
\plc{nest-var} ICV.

%\newpage %% HACK
\format
\begin{ccppspecific}
\begin{ompcFunction}
void omp_set_nested(int \plc{nested});
\end{ompcFunction}
\end{ccppspecific}

\begin{fortranspecific}
\begin{ompfSubroutine}
subroutine omp_set_nested(\plc{nested})
logical \plc{nested}
\end{ompfSubroutine}
\end{fortranspecific}

\binding
The binding task set for an \code{omp_set_nested} region is the generating task.

\effect
For implementations that support nested parallelism, if the argument to
\code{omp_set_nested} evaluates to \plc{true}, nested parallelism is enabled for the current task;
otherwise, nested parallelism is disabled for the current task. For implementations that
do not support nested parallelism, this routine has no effect: the value of \plc{nest-var}
remains \plc{false}. This routine has been deprecated.

\crossreferences
\begin{itemize}
\item \plc{nest-var} ICV, see
\specref{sec:Internal Control Variables}.

\item Determining the number of threads for a \code{parallel} region, see
\specref{subsec:Determining the Number of Threads for a parallel Region}.

\item \code{omp_set_max_active_levels} routine, see
\specref{subsec:omp_set_max_active_levels}.

\item \code{omp_get_max_active_levels} routine, see
\specref{subsec:omp_get_max_active_levels}.

\item \code{omp_get_nested} routine, see
\specref{subsec:omp_get_nested}.

\item \code{OMP_NESTED} environment variable, see
\specref{sec:OMP_NESTED}.
\end{itemize}








\subsection{\hcode{omp_get_nested}}
\index{omp_get_nested@{\code{omp_get_nested}}}
\label{subsec:omp_get_nested}
\summary
The deprecated \code{omp_get_nested} routine returns the value of the \plc{nest-var} ICV, which
determines if nested parallelism is enabled or disabled.

\format
\begin{ccppspecific}
\begin{ompcFunction}
int omp_get_nested(void);
\end{ompcFunction}
\end{ccppspecific}

\begin{fortranspecific}
\begin{ompfFunction}
logical function omp_get_nested()
\end{ompfFunction}
\end{fortranspecific}

\binding
The binding task set for an \code{omp_get_nested} region is the generating task.

\effect
This routine returns \plc{true} if nested parallelism is enabled for the current task; it returns
\plc{false}, otherwise. If an implementation does not support nested parallelism, this routine
always returns \plc{false}. This routine has been deprecated.

\crossreferences
\begin{itemize}
\item \plc{nest-var} ICV, see
\specref{sec:Internal Control Variables}.

\item Determining the number of threads for a \code{parallel} region, see
\specref{subsec:Determining the Number of Threads for a parallel Region}.

\item \code{omp_set_nested} routine, see
\specref{subsec:omp_set_nested}.

\item \code{OMP_NESTED} environment variable, see
\specref{sec:OMP_NESTED}.
\end{itemize}









\subsection{\hcode{omp_set_schedule}}
\index{omp_set_schedule@{\code{omp_set_schedule}}}
\label{subsec:omp_set_schedule}
\summary
The \code{omp_set_schedule} routine affects the schedule that is applied when \code{runtime}
is used as schedule kind, by setting the value of the \plc{run-sched-var} ICV.

%\newpage %% HACK
\format
\begin{ccppspecific}
\begin{ompcFunction}
void omp_set_schedule(omp_sched_t \plc{kind}, int \plc{chunk_size});
\end{ompcFunction}
\end{ccppspecific}

\begin{fortranspecific}
\begin{ompfSubroutine}
subroutine omp_set_schedule(\plc{kind}, \plc{chunk_size})
integer (kind=omp_sched_kind) \plc{kind}
integer \plc{chunk_size}
\end{ompfSubroutine}
\end{fortranspecific}

\constraints
The first argument passed to this routine can be one of the valid OpenMP schedule kinds
(except for \code{runtime}) or any implementation specific schedule. The C/C++ header file
(\code{omp.h}) and the Fortran include file (\code{omp_lib.h}) and/or Fortran~90 module file
(\code{omp_lib}) define the valid constants. The valid constants must include the following,
which can be extended with implementation specific values:

\begin{ccppspecific}
\begin{ompcEnum}
typedef enum omp_sched_t {
  omp_sched_static = 1,
  omp_sched_dynamic = 2,
  omp_sched_guided = 3,
  omp_sched_auto = 4
} omp_sched_t;
\end{ompcEnum}
\end{ccppspecific}

\begin{samepage}
\begin{fortranspecific}
\begin{ompfEnum}
integer(kind=omp_sched_kind), parameter :: omp_sched_static = 1
integer(kind=omp_sched_kind), parameter :: omp_sched_dynamic = 2
integer(kind=omp_sched_kind), parameter :: omp_sched_guided = 3
integer(kind=omp_sched_kind), parameter :: omp_sched_auto = 4
\end{ompfEnum}
\end{fortranspecific}
\end{samepage}

\binding
The binding task set for an \code{omp_set_schedule} region is the generating task.

\effect
The effect of this routine is to set the value of the \plc{run-sched-var} ICV of the current task
to the values specified in the two arguments. The schedule is set to the schedule type
specified by the first argument \plc{kind}. It can be any of the standard schedule types or
any other implementation specific one. For the schedule types \code{static}, \code{dynamic}, and
\code{guided} the \plc{chunk_size} is set to the value of the second argument, or to the default
\plc{chunk_size} if the value of the second argument is less than 1; for the schedule type
\code{auto} the second argument has no meaning; for implementation specific schedule types,
the values and associated meanings of the second argument are implementation defined.

\crossreferences
\begin{itemize}
\item \plc{run-sched-var} ICV, see
\specref{sec:Internal Control Variables}.

\item Determining the schedule of a worksharing loop, see
\specref{subsubsec:Determining the Schedule of a Worksharing Loop}.

\item \code{omp_get_schedule} routine, see
\specref{subsec:omp_get_schedule}.

\item \code{OMP_SCHEDULE} environment variable, see
\specref{sec:OMP_SCHEDULE}.
\end{itemize}









\subsection{\hcode{omp_get_schedule}}
\index{omp_get_schedule@{\code{omp_get_schedule}}}
\label{subsec:omp_get_schedule}
\summary
The \code{omp_get_schedule} routine returns the schedule that is applied when the
runtime schedule is used.
\format
\begin{ccppspecific}
\begin{ompcFunction}
void omp_get_schedule(omp_sched_t *\plc{kind}, int *\plc{chunk_size});
\end{ompcFunction}
\end{ccppspecific}

%\newpage %% HACK
\begin{fortranspecific}
\begin{ompfSubroutine}
subroutine omp_get_schedule(\plc{kind}, \plc{chunk_size})
integer (kind=omp_sched_kind) \plc{kind}
integer \plc{chunk_size}
\end{ompfSubroutine}
\end{fortranspecific}

\binding
The binding task set for an \code{omp_get_schedule} region is the generating task.

\effect
This routine returns the \plc{run-sched-var} ICV in the task to which the routine binds. The
first argument \plc{kind} returns the schedule to be used. It can be any of the standard
schedule types as defined in
\specref{subsec:omp_set_schedule},
or any implementation specific
schedule type. The second argument is interpreted as in the \code{omp_set_schedule} call,
defined in
\specref{subsec:omp_set_schedule}.

\crossreferences
\begin{itemize}
\item \plc{run-sched-var} ICV, see
\specref{sec:Internal Control Variables}.

\item Determining the schedule of a worksharing loop, see
\specref{subsubsec:Determining the Schedule of a Worksharing Loop}.

\item \code{omp_set_schedule} routine, see
\specref{subsec:omp_set_schedule}.

\item \code{OMP_SCHEDULE} environment variable, see
\specref{sec:OMP_SCHEDULE}.
\end{itemize}









\subsection{\hcode{omp_get_thread_limit}}
\index{omp_get_thread_limit@{\code{omp_get_thread_limit}}}
\label{subsec:omp_get_thread_limit}
\summary
The \code{omp_get_thread_limit} routine returns the maximum number of OpenMP
threads available to participate in the current contention group.

%\newpage %% HACK
\format
\begin{ccppspecific}
\begin{ompcFunction}
int omp_get_thread_limit(void);
\end{ompcFunction}
\end{ccppspecific}

\begin{fortranspecific}
\begin{ompfFunction}
integer function omp_get_thread_limit()
\end{ompfFunction}
\end{fortranspecific}

\binding
The binding thread set for an \code{omp_get_thread_limit} region is all threads on the
device. The effect of executing this routine is not related to any specific region
corresponding to any construct or API routine.

\effect
The \code{omp_get_thread_limit} routine returns the value of the \plc{thread-limit-var} ICV.

\crossreferences
\begin{itemize}
\item \plc{thread-limit-var} ICV, see
\specref{sec:Internal Control Variables}.

\item \code{OMP_THREAD_LIMIT} environment variable, see
\specref{sec:OMP_THREAD_LIMIT}.
\end{itemize}









\subsection{\hcode{omp_set_max_active_levels}}
\index{omp_set_max_active_levels@{\code{omp_set_max_active_levels}}}
\label{subsec:omp_set_max_active_levels}
\summary
The \code{omp_set_max_active_levels} routine limits the number of nested active
parallel regions on the device, by setting the \plc{max-active-levels-var} ICV

%\newpage %% HACK
\format
\begin{ccppspecific}
\begin{ompcFunction}
void omp_set_max_active_levels(int \plc{max_levels});
\end{ompcFunction}
\end{ccppspecific}

\begin{fortranspecific}
\begin{ompfSubroutine}
subroutine omp_set_max_active_levels(\plc{max_levels})
integer \plc{max_levels}
\end{ompfSubroutine}
\end{fortranspecific}

\constraints
The value of the argument passed to this routine must evaluate to a non-negative integer,
otherwise the behavior of this routine is implementation defined.

\binding
When called from a sequential part of the program, the binding thread set for an
\code{omp_set_max_active_levels} region is the encountering thread. When called
from within any explicit parallel region, the binding thread set (and binding region, if
required) for the \code{omp_set_max_active_levels} region is implementation defined.

\effect
The effect of this routine is to set the value of the \plc{max-active-levels-var} ICV to the value
specified in the argument.

If the number of parallel levels requested exceeds the number of levels of parallelism
supported by the implementation, the value of the \plc{max-active-levels-var} ICV will be set
to the number of parallel levels supported by the implementation.

This routine has the described effect only when called from a sequential part of the
program. When called from within an explicit \code{parallel} region, the effect of this
routine is implementation defined.

\crossreferences
\begin{itemize}
\item \plc{max-active-levels-var} ICV, see
\specref{sec:Internal Control Variables}.

\item \code{omp_get_max_active_levels} routine, see
\specref{subsec:omp_get_max_active_levels}.

\item \code{OMP_MAX_ACTIVE_LEVELS} environment variable, see
\specref{sec:OMP_MAX_ACTIVE_LEVELS}.
\end{itemize}










\subsection{\hcode{omp_get_max_active_levels}}
\index{omp_get_max_active_levels@{\code{omp_get_max_active_levels}}}
\label{subsec:omp_get_max_active_levels}
\summary
The \code{omp_get_max_active_levels} routine returns the value of the
\plc{max-active-levels-var} ICV, which determines the maximum number of nested active parallel regions
on the device.

\format
\begin{ccppspecific}
\begin{ompcFunction}
int omp_get_max_active_levels(void);
\end{ompcFunction}
\end{ccppspecific}

\begin{fortranspecific}
\begin{ompfFunction}
integer function omp_get_max_active_levels()
\end{ompfFunction}
\end{fortranspecific}

\binding
When called from a sequential part of the program, the binding thread set for an
\code{omp_get_max_active_levels} region is the encountering thread. When called
from within any explicit parallel region, the binding thread set (and binding region, if
required) for the \code{omp_get_max_active_levels} region is implementation defined.

\effect
The \code{omp_get_max_active_levels} routine returns the value of the \plc{max-active-levels-var} ICV,
which determines the maximum number of nested active parallel regions
on the device.

\crossreferences
\begin{itemize}
\item \plc{max-active-levels-var} ICV, see
\specref{sec:Internal Control Variables}.

\item \code{omp_set_max_active_levels} routine, see
\specref{subsec:omp_set_max_active_levels}.

\item \code{OMP_MAX_ACTIVE_LEVELS} environment variable, see
\specref{sec:OMP_MAX_ACTIVE_LEVELS}.
\end{itemize}








\subsection{\hcode{omp_get_level}}
\index{omp_get_level@{\code{omp_get_level}}}
\label{subsec:omp_get_level}
\summary
The \code{omp_get_level} routine returns the value of the \plc{levels-var} ICV.

\format
\begin{ccppspecific}
\begin{ompcFunction}
int omp_get_level(void);
\end{ompcFunction}
\end{ccppspecific}

\begin{fortranspecific}
\begin{ompfFunction}
integer function omp_get_level()
\end{ompfFunction}
\end{fortranspecific}

\binding
The binding task set for an \code{omp_get_level} region is the generating task.

\effect
The effect of the \code{omp_get_level} routine is to return the number of nested
\code{parallel} regions (whether active or inactive) enclosing the current task such that all
of the \code{parallel} regions are enclosed by the outermost initial task region on the
current device.

\crossreferences
\begin{itemize}
\item \plc{levels-var} ICV, see
\specref{sec:Internal Control Variables}.

\item \code{omp_get_active_level} routine, see
\specref{subsec:omp_get_active_level}.

\item \code{OMP_MAX_ACTIVE_LEVELS} environment variable, see
\specref{sec:OMP_MAX_ACTIVE_LEVELS}.
\end{itemize}










\subsection{\hcode{omp_get_ancestor_thread_num}}
\index{omp_get_ancestor_thread_num@{\code{omp_get_ancestor_thread_num}}}
\label{subsec:omp_get_ancestor_thread_num}
\summary
The \code{omp_get_ancestor_thread_num} routine returns, for a given nested level of
the current thread, the thread number of the ancestor of the current thread.

\begin{samepage}
\format
\begin{ccppspecific}
\begin{ompcFunction}
int omp_get_ancestor_thread_num(int \plc{level});
\end{ompcFunction}
\end{ccppspecific}
\end{samepage}

\begin{fortranspecific}
\begin{ompfFunction}
integer function omp_get_ancestor_thread_num(\plc{level})
integer \plc{level}
\end{ompfFunction}
\end{fortranspecific}

\binding
The binding thread set for an \code{omp_get_ancestor_thread_num} region is the
encountering thread. The binding region for an \code{omp_get_ancestor_thread_num}
region is the innermost enclosing \code{parallel} region.

\effect
The \code{omp_get_ancestor_thread_num} routine returns the thread number of the
ancestor at a given nest level of the current thread or the thread number of the current
thread. If the requested nest level is outside the range of 0 and the nest level of the
current thread, as returned by the \code{omp_get_level} routine, the routine returns -1.

\begin{note}
When the \code{omp_get_ancestor_thread_num} routine is called with a value
of \code{level}=0, the routine always returns 0. If \code{level}=\code{omp_get_level()}, the routine
has the same effect as the \code{omp_get_thread_num} routine.
\end{note}

\crossreferences
\begin{itemize}
\item \code{omp_get_thread_num} routine, see
\specref{subsec:omp_get_thread_num}.

\item \code{omp_get_level} routine, see
\specref{subsec:omp_get_level}.

\item \code{omp_get_team_size} routine, see
\specref{subsec:omp_get_team_size}.
\end{itemize}










\subsection{\hcode{omp_get_team_size}}
\index{omp_get_team_size@{\code{omp_get_team_size}}}
\label{subsec:omp_get_team_size}
\summary
The \code{omp_get_team_size} routine returns, for a given nested level of the current
thread, the size of the thread team to which the ancestor or the current thread belongs.

\format
\begin{ccppspecific}
\begin{ompcFunction}
int omp_get_team_size(int \plc{level});
\end{ompcFunction}
\end{ccppspecific}

\begin{fortranspecific}
\begin{ompfFunction}
integer function omp_get_team_size(\plc{level})
integer \plc{level}
\end{ompfFunction}
\end{fortranspecific}

\binding
The binding thread set for an \code{omp_get_team_size} region is the encountering
thread. The binding region for an \code{omp_get_team_size} region is the innermost
enclosing \code{parallel} region.

\effect
The \code{omp_get_team_size} routine returns the size of the thread team to which the
ancestor or the current thread belongs. If the requested nested level is outside the range
of 0 and the nested level of the current thread, as returned by the \code{omp_get_level}
routine, the routine returns -1. Inactive parallel regions are regarded like active parallel
regions executed with one thread.

\begin{note}
When the \code{omp_get_team_size} routine is called with a value of \code{level}=0,
the routine always returns 1. If \code{level}=\code{omp_get_level()}, the routine has the same
effect as the\linebreak \code{omp_get_num_threads} routine.
\end{note}

\crossreferences
\begin{itemize}
\item \code{omp_get_num_threads} routine, see
\specref{subsec:omp_get_num_threads}.

\item \code{omp_get_level} routine, see
\specref{subsec:omp_get_level}.

\item \code{omp_get_ancestor_thread_num} routine, see
\specref{subsec:omp_get_ancestor_thread_num}.
\end{itemize}









\subsection{\hcode{omp_get_active_level}}
\index{omp_get_active_level@{\code{omp_get_active_level}}}
\label{subsec:omp_get_active_level}
\summary
The \code{omp_get_active_level} routine returns the value of the \plc{active-level-vars} ICV..

\format
\begin{ccppspecific}
\begin{ompcFunction}
int omp_get_active_level(void);
\end{ompcFunction}
\end{ccppspecific}

%\pagebreak
\begin{fortranspecific}
\begin{ompfFunction}
integer function omp_get_active_level()
\end{ompfFunction}
\end{fortranspecific}

\binding
The binding task set for the an \code{omp_get_active_level} region is the generating
task.

\effect
The effect of the \code{omp_get_active_level} routine is to return the number of nested,
active \code{parallel} regions enclosing the current task such that all of the \code{parallel}
regions are enclosed by the outermost initial task region on the current device.

\crossreferences
\begin{itemize}
\item \plc{active-levels-var} ICV, see
\specref{sec:Internal Control Variables}.

\item \code{omp_get_level} routine, see
\specref{subsec:omp_get_level}.
\end{itemize}











\subsection{\hcode{omp_in_final}}
\index{omp_in_final@{\code{omp_in_final}}}
\label{subsec:omp_in_final}
\summary
The \code{omp_in_final} routine returns \plc{true} if the routine is executed in a final task
region; otherwise, it returns \plc{false}.

\format
\begin{ccppspecific}
\begin{ompcFunction}
int omp_in_final(void);
\end{ompcFunction}
\end{ccppspecific}

\begin{fortranspecific}
\begin{ompfFunction}
logical function omp_in_final()
\end{ompfFunction}
\end{fortranspecific}

\binding
The binding task set for an \code{omp_in_final} region is the generating task.

\effect
\code{omp_in_final} returns \plc{true} if the enclosing task region is final. Otherwise, it returns
\plc{false}.

\crossreferences
\begin{itemize}
\item \code{task} construct, see
\specref{subsec:task Construct}.
\end{itemize}









\subsection{\hcode{omp_get_proc_bind}}
\index{omp_get_proc_bind@{\code{omp_get_proc_bind}}}
\label{subsec:omp_get_proc_bind}
\summary
The \code{omp_get_proc_bind} routine returns the thread affinity policy to be used for the
subsequent nested \code{parallel} regions that do not specify a \code{proc_bind} clause.

\format
\begin{ccppspecific}
\begin{ompcFunction}
omp_proc_bind_t omp_get_proc_bind(void);
\end{ompcFunction}
\end{ccppspecific}

\begin{fortranspecific}
\begin{ompfFunction}
integer (kind=omp_proc_bind_kind) function omp_get_proc_bind()
\end{ompfFunction}
\end{fortranspecific}

\constraints
The value returned by this routine must be one of the valid affinity policy kinds. The C/
C++ header file (\code{omp.h}) and the Fortran include file (\code{omp_lib.h}) and/or Fortran~90
module file (\code{omp_lib}) define the valid constants. The valid constants must include the
following:

\begin{ccppspecific}
\begin{ompcEnum}
typedef enum omp_proc_bind_t {
  omp_proc_bind_false = 0,
  omp_proc_bind_true = 1,
  omp_proc_bind_master = 2,
  omp_proc_bind_close = 3,
  omp_proc_bind_spread = 4
} omp_proc_bind_t;
\end{ompcEnum}
\end{ccppspecific}

\begin{fortranspecific}
\begin{ompfEnum}
integer (kind=omp_proc_bind_kind), &
                parameter :: omp_proc_bind_false = 0
integer (kind=omp_proc_bind_kind), &
                parameter :: omp_proc_bind_true = 1
integer (kind=omp_proc_bind_kind), &
                parameter :: omp_proc_bind_master = 2
integer (kind=omp_proc_bind_kind), &
                parameter :: omp_proc_bind_close = 3
integer (kind=omp_proc_bind_kind), &
                parameter :: omp_proc_bind_spread = 4
\end{ompfEnum}
\end{fortranspecific}

\binding
The binding task set for an \code{omp_get_proc_bind} region is the generating task

\effect
The effect of this routine is to return the value of the first element of the \plc{bind-var} ICV
of the current task. See \specref{subsec:Controlling OpenMP Thread Affinity}
for the rules governing the thread affinity policy.

\crossreferences
\begin{itemize}
\item \plc{bind-var} ICV, see
\specref{sec:Internal Control Variables}.

\item Controlling OpenMP thread affinity, see
\specref{subsec:Controlling OpenMP Thread Affinity}.

\item \code{OMP_PROC_BIND} environment variable, see
\specref{sec:OMP_PROC_BIND}.
\end{itemize}



%%%%%%%%%% begin 392

\subsection{\hcode{omp_get_num_places}}
\index{omp_get_num_places@{\code{omp_get_num_places}}}
\label{subsec:omp_get_num_places}
\summary
The \code{omp_get_num_places} routine returns the number of places
available to the execution environment in the place list.

\format
\begin{ccppspecific}
\begin{ompcFunction}
int omp_get_num_places(void);
\end{ompcFunction}
\end{ccppspecific}

\begin{fortranspecific}
\begin{ompfFunction}
integer function omp_get_num_places()
\end{ompfFunction}
\end{fortranspecific}

\binding
The binding thread set for an \code{omp_get_num_places}  region is all threads on a device. The effect of executing this routine is not related to any specific region corresponding to any construct or API routine.

\effect

The \code{omp_get_num_places} routine returns the number of places in the place list. This value is equivalent to the number of places in the  \plc{place-partition-var} ICV in the execution environment of the initial task.

\crossreferences
\begin{itemize}
\item \plc{place-partition-var} ICV, see
\specref{sec:Internal Control Variables}.

\item \code{OMP_PLACES} environment variable, see
\specref{sec:OMP_PLACES}.
\end{itemize}





\subsection{\hcode{omp_get_place_num_procs}}
\index{omp_get_place_num_procs@{\code{omp_get_place_num_procs}}}
\label{subsec:omp_get_place_num_procs}

\summary
The \code{omp_get_place_num_procs}  routine returns the number of processors available to the execution environment in the specified place.

\format
\begin{ccppspecific}
\begin{ompcFunction}
int omp_get_place_num_procs(int \plc{place_num});
\end{ompcFunction}
\end{ccppspecific}

\begin{fortranspecific}
\begin{ompfFunction}
integer function omp_get_place_num_procs(\plc{place_num})
integer \plc{place_num}
\end{ompfFunction}
\end{fortranspecific}

\binding
The binding thread set for an \code{omp_get_place_num_procs}  region is all threads on a device. The effect of executing this routine is not related to any specific region corresponding to any construct or API routine.

\effect
The \code{omp_get_place_num_procs} routine returns the number of
processors associated with the place numbered \plc{place_num}. The
routine returns zero when \plc{place_num} is negative, or is equal
to or larger than the value returned by \code{omp_get_num_places()}.

\crossreferences
\begin{itemize}
\item \code{OMP_PLACES} environment variable, see
\specref{sec:OMP_PLACES}.
\end{itemize}




\subsection{\hcode{omp_get_place_proc_ids}}
\index{omp_get_place_proc_ids@{\code{omp_get_place_proc_ids}}}
\label{subsec:omp_get_place_proc_ids}

\summary
The \code{omp_get_place_proc_ids} routine returns the numerical identifiers of the processors available to the execution environment in the specified place.

\format
\begin{ccppspecific}
\begin{ompcFunction}
void omp_get_place_proc_ids(int \plc{place_num}, int *\plc{ids});
\end{ompcFunction}
\end{ccppspecific}

\begin{fortranspecific}
\begin{ompfSubroutine}
subroutine omp_get_place_proc_ids(\plc{place_num}, \plc{ids})
integer \plc{place_num}
integer \plc{ids}(*)
\end{ompfSubroutine}
\end{fortranspecific}

\binding
The binding thread set for an \code{omp_get_place_proc_ids} region is all
threads on a device. The effect of executing this routine is not related to
any specific region corresponding to any construct or API routine.

\effect
The \code{omp_get_place_proc_ids} routine returns the numerical
identifiers of each processor associated with the place numbered
\plc{place_num}. The numerical identifiers are non-negative, and
their meaning is implementation defined.  The numerical identifiers
are returned in the array \plc{ids} and their order in the array is
implementation defined. The array must be sufficiently large to contain
\code{omp_get_place_num_procs(}\plc{place_num}\code{)} integers;
otherwise, the behavior is unspecified.  The routine has no effect when
\plc{place_num} has a negative value, or a value equal or larger
than \code{omp_get_num_places()}.

\crossreferences
\begin{itemize}
\item \code{omp_get_place_num_procs} routine, see
\specref{subsec:omp_get_place_num_procs}.

\item \code{omp_get_num_places} routine, see
\specref{subsec:omp_get_num_places}.

\item \code{OMP_PLACES} environment variable, see
\specref{sec:OMP_PLACES}.
\end{itemize}




\subsection{\hcode{omp_get_place_num}}
\index{omp_get_place_num@{\code{omp_get_place_num}}}
\label{subsec:omp_get_place_num}

\summary
The \code{omp_get_place_num} routine returns the place number of the place to which the encountering thread is bound.

\format
\begin{ccppspecific}
\begin{ompcFunction}
int omp_get_place_num(void);
\end{ompcFunction}
\end{ccppspecific}

\begin{fortranspecific}
\begin{ompfFunction}
integer function omp_get_place_num()
\end{ompfFunction}
\end{fortranspecific}

\binding
The binding thread set for an \code{omp_get_place_num} region is the encountering thread.

\effect
When the encountering thread is bound to a place, the
\code{omp_get_place_num} routine returns the place number associated
with the thread. The returned value is between 0 and one less than the
value returned by \code{omp_get_num_places()}, inclusive. When the
encountering thread is not bound to a place, the routine returns -1.

\crossreferences
\begin{itemize}
\item Controlling OpenMP thread affinity, see
\specref{subsec:Controlling OpenMP Thread Affinity}.

\item \code{omp_get_num_places} routine, see
\specref{subsec:omp_get_num_places}.

\item \code{OMP_PLACES} environment variable, see
\specref{sec:OMP_PLACES}.
\end{itemize}





\subsection{\hcode{omp_get_partition_num_places}}
\index{omp_get_partition_num_places@{\code{omp_get_partition_num_places}}}
\label{subsec:omp_get_partition_num_places}

\summary
The \code{omp_get_partition_num_places} routine returns the number of places in the place partition
of the innermost implicit task.

\format
\begin{ccppspecific}
\begin{ompcFunction}
int omp_get_partition_num_places(void);
\end{ompcFunction}
\end{ccppspecific}

\begin{fortranspecific}
\begin{ompfFunction}
integer function omp_get_partition_num_places()
\end{ompfFunction}
\end{fortranspecific}

\binding
The binding task set for an  \code{omp_get_partition_num_places} region is the encountering implicit task.

\effect
The \code{omp_get_partition_num_places} routine returns the number of places in the \plc{place-partition-var} ICV.

\crossreferences
\begin{itemize}
\item \plc{place-partition-var} ICV, see
\specref{sec:Internal Control Variables}.

\item Controlling OpenMP thread affinity, see
\specref{subsec:Controlling OpenMP Thread Affinity}.

\item \code{OMP_PLACES} environment variable, see
\specref{sec:OMP_PLACES}.
\end{itemize}





\subsection{\hcode{omp_get_partition_place_nums}}
\index{omp_get_partition_place_nums@{\code{omp_get_partition_place_nums}}}
\label{subsec:omp_get_partition_place_nums}

\summary
The \code{omp_get_partition_place_nums} routine returns the list of place numbers corresponding to the places in the \plc{place-partition-var} ICV of the innermost implicit task.

\format
\begin{ccppspecific}
\begin{ompcFunction}
void omp_get_partition_place_nums(int *\plc{place_nums});
\end{ompcFunction}
\end{ccppspecific}

\begin{fortranspecific}
\begin{ompfSubroutine}
subroutine omp_get_partition_place_nums(\plc{place_nums})
integer \plc{place_nums}(*)
\end{ompfSubroutine}
\end{fortranspecific}

\binding
The binding task set for an \code{omp_get_partition_place_nums} region is the encountering implicit task.

\effect
The \code{omp_get_partition_place_nums} routine returns the list of
place numbers corresponding to the places in the \plc{place-partition-var}
ICV of the innermost implicit task. The array must be sufficiently large
to contain \code{omp_get_partition_num_places()} integers; otherwise,
the behavior is unspecified.

\crossreferences
\begin{itemize}
\item \plc{place-partition-var} ICV, see
\specref{sec:Internal Control Variables}.

\item Controlling OpenMP thread affinity, see
\specref{subsec:Controlling OpenMP Thread Affinity}.

\item \code{omp_get_partition_num_places} routine, see
\specref{subsec:omp_get_partition_num_places}.

\item \code{OMP_PLACES} environment variable, see
\specref{sec:OMP_PLACES}.
\end{itemize}

\subsection{\hcode{omp_set_affinity_format}}
\index{omp_set_affinity_format@{\code{omp_set_affinity_format}}}
\label{subsec:omp_set_affinity_format}

\summary
The \code{omp_set_affinity_format} routine sets the affinity format to be used on the device
by setting the value of the \plc{affinity-format-var} ICV.

\format
\begin{ccppspecific}
\begin{ompcFunction}
void omp_set_affinity_format(char const *\plc{format});
\end{ompcFunction}
\end{ccppspecific}

\begin{fortranspecific}
\begin{ompfSubroutine}
subroutine omp_set_affinity_format(\plc{format})
character(len=*),intent(in)::\plc{format}
\end{ompfSubroutine}
\end{fortranspecific}

\binding
When called from a sequential part of the program, the binding thread set for an
\code{omp_set_affinity_format} region is the encountering thread. When called
from within any explicit parallel region, the binding thread set (and binding region, if
required) for the \code{omp_set_affinity_format} region is implementation defined.

\effect
The effect of \code{omp_set_affinity_format} routine is to copy the
character string specified by the \plc{format} argument into the
\plc{affinity-format-var} ICV on the current device.

This routine has the described effect only when called from a sequential part of the
program. When called from within an explicit \code{parallel} region, the effect of this
routine is implementation defined.

\crossreferences
\begin{itemize}
\item Controlling OpenMP thread affinity, see
\specref{subsec:Controlling OpenMP Thread Affinity}.
\item \code{OMP_DISPLAY_AFFINITY} environment variable, see
\specref{sec:OMP_DISPLAY_AFFINITY}.
\item \code{OMP_AFFINITY_FORMAT} environment variable, see
\specref{sec:OMP_AFFINITY_FORMAT}.
\item \code{omp_get_affinity_format} routine, see \specref{subsec:omp_get_affinity_format}.
\item \code{omp_display_affinity} routine, see \specref{subsec:omp_display_affinity}.
\item \code{omp_capture_affinity} routine, see \specref{subsec:omp_capture_affinity}.
\end{itemize}

\subsection{\hcode{omp_get_affinity_format}}
\index{omp_get_affinity_format@{\code{omp_get_affinity_format}}}
\label{subsec:omp_get_affinity_format}

\summary
The \code{omp_get_affinity_format} routine returns the value of the
\plc{affinity-format-var} ICV on the device.

\format
\begin{ccppspecific}
\begin{ompcFunction}
size_t omp_get_affinity_format(char *\plc{buffer}, size_t \plc{size});
\end{ompcFunction}
\end{ccppspecific}

\begin{fortranspecific}
\begin{ompfFunction}
integer function omp_get_affinity_format(\plc{buffer})
character(len=*),intent(out)::\plc{buffer}
\end{ompfFunction}
\end{fortranspecific}

\binding
When called from a sequential part of the program, the binding thread set for an
\code{omp_get_affinity_format} region is the encountering thread. When called
from within any explicit \code{parallel} region, the binding thread set (and binding region, if
required) for the \code{omp_get_affinity_format} region is implementation defined.

\effect
\begin{ccppspecific}
The \code{omp_get_affinity_format} routine returns the number of characters
in the \plc{affinity-format-var} ICV on the current device excluding the terminating
null byte (\code{'\0'}) and if \plc{size} is non-zero, writes the value of the
\plc{affinity-format-var} ICV on the current device to \plc{buffer} followed
by a null byte.  If the return value is larger or equal to \plc{size},
the affinity format specification is truncated, with the terminating
null byte stored to {\pcode{\plc{buffer}[\plc{size}-1]}}.  If \plc{size} is
zero, nothing is stored and \plc{buffer} may be \code{NULL}.
\end{ccppspecific}

\begin{fortranspecific}
The \code{omp_get_affinity_format} routine returns the number of characters required
to hold the \plc{affinity-format-var} ICV on the current device and writes the value of the
\plc{affinity-format-var} ICV on the current device to \plc{buffer}.
If the return value is larger than {\pcode{len(\plc{buffer})}}, the affinity
format specification is truncated.
\end{fortranspecific}

\crossreferences
\begin{itemize}
\item Controlling OpenMP thread affinity, see
\specref{subsec:Controlling OpenMP Thread Affinity}.
\item \code{OMP_DISPLAY_AFFINITY} environment variable, see
\specref{sec:OMP_DISPLAY_AFFINITY}.
\item \code{OMP_AFFINITY_FORMAT} environment variable, see
\specref{sec:OMP_AFFINITY_FORMAT}.
\item \code{omp_set_affinity_format} routine, see \specref{subsec:omp_set_affinity_format}.
\item \code{omp_display_affinity} routine, see \specref{subsec:omp_display_affinity}.
\item \code{omp_capture_affinity} routine, see \specref{subsec:omp_capture_affinity}.
\end{itemize}


\subsection{\hcode{omp_display_affinity}}
\index{omp_display_affinity@{\code{omp_display_affinity}}}
\label{subsec:omp_display_affinity}

\summary
The \code{omp_display_affinity} routine prints the OpenMP thread affinity information using the format specification provided.

\format
\begin{ccppspecific}
\begin{ompcFunction}
void omp_display_affinity(char const *\plc{format});
\end{ompcFunction}
\end{ccppspecific}

%\newpage %% HACK
\begin{fortranspecific}
\begin{ompfSubroutine}
subroutine omp_display_affinity(\plc{format})
character(len=*),intent(in)::\plc{format}
\end{ompfSubroutine}
\end{fortranspecific}

\binding
The binding thread set for an \code{omp_display_affinity} region is the encountering thread.

\effect
The \code{omp_display_affinity} routine prints the thread affinity information of the current thread in the format
specified by the \plc{format} argument, followed by a \plc{new-line}.
If the \plc{format} is \code{NULL} (for C/C++) or a zero-length string
(for Fortran and C/C++), the value of the \plc{affinity-format-var} ICV is used.

\crossreferences
\begin{itemize}
\item Controlling OpenMP thread affinity, see
\specref{subsec:Controlling OpenMP Thread Affinity}.
\item \code{OMP_DISPLAY_AFFINITY} environment variable, see
\specref{sec:OMP_DISPLAY_AFFINITY}.
\item \code{OMP_AFFINITY_FORMAT} environment variable, see
\specref{sec:OMP_AFFINITY_FORMAT}.
\item \code{omp_set_affinity_format} routine, see \specref{subsec:omp_set_affinity_format}.
\item \code{omp_get_affinity_format} routine, see \specref{subsec:omp_get_affinity_format}.
\item \code{omp_capture_affinity} routine, see \specref{subsec:omp_capture_affinity}.
\end{itemize}


\subsection{\hcode{omp_capture_affinity}}
\index{omp_capture_affinity@{\code{omp_capture_affinity}}}
\label{subsec:omp_capture_affinity}

\summary
The \code{omp_capture_affinity} routine prints the OpenMP thread affinity information into a buffer using
the format specification provided.

%\newpage %% HACK
\format
\begin{ccppspecific}
\begin{ompcFunction}
size_t omp_capture_affinity(
  char *\plc{buffer},
  size_t \plc{size},
  char const *\plc{format}
);
\end{ompcFunction}
\end{ccppspecific}

\begin{fortranspecific}
\begin{ompfFunction}
integer function omp_capture_affinity(\plc{buffer},\plc{format})
character(len=*),intent(out)::\plc{buffer}
character(len=*),intent(in)::\plc{format}
\end{ompfFunction}
\end{fortranspecific}

\binding
The binding thread set for an \code{omp_capture_affinity} region is the encountering thread.

\effect
\begin{ccppspecific}
The \code{omp_capture_affinity} routine returns the number of characters
in the entire thread affinity information string excluding the terminating
null byte (\code{'\0'}) and if \plc{size} is non-zero, writes the thread
affinity information of the current thread  in the format specified by the \plc{format} argument
into the character string \code{buffer} followed by null byte.  If the return value is larger
or equal to \plc{size}, the thread affinity information string is truncated, with the terminating
null byte stored to {\pcode{\plc{buffer}[\plc{size}-1]}}.  If \plc{size} is
zero, nothing is stored and \plc{buffer} may be \code{NULL}.
If the \plc{format} is \code{NULL} or a zero-length string, the value of the
\plc{affinity-format-var} ICV is used.
\end{ccppspecific}

\begin{fortranspecific}
The \code{omp_capture_affinity} routine returns the number of characters
required to hold the entire thread affinity information string
and prints the thread affinity information of the current thread
into the character string \code{buffer} with the size of
{\pcode{len(\plc{buffer})}} in the format specified by the \plc{format} argument.
If the \plc{format} is NULL (for C/C++)
or a zero-length string (for Fortran and C/C++), the value of the \plc{affinity-format-var} ICV is used.
If the return value is larger than {\pcode{len(\plc{buffer})}}, the thread affinity
information string is truncated.
If the \plc{format} is a zero-length string, the value of the \plc{affinity-format-var} ICV is used.
\end{fortranspecific}

\crossreferences
\begin{itemize}
\item Controlling OpenMP thread affinity, see
\specref{subsec:Controlling OpenMP Thread Affinity}.
\item \code{OMP_DISPLAY_AFFINITY} environment variable, see
\specref{sec:OMP_DISPLAY_AFFINITY}.
\item \code{OMP_AFFINITY_FORMAT} environment variable, see
\specref{sec:OMP_AFFINITY_FORMAT}.
\item \code{omp_set_affinity_format} routine, see \specref{subsec:omp_set_affinity_format}.
\item \code{omp_get_affinity_format} routine, see \specref{subsec:omp_get_affinity_format}.
\item \code{omp_display_affinity} routine, see \specref{subsec:omp_display_affinity}.
\end{itemize}


%%%%%%%%%%



\subsection{\hcode{omp_set_default_device}}
\index{omp_set_default_device@{\code{omp_set_default_device}}}
\label{subsec:omp_set_default_device}

\summary

The \code{omp_set_default_device} routine controls the default target device by
assigning the value of the \plc{default-device-var} ICV.

\format
\begin{ccppspecific}
\begin{ompcFunction}
void omp_set_default_device(int \plc{device_num});
\end{ompcFunction}
\end{ccppspecific}

\begin{fortranspecific}
\begin{ompfSubroutine}
subroutine omp_set_default_device(\plc{device_num})
integer \plc{device_num}
\end{ompfSubroutine}
\end{fortranspecific}

\binding
The binding task set for an \code{omp_set_default_device} region is the generating
task.

\effect
The effect of this routine is to set the value of the \plc{default-device-var} ICV of the current
task to the value specified in the argument. When called from within a \code{target} region
the effect of this routine is unspecified.

\crossreferences
\begin{itemize}
\item \plc{default-device-var}, see
\specref{sec:Internal Control Variables}.

\item \code{omp_get_default_device}, see
\specref{subsec:omp_get_default_device}.

\item \code{OMP_DEFAULT_DEVICE} environment variable, see
\specref{sec:OMP_DEFAULT_DEVICE}
\end{itemize}










\subsection{\hcode{omp_get_default_device}}
\index{omp_get_default_device@{\code{omp_get_default_device}}}
\label{subsec:omp_get_default_device}
\summary
The \code{omp_get_default_device} routine returns the default target device.

\format
\begin{ccppspecific}
\begin{ompcFunction}
int omp_get_default_device(void);
\end{ompcFunction}
\end{ccppspecific}

\begin{fortranspecific}
\begin{ompfFunction}
integer function omp_get_default_device()
\end{ompfFunction}
\end{fortranspecific}

\binding
The binding task set for an \code{omp_get_default_device} region is the generating
task.

\effect
The \code{omp_get_default_device} routine returns the value of the \plc{default-device-var}
ICV of the current task. When called from within a \code{target} region the effect of this
routine is unspecified.

\crossreferences
\begin{itemize}
\item \plc{default-device-var}, see
\specref{sec:Internal Control Variables}.

\item \code{omp_set_default_device}, see
\specref{subsec:omp_set_default_device}.

\item \code{OMP_DEFAULT_DEVICE} environment variable, see
\specref{sec:OMP_DEFAULT_DEVICE}.
\end{itemize}











\subsection{\hcode{omp_get_num_devices}}
\index{omp_get_num_devices@{\code{omp_get_num_devices}}}
\label{subsec:omp_get_num_devices}
\summary
The \code{omp_get_num_devices} routine returns the number of target devices.

\format
\begin{ccppspecific}
\begin{ompcFunction}
int omp_get_num_devices(void);
\end{ompcFunction}
\end{ccppspecific}

\begin{fortranspecific}
\begin{ompfFunction}
integer function omp_get_num_devices()
\end{ompfFunction}
\end{fortranspecific}

\binding
The binding task set for an \code{omp_get_num_devices} region is the generating task.

\effect
The \code{omp_get_num_devices} routine returns the number of available target devices.
When called from within a \code{target} region the effect of this routine is unspecified.

\crossreferences
None.











\subsection{\hcode{omp_get_device_num}}
\index{omp_get_device_num@{\code{omp_get_device_num}}}
\label{subsec:omp_get_device_num}
\summary
The \code{omp_get_device_num} routine returns the device number of the device on which
the calling thread is executing.

\format
\begin{ccppspecific}
\begin{ompcFunction}
int omp_get_device_num(void);
\end{ompcFunction}
\end{ccppspecific}

%\newpage %% HACK
\begin{fortranspecific}
\begin{ompfFunction}
integer function omp_get_device_num()
\end{ompfFunction}
\end{fortranspecific}

\binding
The binding task set for an \code{omp_get_devices_num} region is the generating task.

\effect
The \code{omp_get_device_num} routine returns the device number of the device on which
the calling thread is executing. When called on the host device, it will return the same
value as the \code{omp_get_initial_device} routine.

\crossreferences
\begin{itemize}
\item \code{omp_get_initial_device} routine, see
\specref{subsec:omp_get_initial_device}.
\end{itemize}










\subsection{\hcode{omp_get_num_teams}}
\index{omp_get_num_teams@{\code{omp_get_num_teams}}}
\label{subsec:omp_get_num_teams}
\summary
The \code{omp_get_num_teams} routine returns the number of initial teams in the current \code{teams}
region.

\format
\begin{ccppspecific}
\begin{ompcFunction}
int omp_get_num_teams(void);
\end{ompcFunction}
\end{ccppspecific}

\begin{fortranspecific}
\begin{ompfFunction}
integer function omp_get_num_teams()
\end{ompfFunction}
\end{fortranspecific}

\binding
The binding task set for an \code{omp_get_num_teams} region is the generating task

\effect
The effect of this routine is to return the number of initial teams in the current \code{teams} region.
The routine returns 1 if it is called from outside of a \code{teams} region.

\crossreferences
\begin{itemize}
\item \code{teams} construct, see
\specref{subsec:teams Construct}.
\end{itemize}











\subsection{\hcode{omp_get_team_num}}
\index{omp_get_team_num@{\code{omp_get_team_num}}}
\label{subsec:omp_get_team_num}
\summary
The \code{omp_get_team_num} routine returns the initial team number of the calling thread.

\format
\begin{ccppspecific}
\begin{ompcFunction}
int omp_get_team_num(void);
\end{ompcFunction}
\end{ccppspecific}

\begin{fortranspecific}
\begin{ompfFunction}
integer function omp_get_team_num()
\end{ompfFunction}
\end{fortranspecific}

\binding
The binding task set for an \code{omp_get_team_num} region is the generating task.

\effect
The \code{omp_get_team_num} routine returns the initial team number of the calling thread. The
initial team number is an integer between 0 and one less than the value returned by
\code{omp_get_num_teams()}, inclusive. The routine returns 0 if it is called outside of a
\code{teams} region.

\crossreferences
\begin{itemize}
\item \code{teams} construct, see
\specref{subsec:teams Construct}.

\item \code{omp_get_num_teams} routine, see
\specref{subsec:omp_get_num_teams}.
\end{itemize}









\subsection{\hcode{omp_is_initial_device}}
\index{omp_is_initial_device@{\code{omp_is_initial_device}}}
\label{subsec:omp_is_initial_device}
\summary
The \code{omp_is_initial_device} routine returns \plc{true} if the current task is executing
on the host device; otherwise, it returns \plc{false}.

\begin{samepage}
\format
\begin{ccppspecific}
\begin{ompcFunction}
int omp_is_initial_device(void);
\end{ompcFunction}
\end{ccppspecific}
\end{samepage}

\begin{fortranspecific}
\begin{ompfFunction}
logical function omp_is_initial_device()
\end{ompfFunction}
\end{fortranspecific}

\binding
The binding task set for an \code{omp_is_initial_device} region is the generating task.

\effect
The effect of this routine is to return \plc{true} if the current task is executing on the host
device; otherwise, it returns \plc{false}.

\crossreferences
\begin{itemize}
\item \code{target} construct, see
\specref{subsec:target Construct}
\end{itemize}





\subsection{\hcode{omp_get_initial_device}}
\index{omp_get_initial_device@{\code{omp_get_initial_device}}}
\label{subsec:omp_get_initial_device}
\summary
The \code{omp_get_initial_device} routine returns a device number representing
the host device.

\begin{samepage}
\format
\begin{ccppspecific}
\begin{ompcFunction}
int omp_get_initial_device(void);
\end{ompcFunction}
\end{ccppspecific}
\end{samepage}

\begin{fortranspecific}
\begin{ompfFunction}
integer function omp_get_initial_device()
\end{ompfFunction}
\end{fortranspecific}

\binding
The binding task set for an \code{omp_get_initial_device} region is the generating task.

\effect
The effect of this routine is to return the device number of the host device.
The value of the device number is implementation defined. If it is between 0
and one less than \code{omp_get_num_devices()} then it is valid for use
with all device constructs and routines; if it is outside that range, then
it is only valid for use with the device memory routines and not in the
\code{device} clause. When called from within a \code{target} region
the effect of this routine is unspecified.

\crossreferences
\begin{itemize}
\item \code{target} construct, see
\specref{subsec:target Construct}

\item Device memory routines, see \specref{sec:Device Memory Routines}.
\end{itemize}




\subsection{\hcode{omp_get_max_task_priority}}
\index{omp_get_max_task_priority@{\code{omp_get_max_task_priority}}}
\label{subsec:omp_get_max_task_priority}
\summary

The \code{omp_get_max_task_priority} routine returns the maximum value that can be
specified in the \code{priority} clause.

\begin{samepage}
\format
\begin{ccppspecific}
\begin{ompcFunction}
int omp_get_max_task_priority(void);
\end{ompcFunction}
\end{ccppspecific}
\end{samepage}

\begin{fortranspecific}
\begin{ompfFunction}
integer function omp_get_max_task_priority()
\end{ompfFunction}
\end{fortranspecific}

\binding

The binding thread set for an \code{omp_get_max_task_priority} region is all threads
on the device. The effect of executing this routine is not related to any specific region
corresponding to any construct or API routine.

\effect

The \code{omp_get_max_task_priority} routine returns the value of the \plc{max-task-priority-var}
ICV, which determines the maximum value that can be specified in the \code{priority} clause.

\crossreferences

\begin{itemize}
\item \plc{max-task-priority-var}, see
\specref{sec:Internal Control Variables}.

\item \code{task} construct, see
\specref{subsec:task Construct}.
\end{itemize}










%% \newpage

% This is an included file. See the master file for more information.
%
% When editing this file:
%
%    1. To change formatting, appearance, or style, please edit openmp.sty.
%
%    2. Custom commands and macros are defined in openmp.sty.
%
%    3. Be kind to other editors -- keep a consistent style by copying-and-pasting to
%       create new content.
%
%    4. We use semantic markup, e.g. (see openmp.sty for a full list):
%         \code{}     % for bold monospace keywords, code, operators, etc.
%         \plc{}      % for italic placeholder names, grammar, etc.
%
%    5. There are environments that provide special formatting, e.g. language bars.
%       Please use them whereever appropriate.  Examples are:
%
%         \begin{fortranspecific}
%         This is text that appears enclosed in blue language bars for Fortran.
%         \end{fortranspecific}
%
%         \begin{note}
%         This is a note.  The "Note -- " header appears automatically.
%         \end{note}
%
%    6. Other recommendations:
%         Use the convenience macros defined in openmp.sty for the minor headers
%         such as Comments, Syntax, etc.
%
%         To keep items together on the same page, prefer the use of
%         \begin{samepage}.... Avoid \parbox for text blocks as it interrupts line numbering.
%         When possible, avoid \filbreak, \pagebreak, \newpage, \clearpage unless that's
%         what you mean. Use \needspace{} cautiously for troublesome paragraphs.
%
%         Avoid absolute lengths and measures in this file; use relative units when possible.
%         Vertical space can be relative to \baselineskip or ex units. Horizontal space
%         can be relative to \linewidth or em units.
%
%         Prefer \emph{} to italicize terminology, e.g.:
%             This is a \emph{definition}, not a placeholder.
%             This is a \plc{var-name}.
%


\section{Lock Routines}
\index{lock routines}
\label{sec:Lock Routines}
The OpenMP runtime library includes a set of general-purpose lock 
routines that can be used for synchronization. These general-purpose 
lock routines operate on OpenMP locks that are represented by OpenMP 
lock variables. OpenMP lock variables must be accessed only through 
the routines described in this section; programs that otherwise
access OpenMP lock variables are non-conforming.

An OpenMP lock can be in one of the following states: \emph{uninitialized}; 
\emph{unlocked}; or \emph{locked}. If a lock is in the \emph{unlocked} state, 
a task can \emph{set} the lock, which changes its state to \emph{locked}. 
The task that sets the lock is then said to \emph{own} the lock. A task 
that owns a lock can \emph{unset} that lock, returning it to the 
\emph{unlocked} state. A program in which a task unsets a lock that is 
owned by another task is non-conforming.

Two types of locks are supported: \emph{simple locks} and \emph{nestable locks}. 
A \emph{nestable lock} can be set multiple times by the same task before being 
unset; a \emph{simple lock} cannot be set if it is already owned by the task 
trying to set it. \emph{Simple lock} variables are associated with 
\emph{simple locks} and can only be passed to \emph{simple lock} routines. 
\emph{Nestable lock} variables are associated with \emph{nestable locks} and 
can only be passed to \emph{nestable lock} routines.

Each type of lock can also have a \emph{synchronization hint} that contains 
information about the intended usage of the lock by the application code.  
The effect of the hint is implementation defined.  An OpenMP implementation 
can use this hint to select a usage-specific lock, but hints do not change 
the mutual exclusion semantics of locks. A conforming implementation can 
safely ignore the hint.

Constraints on the state and ownership of the lock accessed by each of the 
lock routines are described with the routine. If these constraints are not 
met, the behavior of the routine is unspecified.

The OpenMP lock routines access a lock variable such that they always read
and update the most current value of the lock variable. It is not necessary 
for an OpenMP program to include explicit \code{flush} directives to ensure 
that the lock variable's value is consistent among different tasks.

\binding
The binding thread set for all lock routine regions is all threads in the 
contention group. As a consequence, for each OpenMP lock, the lock routine 
effects relate to all tasks that call the routines, without regard to which 
teams the threads in the contention group that are executing the tasks belong.

\littleheader{Simple Lock Routines}
\index{Simple Lock Routines}
\begin{ccppspecific}
The type \code{omp_lock_t} represents a simple lock. For the following 
routines, a simple lock variable must be of \code{omp_lock_t} type. All 
simple lock routines require an argument that is a pointer to a variable 
of type \code{omp_lock_t}. 
\end{ccppspecific}

\begin{fortranspecific}
For the following routines, a simple lock variable must be an integer 
variable of \code.
\end{fortranspecific}

The simple lock routines are as follows:

\begin{itemize}
\item The \code{omp_init_lock} routine initializes a simple lock;
\item The \code{omp_init_lock_with_hint} routine initializes a simple 
      lock and attaches a hint to it;
\item The \code{omp_destroy_lock} routine uninitializes a simple lock;
\item The \code{omp_set_lock} routine waits until a simple lock is 
      available and then sets it;
\item The \code{omp_unset_lock} routine unsets a simple lock; and
\item The \code{omp_test_lock} routine tests a simple lock and sets 
      it if it is available.
\end{itemize}

\littleheader{Nestable Lock Routines}
\begin{ccppspecific}
The type \code{omp_nest_lock_t} represents a nestable lock. For the 
following routines, a nestable lock variable must be of 
\code{omp_nest_lock_t} type. All nestable lock routines require an 
argument that is a pointer to a variable of type \code{omp_nest_lock_t}.
\end{ccppspecific}

\begin{fortranspecific}
For the following routines, a nestable lock variable must be an 
integer variable of \code.
\end{fortranspecific}

The nestable lock routines are as follows:

\begin{itemize}
\item The \code{omp_init_nest_lock} routine initializes a nestable lock;
\item The \code{omp_init_nest_lock_with_hint} routine initializes a 
      nestable lock and attaches a hint to it;
\item The \code{omp_destroy_nest_lock} routine uninitializes a nestable lock;
\item The \code{omp_set_nest_lock} routine waits until a nestable lock 
      is available and then sets it;
\item The \code{omp_unset_nest_lock} routine unsets a nestable lock; and
\item The \code{omp_test_nest_lock} routine tests a nestable lock and 
      sets it if it is available.
\end{itemize}

\restrictions
OpenMP lock routines have the following restriction:

\begin{itemize}
\item The use of the same OpenMP lock in different contention groups 
      results in unspecified behavior.
\end{itemize}



\subsection{\hcode{omp_init_lock} and \hcode{omp_init_nest_lock}}
\label{subsec:omp_init_lock and omp_init_nest_lock}
\index{omp_init_lock@{\code{omp_init_lock}}}
\index{omp_init_nest_lock@{\code{omp_init_nest_lock}}}
\summary
These routines initialize an OpenMP lock without a hint.

\format
\begin{ccppspecific}
\begin{ompcFunction}
void omp_init_lock(omp_lock_t *\plc{lock});
void omp_init_nest_lock(omp_nest_lock_t *\plc{lock});
\end{ompcFunction}
\end{ccppspecific}

\begin{fortranspecific}
\begin{ompfSubroutine}
subroutine omp_init_lock(\plc{svar})
integer (kind=omp_lock_kind) \plc{svar}

subroutine omp_init_nest_lock(\plc{nvar})
integer (kind=omp_nest_lock_kind) \plc{nvar}
\end{ompfSubroutine}
\end{fortranspecific}

\constraints
A program that accesses a lock that is not in the uninitialized state 
through either routine is non-conforming.

\effect
The effect of these routines is to initialize the lock to the unlocked 
state; that is, no task owns the lock. In addition, the nesting count 
for a nestable lock is set to zero.

\events
The \plc{lock-init} event occurs in a thread that executes an 
\code{omp_init_lock} region after initialization of the lock, but 
before it finishes the region. The \plc{nest-lock-init} event occurs 
in a thread that executes an \code{omp_init_nest_lock} region
after initialization of the lock, but before it finishes the region.

\tools
A thread dispatches a registered \code{ompt_callback_lock_init} 
callback with \code{omp_sync_hint_none} as the \plc{hint} argument 
and \code{ompt_mutex_lock} as the \plc{kind} argument for each 
occurrence of a \plc{lock-init} event in that thread. Similarly, 
a thread dispatches a registered \code{ompt_callback_lock_init} 
callback with \code{omp_sync_hint_none} as the \plc{hint} argument 
and \code{ompt_mutex_nest_lock} as the \plc{kind} argument for each 
occurrence of a \plc{nest-lock-init} event in that thread. These 
callbacks have the type signature \code{ompt_callback_mutex_acquire_t}
and occur in the task that encounters the routine.

\begin{crossrefs}
\item \code{ompt_callback_mutex_acquire_t}, see
\specref{sec:ompt_callback_mutex_acquire_t}.
\end{crossrefs}



\subsection[\hcode{omp_init_lock_with_hint} and \\
\hcode{omp_init_nest_lock_with_hint}]{\code{omp_init_lock_with_hint} and \\
\code{omp_init_nest_lock_with_hint}}
\label{subsec:omp_init_lock_with_hint and omp_init_nest_lock_with_hint}
\index{omp_init_lock@{\code{omp_init_lock}}}
\index{omp_init_nest_lock@{\code{omp_init_nest_lock}}}
\summary
These routines initialize an OpenMP lock with a hint. The effect 
of the hint is implementation-defined. The OpenMP implementation
can ignore the hint without changing program semantics.

\format
\begin{ccppspecific}
\begin{ompcFunction}
void omp_init_lock_with_hint(
  omp_lock_t *\plc{lock},
  omp_sync_hint_t \plc{hint}
);
void omp_init_nest_lock_with_hint(
  omp_nest_lock_t *\plc{lock},
  omp_sync_hint_t \plc{hint}
);
\end{ompcFunction}
\end{ccppspecific}


\begin{fortranspecific}
\begin{ompfSubroutine}
subroutine omp_init_lock_with_hint(\plc{svar}, \plc{hint})
integer (kind=omp_lock_kind) \plc{svar}
integer (kind=omp_sync_hint_kind) \plc{hint}

subroutine omp_init_nest_lock_with_hint(\plc{nvar}, \plc{hint})
integer (kind=omp_nest_lock_kind) \plc{nvar}
integer (kind=omp_sync_hint_kind) \plc{hint}
\end{ompfSubroutine}
\end{fortranspecific}

\constraints
A program that accesses a lock that is not in the uninitialized 
state through either routine is non-conforming.

The second argument passed to these routines (\plc{hint}) is a hint
as described in \specref{subsec:Synchronization Hints}.

\effect
The effect of these routines is to initialize the lock to the unlocked 
state and, optionally, to choose a specific lock implementation based 
on the hint. After initialization no task owns the lock. In addition, 
the nesting count for a nestable lock is set to zero.

\events
The \plc{lock-init} event occurs in a thread that executes an 
\code{omp_init_lock_with_hint} region after initialization of the 
lock, but before it finishes the region. The \plc{nest-lock-init_with_hint} 
event occurs in a thread that executes an \code{omp_init_nest_lock} 
region after initialization of the lock, but before it finishes the region.

\tools
A thread dispatches a registered \code{ompt_callback_lock_init} 
callback with the same value for its \plc{hint} argument as the 
\plc{hint} argument of the call to \code{omp_init_lock_with_hint}
and \code{ompt_mutex_lock} as the \plc{kind} argument for each 
occurrence of a \plc{lock-init} event in that thread. Similarly, 
a thread dispatches a registered \code{ompt_callback_lock_init} 
callback with the same value for its \plc{hint} argument as the 
\plc{hint} argument of the call to \code{omp_init_nest_lock_with_hint} 
and \code{ompt_mutex_nest_lock} as the \plc{kind} argument for each 
occurrence of a \plc{nest-lock-init} event in that thread. These 
callbacks have the type signature \code{ompt_callback_mutex_acquire_t}
and occur in the task that encounters the routine.

\begin{crossrefs}
\item Synchronization Hints, see \specref{subsec:Synchronization Hints}.

\item \code{ompt_callback_mutex_acquire_t}, see
\specref{sec:ompt_callback_mutex_acquire_t}.
\end{crossrefs}



\subsection[\hcode{omp_destroy_lock} and \hcode{omp_destroy_nest_lock}]{\code{omp_destroy_lock} and\\ \code{omp_destroy_nest_lock}}
\label{subsec:omp_destroy_lock and omp_destroy_nest_lock}
\index{omp_destroy_lock@{\code{omp_destroy_lock}}}
\index{omp_destroy_nest_lock@{\code{omp_destroy_nest_lock}}}
\summary
These routines ensure that the OpenMP lock is uninitialized.

\format
\begin{ccppspecific}
\begin{ompcFunction}
void omp_destroy_lock(omp_lock_t *\plc{lock});
void omp_destroy_nest_lock(omp_nest_lock_t *\plc{lock});
\end{ompcFunction}
\end{ccppspecific}

\begin{fortranspecific}
\begin{ompfSubroutine}
subroutine omp_destroy_lock(\plc{svar})
integer (kind=omp_lock_kind) \plc{svar}

subroutine omp_destroy_nest_lock(\plc{nvar})
integer (kind=omp_nest_lock_kind) \plc{nvar}
\end{ompfSubroutine}
\end{fortranspecific}

\constraints
A program that accesses a lock that is not in the unlocked state 
through either routine is non-conforming.

\effect
The effect of these routines is to change the state of the lock to uninitialized.

\events
The \plc{lock-destroy} event occurs in a thread that executes an 
\code{omp_destroy_lock} region before it finishes the region. The 
\plc{nest-lock-destroy_with_hint} event occurs in a thread that 
executes an \code{omp_destroy_nest_lock} region before it finishes the region.

\tools
A thread dispatches a registered \code{ompt_callback_lock_destroy} 
callback with \code{ompt_mutex_lock} as the \plc{kind} argument for each 
occurrence of a \plc{lock-destroy} event in that thread. Similarly, 
a thread dispatches a registered \code{ompt_callback_lock_destroy} 
callback with \code{ompt_mutex_nest_lock} as the \plc{kind} argument for each 
occurrence of a \plc{nest-lock-destroy} event in that thread. These 
callbacks have the type signature \code{ompt_callback_mutex_acquire_t}
and occur in the task that encounters the routine.

\begin{crossrefs}
\item \code{ompt_callback_mutex_t}, see
\specref{sec:ompt_callback_mutex_t}.
\end{crossrefs}



\subsection{\hcode{omp_set_lock} and \hcode{omp_set_nest_lock}}
\label{subsec:omp_set_lock and omp_set_nest_lock}
\index{omp_set_lock@{\code{omp_set_lock}}}
\index{omp_set_nest_lock@{\code{omp_set_nest_lock}}}
\summary
These routines provide a means of setting an OpenMP lock. The calling
task region behaves as if it was suspended until the lock can be set
by this task.

\format
\begin{ccppspecific}
\begin{ompcFunction}
void omp_set_lock(omp_lock_t *\plc{lock});
void omp_set_nest_lock(omp_nest_lock_t *\plc{lock});
\end{ompcFunction}
\end{ccppspecific}

\begin{fortranspecific}
\begin{ompfSubroutine}
subroutine omp_set_lock(\plc{svar})
integer (kind=omp_lock_kind) \plc{svar}

subroutine omp_set_nest_lock(\plc{nvar})
integer (kind=omp_nest_lock_kind) \plc{nvar}
\end{ompfSubroutine}
\end{fortranspecific}

\constraints
A program that accesses a lock that is in the uninitialized state 
through either routine is non-conforming. A simple lock accessed by 
\code{omp_set_lock} that is in the locked state must not be owned 
by the task that contains the call or deadlock will result.

\effect
Each of these routines has an effect equivalent to suspension of the task
that is executing the routine until the specified lock is available.

\begin{note} The semantics of these routines is specified \emph{as if} 
they serialize execution of the region guarded by the lock. However, 
implementations may implement them in other ways provided that the 
isolation properties are respected so that the actual execution 
delivers a result that could arise from some serialization.
\end{note}

A simple lock is available if it is unlocked. Ownership of the lock is
granted to the task that executes the routine.

A nestable lock is available if it is unlocked or if it is already owned by
the task that executes the routine. The task that executes the routine is granted,
or retains, ownership of the lock, and the nesting count for the lock is
incremented.

\events
The \plc{lock-acquire} event occurs in a thread that executes an 
\code{omp_set_lock} region before the associated lock is requested.
The \plc{nest-lock-acquire} event occurs in a thread that executes
an \code{omp_set_nest_lock} region before the associated lock is requested.

The \plc{lock-acquired} event occurs in a thread that executes an 
\code{omp_set_lock} region after it acquires the associated lock but 
before it finishes the region. The \plc{nest-lock-acquired} event occurs 
in a thread that executes an \code{omp_set_nest_lock} region if the 
thread did not already own the lock, after it acquires the associated 
lock but before it finishes the region.

The \plc{nest-lock-owned} event occurs in a thread when it already owns the 
lock and executes an \code{omp_set_nest_lock} region. The event occurs after 
the nesting count is incremented but before the thread finishes the region.

\tools
A thread dispatches a registered \code{ompt_callback_mutex_acquire}
callback for each occurrence of a \plc{lock-acquire} or \plc{nest-lock-acquire} 
event in that thread.  This callback has the type signature 
\code{ompt_callback_mutex_acquire_t}.

A thread dispatches a registered \code{ompt_callback_mutex_acquired}
callback for each occurrence of a \plc{lock-acquired} or \plc{nest-lock-acquired} 
event in that thread.  This callback has the type signature 
\code{ompt_callback_mutex_t}.

A thread dispatches a registered \code{ompt_callback_nest_lock}
callback with \code{ompt_scope_begin} as its \plc{endpoint} argument
for each occurrence of a \plc{nest-lock-owned} event in that thread. 
This callback has the type signature \code{ompt_callback_nest_lock_t}.

The above callbacks occur in the task that encounters the lock function.  
The \plc{kind} argument of these callbacks is \code{ompt_mutex_lock} when
the events arise from an \code{omp_set_lock} region while it is 
 \code{ompt_mutex_nest_lock} when the events arise from an 
\code{omp_set_nest_lock} region.

\begin{crossrefs}
\item \code{ompt_callback_mutex_acquire_t}, see
\specref{sec:ompt_callback_mutex_acquire_t}.

\item \code{ompt_callback_mutex_t}, see
\specref{sec:ompt_callback_mutex_t}.

\item \code{ompt_callback_nest_lock_t}, see
\specref{sec:ompt_callback_nest_lock_t}.
\end{crossrefs}



\subsection{\hcode{omp_unset_lock} and \hcode{omp_unset_nest_lock}}
\label{subsec:omp_unset_lock and omp_unset_nest_lock}
\index{omp_unset_lock@{\code{omp_unset_lock}}}
\index{omp_unset_nest_lock@{\code{omp_unset_nest_lock}}}
\summary
These routines provide the means of unsetting an OpenMP lock.

\format
\begin{ccppspecific}
\begin{ompcFunction}
void omp_unset_lock(omp_lock_t *\plc{lock});
void omp_unset_nest_lock(omp_nest_lock_t *\plc{lock});
\end{ompcFunction}
\end{ccppspecific}

\begin{fortranspecific}
\begin{ompfSubroutine}
subroutine omp_unset_lock(\plc{svar})
integer (kind=omp_lock_kind) \plc{svar}

subroutine omp_unset_nest_lock(\plc{nvar})
integer (kind=omp_nest_lock_kind) \plc{nvar}
\end{ompfSubroutine}
\end{fortranspecific}

\constraints
A program that accesses a lock that is not in the locked state or that is
not owned by the task that contains the call through either routine is
non-conforming.

\effect
For a simple lock, the \code{omp_unset_lock} routine causes the lock to 
become unlocked.

For a nestable lock, the \code{omp_unset_nest_lock} routine decrements 
the nesting count, and causes the lock to become unlocked if the resulting 
nesting count is zero.

For either routine, if the lock becomes unlocked, and if one or more task
regions were effectively suspended because the lock was unavailable, the
effect is that one task is chosen and given ownership of the lock.

\events
The \plc{lock-release} event occurs in a thread that executes an 
\code{omp_unset_lock} region after it releases the associated lock but 
before it finishes the region. The \plc{nest-lock-release} event occurs 
in a thread that executes an \code{omp_unset_nest_lock} region after it 
releases the associated lock but before it finishes the region.

The \plc{nest-lock-held} event occurs in a thread that executes an 
\code{omp_unset_nest_lock} region before it finishes the region  when 
the thread still owns the lock after the nesting count is decremented.

\tools
A thread dispatches a registered \code{ompt_callback_mutex_released}
callback with \code{ompt_mutex_lock} as the \plc{kind} argument for 
each occurrence of a \plc{lock-release} event in that thread. Similarly,
a thread dispatches a registered \code{ompt_callback_mutex_released}
callback with \code{ompt_mutex_nest_lock} as the \plc{kind} argument for 
each occurrence of a \plc{nest-lock-release} event in that thread.
These callbacks have the type signature \code{ompt_callback_mutex_t}
and occur in the task that encounters the routine.

A thread dispatches a registered \code{ompt_callback_nest_lock}
callback with \code{ompt_scope_end} as its \plc{endpoint} argument
for each occurrence of a \plc{nest-lock-held} event in that thread. 
This callback has the type signature \code{ompt_callback_nest_lock_t}.

\begin{crossrefs}
\item \code{ompt_callback_mutex_t}, see
\specref{sec:ompt_callback_mutex_t}.

\item \code{ompt_callback_nest_lock_t}, see
\specref{sec:ompt_callback_nest_lock_t}.
\end{crossrefs}



\subsection{\hcode{omp_test_lock} and \hcode{omp_test_nest_lock}}
\label{subsec:omp_test_lock and omp_test_nest_lock}
\index{omp_test_lock@{\code{omp_test_lock}}}
\index{omp_test_nest_lock@{\code{omp_test_nest_lock}}}
\summary
These routines attempt to set an OpenMP lock but do not suspend execution 
of the task that executes the routine.

\format
\begin{ccppspecific}
\begin{ompcFunction}
int omp_test_lock(omp_lock_t *\plc{lock});
int omp_test_nest_lock(omp_nest_lock_t *\plc{lock});
\end{ompcFunction}
\end{ccppspecific}

\begin{fortranspecific}
\begin{ompfFunction}
logical function omp_test_lock(\plc{svar})
integer (kind=omp_lock_kind) \plc{svar}
integer function omp_test_nest_lock(\plc{nvar})
integer (kind=omp_nest_lock_kind) \plc{nvar}
\end{ompfFunction}
\end{fortranspecific}

\constraints
A program that accesses a lock that is in the uninitialized state 
through either routine is non-conforming. The behavior is unspecified 
if a simple lock accessed by \code{omp_test_lock} is in the locked state 
and is owned by the task that contains the call.

\effect
These routines attempt to set a lock in the same manner as \code{omp_set_lock} 
and \code{omp_set_nest_lock}, except that they do not suspend execution of the 
task that executes the routine.

For a simple lock, the \code{omp_test_lock} routine returns \plc{true} if 
the lock is successfully set; otherwise, it returns \plc{false}.

For a nestable lock, the \code{omp_test_nest_lock} routine returns the 
new nesting count if the lock is successfully set; otherwise, it returns zero.

\events
The \plc{lock-test} event occurs in a thread that executes an 
\code{omp_test_lock} region before the associated lock is tested.
The \plc{nest-lock-test} event occurs in a thread that executes 
an \code{omp_test_nest_lock} region before the associated lock is tested.

The \plc{lock-test-acquired} event occurs in a thread that executes an 
\code{omp_test_lock} region before it finishes the region if the associated 
lock was acquired. The \plc{nest-lock-test-acquired} event occurs in a 
thread that executes an \code{omp_test_nest_lock} region before it finishes
the region if the associated lock was acquired and the thread did not already 
own the lock.

The \plc{nest-lock-owned} event occurs in a thread that executes an 
\code{omp_test_nest_lock} region before it finishes the region after 
the nesting count is incremented if the thread already owned the lock.

\tools
A thread dispatches a registered \code{ompt_callback_mutex_acquire}
callback for each occurrence of a \plc{lock-test} or \plc{nest-lock-test} 
event in that thread. This callback has the type signature 
\code{ompt_callback_mutex_acquire_t}.

A thread dispatches a registered \code{ompt_callback_mutex_acquired}
callback for each occurrence of a \plc{lock-test-acquired} or 
\plc{nest-lock-test-acquired} event in that thread.  This callback has 
the type signature \code{ompt_callback_mutex_t}.

A thread dispatches a registered \code{ompt_callback_nest_lock}
callback with \code{ompt_scope_begin} as its \plc{endpoint} argument
for each occurrence of a \plc{nest-lock-owned} event in that thread. 
This callback has the type signature \code{ompt_callback_nest_lock_t}.

The above callbacks occur in the task that encounters the lock function.  
The \plc{kind} argument of these callbacks is \code{ompt_mutex_test_lock} when
the events arise from an \code{omp_test_lock} region while it is 
\code{ompt_mutex_test_nest_lock} when the events arise from an 
\code{omp_test_nest_lock} region.

\begin{crossrefs}
\item \code{ompt_callback_mutex_acquire_t}, see
\specref{sec:ompt_callback_mutex_acquire_t}.

\item \code{ompt_callback_mutex_t}, see
\specref{sec:ompt_callback_mutex_t}.

\item \code{ompt_callback_nest_lock_t}, see
\specref{sec:ompt_callback_nest_lock_t}.
\end{crossrefs}



\section{Timing Routines}
\index{timing routines}
\index{timer}
\index{wall clock timer}
\label{sec:Timing Routines}
This section describes routines that support a portable wall clock timer.



\subsection{\hcode{omp_get_wtime}}
\index{omp_get_wtime@{\code{omp_get_wtime}}}
\label{subsec:omp_get_wtime}
\summary
The \code{omp_get_wtime} routine returns elapsed wall clock time in seconds.

\format
\begin{ccppspecific}
\begin{ompcFunction}
double omp_get_wtime(void);
\end{ompcFunction}
\end{ccppspecific}

\begin{fortranspecific}
\begin{ompfFunction}
double precision function omp_get_wtime()
\end{ompfFunction}
\end{fortranspecific}

\binding
The binding thread set for an \code{omp_get_wtime} region is the 
encountering thread. The routine's return value is not guaranteed 
to be consistent across any set of threads.

\effect
The \code{omp_get_wtime} routine returns a value equal to the elapsed 
wall clock time in seconds since some \emph{time-in-the-past}. The actual 
\emph{time-in-the-past} is arbitrary, but it is guaranteed not to change 
during the execution of the application program. The time returned is 
a \emph{per-thread time}, so it is not required to be globally consistent 
across all threads that participate in an application.

\begin{note}
The routine is anticipated to be used to measure elapsed times as shown
in the following example:

\begin{ccppspecific}
\begin{ompcFunction}
double start;
double end;
start = omp_get_wtime();
\plc{... work to be timed ...}
end = omp_get_wtime();
printf("Work took %f seconds\n", end - start);
\end{ompcFunction}
\end{ccppspecific}

\begin{fortranspecific}
\begin{ompcFunction}
DOUBLE PRECISION START, END
START = omp_get_wtime()
\plc{... work to be timed ...}
END = omp_get_wtime()
PRINT *, "Work took", END - START, "seconds"
\end{ompcFunction}
\end{fortranspecific}
\end{note}



\subsection{\hcode{omp_get_wtick}}
\index{omp_get_wtick@{\code{omp_get_wtick}}}
\label{subsec:omp_get_wtick}
\summary
The \code{omp_get_wtick} routine returns the precision of the timer used 
by \code{omp_get_wtime}.

\format
\begin{ccppspecific}
\begin{ompcFunction}
double omp_get_wtick(void);
\end{ompcFunction}
\end{ccppspecific}

\begin{fortranspecific}
\begin{ompfFunction}
double precision function omp_get_wtick()
\end{ompfFunction}
\end{fortranspecific}

\binding
The binding thread set for an \code{omp_get_wtick} region is the encountering 
thread. The routine's return value is not guaranteed to be consistent across 
any set of threads.

\effect
The \code{omp_get_wtick} routine returns a value equal to the number of seconds 
between successive clock ticks of the timer used by \code{omp_get_wtime}.



\section{Event Routine}
\index{event routines}
\index{event}
\label{sec:Event Routine}

This section describes a routine that supports OpenMP event objects.

\binding

The binding thread set for all event routine regions is the encountering thread.

\subsection{\hcode{omp_fulfill_event}}
\index{omp_fulfill_event@{\code{omp_fulfill_event}}}
\label{subsec:omp_fulfill_event}
\summary
This routine fulfills and destroys an OpenMP event.

\format
\begin{ccppspecific}
\begin{ompcFunction}
void omp_fulfill_event(omp_event_handle_t event);
\end{ompcFunction}
\end{ccppspecific}

\begin{fortranspecific}
\begin{ompfFunction}
subroutine omp_fulfill_event(event)
integer (kind=omp_event_handle_kind) \plc{event}
\end{ompfFunction}
\end{fortranspecific}

\constraints
A program that calls this routine on an event that was already fulfilled 
is non-conforming. A program that calls this routine with an event handle 
that was not created by the \code{detach} clause is non-conforming.

\effect
The effect of this routine is to fulfill the event associated with the event 
handle argument. The effect of fulfilling the event will depend on how the 
event was created. The event is destroyed and cannot be accessed after calling 
this routine, and the event handle becomes unassociated with any event.

\events
The \plc{task-fulfill} event occurs in a thread that executes an
\code{omp_fulfill_event} region before the event is fulfilled if 
the OpenMP event object was created by a \code{detach} clause on a task.

\tools
A thread dispatches a registered \code{ompt_callback_task_schedule}
callback with \code{NULL} as its \plc{next_task_data} argument while
the argument \plc{prior_task_data} binds to the detached task for each 
occurrence of a \plc{task-fulfill} event. If the \plc{task-fulfill} event 
occurs before the detached task finished the execution of the associated
\plc{structured-block}, the callback has \code{ompt_task_early_fulfill} as 
its  \plc{prior_task_status} argument; otherwise the callback has 
\code{ompt_task_late_fulfill} as its \plc{prior_task_status} argument.
This callback has type signature \code{ompt_callback_task_schedule_t}.

\begin{crossrefs}
 \item \code{detach} clause, see \specref{subsec:task Construct}.

\item \code{ompt_callback_task_schedule_t}, see
\specref{sec:ompt_callback_task_schedule_t}.
\end{crossrefs}

% This is an included file. See the master file for more information.
%
% When editing this file:
%
%    1. To change formatting, appearance, or style, please edit openmp.sty.
%
%    2. Custom commands and macros are defined in openmp.sty.
%
%    3. Be kind to other editors -- keep a consistent style by copying-and-pasting to
%       create new content.
%
%    4. We use semantic markup, e.g. (see openmp.sty for a full list):
%         \code{}     % for bold monospace keywords, code, operators, etc.
%         \plc{}      % for italic placeholder names, grammar, etc.
%
%    5. There are environments that provide special formatting, e.g. language bars.
%       Please use them whereever appropriate.  Examples are:
%
%         \begin{fortranspecific}
%         This is text that appears enclosed in blue language bars for Fortran.
%         \end{fortranspecific}
%
%         \begin{note}
%         This is a note.  The "Note -- " header appears automatically.
%         \end{note}
%
%    6. Other recommendations:
%         Use the convenience macros defined in openmp.sty for the minor headers
%         such as Comments, Syntax, etc.
%
%         To keep items together on the same page, prefer the use of
%         \begin{samepage}.... Avoid \parbox for text blocks as it interrupts line numbering.
%         When possible, avoid \filbreak, \pagebreak, \newpage, \clearpage unless that's
%         what you mean. Use \needspace{} cautiously for troublesome paragraphs.
%
%         Avoid absolute lengths and measures in this file; use relative units when possible.
%         Vertical space can be relative to \baselineskip or ex units. Horizontal space
%         can be relative to \linewidth or em units.
%
%         Prefer \emph{} to italicize terminology, e.g.:
%             This is a \emph{definition}, not a placeholder.
%             This is a \plc{var-name}.
%


\begin{ccppspecific}

\section{Device Memory Routines}
\index{device memory routines}
\index{target memory routines}
\label{sec:Device Memory Routines}
This section describes routines that support allocation of memory and
management of pointers in the data environments of target devices.


\subsection{\hcode{omp_target_alloc}}
\index{omp_target_alloc@{\code{omp_target_alloc}}}
\label{subsec:omp_target_alloc}
\summary
The \code{omp_target_alloc} routine allocates memory in a device data
environment.


\format
\begin{ompcFunction}
void* omp_target_alloc(size_t \plc{size}, int \plc{device_num});
\end{ompcFunction}

\effect

The \code{omp_target_alloc} routine returns the device address of a storage
location of \plc{size} bytes. The storage location is dynamically allocated in
the device data environment of the device specified by \plc{device_num}, which
must be greater than or equal to zero and less than the result of
\code{omp_get_num_devices()} or the result of a call to
\code{omp_get_initial_device()}. When called from within a \code{target} region
the effect of this routine is unspecified.

The \code{omp_target_alloc} routine returns \code{NULL} if it cannot dynamically
allocate the memory in the device data environment.

The device address returned by \code{omp_target_alloc} can be used in an
\code{is_device_ptr} clause, \specref{subsec:target Construct}.

Unless \code{unified_address} clause appears on a \code{requires}
directive in the compilation unit, pointer arithmetic is not supported
on the device address returned by \code{omp_target_alloc}.

Freeing the storage returned by \code{omp_target_alloc} with any routine
other than \code{omp_target_free} results in unspecified behavior.

\events
The \plc{target-data-allocation} event occurs when a thread allocates data on a target device.

\tools
A thread invokes a registered \code{ompt_callback_target_data_op}
callback for each occurrence of a \plc{target-data-allocation} event in that thread.
The callback occurs in the context of the target task.  The callback has type signature
\code{ompt_callback_target_data_op_t}.

\crossreferences
\begin{itemize}
\item \code{target} construct, see
\specref{subsec:target Construct}

\item \code{omp_get_num_devices} routine, see
\specref{subsec:omp_get_num_devices}

\item \code{omp_get_initial_device} routine, see
\specref{subsec:omp_get_initial_device}

\item \code{omp_target_free} routine, see
\specref{subsec:omp_target_free}

\item \code{ompt_callback_target_data_op_t}, see
\specref{sec:ompt_callback_target_data_op_t}.
\end{itemize}



\subsection{\hcode{omp_target_free}}
\index{omp_target_free@{\code{omp_target_free}}}
\label{subsec:omp_target_free}
\summary
The \code{omp_target_free} routine frees the device memory allocated 
by the \code{omp_target_alloc} routine.

\format
\begin{ompcFunction}
void omp_target_free(void *\plc{device_ptr}, int \plc{device_num});
\end{ompcFunction}

\constraints

A program that calls \code{omp_target_free} with a non-null pointer
that does not have a value returned from \code{omp_target_alloc} is
non-conforming.  The \plc{device_num} must be greater than or equal to
zero and less than the result of \code{omp_get_num_devices()} or the
 result of a call to \code{omp_get_initial_device()}.

\effect

The \code{omp_target_free} routine frees the memory in the device data
environment associated with \plc{device_ptr}.  If \plc{device_ptr} is
\code{NULL}, the operation is ignored.

Synchronization must be inserted to ensure that all accesses to
\plc{device_ptr} are completed before the call to \code{omp_target_free}.

When called from within a \code{target} region the effect of this routine
is unspecified.

\events
The \plc{target-data-free} event occurs when a thread frees data on a target device.

\tools
A thread invokes a registered \code{ompt_callback_target_data_op}
callback for each occurrence of a \plc{target-data-free} event in that 
thread. The callback occurs in the context of the target task.  The 
callback has type signature \code{ompt_callback_target_data_op_t}.


\crossreferences
\begin{itemize}
\item \code{target} construct, see
\specref{subsec:target Construct}

\item \code{omp_get_num_devices} routine, see
\specref{subsec:omp_get_num_devices}

\item \code{omp_get_initial_device} routine, see
\specref{subsec:omp_get_initial_device}

\item \code{omp_target_alloc} routine, see
\specref{subsec:omp_target_alloc}

\item \code{ompt_callback_target_data_op_t}, see
\specref{sec:ompt_callback_target_data_op_t}.

\end{itemize}



\subsection{\hcode{omp_target_is_present}}
\index{omp_target_is_present@{\code{omp_target_is_present}}}
\label{subsec:omp_target_is_present}
\summary

The \code{omp_target_is_present} routine tests whether a host pointer
has corresponding storage on a given device.

\format
\begin{ompcFunction}
int omp_target_is_present(const void *\plc{ptr}, int \plc{device_num});
\end{ompcFunction}

\constraints
The value of \plc{ptr} must be a valid host pointer or \code{NULL}.
The \plc{device_num}
must be greater than or equal to zero and less than the result of
\code{omp_get_num_devices()} or the result of a call to
\code{omp_get_initial_device()}.

\effect
This routine returns non-zero if the specified pointer
would be found present on device \plc{device_num} by a \code{map}
clause; otherwise, it returns zero.

When called from within a \code{target} region
the effect of this routine is unspecified.

\crossreferences
\begin{itemize}
\item \code{target} construct, see \specref{subsec:target Construct}.

\item \code{map} clause, see \specref{subsec:map Clause}.

\item \code{omp_get_num_devices} routine, see
\specref{subsec:omp_get_num_devices}

\item \code{omp_get_initial_device} routine, see
\specref{subsec:omp_get_initial_device}
\end{itemize}


\subsection{\hcode{omp_target_memcpy}}
\index{omp_target_memcpy@{\code{omp_target_memcpy}}}
\label{subsec:omp_target_memcpy}
\summary

The \code{omp_target_memcpy} routine copies memory between any combination
of host and device pointers.

\format
\begin{ompcFunction}
int omp_target_memcpy(
  void *\plc{dst},
  const void *\plc{src},
  size_t \plc{length},
  size_t \plc{dst_offset},
  size_t \plc{src_offset},
  int \plc{dst_device_num},
  int \plc{src_device_num}
);
\end{ompcFunction}

\constraints
Each device must be compatible with the device pointer specified 
on the same side of the copy. The \plc{dst_device_num} and 
\plc{src_device_num} must be greater than or equal to zero and 
less than the result of \code{omp_get_num_devices()} or equal to 
the result of a call to \code{omp_get_initial_device()}.

\effect
\plc{length} bytes of memory at offset \plc{src_offset} from  \plc{src}
in the device data environment of device \plc{src_device_num} are
copied to \plc{dst} starting at offset \plc{dst_offset} in the device data
environment of device \plc{dst_device_num}. The return value is zero on 
success and non-zero on failure.  The host device and host device data 
environment can be referenced with the device number returned by 
\code{omp_get_initial_device}. This routine contains a task scheduling point.

When called from within a \code{target} region
the effect of this routine is unspecified.

\events
The \plc{target-data-op} event occurs when a thread transfers data on a target device.

\tools
A thread invokes a registered \code{ompt_callback_target_data_op}
callback for each occurrence of a \plc{target-data-op} event in that thread.
The callback occurs in the context of the target task.  The callback has type signature
\code{ompt_callback_target_data_op_t}.

\crossreferences
\begin{itemize}
\item \code{target} construct, see \specref{subsec:target Construct}.

\item \code{omp_get_initial_device} routine, see
\specref{subsec:omp_get_initial_device}

\item \code{omp_target_alloc} routine, see \specref{subsec:omp_target_alloc}.


\item \code{ompt_callback_target_data_op_t}, see
\specref{sec:ompt_callback_target_data_op_t}.
\end{itemize}



\subsection{\hcode{omp_target_memcpy_rect}}
\index{omp_target_memcpy_rect@{\code{omp_target_memcpy_rect}}}
\label{subsec:omp_target_memcpy_rect}
\summary
The \code{omp_target_memcpy_rect} routine copies a rectangular subvolume from
a multi-dimensional array to another multi-dimensional array. The copies can
use any combination of host and device pointers.

\format
\begin{samepage}
\begin{ompcFunction}
int omp_target_memcpy_rect(
  void *\plc{dst},
  const void *\plc{src},
  size_t \plc{element_size},
  int \plc{num_dims},
  const size_t *\plc{volume},
  const size_t *\plc{dst_offsets},
  const size_t *\plc{src_offsets},
  const size_t *\plc{dst_dimensions},
  const size_t *\plc{src_dimensions},
  int \plc{dst_device_num},
  int \plc{src_device_num}
);
\end{ompcFunction}
\end{samepage}

\constraints
The length of the offset and dimension arrays must be at least the
value of \plc{num_dims}. The \code{dst_device_num} and \code{src_device_num}
must be greater than or equal to zero and less than the result of
\code{omp_get_num_devices()} or equal to the result of a call to
\code{omp_get_initial_device()}.

The value of \plc{num_dims} must be between 1 and the implementation-defined
limit, which must be at least three.


\effect
This routine copies a rectangular subvolume of \plc{src},
in the device data environment of device \plc{src_device_num},
to \plc{dst}, in the device data environment of device \plc{dst_device_num}.
The volume is specified in terms of the size of an element,
number of dimensions, and constant arrays of length \plc{num_dims}.  The
maximum number of dimensions supported is at least three, support for higher
dimensionality is implementation defined. The volume array specifies the
length, in number of elements, to copy in each dimension from \plc{src}
to \plc{dst}. The \plc{dst_offsets} (\plc{src_offsets}) parameter specifies
number of elements from the origin of \plc{dst} (\plc{src}) in elements.
The \plc{dst_dimensions} (\plc{src_dimensions}) parameter specifies the
length of each dimension of \plc{dst} (\plc{src})

The routine returns zero if successful. If both \plc{dst} and \plc{src} are
\code{NULL} pointers, the routine returns the number of dimensions supported
by the implementation for the specified device numbers. The host device and
host device data environment can be referenced with the device number returned
by \code{omp_get_initial_device}.  Otherwise, it returns a non-zero value. The
routine contains a task scheduling point.

When called from within a \code{target} region
the effect of this routine is unspecified.

\events
The \plc{target-data-op} event occurs when a thread transfers data on a target device.

\tools

A thread invokes a registered \code{ompt_callback_target_data_op}
callback for each occurrence of a \plc{target-data-op} event in that 
thread. The callback occurs in the context of the target task.  The 
callback has type signature \code{ompt_callback_target_data_op_t}.

\crossreferences
\begin{itemize}
\item \code{target} construct, see \specref{subsec:target Construct}.

\item \code{omp_get_initial_device} routine, see
\specref{subsec:omp_get_initial_device}

\item \code{omp_target_alloc} routine, see \specref{subsec:omp_target_alloc}.

\item \code{ompt_callback_target_data_op_t}, see
\specref{sec:ompt_callback_target_data_op_t}.
\end{itemize}



\subsection{\hcode{omp_target_associate_ptr}}
\index{omp_target_associate_ptr@{\code{omp_target_associate_ptr}}}
\label{subsec:omp_target_associate_ptr}
\summary

The \code{omp_target_associate_ptr} routine maps a device pointer, which may
be returned from \code{omp_target_alloc} or implementation-defined runtime
routines, to a host pointer.

\format
\begin{ompcFunction}
int omp_target_associate_ptr(
  const void *\plc{host_ptr},
  const void *\plc{device_ptr},
  size_t \plc{size},
  size_t \plc{device_offset},
  int \plc{device_num}
);
\end{ompcFunction}

\constraints
The value of \plc{device_ptr} value must be a valid pointer to device
memory for the device denoted by the value of \plc{device_num}. The 
\plc{device_num} argument must be greater than or equal to zero and 
less than the result of \code{omp_get_num_devices()} or equal to the 
result of a call to \code{omp_get_initial_device()}.

\effect
The \code{omp_target_associate_ptr} routine associates a device pointer
in the device data environment of device \plc{device_num}
with a host pointer such that when the host pointer appears in a subsequent
\code{map} clause, the associated device pointer is used as the target for
data motion associated with that host pointer.  The \plc{device_offset}
parameter specifies what offset into \plc{device_ptr} will be used as the
base address for the device side of the mapping.  The reference count of the
resulting mapping will be infinite.  After being successfully associated, the
buffer pointed to by the device pointer is invalidated and accessing data
directly through the device pointer results in unspecified behavior.  The
pointer can be retrieved for other uses by disassociating it.
When called from within a \code{target} region
the effect of this routine is unspecified.

The routine returns zero if successful. Otherwise it returns a non-zero value.

Only one device buffer can be associated with a given host pointer value and
device number pair. Attempting to associate a second buffer will return
non-zero. Associating the same pair of pointers on the same device with the
same offset has no effect and returns zero.  Associating pointers that share
underlying storage will result in unspecified behavior. The
\code{omp_target_is_present} function can be used to test whether a given
host pointer has a corresponding variable in the device data environment.

\events
The \plc{target-data-associate} event occurs when a thread associates 
data on a target device.

\tools
A thread invokes a registered \code{ompt_callback_target_data_op}
callback for each occurrence of a \plc{target-data-associate} event in 
that thread. The callback occurs in the context of the target task.  
The callback has type signature \code{ompt_callback_target_data_op_t}.

\crossreferences
\begin{itemize}
\item \code{target} construct, see \specref{subsec:target Construct}.

\item \code{map} clause, see \specref{subsec:map Clause}.

\item \code{omp_target_alloc} routine, see \specref{subsec:omp_target_alloc}.

\item \code{omp_target_disassociate_ptr} routine, see
\specref{subsec:omp_target_associate_ptr}

\item \code{ompt_callback_target_data_op_t}, see
\specref{sec:ompt_callback_target_data_op_t}.
\end{itemize}



\subsection{\hcode{omp_target_disassociate_ptr}}
\index{omp_target_disassociate_ptr@{\code{omp_target_disassociate_ptr}}}
\label{subsec:omp_target_disassociate_ptr}
\summary

The \code{omp_target_disassociate_ptr} removes the associated pointer for a
given device from a host pointer.

\format
\begin{ompcFunction}
int omp_target_disassociate_ptr(const void *\plc{ptr}, int \plc{device_num});
\end{ompcFunction}

\constraints
The \plc{device_num} must be greater than or equal to zero and less 
than the result of \code{omp_get_num_devices()} or equal to the result 
of a call to \code{omp_get_initial_device()}.

\effect
The \code{omp_target_disassociate_ptr} removes the associated device data
on device \plc{device_num} from the presence table for host pointer
\plc{ptr}. A call to this routine on a pointer that is not
\code{NULL} and does not have associated data on the given device results
in unspecified behavior.  The reference count of the mapping is reduced to
zero, regardless of its current value.

When called from within a \code{target} region
the effect of this routine is unspecified.

The routine returns zero if successful. Otherwise it returns a non-zero value.

After a call to \code{omp_target_disassociate_ptr}, the contents of the device
buffer are invalidated.

\events
The \plc{target-data-disassociate} event occurs when a thread 
disassociates data on a target device.

\tools
A thread invokes a registered \code{ompt_callback_target_data_op}
callback for each occurrence of a \plc{target-data-disassociate} 
event in that thread. The callback occurs in the context of the 
target task.  The callback has type signature \code{ompt_callback_target_data_op_t}.

\crossreferences
\begin{itemize}
\item \code{target} construct, see
\specref{subsec:target Construct}

\item \code{omp_target_associate_ptr} routine, see
\specref{subsec:omp_target_associate_ptr}

\item \code{ompt_callback_target_data_op_t}, see
\specref{sec:ompt_callback_target_data_op_t}.
\end{itemize}

\end{ccppspecific}

\section{Memory Management Routines}
\index{memory management routines}
\label{sec:Memory Management Routines}
This section describes routines that support memory management on the current device.

Instances of memory management types must be accessed only through the routines described in this section; programs that otherwise access instances of these types are non-conforming.

\subsection{Memory Management Types}
\label{subsec:Memory Management Types}

The following type definitions are used by the memory management routines:

\begin{ccppspecific}
\begin{ompEnv}
typedef enum {
  OMP_ATK_THREADMODEL = 1,
  OMP_ATK_ALIGNMENT = 2,
  OMP_ATK_ACCESS = 3,
} omp_alloctrait_key_t;

typedef enum {
  OMP_ATV_FALSE = 0,   /* Reserved for future use */
  OMP_ATV_TRUE = 1,    /* Reserved for future use */
  OMP_ATV_DEFAULT = 2,
  OMP_ATV_CONTENDED = 3,
  OMP_ATV_UNCONTENDED = 4,
  OMP_ATV_SEQUENTIAL = 5,
  OMP_ATV_PRIVATE = 6,
  OMP_ATV_ALL = 7,
  OMP_ATV_THREAD = 8,
  OMP_ATV_PTEAM = 9,
  OMP_ATV_CGROUP = 10
} omp_alloctrait_value_t;

typedef struct {
  omp_alloctrait_key_t key;
  omp_uintptr_t value;
} omp_alloctrait_t;

omp_memspace_t;
omp_allocator_t;

enum { OMP_NULL_ALLOCATOR = NULL };
\end{ompEnv}
\end{ccppspecific}

\begin{fortranspecific}
\begin{ompfEnum}
integer, parameter :: omp_alloctrait_key_kind

integer(kind=omp_alloctrait_key_kind), &
   parameter :: omp_atk_threadmodel = 1
integer(kind=omp_alloctrait_key_kind), &
   parameter :: omp_atk_alignment = 2
integer(kind=omp_alloctrait_key_kind), &
   parameter :: omp_atk_access = 3
   
integer, parameter :: omp_alloctrait_val_kind

integer(kind=omp_alloctratit_val_kind), &
  parameter :: omp_atv_true = 0             ! Reserved for future use
integer(kind=omp_alloctratit_val_kind), &
  parameter :: omp_atv_false = 1            ! Reserved for future use
integer(kind=omp_alloctratit_val_kind), &
  parameter :: omp_atv_default = 2
integer(kind=omp_alloctratit_val_kind), &
  parameter :: omp_atv_contended = 3
integer(kind=omp_alloctratit_val_kind), &
  parameter :: omp_atv_uncontended = 4  
integer(kind=omp_alloctratit_val_kind), &
  parameter :: omp_atv_sequential = 5
integer(kind=omp_alloctratit_val_kind), &
  parameter :: omp_atv_private = 6  
integer(kind=omp_alloctratit_val_kind), &
  parameter :: omp_atv_all = 7
integer(kind=omp_alloctratit_val_kind), &
  parameter :: omp_atv_thread = 8 
integer(kind=omp_alloctratit_val_kind), &
  parameter :: omp_atv_pteam = 9
integer(kind=omp_alloctratit_val_kind), &
  parameter :: omp_atv_cgroup = 10
  
type omp_alloctrait
  integer(kind=omp_alloctrait_key_kind) key
  integer(kind=omp_alloctrait_val_kind) value
end type omp_alloctrait

integer, parameter :: omp_memspace_kind
integer, parameter :: omp_allocator_kind

integer(kind=omp_allocator_kind), &
        parameter :: omp_null_allocator = 0
\end{ompfEnum}
\end{fortranspecific}

\subsection{\hcode{omp_init_allocator}}
\index{omp_init_allocator@{\code{omp_init_allocator}}}
\label{subsec:omp_init_allocator}

\summary
The \code{omp_init_allocator} routine initializes an allocator and associates it with a memory space.

\format
\begin{ccppspecific}
\begin{ompcFunction}
omp_allocator_t * omp_init_allocator ( const omp_memspace_t *\plc{memspace}, const size_t \plc{ntraits}, const omp_alloctrait_t \plc{traits}[])
\end{ompcFunction}
\end{ccppspecific}
\begin{fortranspecific}
\begin{ompfFunction}
integer(kind=omp_allocator_kind) &
function omp_init_allocator ( \plc{memspace}, \plc{ntraits}, \plc{traits} )
integer(kind=omp_memspace_kind),intent(in) :: \plc{memspace}
integer,intent(in) :: \plc{ntraits}
type(omp_alloctrait),intent(in) :: \plc{traits}(*)
\end{ompfFunction}
\end{fortranspecific}

\constraints

The \plc{memspace} argument must be a predefined memory space.

If the \plc{ntraits} argument is greater than zero, then there must be at least as many traits
specified in the \plc{traits} argument. If there are fewer than \plc{ntraits} traits the behavior is
unspecified.

\binding

The binding thread set for an \code{omp_init_allocator} region is all threads on a device.
The effect of executing this routine is not related to any specific region corresponding to any construct or API routine.

\effect

The \code{omp_init_allocator} routine creates a new allocator that is associated with the the \plc{memspace} memory space. 
The allocations done through the created allocator will behave according to the allocator traits specified in the \plc{traits} argument.  The number of traits in the \plc{traits} argument is specified by the \plc{ntraits} argument. Specifying the same allocator trait more than once results in unspecified behavior. The routine returns a handle for the created allocator. If the special \code{OMP_ATV_DEFAULT} value is used for a given trait, then its value will be the default value specified in Table~\ref{tab:Allocator traits} for that given trait.

If the \plc{memspace} is \scode{omp_default_mem_space} and the \scode{traits} argument is an empty set this routine will always return a handle to an allocator. Otherwise if an allocator based on the requirements cannot be created then the special value \scode{OMP_NULL_ALLOCATOR} is returned.

\crossreferences
\begin{itemize}
\item Memory spaces in \specref{sec:Memory Spaces}
\item Allocators in \specref{sec:Memory Allocators}
\end{itemize}

\subsection{\hcode{omp_destroy_allocator}}
\index{omp_destroy_allocator@{\code{omp_destroy_allocator}}}
\label{subsec:omp_destroy_allocator}

\summary
The \code{omp_destroy_allocator} routine releases all resources used by the allocator handle and all memory allocations made through the allocator.

\format
\begin{ccppspecific}
\begin{ompcFunction}
void omp_destroy_allocator ( omp_allocator_t * \plc{allocator});
\end{ompcFunction}
\end{ccppspecific}
\begin{fortranspecific}
\begin{ompfSubroutine}
subroutine omp_destroy_allocator ( \plc{allocator} )
integer(kind=omp_allocator_kind),intent(in) :: \plc{allocator}
\end{ompfSubroutine}
\end{fortranspecific}

\constraints

The \plc{allocator} argument must not be a predefined memory allocator.

\binding

The binding thread set for an \code{omp_destroy_allocator} region is all threads on a device.
The effect of executing this routine is not related to any specific region corresponding to any construct or API routine.

\effect

The \code{omp_destroy_allocator} routine releases all resources used to implement the \plc{allocator} handle. Also, any memory allocated by the allocator but not yet deallocated  is deallocated by this routine. 

If the \plc{allocator} is \code{OMP_NULL_ALLOCATOR} then this routine will have no effect.
 
\crossreferences
\begin{itemize}
\item Allocators in \specref{sec:Memory Allocators}
\end{itemize}


\subsection{\hcode{omp_set_default_allocator}}
\index{omp_set_default_allocator@{\code{omp_set_default_allocator}}}
\label{subsec:omp_set_default_allocator}

\summary
The \code{omp_set_default_allocator} routine sets the default memory allocator to be used by allocation calls, \code{allocate} directives and \code{allocate} clauses that do not specify an allocator.

\format
\begin{ccppspecific}
\begin{ompcFunction}
void omp_set_default_allocator (const omp_allocator_t *\plc{allocator});
\end{ompcFunction}
\end{ccppspecific}
\begin{fortranspecific}
\begin{ompfSubroutine}
subroutine omp_set_default_allocator ( \plc{allocator} )
integer(kind=omp_allocator_kind),intent(in) :: \plc{allocator}
\end{ompfSubroutine}
\end{fortranspecific}

\constraints

The \plc{allocator} argument must point to a valid memory allocator.

\binding
The binding task set for an \code{omp_set_default_allocator} region is the \emph{binding implicit task}.

\effect

The effect of this routine is to set the value of the \plc{def-allocator-var} ICV of the \emph{binding implicit task} to the value specified in the \plc{allocator} argument.

\crossreferences

\begin{itemize}
\item \plc{def-allocator-var} ICV, see \specref{sec:Internal Control Variables}.
\item Memory Allocators, see \specref{sec:Memory Allocators}.
\item \code{omp_alloc} routine, see \specref{subsec:omp_alloc}.
\end{itemize}

\subsection{\hcode{omp_get_default_allocator}}
\index{omp_get_default_allocator@{\code{omp_get_default_allocator}}}
\label{subsec:omp_get_default_allocator}

\summary
The \code{omp_get_default_allocator} routine returns the memory allocator to be used by allocation calls, \code{allocate} directives and \code{allocate} clauses that do not specify an allocator.

\format
\begin{ccppspecific}
\begin{ompcFunction}
const omp_allocator_t * omp_get_default_allocator (void);
\end{ompcFunction}
\end{ccppspecific}
\begin{fortranspecific}
\begin{ompfFunction}
integer(kind=omp_allocator_kind)&
function omp_get_default_allocator ()
\end{ompfFunction}
\end{fortranspecific}

\binding

The binding task set for an \code{omp_get_default_allocator} region is the \emph{binding implicit task}.

\effect

The effect of this routine is to return the value of the \plc{def-allocator-var} ICV of the \emph{binding implicit task}.

\crossreferences
\begin{itemize}
\item \plc{def-allocator-var} ICV, see \specref{sec:Internal Control Variables}.
\item Memory Allocators, see \specref{sec:Memory Allocators}.
\item \code{omp_alloc} routine, see \specref{subsec:omp_alloc}.
\end{itemize}


%\newpage %% HACK
\vspace{3\baselineskip}
\begin{ccppspecific}
\vspace{-3\baselineskip}
\subsection{\hcode{omp_alloc}}
\index{omp_alloc@{\code{omp_alloc}}}
\label{subsec:omp_alloc}

\summary
The \code{omp_alloc} routine requests a memory allocation from a memory allocator.

\format
\begin{cspecific}
\begin{ompcFunction}
void * omp_alloc (size_t \plc{size}, const omp_allocator_t *\plc{allocator});
\end{ompcFunction}
\end{cspecific}
\begin{cppspecific}
\begin{ompcFunction}
void * omp_alloc (
  size_t \plc{size},
  const omp_allocator_t *\plc{allocator}=OMP_NULL_ALLOCATOR
);
\end{ompcFunction}
\end{cppspecific}

\constraints

For \code{omp_alloc} invocations appearing in \code{target} regions the \plc{allocator} argument cannot be \code{OMP_NULL_ALLOCATOR} and it must be a constant expression.

\effect

The \code{omp_alloc} routine requests a memory allocation of \plc{size} bytes from the specified memory allocator. If the \plc{allocator} argument is
\code{OMP_NULL_ALLOCATOR} the memory allocator used by the routine will be the one specified by the \plc{def-allocator-var} ICV of the \emph{binding implicit task}.
Upon success it returns a pointer to the allocated memory. Otherwise, it returns \code{NULL}.

\crossreferences
\begin{itemize}
\item Memory allocators, see \specref{sec:Memory Allocators}.
\end{itemize}

\subsection{\hcode{omp_free}}
\index{omp_free@{\code{omp_free}}}
\label{subsec:omp_free}

\summary
The \code{omp_free} routine deallocates previously allocated memory.

\format

\begin{cspecific}
\begin{ompcFunction}
void omp_free ( void *\plc{ptr}, const omp_allocator_t *\plc{allocator});
\end{ompcFunction}
\end{cspecific}
\begin{cppspecific}
\begin{ompcFunction}
void omp_free (
  void *\plc{ptr},
  const omp_allocator_t *\plc{allocator}=OMP_NULL_ALLOCATOR
);
\end{ompcFunction}
\end{cppspecific}

\effect

The \code{omp_free} routine deallocates the memory to which \plc{ptr} points. The \plc{ptr} argument must point to memory previously allocated with a memory allocator. If the \plc{allocator} argument is specified it must be the memory allocator to which the allocation request was made. If the \plc{allocator} argument is \code{OMP_NULL_ALLOCATOR} the implementation will find the memory allocator used to allocate the memory. Using \code{omp_free} on memory that was already deallocated results in unspecified behavior.

\end{ccppspecific}

% This is an included file. See the master file for more information.
%
% When editing this file:
%
%    1. To change formatting, appearance, or style, please edit openmp.sty.
%
%    2. Custom commands and macros are defined in openmp.sty.
%
%    3. Be kind to other editors -- keep a consistent style by copying-and-pasting to
%       create new content.
%
%    4. We use semantic markup, e.g. (see openmp.sty for a full list):
%         \code{}     % for bold monospace keywords, code, operators, etc.
%         \plc{}      % for italic placeholder names, grammar, etc.
%
%    5. There are environments that provide special formatting, e.g. language bars.
%       Please use them whereever appropriate.  Examples are:
%
%         \begin{fortranspecific}
%         This is text that appears enclosed in blue language bars for Fortran.
%         \end{fortranspecific}
%
%         \begin{note}
%         This is a note.  The "Note -- " header appears automatically.
%         \end{note}
%
%    6. Other recommendations:
%         Use the convenience macros defined in openmp.sty for the minor headers
%         such as Comments, Syntax, etc.
%
%         To keep items together on the same page, prefer the use of
%         \begin{samepage}.... Avoid \parbox for text blocks as it interrupts line numbering.
%         When possible, avoid \filbreak, \pagebreak, \newpage, \clearpage unless that's
%         what you mean. Use \needspace{} cautiously for troublesome paragraphs.
%
%         Avoid absolute lengths and measures in this file; use relative units when possible.
%         Vertical space can be relative to \baselineskip or ex units. Horizontal space
%         can be relative to \linewidth or em units.
%
%         Prefer \emph{} to italicize terminology, e.g.:
%             This is a \emph{definition}, not a placeholder.
%             This is a \plc{var-name}.
%


\section{Tool Control Routines}
\index{tool control}
\label{sec:control_tool}

\summary
The \code{omp_control_tool} routine enables a program to
pass commands to an active tool.

\format
\begin{ccppspecific}
\begin{ompcFunction}
int omp_control_tool(int \plc{command}, int \plc{modifier}, void *\plc{arg});
\end{ompcFunction}
\end{ccppspecific}

\begin{fortranspecific}
\begin{ompfFunction}
integer function omp_control_tool(\plc{command}, \plc{modifier})
integer (kind=omp_control_tool_kind) \plc{command}
integer (kind=omp_control_tool_kind) \plc{modifier}
\end{ompfFunction}
\end{fortranspecific}

\descr
An OpenMP program may use \code{omp_control_tool} to pass commands to
a tool. An application can use \code{omp_control_tool} to request that
a tool starts or restarts data collection when a code region of interest 
is encountered, that a tool pauses data collection when leaving the region 
of interest, that a tool flushes any data that it has collected so far, or
that a tool ends data collection. Additionally, \code{omp_control_tool} can 
be used to pass tool-specific commands to a particular tool.

The following types correspond to return values from \code{omp_control_tool}:

\begin{ccppspecific}
\begin{ompcEnum}
typedef enum omp_control_tool_result_t {
  omp_control_tool_notool = -2,
  omp_control_tool_nocallback = -1,
  omp_control_tool_success = 0,
  omp_control_tool_ignored = 1
} omp_control_tool_result_t;
\end{ompcEnum}
\end{ccppspecific}

\begin{fortranspecific}
\begin{ompfEnum}
integer (kind=omp_control_tool_result_kind), &
        parameter :: omp_control_tool_notool = -2
integer (kind=omp_control_tool_result_kind), &
        parameter :: omp_control_tool_nocallback = -1
integer (kind=omp_control_tool_result_kind), &
        parameter :: omp_control_tool_success = 0
integer (kind=omp_control_tool_result_kind), &
        parameter :: omp_control_tool_ignored = 1
\end{ompfEnum}
\end{fortranspecific}

If the OMPT interface state is inactive, the OpenMP implementation returns
\code{omp_control_tool_notool}. If the OMPT interface state is active, but
no callback is registered for the \plc{tool-control} event, the OpenMP
implementation returns \code{omp_control_tool_nocallback}. An OpenMP 
implementation may return other implementation-defined negative values 
strictly smaller than -64; an application may assume that any negative 
return value indicates that a tool has not received the command. A return 
value of \code{omp_control_tool_success} indicates that the tool has 
performed the specified command. A return value of 
\code{omp_control_tool_ignored} indicates that the tool has ignored the 
specified command. A tool may return other positive values strictly greater 
than 64 that are tool-defined.

\constraints
The following enumeration type defines four standard commands.
Table~\ref{table:std-tool-cmds} describes the
actions that these commands request from a tool.


\begin{ccppspecific}
\begin{ompcEnum}
typedef enum omp_control_tool_t {
  omp_control_tool_start = 1,
  omp_control_tool_pause = 2,
  omp_control_tool_flush = 3,
  omp_control_tool_end = 4
} omp_control_tool_t;
\end{ompcEnum}
\end{ccppspecific}

\begin{fortranspecific}
\begin{ompfEnum}
integer (kind=omp_control_tool_kind), &
          parameter :: omp_control_tool_start = 1
integer (kind=omp_control_tool_kind), &
          parameter :: omp_control_tool_pause = 2
integer (kind=omp_control_tool_kind), &
          parameter :: omp_control_tool_flush = 3
integer (kind=omp_control_tool_kind), &
          parameter :: omp_control_tool_end = 4
\end{ompfEnum}
\end{fortranspecific}

Tool-specific values for \plc{command} must be greater or equal to 64.
Tools must ignore \plc{command} values that they are not explicitly 
designed to handle. Other values accepted by a tool for \plc{command},
and any values for \plc{modifier} and \plc{arg} are tool-defined.



\nolinenumbers
\renewcommand{\arraystretch}{1.5}
\tablefirsthead{%
\hline
\textsf{\textbf{Command}} & \textsf{\textbf{Action}}\\
\hline\\[-3ex]
}
\tablehead{%
\multicolumn{2}{l}{\small\slshape table continued from previous page}\\
\hline
\textsf{\textbf{Command}} & \textsf{\textbf{Action}}\\
\hline\\[-3ex]
}
\tabletail{%
\hline\\[-4ex]
\multicolumn{2}{l}{\small\slshape table continued on next page}\\
}
\tablelasttail{\hline}
\tablecaption{Standard Tool Control Commands\label{table:std-tool-cmds}}
\begin{supertabular}{p{2in} p{3.0in}}
{\scode{omp_control_tool_start}} & Start or restart monitoring if it is 
                                   off. If monitoring is already on, this
                                   command is idempotent. If monitoring has 
                                   already been turned off permanently, this 
                                   command will have no effect.\\
{\scode{omp_control_tool_pause}} & Temporarily turn monitoring off. If 
                                   monitoring is already off, it is idempotent.\\
{\scode{omp_control_tool_flush}} & Flush any data buffered by a tool.
                                   This command may be applied whether 
                                   monitoring is on or off.\\
{\scode{omp_control_tool_end}}   & Turn monitoring off permanently;
                                   the tool finalizes itself and flushes all output.\\
\end{supertabular}

\linenumbers

\events
The \plc{tool-control} event occurs in the thread that encounters a call
to \code{omp_control_tool} at a point inside its corresponding OpenMP region.

\tools
A thread dispatches a registered \code{ompt_callback_control_tool} callback 
for each occurrence of a \plc{tool-control} event. The callback executes in 
the context of the call that occurs in the user program and has type signature 
\code{ompt_callback_control_tool_t}. The callback may return any non-negative 
value, which will be returned to the application by the OpenMP implementation 
as the return value of the \code{omp_control_tool} call that triggered the callback.

Arguments passed to the callback are those passed by the user to
\code{omp_control_tool}. If the call is made in Fortran, the tool
will be passed \code{NULL} as the third argument to the callback. If
any of the four standard commands is presented to a tool, the tool
will ignore the \plc{modifier} and \plc{arg} argument values.



\crossreferences
\begin{itemize}
\item Tool Interface, see
\specchapterref{chap:ToolsSupport}
\item \code{ompt_callback_control_tool_t}, see
\specref{sec:ompt_callback_control_tool_t}
\end{itemize}



