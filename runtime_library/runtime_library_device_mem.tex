% This is an included file. See the master file for more information.
%
% When editing this file:
%
%    1. To change formatting, appearance, or style, please edit openmp.sty.
%
%    2. Custom commands and macros are defined in openmp.sty.
%
%    3. Be kind to other editors -- keep a consistent style by copying-and-pasting to
%       create new content.
%
%    4. We use semantic markup, e.g. (see openmp.sty for a full list):
%         \code{}     % for bold monospace keywords, code, operators, etc.
%         \plc{}      % for italic placeholder names, grammar, etc.
%
%    5. There are environments that provide special formatting, e.g. language bars.
%       Please use them whereever appropriate.  Examples are:
%
%         \begin{fortranspecific}
%         This is text that appears enclosed in blue language bars for Fortran.
%         \end{fortranspecific}
%
%         \begin{note}
%         This is a note.  The "Note -- " header appears automatically.
%         \end{note}
%
%    6. Other recommendations:
%         Use the convenience macros defined in openmp.sty for the minor headers
%         such as Comments, Syntax, etc.
%
%         To keep items together on the same page, prefer the use of 
%         \begin{samepage}.... Avoid \parbox for text blocks as it interrupts line numbering.
%         When possible, avoid \filbreak, \pagebreak, \newpage, \clearpage unless that's
%         what you mean. Use \needspace{} cautiously for troublesome paragraphs.
%
%         Avoid absolute lengths and measures in this file; use relative units when possible.
%         Vertical space can be relative to \baselineskip or ex units. Horizontal space
%         can be relative to \linewidth or em units.
%
%         Prefer \emph{} to italicize terminology, e.g.:
%             This is a \emph{definition}, not a placeholder.
%             This is a \plc{var-name}.
%

\vspace{3\baselineskip}
\begin{ccppspecific}
\vspace{-3\baselineskip}
\section{Device Memory Routines}
\index{device memory routines}
\index{target memory routines}
\label{sec:Device Memory Routines}
This section describes routines that support allocation of memory and 
management of pointers in the data environments of target devices.


\subsection{\code{omp\_target\_alloc}}
\index{omp\_target\_alloc@{\code{omp\_target\_alloc}}}
\label{subsec:omp_target_alloc}
\summary
The \code{omp\_target\_alloc} routine allocates memory in a device data
environment.

\newpage %% HACK
\format
\begin{boxedcode}
void* omp\_target\_alloc(size\_t \plc{size}, int \plc{device\_num});
\end{boxedcode}

% blue line floater at top of this page for "C/C++, cont."
\begin{figure}[t!]
\linewitharrows{-1}{dashed}{C/C++ (cont.)}{8em}
\end{figure}
\effect

The \code{omp\_target\_alloc} routine returns the device address of a storage
location of \plc{size} bytes. The storage location is dynamically allocated in
the device data environment of the device specified by \plc{device\_num}, which
must be greater than or equal to zero and less than the result of
\code{omp\_get\_num\_devices()} or the result of a call to
\code{omp\_get\_initial\_device()}. When called from within a \code{target} region 
the effect of this routine is unspecified.

The \code{omp\_target\_alloc} routine returns \code{NULL} if it cannot dynamically
allocate the memory in the device data environment. 

The device address returned by \code{omp\_target\_alloc} can be used in an 
\code{is\_device\_ptr} clause, \specref{subsec:target Construct}.

Pointer arithmetic is not supported on the device address returned by
\code{omp\_target\_alloc}.

Freeing the storage returned by \code{omp\_target\_alloc} with any routine
other than \code{omp\_target\_free} results in unspecified behavior.

\events
The \plc{target-data-allocation} event occurs when a thread allocates data on a target device.

\tools

A thread invokes a registered \code{ompt\_callback\_target\_data\_op}
callback for each occurrence of a \plc{target-data-allocation} event in that thread. 
The callback occurs in the context of the target task.  The callback has type signature
\code{ompt\_callback\_target\_data\_op\_t}. 


\crossreferences
\begin{itemize}
\item \code{target} construct, see 
\specref{subsec:target Construct}

\item \code{omp\_get\_num\_devices} routine, see 
\specref{subsec:omp_get_num_devices}

\item \code{omp\_get\_initial\_device} routine, see 
\specref{subsec:omp_get_initial_device}

\item \code{omp\_target\_free} routine, see 
\specref{subsec:omp_target_free}

\item \code{ompt\_callback\_target\_data\_op\_t}, see 
\specref{sec:ompt_callback_target_data_op_t}.

\end{itemize}



\subsection{\code{omp\_target\_free}}
\index{omp\_target\_free@{\code{omp\_target\_free}}}
\label{subsec:omp_target_free}
\summary
The \code{omp\_target\_free} routine frees the device memory allocated by the \code{omp\_target\_alloc} routine.

% blue line floater at top of this page for "C/C++, cont."
\begin{figure}[t!]
\linewitharrows{-1}{dashed}{C/C++ (cont.)}{8em}
\end{figure}

\format
\begin{boxedcode}
void omp\_target\_free(void * \plc{device\_ptr}, int \plc{device\_num});
\end{boxedcode}

\constraints

A program that calls \code{omp\_target\_free} with a non-\code{NULL} pointer 
that does not have a value returned from \code{omp\_target\_alloc} is
non-conforming.  The \plc{device\_num} must be greater than or equal to 
zero and less than the result of \code{omp\_get\_num\_devices()} or the
 result of a call to \code{omp\_get\_initial\_device()}.

\effect

The \code{omp\_target\_free} routine frees the memory in the device data
environment associated with \plc{device\_ptr}.  If \plc{device\_ptr} is
\code{NULL}, the operation is ignored.

Synchronization must be inserted to ensure that all accesses to 
\plc{device\_ptr} are completed before the call to \code{omp\_target\_free}.

When called from within a \code{target} region the effect of this routine
is unspecified.

\events
The \plc{target-data-free} event occurs when a thread frees data on a target device.

\tools

A thread invokes a registered \code{ompt\_callback\_target\_data\_op}
callback for each occurrence of a \plc{target-data-free} event in that thread. 
The callback occurs in the context of the target task.  The callback has type signature
\code{ompt\_callback\_target\_data\_op\_t}. 


\crossreferences
\begin{itemize}
\item \code{target} construct, see 
\specref{subsec:target Construct}

\item \code{omp\_get\_num\_devices} routine, see 
\specref{subsec:omp_get_num_devices}

\item \code{omp\_get\_initial\_device} routine, see 
\specref{subsec:omp_get_initial_device}

\item \code{omp\_target\_alloc} routine, see 
\specref{subsec:omp_target_alloc}

\item \code{ompt\_callback\_target\_data\_op\_t}, see 
\specref{sec:ompt_callback_target_data_op_t}.

\end{itemize}



\subsection{\code{omp\_target\_is\_present}}
\index{omp\_target\_is\_present@{\code{omp\_target\_is\_present}}}
\label{subsec:omp_target_is_present}
\summary

The \code{omp\_target\_is\_present} routine tests whether a host pointer
has corresponding storage on a given device.

% blue line floater at top of this page for "C/C++, cont."
\begin{figure}[t!]
\linewitharrows{-1}{dashed}{C/C++ (cont.)}{8em}
\end{figure}

\format
\begin{boxedcode}
int omp\_target\_is\_present(const void * \plc{ptr}, int \plc{device\_num});
\end{boxedcode}

\constraints

The value of \plc{ptr} must be a valid host pointer or \code{NULL}.
The \plc{device\_num}
must be greater than or equal to zero and less than the result of
\code{omp\_get\_num\_devices()} or the result of a call to
\code{omp\_get\_initial\_device()}.

\effect

This routine returns non-zero if the specified pointer
would be found present on device \plc{device\_num} by a \code{map}
clause; otherwise, it returns zero.

When called from within a \code{target} region 
the effect of this routine is unspecified.

\crossreferences
\begin{itemize}
\item \code{target} construct, see \specref{subsec:target Construct}

\item \code{map} clause, see \specref{subsec:map Clause}.

\item \code{omp\_get\_num\_devices} routine, see 
\specref{subsec:omp_get_num_devices}

\item \code{omp\_get\_initial\_device} routine, see 
\specref{subsec:omp_get_initial_device}
\end{itemize}


\subsection{\code{omp\_target\_memcpy}}
\index{omp\_target\_memcpy@{\code{omp\_target\_memcpy}}}
\label{subsec:omp_target_memcpy}
\summary

The \code{omp\_target\_memcpy} routine copies memory between any combination
of host and device pointers.

% blue line floater at top of this page for "C/C++, cont."
\begin{figure}[t!]
\linewitharrows{-1}{dashed}{C/C++ (cont.)}{8em}
\end{figure}

\format
\begin{boxedcode}
int omp\_target\_memcpy(void * \plc{dst}, const void * \plc{src},
                        size\_t \plc{length}, size\_t \plc{dst\_offset},
                        size\_t \plc{src\_offset}, int \plc{dst\_device\_num},
                        int \plc{src\_device\_num});
\end{boxedcode}

\constraints
Each device must
be compatible with the device pointer specified on the same side of the copy.
The \plc{dst\_device\_num} and \plc{src\_device\_num}
must be greater than or equal to zero and less than the result of
\code{omp\_get\_num\_devices()} or equal to the result of a call to
\code{omp\_get\_initial\_device()}.

\effect


\plc{length} bytes of memory at offset \plc{src\_offset} from  \plc{src}
in the device data environment of device \plc{src\_device\_num} are
copied to \plc{dst} starting at offset \plc{dst\_offset} in the device data
environment of device \plc{dst\_device\_num}.  
The return value is zero on success and non-zero on failure.  The host device
and host device data environment can be referenced with the device number
returned by \code{omp\_get\_initial\_device}. This routine contains a task
scheduling point.

When called from within a \code{target} region 
the effect of this routine is unspecified.

\events
The \plc{target-data-op} event occurs when a thread transfers data on a target device.

% blue line floater at top of this page for "C/C++, cont."
\begin{figure}[t!]
\linewitharrows{-1}{dashed}{C/C++ (cont.)}{8em}
\end{figure}

\tools
A thread invokes a registered \code{ompt\_callback\_target\_data\_op}
callback for each occurrence of a \plc{target-data-op} event in that thread. 
The callback occurs in the context of the target task.  The callback has type signature
\code{ompt\_callback\_target\_data\_op\_t}. 


\crossreferences
\begin{itemize}
\item \code{target} construct, see \specref{subsec:target Construct}

\item \code{omp\_get\_initial\_device} routine, see 
\specref{subsec:omp_get_initial_device}

\item \code{omp\_target\_alloc} routine, see \specref{subsec:omp_target_alloc}


\item \code{ompt\_callback\_target\_data\_op\_t}, see 
\specref{sec:ompt_callback_target_data_op_t}.
\end{itemize}


\vspace{-1\baselineskip}
\subsection{\code{omp\_target\_memcpy\_rect}}
\index{omp\_target\_memcpy\_rect@{\code{omp\_target\_memcpy\_rect}}}
\label{subsec:omp_target_memcpy_rect}
\summary

The \code{omp\_target\_memcpy\_rect} routine copies a rectangular subvolume from
a multi-dimensional array to another multi-dimensional array. The copies can
use any combination of host and device pointers.

\format
\begin{samepage}
\begin{boxedcode}
int omp\_target\_memcpy\_rect(
               void * \plc{dst}, const void * \plc{src},
               size\_t \plc{element\_size},
               int \plc{num\_dims},
               const size\_t* \plc{volume},
               const size\_t* \plc{dst\_offsets},
               const size\_t* \plc{src\_offsets},
               const size\_t* \plc{dst\_dimensions},
               const size\_t* \plc{src\_dimensions},
               int \plc{dst\_device\_num}, int \plc{src\_device\_num});
\end{boxedcode}
\end{samepage}

\constraints
The length of the offset and dimension arrays must be at least the
value of \plc{num\_dims}.
The \code{dst\_device\_num} and \code{src\_device\_num}
must be greater than or equal to zero and less than the result of
\code{omp\_get\_num\_devices()} or equal to the result of a call to
\code{omp\_get\_initial\_device()}.

The value of \plc{num\_dims} must be between 1 and the implementation-defined
limit, which must be at least three.


\effect

This routine copies a rectangular subvolume of \plc{src},
in the device data environment of device \plc{src\_device\_num},
to \plc{dst}, in the device data environment of device \plc{dst\_device\_num}.
The volume is specified in terms of the size of an element, 
number of dimensions, and constant arrays of length \plc{num\_dims}.  The 
maximum number of dimensions supported is at least three, support for higher 
dimensionality is implementation defined. The volume array specifies the 
length, in number of elements, to copy in each dimension from \plc{src} 
to \plc{dst}. The \plc{dst\_offsets} (\plc{src\_offsets}) parameter specifies
number of elements from the origin of \plc{dst} (\plc{src}) in elements.  
The \plc{dst\_dimensions} (\plc{src\_dimensions}) parameter specifies the 
length of each dimension of \plc{dst} (\plc{src})

The routine returns zero if successful. If both \plc{dst} and \plc{src} are
\code{NULL} pointers, the routine returns the number of dimensions supported
by the implementation for the specified device numbers. The host device and
host device data environment can be referenced with the device number returned
by \code{omp\_get\_initial\_device}.  Otherwise, it returns a non-zero value. The
routine contains a task scheduling point.

When called from within a \code{target} region 
the effect of this routine is unspecified.

%% As an example, this routine could be used to copy a $3\times3\times3$ cube 
%% from the center of a $9\times9\times9$ cube in \code{src} to a \code{dst} 
%% matrix of size $3\times3\times3$ on the host as follows:
%%\begin{samepage}
%%\begin{boxedcode}
%%int src[9][9][9], dst[3][3][3];
%%size_t volume[]    = \{3, 3, 3\},
%%       dst_offset[]= \{0, 0, 0\},
%%       src_offset[]= \{3, 3, 3\},
%%       dst_dims[]  = \{3, 3, 3\},
%%       src_dims[]  = \{9, 9, 9\};
%%omp_target_memcpy_rect(
%%          dst, src,
%%          sizeof(int), //elem_size
%%          3, //num_dims
%%          volume,
%%          dst_offset,
%%          src_offset,
%%          dst_dims,
%%          src_dims,
%%          0, 0);
%%\end{boxedcode}
%%\end{samepage}

% blue line floater at top of this page for "C/C++, cont."
\begin{figure}[t!]
\linewitharrows{-1}{dashed}{C/C++ (cont.)}{8em}
\end{figure}

\events
The \plc{target-data-op} event occurs when a thread transfers data on a target device.

\tools

A thread invokes a registered \code{ompt\_callback\_target\_data\_op}
callback for each occurrence of a \plc{target-data-op} event in that thread. 
The callback occurs in the context of the target task.  The callback has type signature
\code{ompt\_callback\_target\_data\_op\_t}. 


\crossreferences
\begin{itemize}
\item \code{target} construct, see \specref{subsec:target Construct}

\item \code{omp\_get\_initial\_device} routine, see 
\specref{subsec:omp_get_initial_device}

\item \code{omp\_target\_alloc} routine, see \specref{subsec:omp_target_alloc}


\item \code{ompt\_callback\_target\_data\_op\_t}, see 
\specref{sec:ompt_callback_target_data_op_t}.

\end{itemize}

\subsection{\code{omp\_target\_associate\_ptr}}
\index{omp\_target\_associate\_ptr@{\code{omp\_target\_associate\_ptr}}}
\label{subsec:omp_target_associate_ptr}
\summary

The \code{omp\_target\_associate\_ptr} routine maps a device pointer, which may
be returned from \code{omp\_target\_alloc} or implementation-defined runtime 
routines, to a host pointer.

\format
\begin{boxedcode}
int omp\_target\_associate\_ptr(const void * \plc{host\_ptr},
                                const void * \plc{device\_ptr},
                                size\_t \plc{size}, size\_t \plc{device\_offset},
                                int \plc{device\_num});
\end{boxedcode}

\constraints

The value of \plc{device\_ptr} value must be a valid pointer to device 
memory for the device denoted by the value of \plc{device\_num}.
The \plc{device\_num} argument
must be greater than or equal to zero and less than the result of
\code{omp\_get\_num\_devices()} or equal to the result of a call to
\code{omp\_get\_initial\_device()}.

% blue line floater at top of this page for "C/C++, cont."
\begin{figure}[t!]
\linewitharrows{-1}{dashed}{C/C++ (cont.)}{8em}
\end{figure}

\effect

The \code{omp\_target\_associate\_ptr} routine associates a device pointer
in the device data environment of device \plc{device\_num}
with a host pointer such that when the host pointer appears in a subsequent
\code{map} clause, the associated device pointer is used as the target for
data motion associated with that host pointer.  The \plc{device\_offset}
parameter specifies what offset into \plc{device\_ptr} will be used as the
base address for the device side of the mapping.  The reference count of the
resulting mapping will be infinite.  After being successfully associated, the
buffer pointed to by the device pointer is invalidated and accessing data
directly through the device pointer results in unspecified behavior.  The
pointer can be retrieved for other uses by disassociating it.
When called from within a \code{target} region 
the effect of this routine is unspecified.

The routine returns zero if successful. Otherwise it returns a non-zero value.

Only one device buffer can be associated with a given host pointer value and
device number pair. Attempting to associate a second buffer will return
non-zero. Associating the same pair of pointers on the same device with the
same offset has no effect and returns zero.  Associating pointers that share
underlying storage will result in unspecified behavior. The
\code{omp\_target\_is\_present} region can be used to test whether a given
host pointer has a corresponding variable in the device data environment.

\events
The \plc{target-data-associate} event occurs when a thread associates data on a target device.

% blue line floater at top of this page for "C/C++, cont."
\begin{figure}[t!]
\linewitharrows{-1}{dashed}{C/C++ (cont.)}{8em}
\end{figure}

\tools

A thread invokes a registered \code{ompt\_callback\_target\_data\_op}
callback for each occurrence of a \plc{target-data-associate} event in that thread. 
The callback occurs in the context of the target task.  The callback has type signature
\code{ompt\_callback\_target\_data\_op\_t}. 

\crossreferences
\begin{itemize}
\item \code{target} construct, see \specref{subsec:target Construct}

\item \code{map} clause, see \specref{subsec:map Clause}.

\item \code{omp\_target\_alloc} routine, see \specref{subsec:omp_target_alloc}

\item \code{omp\_target\_disassociate\_ptr} routine, see 
\specref{subsec:omp_target_associate_ptr}

\item \code{ompt\_callback\_target\_data\_op\_t}, see 
\specref{sec:ompt_callback_target_data_op_t}.

\end{itemize}

\subsection{\code{omp\_target\_disassociate\_ptr}}
\index{omp\_target\_disassociate\_ptr@{\code{omp\_target\_disassociate\_ptr}}}
\label{subsec:omp_target_disassociate_ptr}
\summary

The \code{omp\_target\_disassociate\_ptr} removes the associated pointer for a
given device from a host pointer.

% blue line floater at top of this page for "C/C++, cont."
\begin{figure}[t!]
\linewitharrows{-1}{dashed}{C/C++ (cont.)}{8em}
\end{figure}

\format
\begin{boxedcode}
int omp\_target\_disassociate\_ptr(const void * \plc{ptr}, int \plc{device\_num});
\end{boxedcode}

\constraints


The \plc{device\_num}
must be greater than or equal to zero and less than the result of
\code{omp\_get\_num\_devices()} or equal to the result of a call to
\code{omp\_get\_initial\_device()}.

\effect

The \code{omp\_target\_disassociate\_ptr} removes the associated device data
on device \plc{device\_num} from the presence table for host pointer
\plc{ptr}. A call to this routine on a pointer that is not 
\code{NULL} and does not have associated data on the given device results
in unspecified behavior.  The reference count of the mapping is reduced to 
zero, regardless of its current value.

When called from within a \code{target} region 
the effect of this routine is unspecified.

After a call to \code{omp\_target\_disassociate\_ptr}, the contents of the device
buffer are invalidated.

\events
The \plc{target-data-disassociate} event occurs when a thread disassociates data on a target device.

\tools

A thread invokes a registered \code{ompt\_callback\_target\_data\_op}
callback for each occurrence of a \plc{target-data-disassociate} event in that thread. 
The callback occurs in the context of the target task.  The callback has type signature
\code{ompt\_callback\_target\_data\_op\_t}. 


\crossreferences
\begin{itemize}
\item \code{target} construct, see 
\specref{subsec:target Construct}
\item \code{omp\_target\_associate\_ptr} routine, see 
\specref{subsec:omp_target_associate_ptr}
\item \code{ompt\_callback\_target\_data\_op\_t}, see 
\specref{sec:ompt_callback_target_data_op_t}.

\end{itemize}

\end{ccppspecific}
