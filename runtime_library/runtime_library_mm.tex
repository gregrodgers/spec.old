\section{Memory Management Routines}
\index{memory management routines}
\label{sec:Memory Management Routines}
This section describes routines that support memory management on the current device.

Instances of memory management types must be accessed only through the routines described in this section; programs that otherwise access instances of these types are non-conforming.

\subsection{Memory Management Types}
\label{subsec:Memory Management Types}

The following type definitions are used by the memory management routines:

\begin{ccppspecific}
\begin{ompEnv}
omp_allocator_t;

enum { OMP_NULL_ALLOCATOR = NULL };
\end{ompEnv}
\end{ccppspecific}

\begin{fortranspecific}
\begin{ompfEnum}
integer parameter omp_allocator_kind

integer(kind=omp_allocator_kind), &
        parameter :: omp_null_allocator = 0
\end{ompfEnum}
\end{fortranspecific}

\subsection{\hcode{omp_set_default_allocator}}
\index{omp_set_default_allocator@{\code{omp_set_default_allocator}}}
\label{subsec:omp_set_default_allocator}

\summary
The \code{omp_set_default_allocator} routine sets the default memory allocator to be used by allocation calls, \code{allocate} directives and \code{allocate} clauses that do not specify an allocator.

\format
\begin{ccppspecific}
\begin{ompcFunction}
void omp_set_default_allocator (const omp_allocator_t *\plc{allocator});
\end{ompcFunction}
\end{ccppspecific}
\begin{fortranspecific}
\begin{ompfSubroutine}
subroutine omp_set_default_allocator ( \plc{allocator} )
integer(kind=omp_allocator_kind),intent(in) :: allocator
\end{ompfSubroutine}
\end{fortranspecific}

\constraints

The \plc{allocator} argument must point to a valid memory allocator.

\binding
The binding task set for an \code{omp_set_default_allocator} region is the \emph{binding implicit task}.

\effect

The effect of this routine is to set the value of the \plc{def-allocator-var} ICV of the \emph{binding implicit task} to the value specified in the \plc{allocator} argument. 

\crossreferences

\begin{itemize}
\item \plc{def-allocator-var} ICV, see \specref{sec:Internal Control Variables}.
\item Memory Allocators, see \specref{sec:Memory Allocators}.
\item \code{omp_alloc} routine, see \specref{subsec:omp_alloc}.
\end{itemize}

\subsection{\hcode{omp_get_default_allocator}}
\index{omp_get_default_allocator@{\code{omp_get_default_allocator}}}
\label{subsec:omp_get_default_allocator}

\summary
The \code{omp_get_default_allocator} routine returns the memory allocator to be used by allocation calls, \code{allocate} directives and \code{allocate} clauses that do not specify an allocator.

\format
\begin{ccppspecific}
\begin{ompcFunction}
const omp_allocator_t * omp_get_default_allocator (void);
\end{ompcFunction}
\end{ccppspecific}
\begin{fortranspecific}
\begin{ompfFunction}
integer(kind=omp_allocator_kind)
function omp_get_default_allocator ()
\end{ompfFunction}
\end{fortranspecific}

\binding

The binding task set for an \code{omp_get_default_allocator} region is the \emph{binding implicit task}.

\effect

The effect of this routine is to return the value of the \plc{def-allocator-var} ICV of the \emph{binding implicit task}.

\crossreferences
\begin{itemize}
\item \plc{def-allocator-var} ICV, see \specref{sec:Internal Control Variables}.
\item Memory Allocators, see \specref{sec:Memory Allocators}.
\item \code{omp_alloc} routine, see \specref{subsec:omp_alloc}.
\end{itemize}


%\newpage %% HACK
\vspace{3\baselineskip}
\begin{ccppspecific}
\vspace{-3\baselineskip}
\subsection{\hcode{omp_alloc}}
\index{omp_alloc@{\code{omp_alloc}}}
\label{subsec:omp_alloc}

\summary
The \code{omp_alloc} routine requests a memory allocation from a memory allocator.

\format
\begin{cspecific}
\begin{ompcFunction}
void * omp_alloc (size_t \plc{size}, const omp_allocator_t *\plc{allocator});
\end{ompcFunction}
\end{cspecific}
\begin{cppspecific}
\begin{ompcFunction}
void * omp_alloc (
  size_t \plc{size},
  const omp_allocator_t *\plc{allocator}=OMP_NULL_ALLOCATOR
);
\end{ompcFunction}
\end{cppspecific}

\constraints

For \code{omp_alloc} invocations appearing in \code{target} regions the \plc{allocator} argument cannot be \code{OMP_NULL_ALLOCATOR} and it must be a constant expression.

\effect

The \code{omp_alloc} routine requests a memory allocation of \plc{size} bytes from the specified memory allocator. If the \plc{allocator} argument is 
\code{OMP_NULL_ALLOCATOR} the memory allocator used by the routine will be the one specified by the \plc{def-allocator-var} ICV of the \emph{binding implicit task}.
Upon success it returns a pointer to the allocated memory. Otherwise, it returns \code{NULL}.

\crossreferences
\begin{itemize}
\item Memory allocators, see \specref{sec:Memory Allocators}.
\end{itemize}

\subsection{\hcode{omp_free}}
\index{omp_free@{\code{omp_free}}}
\label{subsec:omp_free}

\summary
The \code{omp_free} routine deallocates previously allocated memory. 

\format

\begin{cspecific}
\begin{ompcFunction}
void omp_free ( void *\plc{ptr}, const omp_allocator_t *\plc{allocator});
\end{ompcFunction}
\end{cspecific}
\begin{cppspecific}
\begin{ompcFunction}
void omp_free (
  void *\plc{ptr},
  const omp_allocator_t *\plc{allocator}=OMP_NULL_ALLOCATOR
);
\end{ompcFunction}
\end{cppspecific}

\effect

The \code{omp_free} routine deallocates the memory to which \plc{ptr} points. The \plc{ptr} argument must point to memory previously allocated with a memory allocator. If the \plc{allocator} argument is specified it must be the memory allocator to which the allocation request was made. If the \plc{allocator} argument is \code{OMP_NULL_ALLOCATOR} the implementation will find the memory allocator used to allocate the memory. Using \code{omp_free} on memory that was already deallocated results in unspecified behavior.

\end{ccppspecific}
