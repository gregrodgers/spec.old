\newpage  %% HACK
\section{Memory Management Routines}
\index{memory management routines}
\label{sec:Memory Management Routines}
This section describes routines that support memory management on the current device.

Instances of memory management types must be accessed only through the routines described in this section; programs that otherwise access instances of these types are non-conforming.

\subsection{Memory Management Types}
\label{subsec:Memory Management Types}

The following type definitions are used by the memory management routines:

\begin{ccppspecific}
\begin{ompEnv}
typedef enum {
  OMP_ATK_THREADMODEL = 1,
  OMP_ATK_ALIGNMENT = 2,
  OMP_ATK_ACCESS = 3,
  OMP_ATK_POOL_SIZE = 4,
  OMP_ATK_FALLBACK = 5,
  OMP_ATK_FB_DATA = 6,
  OMP_ATK_PINNED = 7,
  OMP_ATK_PARTITION = 8
} omp_alloctrait_key_t;

typedef enum {
  OMP_ATV_FALSE = 0,   
  OMP_ATV_TRUE = 1,    
  OMP_ATV_DEFAULT = 2,
  OMP_ATV_CONTENDED = 3,
  OMP_ATV_UNCONTENDED = 4,
  OMP_ATV_SEQUENTIAL = 5,
  OMP_ATV_PRIVATE = 6,
  OMP_ATV_ALL = 7,
  OMP_ATV_THREAD = 8,
  OMP_ATV_PTEAM = 9,
  OMP_ATV_CGROUP = 10,
  OMP_ATV_DEFAULT_MEM_FB = 11,
  OMP_ATV_NULL_FB = 12,
  OMP_ATV_ABORT_FB = 13,
  OMP_ATV_ALLOCATOR_FB = 14,
  OMP_ATV_ENVIRONMENT = 15,
  OMP_ATV_NEAREST = 16,
  OMP_ATV_BLOCKED = 17,
  OMP_ATV_INTERLEAVED = 18
} omp_alloctrait_value_t;

typedef struct {
  omp_alloctrait_key_t key;
  omp_uintptr_t value;
} omp_alloctrait_t;

enum { OMP_NULL_ALLOCATOR = NULL };
\end{ompEnv}
\end{ccppspecific}

\begin{fortranspecific}
\begin{ompfEnum}

integer(kind=omp_alloctrait_key_kind), &
   parameter :: omp_atk_threadmodel = 1
integer(kind=omp_alloctrait_key_kind), &
   parameter :: omp_atk_alignment = 2
integer(kind=omp_alloctrait_key_kind), &
   parameter :: omp_atk_access = 3   
integer(kind=omp_alloctrait_key_kind), &   
   parameter :: omp_atk_pool_size = 4
integer(kind=omp_alloctrait_key_kind), &
   parameter :: omp_atk_fallback = 5
integer(kind=omp_alloctrait_key_kind), &
   parameter :: omp_atk_fb_data = 6
integer(kind=omp_alloctrait_key_kind), &
   parameter :: omp_atk_pinned = 7
integer(kind=omp_alloctrait_key_kind), &
   parameter :: omp_atk_partition = 8

integer(kind=omp_alloctrait_val_kind), &
  parameter :: omp_atv_false = 0             ! Reserved for future use
integer(kind=omp_alloctrait_val_kind), &
  parameter :: omp_atv_true = 1              ! Reserved for future use
integer(kind=omp_alloctrait_val_kind), &
  parameter :: omp_atv_default = 2
integer(kind=omp_alloctrait_val_kind), &
  parameter :: omp_atv_contended = 3
integer(kind=omp_alloctrait_val_kind), &
  parameter :: omp_atv_uncontended = 4  
integer(kind=omp_alloctrait_val_kind), &
  parameter :: omp_atv_sequential = 5
integer(kind=omp_alloctrait_val_kind), &
  parameter :: omp_atv_private = 6  
integer(kind=omp_alloctrait_val_kind), &
  parameter :: omp_atv_all = 7
integer(kind=omp_alloctrait_val_kind), &
  parameter :: omp_atv_thread = 8 
integer(kind=omp_alloctrait_val_kind), &
  parameter :: omp_atv_pteam = 9
integer(kind=omp_alloctrait_val_kind), &
  parameter :: omp_atv_cgroup = 10
integer(kind=omp_alloctratit_val_kind), &
  parameter :: omp_atv_default_mem_fb = 11
integer(kind=omp_alloctratit_val_kind), &
  parameter :: omp_atv_null_fb = 12
integer(kind=omp_alloctratit_val_kind), &
  parameter :: omp_atv_abort_fb = 13
integer(kind=omp_alloctratit_val_kind), &
  parameter :: omp_atv_allocator_fb = 14
integer(kind=omp_alloctrait_val_kind), &
  parameter :: omp_atv_environment = 15
integer(kind=omp_alloctrait_val_kind), &
  parameter :: omp_atv_nearest = 16
integer(kind=omp_alloctrait_val_kind), &
  parameter :: omp_atv_blocked = 17
integer(kind=omp_alloctrait_val_kind), &
  parameter :: omp_atv_interleaved = 18

type omp_alloctrait
  integer(kind=omp_alloctrait_key_kind) key
  integer(kind=omp_alloctrait_val_kind) value
end type omp_alloctrait

integer(kind=omp_allocator_kind), &
        parameter :: omp_null_allocator = 0
\end{ompfEnum}
\end{fortranspecific}

\subsection{\hcode{omp_init_allocator}}
\index{omp_init_allocator@{\code{omp_init_allocator}}}
\label{subsec:omp_init_allocator}

\summary
The \code{omp_init_allocator} routine initializes an allocator and associates it with a memory space.

\format
\begin{ccppspecific}
\begin{ompcFunction}
omp_allocator_t * omp_init_allocator ( const omp_memspace_t *\plc{memspace}, const int \plc{ntraits}, const omp_alloctrait_t \plc{traits}[])
\end{ompcFunction}
\end{ccppspecific}
\begin{fortranspecific}
\begin{ompfFunction}
integer(kind=omp_allocator_kind) &
function omp_init_allocator ( \plc{memspace}, \plc{ntraits}, \plc{traits} )
integer(kind=omp_memspace_kind),intent(in) :: \plc{memspace}
integer,intent(in) :: \plc{ntraits}
type(omp_alloctrait),intent(in) :: \plc{traits}(*)
\end{ompfFunction}
\end{fortranspecific}

\constraints

The \plc{memspace} argument must be one of the predefined memory spaces defined in Table~\ref{tab:Predefined Memory Spaces}.

If the \plc{ntraits} argument is greater than zero, then there must be at least as many traits
specified in the \plc{traits} argument. If there are fewer than \plc{ntraits} traits the behavior is
unspecified.

Unless a \code{requires} directive with the \code{dynamic_allocators} clause is present in the same compilation unit, using this routine in a \code{target} region results in unspecified behavior.

\binding

The binding thread set for an \code{omp_init_allocator} region is all threads on a device.
The effect of executing this routine is not related to any specific region corresponding to any construct or API routine.

\effect

The \code{omp_init_allocator} routine creates a new allocator that is associated with the \plc{memspace} memory space. 
The allocations done through the created allocator will behave according to the allocator traits specified in the \plc{traits} argument.  The number of traits in the \plc{traits} argument is specified by the \plc{ntraits} argument. Specifying the same allocator trait more than once results in unspecified behavior. The routine returns a handle for the created allocator. If the special \code{OMP_ATV_DEFAULT} value is used for a given trait, then its value will be the default value specified in Table~\ref{tab:Allocator traits} for that given trait.

If \plc{memspace} is \scode{omp_default_mem_space} and the \scode{traits} argument is an empty set this routine will always return a handle to an allocator. Otherwise if an allocator based on the requirements cannot be created then the special value \scode{OMP_NULL_ALLOCATOR} is returned.

The use of an allocator returned by this routine on a device other than the one on which it was created results in unspecified behavior.

\crossreferences
\begin{itemize}
\item Memory spaces in \specref{subsec:Memory Spaces}
\item Allocators in \specref{subsec:Memory Allocators}
\end{itemize}

\subsection{\hcode{omp_destroy_allocator}}
\index{omp_destroy_allocator@{\code{omp_destroy_allocator}}}
\label{subsec:omp_destroy_allocator}

\summary
The \code{omp_destroy_allocator} routine releases all resources used by the allocator handle.

\format
\begin{ccppspecific}
\begin{ompcFunction}
void omp_destroy_allocator ( omp_allocator_t * \plc{allocator});
\end{ompcFunction}
\end{ccppspecific}
\begin{fortranspecific}
\begin{ompfSubroutine}
subroutine omp_destroy_allocator ( \plc{allocator} )
integer(kind=omp_allocator_kind),intent(in) :: \plc{allocator}
\end{ompfSubroutine}
\end{fortranspecific}

\constraints

The \plc{allocator} argument must not be a predefined memory allocator.

Unless a \code{requires} directive with the \code{dynamic_allocators} clause is present in the same compilation unit, using this routine in a \code{target} region results in unspecified behavior.

\binding

The binding thread set for an \code{omp_destroy_allocator} region is all threads on a device.
The effect of executing this routine is not related to any specific region corresponding to any construct or API routine.

\effect

The \code{omp_destroy_allocator} routine releases all resources used to implement the \plc{allocator} handle. Accessing any memory allocated by the \plc{allocator} after this call results in undefined behavior.

If \plc{allocator} is \code{OMP_NULL_ALLOCATOR} then this routine will have no effect.
 
\crossreferences
\begin{itemize}
\item Allocators in \specref{subsec:Memory Allocators}
\end{itemize}


\subsection{\hcode{omp_set_default_allocator}}
\index{omp_set_default_allocator@{\code{omp_set_default_allocator}}}
\label{subsec:omp_set_default_allocator}

\summary
The \code{omp_set_default_allocator} routine sets the default memory allocator to be used by allocation calls, \code{allocate} directives and \code{allocate} clauses that do not specify an allocator.

\format
\begin{ccppspecific}
\begin{ompcFunction}
void omp_set_default_allocator (const omp_allocator_t *\plc{allocator});
\end{ompcFunction}
\end{ccppspecific}
\begin{fortranspecific}
\begin{ompfSubroutine}
subroutine omp_set_default_allocator ( \plc{allocator} )
integer(kind=omp_allocator_kind),intent(in) :: \plc{allocator}
\end{ompfSubroutine}
\end{fortranspecific}

\constraints

The \plc{allocator} argument must point to a valid memory allocator.

\binding
The binding task set for an \code{omp_set_default_allocator} region is the binding implicit task.

\effect

The effect of this routine is to set the value of the \plc{def-allocator-var} ICV of the binding implicit task to the value specified in the \plc{allocator} argument.

\crossreferences

\begin{itemize}
\item \plc{def-allocator-var} ICV, see \specref{sec:Internal Control Variables}.
\item Memory Allocators, see \specref{subsec:Memory Allocators}.
\item \code{omp_alloc} routine, see \specref{subsec:omp_alloc}.
\end{itemize}

\subsection{\hcode{omp_get_default_allocator}}
\index{omp_get_default_allocator@{\code{omp_get_default_allocator}}}
\label{subsec:omp_get_default_allocator}

\summary
The \code{omp_get_default_allocator} routine returns the memory allocator to be used by allocation calls, \code{allocate} directives and \code{allocate} clauses that do not specify an allocator.

\format
\begin{ccppspecific}
\begin{ompcFunction}
const omp_allocator_t * omp_get_default_allocator (void);
\end{ompcFunction}
\end{ccppspecific}
\begin{fortranspecific}
\begin{ompfFunction}
integer(kind=omp_allocator_kind)&
function omp_get_default_allocator ()
\end{ompfFunction}
\end{fortranspecific}

\binding

The binding task set for an \code{omp_get_default_allocator} region is the binding implicit task.

\effect

The effect of this routine is to return the value of the \plc{def-allocator-var} ICV of the binding implicit task.

\crossreferences
\begin{itemize}
\item \plc{def-allocator-var} ICV, see \specref{sec:Internal Control Variables}.
\item Memory Allocators, see \specref{subsec:Memory Allocators}.
\item \code{omp_alloc} routine, see \specref{subsec:omp_alloc}.
\end{itemize}



\vspace{3\baselineskip}
\begin{ccppspecific}
\vspace{-3\baselineskip}
\subsection{\hcode{omp_alloc}}
\index{omp_alloc@{\code{omp_alloc}}}
\label{subsec:omp_alloc}

\summary
The \code{omp_alloc} routine requests a memory allocation from a memory allocator.

\format
\begin{cspecific}
\begin{ompcFunction}
void * omp_alloc (size_t \plc{size}, const omp_allocator_t *\plc{allocator});
\end{ompcFunction}
\end{cspecific}
\begin{cppspecific}
\begin{ompcFunction}
void * omp_alloc (
  size_t \plc{size},
  const omp_allocator_t *\plc{allocator}=OMP_NULL_ALLOCATOR
);
\end{ompcFunction}
\end{cppspecific}

\constraints

For \code{omp_alloc} invocations appearing in \code{target} regions the \plc{allocator} argument cannot be \code{OMP_NULL_ALLOCATOR} and it must be an expression must evaluable by the compiler.

\effect

The \code{omp_alloc} routine requests a memory allocation of \plc{size} bytes from the specified memory allocator. If the \plc{allocator} argument is
\code{OMP_NULL_ALLOCATOR} the memory allocator used by the routine will be the one specified by the \plc{def-allocator-var} ICV of the binding implicit task.
Upon success it returns a pointer to the allocated memory. Otherwise, the behavior specified by the \code{fallback} trait will be followed.

\crossreferences
\begin{itemize}
\item Memory allocators, see \specref{subsec:Memory Allocators}.
\end{itemize}

\subsection{\hcode{omp_free}}
\index{omp_free@{\code{omp_free}}}
\label{subsec:omp_free}

\summary
The \code{omp_free} routine deallocates previously allocated memory.

\format

\begin{cspecific}
\begin{ompcFunction}
void omp_free ( void *\plc{ptr}, const omp_allocator_t *\plc{allocator});
\end{ompcFunction}
\end{cspecific}
\begin{cppspecific}
\begin{ompcFunction}
void omp_free (
  void *\plc{ptr},
  const omp_allocator_t *\plc{allocator}=OMP_NULL_ALLOCATOR
);
\end{ompcFunction}
\end{cppspecific}

\effect

The \code{omp_free} routine deallocates the memory to which \plc{ptr} points. The \plc{ptr} argument must point to memory previously allocated with a memory allocator. If the \plc{allocator} argument is specified it must be the memory allocator to which the allocation request was made. If the \plc{allocator} argument is \code{OMP_NULL_ALLOCATOR} the implementation will find the memory allocator used to allocate the memory. Using \code{omp_free} on memory that was already deallocated or that was allocated by an allocator that has already been destroyed with \code{omp_destroy_allocator} results in unspecified behavior.

\crossreferences
\begin{itemize}
\item Memory allocators, see \specref{subsec:Memory Allocators}.
\end{itemize}

\end{ccppspecific}
