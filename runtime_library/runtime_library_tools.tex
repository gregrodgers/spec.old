\section{Tool Control Routine}
\index{tool control}
\label{sec:control_tool}

\summary
The \code{omp_control_tool} routine enables a program to
pass commands to an active tool.

\format
\begin{ccppspecific}
\begin{ompcFunction}
int omp_control_tool(int \plc{command}, int \plc{modifier}, void *\plc{arg});
\end{ompcFunction}
\end{ccppspecific}

\begin{fortranspecific}
\begin{ompfFunction}
integer function omp_control_tool(\plc{command}, \plc{modifier})
integer (kind=omp_control_tool_kind) \plc{command}
integer \plc{modifier}
\end{ompfFunction}
\end{fortranspecific}

\descr
An OpenMP program may use \code{omp_control_tool} to pass commands to
a tool. An application can use \code{omp_control_tool} to request that
a tool starts or restarts data collection when a code region of interest 
is encountered, that a tool pauses data collection when leaving the region 
of interest, that a tool flushes any data that it has collected so far, or
that a tool ends data collection. Additionally, \code{omp_control_tool} can 
be used to pass tool-specific commands to a particular tool.

The following types correspond to return values from \code{omp_control_tool}:

\begin{ccppspecific}
\begin{ompcEnum}
typedef enum omp_control_tool_result_t {
  omp_control_tool_notool = -2,
  omp_control_tool_nocallback = -1,
  omp_control_tool_success = 0,
  omp_control_tool_ignored = 1
} omp_control_tool_result_t;
\end{ompcEnum}
\end{ccppspecific}

\begin{fortranspecific}
\begin{ompfEnum}
integer (kind=omp_control_tool_result_kind), &
        parameter :: omp_control_tool_notool = -2
integer (kind=omp_control_tool_result_kind), &
        parameter :: omp_control_tool_nocallback = -1
integer (kind=omp_control_tool_result_kind), &
        parameter :: omp_control_tool_success = 0
integer (kind=omp_control_tool_result_kind), &
        parameter :: omp_control_tool_ignored = 1
\end{ompfEnum}
\end{fortranspecific}

If the OMPT interface state is inactive, the OpenMP implementation returns
\code{omp_control_tool_notool}. If the OMPT interface state is active, but
no callback is registered for the \plc{tool-control} event, the OpenMP
implementation returns \code{omp_control_tool_nocallback}. An OpenMP 
implementation may return other implementation-defined negative values 
strictly smaller than -64; an application may assume that any negative 
return value indicates that a tool has not received the command. A return 
value of \code{omp_control_tool_success} indicates that the tool has 
performed the specified command. A return value of 
\code{omp_control_tool_ignored} indicates that the tool has ignored the 
specified command. A tool may return other positive values strictly greater 
than 64 that are tool-defined.

\constraints
The following enumeration type defines four standard commands.
Table~\ref{table:std-tool-cmds} describes the
actions that these commands request from a tool.


\begin{ccppspecific}
\begin{ompcEnum}
typedef enum omp_control_tool_t {
  omp_control_tool_start = 1,
  omp_control_tool_pause = 2,
  omp_control_tool_flush = 3,
  omp_control_tool_end = 4
} omp_control_tool_t;
\end{ompcEnum}
\end{ccppspecific}

\begin{fortranspecific}
\begin{ompfEnum}
integer (kind=omp_control_tool_kind), &
          parameter :: omp_control_tool_start = 1
integer (kind=omp_control_tool_kind), &
          parameter :: omp_control_tool_pause = 2
integer (kind=omp_control_tool_kind), &
          parameter :: omp_control_tool_flush = 3
integer (kind=omp_control_tool_kind), &
          parameter :: omp_control_tool_end = 4
\end{ompfEnum}
\end{fortranspecific}

Tool-specific values for \plc{command} must be greater or equal to 64.
Tools must ignore \plc{command} values that they are not explicitly 
designed to handle. Other values accepted by a tool for \plc{command},
and any values for \plc{modifier} and \plc{arg} are tool-defined.



\nolinenumbers
\renewcommand{\arraystretch}{1.5}
\tablefirsthead{%
\hline
\textsf{\textbf{Command}} & \textsf{\textbf{Action}}\\
\hline\\[-3ex]
}
\tablehead{%
\multicolumn{2}{l}{\small\slshape table continued from previous page}\\
\hline
\textsf{\textbf{Command}} & \textsf{\textbf{Action}}\\
\hline\\[-3ex]
}
\tabletail{%
\hline\\[-4ex]
\multicolumn{2}{l}{\small\slshape table continued on next page}\\
}
\tablelasttail{\hline}
\tablecaption{Standard Tool Control Commands\label{table:std-tool-cmds}}
\begin{supertabular}{p{2in} p{3.0in}}
{\scode{omp_control_tool_start}} & Start or restart monitoring if it is 
                                   off. If monitoring is already on, this
                                   command is idempotent. If monitoring has 
                                   already been turned off permanently, this 
                                   command will have no effect.\\
{\scode{omp_control_tool_pause}} & Temporarily turn monitoring off. If 
                                   monitoring is already off, it is idempotent.\\
{\scode{omp_control_tool_flush}} & Flush any data buffered by a tool.
                                   This command may be applied whether 
                                   monitoring is on or off.\\
{\scode{omp_control_tool_end}}   & Turn monitoring off permanently;
                                   the tool finalizes itself and flushes all output.\\
\end{supertabular}

\linenumbers

\events
The \plc{tool-control} event occurs in the thread that encounters a call
to \code{omp_control_tool} at a point inside its corresponding OpenMP region.

\tools
A thread dispatches a registered \code{ompt_callback_control_tool} callback 
for each occurrence of a \plc{tool-control} event. The callback executes in 
the context of the call that occurs in the user program and has type signature 
\code{ompt_callback_control_tool_t}. The callback may return any non-negative 
value, which will be returned to the application by the OpenMP implementation 
as the return value of the \code{omp_control_tool} call that triggered the callback.

Arguments passed to the callback are those passed by the user to
\code{omp_control_tool}. If the call is made in Fortran, the tool
will be passed \code{NULL} as the third argument to the callback. If
any of the four standard commands is presented to a tool, the tool
will ignore the \plc{modifier} and \plc{arg} argument values.



\begin{crossrefs}
\item OMPT Interface, see
\specchapterref{sec:OMPT Interface}
\item \code{ompt_callback_control_tool_t}, see
\specref{sec:ompt_callback_control_tool_t}.
\end{crossrefs}

