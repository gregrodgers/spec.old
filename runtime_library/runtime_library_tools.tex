% This is an included file. See the master file for more information.
%
% When editing this file:
%
%    1. To change formatting, appearance, or style, please edit openmp.sty.
%
%    2. Custom commands and macros are defined in openmp.sty.
%
%    3. Be kind to other editors -- keep a consistent style by copying-and-pasting to
%       create new content.
%
%    4. We use semantic markup, e.g. (see openmp.sty for a full list):
%         \code{}     % for bold monospace keywords, code, operators, etc.
%         \plc{}      % for italic placeholder names, grammar, etc.
%
%    5. There are environments that provide special formatting, e.g. language bars.
%       Please use them whereever appropriate.  Examples are:
%
%         \begin{fortranspecific}
%         This is text that appears enclosed in blue language bars for Fortran.
%         \end{fortranspecific}
%
%         \begin{note}
%         This is a note.  The "Note -- " header appears automatically.
%         \end{note}
%
%    6. Other recommendations:
%         Use the convenience macros defined in openmp.sty for the minor headers
%         such as Comments, Syntax, etc.
%
%         To keep items together on the same page, prefer the use of
%         \begin{samepage}.... Avoid \parbox for text blocks as it interrupts line numbering.
%         When possible, avoid \filbreak, \pagebreak, \newpage, \clearpage unless that's
%         what you mean. Use \needspace{} cautiously for troublesome paragraphs.
%
%         Avoid absolute lengths and measures in this file; use relative units when possible.
%         Vertical space can be relative to \baselineskip or ex units. Horizontal space
%         can be relative to \linewidth or em units.
%
%         Prefer \emph{} to italicize terminology, e.g.:
%             This is a \emph{definition}, not a placeholder.
%             This is a \plc{var-name}.
%

\section{Tool Control Routines}
\index{tool control}
\label{sec:control_tool}

\summary
The \code{omp_control_tool} routine enables a program to
pass commands to an active tool.

\format


\begin{ccppspecific}
\begin{ompcFunction}
int omp_control_tool(int \plc{command}, int \plc{modifier}, void *\plc{arg});
\end{ompcFunction}
\end{ccppspecific}


\begin{fortranspecific}
\begin{ompfFunction}
integer function omp_control_tool(\plc{command}, \plc{modifier})
integer (kind=omp_control_tool_kind) \plc{command}
integer (kind=omp_control_tool_kind) \plc{modifier}
\end{ompfFunction}
\end{fortranspecific}

\descr

An OpenMP program may use \code{omp_control_tool} to pass commands to
a tool.
Using \code{omp_control_tool}, an application can request that
a tool start or restart data collection when a code
region of interest is encountered, pause data collection when
leaving the region of interest,
flush any data that it has collected so far, or
end data collection.
Additionally, \code{omp_control_tool} can be used to pass
tool-specific commands to a particular tool.



\begin{ccppspecific}
\begin{ompcEnum}
typedef enum omp_control_tool_result_t {
  omp_control_tool_notool = -2,
  omp_control_tool_nocallback = -1,
  omp_control_tool_success = 0,
  omp_control_tool_ignored = 1
} omp_control_tool_result_t;
\end{ompcEnum}
\end{ccppspecific}



\begin{fortranspecific}
\begin{ompfEnum}
integer (kind=omp_control_tool_result_kind), &
        parameter :: omp_control_tool_notool = -2
integer (kind=omp_control_tool_result_kind), &
        parameter :: omp_control_tool_nocallback = -1
integer (kind=omp_control_tool_result_kind), &
        parameter :: omp_control_tool_success = 0
integer (kind=omp_control_tool_result_kind), &
        parameter :: omp_control_tool_ignored = 1
\end{ompfEnum}
\end{fortranspecific}


If no tool is active, the OpenMP implementation will return
\code{omp_control_tool_notool}. If a tool is active, but it has not
registered a callback for the \plc{tool-control} event, the OpenMP
implementation will return \code{omp_control_tool_nocallback}.
An OpenMP implementation may return other implementation-defined negative
values $< -64$; an application may assume that any negative return value
indicates that a tool has not received the command.  A return value of
\code{omp_control_tool_success} indicates that the tool has performed the
specified command.  A return value of \code{omp_control_tool_ignored}
indicates that the tool has ignored the specified command.
A tool may return other positive values $> 64$ that are tool-defined.

\constraints

The following enumeration type defines four standard commands.
Table~\ref{table:std-tool-cmds} describes the
actions that these commands request from a tool.


\begin{ccppspecific}
\begin{ompcEnum}
typedef enum omp_control_tool_t {
  omp_control_tool_start = 1,
  omp_control_tool_pause = 2,
  omp_control_tool_flush = 3,
  omp_control_tool_end = 4
} omp_control_tool_t;
\end{ompcEnum}
\end{ccppspecific}



\begin{fortranspecific}
\begin{ompfEnum}
integer (kind=omp_control_tool_kind), &
          parameter :: omp_control_tool_start = 1
integer (kind=omp_control_tool_kind), &
          parameter :: omp_control_tool_pause = 2
integer (kind=omp_control_tool_kind), &
          parameter :: omp_control_tool_flush = 3
integer (kind=omp_control_tool_kind), &
          parameter :: omp_control_tool_end = 4
\end{ompfEnum}
\end{fortranspecific}


Tool-specific values for \plc{command} must be $\geq$ 64.
Tools must ignore \plc{command} values that they are not
explicitly designed to handle.
Other values accepted by a tool for \plc{command},
and any values for \plc{modifier} and \plc{arg} are tool-defined.



\nolinenumbers
\renewcommand{\arraystretch}{1.5}
\tablefirsthead{%
\hline
\textsf{\textbf{Command}} & \textsf{\textbf{Action}}\\
\hline\\[-3ex]
}
\tablehead{%
\multicolumn{2}{l}{\small\slshape table continued from previous page}\\
\hline
\textsf{\textbf{Command}} & \textsf{\textbf{Action}}\\
\hline\\[-3ex]
}
\tabletail{%
\hline\\[-4ex]
\multicolumn{2}{l}{\small\slshape table continued on next page}\\
}
\tablelasttail{\hline}
\tablecaption{Standard tool control commands.\label{table:std-tool-cmds}}
\begin{supertabular}{p{2in} p{3.0in}}
{\scode{omp_control_tool_start}} & Start or
restart monitoring if it is off. If monitoring is already on, this
command is idempotent. If monitoring has already been turned off
permanently, this command will have no effect.\\
{\scode{omp_control_tool_pause}} & Temporarily
turn monitoring off. If monitoring is already off, it is idempotent.\\
{\scode{omp_control_tool_flush}} & Flush any data buffered by a tool.
This command may be applied whether monitoring is on or off.\\
{\scode{omp_control_tool_end}} & Turn monitoring off permanently;
the tool finalizes itself and flushes all output.\\
\end{supertabular}

\linenumbers

\events

The \plc{tool-control} event occurs in the thread encountering a call
to \code{omp_control_tool} at a point inside its associated OpenMP region.

\tools

A thread dispatches a registered
\code{ompt_callback_control_tool} callback for each occurrence of a
\plc{tool-control} event.  The callback executes in the context of the
call that occurs in the user program.  This callback has type
signature \code{ompt_callback_control_tool_t}.The callback may return
any non-negative value, which will be returned to the application by
the OpenMP implementation as the return value of the
\code{omp_control_tool} call that triggered the callback.


Arguments passed to the callback are those passed by the user to
\code{omp_control_tool}. If the call is made in Fortran, the tool
will be passed a \code{NULL} as the third argument to the callback. If
any of the four standard commands is presented to a tool, the tool
will ignore the \plc{modifier} and \plc{arg} argument values.



\crossreferences
\begin{itemize}
\item Tool Interface, see
\specchapterref{chap:ToolsSupport}
\item \code{ompt_callback_control_tool_t}, see
\specref{sec:ompt_callback_control_tool_t}
\end{itemize}

