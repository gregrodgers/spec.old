%%%%%%%%%%%%%%%%%%%%%%%%%%%%%%%%%%%%%%%%%%%%%%%%%%%%%%%%%%%%%%%%%%%%%%%%%%%%%%%%%%%%%%%%%%%%%%
% Macros for the black and blue lines and arrows delineating language-specific
% and notes sections. Example:
%
%   \fortranspecificstart
%   This is text that applies to Fortran.
%   \fortranspecificend

% local parameters for use \linewitharrows and \notelinewitharrows:
\newlength{\sbsz}\setlength{\sbsz}{0.05in}  % size of arrows
\newlength{\sblw}\setlength{\sblw}{1.35pt}  % line width (thickness)
\newlength{\sbtw}                           % text width
\newlength{\sblen}                          % total width of horizontal rule
\newlength{\sbht}                           % height of arrows
\newlength{\sbhadj}                         % vertical adjustment for aligning arrows with the line
\newlength{\sbns}\setlength{\sbns}{8\baselineskip}       % arg for \needspace for downward arrows

%
% % \notelinewitharrows is a helper command that makes a black Note marker:
% %     arg 1 = 1 or -1 for up or down arrows
% %     arg 2 = solid or dashed or loosely dashed, etc.
% \newcommand{\notelinewitharrows}[2]{%
%   \ifcsname notebox@#1#2\endcsname
%   \else
%     \expandafter\newsavebox\csname notebox@#1#2\endcsname
%     \expandafter\setbox\csname notebox@#1#2\endcsname=\hbox{
%       % \begin{minipage}{\textwidth}
%         \begin{tikzpicture}%
%             \setlength{\sblen}{.99\linewidth}%
%             \setlength{\sbht}{#1\sbsz}\setlength{\sbht}{1.4\sbht}%
%             \setlength{\sbhadj}{#1\sblw}\setlength{\sbhadj}{0.25\sbhadj}%
%             \filldraw (\sblen, 0) -- (\sblen - \sbsz, \sbht) -- (\sblen - 2\sbsz, 0) -- (\sblen, 0);
%             \draw[line width=\sblw, #2] (2\sbsz - \sblw, \sbhadj) -- (\sblen - 2\sbsz + \sblw, \sbhadj);
%             \filldraw (0, 0) -- (\sbsz, \sbht) -- (0 + 2\sbsz, 0) -- (0, 0);
%         \end{tikzpicture}
%       % \end{minipage}
%       }
%   \fi
%   \vskip 0pt % noindent, but works in itemize; \needspace{0.1\baselineskip} might cause a page break
%   \ifdim#1\sbsz=-1\sbsz \VSPb \else \penalty 10000 \fi %
%   \vbox{%
%     \expandafter\usebox\csname notebox@#1#2\endcsname
%   }
%   \ifdim#1\sbsz=1\sbsz \penalty -100 \VSPa \fi %
% }
%
% % \linewitharrows is a helper command that makes a blue horizontal line, up or down arrows, and some text:
% % arg 1 = 1 or -1 for up or down arrows
% % arg 2 = solid or dashed or loosely dashed, etc.
% % arg 3 = text
% \newcommand{\linewitharrows}[3]{%
%   \ifcsname ompbox@#1#2#3\endcsname
%   \else
%     \expandafter\newsavebox\csname ompbox@#1#2#3\endcsname
%   \fi
%     \expandafter\savebox\csname ompbox@#1#2#3\endcsname{
%       \settowidth{\sbtw}{\large \textsf{\textup{xxxx#3}}}
%       % \begin{minipage}{\textwidth}
%         \begin{tikzpicture}[overlay]%
%           \setlength{\sblen}{.99\linewidth}%
%           \setlength{\sbht}{#1\sbsz}\setlength{\sbht}{1.4\sbht}%
%           \setlength{\sbhadj}{#1\sblw}\setlength{\sbhadj}{0.25\sbhadj}%
%           \filldraw[color=blue!40] (\sblen, 0) -- (\sblen - \sbsz, \sbht) -- (\sblen - 2\sbsz, 0) -- (\sblen, 0);%
%           \draw[line width=\sblw, color=blue!40, #2] (2\sbsz - \sblw, \sbhadj) -- (0.5\sblen - 0.5\sbtw, \sbhadj);%
%           \draw[line width=\sblw, color=blue!40, #2] (0.5\sblen + 0.5\sbtw, \sbhadj) -- (\sblen - 2\sbsz + \sblw, \sbhadj);%
%           \filldraw[color=blue!40] (0, 0) -- (\sbsz, \sbht) -- (0 + 2\sbsz, 0) -- (0, 0);%
%           \node[color=blue!90] at (0.5\sblen, 0) {\large \textsf{\textup{#3}}};%
%         \end{tikzpicture}
%       % \end{minipage}
%     }
%   \vskip 0pt % noindent, but works in itemize; \needspace{0.1\baselineskip} might cause a page break
%   \ifdim#1\sbsz=-1\sbsz \VSPb \else \penalty 10000 \fi %
%   \vbox to 12pt {%  %% TODO: we may want to determine the font size automatically
%     \vfill%
%     \usebox{\csname ompbox@#1#2#3\endcsname}
%   }
%   \ifdim#1\sbsz=1\sbsz \penalty -100 \VSPa \fi %
% }


% \notelinewitharrows is a helper command that makes a black Note marker:
%     arg 1 = 1 or -1 for up or down arrows
%     arg 2 = solid or dashed or loosely dashed, etc.
\newcommand{\notelinewitharrows}[2]{%
  \vskip 0pt % noindent, but works in itemize; \needspace{0.1\baselineskip} might cause a page break
  \ifdim#1\sbsz=-1\sbsz \VSPb \else \penalty 10000 \fi %
  \vbox{%
    \begin{tikzpicture}[overlay]%
        \setlength{\sblen}{.99\linewidth}%
        \setlength{\sbht}{#1\sbsz}\setlength{\sbht}{1.4\sbht}%
        \setlength{\sbhadj}{#1\sblw}\setlength{\sbhadj}{0.25\sbhadj}%
        \filldraw (\sblen, 0) -- (\sblen - \sbsz, \sbht) -- (\sblen - 2\sbsz, 0) -- (\sblen, 0);
        \draw[line width=\sblw, #2] (2\sbsz - \sblw, \sbhadj) -- (\sblen - 2\sbsz + \sblw, \sbhadj);
        \filldraw (0, 0) -- (\sbsz, \sbht) -- (0 + 2\sbsz, 0) -- (0, 0);
    \end{tikzpicture}
  }
  \ifdim#1\sbsz=1\sbsz \penalty -100 \VSPa \fi %
}

% \linewitharrows is a helper command that makes a blue horizontal line, up or down arrows, and some text:
% arg 1 = 1 or -1 for up or down arrows
% arg 2 = solid or dashed or loosely dashed, etc.
% arg 3 = text
\newcommand{\linewitharrows}[3]{%
  \settowidth{\sbtw}{\large \textsf{\textup{xxxx#3}}}
  \ifdim#1\sbsz=-1\sbsz \VSPb \else \penalty 10000 \fi %
  \vbox{%  %% TODO: we may want to determine the font size automatically
    \begin{tikzpicture}[overlay]%
        \setlength{\sblen}{.99\linewidth}%
        \setlength{\sbht}{#1\sbsz}\setlength{\sbht}{1.4\sbht}%
        \setlength{\sbhadj}{#1\sblw}\setlength{\sbhadj}{0.25\sbhadj}%
        \filldraw[color=blue!40] (\sblen, 0) -- (\sblen - \sbsz, \sbht) -- (\sblen - 2\sbsz, 0) -- (\sblen, 0);%
        \draw[line width=\sblw, color=blue!40, #2] (2\sbsz - \sblw, \sbhadj) -- (0.5\sblen - 0.5\sbtw, \sbhadj);%
        \draw[line width=\sblw, color=blue!40, #2] (0.5\sblen + 0.5\sbtw, \sbhadj) -- (\sblen - 2\sbsz + \sblw, \sbhadj);%
        \filldraw[color=blue!40] (0, 0) -- (\sbsz, \sbht) -- (0 + 2\sbsz, 0) -- (0, 0);%
        \node[color=blue!90] at (0.5\sblen, 0) {\large \textsf{\textup{#3}}};%
    \end{tikzpicture}
  }
  \ifdim#1\sbsz=1\sbsz \penalty -100 \VSPa \fi %
}



\newcommand{\VSPb}{\vspace{0.25ex plus 5ex minus 0.25ex}}
\newcommand{\VSPa}{\vspace{0.25ex plus 5ex minus 0.25ex}}

% \newcommand{\defompsavebox}[2]{
% \expandafter\newsavebox\csname #1\endcsname
% \expandafter\savebox\csname #1\endcsname{
%   #2
% }
% }
%
% \defompsavebox{csdown}{\linewitharrows{-1}{solid}{C}}
\newsavebox{\csdown}\setbox\csdown=\vbox{{\linewitharrows{-1}{solid}{C}}}
\newsavebox{\csup}\setbox\csup=\vbox{{\linewitharrows{1}{solid}{C}}}
\newsavebox{\ccppsdown}\setbox\ccppsdown=\vbox{{\linewitharrows{-1}{solid}{C / C++}}}
\newsavebox{\ccppsup}\setbox\ccppsup=\vbox{{\linewitharrows{1}{solid}{C / C++}}}
\newsavebox{\cppsdown}\setbox\cppsdown=\vbox{ {\linewitharrows{-1}{solid}{C++}}}
\newsavebox{\cppsup}\setbox\cppsup=\vbox{{\linewitharrows{1}{solid}{C++}}}
\newsavebox{\cninetysdown}\setbox\cninetysdown=\vbox{{\linewitharrows{-1}{solid}{C90}}}
\newsavebox{\cninetysup}\setbox\cninetysup=\vbox{{\linewitharrows{1}{solid}{C90}}}
\newsavebox{\cninetyninesdown}\setbox\cninetyninesdown=\vbox{{\linewitharrows{-1}{solid}{C99}}}
\newsavebox{\cninetyninesup}\setbox\cninetyninesup=\vbox{{\linewitharrows{1}{solid}{C99}}}
\newsavebox{\fortransdown}\setbox\fortransdown=\vbox{{\linewitharrows{-1}{solid}{Fortran}}}
\newsavebox{\fortransup}\setbox\fortransup=\vbox{{\linewitharrows{1}{solid}{Fortran}}}
\newsavebox{\noteboxdown}\setbox\noteboxdown=\vbox{{\notelinewitharrows{-1}{solid}}}
\newsavebox{\noteboxup}\setbox\noteboxup=\vbox{{\notelinewitharrows{1}{solid}}}

\newcommand{\useompbox}[2]{
  \vskip 0pt % noindent, but works in itemize; \needspace{0.1\baselineskip} might cause a page break
  \ifdim#1\sbsz=-1\sbsz \VSPb \else \penalty 10000 \fi %
  \vtop{
    \usebox{#2}
  }
  \ifdim#1\sbsz=1\sbsz \penalty -100 \VSPa \fi %
}

% C
\newcommand{\cspecificstart}{\needspace{\sbns}\useompbox{-1}{\csdown}}
\newcommand{\cspecificend}{\useompbox{1}{\csup}}
\newenvironment{cspecific}{\cspecificstart}{\cspecificend}

% C/C++
\newcommand{\ccppspecificstart}{\needspace{\sbns}\useompbox{-1}{\ccppsdown}}\specificstartmultipage{ccppcont}
\newcommand{\ccppspecificend}{\useompbox{1}{\ccppsup}}\medskip\specificendmultipage{ccppcont}
\newenvironment{ccppspecific}{\ccppspecificstart}{\ccppspecificend}

% C++
\newcommand{\cppspecificstart}{\needspace{\sbns}\useompbox{-1}{\cppsdown}}
\newcommand{\cppspecificend}{\useompbox{1}{\cppsup}}\medskip
\newenvironment{cppspecific}{\cppspecificstart}{\cppspecificend}

% C90
\newcommand{\cNinetyspecificstart}{\needspace{\sbns}\useompbox{-1}{\cninetysdown}}
\newcommand{\cNinetyspecificend}{\useompbox{1}{\cninetysup}}\medskip
\newenvironment{c90specific}{\cNinetyspecificstart}{\cNinetyspecificend}

% C99
\newcommand{\cNinetyNinespecificstart}{\needspace{\sbns}\useompbox{-1}{\cninetyninesdown}}
\newcommand{\cNinetyNinespecificend}{\useompbox{1}{\cninetyninesup}}\medskip
\newenvironment{c99specific}{\cNinetyNinespecificstart}{\cNinetyNinespecificend}

% Fortran
\newcommand{\fortranspecificstart}{\needspace{\sbns}\useompbox{-1}{\fortransdown}\specificstartmultipage{fcont}}
\newcommand{\fortranspecificend}{\useompbox{1}{\fortransup}\medskip\specificendmultipage{fcont}}
\newenvironment{fortranspecific}{\fortranspecificstart}{\fortranspecificend}

% Note
\newcommand{\notestart}{\needspace{\sbns}{\usebox{\noteboxdown}}}
\newcommand{\noteend}{\vspace{12pt}{\usebox{\noteboxup}}\medskip}
\newcommand{\noteheader}{\needspace{1.0\baselineskip}{\textrm{\textsf{\textbf\textup\normalsize{{{{Note }}}}}}}}
\newenvironment{note}{\notestart\noteheader -- }{\noteend}

