% This is tool_support.tex (Chapter 4) of the OpenMP specification.
% This is an included file. See the master file for more information.
%
% When editing this file:
%
%    1. To change formatting, appearance, or style, please edit openmp.sty.
%
%    2. Custom commands and macros are defined in openmp.sty.
%
%    3. Be kind to other editors -- keep a consistent style by copying-and-pasting to
%       create new content.
%
%    4. We use semantic markup, e.g. (see openmp.sty for a full list):
%         \code{}     % for bold monospace keywords, code, operators, etc.
%         \plc{}      % for italic placeholder names, grammar, etc.
%
%    5. There are environments that provide special formatting, e.g. language bars.
%       Please use them whereever appropriate.  Examples are:
%
%         \begin{fortranspecific}
%         This is text that appears enclosed in blue language bars for Fortran.
%         \end{fortranspecific}
%
%         \begin{note}
%         This is a note.  The "Note -- " header appears automatically.
%         \end{note}
%
%    6. Other recommendations:
%         Use the convenience macros defined in openmp.sty for the minor headers
%         such as Comments, Syntax, etc.
%
%         To keep items together on the same page, prefer the use of
%         \begin{samepage}.... Avoid \parbox for text blocks as it interrupts line numbering.
%         When possible, avoid \filbreak, \pagebreak, \newpage, \clearpage unless that's
%         what you mean. Use \needspace{} cautiously for troublesome paragraphs.
%
%         Avoid absolute lengths and measures in this file; use relative units when possible.
%         Vertical space can be relative to \baselineskip or ex units. Horizontal space
%         can be relative to \linewidth or em units.
%
%         Prefer \emph{} to italicize terminology, e.g.:
%             This is a \emph{definition}, not a placeholder.
%             This is a \plc{var-name}.
%

\newcommand{\devicedesc}{
The argument \plc{device} is a pointer to an opaque object that
represents the target device instance. The pointer to the device
instance object is used by functions in the device tracing interface
to identify the device being addressed.
}

\newcommand{\epdesc}{
The argument \plc{endpoint} indicates whether the callback is
signalling the beginning or the end of a scope.
}

\newcommand\codeptrdesc{
The argument \plc{codeptr_ra} is used to relate the implementation
of an OpenMP region back to its source code.  In cases where a runtime
routine implements the region associated with this callback,
\plc{codeptr_ra} is expected to contain the return address of the
call to the runtime routine.  In cases where the implementation of
this feature is inlined, \plc{codeptr_ra} is expected to contain the
return address of the invocation of this callback.  In cases where
attribution to source code is impossible or inappropriate,
\plc{codeptr_ra} may be \code{NULL}.
}


\chapter{Tool Support}
\index{tool support}
\label{chap:ToolsSupport}

%To enable development of high-quality, portable, \emph{first-party} tools
%that support monitoring and performance analysis of OpenMP programs
%developed using any implementation of the OpenMP API, the OpenMP API
%includes a tool interface known as OMPT.

%OMPD-TODO: rewrite to add debugging interface here. Distrinction of 1st/3rd
%party

This chapter describes OMPT and OMPD, which are a pair of interfaces for first-party and third-party
tools, respectively.  Section~\ref{sec:ompt-overview} describes OMPT---an interface for first-party tools.
The section begins with a description of how to initialize (Section~\ref{sec:ompt-initialization})
and finalize (Section~\ref{sec:ompt-finalization}) a tool.
Subsequent sections describe details of the interface, including
data types shared between an OpenMP implementation and a tool
(Section~\ref{sec:ompt-data-types}),
an interface that enables an OpenMP implementation to determine that a
tool is available (Section~\ref{sec:ompt-check-tool}),
type signatures for tool callbacks
that an OpenMP implementation may dispatch for OpenMP events
(Section~\ref{sec:ompt-tool-callbacks}), and
\emph{runtime entry points}---function interfaces
provided by an OpenMP implementation for use by a tool
(Section~\ref{sec:entry-points}).

Section~\ref{sec:ompd-overview} describes
OMPD---an interface for  third-party tools such as debuggers.
Unlike OMPT, a third-party tool exists in a separate process from
the OpenMP program.
An OpenMP implementation need not maintain any extra information to support OMPD inquiries from third-party tools
\emph{unless} it is explicitly instructed to do so.
Section~\ref{subsec:activating} discusses the mechanisms for
activating support for OMPD in the OpenMP runtime.
Section~\ref{subsec:ompd-data-types}  describes the data types shared between the OMPD library and a third-party tool.
Section~\ref{sec:third-party-tool-callback-interface} describes the API provided by the OMPD library for use by a third-party tool.
An OMPD library will not interact directly with the OpenMP
runtime for which it is designed to operate.
Instead, the third-party tool must provide the OMPD library with a set of
callbacks that the OMPD library uses to access the OpenMP runtime.
This interface is given in
Section~\ref{sec:third-party-tool-callback-interface}.
In general, a third-party's tool's OpenMP-related activity will be
conducted through the OMPD interface.
However, there are a few instances where the third-party tool needs
to access the OpenMP runtime directly;
these cases are discussed in
Section~\ref{subsec:runtime-entry-points-for-ompd}.

\section{Tool Interfaces Definitions}
\index{tool interfaces definitions}
\index{tools header files}
\label{sec:tool_interfaces_definitions}

\begin{ccppspecific}

A compliant implementation must supply a set of definitions for the OMPT runtime entry 
points, OMPT callback signatures, OMPD runtime entry points, OMPD tool callback 
signatures, OMPD tool interface routines, and the special data types of their parameters 
and return values.

The set of definitions is provided in a header file named \code{omp-tools.h} and must 
contain a declaration for each of the types defined in 
Sections~\ref{sec:ompt-data-types} - \ref{sec:entry-points} and
\ref{subsec:ompd-data-types} - \ref{subsec:runtime-entry-points-for-ompd}. 

In addition, the set of definitions may specify other implementation specific values.

The \code{ompt_start_tool} function, the \code{ompd_dll_locations} function, all OMPD 
tool interface functions, and all OMPD runtime entry points are external functions with 
``C'' linkage.
	
			
\end{ccppspecific}



% OMPT
% This is an included file. See the master file for more information.
%
% When editing this file:
%
%    1. To change formatting, appearance, or style, please edit openmp.sty.
%
%    2. Custom commands and macros are defined in openmp.sty.
%
%    3. Be kind to other editors -- keep a consistent style by copying-and-pasting to
%       create new content.
%
%    4. We use semantic markup, e.g. (see openmp.sty for a full list):
%         \code{}     % for bold monospace keywords, code, operators, etc.
%         \plc{}      % for italic placeholder names, grammar, etc.
%
%    5. There are environments that provide special formatting, e.g. language bars.
%       Please use them whereever appropriate.  Examples are:
%
%         \begin{fortranspecific}
%         This is text that appears enclosed in blue language bars for Fortran.
%         \end{fortranspecific}
%
%         \begin{note}
%         This is a note.  The "Note -- " header appears automatically.
%         \end{note}
%
%    6. Other recommendations:
%         Use the convenience macros defined in openmp.sty for the minor headers
%         such as Comments, Syntax, etc.
%
%         To keep items together on the same page, prefer the use of
%         \begin{samepage}.... Avoid \parbox for text blocks as it interrupts line numbering.
%         When possible, avoid \filbreak, \pagebreak, \newpage, \clearpage unless that's
%         what you mean. Use \needspace{} cautiously for troublesome paragraphs.
%
%         Avoid absolute lengths and measures in this file; use relative units when possible.
%         Vertical space can be relative to \baselineskip or ex units. Horizontal space
%         can be relative to \linewidth or em units.
%
%         Prefer \emph{} to italicize terminology, e.g.:
%             This is a \emph{definition}, not a placeholder.
%             This is a \plc{var-name}.
%


\section{OMPT}
\label{sec:ompt-overview}

The OMPT interface defines mechanisms for initializing a tool,
assessing implementation-dependent details of an OpenMP implementation
(such as supported states and mutual exclusion implementations),
examining OpenMP state
associated with an OpenMP thread, interpreting an OpenMP thread's call stack,
receiving notification about OpenMP \emph{events}, tracing activity on
OpenMP target devices, and controlling a tool from an OpenMP application.

\subsection{Activating an OMPT Tool}
\label{sec:ompt-initialization}

There are three steps to activate a tool. First, an OpenMP
implementation determines whether a tool should be initialized.  If
so, the OpenMP implementation invokes the tool's initializer, enabling
the tool to prepare to monitor the execution on the host. Finally, a
tool may arrange to monitor computation that execute
on target devices. This section explains how the tool and an
OpenMP implementation interact to accomplish these tasks.

\subsubsection{Determining Whether an OMPT Tool Should be Initialized}
\label{sec:ompt-check-tool}

A tool indicates its interest in using the OMPT interface
by providing a non-null pointer to an
\code{ompt_start_tool_result_t}
structure to an OpenMP implementation as a return value from
\code{ompt_start_tool}. There are three ways
that a tool can provide a definition of \code{ompt_start_tool} to an
OpenMP implementation:

\begin{itemize}
\item statically-linking the tool's definition of \code{ompt_start_tool}
  into an OpenMP application,
\item introducing a dynamically-linked library that includes the tool's definition
  of \code{ompt_start_tool} into the application's address space, or
\item providing the name of a dynamically-linked library appropriate
  for the architecture and operating system used by the application
  in the \plc{tool-libraries-var} ICV.
\end{itemize}

Immediately before an OpenMP implementation initializes itself, it
determines whether it should check for the presence of a tool
interested in using the OMPT interface by examining the \plc{tool-var}
ICV.  If value of \plc{tool-var} is \plc{disabled}, the OpenMP
implementation will initialize itself without checking whether a
tool is present and the functionality of the OMPT interface will be
unavailable as the program executes.

If the value of \plc{tool-var} is \plc{enabled}, the OpenMP
implementation will check if a tool has provided an
implementation of \code{ompt_start_tool}.  The OpenMP implementation first
checks if a tool-provided implementation of \code{ompt_start_tool} is
available in the address space, either statically-linked into the
application or in a dynamically-linked library loaded in the address
space. If multiple implementations of \code{ompt_start_tool} are available,
the OpenMP implementation will use the first tool-provided
implementation of \code{ompt_start_tool} found.

If no tool-provided implementation of \code{ompt_start_tool} is found in
the address space, the OpenMP implementation will consult the
\plc{tool-libraries-var} ICV, which contains a (possibly empty) list
of dynamically-linked libraries.  As described in detail in
Section~\ref{sec:OMP_TOOL_LIBRARIES}, the libraries in
\plc{tool-libraries-var} will be searched for the first usable
implementation of \code{ompt_start_tool} provided by one of the libraries
in the list.

If a tool-provided definition of \code{ompt_start_tool} is found using
either method, the OpenMP implementation will invoke it; if it returns
a non-null pointer to an \code{ompt_start_tool_result_t} structure,
the OpenMP implementation will know
that a tool expects to use the OMPT interface.

Next, the OpenMP implementation will initialize itself. If a tool
provides a non-null pointer to an \code{ompt_start_tool_result_t} structure,
the OpenMP runtime will initialize the OMPT interface for use by a tool.

\crossreferences
\begin{itemize}
\item \plc{tool-libraries-var} and \plc{tool-var} ICV, see \specref{sec:Internal Control 
Variables}.
\item \code{ompt_start_tool_result_t}, see \specref{sec:ompt_start_tool_result_t}.
\item \code{ompt_start_tool}, see \specref{sec:ompt_start_tool}.
\end{itemize}

\subsubsection{Initializing an OMPT Tool}
\index{tool initialization}
\label{sec:tool-initialize}

If a tool-provided implementation of \code{ompt_start_tool} returns a
non-null pointer to an \code{ompt_start_tool_result_t} structure,
the OpenMP implementation will invoke the tool initializer specified
in this structure prior to the occurrence of any OpenMP \emph{event}.

A tool's initializer, described in Section~\ref{sec:ompt_initialize_t},
uses its argument \plc{lookup} to look up pointers
to OMPT interface runtime entry points provided by the OpenMP
implementation; this process is described in Section~\ref{sec:ompt-bind}.
Typically, a tool initializer will
obtain a pointer to the OpenMP runtime entry point known as
\code{ompt_set_callback} with type signature
\code{ompt_set_callback_t} and then use this runtime entry point to
register tool callbacks for OpenMP events, as described in
Section~\ref{sec:ompt-register-callbacks}.

A tool initializer may use the OMPT interface runtime
entry points known as \code{ompt_enumerate_states} and
\code{ompt_enumerate_mutex_impls}, which have the type signatures
\code{ompt_enumerate_states_t} and
\code{ompt_enumerate_mutex_impls_t}, to determine the thread
states and implementations of mutual exclusion that a particular OpenMP
implementation employs.

%The descriptions of the enumeration runtime entry point type signatures show how to use them to determine what thread states and mutual exclusion mechanisms an OpenMP implementation supports.

If a tool initializer returns a non-zero value, the tool will be
\emph{activated} for the execution; otherwise, the tool will be
inactive.

\crossreferences
\begin{itemize}
\item \code{ompt_start_tool_result_t}, see
  \specref{sec:ompt_start_tool_result_t}.
\item \code{ompt_initialize_t}, see \specref{sec:ompt_initialize_t}.
\item \code{ompt_callback_thread_begin_t}, see \specref{sec:ompt_callback_thread_begin_t}.
\item \code{ompt_enumerate_states_t}, see \specref{sec:ompt_enumerate_states_t}.
\item \code{ompt_enumerate_mutex_impls_t}, see   \specref{sec:ompt_enumerate_mutex_impls_t}.
\item \code{ompt_set_callback_t}, see \specref{sec:ompt_set_callback_t}.
\item \code{ompt_function_lookup_t}, see \specref{sec:ompt_function_lookup_t}.
\item \code{ompt_start_tool}, see \specref{sec:ompt_start_tool}.
\end{itemize}


\subsubsubsection{Binding Entry Points in the OMPT Callback Interface}
\label{sec:ompt-bind}

Functions that an OpenMP implementation provides to support the OMPT interface
are not defined as global function symbols. Instead, they are defined as runtime entry points
that a tool can only identify using the \plc{lookup} function provided as an
argument to the tool's initializer. This design avoids tool
implementations that
will fail in certain circumstances where functions defined as part of
the OpenMP runtime are not visible to a tool, even though the tool and
the OpenMP runtime are both present in the same address space.
It also prevents inadvertent use of a tool support routine by
applications.

A tool's initializer receives a function pointer to a \plc{lookup}
runtime entry point with the type signature
\code{ompt_function_lookup_t} as its first argument. Using this
function, a tool initializer may obtain a pointer to each of the
runtime entry points that an OpenMP implementation provides to support
the OMPT interface. Once a tool has obtained a
\plc{lookup} function, it may employ it at any point in the future.

For each runtime entry point in the OMPT interface for the host device,
Table~\ref{table:ompt-callback-interface-functions} provides the string
name by which it is known and its associated type signature. Implementations
can provide additional implementation-specific names and corresponding
entry points.  Any names that begin with \code{ompt_} are reserved names.

During initialization, a tool should look up each runtime entry point in the
OMPT interface by name and bind a pointer maintained by the tool
that can be used at a later time to invoke the entry point as needed. The entry points
described in Table~\ref{table:ompt-callback-interface-functions}
enable a tool to assess
the thread states and mutual exclusion implementations that an OpenMP
implementation supports,
register tool callbacks, inspect callbacks registered,
introspect OpenMP state associated with threads, and use tracing to monitor
computations that execute on target devices.

Detailed information about each runtime entry point listed in
Table~\ref{table:ompt-callback-interface-functions} is included as
part of the description of its type signature.

\crossreferences
\begin{itemize}
\item \code{ompt_enumerate_states_t}, see \specref{sec:ompt_enumerate_states_t}.
\item \code{ompt_enumerate_mutex_impls_t}, see  \specref{sec:ompt_enumerate_mutex_impls_t}.
\item \code{ompt_set_callback_t}, see \specref{sec:ompt_set_callback_t}.
\item \code{ompt_get_callback_t}, see \specref{sec:ompt_get_callback_t}.
\item \code{ompt_get_thread_data_t}, see \specref{sec:ompt_get_thread_data_t}.
\item \code{ompt_get_num_procs_t}, see \specref{sec:ompt_get_num_procs_t}.
\item \code{ompt_get_num_places_t}, see \specref{sec:ompt_get_num_places_t}.
\item \code{ompt_get_place_proc_ids_t}, see \specref{sec:ompt_get_place_proc_ids_t}.
\item \code{ompt_get_place_num_t}, see \specref{sec:ompt_get_place_num_t}.
\item \code{ompt_get_partition_place_nums_t}, see \specref{sec:ompt_get_partition_place_nums_t}.
\item \code{ompt_get_proc_id_t}, see \specref{sec:ompt_get_proc_id_t}.
\item \code{ompt_get_state_t}, see \specref{sec:ompt_get_state_t}.
\item \code{ompt_get_parallel_info_t}, see \specref{sec:ompt_get_parallel_info_t}.
\item \code{ompt_get_task_info_t}, see \specref{sec:ompt_get_task_info_t}.
\item \code{ompt_get_task_memory_t}, see \specref{sec:ompt_get_task_memory_t}.
\item \code{ompt_get_target_info_t}, see \specref{sec:ompt_get_target_info_t}.
\item \code{ompt_get_num_devices_t}, see \specref{sec:ompt_get_num_devices_t}.
\item \code{ompt_get_unique_id_t}, see \specref{sec:ompt_get_unique_id_t}.
\item \code{ompt_finalize_tool_t}, see \specref{sec:ompt_finalize_tool_t}.
\item \code{ompt_function_lookup_t}, see \specref{sec:ompt_function_lookup_t}.
\end{itemize}

\begin{table}[p]
    \caption{OMPT Callback Interface Runtime Entry Point Names and their Type Signatures\label{table:ompt-callback-interface-functions}}
    \begin{tabular}{ll}\hline
        {\small \textbf{\textsf{Entry Point String Name}}} & {\small \textbf{\textsf{Type signature}}}\\\hline
        ``{\scode{ompt_enumerate_states}}'' & {\scode{ompt_enumerate_states_t}}\\
        ``{\scode{ompt_enumerate_mutex_impls}}'' & {\scode{ompt_enumerate_mutex_impls_t}}\\
        ``{\scode{ompt_set_callback}}'' & {\scode{ompt_set_callback_t}}\\
        ``{\scode{ompt_get_callback}}'' & {\scode{ompt_get_callback_t}}\\
        ``{\scode{ompt_get_thread_data}}'' & {\scode{ompt_get_thread_data_t}}\\
        ``{\scode{ompt_get_num_places}}'' & {\scode{ompt_get_num_places_t}}\\
        ``{\scode{ompt_get_place_proc_ids}}'' & {\scode{ompt_get_place_proc_ids_t}}\\
        ``{\scode{ompt_get_place_num}}'' & {\scode{ompt_get_place_num_t}}\\
        ``{\scode{ompt_get_partition_place_nums}}'' & {\scode{ompt_get_partition_place_nums_t}}\\
        ``{\scode{ompt_get_proc_id}}'' & {\scode{ompt_get_proc_id_t}}\\
        ``{\scode{ompt_get_state}}'' & {\scode{ompt_get_state_t}}\\
        ``{\scode{ompt_get_parallel_info}}'' & {\scode{ompt_get_parallel_info_t}}\\
        ``{\scode{ompt_get_task_info}}'' & {\scode{ompt_get_task_info_t}}\\
        ``{\scode{ompt_get_task_memory}}'' & {\scode{ompt_get_task_memory_t}}\\
        ``{\scode{ompt_get_num_devices}}'' & {\scode{ompt_get_num_devices_t}}\\
        ``{\scode{ompt_get_num_procs}}'' & {\scode{ompt_get_num_procs_t}}\\
        ``{\scode{ompt_get_target_info}}'' & {\scode{ompt_get_target_info_t}}\\
        ``{\scode{ompt_get_unique_id}}'' & {\scode{ompt_get_unique_id_t}}\\
        ``{\scode{ompt_finalize_tool}}'' & {\scode{ompt_finalize_tool_t}}\\\hline
        % ``{\scode{ompt_callback_device_initialize}}'' & 
        %{\scode{ompt_callback_device_initialize_t}}\\\hline
    \end{tabular}
    
\end{table}

\subsubsection{Monitoring Activity on the Host with OMPT}
\index{event callback registration}
\label{sec:ompt-register-callbacks}

To monitor the execution of an OpenMP program on the host device, a tool's
initializer must register to receive notification
of events that occur as an OpenMP program is executed.
A tool can register callbacks for OpenMP events using
the runtime entry point known as
\code{ompt_set_callback}.  The possible return codes for
\code{ompt_set_callback} and their meanings are shown in
Table~\ref{table:ToolsSupport_set_rc}.
If the \code{ompt_set_callback} runtime entry point is
called outside a tool's initializer, registration of supported
callbacks may fail with a return code of \code{ompt_set_error}.

All callbacks registered with \code{ompt_set_callback} or returned
by \code{ompt_get_callback} use the dummy type signature
\code{ompt_callback_t}.

Table~\ref{table:valid_rc} indicates the return codes permissible
when trying to register various callbacks. For callbacks where the only registration return code
allowed is \code{ompt_set_always}, an
OpenMP implementation must guarantee that the callback will be
invoked every time a runtime event associated with it occurs. Support
for such callbacks is required in a minimal implementation of the
OMPT interface. For other callbacks where registration is allowed to return values
other than \code{ompt_set_always}, its implementation-defined
whether an OpenMP implementation invokes a registered callback
never, sometimes, or always. If registration for a callback allows
a return code of \code{omp_set_never}, support for invoking such
a callback need not be present in a minimal implementation of the
OMPT interface.  The return code from registering a callback
enables a tool to know what to expect when the level
of support for the callback can be implementation defined.



\begin{table}
\renewcommand{\arraystretch}{1.2}
\caption{Valid Return Codes of \code{ompt_set_callback} for each Callback\label{table:valid_rc}}
\begin{tabular}{lp{3em}p{3em}p{3em}p{3em}}
%                                & \rot{\small{\scode{ompt_set_never}}}
%                                & \rot{\small{\scode{ompt_set_sometimes}}
%                                             {\scode{ompt_set_sometimes_paired}}}
%                                & \rot{\small{\scode{ompt_set_always}}}\\
                                \midrule
Return code abbreviation                      & N &S/P& A \\\hline
{\scode{ompt_callback_thread_begin}}          &   &   & * \\
{\scode{ompt_callback_thread_end}}            &   &   & * \\
{\scode{ompt_callback_parallel_begin}}        &   &   & * \\
{\scode{ompt_callback_parallel_end}}          &   &   & * \\
{\scode{ompt_callback_task_create}}           &   &   & * \\
{\scode{ompt_callback_task_schedule}}         &   &   & * \\
{\scode{ompt_callback_implicit_task}}         &   &   & * \\
{\scode{ompt_callback_target}}                 &   &   & * \\
{\scode{ompt_callback_target_data_op}}       &   &   & * \\
{\scode{ompt_callback_target_submit}}         &   &   & * \\
{\scode{ompt_callback_control_tool}}          &   &   & * \\
{\scode{ompt_callback_device_initialize}}     &   &   & * \\
{\scode{ompt_callback_device_finalize}}       &   &   & * \\
{\scode{ompt_callback_device_load}}           &   &   & * \\
{\scode{ompt_callback_device_unload}}         &   &   & * \\
{\scode{ompt_callback_sync_region_wait}}     & * & * & * \\
{\scode{ompt_callback_mutex_released}}        & * & * & * \\
{\scode{ompt_callback_dependences}}          & * & * & * \\
{\scode{ompt_callback_task_dependence}}       & * & * & * \\
{\scode{ompt_callback_work}}                   & * & * & * \\
{\scode{ompt_callback_master}}                 & * & * & * \\
{\scode{ompt_callback_target_map}}            & * & * & * \\
{\scode{ompt_callback_sync_region}}           & * & * & * \\
{\scode{ompt_callback_reduction}}             & * & * & * \\
{\scode{ompt_callback_lock_init}}             & * & * & * \\
{\scode{ompt_callback_lock_destroy}}          & * & * & * \\
{\scode{ompt_callback_mutex_acquire}}         & * & * & * \\
{\scode{ompt_callback_mutex_acquired}}        & * & * & * \\
{\scode{ompt_callback_nest_lock}}             & * & * & * \\
{\scode{ompt_callback_flush}}                  & * & * & * \\
{\scode{ompt_callback_cancel}}                 & * & * & * \\
{\scode{ompt_callback_dispatch}}              & * & * & * \\
\bottomrule
N = {\scode{ompt_set_never}}                   &  \multicolumn{3}{l}{S = {\scode{ompt_set_sometimes}}} \\
P = {\scode{ompt_set_sometimes_paired}}        &  \multicolumn{3}{l}{A = {\scode{ompt_set_always}}} \\
\end{tabular}

\end{table}

To avoid a tool interface specification that enables a tool to
register unique callbacks for an overwhelming number of events,
the interface was collapsed in several ways.
First, in cases where events are naturally paired, for example, the beginning and
end of a region, and the arguments needed by the callback at each
endpoint were identical, the pair of events was collapsed so that
a tool registers a single callback that will be invoked at both endpoints
with \code{ompt_scope_begin} or \code{ompt_scope_end} provided
as an argument to identify which endpoint the callback invocation reflects.
Second, when a whole class of events is amenable to uniform treatment, only a
single callback is provided for a family of events, for example,  an
\code{ompt_callback_sync_region_wait} callback is used for multiple
kinds of synchronization regions, such as barrier, taskwait, and taskgroup
regions. Some events involve both kinds of collapsing: the aforementioned
\code{ompt_callback_sync_region_wait} represents
a callback that will be invoked at each endpoint for different kinds
of synchronization regions.


\crossreferences
\begin{itemize}
\item \code{ompt_set_callback_t}, see \specref{sec:ompt_set_callback_t}.
\item \code{ompt_get_callback_t}, see \specref{sec:ompt_get_callback_t}.
\end{itemize}




\subsubsection{Tracing Activity on Target Devices with OMPT}
\index{tracing device activity}
\label{sec:tracing-device-activity}

A target device may or may not initialize a full OpenMP runtime system.
Unless it does, it may not be possible to monitor activity
on a device using a tool interface based on callbacks.
To accommodate such cases, the OMPT interface defines
a monitoring interface for tracing activity on target
devices. Tracing activity on a target device involves the following
steps:

\begin{itemize}
\item To prepare to trace activity on a target device, a tool must register for an
  \code{ompt_callback_device_initialize} callback.  A tool may also register for an
  \code{ompt_callback_device_load} callback to be notified when code is loaded onto a target device or
an \code{ompt_callback_device_unload} callback to be notified when code is unloaded from a target device.
A tool may also optionally register an \code{ompt_callback_device_finalize} callback.
\item When an OpenMP implementation initializes a target device, the
  OpenMP implementation will dispatch the tool's device initialization
  callback on the host device. If the OpenMP implementation or target device does not support tracing,
  the OpenMP implementation will pass a \code{NULL} to the tool's device initializer for its
  \plc{lookup} argument; otherwise, the OpenMP implementation will pass
  a pointer to a device-specific runtime entry point with type
  signature \code{ompt_function_lookup_t} to the tool's device initializer.
\item If the device initializer for the tool receives a
  non-null \plc{lookup} pointer, the tool may use it to query
  which runtime entry points in the tracing interface are available for a target device
  and bind the function pointers returned to tool variables.
  Table~\ref{table:ompt-tracing-interface-functions} indicates the
  names of the runtime entry points that a target device may provide for use
  by a tool.
  Implementations
may provide additional implementation-specific names and corresponding
entry points as long as they don't use names that start with the prefix
\code{ompt_} which is reserved for the OpenMP specification.

  If \plc{lookup} is non-null, the driver for a device will
  provide runtime entry points that enable a tool to control the device's
  interface for collecting traces in its \emph{native} trace format,
  which may be device specific.
  The kinds of trace records available for a device are
  implementation-defined.
  Some devices may allow a tool to
  collect traces of records in a standard format known as OMPT trace
  records. Each OMPT trace record serves as a substitute for an OMPT
  callback that cannot be made on a device. The fields in each trace
  record type are defined in the description of the callback that the
  record represents.  If so, the \plc{lookup} function will
  return values for the runtime entry points
  \code{ompt_set_trace_ompt} and \code{ompt_get_record_ompt}, which support
  collecting and decoding OMPT traces.
  These runtime entry points are not required for all devices and will only be available for target devices that support
  collection of standard traces in OMPT format.
  For some devices, their native
  tracing format may be OMPT format. In that case, tracing can be
  controlled using either the runtime entry points for native or OMPT
  tracing.

\begin{table}
{\small
\caption{OMPT Tracing Interface Runtime Entry Point Names and their Type Signatures\label{table:ompt-tracing-interface-functions}}
\begin{tabular}{ll}\hline
\textbf{\textsf{Entry Point String Name}} & \textbf{\textsf{Type Signature}}\\\hline
``{\scode{ompt_get_device_num_procs}}'' & {\scode{ompt_get_device_num_procs_t}}\\
``{\scode{ompt_get_device_time}}'' & {\scode{ompt_get_device_time_t}}\\
``{\scode{ompt_translate_time}}'' & {\scode{ompt_translate_time_t}}\\
``{\scode{ompt_set_trace_ompt}}'' & {\scode{ompt_set_trace_ompt_t}}\\
``{\scode{ompt_set_trace_native}}'' & {\scode{ompt_set_trace_native_t}}\\
``{\scode{ompt_start_trace}}'' & {\scode{ompt_start_trace_t}}\\
``{\scode{ompt_pause_trace}}'' & {\scode{ompt_pause_trace_t}}\\
``{\scode{ompt_flush_trace}}'' & {\scode{ompt_flush_trace_t}}\\
``{\scode{ompt_stop_trace}}'' & {\scode{ompt_stop_trace_t}}\\
``{\scode{ompt_advance_buffer_cursor}}'' & {\scode{ompt_advance_buffer_cursor_t}}\\
``{\scode{ompt_get_record_type}}'' & {\scode{ompt_get_record_type_t}}\\
``{\scode{ompt_get_record_ompt}}'' & {\scode{ompt_get_record_ompt_t}}\\
``{\scode{ompt_get_record_native}}'' & {\scode{ompt_get_record_native_t}}\\
``{\scode{ompt_get_record_abstract}}'' & {\scode{ompt_get_record_abstract_t}}\\\hline
\end{tabular}
}

\end{table}


\item The tool will use the \code{ompt_set_trace_native}
  and/or the \code{ompt_set_trace_ompt} runtime entry point to specify what
  types of events or activities to monitor on the target device.
  If the \code{ompt_set_trace_native} and/or the
  \code{ompt_set_trace_ompt} runtime entry point is called outside a device
  initializer, registration of supported callbacks may fail with a return code of
  \code{ompt_set_error}.
\item The tool will initiate tracing on the target device by
  invoking \code{ompt_start_trace}. Arguments to \code{ompt_start_trace}
  include two tool callbacks for use by the OpenMP implementation to manage
  traces associated with the target device: one to allocate
  a buffer where the target device can deposit trace events and a
  second to process a buffer of trace events from the target device.
\item When the target device needs a trace buffer, the OpenMP implementation
  will invoke the tool-supplied callback function on the host device to request a new buffer.
\item The OpenMP implementation will monitor the execution of OpenMP constructs on the target device as
  directed and record a trace of events or activities into a trace
  buffer. If the device is capable, device trace records will be
  marked with a \plc{host_op_id}---an identifier used to associate
  device activities with the target operation initiated on the host
  that caused these activities.  To correlate activities on the host
  with activities on a device, a tool can register a
  \code{ompt_callback_target_submit} callback.
  Before the host initiates each distinct activity associated with a structured block for a \code{target} construct
  on a target device, the OpenMP implementation will dispatch the \code{ompt_callback_target_submit} callback
  on the host in the thread executing the task that encounters the \code{target} construct.
  Examples of activities that could cause an \code{ompt_callback_target_submit} callback to be dispatched
  include an explicit data copy between a host and target device or execution of a computation.
  This callback provides the tool with a pair of identifiers: one that identifies the target region and a second
  that uniquely identifies an activity associated with that region.
  These identifiers help the tool correlate activities on the target device with their target region.
\item When appropriate, for example, when a trace buffer fills or needs to be
  flushed, the OpenMP implementation will invoke the tool-supplied buffer
  completion callback to process a non-empty sequence of
  records in a trace buffer associated with the target device.

\item The tool-supplied buffer completion callback may return
  immediately, ignoring records in the trace buffer, or it may iterate
  through them using the \code{ompt_advance_buffer_cursor} entry
  point
  and inspect each one. A tool may inspect the type of the record at
  the current cursor position using the \code{ompt_get_record_type}
  runtime entry point.  A tool may choose to inspect the contents of some or
  all records in a trace buffer using the \code{ompt_get_record_ompt},
  \code{ompt_get_record_native}, or
  \code{ompt_get_record_abstract} runtime entry point.  Presumably, a tool that
  chooses to use the \code{ompt_get_record_native} runtime entry point to
  inspect records will have some knowledge about a device's native
  trace format.  A tool may always use the
  \code{ompt_get_record_abstract} runtime entry point to inspect a trace
  record; this runtime entry point will decode the contents of a native trace record
  and summarize them in a standard format, namely an
  \code{ompt_record_abstract_t} record.
  Only a record in OMPT format can be retrieved using the
  \code{ompt_get_record_ompt} runtime entry point.
\item Once tracing has been started on a device, a tool may pause or resume
  tracing on the device at any time by invoking
  \code{ompt_pause_trace} with an appropriate flag value as an
  argument.
\item A tool may request that a device flush any pending trace records
  at any time between device initialization and finalization
  by invoking the \code{ompt_flush_trace} runtime entry point for the device.
\item A tool may start or stop tracing on a device at any time using the
  \code{ompt_start_trace} or \code{ompt_stop_trace} runtime entry points,
  respectively. When tracing is stopped on a device, the OpenMP implementation will eventually
  gather all trace records already collected on the device and present them to the tool using
  the buffer completion callback provided by the tool.
\item It is legal to shut down an OpenMP implementation while device tracing
is in progress.
\item When an OpenMP implementation begins to shut down, it will
  finalize each target device.  Device finalization occurs in three steps.
  First, the OpenMP implementation halts any tracing in progress for the device. Second,
  the OpenMP implementation flushes all trace records collected for the device and presents them to
  the tool using the buffer completion callback associated with that device.
  Finally, the OpenMP implementation dispatches
  any \code{ompt_callback_device_finalize} callback that was previously
  registered by the tool.

\end{itemize}


\crossreferences
\begin{itemize}
\item \code{ompt_callback_device_initialize_t}, see \specref{sec:ompt_callback_device_initialize_t}.
\item \code{ompt_callback_device_finalize_t}, see \specref{sec:ompt_callback_device_finalize_t}.
\item \code{ompt_get_device_num_procs_t}, see \specref{sec:ompt_get_device_num_procs_t}.
\item \code{ompt_get_device_time}, see \specref{sec:ompt_get_device_time_t}.
\item \code{ompt_translate_time}, see \specref{sec:ompt_translate_time_t}.
\item\code{ompt_set_trace_ompt}, see \specref{sec:ompt_set_trace_ompt_t}.
\item \code{ompt_set_trace_native}, see \specref{sec:ompt_set_trace_native_t}.
\item \code{ompt_start_trace}, see \specref{sec:ompt_start_trace_t}.
\item \code{ompt_pause_trace}, see \specref{sec:ompt_pause_trace_t}.
\item \code{ompt_flush_trace}, see \specref{sec:ompt_flush_trace_t}.
\item \code{ompt_stop_trace}, see \specref{sec:ompt_stop_trace_t}.
\item \code{ompt_advance_buffer_cursor}, see \specref{sec:ompt_advance_buffer_cursor_t}.
\item \code{ompt_get_record_type}, see \specref{sec:ompt_get_record_type_t}.
\item \code{ompt_get_record_ompt}, see \specref{sec:ompt_get_record_ompt_t}.
\item \code{ompt_get_record_native}, see \specref{sec:ompt_get_record_native_t}.
\item \code{ompt_get_record_abstract}, see \specref{sec:ompt_get_record_abstract_t}.
\end{itemize}


\subsection{Finalizing an OMPT Tool}
\label{sec:ompt-finalization}

If \code{ompt_start_tool} returned a non-null pointer when an OpenMP
implementation was initialized, the tool finalizer, of type signature
\code{ompt_finalize_t}, specified by the
\plc{finalize} field in this structure will be called as the OpenMP
implementation shuts down.

\crossreferences
\begin{itemize}
\item \code{ompt_finalize_t}, \specref{sec:ompt_finalize_t}
\end{itemize}

\subsection{OMPT Data Types}
\label{sec:ompt-data-types}

\subsubsection{Tool Initialization and Finalization}
\label{sec:ompt_start_tool_result_t}

\summary
A tool's implementation of \code{ompt_start_tool} returns a pointer to an
\code{ompt_start_tool_result_t} structure, which
contains pointers to the tool's
initialization and finalization callbacks as well as an
\code{ompt_data_t} object for use by the tool.


\begin{ccppspecific}
\begin{omptOther}
typedef struct ompt_start_tool_result_t {
  ompt_initialize_t \plc{initialize};
  ompt_finalize_t \plc{finalize};
  ompt_data_t \plc{tool_data};
} ompt_start_tool_result_t;
\end{omptOther}
\end{ccppspecific}


\restrictions

The \plc{initialize} and \plc{finalize} callback pointer values in an
\code{ompt_start_tool_result_t} structure returned by \code{ompt_start_tool} must be
non-null.

\crossreferences
\begin{itemize}
\item \code{ompt_data_t}, see \specref{sec:ompt_data_t}.
\item \code{ompt_initialize_t}, see \specref{sec:ompt_initialize_t}.
\item \code{ompt_finalize_t}, see \specref{sec:ompt_finalize_t}.
\item \code{ompt_start_tool}, see \specref{sec:ompt_start_tool}.
\end{itemize}

\subsubsection{Callbacks}
\label{sec:ompt_callbacks_t}

The following enumeration type indicates the integer codes used to identify
OpenMP callbacks when registering or querying them.


\begin{ccppspecific}
\begin{omptEnum}
typedef enum ompt_callbacks_t {
  ompt_callback_thread_begin             = 1,
  ompt_callback_thread_end               = 2,
  ompt_callback_parallel_begin           = 3,
  ompt_callback_parallel_end             = 4,
  ompt_callback_task_create              = 5,
  ompt_callback_task_schedule            = 6,
  ompt_callback_implicit_task            = 7,
  ompt_callback_target                   = 8,
  ompt_callback_target_data_op           = 9,
  ompt_callback_target_submit            = 10,
  ompt_callback_control_tool             = 11,
  ompt_callback_device_initialize        = 12,
  ompt_callback_device_finalize          = 13,
  ompt_callback_device_load              = 14,
  ompt_callback_device_unload            = 15,
  ompt_callback_sync_region_wait         = 16,
  ompt_callback_mutex_released           = 17,
  ompt_callback_dependences              = 18,
  ompt_callback_task_dependence          = 19,
  ompt_callback_work                     = 20,
  ompt_callback_master                   = 21,
  ompt_callback_target_map               = 22,
  ompt_callback_sync_region              = 23,
  ompt_callback_lock_init                = 24,
  ompt_callback_lock_destroy             = 25,
  ompt_callback_mutex_acquire            = 26,
  ompt_callback_mutex_acquired           = 27,
  ompt_callback_nest_lock                = 28,
  ompt_callback_flush                    = 29,
  ompt_callback_cancel                   = 30,
  ompt_callback_reduction                = 31,
  ompt_callback_dispatch                 = 32
} ompt_callbacks_t;
\end{omptEnum}
\end{ccppspecific}



%\subsubsubsection{Triggers for Miscellaneous Events}
%Most events trigger during the execution of OpenMP directives. Other
%events trigger when an application calls certain runtime library
%routines, e.g., those for setting and unsetting locks.
%This section describes events triggered during initialization and
%finialization of an OpenMP implementation.
%
%\ompteventswithoutdirectives{\code{ompt_callback_thread_begin}}
%\label{sec:ompt_callback_thread_begin}
%
%An OpenMP implementation invokes this callback in the context of an
%initial thread just after it initializes the runtime, or in the
%context of a new thread created by the runtime just after the thread
%initializes itself. In either case, this callback must be the first
%callback for a thread and must occur before the thread executes any
%OpenMP tasks. This callback has type signature
%\code{ompt_callback_thread_begin_t}.  This callback argument
%\code{thread_type} indicates the type of the thread: initial, worker,
%or other.
%
%\ompteventswithoutdirectives{\code{ompt_callback_thread_end}}
%\label{sec:ompt_callback_thread_end}
%
%An OpenMP implementation invokes this callback after an OpenMP thread
%completes all of its tasks but before the thread is destroyed. The
%callback executes in the context of the OpenMP thread. This callback
%must be the last callback event for any worker thread; it is optional
%for other types of threads.  This callback has type signature
%\code{ompt_callback_thread_end_t}.
%
%
%\ompteventswithoutdirectives{\code{ompt_callback_sync_region_wait}}
%\label{sec:ompt_callback_sync_region_wait}
%
%If the \code{ompt_callback_sync_region_wait} callback is registered,
%an OpenMP implementation will invoke this callback when a task starts
%and stops waiting in a barrier region, taskwait region, or taskgroup
%region.  This callback has type signature
%\code{ompt_callback_sync_region_t}.  One region may generate
%multiple pairs of start/stop callbacks if another task is scheduled on
%the thread while the task awaiting completion of the region is
%stalled.  This callback executes in the context of the task that
%encountered the barrier, taskwait, or taskgroup construct.
%
%\ompteventswithoutdirectives{\code{ompt_callback_runtime_shutdown}}
%\label{sec:ompt_callback_runtime_shutdown}
%
%An OpenMP implementation invokes this callback before it shuts down
%the runtime system.  This callback enables a tool to clean up its
%state and record or report information gathered. A runtime may later
%restart and reinitialize the tool by calling the tool initializer
%function (described in Section~\ref{sec:tool-initialize}) again.  This
%callback has type signature \code{ompt_callback_t}.


\subsubsection{Tracing}
\label{sec:ompt-records}

\subsubsubsection{Record Type}

\begin{ccppspecific}
\begin{omptEnum}
typedef enum ompt_record_t {
  ompt_record_ompt               = 1,
  ompt_record_native             = 2,
  ompt_record_invalid            = 3
} ompt_record_t;
\end{omptEnum}
\end{ccppspecific}


\subsubsubsection{Native Record Kind}
\label{sec:ompt_record_native_t}


\begin{ccppspecific}
\begin{omptEnum}
typedef enum ompt_record_native_t {
  ompt_record_native_info  = 1,
  ompt_record_native_event = 2
} ompt_record_native_t;
\end{omptEnum}
\end{ccppspecific}


\subsubsubsection{Native Record Abstract Type}
\label{sec:ompt_record_abstract_t}


\begin{ccppspecific}
\begin{omptRecord}
typedef struct ompt_record_abstract_t {
  ompt_record_native_t \plc{rclass};
  const char *\plc{type};
  ompt_device_time_t \plc{start_time};
  ompt_device_time_t \plc{end_time};
  ompt_hwid_t \plc{hwid};
} ompt_record_abstract_t;
\end{omptRecord}
\end{ccppspecific}


\descr

An \code{ompt_record_abstract_t} record contains several
pieces of information that a tool can use to process a native record
that it may not fully understand. The \plc{rclass} field
indicates whether the record is informational
or represents an event; knowing this can help a tool determine
how to present the record. The record \plc{type} field points to
a statically-allocated, immutable character string that provides
a meaningful name that a tool might want to use to describe the event
to a user. The \plc{start_time} and \plc{end_time} fields are
used to place an event in time. The times are relative to the device
clock. If an event has no associated \plc{start_time} and/or
\plc{end_time}, its value will be
\code{ompt_time_none}. The hardware identifier field,
\plc{hwid},  is used to indicate the location on the device where
the event occurred. A \plc{hwid} may represent a hardware abstraction
such as a core or a hardware thread identifier. The meaning of a \plc{hwid}
value for a device is defined by the implementer of the software
stack for the device. If there is no hardware abstraction associated
with the record, the value of \plc{hwid}
will be \code{ompt_hwid_none}.

\subsubsubsection{Record Type}
\label{sec:ompt_record_ompt_t}

\begin{ccppspecific}
\begin{omptRecord}
typedef struct ompt_record_ompt_t {
  ompt_callbacks_t \plc{type};
  ompt_device_time_t \plc{time};
  ompt_id_t \plc{thread_id};
  ompt_id_t \plc{target_id};
  union {
    ompt_record_thread_begin_t \plc{thread_begin};
    ompt_record_idle_t \plc{idle};
    ompt_record_parallel_begin_t \plc{parallel_begin};
    ompt_record_parallel_end_t \plc{parallel_end};
    ompt_record_task_create_t \plc{task_create};
    ompt_record_dependences_t \plc{deps};
    ompt_record_task_dependence_t \plc{task_dep};
    ompt_record_task_schedule_t \plc{task_sched};
    ompt_record_implicit_t \plc{implicit};
    ompt_record_sync_region_t \plc{sync_region};
    ompt_record_target_t \plc{target_record};
    ompt_record_target_data_op_t \plc{target_data_op};
    ompt_record_target_map_t \plc{target_map};
    ompt_record_target_kernel_t \plc{kernel};
    ompt_record_lock_init_t \plc{lock_init};
    ompt_record_lock_destroy_t \plc{lock_destroy};
    ompt_record_mutex_acquire_t \plc{mutex_acquire};
    ompt_record_mutex_t \plc{mutex};
    ompt_record_nest_lock_t \plc{nest_lock};
    ompt_record_master_t \plc{master};
    ompt_record_work_t \plc{work};
    ompt_record_flush_t \plc{flush};
  } \plc{record};
} ompt_record_ompt_t;
\end{omptRecord}
\end{ccppspecific}


\descr
The field \plc{type} specifies the type of record provided by this
structure.
According to the type, event specific information is stored in the matching
\plc{record} entry.
\restrictions
If the \plc{type} is set to \code{ompt_callback_thread_end_t}, the value
of \plc{record} is undefined.

%\begin{note}
%There is no trace record defined for \code{ompt_callback_thread_end_t}, because the thread id is already stored in \code{ompt_record_ompt_t} and the callback signature does not have any further parameters.
%\end{note}

\subsubsection{Miscellaneous Type Definitions}
\label{sec:ompt-types:misc}
This section describes miscellaneous types and enumerations used by the tool interface.

\subsubsubsection{\hcode{ompt_callback_t}}
\label{sec:ompt_callback_t}

Pointers to tool callback functions with different type
signatures are passed to the \code{ompt_set_callback} runtime entry point and
returned by the \code{ompt_get_callback} runtime entry point. For convenience,
these runtime entry points expect all type signatures to be cast to
a dummy type \code{ompt_callback_t}.


\begin{ccppspecific}
\begin{omptCallback}
typedef void (*ompt_callback_t) (void);
\end{omptCallback}
\end{ccppspecific}


% ompt_id_t

\subsubsubsection{\hcode{ompt_id_t}}
\label{sec:ompt_id_t}
When tracing asynchronous activity on devices, tools need identifiers to correlate target regions and operations initiated by the host with
associated activities on a target device. In addition, tools need identifiers to refer to parallel regions and tasks that execute on a device.
OpenMP implementations use identifiers of type \code{ompt_id_t} type for each of these 
purposes.

\begin{ccppspecific}
\begin{omptOther}
typedef uint64_t ompt_id_t;
\end{omptOther}
\end{ccppspecific}

\code{ompt_id_none} is defined as an instance of type \code{ompt_id_t} with the value 0.

Identifiers created on each device must be unique from the time an OpenMP implementation is initialized until it is shut down.
Specifically, this means that (1) identifiers for each target region and target operation instance initiated by the host device must be unique over time on the host,
and (2) identifiers for parallel and task region instances that execute on a device must be unique over time within that device.

Tools should not assume that \code{ompt_id_t} values are small or densely allocated.

% ompt_data_t

\subsubsubsection{\hcode{ompt_data_t}}
\label{sec:ompt_data_t}
Threads, parallel regions, and task regions
each have an associated data object of type \code{ompt_data_t} reserved for use by a tool.
When an OpenMP implementation creates a thread or an instance of a parallel or task region,
it will initialize its associated \code{ompt_data_t} object with the value \code{ompt_data_none}.


\begin{ccppspecific}
\begin{omptOther}
typedef union ompt_data_t {
  uint64_t \plc{value};
  void *\plc{ptr};
} ompt_data_t;
\end{omptOther}
\end{ccppspecific}

\code{ompt_data_none} is defined as an instance of type \code{ompt_data_t} with the data 
and pointer fields equal to 0.

% ompt_device_t

\subsubsubsection{\hcode{ompt_device_t}}
\label{sec:ompt_device_t}
\code{ompt_device_t} is an opaque object representing a device.


\begin{ccppspecific}
\begin{omptOther}
typedef void ompt_device_t;
\end{omptOther}
\end{ccppspecific}




% ompt_device_time_t

\subsubsubsection{\hcode{ompt_device_time_t}}
\label{sec:ompt_device_time_t}
\code{ompt_device_time_t} is an opaque object representing a raw time value from a device.
\label{sec:ompt_time_none}
\code{ompt_time_none} refers to an uknown or unspecified time.


\begin{ccppspecific}
\begin{omptOther}
typedef uint64_t ompt_device_time_t;
\end{omptOther}
\end{ccppspecific}

\code{ompt_time_none} is defined as an instance of type \code{ompt_device_time_t} with 
the value 0.


% ompt_buffer_t

\subsubsubsection{\hcode{ompt_buffer_t}}
\label{sec:ompt_buffer_t}
\code{ompt_buffer_t} is an opaque object handle for a target buffer.


\begin{ccppspecific}
\begin{omptOther}
typedef void ompt_buffer_t;
\end{omptOther}
\end{ccppspecific}




% ompt_buffer_cursor_t

\subsubsubsection{\hcode{ompt_buffer_cursor_t}}
\label{sec:ompt_buffer_cursor_t}
\code{ompt_buffer_cursor_t} is an opaque handle for a position in a target buffer.


\begin{ccppspecific}
\begin{omptOther}
typedef uint64_t ompt_buffer_cursor_t;
\end{omptOther}
\end{ccppspecific}




% ompt_dependence_t

\subsubsubsection{\hcode{ompt_dependence_t}}
\label{sec:ompt_dependence_t}
\code{ompt_dependence_t} is a task dependence.


\begin{ccppspecific}
\begin{omptOther}
typedef struct ompt_dependence_t {
  ompt_data_t \plc{variable};
  ompt_dependence_type_t \plc{dependence_type};
} ompt_dependence_t;
\end{omptOther}
\end{ccppspecific}


\descr
\code{ompt_dependence_t} is a structure to hold information
about a depend clause.
For task dependences, the element \plc{variable} points to the storage location of the 
dependence. 
For doacross dependences, the element \plc{variable} contains the value of a vector 
element, describing 
the dependence.
The element \plc{dependence_type} indicates the type of dependence described.

\crossreferences
\begin{itemize}
\item \code{ompt_dependence_type_t}, see
\specref{sec:ompt_dependence_type_t}.
\end{itemize}


% ompt_thread_t

\subsubsubsection{\hcode{ompt_thread_t}}
\label{sec:ompt_thread_t}
\code{ompt_thread_t} is an enumeration that defines the valid thread type values.


\begin{ccppspecific}
\begin{omptEnum}
typedef enum ompt_thread_t {
  ompt_thread_initial                 = 1,
  ompt_thread_worker                  = 2,
  ompt_thread_other                   = 3,
  ompt_thread_unknown                 = 4
} ompt_thread_t;
\end{omptEnum}
\end{ccppspecific}


Any \plc{initial thread} has thread type \code{ompt_thread_initial}.
All \plc{OpenMP threads} that are not initial threads have thread
type \code{ompt_thread_worker}.  A thread employed by an OpenMP
implementation that does not execute user code has thread type
\code{ompt_thread_other}.  Any thread created outside an OpenMP
implementation that is not an \plc{initial thread} has thread type
\code{ompt_thread_unknown}.

% ompt_scope_endpoint_t

\subsubsubsection{\hcode{ompt_scope_endpoint_t}}
\label{sec:ompt_scope_endpoint_t}
\code{ompt_scope_endpoint_t} is an enumeration that defines valid scope endpoint values.


\begin{ccppspecific}
\begin{omptEnum}
typedef enum ompt_scope_endpoint_t {
  ompt_scope_begin                    = 1,
  ompt_scope_end                      = 2
} ompt_scope_endpoint_t;
\end{omptEnum}
\end{ccppspecific}



% ompt_dispatch_t

\subsubsubsection{\hcode{ompt_dispatch_t}}
\label{sec:ompt_dispatch_t}
\code{ompt_dispatch_t} is an enumeration that defines the valid dispatch kind values.


\begin{ccppspecific}
\begin{omptEnum}
typedef enum ompt_dispatch_t {
  ompt_dispatch_iteration             = 1,
  ompt_dispatch_section               = 2
} ompt_dispatch_t;
\end{omptEnum}
\end{ccppspecific}



% ompt_sync_region_t

\subsubsubsection{\hcode{ompt_sync_region_t}}
\label{sec:ompt_sync_region_t}
\code{ompt_sync_region_t} is an enumeration that defines the valid synchronization region kind values.


\begin{ccppspecific}
\begin{omptEnum}
typedef enum ompt_sync_region_t {
  ompt_sync_region_barrier                = 1,
  ompt_sync_region_barrier_implicit       = 2,
  ompt_sync_region_barrier_explicit       = 3,
  ompt_sync_region_barrier_implementation = 4,
  ompt_sync_region_taskwait               = 5,
  ompt_sync_region_taskgroup              = 6,
  ompt_sync_region_reduction              = 7
} ompt_sync_region_t;
\end{omptEnum}
\end{ccppspecific}




% ompt_target_data_op_t

\subsubsubsection{\hcode{ompt_target_data_op_t}}
\label{sec:ompt_target_data_op_t}
\code{ompt_target_data_op_t} is an enumeration that defines the valid target data operation values.


\begin{ccppspecific}
\begin{omptEnum}
typedef enum ompt_target_data_op_t {
  ompt_target_data_alloc                = 1,
  ompt_target_data_transfer_to_device   = 2,
  ompt_target_data_transfer_from_device = 3,
  ompt_target_data_delete               = 4,
  ompt_target_data_associate            = 5,
  ompt_target_data_disassociate         = 6
} ompt_target_data_op_t;
\end{omptEnum}
\end{ccppspecific}




% ompt_work_t

\subsubsubsection{\hcode{ompt_work_t}}
\label{sec:ompt_work_t}
\code{ompt_work_t} is an enumeration that defines the valid work type values.


\begin{ccppspecific}
\begin{omptEnum}
typedef enum ompt_work_t {
  ompt_work_loop               = 1,
  ompt_work_sections           = 2,
  ompt_work_single_executor    = 3,
  ompt_work_single_other       = 4,
  ompt_work_workshare          = 5,
  ompt_work_distribute         = 6,
  ompt_work_taskloop           = 7
} ompt_work_t;
\end{omptEnum}
\end{ccppspecific}




% ompt_mutex_t

\subsubsubsection{\hcode{ompt_mutex_t}}
\label{sec:ompt_mutex_t}
\code{ompt_mutex_t} is an enumeration that defines the valid mutex kind values.


\begin{ccppspecific}
\begin{omptEnum}
typedef enum ompt_mutex_t {
  ompt_mutex_lock                     = 1,
  ompt_mutex_nest_lock                = 2,
  ompt_mutex_critical                 = 3,
  ompt_mutex_atomic                   = 4,
  ompt_mutex_ordered                  = 5
} ompt_mutex_t;
\end{omptEnum}
\end{ccppspecific}




% ompt_native_mon_flag_t

\subsubsubsection{\hcode{ompt_native_mon_flag_t}}
\label{sec:ompt_native_mon_flag_t}
\code{ompt_native_mon_flag_t} is an enumeration that defines the valid native monitoring flag values.


\begin{ccppspecific}
\begin{omptEnum}
typedef enum ompt_native_mon_flag_t {
  ompt_native_data_motion_explicit    = 0x01,
  ompt_native_data_motion_implicit    = 0x02,
  ompt_native_kernel_invocation       = 0x04,
  ompt_native_kernel_execution        = 0x08,
  ompt_native_driver                  = 0x10,
  ompt_native_runtime                 = 0x20,
  ompt_native_overhead                = 0x40,
  ompt_native_idleness                = 0x80
} ompt_native_mon_flag_t;
\end{omptEnum}
\end{ccppspecific}




% ompt_task_flag_t

\subsubsubsection{\hcode{ompt_task_flag_t}}
\label{sec:ompt_task_flag_t}
\code{ompt_task_flag_t} is an enumeration that defines the valid task type values.
The least significant byte provides information about the general classification of the task.
The other bits represent properties of the task.

\begin{ccppspecific}
\begin{omptEnum}
typedef enum ompt_task_flag_t {
  ompt_task_initial                   = 0x00000001,
  ompt_task_implicit                  = 0x00000002,
  ompt_task_explicit                  = 0x00000004,
  ompt_task_target                    = 0x00000008,
  ompt_task_undeferred                = 0x08000000,
  ompt_task_untied                    = 0x10000000,
  ompt_task_final                     = 0x20000000,
  ompt_task_mergeable                 = 0x40000000,
  ompt_task_merged                    = 0x80000000
} ompt_task_flag_t;
\end{omptEnum}
\end{ccppspecific}




%xxx
% ompt_task_status

\subsubsubsection{\hcode{ompt_task_status_t}}
\label{sec:ompt_task_status_t}
\code{ompt_task_status_t} is an enumeration that explains the
reasons for switching a task that reached  a task scheduling point.


\begin{ccppspecific}
\begin{omptEnum}
typedef enum ompt_task_status_t {
  ompt_task_complete  = 1,
  ompt_task_yield     = 2,
  ompt_task_cancel    = 3,
  ompt_task_switch    = 4
} ompt_task_status_t;
\end{omptEnum}
\end{ccppspecific}


The value \code{ompt_task_complete} indicates the completion of task that
encountered the task scheduling point. The value \code{ompt_task_yield} indicates
that the task encountered a \code{taskyield} construct. The value \code{ompt_task_cancel} indicates
that the task is canceled due to the encountering of an active cancellation point resulting in the
cancellation of that task.
The value \code{ompt_task_switch} is used in the remaining cases of task switches.

% ompt_target_t

\subsubsubsection{\hcode{ompt_target_t}}
\label{sec:ompt_target_t}
\code{ompt_target_t} is an enumeration that defines the valid target type values.


\begin{ccppspecific}
\begin{omptEnum}
typedef enum ompt_target_t {
  ompt_target                         = 1,
  ompt_target_enter_data              = 2,
  ompt_target_exit_data               = 3,
  ompt_target_update                  = 4
} ompt_target_t;
\end{omptEnum}
\end{ccppspecific}




% ompt_parallel_t

\subsubsubsection{\hcode{ompt_parallel_flag_t}}
\label{sec:ompt_parallel_flag_t}
\code{ompt_parallel_flag_t} is an enumeration that defines the valid invoker values.


\begin{ccppspecific}
\begin{omptEnum}
typedef enum ompt_parallel_flag_t {
  ompt_parallel_invoker_program = 0x00000001,
  ompt_parallel_invoker_runtime = 0x00000002,
  ompt_parallel_league          = 0x40000000,
  ompt_parallel_team            = 0x80000000
} ompt_parallel_flag_t;
\end{omptEnum}
\end{ccppspecific}

\descr

The value \code{ompt_parallel_invoker_program} indicates that on the master thread for a parallel region,
the outlined function associated with implicit tasks for the region
is invoked directly by the application.

The value \code{ompt_parallel_invoker_runtime} indicates that on the master thread for a parallel region, 
the outlined function associated with implicit tasks for the region
is invoked by the runtime.

The value \code{ompt_parallel_league} indicates that the callback indicates
the creation of a league of teams by a \code{teams} construct.

The value \code{ompt_parallel_team} indicates that the callback indicates
the creation of a team of threads by a \code{parallel} construct.





% ompt_target_map_flag_t

\subsubsubsection{\hcode{ompt_target_map_flag_t}}
\label{sec:ompt_target_map_flag_t}
\code{ompt_target_map_flag_t} is an enumeration that defines the valid target map flag values.


\begin{ccppspecific}
\begin{omptEnum}
typedef enum ompt_target_map_flag_t {
  ompt_target_map_flag_to             = 0x01,
  ompt_target_map_flag_from           = 0x02,
  ompt_target_map_flag_alloc          = 0x04,
  ompt_target_map_flag_release        = 0x08,
  ompt_target_map_flag_delete         = 0x10,
  ompt_target_map_flag_implicit       = 0x20
} ompt_target_map_flag_t;
\end{omptEnum}
\end{ccppspecific}




% ompt_dependence_type_t

\subsubsubsection{\hcode{ompt_dependence_type_t}}
\label{sec:ompt_dependence_type_t}
\code{ompt_dependence_type_t} is an enumeration that defines the valid task dependence 
type values.


\begin{ccppspecific}
\begin{omptEnum}
typedef enum ompt_dependence_type_t {
  ompt_dependence_type_in              = 1,
  ompt_dependence_type_out             = 2,
  ompt_dependence_type_inout           = 3,
  ompt_dependence_type_mutexinoutset   = 4,
  ompt_dependence_type_source          = 5,
  ompt_dependence_type_sink            = 6
} ompt_dependence_type_t;
\end{omptEnum}
\end{ccppspecific}





% ompt_cancel_flag_t

\subsubsubsection{\hcode{ompt_cancel_flag_t}}
\label{sec:ompt_cancel_flag_t}
\code{ompt_cancel_flag_t} is an enumeration that defines the valid cancel flag values.


\begin{ccppspecific}
\begin{omptEnum}
typedef enum ompt_cancel_flag_t {
  ompt_cancel_parallel       = 0x01,
  ompt_cancel_sections       = 0x02,
  ompt_cancel_loop           = 0x04,
  ompt_cancel_taskgroup      = 0x08,
  ompt_cancel_activated      = 0x10,
  ompt_cancel_detected       = 0x20,
  ompt_cancel_discarded_task = 0x40
} ompt_cancel_flag_t;
\end{omptEnum}
\end{ccppspecific}


% ompt_hwid_t

\subsubsubsection{\hcode{ompt_hwid_t}}
\label{sec:ompt_hwid_t}
\code{ompt_hwid_t} is an opaque object representing a hardware identifier for a target device.
\label{sec:ompt_hwid_none}
\code{ompt_hwid_none} refers to an unknown or unspecified hardware identifier.
If there is no \plc{hwid} associated with an
\code{ompt_record_abstract_t}, the value of \plc{hwid} shall be
\code{ompt_hwid_none}.

\crossreferences
\begin{itemize}
\item \code{ompt_record_abstract_t} data type, see \specref{sec:ompt_record_abstract_t}.
\end{itemize}


\begin{ccppspecific}
\begin{omptOther}
typedef uint64_t ompt_hwid_t;
\end{omptOther}
\end{ccppspecific}

\code{ompt_hwid_none} is defined as an instance of type \code{ompt_hwid_t} with the value 
0.



% end miscellaneous types

% This is an included file. See the master file for more information.
%
% When editing this file:
%
%    1. To change formatting, appearance, or style, please edit openmp.sty.
%
%    2. Custom commands and macros are defined in openmp.sty.
%
%    3. Be kind to other editors -- keep a consistent style by copying-and-pasting to
%       create new content.
%
%    4. We use semantic markup, e.g. (see openmp.sty for a full list):
%         \code{}     % for bold monospace keywords, code, operators, etc.
%         \plc{}      % for italic placeholder names, grammar, etc.
%
%    5. There are environments that provide special formatting, e.g. language bars.
%       Please use them whereever appropriate.  Examples are:
%
%         \begin{fortranspecific}
%         This is text that appears enclosed in blue language bars for Fortran.
%         \end{fortranspecific}
%
%         \begin{note}
%         This is a note.  The "Note -- " header appears automatically.
%         \end{note}
%
%    6. Other recommendations:
%         Use the convenience macros defined in openmp.sty for the minor headers
%         such as Comments, Syntax, etc.
%
%         To keep items together on the same page, prefer the use of 
%         \begin{samepage}.... Avoid \parbox for text blocks as it interrupts line numbering.
%         When possible, avoid \filbreak, \pagebreak, \newpage, \clearpage unless that's
%         what you mean. Use \needspace{} cautiously for troublesome paragraphs.
%
%         Avoid absolute lengths and measures in this file; use relative units when possible.
%         Vertical space can be relative to \baselineskip or ex units. Horizontal space
%         can be relative to \linewidth or em units.
%
%         Prefer \emph{} to italicize terminology, e.g.:
%             This is a \emph{definition}, not a placeholder.
%             This is a \plc{var-name}.
%


\subsection{OMPT Tool Callback Signatures and Trace Records}
\label{sec:ompt-tool-callbacks}

\restrictions
Tool callbacks may not use OpenMP directives or call any runtime library routines
described in Section~\ref{chap:Runtime Library Routines}.

\subsubsection{Initialization and Finalization Callback Signature}

\omptcallbacksignature{\code{ompt\_initialize\_t}}
\label{sec:ompt_initialize_t}

\summary
A tool implements an initializer with the type signature
\code{ompt\_initialize\_t} to initialize the tool's use of
the OMPT interface.

\format
\vbox{
\begin{ccppspecific}
\begin{boxedcode}
typedef int (*ompt_initialize_t) (
  ompt_function_lookup_t \plc{lookup},
  ompt_data_t *\plc{tool_data}
);
\end{boxedcode}
\end{ccppspecific}
}

\descr
For a tool to use the OMPT interface of an OpenMP implementation,
the tool's implementation of \code{ompt\_start\_tool} must return a 
non-\code{NULL} pointer to an
\code{ompt\_start\_tool\_result\_t} structure that contains a
non-\code{NULL} pointer to a tool initializer with
type signature \code{ompt\_initialize\_t}.
An OpenMP implementation will call the tool initializer
after fully initializing itself but before
beginning execution of any OpenMP construct
or completing execution of any environment routine invocation.  

The initializer returns a non-zero value if it succeeds.

\argdesc
The \callbackarg{} \plc{lookup} is a callback 
to an OpenMP runtime routine that a tool must use to 
obtain a pointer to each runtime entry point in the OMPT interface.
The \callbackarg{} \plc{tool\_data} is a pointer to the
\plc{tool\_data} field in the \code{ompt\_start\_tool\_result\_t}
structure returned by \code{ompt\_start\_tool}.
The expected actions of a tool initializer are described in
\specref{sec:tool-initialize}.

\crossreferences
\begin{itemize}
\item \code{ompt\_start\_tool\_result\_t}, see
  \specref{sec:ompt_start_tool_result_t}.
\item \code{ompt\_data\_t}, see \specref{sec:ompt_data_t}.
\item \code{ompt\_start\_tool}, see \specref{sec:ompt_start_tool}.
\item \code{ompt\_function\_lookup\_t}, see
  \specref{sec:ompt_function_lookup_t}.
\end{itemize}


\omptcallbacksignature{\code{ompt\_finalize\_t}}
\label{sec:ompt_finalize_t}

\summary
A tool implements a finalizer with the type signature
\code{ompt\_finalize\_t} to finalize the tool's use of
the OMPT interface.

\format
\vbox{
\begin{ccppspecific}
\begin{boxedcode}
typedef void (*ompt_finalize_t) (
  ompt_data_t *\plc{tool_data}
);
\end{boxedcode}
\end{ccppspecific}
}

\descr
For a tool to use the OMPT interface of an OpenMP implementation,
the tool's implementation of \code{ompt\_start\_tool} must return a
non-\code{NULL} pointer to an
\code{ompt\_start\_tool\_result\_t} structure that contains a
non-\code{NULL} pointer to a tool finalizer with
type signature \code{ompt\_finalize\_t}.
An OpenMP implementation will call the tool finalizer
after the last OMPT \plc{event} as the OpenMP implementation shuts down.

\argdesc
The \callbackarg{} \plc{tool\_data} is a pointer to the
\plc{tool\_data} field in the \code{ompt\_start\_tool\_result\_t}
structure returned by \code{ompt\_start\_tool}.

\crossreferences
\begin{itemize}
\item \code{ompt\_start\_tool\_result\_t}, see
  \specref{sec:ompt_start_tool_result_t}.
\item \code{ompt\_data\_t}, see \specref{sec:ompt_data_t}.
\item \code{ompt\_start\_tool}, see \specref{sec:ompt_start_tool}.
\end{itemize}


\subsubsection{Event Callback Signatures and Trace Records}
\index{event callback signatures}
\label{sec:ToolsSupport_callback_signatures}

This section describes the signatures of tool callback functions that an OMPT
tool might register and that are called during runtime of an OpenMP program.

\omptcallbacksignature{\code{ompt\_callback\_thread\_begin\_t}}
\index{ompt\_callback\_thread\_begin\_t@{\code{ompt\_callback\_thread\_begin\_t}}}
\label{sec:ompt_callback_thread_begin_t}
\format
\vbox{
\begin{ccppspecific}
\begin{boxedcode}
typedef void (*ompt_callback_thread_begin_t) (
  ompt_thread_type_t \plc{thread_type},
  ompt_data_t *\plc{thread_data}
);
\end{boxedcode}
\end{ccppspecific}
}

\record
\vbox{
\begin{ccppspecific}
\begin{boxedcode}
typedef struct ompt_record_thread_begin_t \{
  ompt_thread_type_t \plc{thread_type};
\} ompt_record_thread_begin_t;
\end{boxedcode}
\end{ccppspecific}
}

\argdesc

The \callbackarg{} \plc{thread\_type}
indicates the type of the new thread: initial, worker, or other.

The binding of \callbackarg{} \plc{thread\_data} is the new thread.

\crossreferences
\begin{itemize}
\item \code{ompt\_data\_t} type, see
\specref{sec:ompt_data_t}.
\item \code{ompt\_thread\_type\_t} type, see
\specref{sec:ompt_thread_type_t}.
\end{itemize}



\omptcallbacksignature{\code{ompt\_callback\_thread\_end\_t}}
\index{ompt\_callback\_thread\_end\_t@{\code{ompt\_callback\_thread\_end\_t}}}
\label{sec:ompt_callback_thread_end_t}
\format
\vbox{
\begin{ccppspecific}
\begin{boxedcode}
typedef void (*ompt_callback_thread_end_t) (
  ompt_data_t *\plc{thread_data}
);
\end{boxedcode}
\end{ccppspecific}
}

\argdesc

The binding of \callbackarg{} \plc{thread\_data} is the thread that is
terminating.

\crossreferences
\begin{itemize}
\item \code{ompt\_data\_t} type, see
\specref{sec:ompt_data_t}.
\item \code{ompt\_record\_ompt\_t} type, see
\specref{sec:ompt_record_ompt_t}.
\end{itemize}


\omptcallbacksignature{\code{ompt\_callback\_idle\_t}}
\index{ompt\_callback\_idle\_t@{\code{ompt\_callback\_idle\_t}}}
\label{sec:ompt_callback_idle_t}
\format
\vbox{
\begin{ccppspecific}
\begin{boxedcode}
typedef void (*ompt_callback\_idle_t) (
  ompt_scope_endpoint_t \plc{endpoint}
);
\end{boxedcode}
\end{ccppspecific}
}

\record
\vbox{
\begin{ccppspecific}
\begin{boxedcode}
typedef struct ompt_record_idle_t \{
  ompt_scope_endpoint_t \plc{endpoint};
\} ompt_record_idle_t;
\end{boxedcode}
\end{ccppspecific}
}

\argdesc

The \callbackarg{} \plc{endpoint} indicates whether the callback is
signalling the beginning or end of an idle interval.

\crossreferences
\begin{itemize}
\item \code{ompt\_scope\_endpoint\_t} type, see
\specref{sec:ompt_scope_endpoint_t}.
\end{itemize}



\omptcallbacksignature{\code{ompt\_callback\_parallel\_begin\_t}}
\index{ompt\_callback\_parallel\_begin\_t@{\code{ompt\_callback\_parallel\_begin\_t}}}
\label{sec:ompt_callback_parallel_begin_t}
\format
\vbox{
\begin{ccppspecific}
\begin{boxedcode}
typedef void (*ompt_callback_parallel_begin_t) (
  ompt_data_t *\plc{encountering_task_data},
  const ompt_frame_t *\plc{encountering_task_frame},
  ompt_data_t *\plc{parallel_data},
  unsigned int \plc{requested_team_size},
  ompt_invoker_t \plc{invoker},
  const void *\plc{codeptr_ra}
);
\end{boxedcode}
\end{ccppspecific}
}

\record
\vbox{
\begin{ccppspecific}
\begin{boxedcode}
typedef struct ompt_record_parallel_begin_t \{
  ompt_id_t \plc{encountering_task_id};
  ompt_id_t \plc{parallel_id};
  unsigned int \plc{requested_team_size};
  ompt_invoker_t \plc{invoker};
  const void *\plc{codeptr_ra};
\} ompt_record_parallel_begin_t;
\end{boxedcode}
\end{ccppspecific}
}

\argdesc

The binding of \callbackarg{} \plc{encountering\_task\_data} is the
encountering task.

The \callbackarg{} \plc{encountering\_task\_frame} points to the frame object
associated with the encountering task.

The binding of \callbackarg{} \plc{parallel\_data} is the parallel
region that is beginning.

The \callbackarg{} \plc{requested\_team\_size}
indicates the number of threads requested by the user. 

The \callbackarg{} \plc{invoker} indicates whether the code for the
parallel region is inlined into the application or invoked by the
runtime.

\codeptrdesc

\crossreferences
\begin{itemize}
\item \code{ompt\_data\_t} type, see \specref{sec:ompt_data_t}.
\item \code{omp\_frame\_t} type, see \specref{sec:omp_frame_t}.
\item \code{ompt\_invoker\_t} type, see \specref{sec:ompt_invoker_t}.
\end{itemize}



\omptcallbacksignature{\code{ompt\_callback\_parallel\_end\_t}}
\index{ompt\_callback\_parallel\_end\_t@{\code{ompt\_callback\_parallel\_end\_t}}}
\label{sec:ompt_callback_parallel_end_t}
\format
\vbox{
\begin{ccppspecific}
\begin{boxedcode}
typedef void (*ompt_callback_parallel_end_t) (
  ompt_data_t *\plc{parallel_data},
  ompt_data_t *\plc{encountering_task_data},
  ompt_invoker_t \plc{invoker},
  const void *\plc{codeptr_ra}
);
\end{boxedcode}
\end{ccppspecific}
}

\record
\vbox{
\begin{ccppspecific}
\begin{boxedcode}
typedef struct ompt_record_parallel_end_t \{
  ompt_id_t \plc{parallel_id}; 
  ompt_id_t \plc{encountering_task_id};
  ompt_invoker_t \plc{invoker};
  const void *\plc{codeptr_ra};
\} ompt_record_parallel_end_t;
\end{boxedcode}
\end{ccppspecific}
}

\argdesc

The binding of \callbackarg{} \plc{parallel\_data} is the parallel
region that is ending.

The binding of \callbackarg{} \plc{encountering\_task\_data} is the encountering
task.

The \callbackarg{} \plc{invoker} explains whether the execution of the
parallel region code is inlined into the application code or started
by the runtime.

\codeptrdesc

\crossreferences
\begin{itemize}
\item \code{ompt\_data\_t} type signature, see
\specref{sec:ompt_data_t}.
\item \code{ompt\_invoker\_t} type signature, see
\specref{sec:ompt_invoker_t}.
\end{itemize}

\omptcallbacksignature{\code{ompt\_callback\_master\_t}}
\index{ompt\_callback\_master\_t@{\code{ompt\_callback\_master\_t}}}
\label{sec:ompt_callback_master_t}
\format
\vbox{
\begin{ccppspecific}
\begin{boxedcode}
typedef void (*ompt_callback_master_t) (
  ompt_scope_endpoint_t \plc{endpoint},
  ompt_data_t *\plc{parallel_data},
  ompt_data_t *\plc{task_data},
  const void *\plc{codeptr_ra}
);
\end{boxedcode}
\end{ccppspecific}
}

\record
\vbox{
\begin{ccppspecific}
\begin{boxedcode}
typedef struct ompt_record_master_t \{
  ompt_scope_endpoint_t \plc{endpoint};
  ompt_id_t \plc{parallel_id};
  ompt_id_t \plc{task_id};
  const void *\plc{codeptr_ra};
\} ompt_record_master_t;
\end{boxedcode}
\end{ccppspecific}
}

\argdesc

\epdesc

The binding of \callbackarg{} \plc{parallel\_data} is the current parallel region.

The binding of \callbackarg{} \plc{task\_data} is the encountering task.

\codeptrdesc

%\effect
% ompt events have no effect

\crossreferences
\begin{itemize}
\item \code{ompt\_data\_t} type signature, see \specref{sec:ompt_data_t}.
\item \code{ompt\_scope\_endpoint\_t} type, see \specref{sec:ompt_scope_endpoint_t}.
\end{itemize}





\omptcallbacksignature{\code{ompt\_callback\_task\_create\_t}}
\index{ompt\_callback\_task\_create\_t@{\code{ompt\_callback\_task\_create\_t}}}
\label{sec:ompt_callback_task_create_t}
\format
\vbox{
\begin{ccppspecific}
\begin{boxedcode}
typedef void (*ompt_callback_task_create_t) (
  ompt_data_t *\plc{encountering_task_data},
  const ompt_frame_t *\plc{encountering_task_frame},
  ompt_data_t *\plc{new_task_data},
  int \plc{type},
  int \plc{has_dependences},
  const void *\plc{codeptr_ra}
);
\end{boxedcode}
\end{ccppspecific}
}

\record
\vbox{
\begin{ccppspecific}
\begin{boxedcode}
typedef struct ompt_record_task_create_t \{
  ompt_id_t \plc{encountering_task_id};
  ompt_id_t \plc{new_task_id};
  int \plc{type};
  int \plc{has_dependences};
  const void *\plc{codeptr_ra};
\} ompt_record_task_create_t;
\end{boxedcode}
\end{ccppspecific}
}

\argdesc

The binding of \callbackarg{} \plc{encountering\_task\_data} is the
encountering task.  
This parameter is \code{NULL} for an initial task.

The \callbackarg{} \plc{encountering\_task\_frame} points to the frame object
associated with the encountering task. 
This parameter is \code{NULL} for an initial task.

The binding of \callbackarg{} \plc{new\_task\_data} is the created
task.

The \callbackarg{} \plc{type} indicates the kind of the task: initial,
explicit or target.
Values for \plc{type} are composed by or-ing elements of enum
\code{ompt\_task\_type\_t}.

The \callbackarg{} \plc{has\_dependences} indicates whether created
task has dependences.

\codeptrdesc

\crossreferences
\begin{itemize}
\item \code{ompt\_data\_t} type, see
\specref{sec:ompt_data_t}.
\item \code{omp\_frame\_t} type, see
\specref{sec:omp_frame_t}.
\item \code{ompt\_task\_type\_t} type, see
\specref{sec:ompt_task_type_t}.
\end{itemize}



\omptcallbacksignature{\code{ompt\_callback\_task\_dependences\_t}}
\index{ompt\_callback\_task\_dependences\_t@{\code{ompt\_callback\_task\_dependences\_t}}}
\label{sec:ompt_callback_task_dependences_t}
\format
\vbox{
\begin{ccppspecific}
\begin{boxedcode}
typedef void (*ompt_callback_task_dependences_t) (
  ompt_data_t *\plc{task_data},
  const ompt_task_dependence_t *\plc{deps},
  int \plc{ndeps}
);
\end{boxedcode}
\end{ccppspecific}
}

\record
\vbox{
\begin{ccppspecific}
\begin{boxedcode}
typedef struct ompt_record_task_dependences_t \{
  ompt_id_t \plc{task_id};
  ompt_task_dependence_t \plc{dep};
  int \plc{ndeps};
\} ompt_record_task_dependences_t;
\end{boxedcode}
\end{ccppspecific}
}


\argdesc

The binding of \callbackarg{} \plc{task\_data} is the task being created.

The \callbackarg{} \plc{deps} lists all dependences of a new task.

The \callbackarg{} \plc{ndeps} specifies the length of the list.
The memory for \plc{deps} is owned by the caller; the tool cannot rely on
the data after the callback returns.

The performance monitor interface for tracing activity on target devices will provide one record per dependence.

\crossreferences
\begin{itemize}
\item \code{ompt\_data\_t} type, see
\specref{sec:ompt_data_t}.
\item \code{ompt\_task\_dependence\_t} type, see
\specref{sec:ompt_task_dependence_t}.
\end{itemize}



\omptcallbacksignature{\code{ompt\_callback\_task\_dependence\_t}}
\index{ompt\_callback\_task\_dependence\_t@{\code{ompt\_callback\_task\_dependence\_t}}}
\label{sec:ompt_callback_task_dependence_t}
\format
\vbox{
\begin{ccppspecific}
\begin{boxedcode}
typedef void (*ompt_callback_task_dependence_t) (
  ompt_data_t *\plc{src_task_data},
  ompt_data_t *\plc{sink_task_data}
);
\end{boxedcode}
\end{ccppspecific}
}

\record
\vbox{
\begin{ccppspecific}
\begin{boxedcode}
typedef struct ompt_record_task_dependence_t \{
  ompt_id_t \plc{src_task_id};
  ompt_id_t \plc{sink_task_id};
\} ompt_record_task_dependence_t;
\end{boxedcode}
\end{ccppspecific}
}

\argdesc

The binding of \callbackarg{} \plc{src\_task\_data} is a running task
with an outgoing dependence.

The binding of \callbackarg{} \plc{sink\_task\_data} is a task with an
unsatisfied incoming dependence.


\crossreferences
\begin{itemize}
\item \code{ompt\_data\_t} type signature, see
\specref{sec:ompt_data_t}.
\end{itemize}



\omptcallbacksignature{\code{ompt\_callback\_task\_schedule\_t}}
\index{ompt\_callback\_task\_schedule\_t@{\code{ompt\_callback\_task\_schedule\_t}}}
\label{sec:ompt_callback_task_schedule_t}
\format
\vbox{
\begin{ccppspecific}
\begin{boxedcode}
typedef void (*ompt_callback_task_schedule_t) (
  ompt_data_t *\plc{prior_task_data},
  ompt_task_status_t \plc{prior_task_status},
  ompt_data_t *\plc{next_task_data}
);
\end{boxedcode}
\end{ccppspecific}
}

\record
\vbox{
\begin{ccppspecific}
\begin{boxedcode}
typedef struct ompt_record_task_schedule_t \{
  ompt_id_t \plc{prior_task_id};
  ompt_task_status_t \plc{prior_task_status},
  ompt_id_t \plc{next_task_id};
\} ompt_record_task_schedule_t;
\end{boxedcode}
\end{ccppspecific}
}

\argdesc

The \callbackarg{} \plc{prior\_task\_status} indicates the status of
the task that arrived at a task scheduling point.

The binding of \callbackarg{} \plc{prior\_task\_data} is the task that
arrived at the scheduling point.

The binding of \callbackarg{} \plc{next\_task\_data} is the task that
will resume at the scheduling point.

\crossreferences
\begin{itemize}
\item \code{ompt\_data\_t} type, see
\specref{sec:ompt_data_t}.

\item \code{ompt\_task\_status\_t} type, see
\specref{sec:ompt_task_status_t}.
\end{itemize}



\omptcallbacksignature{\code{ompt\_callback\_implicit\_task\_t}}
\index{ompt\_callback\_implicit\_task\_t@{\code{ompt\_callback\_implicit\_task\_t}}}
\label{sec:ompt_callback_implicit_task_t}
\format
\vbox{
\begin{ccppspecific}
\begin{boxedcode}
typedef void (*ompt_callback_implicit_task_t) (
  ompt_scope_endpoint_t \plc{endpoint},
  ompt_data_t *\plc{parallel_data},
  ompt_data_t *\plc{task_data},
  unsigned int \plc{team_size},
  unsigned int \plc{thread_num}
);
\end{boxedcode}
\end{ccppspecific}
}

\record
\vbox{
\begin{ccppspecific}
\begin{boxedcode}
typedef struct ompt_record_implicit_t \{
  ompt_scope_endpoint_t \plc{endpoint};
  ompt_id_t \plc{parallel_id};
  ompt_id_t \plc{task_id};
  unsigned int \plc{team_size},
  unsigned int \plc{thread_num};
\} ompt_record_implicit_t;
\end{boxedcode}
\end{ccppspecific}
}

\argdesc

\epdesc

The binding of \callbackarg{} \plc{parallel\_data} is the current
parallel region.
For the \plc{implicit-task-end} event, this argument is \code{NULL}.

The binding of \callbackarg{} \plc{task\_data} is the implicit task
executing the parallel region's structured block.

The \callbackarg{} \plc{team\_size} indicates the number of
threads in the team.

The \callbackarg{} \plc{thread\_num} indicates
the thread number of the calling thread, within the team executing the parallel
region to which the implicit region binds.

\crossreferences
\begin{itemize}
\item \code{ompt\_data\_t} type, see
\specref{sec:ompt_data_t}.
\item \code{ompt\_scope\_endpoint\_t} enumeration type, see
\specref{sec:ompt_scope_endpoint_t}.
\end{itemize}



\omptcallbacksignature{\code{ompt\_callback\_sync\_region\_t}}
\index{ompt\_callback\_sync\_region\_t@{\code{ompt\_callback\_sync\_region\_t}}}
\label{sec:ompt_callback_sync_region_t}
\format
\vbox{
\begin{ccppspecific}
\begin{boxedcode}
typedef void (*ompt_callback_sync_region_t) (
  ompt_sync_region_kind_t \plc{kind},
  ompt_scope_endpoint_t \plc{endpoint},
  ompt_data_t *\plc{parallel_data},
  ompt_data_t *\plc{task_data},
  const void *\plc{codeptr_ra}
);
\end{boxedcode}
\end{ccppspecific}
}

\record
\vbox{
\begin{ccppspecific}
\begin{boxedcode}
typedef struct ompt_record_sync_region_t \{
  ompt_sync_region_kind_t \plc{kind};
  ompt_scope_endpoint_t \plc{endpoint};
  ompt_id_t \plc{parallel_id};
  ompt_id_t \plc{task_id};
  const void *\plc{codeptr_ra};
\} ompt_record_sync_region_t;
\end{boxedcode}
\end{ccppspecific}
}

% \descr
% A tool executes a callback with type signature
% \code{ompt\_callback\_sync\_region\_t} to receive notification of
% event \code{ompt\_callback\_sync\_region\_t} when an OpenMP
% implementation starts and stops waiting in a barrier region, taskwait
% region, or taskgroup region.

\argdesc

The \callbackarg{} \plc{kind} indicates the kind of
synchronization region.

\epdesc

The binding of \callbackarg{} \plc{parallel\_data} is the current
parallel region.
For the \plc{barrier-end} event at the end of a parallel region, 
this argument is \code{NULL}.

The binding of \callbackarg{} \plc{task\_data} is the current task.

\codeptrdesc

\crossreferences
\begin{itemize}
\item \code{ompt\_data\_t} type, see
\specref{sec:ompt_data_t}.
\item \code{ompt\_sync\_region\_kind\_t} type, see
\specref{sec:ompt_sync_region_kind_t}.
\item \code{ompt\_scope\_endpoint\_t} type, see
\specref{sec:ompt_scope_endpoint_t}.
\end{itemize}


\omptcallbacksignature{\code{ompt\_callback\_mutex\_acquire\_t}}
\index{ompt\_callback\_mutex\_acquire\_t@{\code{ompt\_callback\_mutex\_acquire\_t}}}
\label{sec:ompt_callback_mutex_acquire_t}
\format
\vbox{
\begin{ccppspecific}
\begin{boxedcode}
typedef void (*ompt_callback_mutex_acquire_t) (
  ompt_mutex_kind_t \plc{kind},
  unsigned int \plc{hint},
  unsigned int \plc{impl},
  omp_wait_id_t \plc{wait_id},
  const void *\plc{codeptr_ra}
);
\end{boxedcode}
\end{ccppspecific}
}

\record
\vbox{
\begin{ccppspecific}
\begin{boxedcode}
typedef struct ompt_record_mutex_acquire_t \{
  ompt_mutex_kind_t \plc{kind};
  unsigned int \plc{hint};
  unsigned int \plc{impl};
  omp_wait_id_t \plc{wait_id};
  const void *\plc{codeptr_ra};
\} ompt_record_mutex_acquire_t;
\end{boxedcode}
\end{ccppspecific}
}

\argdesc

The \callbackarg{} \plc{kind} indicates the kind of the lock.

The \callbackarg{} \plc{hint} indicates the hint provided when initializing
an implementation of mutual exclusion.
If no hint is available when a thread initiates acquisition of mutual exclusion,
the runtime may supply \code{omp\_lock\_hint\_none} as the value for \plc{hint}.

The \callbackarg{} \plc{impl} indicates the mechanism chosen by the
runtime to implement the mutual exclusion.  

The \callbackarg{} \plc{wait\_id} indicates the object being awaited. 

\codeptrdesc

\crossreferences
\begin{itemize}
\item \code{omp\_wait\_id\_t} type, see
\specref{sec:omp_wait_id_t}.
\item \code{ompt\_mutex\_kind\_t} type, see
\specref{sec:ompt_mutex_kind_t}.
\end{itemize}



\omptcallbacksignature{\code{ompt\_callback\_mutex\_t}}
\index{ompt\_callback\_mutex\_t@{\code{ompt\_callback\_mutex\_t}}}
\label{sec:ompt_callback_mutex_t}
\format
\vbox{
\begin{ccppspecific}
\begin{boxedcode}
typedef void (*ompt_callback_mutex_t) (
  ompt_mutex_kind_t \plc{kind},
  omp_wait_id_t \plc{wait_id},
  const void *\plc{codeptr_ra}
);
\end{boxedcode}
\end{ccppspecific}
}

\record
\vbox{
\begin{ccppspecific}
\begin{boxedcode}
typedef struct ompt_record_mutex_t \{
  ompt_mutex_kind_t \plc{kind};
  omp_wait_id_t \plc{wait_id};
  const void *\plc{codeptr_ra};
\} ompt_record_mutex_t;
\end{boxedcode}
\end{ccppspecific}
}

\argdesc

The \callbackarg{} \plc{kind} indicates the kind of mutual exclusion event.

The \callbackarg{} \plc{wait\_id} indicates the object being awaited. 

\codeptrdesc

\crossreferences
\begin{itemize}
\item \code{omp\_wait\_id\_t} type signature, see
\specref{sec:omp_wait_id_t}.
\item \code{ompt\_mutex\_kind\_t} type signature, see
\specref{sec:ompt_mutex_kind_t}.
\end{itemize}



\omptcallbacksignature{\code{ompt\_callback\_nest\_lock\_t}}
\index{ompt\_callback\_nest\_lock\_t@{\code{ompt\_callback\_nest\_lock\_t}}}
\label{sec:ompt_callback_nest_lock_t}
\format
\vbox{
\begin{ccppspecific}
\begin{boxedcode}
typedef void (*ompt_callback_nest_lock_t) (
  ompt_scope_endpoint_t \plc{endpoint},
  omp_wait_id_t \plc{wait_id},
  const void *\plc{codeptr_ra}
);
\end{boxedcode}
\end{ccppspecific}
}

\record
\vbox{
\begin{ccppspecific}
\begin{boxedcode}
typedef struct ompt_record_nest_lock_t \{
  ompt_scope_endpoint_t \plc{endpoint};
  omp_wait_id_t \plc{wait_id};
  const void *\plc{codeptr_ra};
\} ompt_record_nest_lock_t;
\end{boxedcode}
\end{ccppspecific}
}

\argdesc

\epdesc

The \callbackarg{} \plc{wait\_id} indicates the object being awaited. 

\codeptrdesc

\crossreferences
\begin{itemize}
\item \code{omp\_wait\_id\_t} type signature, see
\specref{sec:omp_wait_id_t}.
\item \code{ompt\_scope\_endpoint\_t} type signature, see
\specref{sec:ompt_scope_endpoint_t}.
\end{itemize}



\omptcallbacksignature{\code{ompt\_callback\_work\_t}}
\index{ompt\_callback\_work\_t@{\code{ompt\_callback\_work\_t}}}
\label{sec:ompt_callback_work_t}
\format
\vbox{
\begin{ccppspecific}
\begin{boxedcode}
typedef void (*ompt_callback_work_t) (
  ompt_work_type_t \plc{wstype},
  ompt_scope_endpoint_t \plc{endpoint},
  ompt_data_t *\plc{parallel_data},
  ompt_data_t *\plc{task_data},
  \longlongint{} \plc{count},
  const void *\plc{codeptr_ra}
);
\end{boxedcode}
\end{ccppspecific}
}

\record
\vbox{
\begin{ccppspecific}
\begin{boxedcode}
typedef struct ompt_record_work_t \{
  ompt_work_type_t \plc{wstype};
  ompt_scope_endpoint_t \plc{endpoint};
  ompt_id_t \plc{parallel_id};
  ompt_id_t \plc{task_id};
  \longlongint{} \plc{count};
  const void *\plc{codeptr_ra};
\} ompt_record_work_t;
\end{boxedcode}
\end{ccppspecific}
}

\argdesc

The \callbackarg{} \plc{wstype} indicates the kind of worksharing
region.

\epdesc

The binding of \callbackarg{} \plc{parallel\_data} is the current
parallel region.

The binding of \callbackarg{} \plc{task\_data} is the current task.

The \callbackarg{} \plc{count} is a measure of the quantity of work involved in the worksharing construct. 
For a loop construct, \plc{count} represents the number of iterations of the loop.
For a \code{taskloop} construct, \plc{count} represents the number of iterations in the iteration space, 
which may be the result of collapsing several associated loops.
For a \code{sections} construct, \plc{count} represents the number of sections. 
For a \code{workshare} construct, \plc{count} represents the units of work, as defined by the \code{workshare} construct.
For a \code{single} construct, \plc{count} is always 1.
When the \plc{endpoint} argument is signaling the end of a scope, a \plc{count} value of 0 indicates that the actual \plc{count} value is not available. 

\codeptrdesc


\crossreferences
\begin{itemize}
\item worksharing constructs, see \specref{sec:Worksharing Constructs}.
\item \code{ompt\_data\_t} type signature, see
\specref{sec:ompt_data_t}.
\item \code{ompt\_scope\_endpoint\_t} type signature, see
\specref{sec:ompt_scope_endpoint_t}.
\item \code{ompt\_work\_type\_t} type signature, see
\specref{sec:ompt_work_type_t}.
\end{itemize}



\omptcallbacksignature{\code{ompt\_callback\_flush\_t}}
\index{ompt\_callback\_flush\_t@{\code{ompt\_callback\_flush\_t}}}
\label{sec:ompt_callback_flush_t}
\format
\vbox{
\begin{ccppspecific}
\begin{boxedcode}
typedef void (*ompt_callback_flush_t) (
  ompt_data_t *\plc{thread_data},
  const void *\plc{codeptr_ra}
);
\end{boxedcode}
\end{ccppspecific}
}

\record
\vbox{
\begin{ccppspecific}
\begin{boxedcode}
typedef struct ompt_record_flush_t \{
  const void *\plc{codeptr_ra};
\} ompt_record_flush_t;
\end{boxedcode}
\end{ccppspecific}
}

\argdesc

\codeptrdesc

\crossreferences
\begin{itemize}
\item \code{ompt\_data\_t} type signature, see
\specref{sec:ompt_data_t}.
\end{itemize}



\omptcallbacksignature{\code{ompt\_callback\_target\_t}}
\index{ompt\_callback\_target\_t@{\code{ompt\_callback\_target\_t}}}
\label{sec:ompt_callback_target_t}
\format
\vbox{
\begin{ccppspecific}
\begin{boxedcode}
typedef void (*ompt_callback_target_t) (
  ompt_target_type_t \plc{kind},
  ompt_scope_endpoint_t \plc{endpoint},
  uint64_t \plc{device_num},
  ompt_data_t *\plc{task_data},
  ompt_id_t \plc{target_id},
  const void *\plc{codeptr_ra}
);
\end{boxedcode}
\end{ccppspecific}
}

\record
\vbox{
\begin{ccppspecific}
\begin{boxedcode}
typedef struct ompt_record_target_t \{
  ompt_target_type_t \plc{kind};
  ompt_scope_endpoint_t \plc{endpoint};
  uint64_t \plc{device_num};
  ompt_data_t *\plc{task_data};
  ompt_id_t \plc{target_id};
  const void *\plc{codeptr_ra};
\} ompt_record_target_t;
\end{boxedcode}
\end{ccppspecific}
}

\argdesc

The \callbackarg{} \plc{kind} indicates the kind of target region.

\epdesc

The \callbackarg{} \plc{device\_num} indicates the id of the device
which will execute the target region.

The binding of \callbackarg{} \plc{task\_data} is the target task.

The binding of \callbackarg{} \plc{target\_id} is the target region.

\codeptrdesc

\crossreferences
\begin{itemize}
\item \code{ompt\_id\_t} type, see
\specref{sec:ompt_id_t}.
\item \code{ompt\_data\_t} type signature, see
\specref{sec:ompt_data_t}.
\item \code{ompt\_scope\_endpoint\_t} type signature, see
\specref{sec:ompt_scope_endpoint_t}.
\item \code{ompt\_target\_type\_t} type signature, see
\specref{sec:ompt_target_type_t}.
\end{itemize}

\omptcallbacksignature{\code{ompt\_callback\_device\_load\_t}}
\index{ompt\_callback\_target\_code\_t@{\code{ompt\_callback\_device\_load\_t}}}
\label{sec:ompt_callback_device_load_t}

\summary
The OpenMP runtime invokes this callback to notify a tool immediately after loading code onto the specified device.

\format

\vbox{
\begin{ccppspecific}
\begin{boxedcode}
typedef void (*ompt_callback_device_load_t) (
  uint64_t \plc{device_num},
  const char *\plc{filename},
  int64_t \plc{offset_in_file},
  void *\plc{vma_in_file}, 
  size_t \plc{bytes}
  void *\plc{host_addr},
  void *\plc{device_addr},
  uint64_t \plc{module_id}
);

#define ompt_addr_unknown  ((void *) ~0)
\end{boxedcode}

\end{ccppspecific}
}

\argdesc
The \callbackarg{} \plc{device\_num} specifies the device. 

The \callbackarg{} \plc{filename} indicates the name of a file in which the device code can be found. A NULL \plc{filename} indicates that the code is not available in a file in the file system.

The \callbackarg{} \plc{offset\_in\_file} indicates an offset into \plc{filename} at which the code can be found. A value of -1 indicates that
no offset is provided.

The \callbackarg{} \plc{vma\_in\_file} indicates an virtual address in \plc{filename} at which the code can be found. 
A value of \plc{ompt\_addr\_unknown} indicates that a virtual address in the file is not available.

The \callbackarg{} \plc{bytes} indicates the size of the device code object in bytes.

The \callbackarg{} \plc{host\_addr} indicates where a copy of the device code is available in host memory.
A value of \plc{ompt\_addr\_unknown} indicates that a host code address is not available.

The \callbackarg{} \plc{device\_addr} indicates where the device code has been loaded in device memory. 
A value of \plc{ompt\_addr\_unknown} indicates that a device code address is not available.

The \callbackarg{} \plc{module\_id} is an identifier that is associated with the device code object. 


\omptcallbacksignature{\code{ompt\_callback\_device\_unload\_t}}
\index{ompt\_callback\_target\_code\_t@{\code{ompt\_callback\_device\_unload\_t}}}
\label{sec:ompt_callback_device_unload_t}

\summary
The OpenMP runtime invokes this callback to notify a tool immediately prior to unloading code from the specified device.

\format

\vbox{
\begin{ccppspecific}
\begin{boxedcode}
typedef void (*ompt_callback_device_unload_t) (
  uint64_t \plc{device_num},
  uint64_t \plc{module_id}
);
\end{boxedcode}

\end{ccppspecific}
}

\argdesc

The \callbackarg{} \plc{device\_num} specifies the device. 

The \callbackarg{} \plc{module\_id} is an identifier that is associated with the device code object. 


\omptcallbacksignature{\code{ompt\_callback\_target\_data\_op\_t}}
\index{ompt\_callback\_target\_data\_op\_t@{\code{ompt\_callback\_target\_data\_op\_t}}}
\label{sec:ompt_callback_target_data_op_t}
\format

\vbox{
\begin{ccppspecific}
\begin{boxedcode}
typedef void (*ompt_callback_target_data_op_t) (
  ompt_id_t \plc{target_id},
  ompt_id_t \plc{host_op_id},
  ompt_target_data_op_t \plc{optype},
  void *\plc{src_addr},
  int \plc{src_device_num},
  void *\plc{dest_addr},
  int \plc{dest_device_num},
  size_t \plc{bytes},
  const void *\plc{codeptr_ra}
);
\end{boxedcode}
\end{ccppspecific}
}

\descr
An OpenMP implementation will dispatch a registered \code{ompt\_callback\_target\_data\_op} callback
when device memory is allocated or freed, as well as when data is copied to or from a device. 

\begin{note}

An OpenMP implementation may aggregate program variables and data
operations upon them.  For instance, an OpenMP implementation may
synthesize a composite to represent multiple scalars and then
allocate, free, or copy this composite as a whole rather than
performing data operations on each scalar individually.  For that
reason, a tool should not expect to see separate data operations
on each variable.

\end{note}

\record
\vbox{
\begin{ccppspecific}
\begin{boxedcode}
typedef struct ompt_record_target_data_op_t \{
  ompt_id_t \plc{host\_op\_id};
  ompt_target_data_op_t \plc{optype};
  void *\plc{src\_addr};
  int \plc{src_device_num};
  void *\plc{dest\_addr};
  int \plc{dest_device_num};
  size_t \plc{bytes};
  ompt_device_time_t \plc{end\_time};
  const void *\plc{codeptr_ra};
\} ompt_record_target_data_op_t;
\end{boxedcode}
\end{ccppspecific}
}

\argdesc

The \callbackarg{} \plc{host\_op\_id} is a unique identifer for a data
operations on a target device.

The \callbackarg{} \plc{optype} indicates the kind of data mapping.

The \callbackarg{} \plc{src\_addr} indicates the address of data 
before the operation, where applicable.

The \callbackarg{} \plc{src\_device\_num} indicates the source device number
for the data operation, where applicable.

The \callbackarg{} \plc{dest\_addr} indicates the address of data 
after the operation.

The \callbackarg{} \plc{dest\_device\_num} indicates the destination device
number for the data operation.

It is implementation defined whether in some operations \plc{src\_addr} 
or \plc{dest\_addr} might point to an intermediate buffer.

The \callbackarg{} \plc{bytes} indicates the size of data.

\codeptrdesc

\crossreferences
\begin{itemize}
\item \code{ompt\_id\_t} type, see
\specref{sec:ompt_id_t}.
\item \code{ompt\_target\_data\_op\_t} type signature, see
\specref{sec:ompt_target_data_op_t}.
\end{itemize}



\omptcallbacksignature{\code{ompt\_callback\_target\_map\_t}}
\index{ompt\_callback\_target\_map\_t@{\code{ompt\_callback\_target\_map\_t}}}
\label{sec:ompt_callback_target_map_t}
\format
\vbox{
\begin{ccppspecific}
\begin{boxedcode}
typedef void (*ompt_callback_target_map_t) (
  ompt_id_t \plc{target_id},
  unsigned int \plc{nitems},
  void **\plc{host_addr},
  void **\plc{device_addr},
  size_t *\plc{bytes},
  unsigned int *\plc{mapping_flags},
  const void *\plc{codeptr_ra}
);
\end{boxedcode}
\end{ccppspecific}
}

\record
\vbox{
\begin{ccppspecific}
\begin{boxedcode}
typedef struct ompt_record_target_map_t \{
  ompt_id_t \plc{target_id};
  unsigned int \plc{nitems};
  void **\plc{host_addr};
  void **\plc{device_addr};
  size_t *\plc{bytes};
  unsigned int *\plc{mapping_flags};
  const void *\plc{codeptr_ra};
\} ompt_record_target_map_t;
\end{boxedcode}
\end{ccppspecific}
}

\descr
An instance of a \code{target}, \code{target data}, or \code{target enter data} construct may contain one or more \code{map} clauses. 
An OpenMP implementation may report the set of mappings associated with \code{map} clauses for a construct 
with a single \code{ompt\_callback\_target\_map} callback to report the effect of all mappings or multiple 
\code{ompt\_callback\_target\_map} callbacks with each reporting a subset of the mappings. 
Furthermore, an OpenMP implementation may omit mappings that it determines are unnecessary.
If an OpenMP implementation issues multiple \code{ompt\_callback\_target\_map}
callbacks, these callbacks may be interleaved with \code{ompt\_callback\_target\_data\_op} callbacks 
used to report data operations associated with the mappings. 

\argdesc

The binding of \callbackarg{} \plc{target\_id} is the target region.

The \callbackarg{} \plc{nitems} indicates the number of data mappings being reported by this callback.

The \callbackarg{} \plc{host\_addr} indicates an array of addresses of
data on host side.

The \callbackarg{} \plc{device\_addr} indicates an array of addresses
of data on device side.

The \callbackarg{} \plc{bytes} indicates an array of size of data.

The \callbackarg{} \plc{mapping\_flags} indicates the kind of data
mapping. Flags for a mapping include one or more values specified by the type
\code{ompt\_target\_map\_flag\_t}.

\codeptrdesc


\crossreferences
\begin{itemize}
\item \code{ompt\_id\_t} type, see
\specref{sec:ompt_id_t}.
\item \code{ompt\_callback\_target\_data\_op\_t}, 
see \specref{sec:ompt_callback_target_data_op_t}.
\item \code{ompt\_target\_map\_flag\_t} type, see
\specref{sec:ompt_target_map_flag_t}.
\end{itemize}


\omptcallbacksignature{\code{ompt\_callback\_target\_submit\_t}}
\index{ompt\_callback\_target\_submit\_t@{\code{ompt\_callback\_target\_submit\_t}}}
\label{sec:ompt_callback_target_submit_t}
\format
\vbox{
\begin{ccppspecific}
\begin{boxedcode}
typedef void (*ompt_callback_target_submit_t) (
  ompt_id_t \plc{target_id},
  ompt_id_t \plc{host_op_id},
  unsigned int \plc{requested_num_teams}
);
\end{boxedcode}
\end{ccppspecific}
}

\descr
A thread dispatches a registered \code{ompt\_callback\_target\_submit} callback on the host when
a target task creates an initial task on a target device.

\argdesc

The \callbackarg{} \plc{target\_id} is a unique identifier for the
associated target region.

The \callbackarg{} \plc{host\_op\_id} is a unique identifer for the
initial task on the target device.

The \callbackarg{} \plc{requested\_num\_teams} is the number of teams that the host is requesting to
execute the kernel. The actual number of teams that execute the kernel may be smaller and generally won't be
known until the kernel begins to execute on the device.

\constraints

The \callbackarg{} \plc{target\_id} indicates the instance of the
target construct to which the computation belongs.

The \callbackarg{} \plc{host\_op\_id} provides a unique host-side
identifier that represents the computation on the device.

\record
\vbox{
\begin{ccppspecific}
\begin{boxedcode}
typedef struct ompt_record_target_kernel_t \{
  ompt_id_t \plc{host_op_id};
  unsigned int \plc{requested_num_teams};
  unsigned int \plc{granted_num_teams};
  ompt_device_time_t \plc{end_time};
\} ompt_record_target_kernel_t;
\end{boxedcode}
\end{ccppspecific}
}


If a tool has configured a device to trace kernel execution using \code{ompt\_set\_trace\_ompt},
the device will log a \code{ompt\_record\_target\_kernel\_t} record in a trace. The fields in the record
are as follows:

\begin{itemize}
\item The \plc{host\_op\_id} field contains a unique identifier that a tool can use to correlate a
\code{ompt\_record\_target\_kernel\_t} record with its associated \code{ompt\_callback\_target\_submit} callback
on the host.

\item The \plc{requested\_num\_teams} field contains the number of teams that the host requested to execute the kernel.

\item The \plc{granted\_num\_teams} field contains the number of teams that the device actually used to execute the kernel.

\item The time when the initial task began execution on the device is recorded in the \plc{time} field of an enclosing
\code{ompt\_record\_t} structure; the time when the initial task completed execution on the device is recorded in the \plc{end\_time} field.
\end{itemize}

\crossreferences
\begin{itemize}
\item \code{ompt\_id\_t} type, see
\specref{sec:ompt_id_t}.
\end{itemize}


\omptcallbacksignature{\code{ompt\_callback\_buffer\_request\_t}}
\index{ompt\_callback\_buffer\_request\_t@{\code{ompt\_callback\_buffer\_request\_t}}}
\label{sec:ompt_callback_buffer_request_t}

\summary
The OpenMP runtime will invoke a callback with type signature  
\code{ompt\_callback\_buffer\_request\_t} to request a
buffer to store event records for a device.

\format
\vbox{
\begin{ccppspecific}
\begin{boxedcode}
typedef void (*ompt_callback_buffer_request_t) (
  uint64_t \plc{device_num},
  ompt_buffer_t **\plc{buffer},
  size_t *\plc{bytes}
);
\end{boxedcode}
\end{ccppspecific}
}

\descr
This callback requests a buffer to store trace records for the
specified device.

A buffer request callback may set \plc{*bytes} to 0 if it does not
want to provide a buffer for any reason. If a callback sets
\plc{*bytes} to 0, further recording of events for the device will be
disabled until the next invocation of \code{ompt\_start\_trace}.  This
will cause the device to drop future trace records until recording is
restarted.

The buffer request callback is not required to be \emph{async signal safe}.

\argdesc

The \callbackarg{} \plc{device\_num} specifies the device. 

A tool should set \plc{*buffer} to point to a buffer where device events
may be recorded and \plc{*bytes} to the length of that buffer.  

\crossreferences
\begin{itemize}
\item \code{ompt\_buffer\_t} type, see
\specref{sec:ompt_buffer_t}.
\end{itemize}

\omptcallbacksignature{\code{ompt\_callback\_buffer\_complete\_t}}
\index{ompt\_callback\_buffer\_complete\_t@{\code{ompt\_callback\_buffer\_complete\_t}}}
\label{sec:ompt_callback_buffer_complete_t}
\summary
A device triggers a call to \code{ompt\_callback\_buffer\_complete\_t} when no further records will be recorded in an event buffer and all records written to the buffer are valid. 

\format
\vbox{
\begin{ccppspecific}
\begin{boxedcode}
typedef void (*ompt_callback_buffer_complete_t) (
  uint64_t \plc{device_num},
  ompt_buffer_t *\plc{buffer},
  size_t \plc{bytes},
  ompt_buffer_cursor_t \plc{begin},
  int \plc{buffer_owned}
);
\end{boxedcode}
\end{ccppspecific}
}

\descr 
This callback provides a tool with a buffer containing trace records for the
specified device. Typically, a tool will iterate through the records
in the buffer and process them.

The OpenMP implementation will make these callbacks
on a thread that is not an OpenMP master or worker. 

The callee may delete the buffer if the \callbackarg{}
\plc{buffer\_owned}=0.

The buffer completion callback is not 
required to be \emph{async signal safe}.

\argdesc

The \callbackarg{} \plc{device\_num} indicates the device whose events
the buffer contains.

The \callbackarg{} \plc{buffer} is the address of a buffer previously
allocated by a \emph{buffer request} callback.

The \callbackarg{} \plc{bytes} indicates the full size of the buffer.

The \callbackarg{} \plc{begin} is an opaque cursor that indicates the
position at the beginning of the first record in the buffer.

The \callbackarg{} \plc{buffer\_owned} is 1 if the data pointed to by
buffer can be deleted by the callback and 0 otherwise. If multiple
devices accumulate trace events into a single buffer, this callback
might be invoked with a pointer to one or more trace records in a
shared buffer with \plc{buffer\_owned} = 0. In this case, the callback
may not delete the buffer.

\crossreferences
\begin{itemize}
\item \code{ompt\_buffer\_t} type, see
\specref{sec:ompt_buffer_t}.
\item \code{ompt\_buffer\_cursor\_t} type, see
\specref{sec:ompt_buffer_cursor_t}.
\end{itemize}

\omptcallbacksignature{\code{ompt\_callback\_control\_tool\_t}}
\index{ompt\_callback\_control\_tool\_t@{\code{ompt\_callback\_control\_tool\_t}}}
\label{sec:ompt_callback_control_tool_t}
\format
\vbox{
\begin{ccppspecific}
\begin{boxedcode}
typedef int (*ompt_callback_control\_tool_t) (
  \longlongint{} \plc{command},
  \longlongint{} \plc{modifier},
  void *\plc{arg},
  const void *\plc{codeptr_ra}
);
\end{boxedcode}
\end{ccppspecific}
}

\descr

The tool control callback may return any non-negative value, which will be returned to the 
application by the OpenMP implementation as the return value of the 
\code{omp\_control\_tool} call that triggered the callback.

\argdesc 

The \callbackarg{} \plc{command} passes a command from an application
to a tool.  Standard values for \plc{command} are defined by
\code{omp\_control\_tool\_t}.  defined in \specref{sec:control_tool}.

The \callbackarg{} \plc{modifier} passes a command modifier from an
application to a tool.

This callback allows tool-specific values for \plc{command} and
\plc{modifier}.  Tools must ignore \plc{command} values that they are
not explicitly designed to handle.

The \callbackarg{} \plc{arg} is a void pointer that enables a tool and
an application to pass arbitrary state back and forth. 
The \callbackarg{} \plc{arg} may be \code{NULL}.

\codeptrdesc


\constraints
Tool-specific values for \plc{command} must be $\geq$ 64.

\crossreferences
\begin{itemize}
\item \code{omp\_control\_tool\_t} enumeration type, see \specref{sec:control_tool}.
\end{itemize}

\omptcallbacksignature{\code{ompt\_callback\_cancel\_t}}
\index{ompt\_callback\_cancel\_t@{\code{ompt\_callback\_cancel\_t}}}
\label{sec:ompt_callback_cancel_t}
\format
\vbox{
\begin{ccppspecific}
\begin{boxedcode}
typedef void (*ompt_callback_cancel_t) (
  ompt_data_t *\plc{task_data},
  int \plc{flags},
  const void *\plc{codeptr_ra}
  );
\end{boxedcode}
\end{ccppspecific}
}

\argdesc 

The \callbackarg{} \plc{task\_data} corresponds to the task
encountering a \code{cancel} construct, a \code{cancellation point}
construct, or a construct defined as having an implicit cancellation
point.

The \callbackarg{} \plc{flags}, defined by the enumeration
\code{ompt\_cancel\_flag\_t}, indicates whether the cancel is
activated by the current task, or detected as being activated by
another task.  The construct being canceled is also described in the
\plc{flags}. When several constructs are detected as being
concurrently canceled, each corresponding bit in the flags will be
set.

\codeptrdesc

\crossreferences
\begin{itemize}
\item \code{omp\_cancel\_flag\_t} enumeration type, see \specref{sec:ompt_cancel_flag_t}.
\end{itemize}


\omptcallbacksignature{\code{ompt\_callback\_device\_initialize\_t}}
\index{ompt\_callback\_flush\_t@{\code{ompt\_callback\_device\_initialize\_t}}}
\label{sec:ompt_callback_device_initialize_t}

\summary The tool callback with type signature
\code{ompt\_callback\_device\_initialize\_t} initializes a
tool's tracing interface for a device.

\format
\vbox{
\begin{ccppspecific}
\begin{boxedcode}
typedef void (*ompt_callback_device_initialize_t) (
  uint64_t \plc{device_num},
  const char *\plc{type},
  ompt_device_t *\plc{device},
  ompt_function_lookup_t \plc{lookup},
  const char *\plc{documentation}
);
\end{boxedcode}
\end{ccppspecific}
}

\descr 

A tool that wants to asynchronously collect a trace of
activities on a device should register a callback with type signature
\code{ompt\_callback\_device\_initialize\_t} for the
\code{ompt\_callback\_device\_initialize} OpenMP event. An OpenMP
implementation will invoke this callback for a device after OpenMP is
initialized for the device but before beginning execution of any
OpenMP construct on the device.

\argdesc

The \callbackarg{} \plc{device\_num} identifies the logical device
being initialized.

The \callbackarg{} \plc{type} is a character string indicating the
type of the device. A device type string is a semicolon separated
character string that includes at a minimum the vendor and model name
of the device. This may be followed by a semicolon-separated sequence
of properties that describe a device's hardware or software.

\devicedesc

The \callbackarg{} \plc{lookup} is a pointer to a runtime callback
that a tool must use to obtain pointers to runtime entry points in the
device's OMPT tracing interface. If a device does not support tracing,
it should provide \code{NULL} for \plc{lookup}.

The \callbackarg{} \plc{documentation} is a string that describes
how to use any device-specific runtime
entry points that can be obtained using \plc{lookup}. This
documentation string could simply be a pointer to external
documentation, or it could be inline descriptions 
that includes names and type signatures for any
device-specific interfaces that are available through \plc{lookup}
along with descriptions of how to use these interface functions to
control monitoring and analysis of device traces.

\constraints
The \callbackarg{}s \plc{type} and \plc{documentation} must be
immutable strings that are defined for the lifetime of a program
execution.

\effect

A tool's device initializer has several duties.  First, it should use
\plc{type} to determine whether the tool has any special knowledge
about a device's hardware and/or software.  Second, it should use
\plc{lookup} to look up pointers to runtime entry points in the OMPT tracing
interface for the device.  Finally, using these runtime entry points, it can
then set up tracing for a device.

Initializing tracing for a target device is described in section
\specref{sec:tracing-device-activity}. 

\crossreferences
\begin{itemize}
\item \code{ompt\_function\_lookup\_t}, see
  \specref{sec:ompt_function_lookup_t}.
\end{itemize}


\omptcallbacksignature{\code{ompt\_callback\_device\_finalize\_t}}
\index{ompt\_callback\_flush\_t@{\code{ompt\_callback\_device\_finalize\_t}}}
\label{sec:ompt_callback_device_finalize_t}

\summary The tool callback with type signature
\code{ompt\_callback\_device\_finalize\_t} finalizes a
tool's tracing interface for a device.

\format
\vbox{
\begin{ccppspecific}
\begin{boxedcode}
typedef void (*ompt_callback_device_finalize_t) (
  uint64_t \plc{device_num}
);
\end{boxedcode}
\end{ccppspecific}
}

\argdesc

The \callbackarg{} \plc{device\_num} identifies the logical device
being finalized.

\descr 

A device finalization callback for a tool occurs after all tracing activity 
on a device is complete and immediately prior to the finalization of the device by the OpenMP implementation.





% This is an included file. See the master file for more information.
%
% When editing this file:
%
%    1. To change formatting, appearance, or style, please edit openmp.sty.
%
%    2. Custom commands and macros are defined in openmp.sty.
%
%    3. Be kind to other editors -- keep a consistent style by copying-and-pasting to
%       create new content.
%
%    4. We use semantic markup, e.g. (see openmp.sty for a full list):
%         \code{}     % for bold monospace keywords, code, operators, etc.
%         \plc{}      % for italic placeholder names, grammar, etc.
%
%    5. There are environments that provide special formatting, e.g. language bars.
%       Please use them whereever appropriate.  Examples are:
%
%         \begin{fortranspecific}
%         This is text that appears enclosed in blue language bars for Fortran.
%         \end{fortranspecific}
%
%         \begin{note}
%         This is a note.  The "Note -- " header appears automatically.
%         \end{note}
%
%    6. Other recommendations:
%         Use the convenience macros defined in openmp.sty for the minor headers
%         such as Comments, Syntax, etc.
%
%         To keep items together on the same page, prefer the use of
%         \begin{samepage}.... Avoid \parbox for text blocks as it interrupts line numbering.
%         When possible, avoid \filbreak, \pagebreak, \newpage, \clearpage unless that's
%         what you mean. Use \needspace{} cautiously for troublesome paragraphs.
%
%         Avoid absolute lengths and measures in this file; use relative units when possible.
%         Vertical space can be relative to \baselineskip or ex units. Horizontal space
%         can be relative to \linewidth or em units.
%
%         Prefer \emph{} to italicize terminology, e.g.:
%             This is a \emph{definition}, not a placeholder.
%             This is a \plc{var-name}.
%


\subsection{OMPT Runtime Entry Points for Tools}
\label{sec:entry-points}

The OMPT interface supports two principal sets of runtime entry points for tools. One
set of runtime entry points enables a tool to register callbacks for OpenMP
events and to inspect the state of an OpenMP thread while
executing in a tool callback or a signal handler. The second set of runtime entry points enables a
tool to trace activities on a device. When directed by the tracing
interface, an OpenMP implementation will trace activities on a device, collect
buffers full of trace records, and invoke callbacks on the host to
process these records.
Runtime entry points for tools in an OpenMP implementation
should not be global symbols since tools cannot rely on the visibility
of such symbols in general.

In addition, the OMPT interface supports
runtime entry points for two classes of lookup routines. The first
class of lookup routines contains a single member: a
routine that returns runtime entry points in the OMPT callback interface.
The second class of lookup routines includes
a unique lookup routine for each kind of
device that can return runtime entry points in a device's OMPT tracing interface.

\subsubsection{Entry Points in the OMPT Callback Interface}
\label{sec:ompt-callback-entry-points}

Entry points in the OMPT callback interface enable a tool to register
callbacks for OpenMP events and to inspect the state of an OpenMP thread while
executing in a tool callback or a signal handler.
A tool obtains pointers to these runtime entry points
using the lookup function passed to the tool's initializer for the
callback interface.

\subsubsubsection{\hcode{ompt_enumerate_states_t}}
\label{sec:ompt_enumerate_states_t}
\label{sec:ompt_enumerate_states}

\summary
A runtime entry point known as \code{ompt_enumerate_states}
with type signature \code{ompt_enumerate_states_t}
enumerates the thread states supported by an OpenMP
implementation.

\format


\begin{ccppspecific}
\begin{omptInquiry}
typedef int (*ompt_enumerate_states_t) (
  int \plc{current_state},
  int *\plc{next_state},
  const char **\plc{next_state_name}
);
\end{omptInquiry}
\end{ccppspecific}


\descr
An OpenMP implementation may support only a subset of the states defined by
the \code{omp_state_t} enumeration type. In addition, an
OpenMP implementation may support implementation-specific states.
The \code{ompt_enumerate_states} runtime entry point enables a tool to
enumerate the thread states supported by an OpenMP implementation.

When a thread state supported by an OpenMP implementation is passed
as the first argument to the runtime entry point,
the runtime entry point will assign the next thread state in the enumeration to
the variable passed by reference as the runtime entry point's second argument
and assign the name associated with the next thread state
to the character pointer passed by reference as the third argument.

Whenever one or more states are left in the enumeration,
the enumerate states runtime entry point will return $1$.
When the last state in the enumeration is passed
as the first argument, the runtime entry point will return $0$
indicating that the enumeration is complete.



\argdesc

The argument \plc{current_state} must be a thread state
supported by the OpenMP implementation.  To begin enumerating the
states that an OpenMP implementation supports, a tool should pass
\code{omp_state_undefined} as \plc{current_state}.  Subsequent
invocations of the runtime entry point by the tool should pass the
value assigned to the variable passed by reference as the second
argument to the previous call.

The argument \plc{next_state} is a pointer to an integer where
the entry point will return the value of the next state in the
enumeration.

The argument \plc{next_state_name} is a pointer to a
character string pointer, where the entry point will return a string
describing the next state.

\constraints
Any string returned through the argument
\plc{next_state_name} must be immutable and defined
for the lifetime of a program execution.

\crossreferences
\begin{itemize}
\item \code{omp_state_t}, see \specref{sec:thread-states}.
\end{itemize}

\subsubsubsection{\hcode{ompt_enumerate_mutex_impls_t}}
\label{sec:ompt_enumerate_mutex_impls_t}
\label{sec:ompt_enumerate_mutex_impls}
\label{sec:ompt_mutex_impl_none}

\summary

A runtime entry point known as \code{ompt_enumerate_mutex_impls}
with type signature \code{ompt_enumerate_mutex_impls_t}
enumerates the kinds of mutual exclusion implementations that
an OpenMP implementation employs.


\format


\begin{ccppspecific}
\begin{omptInquiry}
typedef int (*ompt_enumerate_mutex_impls_t) (
  int \plc{current_impl},
  int *\plc{next_impl},
  const char **\plc{next_impl_name}
);
\end{omptInquiry}
\end{ccppspecific}

\code{ompt_mutex_impl_none} is defined as an integer with the value 0.

\descr
An OpenMP implementation may implement mutual exclusion for locks,
nest locks, critical sections, and atomic regions in several different
ways.  The \code{ompt_enumerate_mutex_impls} runtime entry point
enables a tool to enumerate the
kinds of mutual exclusion implementations that an OpenMP implementation
employs.
The value \code{ompt_mutex_impl_none} is reserved to indicate an invalid
implementation.



When a mutex kind supported by an OpenMP implementation is passed
as the first argument to the runtime entry point,
the runtime entry point will assign the next mutex kind in the enumeration to
the variable passed by reference as the runtime entry point's second argument
and assign the name associated with the next mutex kind
to the character pointer passed by reference as the third argument.

Whenever one or more mutex kinds are left in the enumeration,
the runtime entry point to enumerate mutex implementations will return $1$.
When the last mutex kind in the enumeration is passed
as the first argument, the runtime entry point will return $0$
indicating that the enumeration is complete.

\argdesc

The argument \plc{current_impl} must be a mutex implementation
kind supported by an OpenMP implementation.  To begin enumerating the
mutex implementation kinds that an OpenMP implementation supports, a
tool should pass \code{ompt_mutex_impl_none} as the first
argument of the enumerate mutex kinds runtime entry point.  Subsequent
invocations of the runtime entry point by the tool should pass the
value assigned to the variable passed by reference as the second
argument to the previous call.

The argument \plc{next_impl} is a pointer to an integer where
the entry point will return the value of the next mutex implementation
in the enumeration.

The argument \plc{next_impl_name} is a pointer to a character
string pointer, where the entry point will return a string describing
the next mutex implementation.

\constraints
Any string returned through the argument
\plc{next_impl_name} must be immutable and defined
for the lifetime of a program execution.

\subsubsubsection{\hcode{ompt_set_callback_t}}
\label{sec:ompt_set_callback_t}
\label{sec:ompt_set_callback}

\summary

A runtime entry point known as \code{ompt_set_callback}
with type signature \code{ompt_set_callback_t} registers a
pointer to a tool callback that an OpenMP implementation will invoke when a host
OpenMP event occurs.

\format

\begin{ccppspecific}
\begin{omptCallback}
typedef int (*ompt_set_callback_t) (
  ompt_callbacks_t \plc{which},
  ompt_callback_t \plc{callback}
);
\end{omptCallback}
\end{ccppspecific}


\descr

OpenMP implementations can inform tools about events that occur during
the execution of an OpenMP program using callbacks.
To register a tool callback for an OpenMP event on the current device,
a tool uses the runtime entry point
known as \code{ompt_set_callback}
with type signature \code{ompt_set_callback_t}.

The return value of the \code{ompt_set_callback} runtime entry point may indicate several possible
outcomes. Callback registration may fail if it is called outside the initializer for the
callback interface, returning \code{omp_set_error}.
Otherwise, the return value of \code{ompt_set_callback}
indicates whether \emph{dispatching} a callback leads to its invocation.
A return value of \code{ompt_set_never} indicates that the callback
will never be invoked at runtime.
A return value of \code{ompt_set_sometimes} indicates that the callback
will be invoked at runtime for an implementation-defined subset of
associated event occurrences.
A return value of \code{ompt_set_sometimes_paired} is similar to
\code{ompt_set_sometimes}, but provides an additional guarantee for
callbacks with an \plc{endpoint} parameter. Namely, it guarantees that a callback
with an \plc{endpoint} value of \code{ompt_scope_begin} is invoked if and only if
the same callback with \plc{endpoint} value of \code{ompt_scope_end} will
also be invoked sometime in the future.
A return value of \code{ompt_set_always} indicates that the callback
will be always invoked at runtime for associated event occurrences.

\argdesc

The argument \plc{which} indicates the callback being registered.

The argument \plc{callback} is a tool callback function.

A tool may pass a \code{NULL} value for \plc{callback} to disable
any callback associated with \plc{which}. If disabling was successful,
\code{ompt_set_always} is returned.

\constraints
When a tool registers a callback for an event, the type
signature for the callback must match the type signature appropriate for the
event.

\begin{table}
\caption{Return codes for \code{ompt_set_callback} and
    \code{ompt_set_trace_ompt}.\label{table:ToolsSupport_set_rc}}
\begin{omptEnum}
typedef enum ompt_set_result_t {
  ompt_set_error            = 0,
  ompt_set_never            = 1,
  ompt_set_sometimes        = 2,
  ompt_set_sometimes_paired = 3,
  ompt_set_always           = 4
} ompt_set_result_t;
\end{omptEnum}
\vspace*{1ex}
\end{table}

\crossreferences
\begin{itemize}
\item \code{ompt_callbacks_t} enumeration type, see \specref{sec:ompt_callbacks_t}.
\item \code{ompt_callback_t} type, see \specref{sec:ompt_callback_t}.
\item \code{ompt_get_callback_t} host callback type signature,
see \specref{sec:ompt_get_callback_t}.
\end{itemize}

\subsubsubsection{\hcode{ompt_get_callback_t}}
\label{sec:ompt_get_callback_t}
\label{sec:ompt_get_callback}

\summary

A runtime entry point known as \code{ompt_get_callback}
with type signature \code{ompt_get_callback_t} retrieves a pointer
to a tool callback routine (if any)
that an OpenMP implementation will invoke when an OpenMP event occurs.

\format

\begin{ccppspecific}
\begin{omptCallback}
typedef int (*ompt_get_callback_t) (
  ompt_callbacks_t \plc{which},
  ompt_callback_t *\plc{callback}
);
\end{omptCallback}
\end{ccppspecific}


\descr
A tool uses the runtime entry point known
as \code{ompt_get_callback}
with type signature \code{ompt_get_callback_t}
to obtain a pointer to the tool callback that
an OpenMP implementation will invoke when a host OpenMP event occurs.
If a non-\code{NULL} tool callback is registered for the specified event,
the pointer to the tool callback will be assigned to the variable
passed by reference as the second argument and the entry
point will return 1; otherwise, it will return 0. If the entry point
returns 0, the value of the variable passed by reference as the second
argument is undefined.

\argdesc

The argument \plc{which} indicates the callback being inspected.

The argument \plc{callback} returns a pointer to the callback being inspected.

\constraints
The second argument passed to the entry point must be a reference
to a variable of specified type.

\crossreferences
\begin{itemize}
\item \code{ompt_callbacks_t} enumeration type, see \specref{sec:ompt_callbacks_t}.
\item \code{ompt_callback_t} type, see \specref{sec:ompt_callback_t}.
\item \code{ompt_set_callback_t} type signature,
see \specref{sec:ompt_set_callback_t}.
\end{itemize}


\subsubsubsection{\hcode{ompt_get_thread_data_t}}
\label{sec:ompt_get_thread_data_t}
\label{sec:ompt_get_thread_data}

\summary
A runtime entry point known as \code{ompt_get_thread_data}
with type signature \code{ompt_get_thread_data_t}
returns the address of the thread data object for the current thread.

\format
\begin{ccppspecific}
\begin{omptInquiry}
typedef ompt_data_t *(*ompt_get_thread_data_t) (void);
\end{omptInquiry}
\end{ccppspecific}

\descr

Each OpenMP thread has an associated thread data object of type
\code{ompt_data_t}.
A tool uses the runtime entry point known as
\code{ompt_get_thread_data}
with type signature \code{ompt_get_thread_data_t}
to obtain a pointer to the thread data object, if any, associated with the
current thread. If the current thread is unknown to the OpenMP
runtime, the entry point returns \code{NULL}.

A tool may use a pointer to an OpenMP thread's data object
obtained from this runtime entry point to
inspect or modify the value of the data object.
When an OpenMP thread is created, its data object will be initialized
with value \code{ompt_data_none}.

This runtime entry point is \emph{async signal safe}.

\crossreferences
\begin{itemize}
\item \code{ompt_data_t} type, see \specref{sec:ompt_data_t}.
\end{itemize}


\subsubsubsection{\hcode{ompt_get_num_procs_t}}
\label{sec:ompt_get_num_procs_t}

\summary

A runtime entry point known as
\code{ompt_get_num_procs} with type signature
\code{ompt_get_num_procs_t}  returns
the number of processors currently available to the execution
environment on the host device.

\format

\begin{ccppspecific}
\begin{omptInquiry}
typedef int (*ompt_get_num_procs_t) (void);
\end{omptInquiry}
\end{ccppspecific}


\binding

The binding thread set for runtime entry point known as
\code{ompt_get_num_procs} is all threads on a device. The effect
of executing this routine is not related to any specific region
corresponding to any construct or API routine.

\descr
The \code{ompt_get_num_procs} runtime entry point returns the
number of processors that are available on the host device at the time
the routine is called. This value may change between the time that
it is determined and the time that it is read in the calling context due to
system actions outside the control of the OpenMP implementation.

This runtime entry point is \emph{async signal safe}.

%%%%%%%%%%%%%%%%%%%%%%%%%%%%%%%%%%%%%
%% Alex new proc_bind calls

\subsubsubsection{\hcode{ompt_get_num_places_t}}
\label{sec:ompt_get_num_places_t}
\label{sec:ompt_get_num_place}

\summary

A runtime entry point known as
\code{ompt_get_num_places} with type signature
\code{ompt_get_num_places_t}  returns
the number of places available to the execution
environment in the place list.

\format

\begin{ccppspecific}
\begin{omptInquiry}
typedef int (*ompt_get_num_places_t) (void);
\end{omptInquiry}
\end{ccppspecific}


\binding

The binding thread set for the region
of the runtime entry point known as \code{ompt_get_num_places}
is all threads on a device. The effect of executing this
routine is not related to any specific region corresponding
to any construct or API routine.

\descr

The runtime entry point known as  \code{ompt_get_num_places}
returns the number of places in the place list.
This value is equivalent to the number of places in
the  \plc{place-partition-var} ICV in the execution environment
of the initial task.

This runtime entry point is \emph{async signal safe}.

\crossreferences
\begin{itemize}
\item \plc{place-partition-var} ICV, see
\specref{sec:Internal Control Variables}.

\item \code{OMP_PLACES} environment variable, see
\specref{sec:OMP_PLACES}.
\end{itemize}




\subsubsubsection{\hcode{ompt_get_place_proc_ids_t}}
\label{sec:ompt_get_place_proc_ids_t}
\label{sec:ompt_get_place_proc_ids}

\summary

A runtime entry point known as
\code{ompt_get_place_proc_ids} with type signature
\code{ompt_get_place_proc_ids_t}
returns the numerical identifiers of the processors
available to the execution environment in the specified place.

\format

\begin{ccppspecific}
\begin{omptInquiry}
typedef int (*ompt_get_place_proc_ids_t) (
  int \plc{place_num},
  int \plc{ids_size},
  int *\plc{ids}
);
\end{omptInquiry}
\end{ccppspecific}


\binding

The binding thread set for the region
of the runtime entry point known as \code{ompt_get_place_proc_ids}
is all threads on a device. The effect of executing this
routine is not related to any specific region corresponding
to any construct or API routine.

\descr

The runtime entry point known as
\code{ompt_get_place_proc_ids} with type signature
\code{ompt_get_place_proc_ids_t} returns
the numerical identifiers of each processor
associated with the specified place.
The numerical identifiers returned are non-negative, and
their meaning is implementation defined.

\argdesc

The argument \plc{place_num} specifies the place being
queried.

The argument \plc{ids_size} indicates the size of the result
array specified by argument \plc{ids}.

The argument \plc{ids} is an array where the routine can return
a vector of processor identifiers in the specified place.

\effect

If the array \plc{ids} of size \plc{ids_size} is large enough to
contain all identifiers, they are returned in \plc{ids} and
their order in the array is implementation defined.

Otherwise, if the \plc{ids} array is too small, the values in \plc{ids} when the function returns are unspecified.

In both cases, the routine returns the number of numerical identifiers
available to the execution environment in the specified place.

% This runtime entry point is \emph{async signal safe}.


\subsubsubsection{\hcode{ompt_get_place_num_t}}
\label{sec:ompt_get_place_num_t}
\label{sec:ompt_get_place_num}

\summary

A runtime entry point known as
\code{ompt_get_place_num} with type signature
\code{ompt_get_place_num_t} returns
the place number of the place to which the current
thread is bound.

\format

\begin{ccppspecific}
\begin{omptInquiry}
typedef int (*ompt_get_place_num_t) (void);
\end{omptInquiry}
\end{ccppspecific}


\binding

The binding thread set
of the runtime entry point known as \code{ompt_get_place_num}
is the current thread.

\descr

When the current thread is bound to a place,
the runtime entry point known as \code{ompt_get_place_num}
returns the place number associated with the thread.
The returned value is between 0 and one less than the value returned
by runtime entry point known as \code{ompt_get_num_places}, inclusive.
When the current thread is not bound to a place, the routine returns -1.

This runtime entry point is \emph{async signal safe}.


\subsubsubsection{\hcode{ompt_get_partition_place_nums_t}}
\label{sec:ompt_get_partition_place_nums_t}
\label{sec:ompt_get_partition_place_nums}

\summary

A runtime entry point known as
\code{ompt_get_partition_place_nums} with type signature
\code{ompt_get_partition_place_nums_t}
returns the list of place numbers corresponding to the places in the \plc{place-partition-var}
ICV of the innermost implicit task.

\format

\begin{ccppspecific}
\begin{omptInquiry}
typedef int (*ompt_get_partition_place_nums_t) (
  int \plc{place_nums_size},
  int *\plc{place_nums}
);
\end{omptInquiry}
\end{ccppspecific}


\binding

The binding task set for
the runtime entry point known as \code{ompt_get_partition_place_nums}
is the current implicit task.

\descr

The runtime entry point known as
\code{ompt_get_partition_place_nums} with type signature
\code{ompt_get_partition_place_nums_t} returns the list of place
numbers corresponding to the places in the \plc{place-partition-var}
ICV of the innermost implicit task.

This runtime entry point is \emph{async signal safe}.

\argdesc

The argument \plc{place_nums_size} indicates the size of the result
array specified by argument \plc{place_nums}.

The argument \plc{place_nums} is an array where the routine can return
a vector of place identifiers.

\effect

If the array \plc{place_nums} of size \plc{place_nums_size} is
large enough to contain all identifiers, they are returned in
\plc{place_nums} and their order in the array is implementation
defined.

Otherwise, if the \plc{place_nums} array is too small, the values in \plc{place_nums} when the function returns are unspecified.

In both cases, the routine returns the number of places in the
\plc{place-partition-var} ICV of the innermost implicit task.

\crossreferences
\begin{itemize}
\item \plc{place-partition-var} ICV, see
\specref{sec:Internal Control Variables}.

\item \code{OMP_PLACES} environment variable, see
\specref{sec:OMP_PLACES}.

\end{itemize}





\subsubsubsection{\hcode{ompt_get_proc_id_t}}
\label{sec:ompt_get_proc_id_t}
\label{sec:ompt_get_proc_id}

\summary

A runtime entry point known as
\code{ompt_get_proc_id} with type signature
\code{ompt_get_proc_id_t} returns the numerical identifier
of the processor of the current thread.

\format

\begin{ccppspecific}
\begin{omptInquiry}
typedef int (*ompt_get_proc_id_t) (void);
\end{omptInquiry}
\end{ccppspecific}


\binding

The binding thread set for
the runtime entry point known as \code{ompt_get_proc_id}
is the current thread.

\descr

The runtime entry point known as
\code{ompt_get_proc_id} returns the numerical identifier
of the processor of the current thread.
A defined numerical identifier is non-negative, and
its meaning is implementation defined.
A negative number indicates a failure to retrieve the numerical identifier.

This runtime entry point is \emph{async signal safe}.






\subsubsubsection{\hcode{ompt_get_state_t}}
\label{sec:ompt_get_state_t}
\label{sec:ompt_get_state}

\summary
A runtime entry point known as \code{ompt_get_state}
with type signature \code{ompt_get_state_t}
returns the state and the wait identifier of the
current thread.

\format
\begin{ccppspecific}
\begin{omptInquiry}
typedef int (*ompt_get_state_t) (
  omp_wait_id_t *\plc{wait_id}
);
\end{omptInquiry}
\end{ccppspecific}

\descr

Each OpenMP thread has an associated state and a wait identifier.  If
a thread's state indicates that the thread is waiting for mutual
exclusion, the thread's wait identifier will contain an opaque handle
that indicates the data object upon which the thread is waiting.

To retrieve the state and wait identifier for the current thread,
a tool uses the runtime entry point known as
\code{ompt_get_state} with type signature \code{ompt_get_state_t}.

The returned value may be any one of the states predefined by
\code{omp_state_t} or a value that represents any implementation 
specific state. 
The tool may obtain a string representation for each state with the
function known as \code{ompt_enumerate_states}.

If the returned state indicates that the thread is waiting for a
lock, nest lock, critical section, atomic region, or ordered region
the value of the thread's wait identifier will be assigned to a
non-\code{NULL} wait identifier passed as an argument.

This runtime entry point is \emph{async signal safe}.

\argdesc

The argument \plc{wait_id} is a pointer to an opaque handle
available to receive the value of the thread's wait identifier.  If
the \plc{wait_id} pointer is not \code{NULL}, the entry point
will assign the value of the thread's wait identifier
*\plc{wait_id}.  If the returned state is not one of the specified
wait states, the value of *\plc{wait_id} is undefined after the call.

\constraints
The argument passed to the entry point must be a reference
to a variable of the specified type or \code{NULL}.

\crossreferences
\begin{itemize}
\item \code{omp_wait_id_t} type, see \specref{sec:omp_wait_id_t}.
\item \code{omp_state_t} type, see \specref{sec:omp_state_t}.
\item \code{ompt_enumerate_states_t} type, see \specref{sec:ompt_enumerate_states_t}.
\end{itemize}


\subsubsubsection{\hcode{ompt_get_parallel_info_t}}
\label{sec:ompt_get_parallel_info_t}
\label{sec:ompt_get_parallel_info}

\summary

A runtime entry point known as \code{ompt_get_parallel_info}
with type signature \code{ompt_get_parallel_info_t}
returns information about
the parallel region, if any, at the specified ancestor level
for the current execution context.

\format
\begin{ccppspecific}
\begin{omptInquiry}
typedef int (*ompt_get_parallel_info_t) (
  int \plc{ancestor_level},
  ompt_data_t **\plc{parallel_data},
  int *\plc{team_size}
);
\end{omptInquiry}
\end{ccppspecific}

\descr
During execution, an OpenMP program may employ nested parallel
regions.
To obtain information about a parallel region,
a tool uses the runtime entry point known as
\code{ompt_get_parallel_info}
with type signature \code{ompt_get_parallel_info_t}.
This runtime entry point
can be used to obtain information about the current parallel region,
if any, and any enclosing parallel regions
for the current execution context.

The entry point returns 2 if there is a parallel region at the
specified ancestor level and the information is available,
1 if there is a parallel region at the specified ancestor level
but the information is currently unavailable, and 0 otherwise.

A tool may use the pointer to a parallel region's data object that it
obtains from this runtime entry point to inspect or modify the value
of the data object.  When a parallel region is created, its data
object will be initialized with the value \code{ompt_data_none}.

This runtime entry point is \emph{async signal safe}.

\argdesc

The argument \plc{ancestor_level} specifies the parallel region
of interest to a tool by its ancestor level.  Ancestor level 0 refers
to the innermost parallel region; information about enclosing parallel
regions may be obtained using larger ancestor levels.

If a parallel region exists at the specified ancestor level
and the information is currently available,
information will be returned in the variables \plc{parallel_data} and
\plc{team_size} passed by reference to the entry point.
Specifically, a reference to the parallel region's associated data
object will be assigned to *\plc{parallel_data} and the number of
threads in the parallel region's team will be assigned to
*\plc{team_size}.

If no enclosing parallel region exists at the specified ancestor
level, or the information is currently unavailable,
the values of variables passed by reference
*\plc{parallel_data} and *\plc{team_size} will be undefined when the
entry point returns.

\constraints
While argument \plc{ancestor_level} is passed by
value, all other arguments to the entry point must be references
to variables of the specified types.


\restrictions
Between a \emph{parallel-begin} event and an \emph{implicit-task-begin}
event, a call to \code{ompt_get_parallel_info(0,...)} may return
information about the outer parallel team, the new parallel team or an
inconsistent state.

If a thread is in the state \code{omp_state_wait_barrier_implicit_parallel},
a call to \code{ompt_get_parallel_info}
may return a pointer to a copy of the specified parallel region's \plc{parallel_data}
rather than a pointer to the data word for the region itself. This convention enables the master thread
for a parallel region to free storage for the region immediately after the region ends, yet
avoid having some other thread in the region's team
potentially reference the region's \plc{parallel_data} object after it has been freed.

\crossreferences
\begin{itemize}
\item \code{ompt_data_t} type, see \specref{sec:ompt_data_t}.
\end{itemize}

\subsubsubsection{\hcode{ompt_get_task_info_t}}
\label{sec:ompt_get_task_info_t}
\label{sec:ompt_get_task_info}

\summary
A runtime entry point known as \code{ompt_get_task_info}
with type signature \code{ompt_get_task_info_t} provides information about the
task, if any, at the specified ancestor level in the current execution
context.

\format
\begin{ccppspecific}
\begin{omptInquiry}
typedef int (*ompt_get_task_info_t) (
  int \plc{ancestor_level},
  int *\plc{type},
  ompt_data_t **\plc{task_data},
  omp_frame_t **\plc{task_frame},
  ompt_data_t **\plc{parallel_data},
  int *\plc{thread_num}
);
\end{omptInquiry}
\end{ccppspecific}

\descr
During execution, an OpenMP thread may be executing an OpenMP task.
Additionally, the thread's stack may contain
procedure frames associated with suspended OpenMP tasks or
OpenMP runtime system routines.
To obtain information about any task on the current thread's stack,
a tool uses the runtime entry point known as
\code{ompt_get_task_info}
with type signature \code{ompt_get_task_info_t}.

Ancestor level 0 refers to the active task; information about
other tasks with associated frames present on the stack in the current execution context may be queried at
higher ancestor levels.

The \code{ompt_get_task_info} runtime entry point
returns 2 if there is a task region at the
specified ancestor level
and the information is available, 1 if there is a task region at the
specified ancestor level but the information is currently unavailable,
and 0 otherwise.

If a task exists at the specified ancestor level and the information is available,
information will be returned in the variables passed by reference to the entry
point.  If no task region exists at the specified ancestor level
or the information is unavailable,
the values of variables passed by reference to the entry point will be
undefined when the entry point returns.

A tool may use a pointer to a data object for a task or parallel
region that it obtains from this runtime entry point to inspect or modify the
value of the data object.  When either a parallel region or a task
region is created, its data object will be initialized with the value
\code{ompt_data_none}.

This runtime entry point is \emph{async signal safe}.

\argdesc

The argument \plc{ancestor_level} specifies the task region
of interest to a tool by its ancestor level.  Ancestor level 0 refers
to the active task; information about
ancestor tasks found in the current execution context may be queried at
higher ancestor levels.

The argument \plc{type} returns the task type
if the argument is not \code{NULL}.

The argument \plc{task_data} returns the task data
if the argument is not \code{NULL}.

The argument \plc{task_frame} returns the task frame
pointer
if the argument is not \code{NULL}.

The argument \plc{parallel_data} returns the parallel data
if the argument is not \code{NULL}.

The argument \plc{thread_num} returns the thread number
if the argument is not \code{NULL}.

\effect


If the runtime entry point returns 0 or 1, no argument is modified.
Otherwise, the entry point has the effects described below.

If a non-\code{NULL} value was passed for \plc{type},
the value returned in *\plc{type} represents the type of the task
at the specified level.
Task types that a tool may observe on a thread's stack include
initial, implicit, explicit, and target tasks.

If a non-\code{NULL} value was passed for \plc{task_data},
the value returned in *\plc{task_data} is a pointer to a data word
associated with the task at the specified level.

If a non-\code{NULL} value was passed for \plc{task_frame},
the value returned in *\plc{task_frame} is a pointer to the
\code{omp_frame_t} structure associated with the task at the specified level.
Appendix~\ref{chap:frames} discusses an example that
illustrates the use of \code{omp_frame_t} structures with multiple
threads and nested parallelism.

If a non-\code{NULL} value was passed for \plc{parallel_data},
the value returned in *\plc{parallel_data} is a pointer to a data word
associated with the parallel region containing the task at the specified level.
If the task at the specified level is an initial task,
the value of *\plc{parallel_data} will be \code{NULL}.

If a non-\code{NULL} value was passed for \plc{thread_num},
the value returned in *\plc{thread_num}
indicates the number of the thread in the parallel region executing the task.

\crossreferences
\begin{itemize}
\item \code{ompt_data_t} type, see \specref{sec:ompt_data_t}.
\item \code{omp_frame_t} type, see \specref{sec:omp_frame_t}.
\item \code{ompt_task_t} type, see
  \specref{sec:ompt_task_t}.
\end{itemize}

\subsubsubsection{\hcode{ompt_get_target_info_t}}
\label{sec:ompt_get_target_info_t}
\label{sec:ompt_get_target_info}

\summary
A runtime entry point known as \code{ompt_get_target_info}
with type signature \code{ompt_get_target_info_t} returns identifiers that specify a
thread's current target region and target operation id, if any.

\format
\begin{ccppspecific}
\begin{omptInquiry}
typedef int (*ompt_get_target_info_t) (
  uint64_t *\plc{device_num},
  ompt_id_t *\plc{target_id},
  ompt_id_t *\plc{host_op_id}
);
\end{omptInquiry}
\end{ccppspecific}

\descr
A tool can query whether an OpenMP thread is in a target region by
invoking the entry point known as \code{ompt_get_target_info}
with type signature \code{ompt_get_target_info_t}.
This runtime entry point returns 1 if the current thread is
in a target region and 0 otherwise. If the entry point returns 0,
the values of the variables passed by reference as its arguments
are undefined.

If the current thread is in a target region, the entry point will
return information about the current device, active target region, and
active host operation, if any.

This runtime entry point is \emph{async signal safe}.

\argdesc

If the host is in a \code{target} region,
\plc{device_num} returns the target device.

If the host is in a \code{target} region,
\plc{target_id} returns the \code{target} region identifier.

If the current thread is in the process of initiating an
operation on a target device (e.g., copying data to or from an
accelerator or launching a kernel) \plc{host_op_id} returns
the identifier for the operation; otherwise,
\plc{host_op_id} returns \code{ompt_id_none}.

\constraints

Arguments passed to the entry point must be valid
references to variables of the specified types.

\crossreferences
\begin{itemize}
\item \code{ompt_id_t} type, see \specref{sec:ompt_id_t}.
\end{itemize}

\subsubsubsection{\hcode{ompt_get_num_devices_t}}
\label{sec:ompt_get_num_devices_t}
\label{sec:ompt_get_num_devices}

\summary
A runtime entry point known as \code{ompt_get_num_devices}
with type signature \code{ompt_get_num_devices_t}
returns the number of available devices.

\format
\begin{ccppspecific}
\begin{omptInquiry}
typedef int (*ompt_get_num_devices_t) (void);
\end{omptInquiry}
\end{ccppspecific}

\descr

An OpenMP program may execute on one or more devices.
A tool may determine the number of devices available to an OpenMP
program by invoking a runtime entry point
known as \code{ompt_get_num_devices}
with type signature \code{ompt_get_num_devices_t}.

This runtime entry point is \emph{async signal safe}.



\subsubsubsection{\hcode{ompt_get_unique_id_t}}
\label{sec:ompt_get_unique_id_t}
\label{sec:ompt_get_unique_id}

\summary
A runtime entry point known as \code{ompt_get_unique_id}
with type signature \code{ompt_get_unique_id_t}
returns a unique number.

\format
\begin{ccppspecific}
\begin{omptInquiry}
typedef uint64_t (*ompt_get_unique_id_t) (void);
\end{omptInquiry}
\end{ccppspecific}

\descr

A tool may obtain a number that is unique for the duration of an
OpenMP program by invoking a runtime entry point
known as \code{ompt_get_unique_id}
with type signature \code{ompt_get_unique_id_t}.
Successive invocations may not result in
consecutive or even increasing numbers.

This runtime entry point is \emph{async signal safe}.




\subsubsection{Entry Points in the OMPT Device Tracing Interface}
\label{sec:ompt-tracing-entry-points}


\subsubsubsection{\hcode{ompt_get_device_num_procs_t}}
\label{sec:ompt_get_device_num_procs_t}

\summary

A runtime entry point for a device known as
\code{ompt_get_device_num_procs} with type signature
\code{ompt_get_device_num_procs_t} returns
the number of processors currently available to the execution
environment on the specified device.

\format
\begin{ccppspecific}
\begin{omptInquiry}
typedef int (*ompt_get_device_num_procs_t) (
  ompt_device_t *\plc{device}
);
\end{omptInquiry}
\end{ccppspecific}

\descr
A runtime entry point for a device known
as \code{ompt_get_device_num_procs} with type signature
\code{ompt_get_device_num_procs_t} returns the
number of processors that are available on the device at the time
the routine is called. This value may change between the time that
it is determined and the time that it is read in the calling context due to
system actions outside the control of the OpenMP implementation.

\argdesc

The argument \plc{device} is a pointer to an opaque object that
represents the target device instance. The pointer to the device
instance object is used by functions in the device tracing interface
to identify the device being addressed.

\crossreferences
\begin{itemize}
\item \code{ompt_device_t},
see \specref{sec:ompt_device_t}.
\end{itemize}

\subsubsubsection{\hcode{ompt_get_device_time_t}}
\label{sec:ompt_get_device_time_t}

\summary
A runtime entry point for a device known
as \code{ompt_get_device_time}
with type signature \code{ompt_get_device_time_t}
returns the current time on the specified device.

\format
\begin{ccppspecific}
\begin{omptInquiry}
typedef ompt_device_time_t (*ompt_get_device_time_t) (
  ompt_device_t *\plc{device}
);
\end{omptInquiry}
\end{ccppspecific}

\descr
Host and target devices are typically distinct and run independently.
If host and target devices are different hardware components, they
may use different clock generators. For this reason,  there may be
no common time base for ordering host-side and device-side events.

A runtime entry point for a device known
as \code{ompt_get_device_time} with type signature
\code{ompt_get_device_time_t}
returns the current time on the specified device.
A tool can use this information
to align time stamps from different devices.

\argdesc

The argument \plc{device} is a pointer to an opaque object that
represents the target device instance. The pointer to the device
instance object is used by functions in the device tracing interface
to identify the device being addressed.

\crossreferences
\begin{itemize}
\item \code{ompt_device_t},
see \specref{sec:ompt_device_t}.
\item \code{ompt_device_time_t},
see \specref{sec:ompt_device_time_t}.
\end{itemize}

\subsubsubsection{\hcode{ompt_translate_time_t}}
\label{sec:ompt_translate_time_t}

\summary
A runtime entry point for a device known
as \code{ompt_translate_time}
with type signature \code{ompt_translate_time_t} translates
a time value obtained from the specified device to a corresponding time
value on the host device.

\format
\begin{ccppspecific}
\begin{omptInquiry}
typedef double (*ompt_translate_time_t) (
  ompt_device_t *\plc{device},
  ompt_device_time_t \plc{time}
);
\end{omptInquiry}
\end{ccppspecific}

\descr
A runtime entry point for a device known as \code{ompt_translate_time}
with type signature \code{ompt_translate_time_t} translates
a time value obtained from the specified device to a corresponding time
value on the host device. The returned value for the host time has
the same meaning as the value returned from \code{omp_get_wtime}.

\needspace{6\baselineskip}\begin{note}
The accuracy of time translations may degrade if they are not
performed promptly after a device time value is received if either
the host or device vary their clock speeds. Prompt translation of
device times to host times is recommended.
\end{note}

\argdesc

The argument \plc{device} is a pointer to an opaque object that
represents the target device instance. The pointer to the device
instance object is used by functions in the device tracing interface
to identify the device being addressed.

The argument \plc{time} is a time from the specified device.

\crossreferences
\begin{itemize}
\item \code{ompt_device_t},
see \specref{sec:ompt_device_t}.
\item \code{ompt_device_time_t},
see \specref{sec:ompt_device_time_t}.
\end{itemize}

\subsubsubsection{\hcode{ompt_set_trace_ompt_t}}
\label{sec:ompt_set_trace_ompt_t}

\summary
A runtime entry point for a device known as \code{ompt_set_trace_ompt}
with type signature \code{ompt_set_trace_ompt_t}
enables or disables the recording of trace records for one or more
types of OMPT events.

\format
\begin{ccppspecific}
\begin{omptInquiry}
typedef int (*ompt_set_trace_ompt_t) (
  ompt_device_t *\plc{device},
  unsigned int \plc{enable},
  unsigned int \plc{etype}
);
\end{omptInquiry}
\end{ccppspecific}

\argdesc

\devicedesc

The argument \plc{enable} indicates whether tracing should be
enabled or disabled for the event or events specified by
argument \plc{etype}. A positive value for \plc{enable}
indicates that recording of one or more events specified by
\plc{etype} should be enabled; a value of 0 for \plc{enable} indicates
that recording of events should be disabled by this invocation.

An argument \plc{etype} value 0 indicates that traces for all
event types will be enabled or disabled.  Passing a positive value for
\plc{etype} inidicates that recording should be enabled or disabled
for the event in \code{ompt_callbacks_t} that matches \plc{etype}.


\effect

Table~\ref{table:record_set} shows the possible return
codes for \code{ompt_set_trace_ompt}.  If a single invocation
of \code{ompt_set_trace_ompt} is used to enable or disable
more than one event (i.e., \code{etype}=0), the return code will
be 3 if tracing is possible for one or more events but not for
others.

\nolinenumbers
\renewcommand{\arraystretch}{1.5}
\tablefirsthead{%
\hline
\textsf{\textbf{Return Code}} & \textsf{\textbf{Meaning}}\\
\hline\\[-3ex]
}
\tablehead{%
\multicolumn{2}{l}{\small\slshape table continued from previous page}\\
\hline
\textsf{\textbf{Return Code}} & \textsf{\textbf{Meaning}}\\
\hline\\[-3ex]
}
\tabletail{%
\hline\\[-4ex]
\multicolumn{2}{l}{\small\slshape table continued on next page}\\
}
\tablelasttail{\hline}
\tablecaption{Meaning of return codes for \code{ompt_set_trace_ompt} and
    \code{ompt_set_trace_native}.\label{table:record_set}}
\begin{supertabular}{p{2.0in} p{3.0in}}
0 & error\\
1 & event will never occur\\
2 & event may occur but no tracing is possible\\
3 & event may occur and will be traced when convenient\\
4 & event may occur and will always be traced if event occurs\\
\end{supertabular}

\linenumbers

\crossreferences
\begin{itemize}
\item \code{ompt_callbacks_t},
see \specref{sec:ompt_callbacks_t}.
\item \code{ompt_device_t},
see \specref{sec:ompt_device_t}.
\end{itemize}

\subsubsubsection{\hcode{ompt_set_trace_native_t}}
\label{sec:ompt_set_trace_native_t}

\summary
A runtime entry point for a device known as \code{ompt_set_trace_native}
with type signature \code{ompt_set_trace_native_t}
enables or disables the recording of native trace records for a device.


\format
\begin{ccppspecific}
\begin{omptInquiry}
typedef int (*ompt_set_trace_native_t) (
  ompt_device_t *\plc{device},
  int \plc{enable},
  int \plc{flags}
);
\end{omptInquiry}
\end{ccppspecific}

\descr
This interface is designed for use by a tool with no knowledge about
an attached device. If a tool knows how to program a particular
attached device, it may opt to invoke native control functions
directly using pointers obtained through the \plc{lookup} function
associated with the device and described in the \plc{documentation}
string that is provided to the device initializer callback.

\argdesc
\devicedesc

The argument \plc{enable} indicates whether recording of events
should be enabled or disabled by this invocation.

The argument \plc{flags} specifies the kinds of native device
monitoring to enable or disable.
Each kind of monitoring is specified by a flag bit.
Flags can be composed by using logical {\ttfamily or}  to combine enumeration
values from type \code{ompt_native_mon_flags_t}.
Table~\ref{table:record_set} shows the possible return codes for
\code{ompt_set_trace_native}.  If a single invocation of
\code{ompt_set_trace_ompt} is used to enable/disable more
than one kind of monitoring, the return code will be 3 if tracing
is possible for one or more kinds of monitoring but not for others.

To start, pause, or stop tracing for a specific target device
associated with the handle \plc{device}, a tool calls the functions
\code{ompt_start_trace}, \code{ompt_pause_trace}, or
\code{ompt_stop_trace}.


\crossreferences
\begin{itemize}
\item \code{ompt_device_t},
see \specref{sec:ompt_device_t}.
\end{itemize}

\subsubsubsection{\hcode{ompt_start_trace_t}}
\label{sec:ompt_start_trace_t}

\summary
A runtime entry point for a device known as \code{ompt_start_trace}
with type signature \code{ompt_start_trace_t}
starts tracing of activity on a specific device.

\format
\begin{ccppspecific}
\begin{omptInquiry}
typedef int (*ompt_start_trace_t) (
  ompt_device_t *\plc{device},
  ompt_callback_buffer_request_t \plc{request},
  ompt_callback_buffer_complete_t \plc{complete}
);
\end{omptInquiry}
\end{ccppspecific}

\descr
A tool may initiate tracing on a device by invoking the device's \code{ompt_start_trace}
runtime entry point.

Under normal operating conditions, every event buffer provided to
a device by a tool callback will be returned to the tool
before the OpenMP runtime shuts down.
If an exceptional condition terminates  execution of an OpenMP
program, the OpenMP runtime may not return buffers provided to the
device.

An invocation of \code{ompt_start_trace} returns 1 if the command
succeeded and 0 otherwise.

\argdesc

\devicedesc

The argument \emph{buffer request} specifies a tool callback
that will supply a device with a buffer to deposit events.

The argument \emph{buffer complete} specifies a tool callback
that will be invoked by the OpenMP implmementation to empty a buffer
containing event records.

\crossreferences
\begin{itemize}
\item \code{ompt_device_t},
see \specref{sec:ompt_device_t}.
\item \code{ompt_callback_buffer_request_t},
see \specref{sec:ompt_callback_buffer_request_t}.
\item \code{ompt_callback_buffer_complete_t},
see \specref{sec:ompt_callback_buffer_complete_t}.
\end{itemize}

\subsubsubsection{\hcode{ompt_pause_trace_t}}
\label{sec:ompt_pause_trace_t}

\summary
A runtime entry point for a device known as \code{ompt_pause_trace}
with type signature \code{ompt_pause_trace_t}
pauses or restarts activity tracing on a specific device.

\begin{ccppspecific}
\begin{omptInquiry}
typedef int (*ompt_pause_trace_t) (
  ompt_device_t *\plc{device},
  int \plc{begin_pause}
);
\end{omptInquiry}
\end{ccppspecific}

\descr

A tool may pause or resume tracing on a device by invoking the device's
\code{ompt_pause_trace} runtime entry point.
An invocation of \code{ompt_pause_trace} returns 1 if the command
succeeded and 0 otherwise.

Redundant pause or resume commands are idempotent and will
return 1 indicating success.

\argdesc

\devicedesc

The argument \plc{begin_pause} indicates whether to pause or
resume tracing.
To resume tracing, zero should be supplied for \plc{begin_pause}.


\crossreferences
\begin{itemize}
\item \code{ompt_device_t},
see \specref{sec:ompt_device_t}.
\end{itemize}

\subsubsubsection{\hcode{ompt_stop_trace_t}}
\label{sec:ompt_stop_trace_t}

\summary
A runtime entry point for a device known as \code{ompt_stop_trace}
with type signature \code{ompt_stop_trace_t}
stops tracing for a device.

\begin{ccppspecific}
\begin{omptInquiry}
typedef int (*ompt_stop_trace_t) (
  ompt_device_t *\plc{device}
);
\end{omptInquiry}
\end{ccppspecific}

\descr

A tool may halt tracing on a device and request that the device flush any pending trace records
by invoking the \code{ompt_stop_trace} runtime entry point for the device.
An invocation of \code{ompt_stop_trace} returns 1 if the command
succeeded and 0 otherwise.

\argdesc

\devicedesc


%%? johnmc says: we should export one more function from a target device: ompt_target_scope(begin/end, host_id)
%% this function will be used by the OpenMP runtime when a target device is being monitored to signal the target device when
%% (1) entering and leaving a target region
%% (2) before and after launching a kernel
%% (3) before and after performing a data operation: copy, allocation, release, ...
%% this function will enable the target device to associate device_activity_ids with some host_id that either represents a
%% target region, target data operation, or target kernel submission

\crossreferences
\begin{itemize}
\item \code{ompt_device_t},
see \specref{sec:ompt_device_t}.
\end{itemize}

\subsubsubsection{\hcode{ompt_advance_buffer_cursor_t}}
\label{sec:ompt_advance_buffer_cursor_t}

%There are several routines that need to be used together to process %target event records deposited in a buffer by a device.

\summary
A runtime entry point for a device known as \code{ompt_advance_buffer_cursor}
with type signature \code{ompt_advance_buffer_cursor_t}
advances a trace buffer cursor to the next record.

\format
\begin{ccppspecific}
\begin{omptInquiry}
typedef int (*ompt_advance_buffer_cursor_t) (
  ompt_buffer_t *\plc{buffer},
  size_t \plc{size},
  ompt_buffer_cursor_t \plc{current},
  ompt_buffer_cursor_t *\plc{next}
);
\end{omptInquiry}
\end{ccppspecific}

\descr
It returns \plc{true} if the advance is successful and the next
position in the buffer is valid.

\argdesc

\devicedesc

The argument \plc{buffer} indicates a trace buffer associated
with the cursors.

The argument \plc{size} indicates the size of \plc{buffer} in
bytes.

The argument \plc{current} is an opaque buffer cursor.

The argument \plc{next} returns the next value of a opaque buffer cursor.


\crossreferences
\begin{itemize}
\item \code{ompt_device_t},
see \specref{sec:ompt_device_t}.
\item \code{ompt_buffer_cursor_t},
see \specref{sec:ompt_buffer_cursor_t}.
\end{itemize}

\subsubsubsection{\hcode{ompt_get_record_type_t}}
\label{sec:ompt_get_record_type_t}

\summary
A runtime entry point for a device known as
\code{ompt_get_record_type} with type signature
\code{ompt_get_record_type_t} inspects the type
of a trace record for a device.

\format
\begin{ccppspecific}
\begin{omptInquiry}
typedef ompt_record_t (*ompt_get_record_type_t) (
  ompt_buffer_t *\plc{buffer},
  ompt_buffer_cursor_t \plc{current}
);
\end{omptInquiry}
\end{ccppspecific}

\descr

Trace records for a device may be in one of two forms: a
\emph{native} record format, which may be device-specific,
or an \emph{OMPT} record format, where each trace record
corresponds to an OpenMP \emph{event} and fields in the record
structure are mostly the arguments that would be passed to the
OMPT callback for the event.

A runtime entry point for a device known as
\code{ompt_get_record_type} with type signature
\code{ompt_get_record_type_t} inspects the type
of a trace record and indicates whether the record at the current
position in the provided trace buffer is an OMPT record,
a native record, or an invalid record. An invalid record type
is returned if the cursor is out of bounds.

\argdesc
The argument \plc{buffer} indicates a trace buffer.

The argument \plc{current} is an opaque buffer cursor.




\crossreferences
\begin{itemize}
\item \code{ompt_buffer_t},
see \specref{sec:ompt_buffer_t}.
\item \code{ompt_buffer_cursor_t},
see \specref{sec:ompt_buffer_cursor_t}.
\end{itemize}

\subsubsubsection{\hcode{ompt_get_record_ompt_t}}
\label{sec:ompt_get_record_ompt_t}

\summary
A runtime entry point for a device known as \code{ompt_get_record_ompt}
with type signature \code{ompt_get_record_ompt_t}
obtains a pointer to an OMPT trace record from a trace buffer associated with a device.

\format
\begin{ccppspecific}
\begin{omptInquiry}
typedef ompt_record_ompt_t *(*ompt_get_record_ompt_t) (
  ompt_buffer_t *\plc{buffer},
  ompt_buffer_cursor_t \plc{current}
);
\end{omptInquiry}
\end{ccppspecific}

\descr

This function returns a pointer that may point to a record in the
trace buffer, or it may point to a record in thread local storage
where the information extracted from a record was assembled. The
information available for an event depends upon its type.

The return value of type \code{ompt_record_ompt_t}
defines a union type that can represent
information for any OMPT event record type.
Another call to the runtime entry point may overwrite the
contents of the fields in a record returned by a prior invocation.

\argdesc
The argument \plc{buffer} indicates a trace buffer.

The argument \plc{current} is an opaque buffer cursor.

\crossreferences
\begin{itemize}
\item \code{ompt_record_ompt_t},
see \specref{sec:ompt_record_ompt_t}.
\item \code{ompt_device_t},
see \specref{sec:ompt_device_t}.
\item \code{ompt_buffer_cursor_t},
see \specref{sec:ompt_buffer_cursor_t}.
\end{itemize}

\subsubsubsection{\hcode{ompt_get_record_native_t}}
\label{sec:ompt_get_record_native_t}

\summary

A runtime entry point for a device known as
\code{ompt_get_record_native} with type signature
\code{ompt_get_record_native_t}
obtains a pointer to a native trace record from a trace buffer associated with a device.

\format
\begin{ccppspecific}
\begin{omptInquiry}
typedef void *(*ompt_get_record_native_t) (
  ompt_buffer_t *\plc{buffer},
  ompt_buffer_cursor_t \plc{current},
  ompt_id_t *\plc{host_op_id}
);
\end{omptInquiry}
\end{ccppspecific}

\descr

The pointer returned  may point into the specified trace buffer, or into
thread local storage where the information extracted from a trace
record was assembled. The information available for a native event
depends upon its type. If the function returns a non-NULL result,
it will also set \code{*host_op_id} to identify host-side identifier
for the operation associated with the record.  A subsequent call
to \code{ompt_get_record_native} may overwrite the
contents of the fields in a record returned by a prior invocation.

\argdesc
The argument \plc{buffer} indicates a trace buffer.

The argument \plc{current} is an opaque buffer cursor.

The argument \plc{host_op_id} is a pointer to an identifier
that will be returned by the function. The entry point will set
*\plc{host_op_id} to the value of a host-side identifier for an operation on
a target device that was created when the operation was initiated by
the host.


\crossreferences
\begin{itemize}
\item \code{ompt_id_t},
see \specref{sec:ompt_id_t}.
\item \code{ompt_buffer_t},
see \specref{sec:ompt_buffer_t}.
\item \code{ompt_buffer_cursor_t},
see \specref{sec:ompt_buffer_cursor_t}.
\end{itemize}

\subsubsubsection{\hcode{ompt_get_record_abstract_t}}
\label{sec:ompt_get_record_abstract_t}

\summary
A runtime entry point for a device known as
\code{ompt_get_record_abstract}
with type signature
\code{ompt_get_record_abstract_t}
summarizes the context of a native (device-specific) trace record.

\format
\begin{ccppspecific}
\begin{omptOther}
typedef ompt_record_abstract_t *
(*ompt_get_record_abstract_t) (
  void *\plc{native_record}
);
\end{omptOther}
\end{ccppspecific}

\descr
An OpenMP implementation may execute on a device that logs
trace records in a native (device-specific) format unknown to a tool.
A tool can use the \code{ompt_get_record_abstract}
runtime entry point for the device with type signature
\code{ompt_get_record_abstract_t}
to decode a native trace record that it
does not understand into a standard form that it can interpret.

\argdesc

The argument \plc{native_record} is a pointer to a native trace
record.

\crossreferences
\begin{itemize}
\item \code{ompt_record_abstract_t},
see \specref{sec:ompt_record_abstract_t}.
\end{itemize}

\subsubsection{Lookup Entry Point}

\subsubsubsection{\hcode{ompt_function_lookup_t}}
\label{sec:ompt_function_lookup_t}
\label{sec:ompt_function_lookup}

\summary
A tool uses a lookup routine with type signature
\code{ompt_function_lookup_t}
to obtain pointers to runtime entry points that are
part of the OMPT interface.

\format

\begin{ccppspecific}
\begin{omptInquiry}
typedef void (*ompt_interface_fn_t) (void);

typedef ompt_interface_fn_t (*ompt_function_lookup_t) (
  const char *\plc{interface_function_name}
);
\end{omptInquiry}
\end{ccppspecific}


\descr

An OpenMP implementation provides a pointer to a lookup routine as an
argument to tool callbacks used to initialize tool support for
monitoring an OpenMP device using either tracing or callbacks.



When an OpenMP implementation invokes a tool initializer to configure
the OMPT callback interface, the OpenMP implementation will pass the
initializer a lookup function that the tool can use to obtain
pointers to runtime entry points that implement routines that are part of
the OMPT callback interface.

When an OpenMP implementation invokes a tool initializer to configure
the OMPT tracing interface for a device, the Open implementation will
pass the device tracing initializer a lookup function that the tool
can use to obtain pointers to runtime entry points that implement
tracing control routines appropriate for that device.

A tool can call the lookup function to obtain a pointer to a runtime
entry point.

\argdesc
The argument \plc{interface_function_name} is a C string
that represents the name of a runtime entry point.

\crossreferences
\begin{itemize}
\item Entry points in the OMPT callback interface, see
  \tabref{table:ompt-callback-interface-functions} for a list and
  \specref{sec:ompt-callback-entry-points} for detailed definitions.
\item Tool initializer for a device's OMPT tracing interface, \specref{sec:tracing-device-activity}.
\item Entry points in the OMPT tracing interface, see
  \tabref{table:ompt-tracing-interface-functions} for a list and
  \specref{sec:ompt-tracing-entry-points} for detailed definitions.
\item Tool initializer for the OMPT callback interface, \specref{sec:ompt_initialize_t}
\end{itemize}
% This is the end of ch4-toolsSupport.tex


% OMPD
%%%% \ompdsection{Overview on OMPD Tool Interface}
%%%% \label{sec:ompd-overview}

\section{OMPD}
\label{sec:ompd-overview}
\label{sec:third-party-tool-callback-interface}

OMPD allows \emph{third-party tools} such as a debuggers to inspect the 
OpenMP state of a live program or core file in an implementation-agnostic 
manner. That is, a tool that uses OMPD should work with any conforming 
OpenMP implementation. An OpenMP implementor provides a library for OMPD 
that a third-party tool can dynamically load. Using the interface exported 
by the OMPD library, the external tool can inspect the OpenMP state of a 
program. In order to satisfy requests from the third-party tool, the OMPD 
library may need to read data from, or to find the addresses of symbols in 
the OpenMP program. The OMPD library provides this functionality through a 
callback interface that the third-party tool must instantiate for the OMPD library.

To use OMPD, the third-party tool loads the OMPD library. The OMPD library exports 
the API that is defined throughout this section and that the tool uses to 
determine OpenMP information about the OpenMP program. The OMPD library must
look up the symbols and read data out of the program. It does not perform
these operations directly, but instead it use the callback interface that the
tool exports to cause the tool to perform them.

The OMPD architecture insulates tools from the internal structure of the 
OpenMP runtime while the OMPD library is insulated from the details of how 
to access the OpenMP program. This decoupled design allows for flexibility in how 
the OpenMP program and tool are deployed, so that, for example, the tool and the 
OpenMP program are not required to execute on the same machine.

Generally the tool does not interact directly with the OpenMP runtime and, 
instead, interacts with it through the OMPD library. However, a few instances 
require the tool to access the OpenMP runtime directly. These cases fall into 
two broad categories. The first is during initialization, where the tool must
look up symbols and read variables in the OpenMP runtime in order to identify 
the OMPD library that it should use, which is discussed in 
\specref{subsubsec:ompd_dll_locations} and 
\specref{subsubsec:ompd_dll_locations_valid}. The second category relates to 
arranging for the tool to be notified when certain events occur during the 
execution of the OpenMP program. For this purpose, the OpenMP implementation
must define certain symbols in the runtime code, as is discussed in 
\specref{subsec:runtime-entry-points-for-ompd}. Each of these symbols corresponds 
to an event type. The runtime must ensure that control passes through the 
appropriate named location when events occur. If the tool requires notification 
of an event, it can plant a breakpoint at the matching location. The location 
can, but may not, be a function. It can, for example, simply be a label. However, 
the names of the locations must have external \texttt{C} linkage.



%OMPD-TODO: OMPD-tr §1-3 to go here
%OMPD-TODO: OMPD-tr §5-6 to go here

% %%%%%%%%%%%%%%%%%%%%%%%%%%%%%%%%%%%%%%%%%%%%%%%%%%%%%%%%%%%%%%%%%%%%%%%%%%%

\subsection{Activating an OMPD Tool}
\label{subsec:activating}

The tool and the OpenMP program exist as separate processes. 
Thus, coordination is required between the OpenMP runtime
and the external tool for OMPD.

\subsubsection{Enabling the Runtime for OMPD}
\label{subsubsec:enabling-ompd}

In order to support third-party tools, the OpenMP runtime may need to collect
and to maintain information that it might not otherwise. The OpenMP runtime 
collects whatever information is necessary to support OMPD if the environment 
variable \code{OMP_DEBUG} is set to \plc{enabled}.

\crossreferences
\begin{itemize}
\item Activating an OMPT Tool, \specref{sec:ompt-initialization}

\item   \code{OMP_DEBUG}, \specref{sec:OMP_DEBUG}
\end{itemize}



\subsubsection{Finding the OMPD Library}
\label{subsubsec:finding-the-ompd}

An OpenMP runtime may have more than one OMPD libary. The tool must be able 
to locate the right library to use for the OpenMP program that it is examining.

As part of the OpenMP interface, OMPD requires the OpenMP runtime system to
provide a public variable \code{ompd_dll_locations}, which is an \code{argv}-style
vector of filename string pointers that provides the name(s) of any compatible 
OMPD library. This variable must have \code{C} linkage. The tool uses the name 
of the variable verbatim and, in particular, does not apply any name mangling 
before performing the look up.

The \code{ompd_dll_locations} variable points to a NULL-terminated vector of 
zero or more NULL-terminated pathname strings that do not have any filename 
conventions. The vector of string pointers must be fully initialized 
\emph{before} \code{ompd_dll_locations} is set to a non-null value, such that 
if a tool, such as a debugger, stops execution of the OpenMP program at any 
point at which \code{ompd_dll_locations} is non-null, then the vector of 
strings to which it points is valid and complete.

The programming model or architecture of the tool (and hence
that of the required OMPD) does not have to match that of the OpenMP program
being examined.
It is the responsibility of the tool to interpret the contents
of \code{ompd_dll_locations} to find a suitable OMPD that matches
its own architectural characteristics.
On platforms that support different programming models
(for example, 32-bit vs 64-bit), OpenMP implementers are encouraged
to provide OMPD libraries for all models, and which can handle
OpenMP programs of any model.
Thus, for example, a 32-bit debugger using OMPD should be able
to debug a 64-bit OpenMP program
by loading a 32-bit OMPD that can manage a 64-bit OpenMP runtime.

\crossreferences
\begin{itemize}
	\item Identifying the Matching OMPD, \specref{subsubsec:ompd_dll_locations}
	%\item \code{ompd_enable}, \specref{subsubsec:ompd_enable}
	%\item Activating an OMPT Tool, \specref{ompd:ompt-initialization}
\end{itemize}





%%%% \ompdsection{Activating a OMPD Tool}
%%%% \label{sec:ompd-initialization}
%%%% %OMPD-TODO: OMPD-tr §4 to go here

% 
%%%%%%%%%%%%%%%%%%%%%%%%%%%%%%%%%%%%%%%%%%%%%%%%%%%%%%%%%%%%%%%%%%%%%%%%%%%%%%%%%%%%%%%%%%%%%%%%%%%%%%%%%%%%%%%%%%%%%%%%%%%%%%%%%%%%%%%%%%%%%%%%%%%%%%%

\subsection{OMPD Data Types}
\label{sec:ompd-data-types}
%OMPD-TODO: OMPD-tr §15 to go here

In this section, we define the types, structures, and functions for the OMPD API.

\subsubsection{Basic Types}
The following describes the basic OMPD API types.

\subsubsubsection{Size Type}
\label{subsubsubsec:ompd_size_t}

This type is used to specify the number of bytes in opaque data objects passed across the OMPD API.

\format

	\begin{ccppspecific}
	\begin{ompSyntax}
typedef uint64_t ompd_size_t;
	\end{ompSyntax}
	\end{ccppspecific}


\subsubsubsection{Wait ID Type}
\label{subsubsubsec:ompd_wait_id_t}

This type identifies what a thread is waiting for.

\format

	\begin{ccppspecific}
	\begin{ompSyntax}
typedef uint64_t ompd_wait_id_t;
	\end{ompSyntax}
	\end{ccppspecific}


\subsubsubsection{Basic Value Types}
\label{subsubsubsec:ompd_addr_t}
\label{subsubsubsec:ompd_word_t}
\label{subsubsubsec:ompd_seg_t}

These definitions represent a word, address, and segment value types.

\format

\begin{ccppspecific}
\begin{ompSyntax}
typedef uint64_t ompd_addr_t;
typedef int64_t  ompd_word_t;
typedef uint64_t ompd_seg_t;
\end{ompSyntax}
\end{ccppspecific}


The \plc{ompd_addr_t} type represents an unsigned integer address in an 
OpenMP process.
The \plc{ompd_word_t} type represents a signed version of  \plc{ompd_addr_t} to hold a signed 
integer of the OpenMP process.
The \plc{ompd_seg_t} type represents an unsigned integer segment value.

\subsubsubsection{Address Type}
\label{subsubsubsec:ompd_address_t}

This type is a structure that OMPD uses to specify device addresses, 
which may or may not be segmented.

\format

\begin{ccppspecific}
\begin{ompEnv}
typedef struct {
  ompd_seg_t \plc{segment};
  ompd_addr_t \plc{address};
} ompd_address_t;

#define OMPD_SEGMENT_UNSPECIFIED 0
\end{ompEnv}
\end{ccppspecific}


For non segmented architectures, use OMPD\_SEGMENT\_UNSPECIFIED in the \plc{segment} 
field of \code{ompd_address_t}.

\subsubsection{System Device Identifiers}

Different OpenMP runtimes may utilize different underlying devices.
The type used to hold an device identifier can vary in size and format, and 
therefore is not explicitly represented in the OMPD API. Device identifiers are 
passed across the interface using a device-identifier `kind', a pointer to where
the device identifier is stored, and the size of the device identifier in bytes.
The OMPD library and tool using it must agree on the format
of what is being passed.
Each different kind of device identifier uses a unique
unsigned 64-bit integer value.

Recommended values of \code{omp_device_kind_t} are defined in the \code{ompd_types.h} 
header file, which is available on \url{http://www.openmp.org/}. 

\label{ompd:omp_device_kind_t}
\format

	\begin{ccppspecific}
	\begin{ompSyntax}
typedef uint64_t omp_device_kind_t;
	\end{ompSyntax}
	\end{ccppspecific}


\subsubsection{Thread Identifiers}

Different OpenMP runtimes may use different underlying native
thread implementations.
The type used to hold a thread identifier can vary in size and format, and 
therefore is not explicitly represented in the OMPD API. Thread identifiers are 
passed across the interface using a thread-identifier `kind', a pointer to where
the thread identifier is stored, and the size of the thread identifier in bytes.
The OMPD library and tool using it must agree on the format
of what is being passed.
Each different kind of thread identifier uses a unique
unsigned 64-bit integer value.

Recommended values of \code{ompd_thread_id_kind_t} are defined in the \code{ompd_types.h} 
header file, which is available on \url{http://www.openmp.org/}. 

\label{ompd:ompd_thread_id_kind_t}
\format

\begin{ccppspecific}
\begin{ompSyntax}
typedef uint64_t ompd_thread_id_kind_t;
\end{ompSyntax}
\end{ccppspecific}


\subsubsection{OMPD Handle Types}
\label{subsubsec:ompd_address_space_handle_t}
\label{subsubsec:ompd_thread_handle_t}
\label{subsubsec:ompd_parallel_handle_t}
\label{subsubsec:ompd_task_handle_t}

Each operation of the OMPD interface that applies to a particular address space, thread, parallel 
region, or task must explicitly specify
a \emph{handle} for the operation.
OMPD employs handles for address spaces (for a host or target device), threads, parallel regions, 
and tasks. A handle for an entity is constant while the entity itself is alive. Handles are defined by 
the OMPD plugin, and are opaque to the tool. The following defines the OMPD 
handle types:

\format

\begin{ccppspecific}
\begin{ompSyntax}
typedef struct _ompd_aspace_handle_s ompd_address_space_handle_t;
typedef struct _ompd_thread_handle_s ompd_thread_handle_t;
typedef struct _ompd_parallel_handle_s ompd_parallel_handle_t;
typedef struct _ompd_task_handle_s ompd_task_handle_t;
\end{ompSyntax}
\end{ccppspecific}


Defining externally visible type names in this way introduces type safety to the interface, and helps 
to catch instances where incorrect handles are passed by the tool to the OMPD 
library. The \code{struct}s do not need to be defined at all. The OMPD library 
must cast incoming (pointers to) handles to the appropriate internal, private types.

\subsubsection{Tool Context Types}
\label{subsubsec:ompd_address_space_context_t}
\label{subsubsec:ompd_thread_context_t}

A third-party tool uses contexts to uniquely  identifies abstractions. These contexts are opaque to 
the OMPD library and are defined as follows:

\format

\begin{ccppspecific}
\begin{ompSyntax}
typedef struct _ompd_aspace_cont_s ompd_address_space_context_t;
typedef struct _ompd_thread_cont_s ompd_thread_context_t;
\end{ompSyntax}
\end{ccppspecific}


\subsubsection{Return Code Types}
\label{subsubsec:ompd_rc_t}

Each OMPD operation has a return code. The return code types and their semantics are defined as 
follows:

\format

\begin{ccppspecific}
\begin{ompSyntax}
typedef enum {
  ompd_rc_ok = 0,
  ompd_rc_unavailable = 1,
  ompd_rc_stale_handle = 2,
  ompd_rc_bad_input = 3,
  ompd_rc_error = 4,
  ompd_rc_unsupported = 5,
  ompd_rc_needs_state_tracking = 6,
  ompd_rc_incompatible = 7,
  ompd_rc_device_read_error = 8,
  ompd_rc_device_write_error = 9,
  ompd_rc_nomem = 10,
} ompd_rc_t;	
\end{ompSyntax}
\end{ccppspecific}


\descr
\label{ompd:ompd_rc_ok}
\code{ompd_rc_ok} is returned when the operation is successful.

\label{ompd:ompd_rc_unavailable}
\code{ompd_rc_unavailable} is returned when 
information is not available for the specified context.

\label{ompd:ompd_rc_stale_handle}
\code{ompd_rc_stale_handle} is returned when
the specified handle is no longer valid.

\label{ompd:ompd_rc_bad_input}
\code{ompd_rc_bad_input} is returned when
the input parameters (other than handle) are invalid. 

\label{ompd:ompd_rc_error}
\code{ompd_rc_error} is returned when
a fatal error occurred.

\label{ompd:ompd_rc_unsupported}
\code{ompd_rc_unsupported} is returned when
the requested operation is not supported.

\label{ompd:ompd_rc_needs_state_tracking}
\code{ompd_rc_needs_state_tracking} is returned when
the state tracking operation failed because state tracking is not currently enabled.

\label{ompd:ompd_rc_incompatible}
\code{ompd_rc_incompatible} is returned when
this OMPD is incompatible with the OpenMP program.

\label{ompd:ompd_rc_device_read_error}
\code{ompd_rc_device_read_error} is returned when
a read operation failed on the device

\label{ompd:ompd_rc_device_write_error}
\code{ompd_rc_device_write_error} is returned when
a write operation failed to the device.

\label{ompd:ompd_rc_nomem}
\code{ompd_rc_nomem} is returned when
unable to allocate memory.

\subsubsection{OpenMP Scheduling Types}
\label{subsubsec:ompd_sched_t}

The following enumeration defines \code{ompd_sched_t}, which is the OMPD API definition 
corresponding to the OpenMP enumeration type \code{omp_sched_t} (see 

\specref{subsec:omp_set_schedule}).
\code{ompd_sched_t} also defines \code{ompd_sched_vendor_lo} and
\code{ompd_sched_vendor_hi} to define the range of implementation-specific 
\code{omp_sched_t} values than can be handle by the OMPD API.

\begin{quote}
	\begin{lstlisting}

	\end{lstlisting}
\end{quote}

\format

\begin{ccppspecific}
\begin{ompSyntax}
typedef enum {
  ompd_sched_static = 1,
  ompd_sched_dynamic = 2,
  ompd_sched_guided = 3,
  ompd_sched_auto = 4,
  ompd_sched_vendor_lo = 5,
  ompd_sched_vendor_hi = 0x7fffffff
} ompd_sched_t;
\end{ompSyntax}
\end{ccppspecific}


\subsubsection{OpenMP Proc Binding Types}
\label{subsubsec:ompd_proc_bind_t}

The following enumeration defines \code{ompd_proc_bind_t}, which is the OMPD
API definition corresponding to the OpenMP enumeration type
\code{omp_proc_bind_t} (\specref{subsec:omp_get_proc_bind}).

\format

\begin{ccppspecific}
\begin{ompSyntax}
typedef enum {
  ompd_proc_bind_false = 0,
  ompd_proc_bind_true = 1,
  ompd_proc_bind_master = 2,
  ompd_proc_bind_close = 3,
  ompd_proc_bind_spread = 4
} ompd_proc_bind_t;
\end{ompSyntax}
\end{ccppspecific}


\subsubsection{Primitive Types}
\label{subsubsec:ompd_device_type_sizes_t}
The following structure contains members that the OMPD library can use
to interrogate the tool about the ``sizeof'' of primitive types in the OpenMP architecture 
address space.

\format

\begin{ccppspecific}
\begin{ompSyntax}
typedef struct {
  uint8_t \plc{sizeof_char};
  uint8_t \plc{sizeof_short};
  uint8_t \plc{sizeof_int};
  uint8_t \plc{sizeof_long};
  uint8_t \plc{sizeof_long_long};
  uint8_t \plc{sizeof_pointer};
} ompd_device_type_sizes_t;
\end{ompSyntax}
\end{ccppspecific}


\descr
The fields of \code{ompd_device_type_sizes_t} give the sizes of
the eponymous basic types used by the OpenMP runtime.
As the tool and the OMPD plugin, by definition, have the same
architecture and programming model, the size of the fields can be given
as \code{int}.

\crossreferences
\begin{itemize}
	\item
	\code{ompd_callback_sizeof_fn_t}, \specref{subsubsubsec:ompd_callback_sizeof_fn_t}
\end{itemize}

\subsubsection{Runtime State Types}

The OMPD runtime states mirror those in OMPT (\specref{sec:ompt_get_state_t}). Note that there is no guarantee that 
the numeric values of the corresponding members of the enumerations are identical.

\format

\begin{ccppspecific}
\begin{ompSyntax}
typedef enum {
  ompd_state_work_serial = 0x00,
  ompd_state_work_parallel = 0x01,
  ompd_state_work_reduction = 0x02,
  ompd_state_idle = 0x10,
  ompd_state_overhead = 0x20,
  ompd_state_wait_barrier = 0x40,
  ompd_state_wait_barrier_implicit = 0x41,
  ompd_state_wait_barrier_explicit = 0x42,
  ompd_state_wait_taskwait = 0x50,
  ompd_state_wait_taskgroup = 0x51,
  ompd_state_wait_mutex = 0x60,
  ompd_state_wait_lock = 0x61,
  ompd_state_wait_critical = 0x62,
  ompd_state_wait_atomic = 0x63,
  ompd_state_wait_ordered = 0x64,
  ompd_state_undefined = 0x70,
  ompd_state_first = 0x71,
} ompd_state_t;
\end{ompSyntax}
\end{ccppspecific}


\descr
\label{ompd_state_work_serial}
\code{ompd_state_work_serial} - 
working outside parallel
 
\label{ompd_state_work_parallel}
\code{ompd_state_work_parallel} - 
working within parallel

\label{ompd_state_work_reduction}
\code{ompd_state_work_reduction} - 
performing a reduction

\label{ompd_state_idle}
\code{ompd_state_idle} - 
waiting for work
 
\label{ompd_state_overhead}
\code{ompd_state_overhead} - 
non-wait overhead

\label{ompd_state_wait_barrier}
\code{ompd_state_wait_barrier} - 
generic barrier

\label{ompd_state_wait_barrier_implicit}
\code{ompd_state_wait_barrier_implicit} - 
implicit barrier

\label{ompd_state_wait_barrier_explicit}
\code{ompd_state_wait_barrier_explicit} - 
explicit barrier

\label{ompd_state_wait_taskwait}
\code{ompd_state_wait_taskwait} - 
waiting at a taskwait

\label{ompd_state_wait_taskgroup}
\code{ompd_state_wait_taskgroup} - 
waiting at a taskgroup

\label{ompd_state_wait_mutex}
\code{ompd_state_wait_mutex} - 
waiting for any mutex kind

\label{ompd_state_wait_lock}
\code{ompd_state_wait_lock} - 
waiting for lock

\label{ompd_state_wait_critical}
\code{ompd_state_wait_critical} - 
waiting for critical

\label{ompd_state_wait_atomic}
\code{ompd_state_wait_atomic} - 
waiting for atomic

\label{ompd_state_wait_ordered}
\code{ompd_state_wait_ordered} - 
waiting for ordered

\label{ompd_state_undefined}
\code{ompd_state_undefined} - 
undefined thread state

\label{ompd_state_first}
\code{ompd_state_first} - 
initial enumeration state


\subsection{OMPD Tool Callback Interface}
\label{subsec:ompd-tool-callback-interface}

For the OMPD library to provide information about the internal state
of the OpenMP runtime system in an OpenMP process or core file,
it must have a means to extract information from
the OpenMP process that the tool is debugging.
The OpenMP process that the tool is operating on may be either a ``live'' process or a core file,
and a thread may be either a ``live'' thread in an OpenMP process,
or a thread in a core file.
To enable the OMPD library to extract state information from an OpenMP process or core file,
the tool must supply the OMPD library with callback functions to inquire
about the size of primitive types in the device of the OpenMP process,
look up the addresses of symbols,
as well as read and write memory in the device.
The OMPD library then uses these callbacks to implement its interface operations.
The OMPD library will only call the callback functions in direct response
to calls made by the tool to the OMPD library.
Signatures for the tool callbacks used by the OMPD library are given below.


\subsubsection{Memory Management of OMPD Library}
The OMPD library must not access the heap manager directly. Instead, if it needs heap memory it
must use the memory allocation and deallocation callback functions that are described in this
section, \code{ompd_callback_memory_alloc_fn_t} (see~\specref{subsubsubsec:ompd_callback_memory_alloc_fn_t})
and 
\code{ompd_callback_memory_free_fn_t} (see~\specref{subsubsubsec:ompd_callback_memory_free_fn_t}),
which are
provided by the tool to obtain and release heap memory. This will ensure that the library does not
interfere with any custom memory management scheme that the tool may use.

If the OMPD library is implemented in \code{C++}, memory management operators like \code{new}
and \code{delete} in all their variants, \emph{must all} be overloaded and implemented in terms of
the callbacks provided by the tool. The OMPD library must be coded so that any of its
definitions of \code{new} or \code{delete} do not interfere with any defined by the tool.

In some cases, the OMPD library will need to allocate memory to return results to the
tool. This memory will then be `owned' by the tool, which will be responsible for releasing it. It is
therefore vital that the OMPD library and the tool use the same memory manager.

OMPD handles are created by the OMPD library. These are opaque to the
tool, and depending on the specific implementation of OMPD may have complex
internal structure. The tool cannot know whether the handle pointers returned by the API
correspond to discrete heap allocations. Consequently, the tool must not simply deallocate a
handle by passing an address it receives from the OMPD library to its own memory manager.
Instead, the API includes functions that the tool must use when it no longer needs a handle.

Contexts are created by a tool and passed to the OMPD library. The OMPD
library does not need to release contexts; instead this will be done by the tool after it releases any
handles that may be referencing the contexts.

\subsubsubsection{\hcode{ompd_callback_memory_alloc_fn_t}}
\label{subsubsubsec:ompd_callback_memory_alloc_fn_t}
\index{ompd_callback_memory_alloc_fn_t@{\code{ompd_callback_memory_alloc_fn_t}}}

\summary
The type signature of the callback routine provided by the tool
to be used by the OMPD library to allocate memory.


% the odd-looking spacing between type and argument name ensures
% they line up in the pdf
\begin{cspecific}
\begin{ompSyntax}
typedef ompd_rc_t (*ompd_callback_memory_alloc_fn_t) (
  ompd_size_t \plc{nbytes},
  void **\plc{ptr}
);
\end{ompSyntax}
\end{cspecific}


\descr
The OMPD library may call the \code{ompd_callback_memory_alloc_fn_t} callback function to allocate memory.

\argdesc
The argument \plc{nbytes} gives the size in bytes of the block of memory the
OMPD library wants allocated.

The address of the newly allocated
block of memory is returned in \plc{*ptr}.
The newly allocated block is suitably aligned for any type of variable,
and is not guaranteed to be zeroed.

\crossreferences
\begin{itemize}
\item
  \code{ompd_size_t}, \specref{subsubsubsec:ompd_size_t}
\item
  \code{ompd_rc_t}, \specref{subsubsec:ompd_rc_t}
\end{itemize}

\subsubsubsection{\hcode{ompd_callback_memory_free_fn_t}}
\label{subsubsubsec:ompd_callback_memory_free_fn_t}
\index{ompd_callback_memory_free_fn_t@{\code{ompd_callback_memory_free_fn_t}}}

\summary
The type signature of the callback routine provided by the
tool to be used by the OMPD library to deallocate memory.


\begin{cspecific}
\begin{ompSyntax}
typedef ompd_rc_t (*ompd_callback_memory_free_fn_t) (
  void *\plc{ptr}
);
\end{ompSyntax}
\end{cspecific}

\descr
The OMPD library calls the \code{ompd_callback_memory_free_fn_t} callback function to
deallocate memory obtained from a prior call to the \code{ompd_callback_memory_alloc_fn_t}
callback function.

\argdesc
\plc{ptr} is the address of the block to be deallocated.

\crossreferences
\begin{itemize}
\item
  \code{ompd_callback_memory_alloc_fn_t}, \specref{subsubsubsec:ompd_callback_memory_alloc_fn_t}
\item
  \code{ompd_rc_t}, \specref{subsubsec:ompd_rc_t}
\item
  \code{ompd_callbacks_t}, \specref{subsubsec:ompd_callbacks_t}
\end{itemize}

\subsubsection{Context Management and Navigation}

The tool provides the OMPD library with callbacks
to manage and navigate context relationships.

\subsubsubsection{\hcode{ompd_callback_get_thread_context_for_thread_id_fn_t}}
\label{subsubsubsec:ompd_callback_get_thread_context_for_thread_id_fn_t}
% the use of \discretionary below overrides the character used for
% hyphenation when an optional linebreak is used.
% if \- were used, we'd get a "-" at the end of the line being
% broken.  We probably don't wantthat, but instead want just a plain
% linebreak without the "-".
\index{ompd_callback_get_thread_context_for_thread_id_fn_t@{{\scode{ompd_callback_get_thread_context_for_\discretionary{}{}{}thread_id_fn_t}}}}

\summary
The type signature of the callback routine provided by the
third party tool the OMPD library can use to map a
thread identifier to a tool \plc{thread context}.


% the odd-looking spacing between type and argument name ensures
% they line up in the pdf
\begin{cspecific}
\begin{ompSyntax}
typedef ompd_rc_t
(*ompd_callback_get_thread_context_for_thread_id_fn_t) (
  ompd_address_space_context_t *\plc{address_space_context},
  ompd_thread_id_t \plc{kind},
  ompd_size_t \plc{sizeof_thread_id},
  const void *\plc{thread_id},
  ompd_thread_context_t **\plc{thread_context}
);
\end{ompSyntax}
\end{cspecific}


\descr

This callback maps a thread identifier within the address
space identified by \plc{address_space_context} to a tool thread context. The OMPD library
can use the thread context, for example, to access
thread local storage (TLS).

\argdesc

The input argument \plc{address_space_context} is an opaque handle provided by the tool
to reference an address space.
The input arguments \plc{kind},  \plc{sizeof_thread_id}, and \plc{thread_id} represent a thread identifier.
On return the output argument \plc{thread_context} provides an opaque handle to the OMPD library 
that maps a thread identifier to a tool thread context.

\restrictions
The \plc{thread_context} provided by this function is valid until the OMPD library returns 
from the OMPD tool interface routine.

\crossreferences
\begin{itemize}
\item
  \code{ompd_rc_t}, \specref{subsubsec:ompd_rc_t}
\item
  \code{ompd_address_space_context_t}, \specref{subsubsec:ompd_address_space_context_t}
\item
  \code{ompd_thread_id_t}, \specref{ompd:ompd_thread_id_t}
\item
  \code{ompd_size_t}, \specref{subsubsubsec:ompd_size_t}
\item
  \code{ompd_thread_context_t}, \specref{subsubsec:ompd_thread_context_t}
\end{itemize}

%On success the tool \plc{thread context} is returned in \plc{*thread_context}.
%The \plc{thread context} is created, and remains owned, by the tool.
%The OMPD plugin can assume that the \plc{thread context} is valid for
%as long as the tool is holding any references to \plc{thread handles}
%that may contain the \plc{thread context}.
%The OMPD plugin library can use the \plc{thread context}, for example, to access
%thread local storage (TLS).

%On failure, the callback returns an error code from \code{ompd_rc_t}.


\subsubsubsection{\hcode{ompd_callback_sizeof_fn_t}}
\label{subsubsubsec:ompd_callback_sizeof_fn_t}
\index{ompd_callback_sizeof_fn_t@{\code{ompd_callback_sizeof_fn_t}}}

\summary
The type signature of the callback routine provided by the
tool the OMPD library can use to find the sizes of the primitive types
in an address space.


% the odd-looking spacing between type and argument name ensures
% they line up in the pdf
\begin{cspecific}
\begin{ompSyntax}
typedef ompd_rc_t (*ompd_callback_sizeof_fn_t) (
  ompd_address_space_context_t *\plc{address_space_context},
  ompd_device_type_sizes_t *\plc{sizes}
);
\end{ompSyntax}
\end{cspecific}

\descr
This callback may be called by the OMPD library to obtain the sizes of the basic
primitive types for a given address space.

\argdesc
The callback returns the sizes of
the basic primitive types used by the \plc{address_space_context}
in \plc{*sizes}.

\crossreferences
\begin{itemize}
\item
  \code{ompd_address_space_context_t}, \specref{subsubsec:ompd_address_space_context_t}
\item
  \code{ompd_device_type_sizes_t}, \specref{subsubsec:ompd_device_type_sizes_t}
\item
  \code{ompd_rc_t}, \specref{subsubsec:ompd_rc_t}
\item
  \code{ompd_callbacks_t}, \specref{subsubsec:ompd_callbacks_t}
\end{itemize}

\subsubsection{Accessing Memory in the OpenMP Program or Runtime}

The OMPD library may need to read from, or write to, the OpenMP program.
It cannot do this directly, but instead must use the callbacks provided
to it by the tool, which will perform the operation
on its behalf.

\subsubsubsection{\hcode{ompd_callback_symbol_addr_fn_t}}
\label{subsubsubsec:ompd_callback_symbol_addr_fn_t}
\index{ompd_callback_symbol_addr_fn_t@{\code{ompd_callback_symbol_addr_fn_t}}}

\summary
The type signature of the callback provided by the tool the
OMPD library can use to look up the addresses of symbols in an OpenMP program.


% the odd-looking spacing between type and argument name ensures
% they line up in the pdf
\begin{cspecific}
\begin{ompSyntax}
typedef ompd_rc_t (*ompd_callback_symbol_addr_fn_t) (
  ompd_address_space_context_t *\plc{address_space_context},
  ompd_thread_context_t *\plc{thread_context},
  const char *\plc{symbol_name},
  ompd_address_t *\plc{symbol_addr},
  const char *\plc{file_name}
);
\end{ompSyntax}
\end{cspecific}

\descr
This callback function may be called by the OMPD library to look up
addresses of symbols within an specified address space of the tool.

\argdesc
This callback looks up the symbol provided in \plc{symbol_name}.

The \plc{address_space_context} input parameter is the tool's representation of the address 
space of the process, core file, or device. The use of a NULL \plc{address_space_context}
results in unspecified behavior.

The \plc{thread_context} is an optional input parameter which should be NULL for global memory 
access. If  \plc{thread_context} is not NULL, it gives the thread specific context for the symbol 
lookup, for the purpose of calculating thread local storage~(TLS) addresses.
If the \plc{thread_context} parameter is not NULL, the thread that the \plc{thread_context} 
argument refers to must be associated either to the process or to the device that corresponds to 
the \plc{address_space_context} argument.

The \plc{symbol_name} supplied by the OMPD library is used verbatim by the tool, and in particular, 
no name mangling, demangling or other transformations are performed prior to the lookup.
The \plc{symbol_name} parameter must correspond to a statically allocated symbol within the 
specified address space. The symbol can correspond to any type of object, such as a variable, 
thread local storage variable, function, or untyped label. The symbol must be defined and can have 
a local, global, or weak binding.

The \plc{file_name} parameter is an optional input parameter that indicates the 
name of the shared library where the symbol is defined, and is intended to help the third party tool 
disambiguate symbols that are defined multiple times across the executable or shared library files. 
The shared library name may not be an exact match for the name seen by the tool. If the 
\plc{file_name} parameter is NULL, the tool will try first finding the symbol in the executable file, 
and, if the symbol is not found, the tool will try finding the symbol in the shared libraries in the order 
in which the shared libraries are loaded into the address space. If the \plc{file_name} parameter is 
not NULL, the tool will try first finding the symbol in the libraries that match the name in the 
\plc{file_name} parameter, and, if the symbol is not found, the tool will find the symbol following
the procedure as if the \plc{file_name} parameter is NULL.

The callback does not support finding symbols that are dynamically allocated on the call 
stack, or statically allocated symbols defined within the scope of a function or subroutine.

The callback returns the address of the symbol in \plc{*symbol_addr}.

\crossreferences
\begin{itemize}
\item
  \code{ompd_address_space_context_t}, \specref{subsubsec:ompd_address_space_context_t}
\item
  \code{ompd_thread_context_t}, \specref{subsubsec:ompd_thread_context_t}
\item
  \code{ompd_address_t}, \specref{subsubsubsec:ompd_address_t}
\item
  \code{ompd_rc_t}, \specref{subsubsec:ompd_rc_t}
\item
  \code{ompd_callbacks_t}, \specref{subsubsec:ompd_callbacks_t}
\end{itemize}

\subsubsubsection{\hcode{ompd_callback_memory_read_fn_t}}
\label{subsubsubsec:ompd_callback_memory_read_fn_t}
\index{ompd_callback_memory_read_fn_t@{\code{ompd_callback_memory_read_fn_t}}}

\summary

The type signature of the callback provided by the tool the
OMPD library can use to read data out of an OpenMP program.



% the odd-looking spacing between type and argument name ensures
% they line up in the pdf
\begin{cspecific}
\begin{ompSyntax}
typedef ompd_rc_t (*ompd_callback_memory_read_fn_t) (
  ompd_address_space_context_t *\plc{address_space_context},
  ompd_thread_context_t *\plc{thread_context},
  const ompd_address_t *\plc{addr},
  ompd_size_t \plc{nbytes},
  void *\plc{buffer}
);
\end{ompSyntax}
\end{cspecific}

\descr
The function \code{read_memory} of this type copies a block of data from \plc{addr} 
within the address space to the tool \plc{buffer}. 

The function \code{read_string} of this type copies a string pointed to by \plc{addr}, 
including the terminating null byte (\code{'\0'}), to the tool \plc{buffer}. 
At most \plc{nbytes} bytes are copied. 
If there  is no null byte among the first \plc{nbytes} bytes, the string placed in 
\plc{buffer} will not be null-terminated.


\argdesc
The address from which the data are to be read out of the OpenMP program
specified by \plc{address_space_context} is given by \plc{addr}.
\plc{nbytes} gives the number of bytes to be transfered.
The \plc{thread_context} argument is optional for global memory access,
and in this case should be NULL.
If it is not NULL, \plc{thread_context} identifies the thread
specific context for the memory access for the purpose of accessing
thread local storage (TLS).

The data are returned through \plc{buffer}, which is allocated and
owned by the OMPD library.
The contents of the buffer are unstructured, raw bytes.
It is the responsibility of the OMPD library to arrange for
any transformations such as byte-swapping that may be necessary
(see~\specref{subsubsubsec:ompd_callback_device_host_fn_t}) to interpret the
data returned.

\crossreferences
\begin{itemize}
\item
  \code{ompd_address_space_context_t}, \specref{subsubsec:ompd_address_space_context_t}
\item
  \code{ompd_thread_context_t}, \specref{subsubsec:ompd_thread_context_t}
\item
  \code{ompd_address_t}, \specref{subsubsubsec:ompd_address_t}
\item
  \code{ompd_size_t}, \specref{subsubsubsec:ompd_size_t}
\item
  \code{ompd_rc_t}, \specref{subsubsec:ompd_rc_t}
\item
  \code{ompd_callback_device_host_fn_t}, \specref{subsubsubsec:ompd_callback_device_host_fn_t}
\item
  \code{ompd_callbacks_t}, \specref{subsubsec:ompd_callbacks_t}
\end{itemize}

\subsubsubsection{\hcode{ompd_callback_memory_write_fn_t}}
\label{subsubsubsec:ompd_callback_memory_write_fn_t}
\index{ompd_callback_memory_write_fn_t@{\code{ompd_callback_memory_write_fn_t}}}

\summary

The type signature of the callback provided by the tool the
OMPD library can use to write data to an OpenMP program.


% the odd-looking spacing between type and argument name ensures
% they line up in the pdf
\begin{cspecific}
\begin{ompSyntax}
typedef ompd_rc_t (*ompd_callback_memory_write_fn_t) (
  ompd_address_space_context_t *\plc{address_space_context},
  ompd_thread_context_t *\plc{thread_context},
  const ompd_address_t *\plc{addr},
  ompd_size_t \plc{nbytes},
  const void *\plc{buffer}
);
\end{ompSyntax}
\end{cspecific}

\descr
The OMPD library may call this function callback to have the tool write a block of data
to a location within an address space from a provided buffer.

\argdesc
The address to which the data are to be written in the OpenMP program
specified by \plc{address_space_context} is given by \plc{addr}.
\plc{nbytes} gives the number of bytes to be transfered.
The \plc{thread_context} argument is optional for global memory access,
and in this case should be NULL.
If it is not NULL, \plc{thread_context} identifies the thread
specific context for the memory access for the purpose of accessing
thread local storage (TLS).

The data to be written are passed through \plc{buffer}, which is allocated and
owned by the OMPD library.
The contents of the buffer are unstructured, raw bytes.
It is the responsibility of the OMPD library to arrange for
any transformations such as byte-swapping that may be necessary
(see~\specref{subsubsubsec:ompd_callback_device_host_fn_t})
to render the data into a form compatible with the OpenMP runtime.

\crossreferences
\begin{itemize}
\item
  \code{ompd_address_space_context_t} \specref{subsubsec:ompd_address_space_context_t}
\item
  \code{ompd_thread_context_t}, \specref{subsubsec:ompd_thread_context_t}
\item
  \code{ompd_address_t}, \specref{subsubsubsec:ompd_address_t}
\item
  \code{ompd_size_t}, \specref{subsubsubsec:ompd_size_t}
\item
  \code{ompd_rc_t}, \specref{subsubsec:ompd_rc_t}
\item
  \code{ompd_callback_device_host_fn_t}, \specref{subsubsubsec:ompd_callback_device_host_fn_t}
\item
  \code{ompd_callbacks_t}, \specref{subsubsec:ompd_callbacks_t}
\end{itemize}

\subsubsection{Data Format Conversion}
\label{subsubsec:data-format-conversion}

The architecuture and/or programming-model of tool and
OMPD library may be different from that of the OpenMP program being
examined.
Consequently, the conventions for representing data will differ.
The callback interface includes operations for converting between
the conventions, such as byte order (`endianness'),
used by the tool and OMPD library on the
one hand, and the OpenMP program on the other.

\subsubsubsection{\hcode{ompd_callback_device_host_fn_t}}
\label{subsubsubsec:ompd_callback_device_host_fn_t}
\index{ompd_callback_device_host_fn_t@{\code{ompd_callback_device_host_fn_t}}}

\summary

The type signature of the callback provided by the tool the
OMPD library can use to convert data between the formats used by the
tool and OMPD library, and the OpenMP program.


% the odd-looking spacing between type and argument name ensures
% they line up in the pdf
\begin{cspecific}
\begin{ompSyntax}
typedef ompd_rc_t (*ompd_callback_device_host_fn_t) (
  ompd_address_space_context_t *\plc{address_space_context},
  const void *\plc{input},
  ompd_size_t \plc{unit_size},
  ompd_size_t \plc{count},
  void *\plc{output}
);
\end{ompSyntax}
\end{cspecific}

\descr
This callback function may be called by the OMPD library to convert
data between formats used by the tool and OMPD library, and the OpenMP program.
\argdesc
The OpenMP address space associated with the data is given by
\plc{address_space_context}.
The source and destination buffers are given by \plc{input}
and \plc{output}, respectively.
\plc{unit_size} gives the size of each of the elements to be converted.
\plc{count} is the number of elements to be transformed.

The input and output buffers are allocated and owned by the OMPD library,
and it is its responsibility to ensure that the buffers are the correct
size, and eventually deallocated when they are no longer needed.

\crossreferences
\begin{itemize}
\item
  \code{ompd_address_space_context_t}, \specref{subsubsec:ompd_address_space_context_t}
\item
  \code{ompd_rc_t}, \specref{subsubsec:ompd_rc_t}
\item
  \code{ompd_callbacks_t}, \specref{subsubsec:ompd_callbacks_t}
\item
  \code{ompd_size_t}, \specref{subsubsubsec:ompd_size_t}
\end{itemize}

\subsubsection{Output}
\label{subsubsec:output}

\subsubsubsection{\hcode{ompd_callback_print_string_fn_t}}
\label{subsubsubsec:ompd_callback_print_string_fn_t}
\index{ompd_callback_print_string_fn_t@{\code{ompd_callback_print_string_fn_t}}}

\summary

The type signature of the callback provided by the tool the
OMPD library can use to emit output.

% the odd-looking spacing between type and argument name ensures
% they line up in the pdf
\begin{cspecific}
\begin{ompSyntax}
typedef ompd_rc_t (*ompd_callback_print_string_fn_t) (
  const char *\plc{string},
  int \plc{category}
);
\end{ompSyntax}
\end{cspecific}

\descr

The OMPD library may call the \code{ompd_callback_print_string_fn_t} callback function to emit 
output, such as logging or debug information. If the tool does not want to allow the OMPD library to 
emit output, the tool can provide to the OMPD library a \code{NULL} value for the  
\code{ompd_callback_print_string_fn_t} callback function. Note that the OMPD library is prohibited 
from writing to file descriptors that it did not open.

\argdesc
The input \plc{string} parameter is the null-terminated string to be printed.
No conversion or formating is performed on the string.

The input \plc{category} parameter is the category of the string to be printed. The value of 
\plc{category} is implementation defined.

\crossreferences
\begin{itemize}
\item
  \code{ompd_rc_t}, \specref{subsubsec:ompd_rc_t}
\item
  \code{ompd_callbacks_t}, \specref{subsubsec:ompd_callbacks_t}
\end{itemize}

\subsubsection{The Callback Interface}
\label{subsubsec:ompd_callbacks_t}
\index{ompd_callbacks_t@{\code{ompd_callbacks_t}}}

\summary

All the OMPD library's interactions with the OpenMP program
must be through a set of callbacks provided to it by the
tool which loaded it.
These callbacks must also be used for allocating or releasing resources,
such as memory, that the library needs.


% the odd-looking spacing between type and argument name ensures
% they line up in the pdf
\begin{cspecific}
{\small
\begin{ompSyntax}
typedef struct {
  ompd_callback_memory_alloc_fn_t \plc{alloc_memory};
  ompd_callback_memory_free_fn_t \plc{free_memory};
  ompd_callback_print_string_fn_t \plc{print_string};
  ompd_callback_sizeof_fn_t \plc{sizeof_type};
  ompd_callback_symbol_addr_fn_t \plc{symbol_addr_lookup};
  ompd_callback_memory_read_fn_t \plc{read_memory};
  ompd_callback_memory_write_fn_t \plc{write_memory};
  ompd_callback_memory_read_fn_t \plc{read_string};
  ompd_callback_device_host_fn_t \plc{device_to_host};
  ompd_callback_device_host_fn_t \plc{host_to_device};
  ompd_callback_get_thread_context_for_thread_id_fn_t
    \plc{get_thread_context_for_thread_id};
} ompd_callbacks_t;
\end{ompSyntax}
}
\end{cspecific}


\descr
The set of callbacks the OMPD library should use is collected
in the \code{ompd_callbacks_t} record structure.
An instance of this type is passed to the OMPD library
as a parameter to \code{ompd_initialize} (see~\specref{subsubsubsec:ompd_initialize}).
Each field points to a function that the OMPD library should use
for interacting with the OpenMP program, or getting memory from
the tool.

The \plc{alloc_memory} and \plc{free_memory} fields are
pointers to functions the OMPD library uses to allocate and release
dynamic memory.

\plc{print_string} points to a function that prints a string.

The architectures or programming models of the OMPD library and
party tool may be different from that of the OpenMP
program being examined.
\plc{sizeof_type} points to function that allows
the OMPD library to determine the sizes of the basic integer
and pointer types used by the OpenMP program.
Because of the differences in architecture or programming model,
the conventions for representing data in the OMPD library and
the OpenMP program may be different.
The \plc{device_to_host} field points to a function which translates
data from the conventions used by the OpenMP program to that used
by the tool and OMPD library.
The reverse operation is performed by the function pointed
to by the \plc{host_to_device} field.

The OMPD library may need to access memory in the OpenMP program.
The \plc{symbol_addr_lookup} field points to a callback the
OMPD library can use to find the address of a global or thread
local storage (TLS) symbol.
The \plc{read_memory}, \plc{write_memory} and \plc{read_string} fields are
pointers to functions for reading from, and writing to, global or TLS
memory in the OpenMP program, respectively.

\plc{get_thread_context_for_thread_id} is a pointer to a function
the OMPD library can use to obtain a thread context that corresponds to
a thread identifier.


\crossreferences
\begin{itemize}
\item
  \code{ompd_callback_memory_alloc_fn_t}, \specref{subsubsubsec:ompd_callback_memory_alloc_fn_t}
\item
  \code{ompd_callback_memory_free_fn_t}, \specref{subsubsubsec:ompd_callback_memory_free_fn_t}
\item
  \code{ompd_callback_print_string_fn_t}, \specref{subsubsubsec:ompd_callback_print_string_fn_t}
\item
  \code{ompd_callback_sizeof_fn_t}, \specref{subsubsubsec:ompd_callback_sizeof_fn_t}
\item
  \code{ompd_callback_symbol_addr_fn_t}, \specref{subsubsubsec:ompd_callback_symbol_addr_fn_t}
\item
  \code{ompd_callback_memory_read_fn_t}, \specref{subsubsubsec:ompd_callback_memory_read_fn_t}
\item
  \code{ompd_callback_memory_write_fn_t}, \specref{subsubsubsec:ompd_callback_memory_write_fn_t}
\item
  \code{ompd_callback_device_host_fn_t}, \specref{subsubsubsec:ompd_callback_device_host_fn_t}
\item
  \code{ompd_callback_get_thread_context_for_thread_id_fn_t}, \specref{subsubsubsec:ompd_callback_get_thread_context_for_thread_id_fn_t}
\end{itemize}




\section{OMPD Tool Interface Routines}
\label{subsec:ompd-api}

\subsection{Per OMPD Library Initialization and Finalization}

The OMPD library must be initialized exactly once after it is loaded, 
and finalized exactly once before it is unloaded. Per OpenMP process 
or core file initialization and finalization are also required.

Once loaded, the tool can determine the version of the OMPD API that 
the library supports by calling \code{ompd_get_api_version} (see
\specref{subsubsubsec:ompd_get_api_version}). If the tool supports the 
version that \code{ompd_get_api_version} returns, the tool starts the 
initialization by calling \code{ompd_initialize} (see
\specref{subsubsubsec:ompd_initialize}) using the version of the OMPD API 
that the library supports. If the tool does not support the version that
\code{ompd_get_api_version} returns, it may attempt to call 
\code{ompd_initialize} with a different version.



\subsubsection{\hcode{ompd_initialize}}
\label{subsubsubsec:ompd_initialize}

\summary
The \code{ompd_initialize} function initializes the OMPD library.

\format
\begin{cspecific}
\begin{ompSyntax}
ompd_rc_t ompd_initialize(
  ompd_word_t \plc{api_version},
  const ompd_callbacks_t *\plc{callbacks}
);
\end{ompSyntax}
\end{cspecific}

\descr
A tool that uses OMPD calls \code{ompd_initialize} to initialize each OMPD 
library that it loads. More than one library may be present in a third-party 
tool, such as a debugger, because the tool may control multiple devices, which
may use different runtime systems that require different OMPD libraries. This 
initialization must be performed exactly once before the tool can begin to 
operate on an OpenMP process or core file.

\argdesc
The \plc{api_version} argument is the OMPD API version that the tool requests
to use. The tool may call \code{ompd_get_api_version} to obtain the latest 
version that the OMPD library supports.

The tool provides the OMPD library with a set of callback functions in the 
\plc{callbacks} input argument which enables the OMPD library to allocate and 
to deallocate memory in the tool's address space, to lookup the sizes of basic 
primitive types in the device, to lookup symbols in the device, and to read and
to write memory in the device.

\crossreferences
\begin{itemize}
\item \code{ompd_rc_t} type, see \specref{subsubsec:ompd_rc_t}.

\item \code{ompd_callbacks_t} type, see \specref{subsubsec:ompd_callbacks_t}.

\item \code{ompd_get_api_version} call, 
see \specref{subsubsubsec:ompd_get_api_version}.
\end{itemize}



\subsubsection{\hcode{ompd_get_api_version}}
\label{subsubsubsec:ompd_get_api_version}

\summary
The \code{ompd_get_api_version} function returns the OMPD API version.

\format
\begin{cspecific}
\begin{ompSyntax}
ompd_rc_t ompd_get_api_version(ompd_word_t *\plc{version});
\end{ompSyntax}
\end{cspecific}

\descr
The tool may call the \code{ompd_get_api_version} function to obtain the 
latest OMPD API version number of the OMPD library.

\argdesc
The latest version number is returned into the location to which the 
\plc{version} argument points.

\crossreferences
\begin{itemize}
\item \code{ompd_rc_t} type, see \specref{subsubsec:ompd_rc_t}.
\end{itemize}



\subsubsection{\hcode{ompd_get_version_string}}
\label{subsubsubsec:ompd_get_version_string}

\summary
The \code{ompd_get_version_string} function returns a descriptive 
string for the OMPD API version.

\format
\begin{cspecific}
\begin{ompSyntax}
ompd_rc_t ompd_get_version_string(const char **\plc{string});
\end{ompSyntax}
\end{cspecific}


\descr
The tool may call this function to obtain a pointer to a descriptive 
version string of the OMPD API  version.

\argdesc
A pointer to a descriptive version string is placed into the location
to which \plc{string} output argument points. The OMPD library owns the string 
that the OMPD library returns; the tool must not modify or release this string.
The string remains valid for as long as the library is loaded. The
\code{ompd_get_version_string} function may be called before 
\code{ompd_initialize} (see \specref{subsubsubsec:ompd_initialize}).
Accordingly, the OMPD library must not use heap or stack memory for the string.

The signatures of \code{ompd_get_api_version} (see
\specref{subsubsubsec:ompd_get_api_version}) and \code{ompd_get_version_string} 
are guaranteed not to change in future versions of the API. In contrast, the 
type definitions and prototypes in the rest of the API do not carry the same 
guarantee. Therefore a tool that uses OMPD should check the version of the API 
of the loaded OMPD library before it calls any other function of the API.

\crossreferences
\begin{itemize}
\item \code{ompd_rc_t} type, see \specref{subsubsec:ompd_rc_t}.
\end{itemize}



\subsubsection{\hcode{ompd_finalize}}
\label{subsubsubsec:ompd_finalize}

\summary
When the tool is finished with the OMPD library it should call 
\code{ompd_finalize} before it unloads the library.

\format
\begin{cspecific}
\begin{ompSyntax}
ompd_rc_t ompd_finalize(void);
\end{ompSyntax}
\end{cspecific}

\descr
The call to \code{ompd_finalize} must be the last OMPD call that the tool 
makes before it unloads the library. This call allows the OMPD library to
free any resources that it may be holding.

The OMPD library may implement a \emph{finalizer} section, which executes
as the library is unloaded and therefore after the call to \code{ompd_finalize}. 
During finalization, the OMPD library may use the callbacks that the tool  
earlier provided after the call to \code{ompd_initialize}.

\crossreferences
\begin{itemize}
\item \code{ompd_rc_t} type, see \specref{subsubsec:ompd_rc_t}.
\end{itemize}



\subsection{Per OpenMP Process Initialization and Finalization}



\subsubsection{\hcode{ompd_process_initialize}}
\label{subsubsubsec:ompd_process_initialize}
\summary
A tool calls \code{ompd_process_initialize} to obtain an address space 
handle when it initializes a session on a live process or core file.

\format
\begin{cspecific}
\begin{ompSyntax}
ompd_rc_t ompd_process_initialize(
  ompd_address_space_context_t *\plc{context},
  ompd_address_space_handle_t **\plc{handle}
);
\end{ompSyntax}
\end{cspecific}

\descr
A tool calls \code{ompd_process_initialize} to obtain an address space 
handle when it initializes a session on a live process or core file.
On return from \code{ompd_process_initialize}, the tool owns the address
space handle, which it must release with \code{ompd_rel_address_space_handle}.
The initialization function must be called before any OMPD operations
are performed on the OpenMP process. This call allows the OMPD library 
to confirm that it can handle the OpenMP process or core file that the 
\plc{context} identifies. Incompatibility is signaled by a 
return value of \code{ompd_rc_incompatible}. 

\argdesc
The \plc{context} argument is an opaque handle that the tool provides to 
address an address space. On return, the \plc{handle} argument provides 
an opaque handle to the tool for this address space, which the tool must
release when it is no longer needed.

\crossreferences
\begin{itemize}
\item \code{ompd_address_space_handle_t} type, 
see \specref{subsubsec:ompd_address_space_handle_t}.

\item \code{ompd_address_space_context_t} type, 
see \specref{subsubsec:ompd_address_space_context_t}.

\item \code{ompd_rc_t} type, see \specref{subsubsec:ompd_rc_t}.

\item \code{ompd_rel_address_space_handle} type, 
see \specref{subsubsubsec:ompd_rel_address_space_handle}.
\end{itemize}



\subsubsection{\hcode{ompd_device_initialize}}
\label{subsubsubsec:ompd_device_initialize}

\summary
A tool calls \code{ompd_device_initialize} to obtain an address space 
handle for a device that has at least one  active target region.

\format
\begin{cspecific}
\begin{ompSyntax}
ompd_rc_t ompd_device_initialize(
  ompd_address_space_handle_t *\plc{process_handle},
  ompd_address_space_context_t *\plc{device_context},
  omp_device_t \plc{kind},
  ompd_size_t \plc{sizeof_id},
  void *\plc{id},
  ompd_address_space_handle_t **\plc{device_handle}
);
\end{ompSyntax}
\end{cspecific}

\descr
A tool calls \code{ompd_device_initialize} to obtain an address space 
handle for a device that has at least one  active target region. On return 
from \code{ompd_device_initialize}, the tool owns the address space handle.

\argdesc
The \plc{process_handle} argument is an opaque handle that the tool provides
to reference the address space of the OpenMP process. The \plc{device_context} 
argument is an opaque handle that the tool provides to reference a device 
address space. The \plc{kind}, \plc{sizeof_id}, and \plc{id} arguments represent 
a device identifier. On return the \plc{device_handle} argument provides an 
opaque handle to the tool for this address space.

\crossreferences
\begin{itemize}
\item \code{ompd_size_t} type, see \specref{subsubsubsec:ompd_size_t}.

\item \code{omp_device_t} type, see \specref{ompd:omp_device_t}.

\item \code{ompd_address_space_handle_t} type, 
see \specref{subsubsec:ompd_address_space_handle_t}.

\item \code{ompd_address_space_context_t} type, 
see \specref{subsubsec:ompd_address_space_context_t}.

\item \code{ompd_rc_t} type, see \specref{subsubsec:ompd_rc_t}.
\end{itemize}



\subsubsection{\hcode{ompd_rel_address_space_handle}}
\label{subsubsubsec:ompd_rel_address_space_handle}

\summary
A tool calls \code{ompd_rel_address_space_handle} to release an address space
handle.

\format
\begin{cspecific}
\begin{ompSyntax}
ompd_rc_t ompd_rel_address_space_handle(
  ompd_address_space_handle_t *\plc{handle}
);
\end{ompSyntax}
\end{cspecific}

\descr
When the tool is finished with the OpenMP process address space handle it
should call \code{ompd_rel_address_space_handle} to release the handle, 
which allows the OMPD library to release any resources that it has related 
to the address space.

\argdesc
The \plc{handle} argument is an opaque handle for the address space to be released.

\restrictions
The \code{ompd_rel_address_space_handle} has the following restriction:

\begin{itemize}
\item An address space context must not be used after the corresponding 
      address space handle is released.
\end{itemize}

\crossreferences
\begin{itemize}
\item \code{ompd_address_space_handle_t} type, 
see \specref{subsubsec:ompd_address_space_handle_t}.

\item \code{ompd_rc_t} type, see \specref{subsubsec:ompd_rc_t}.
\end{itemize}



\subsection{Thread and Signal Safety}

The OMPD library does not need to be reentrant. The tool must ensure that only 
one thread enters the OMPD library at a time. The OMPD library must not install 
signal handlers or otherwise interfere with the tool's signal configuration.



\subsection{Address Space Information}

\subsubsection{\hcode{ompd_get_omp_version}}
\label{subsubsubsec:ompd_get_omp_version}

\summary
The tool may call the \code{ompd_get_omp_version} function to obtain the version 
of the OpenMP API that is associated with an address space.

\format
\begin{cspecific}
\begin{ompSyntax}
ompd_rc_t ompd_get_omp_version(
  ompd_address_space_handle_t *\plc{address_space},
  ompd_word_t *\plc{omp_version}
);
\end{ompSyntax}
\end{cspecific}

\descr
The tool may call the \code{ompd_get_omp_version} function to obtain the 
version of the OpenMP API that is associated with the address space.

\argdesc
The \plc{address_space} argument is an opaque handle that the tool provides
to reference the address space of the OpenMP process or device.

Upon return, the \plc{omp_version} argument contains the version of the OpenMP 
runtime in the \code{_OPENMP} version macro format.

\crossreferences
\begin{itemize}
\item \code{ompd_address_space_handle_t} type, 
see \specref{subsubsec:ompd_address_space_handle_t}.

\item \code{ompd_rc_t} type, see \specref{subsubsec:ompd_rc_t}.
\end{itemize}



\subsubsection{\hcode{ompd_get_omp_version_string}}
\label{subsubsubsec:ompd_get_omp_version_string}

\summary
The \code{ompd_get_omp_version_string} function returns a descriptive 
string for the OpenMP API version that is associated with an address space.

\format
\begin{cspecific}
\begin{ompSyntax}
ompd_rc_t ompd_get_omp_version_string(
  ompd_address_space_handle_t *\plc{address_space},
  const char **\plc{string}
);
\end{ompSyntax}
\end{cspecific}

\descr
After initialization, the tool may call the \code{ompd_get_omp_version_string} 
function to obtain the version of the OpenMP API that is associated with an 
address space.

\argdesc
The \plc{address_space} argument is an opaque handle that the tool provides
to reference the address space of the OpenMP process or device. A pointer to 
a descriptive version string is placed into the location to which the 
\plc{string} output argument points. After returning from the call, the tool 
owns the string. The OMPD library must use the memory allocation callback that 
the tool provides to allocate the string storage. The tool is responsible 
for releasing the memory.

\crossreferences
\begin{itemize}
\item \code{ompd_address_space_handle_t} type, 
see \specref{subsubsec:ompd_address_space_handle_t}.

\item \code{ompd_rc_t} type, see \specref{subsubsec:ompd_rc_t}.
\end{itemize}



\subsection{Thread Handles}



\subsubsection{\hcode{ompd_get_thread_in_parallel}}
\label{subsubsubsec:ompd_get_thread_in_parallel}

\summary
The  \code{ompd_get_thread_in_parallel} function enables a tool to obtain 
handles for OpenMP threads that are associated with a parallel region.

\format
\begin{cspecific}
\begin{ompSyntax}
ompd_rc_t ompd_get_thread_in_parallel(
  ompd_parallel_handle_t *\plc{parallel_handle},
  int \plc{thread_num},
  ompd_thread_handle_t **\plc{thread_handle}
);
\end{ompSyntax}
\end{cspecific}


\descr
A successful invocation of  \code{ompd_get_thread_in_parallel} returns a 
pointer to a thread handle in the location to which \code{thread_handle}
points. This call yields  meaningful results only if all OpenMP threads 
in the parallel region are stopped.

\argdesc
The \plc{parallel_handle} argument is an opaque handle for a parallel region
and selects the parallel region on which to operate. The \plc{thread_num} 
argument selects the thread of the team to be returned. On return, the 
\plc{thread_handle} argument is an opaque handle for the selected thread.

\restrictions
The \code{ompd_get_thread_in_parallel} function has the following restriction:

\begin{itemize}
\item The value of \plc{thread_num} must be a non-negative integer smaller 
      than the team size that was provided as the \plc{ompd-team-size-var} 
      from \code{ompd_get_icv_from_scope}.
\end{itemize}

\crossreferences
\begin{itemize}
\item \code{ompd_parallel_handle_t} type, 
see \specref{subsubsec:ompd_parallel_handle_t}.

\item \code{ompd_thread_handle_t} type, 
see \specref{subsubsec:ompd_thread_handle_t}.

\item \code{ompd_rc_t} type, see \specref{subsubsec:ompd_rc_t}.

\item \code{ompd_get_icv_from_scope} call, 
see \specref{subsubsubsec:ompd_get_icv_from_scope}.
\end{itemize}



\subsubsection{\hcode{ompd_get_thread_handle}}
\label{subsubsubsec:ompd_get_thread_handle}

\summary
The \code{ompd_get_thread_handle} function maps a native thread 
to an OMPD thread handle.

\format
\begin{cspecific}
\begin{ompSyntax}
ompd_rc_t ompd_get_thread_handle(
  ompd_address_space_handle_t *\plc{handle},
  ompd_thread_id_t \plc{kind},
  ompd_size_t \plc{sizeof_thread_id},
  const void *\plc{thread_id},
  ompd_thread_handle_t **\plc{thread_handle}
);
\end{ompSyntax}
\end{cspecific}

\descr
The \code{ompd_get_thread_handle} function determines if the native thread 
identifier to which \plc{thread_id} points represents an OpenMP thread. If 
so, the function returns \code{ompd_rc_ok} and the location to which 
\plc{thread_handle} points is set to the thread handle for the OpenMP thread.

\argdesc
The \plc{handle} argument is an opaque handle that the tool provides
to reference an address space. The \plc{kind}, \plc{sizeof_thread_id}, 
and \plc{thread_id} arguments represent a native thread identifier.
On return, the \plc{thread_handle} argument provides an opaque handle 
to the thread within the provided address space.

The native thread identifier to which \plc{thread_id} points is guaranteed 
to be valid  for the duration of the call. If the OMPD library must retain 
the native thread identifier, it must copy it.

\crossreferences
\begin{itemize}
\item \code{ompd_size_t} type, see \specref{subsubsubsec:ompd_size_t}.

\item \code{ompd_thread_id_t} type, see \specref{ompd:ompd_thread_id_t}.

\item \code{ompd_address_space_handle_t} type, 
see \specref{subsubsec:ompd_address_space_handle_t}.

\item \code{ompd_thread_handle_t} type, 
see \specref{subsubsec:ompd_thread_handle_t}.

\item \code{ompd_rc_t} type, see \specref{subsubsec:ompd_rc_t}.
\end{itemize}



\subsubsection{\hcode{ompd_rel_thread_handle}}
\label{subsubsubsec:ompd_rel_thread_handle}
\summary
The \code{ompd_rel_thread_handle} function releases a thread handle.

\format
\begin{cspecific}
\begin{ompSyntax}
ompd_rc_t ompd_rel_thread_handle(
  ompd_thread_handle_t *\plc{thread_handle}
);
\end{ompSyntax}
\end{cspecific}

\descr
Thread handles are opaque to tools, which therefore cannot release them 
directly. Instead, when the tool is finished with a thread handle it must 
pass it to \code{ompd_rel_thread_handle} for disposal.

\argdesc
The \plc{thread_handle} argument is an opaque handle for a thread to be released.

\crossreferences
\begin{itemize}
\item \code{ompd_thread_handle_t} type, 
see \specref{subsubsec:ompd_thread_handle_t}.

\item \code{ompd_rc_t} type, see \specref{subsubsec:ompd_rc_t}.
\end{itemize}



\subsubsection{\hcode{ompd_thread_handle_compare}}
\label{subsubsubsec:ompd_thread_handle_compare}
\summary
The \code{ompd_thread_handle_compare} function allows tools to compare
two thread handles.

\format
\begin{cspecific}
\begin{ompSyntax}
ompd_rc_t ompd_thread_handle_compare(
  ompd_thread_handle_t *\plc{thread_handle_1},
  ompd_thread_handle_t *\plc{thread_handle_2},
  int *\plc{cmp_value}
);
\end{ompSyntax}
\end{cspecific}

\descr
The internal structure of thread handles is opaque to a tool. While the 
tool can easily compare pointers to thread handles, it cannot determine 
whether handles of two different addresses refer to the same underlying 
thread. The \code{ompd_thread_handle_compare} function compares thread handles.

On success, \code{ompd_thread_handle_compare} returns in the location to 
which \plc{cmp_value} points a signed integer value that indicates how the 
underlying threads compare: a value less than, equal to, or greater than 0 
indicates that the thread corresponding to \plc{thread_handle_1} is, respectively,
less than, equal to, or greater than that corresponding to \plc{thread_handle_2}.

\argdesc
The \plc{thread_handle_1} and \plc{thread_handle_2} arguments are opaque 
handles for threads. On return the \plc{cmp_value} argument is set to a 
signed integer value.

\crossreferences
\begin{itemize}
\item \code{ompd_thread_handle_t} type, 
see \specref{subsubsec:ompd_thread_handle_t}.

\item \code{ompd_rc_t} type, see \specref{subsubsec:ompd_rc_t}.
\end{itemize}



\subsubsection{\hcode{ompd_get_thread_id}}
\label{subsubsubsec:ompd_get_thread_id}
\summary
The \code{ompd_get_thread_id} maps an OMPD thread handle to a native thread.

\format
\begin{cspecific}
\begin{ompSyntax}
ompd_rc_t ompd_get_thread_id(
  ompd_thread_handle_t *\plc{thread_handle},
  ompd_thread_id_t \plc{kind},
  ompd_size_t \plc{sizeof_thread_id},
  void *\plc{thread_id}
);
\end{ompSyntax}
\end{cspecific}

\descr
The \code{ompd_get_thread_id} function maps an OMPD thread handle to a 
native thread identifier.

\argdesc
The \plc{thread_handle} argument is an opaque thread handle. The \plc{kind} 
argument represents the native thread identifier. The \plc{sizeof_thread_id}
argument represents the size of the native thread identifier. On return, the 
\plc{thread_id} argument is a buffer that represents a native thread identifier.

\crossreferences
\begin{itemize}
\item \code{ompd_size_t} type, see \specref{subsubsubsec:ompd_size_t}.

\item \code{ompd_thread_id_t} type, see \specref{ompd:ompd_thread_id_t}.

\item \code{ompd_thread_handle_t} type, 
see \specref{subsubsec:ompd_thread_handle_t}.

\item \code{ompd_rc_t} type, see \specref{subsubsec:ompd_rc_t}.
\end{itemize}



\subsection{Parallel Region Handles}

\subsubsection{\hcode{ompd_get_curr_parallel_handle}}
\label{subsubsubsec:ompd_get_curr_parallel_handle}
\summary
The  \code{ompd_get_curr_parallel_handle} function obtains a pointer 
to the parallel handle for an OpenMP thread's current parallel region.

\format
\begin{cspecific}
\begin{ompSyntax}
ompd_rc_t ompd_get_curr_parallel_handle(
  ompd_thread_handle_t *\plc{thread_handle},
  ompd_parallel_handle_t **\plc{parallel_handle}
);
\end{ompSyntax}
\end{cspecific}

\descr
The  \code{ompd_get_curr_parallel_handle} function enables the tool 
to obtain a pointer to the parallel handle for the current parallel region 
that is associated with an OpenMP thread. This call is meaningful only if 
the associated thread is stopped. The parallel handle must be released by 
calling \code{ompd_rel_parallel_handle}.

\argdesc
The \plc{thread_handle} argument is an opaque handle for a thread and selects 
the thread on which to operate. On return, the \plc{parallel_handle} 
argument is set to a handle for the parallel region that the associated thread 
is currently executing, if any.

\crossreferences
\begin{itemize}
\item \code{ompd_thread_handle_t} type, 
see \specref{subsubsec:ompd_thread_handle_t}.

\item \code{ompd_parallel_handle_t} type, 
see \specref{subsubsec:ompd_parallel_handle_t}.

\item \code{ompd_rc_t} type, see \specref{subsubsec:ompd_rc_t}.

\item \code{ompd_rel_parallel_handle} call, 
see \specref{subsubsubsec:ompd_rel_parallel_handle}.
\end{itemize}



\subsubsection{\hcode{ompd_get_enclosing_parallel_handle}}
\label{subsubsubsec:ompd_get_enclosing_parallel_handle}

\summary
The  \code{ompd_get_enclosing_parallel_handle} function obtains a pointer 
to the parallel handle for an enclosing parallel region.

\format
\begin{cspecific}
\begin{ompSyntax}
ompd_rc_t ompd_get_enclosing_parallel_handle(
  ompd_parallel_handle_t *\plc{parallel_handle},
  ompd_parallel_handle_t **\plc{enclosing_parallel_handle}
);
\end{ompSyntax}
\end{cspecific}

\descr
The  \code{ompd_get_enclosing_parallel_handle} function enables a tool 
to obtain a pointer to the parallel handle for the parallel region that
encloses the parallel region that \code{parallel_handle} specifies. This 
call is meaningful only if at least one thread in the parallel region 
is stopped. A pointer to the parallel handle for the enclosing region 
is returned in the location to which \plc{enclosing_parallel_handle}
points. After the call, the tool owns the handle; the tool must release the 
handle with \code{ompd_rel_parallel_handle} when it is no longer required.

\argdesc
The \plc{parallel_handle} argument is an opaque handle for a parallel 
region that selects the parallel region on which to operate. On return, 
the \plc{enclosing_parallel_handle} argument is set to a handle for the 
parallel region that encloses the selected parallel region.

\crossreferences
\begin{itemize}
\item \code{ompd_parallel_handle_t} type, 
see \specref{subsubsec:ompd_parallel_handle_t}.

\item \code{ompd_rc_t} type, see \specref{subsubsec:ompd_rc_t}.

\item \code{ompd_rel_parallel_handle} call, 
see \specref{subsubsubsec:ompd_rel_parallel_handle}.
\end{itemize}



\subsubsection{\hcode{ompd_get_task_parallel_handle}}
\label{subsubsubsec:ompd_get_task_parallel_handle}
\summary
The  \code{ompd_get_task_parallel_handle} function obtains a pointer to 
the parallel handle for the parallel region that encloses a task region.

\format
\begin{cspecific}
\begin{ompSyntax}
ompd_rc_t ompd_get_task_parallel_handle(
  ompd_task_handle_t *\plc{task_handle},
  ompd_parallel_handle_t **\plc{task_parallel_handle}
);
\end{ompSyntax}
\end{cspecific}

\descr
The  \code{ompd_get_task_parallel_handle} function enables a tool to obtain a
pointer to the parallel handle for the parallel region that encloses the task 
region that \plc{task_handle} specifies. This call is meaningful only if at 
least one thread in the parallel region is stopped. A pointer to the parallel 
regions handle is returned in the location to which \plc{task_parallel_handle}
points. The tool owns that parallel handle, which it must release with 
\code{ompd_rel_parallel_handle}.

\argdesc
The \plc{task_handle} argument is an opaque handle that selects the task on 
which to operate. On return, the \plc{parallel_handle} argument is set 
to a handle for the parallel region that encloses the selected task.

\crossreferences
\begin{itemize}
\item \code{ompd_task_handle_t} type, see \specref{subsubsec:ompd_task_handle_t}.

\item \code{ompd_parallel_handle_t} type, 
see \specref{subsubsec:ompd_parallel_handle_t}.

\item \code{ompd_rc_t} type, see \specref{subsubsec:ompd_rc_t}.

\item \code{ompd_rel_parallel_handle} call, 
see \specref{subsubsubsec:ompd_rel_parallel_handle}.
\end{itemize}



\subsubsection{\hcode{ompd_rel_parallel_handle}}
\label{subsubsubsec:ompd_rel_parallel_handle}
\summary
The \code{ompd_rel_parallel_handle} function releases a parallel region handle.

\format
\begin{cspecific}
\begin{ompSyntax}
ompd_rc_t ompd_rel_parallel_handle(
  ompd_parallel_handle_t *\plc{parallel_handle}
);
\end{ompSyntax}
\end{cspecific}

\descr
Parallel region handles are opaque so tools cannot release them directly. 
Instead, a tool must pass a parallel region handle to the 
\code{ompd_rel_parallel_handle} function for disposal when finished with it.

\argdesc
The \plc{parallel_handle} argument is an opaque handle to be released.

\crossreferences
\begin{itemize}
\item \code{ompd_parallel_handle_t} type, 
see \specref{subsubsec:ompd_parallel_handle_t}.

\item \code{ompd_rc_t} type, see \specref{subsubsec:ompd_rc_t}.
\end{itemize}



\subsubsection{\hcode{ompd_parallel_handle_compare}}
\label{subsubsubsec:ompd_parallel_handle_compare}

\summary
The \code{ompd_parallel_handle_compare} function compares two parallel 
region handles.

\format
\begin{cspecific}
\begin{ompSyntax}
ompd_rc_t ompd_parallel_handle_compare(
  ompd_parallel_handle_t *\plc{parallel_handle_1},
  ompd_parallel_handle_t *\plc{parallel_handle_2},
  int *\plc{cmp_value}
);
\end{ompSyntax}
\end{cspecific}

\descr
The internal structure of parallel region handles is opaque to tools. While 
tools can easily compare pointers to parallel region handles, they cannot 
determine whether handles at two different addresses refer to the same 
underlying parallel region and, instead must use the 
\code{ompd_parallel_handle_compare} function.

On success, \code{ompd_parallel_handle_compare} returns a signed integer value 
in the location to which \plc{cmp_value} points that indicates how the underlying 
parallel regions compare. A value less than, equal to, or greater than 0 indicates
that the region corresponding to \plc{parallel_handle_1} is, respectively, less 
than, equal to, or greater than that corresponding to \plc{parallel_handle_2}.
This function is provided since the means by which parallel region handles are 
ordered is implementation defined.

\argdesc
The \plc{parallel_handle_1} and \plc{parallel_handle_2} arguments are 
opaque handles that correspond to parallel regions. On return the \plc{cmp_value}
argument points to a signed integer value that indicates how the underlying 
parallel regions compare.

\crossreferences
\begin{itemize}
\item \code{ompd_parallel_handle_t} type, 
see \specref{subsubsec:ompd_parallel_handle_t}.

\item \code{ompd_rc_t} type, see \specref{subsubsec:ompd_rc_t}.
\end{itemize}



\subsection{Task Handles}



\subsubsection{\hcode{ompd_get_curr_task_handle}}
\label{subsubsubsec:ompd_get_curr_task_handle}

\summary
The \code{ompd_get_curr_task_handle} function obtains a pointer to the task 
handle for the current task region that is associated with an OpenMP thread.

\format
\begin{cspecific}
\begin{ompSyntax}
ompd_rc_t ompd_get_curr_task_handle(
  ompd_thread_handle_t *\plc{thread_handle},
  ompd_task_handle_t **\plc{task_handle}
);
\end{ompSyntax}
\end{cspecific}

\descr
The \code{ompd_get_curr_task_handle} function obtains a pointer to the task 
handle for the current task region that is associated with an OpenMP thread.
This call is meaningful only if the thread for which the handle is provided 
is stopped. The task handle must be released with \code{ompd_rel_task_handle}.

\argdesc
The \plc{thread_handle} argument is an opaque handle that selects the thread 
on which to operate. On return, the \plc{task_handle} argument points to a 
location that points to a handle for the task that the thread is currently 
executing.

\crossreferences
\begin{itemize}
\item \code{ompd_thread_handle_t} type, 
see \specref{subsubsec:ompd_thread_handle_t}.

\item \code{ompd_task_handle_t} type, see \specref{subsubsec:ompd_task_handle_t}.

\item \code{ompd_rc_t} type, see \specref{subsubsec:ompd_rc_t}.

\item \code{ompd_rel_task_handle} call, 
see \specref{subsubsubsec:ompd_rel_task_handle}.
\end{itemize}



\subsubsection{\hcode{ompd_get_generating_task_handle}}
\label{subsubsubsec:ompd_get_generating_task_handle}

\summary
The \code{ompd_get_generating_task_handle} function obtains a pointer 
to the task handle of the generating task region.

\format
\begin{cspecific}
\begin{ompSyntax}
ompd_rc_t ompd_get_generating_task_handle(
  ompd_task_handle_t *\plc{task_handle},
  ompd_task_handle_t **\plc{generating_task_handle}
);
\end{ompSyntax}
\end{cspecific}


\descr
The \code{ompd_get_generating_task_handle} function obtains a pointer to 
the task handle for the task that encountered the OpenMP task construct 
that generated the task represented by \plc{task_handle}. The generating 
task is the OpenMP task that was active when the task specified by 
\plc{task_handle} was created. This call is meaningful only if the thread 
that is executing the task that \plc{task_handle} specifies is stopped. 
The generating task handle must be released with \code{ompd_rel_task_handle}.

\argdesc
The \plc{task_handle} argument is an opaque handle that selects the task on
which to operate. On return, the \plc{generating_task_handle} argument points
to a location that points to a handle for the generating task.

\crossreferences
\begin{itemize}
\item \code{ompd_task_handle_t} type, see \specref{subsubsec:ompd_task_handle_t}.

\item \code{ompd_rc_t} type, see \specref{subsubsec:ompd_rc_t}.

\item \code{ompd_rel_task_handle} call, 
see \specref{subsubsubsec:ompd_rel_task_handle}.
\end{itemize}



\subsubsection{\hcode{ompd_get_scheduling_task_handle}}
\label{subsubsubsec:ompd_get_scheduling_task_handle}
\summary
The \code{ompd_get_scheduling_task_handle} function obtains a task handle 
for the task that was active at a task scheduling point.

\format
\begin{cspecific}
\begin{ompSyntax}
ompd_rc_t ompd_get_scheduling_task_handle(
  ompd_task_handle_t *\plc{task_handle},
  ompd_task_handle_t **\plc{scheduling_task_handle}
);
\end{ompSyntax}
\end{cspecific}

\descr
The \code{ompd_get_scheduling_task_handle} function obtains a task handle 
for the task that was active when the task that \plc{task_handle} represents
was scheduled. This call is meaningful only if the thread that is executing 
the task that \plc{task_handle} specifies is stopped. The scheduling task 
handle must be released with \code{ompd_rel_task_handle}.

\argdesc
The \plc{task_handle} argument is an opaque handle for a task and selects 
the task on which to operate. On return, the \plc{scheduling_task_handle} 
argument points to a location that points to a handle for the task that is
still on the stack of execution on the same thread and was deferred in favor 
of executing the selected task.

\crossreferences
\begin{itemize}
\item \code{ompd_task_handle_t} type, see \specref{subsubsec:ompd_task_handle_t}.

\item \code{ompd_rc_t} type, see \specref{subsubsec:ompd_rc_t}.

\item \code{ompd_rel_task_handle} call, see 
\specref{subsubsubsec:ompd_rel_task_handle}.
\end{itemize}



\subsubsection{\hcode{ompd_get_task_in_parallel}}
\label{subsubsubsec:ompd_get_task_in_parallel}
\summary
The  \code{ompd_get_task_in_parallel} function obtains handles for
the implicit tasks that are associated with a parallel region.

\format
\begin{cspecific}
\begin{ompSyntax}
ompd_rc_t ompd_get_task_in_parallel(
  ompd_parallel_handle_t *\plc{parallel_handle},
  int \plc{thread_num},
  ompd_task_handle_t **\plc{task_handle}
);
\end{ompSyntax}
\end{cspecific}

\descr
The \code{ompd_get_task_in_parallel} function obtains handles for
the implicit tasks that are associated with a parallel region. A 
successful invocation of  \code{ompd_get_task_in_parallel} returns 
a pointer to a task handle in the location to which \plc{task_handle}
points. This call yields meaningful results only if all OpenMP threads 
in the parallel region are stopped.

\argdesc
The \plc{parallel_handle} argument is an opaque handle that selects the 
parallel region on which to operate. The \plc{thread_num} argument selects 
the implicit task of the team that is returned. The selected implicit task 
would return \plc{thread_num} from a call of the \code{omp_get_thread_num()} 
routine. On return, the \plc{task_handle} argument points to a location 
that points to an opaque handle for the selected implicit task.

\restrictions

The following restriction applies to the \code{ompd_get_task_in_parallel} function:

\begin{itemize}
\item The value of \plc{thread_num} must be a non-negative integer that is 
      smaller than the size of the team size that is the value of the 
      \plc{ompd-team-size-var} that \code{ompd_get_icv_from_scope} returns.
\end{itemize}

\crossreferences
\begin{itemize}
\item \code{ompd_parallel_handle_t} type, 
see \specref{subsubsec:ompd_parallel_handle_t}.

\item \code{ompd_task_handle_t} type, see \specref{subsubsec:ompd_task_handle_t}.

\item \code{ompd_rc_t} type, see \specref{subsubsec:ompd_rc_t}.

\item \code{ompd_get_icv_from_scope} call, 
see \specref{subsubsubsec:ompd_get_icv_from_scope}.
\end{itemize}



\subsubsection{\hcode{ompd_rel_task_handle}}
\label{subsubsubsec:ompd_rel_task_handle}
\summary
This \code{ompd_rel_task_handle} function releases a task handle.

\format
\begin{cspecific}
\begin{ompSyntax}
ompd_rc_t ompd_rel_task_handle(
  ompd_task_handle_t *\plc{task_handle}
);
\end{ompSyntax}
\end{cspecific}

\descr
Task handles are opaque so tools cannot release them directly. Instead, 
when a tool is finished with a task handle it must use the 
\code{ompd_rel_task_handle} function to release it.

\argdesc
The \plc{task_handle} argument is an opaque task handle to be released.

\crossreferences
\begin{itemize}
\item \code{ompd_task_handle_t} type, see \specref{subsubsec:ompd_task_handle_t}.

\item \code{ompd_rc_t} type, see \specref{subsubsec:ompd_rc_t}.
\end{itemize}



\subsubsection{\hcode{ompd_task_handle_compare}}
\label{subsubsubsec:ompd_task_handle_compare}
\summary
The \code{ompd_task_handle_compare} function compares task handles.

\format
\begin{cspecific}
\begin{ompSyntax}
ompd_rc_t ompd_task_handle_compare(
  ompd_task_handle_t *\plc{task_handle_1},
  ompd_task_handle_t *\plc{task_handle_2},
  int *\plc{cmp_value}
);
\end{ompSyntax}
\end{cspecific}


\descr
The internal structure of task handles is opaque so tools  cannot directly
determine if handles at two different addresses refer to the same underlying 
task. The \code{ompd_task_handle_compare} function compares task handles.
After a successful call to \code{ompd_task_handle_compare}, the value of 
the location to which \plc{cmp_value} points is a signed integer that 
indicates how the underlying tasks compare: a value less than, equal to, 
or greater than 0 indicates that the task that corresponds to \plc{task_handle_1} 
is, respectively, less than, equal to, or greater than the task that 
corresponds to \plc{task_handle_2}. The means by which task handles are 
ordered is implementation defined.

\argdesc
The \plc{task_handle_1} and \plc{task_handle_2} arguments are opaque handles
that correspond to tasks. On return, the \plc{cmp_value} argument points to
a location in which a signed integer value indicates how the underlying tasks 
compare.

\crossreferences
\begin{itemize}
\item \code{ompd_task_handle_t} type, see \specref{subsubsec:ompd_task_handle_t}.

\item \code{ompd_rc_t} type, see \specref{subsubsec:ompd_rc_t}.
\end{itemize}



\subsubsection{\hcode{ompd_get_task_function}}
\label{subsubsubsec:ompd_get_task_function}

\summary
This \code{ompd_get_task_function} function returns the entry point of 
the code that corresponds to the body of a task.

\format
\begin{cspecific}
\begin{ompSyntax}
ompd_rc_t ompd_get_task_function (
  ompd_task_handle_t *\plc{task_handle},
  ompd_address_t *\plc{entry_point}
);
\end{ompSyntax}
\end{cspecific}

\descr
The \code{ompd_get_task_function} function returns the entry point of the code
that corresponds to the body of code that the task executes.

\argdesc
The \plc{task_handle} argument is an opaque handle that selects the task 
on which to operate. On return, the \plc{entry_point} argument is set to
an address that describes the beginning of application code that executes 
the task region.

\crossreferences
\begin{itemize}
\item \code{ompd_address_t} type, see \specref{subsubsubsec:ompd_address_t}.

\item \code{ompd_task_handle_t} type, see \specref{subsubsec:ompd_task_handle_t}.

\item \code{ompd_rc_t} type, see \specref{subsubsec:ompd_rc_t}.
\end{itemize}



\subsubsection{\hcode{ompd_get_task_frame}}
\label{subsubsubsec:ompd_get_task_frame}
\summary
The \code{ompd_get_task_frame} function extracts the frame pointers of a task.

\format
\begin{cspecific}
\begin{ompSyntax}
ompd_rc_t ompd_get_task_frame (
  ompd_task_handle_t *\plc{task_handle},
  ompd_frame_info_t *\plc{exit_frame},
  ompd_frame_info_t *\plc{enter_frame}
);
\end{ompSyntax}
\end{cspecific}

\descr
An OpenMP implementation  maintains an \code{ompt_frame_t} object for every 
implicit or explicit task. The \code{ompd_get_task_frame} function extracts 
the \plc{enter_frame} and \plc{exit_frame} fields of the \code{ompt_frame_t} 
object of the task that \plc{task_handle} identifies.

\argdesc
The \plc{task_handle} argument specifies an OpenMP task. On return, the 
\plc{exit_frame} argument points to an \code{ompd_frame_info_t} object  
that has the frame information with the same semantics as the \plc{exit_frame} 
field in the \code{ompt_frame_t} object that is associated with the specified 
task. On return, the \plc{enter_frame} argument points to an 
\code{ompd_frame_info_t} object that has the frame information with the same 
semantics as the \plc{enter_frame} field in the \code{ompt_frame_t} object 
that is associated with the specified task. 

\crossreferences
\begin{itemize}
\item \code{ompt_frame_t} type, see \specref{sec:ompt_frame_t}.

\item \code{ompd_address_t} type, see \specref{subsubsubsec:ompd_address_t}.

\item \code{ompd_frame_info_t} type, see \specref{subsubsubsec:ompd_frame_info_t}.

\item \code{ompd_task_handle_t} type, see \specref{subsubsec:ompd_task_handle_t}.

\item \code{ompd_rc_t} type, see \specref{subsubsec:ompd_rc_t}.
\end{itemize}



\subsubsection{\hcode{ompd_enumerate_states}}
\label{subsubsubsec:ompd_enumerate_states}

\summary
The \code{ompd_enumerate_states} function enumerates thread states 
that an OpenMP implementation supports.

\format
\begin{cspecific}
\begin{ompSyntax}
ompd_rc_t ompd_enumerate_states (
  ompd_address_space_handle_t *\plc{address_space_handle},
  ompd_word_t \plc{current_state},
  ompd_word_t *\plc{next_state},
  const char **\plc{next_state_name},
  ompd_word_t *\plc{more_enums}
);
\end{ompSyntax}
\end{cspecific}

\descr
An OpenMP implementation may support only a subset of the states that 
the \code{ompt_state_t} enumeration type defines. In addition, an
OpenMP implementation may support implementation-specific states.
The \code{ompd_enumerate_states} call enables a tool to enumerate 
the thread states that an OpenMP implementation supports.

When the \plc{current_state} argument is a thread state that an OpenMP 
implementation supports, the call assigns the value and string name of 
the next thread state in the enumeration to the locations to which the 
\plc{next_state} and \plc{next_state_name} arguments point.

On return, the third-party tool owns the \plc{next_state_name} string.
The OMPD library allocates storage for the string with the memory allocation 
callback that the tool provides. The tool is responsible for releasing the memory.

On return, the location to which the \plc{more_enums} argument points has
the value $1$ whenever one or more states are left in the enumeration. On
return, the location to which the \plc{more_enums} argument points
has the value $0$ when \plc{current_state} is the last state in the enumeration.

\argdesc
The \plc{address_space_handle} argument identifies the address space. 
The \plc{current_state} argument must be a thread state that the OpenMP 
implementation supports. To begin enumerating the supported states, a 
tool should pass \code{ompt_state_undefined} as the value of \plc{current_state}.
Subsequent calls to \code{ompd_enumerate_states} by the tool should pass the
value that the call returned in the \plc{next_state} argument. On return, the 
\plc{next_state} argument points to an integer with the value of the next state 
in the enumeration. On return, the \plc{next_state_name} argument points to a 
character string that describes the next state. On return, the \plc{more_enums} 
argument points to an integer with a value of $1$ when more states are left to 
enumerate and a value of $0$ when no more states are left.

\constraints
Any string that is returned through the \plc{next_state_name} argument 
must be immutable and defined for the lifetime of program execution.

\crossreferences
\begin{itemize}
\item \code{ompt_state_t} type, see \specref{sec:thread-states}.

\item \code{ompd_address_space_handle_t} type, 
see \specref{subsubsec:ompd_address_space_handle_t}.

\item \code{ompd_rc_t} type, see \specref{subsubsec:ompd_rc_t}.
\end{itemize}



\subsubsection{\hcode{ompd_get_state}}
\label{subsubsubsec:ompd_get_state}
\summary
The \code{ompd_get_state} function obtains the state of a thread.

\format
\begin{cspecific}
\begin{ompSyntax}
ompd_rc_t ompd_get_state (
  ompd_thread_handle_t *\plc{thread_handle},
  ompd_word_t *\plc{state},
  ompt_wait_id_t *\plc{wait_id}
);
\end{ompSyntax}
\end{cspecific}

\descr
The \code{ompd_get_state} function returns the state of an OpenMP thread.

\argdesc
The \plc{thread_handle} argument identifies the thread. The \plc{state} 
argument represents the state of that thread as represented by a value 
that \code{ompd_enumerate_states} returns. On return, if the \plc{wait_id} 
argument is non-null then it points to a handle that corresponds to the 
\plc{wait_id} wait identifier of the thread. If the thread state is not 
one of the specified wait states, the value to which \plc{wait_id} 
points is undefined.

\crossreferences
\begin{itemize}
\item \code{ompd_wait_id_t} type, see \specref{subsubsubsec:ompd_wait_id_t}.

\item \code{ompd_thread_handle_t} type, 
see \specref{subsubsec:ompd_thread_handle_t}.

\item \code{ompd_rc_t} type, see \specref{subsubsec:ompd_rc_t}.

\item \code{ompd_enumerate_states} call, 
see \specref{subsubsubsec:ompd_enumerate_states}.
\end{itemize}



\subsection{Display Control Variables}



\subsubsection{\hcode{ompd_get_display_control_vars}}
\label{subsubsubsec:ompd_get_display_control_vars}

\summary
The \code{ompd_get_display_control_vars} function returns a list of 
name/value pairs for OpenMP control variables.

\format
\begin{cspecific}
\begin{ompSyntax}
ompd_rc_t ompd_get_display_control_vars (
  ompd_address_space_handle_t *\plc{address_space_handle},
  const char * const **\plc{control_vars}
);
\end{ompSyntax}
\end{cspecific}

\descr
The \code{ompd_get_display_control_vars} function returns a NULL-terminated 
vector of NULL-terminated strings of name/value pairs of control variables 
that have user controllable settings and are important to the operation or 
performance of an OpenMP runtime system. The control variables that this 
interface exposes include all OpenMP environment variables, settings that 
may come from vendor or platform-specific environment variables, and other 
settings that affect the operation or functioning of an OpenMP runtime.

The format of the strings is \code{name=a string}.

On return, the third-party tool owns the vector and the strings. The OMP library
must satisfy the termination constraints; it may use static or dynamic memory 
for the vector and/or the strings and is unconstrained in how it arranges them 
in memory. If it uses dynamic memory then the OMPD library must use the allocate 
callback that the tool provides to \code{ompd_initialize}. The tool must use 
\code{ompd_rel_display_control_vars()} to release the vector and the strings.

\argdesc
The \plc{address_space_handle} argument identifies the address space. On return, 
the  \plc{control_vars} argument points to the vector of display control variables.

\crossreferences
\begin{itemize}
\item \code{ompd_address_space_handle_t} type, 
see \specref{subsubsec:ompd_address_space_handle_t}.

\item \code{ompd_rc_t} type, see \specref{subsubsec:ompd_rc_t}.

\item \code{ompd_initialize} call, see \specref{subsubsubsec:ompd_initialize}.

\item \code{ompd_rel_display_control_vars} type, 
see \specref{subsubsubsec:ompd_rel_display_control_vars}.
\end{itemize}



\subsubsection{\hcode{ompd_rel_display_control_vars}}
\label{subsubsubsec:ompd_rel_display_control_vars}

\summary
The \code{ompd_rel_display_control_vars} releases a list of name/value pairs 
of OpenMP control variables previously acquired with 
\code{ompd_get_display_control_vars}.

\format
\begin{cspecific}
\begin{ompSyntax}
ompd_rc_t ompd_rel_display_control_vars (
  const char * const **\plc{control_vars}
);
\end{ompSyntax}
\end{cspecific}

\descr
The third-party tool owns the vector and strings that 
\code{ompd_get_display_control_vars} returns. The tool must call
\code{ompd_rel_display_control_vars} to release the vector and the strings.

\argdesc
The \plc{control_vars} argument is the vector of display control variables 
to be released.

\crossreferences
\begin{itemize}
 \item \code{ompd_rc_t} type, see \specref{subsubsec:ompd_rc_t}.

\item \code{ompd_get_display_control_vars} call, 
see \specref{subsubsubsec:ompd_get_display_control_vars}.
\end{itemize}



\subsection{Accessing Scope-Specific Information}



\subsubsection{\hcode{ompd_enumerate_icvs}}
\label{subsubsubsec:ompd_enumerate_icvs}

\summary
The \code{ompd_enumerate_icvs} function enumerates ICVs.

\format
\begin{cspecific}
\begin{ompSyntax}
ompd_rc_t ompd_enumerate_icvs (
  ompd_address_space_handle *\plc{handle}, 
  ompd_icv_id_t \plc{current},
  ompd_icv_id_t *\plc{next_id},
  const char **\plc{next_icv_name},
  ompd_scope_t *\plc{next_scope},
  int *\plc{more}
);
\end{ompSyntax}
\end{cspecific}

\descr
In addition to the ICVs listed in Table~\ref{tab:ICV Initial Values}, an 
OpenMP implementation must support the OMPD specific ICVs listed in 
Table~\ref{tab:OMPD internal varibales}. An OpenMP implementation may 
support additional implementation specific variables. An implementation 
may store ICVs in a different scope than Table~\ref{tab:Scopes of ICVs}
indicates. The \code{ompd_enumerate_icvs} function enables a tool to 
enumerate the ICVs that an OpenMP implementation supports and their related scopes.

When the \plc{current} argument is set to the identifier of a supported ICV, 
\code{ompd_enumerate_icvs} assigns the value, string name, and scope of the 
next ICV in the enumeration to the locations to which the \plc{next_id}, 
\plc{next_icv_name}, and \plc{next_scope} arguments point. On return, the
third-party tool owns the \plc{next_icv_name} string. The OMPD library uses
the  memory allocation callback that the tool provides to allocate the string 
storage; the tool is responsible for releasing the memory.

On return, the location to which the \plc{more} argument points has the value of 
$1$ whenever one or more ICV are left in the enumeration. on return, that location
has the value $0$ when \plc{current} is the last ICV in the enumeration.

\argdesc
The \plc{address_space_handle} argument identifies the address space. The 
\plc{current} argument must be an ICV that the OpenMP implementation supports.  
To begin enumerating the ICVs, a tool should pass \code{ompd_icv_undefined} as 
the value of \plc{current}. Subsequent calls to \code{ompd_enumerate_icvs} 
should pass the value returned by the call in the \plc{next_id} output argument.
On return, the \plc{next_id} argument points to an integer with the value of the
ID of the next ICV in the enumeration. On return, the \plc{next_icv} argument 
points to a character string with the name of the next ICV. On return, the 
\plc{next_scope} argument points to the scope enum value of the scope of the 
next ICV. On return, the \plc{more_enums} argument points to an integer with 
the value of $1$ when more ICVs are left to enumerate and the value of $0$ 
when no more ICVs are left.

\constraints
Any string that \plc{next_icv} returns must be immutable and defined
for the lifetime of a program execution.

\begin{table}[h!]
\caption{OMPD-specific ICVs\label{tab:OMPD internal varibales}}
\begin{tabular}{p{1.5in} p{0.5in} p{2.7in}}
\hline
\textsf{\textbf{Variable}} & \textsf{\textbf{Scope}} & \textsf{\textbf{Meaning}}\\
\hline
{\splc{ompd-num-procs-var}}  & device & return value of \scode{omp_get_num_procs()} 
                                        when executed on this device \\
{\splc{ompd-thread-num-var}} & task   & return value of \scode{omp_get_thread_num()}
                                        when executed in this task \\
{\splc{ompd-final-var}}      & task   & return value of \scode{omp_in_final()} when 
                                        executed in this task \\
{\splc{ompd-implicit-var}}   & task   & the task is an implicit task\\
{\splc{ompd-team-size-var}}  & team   & return value of 
                                        \scode{omp_get_num_threads()} 
                                        when executed in this team \\
\hline
\end{tabular}
\end{table}

\crossreferences
\begin{itemize}
\item \code{ompd_address_space_handle_t} type, 
see \specref{subsubsec:ompd_address_space_handle_t}.

\item \code{ompd_scope_t} type, see \specref{subsubsec:ompd_scope_t}.

\item \code{ompd_icv_id_t} type, see \specref{subsubsec:ompd_icv_id_t}.

\item \code{ompd_rc_t} type, see \specref{subsubsec:ompd_rc_t}.
\end{itemize}



\subsubsection{\hcode{ompd_get_icv_from_scope}}
\label{subsubsubsec:ompd_get_icv_from_scope}

\summary
The \code{ompd_get_icv_from_scope} function returns the value of an ICV. 

\format
\begin{cspecific}
\begin{ompSyntax}
ompd_rc_t ompd_get_icv_from_scope (
  void *\plc{handle}, 
  ompd_scope_t \plc{scope},
  ompd_icv_id_t \plc{icv_id},
  ompd_word_t *\plc{icv_value}
); 
\end{ompSyntax}
\end{cspecific}

\descr
The \code{ompd_get_icv_from_scope} function provides access to the ICVs
that \code{ompd_enumerate_icvs} identifies.

\argdesc
The \plc{handle} argument provides an OpenMP scope handle. The \plc{scope} 
argument specifies the kind of scope provided in \plc{handle}. The 
\plc{icv_id} argument specifies the ID of the requested ICV. On return, 
the \plc{icv_value} argument points to a location with the value of the 
requested ICV.

\constraints
If the ICV cannot be represented by an integer type value then the 
function returns \code{ompd_rc_incompatible}. 

The provided \plc{handle} must match the \plc{scope} as defined in 
\specref{subsubsec:ompd_icv_id_t}. 

The provided \plc{scope} must match the scope for \plc{icv_id} as requested by 
\code{ompd_enumerate_icvs}. 

\crossreferences
\begin{itemize}
\item \code{ompd_address_space_handle_t} type, 
see \specref{subsubsec:ompd_address_space_handle_t}.

\item \code{ompd_thread_handle_t} type, 
see \specref{subsubsec:ompd_thread_handle_t}.

\item \code{ompd_parallel_handle_t} type, 
see \specref{subsubsec:ompd_parallel_handle_t}.

\item \code{ompd_task_handle_t} type, see \specref{subsubsec:ompd_task_handle_t}.

\item \code{ompd_scope_t} type, see \specref{subsubsec:ompd_scope_t}.

\item \code{ompd_icv_id_t} type, see \specref{subsubsec:ompd_icv_id_t}.

\item \code{ompd_rc_t} type, see \specref{subsubsec:ompd_rc_t}.

\item \code{ompd_enumerate_icvs}, see \specref{subsubsubsec:ompd_enumerate_icvs}.
\end{itemize}



\subsubsection{\hcode{ompd_get_icv_string_from_scope}}
\label{subsubsubsec:ompd_get_icv_string_from_scope}

\summary
The \code{ompd_get_icv_string_from_scope} function returns the value of an ICV.

\format
\begin{cspecific}
\begin{ompSyntax}
ompd_rc_t ompd_get_icv_string_from_scope (
  void *\plc{handle},
  ompd_scope_t \plc{scope},
  ompd_icv_id_t \plc{icv_id},
  const char **\plc{icv_string}
); 
\end{ompSyntax}
\end{cspecific}

\descr
The \code{ompd_get_icv_string_from_scope} function provides access to 
the ICVs that \code{ompd_enumerate_icvs} identifies.

\argdesc
The \plc{handle} argument provides an OpenMP scope handle. The \plc{scope} 
argument specifies the kind of scope provided in \plc{handle}. The \plc{icv_id} 
argument specifies the ID of the requested ICV. On return, the \plc{icv_string} 
argument points to a string representation of the requested ICV.

On return, the third-party tool owns the \plc{icv_string} string. The OMPD
library allocates the string storage with the memory allocation callback that
the tool provides. The tool is responsible for releasing the memory.

\constraints
The provided \plc{handle} must match the \plc{scope} as defined in 
\specref{subsubsec:ompd_icv_id_t}. 

The provided \plc{scope} must match the scope for \plc{icv_id} as requested by 
\code{ompd_enumerate_icvs}. 

\crossreferences
\begin{itemize}
\item \code{ompd_address_space_handle_t} type, 
see \specref{subsubsec:ompd_address_space_handle_t}.

\item \code{ompd_thread_handle_t} type, 
see \specref{subsubsec:ompd_thread_handle_t}.

\item \code{ompd_parallel_handle_t} type, 
see \specref{subsubsec:ompd_parallel_handle_t}.

\item \code{ompd_task_handle_t} type, see \specref{subsubsec:ompd_task_handle_t}.

\item \code{ompd_scope_t} type, see \specref{subsubsec:ompd_scope_t}.

\item \code{ompd_icv_id_t} type, see \specref{subsubsec:ompd_icv_id_t}.

\item \code{ompd_rc_t} type, see \specref{subsubsec:ompd_rc_t}.

\item \code{ompd_enumerate_icvs}, see \specref{subsubsubsec:ompd_enumerate_icvs}.
\end{itemize}



\subsubsection{\hcode{ompd_get_tool_data}}
\label{subsubsubsec:ompd_get_tool_data}

\summary
The \code{ompd_get_tool_data} function provides access to the OMPT data variable 
stored for each OpenMP scope.

\format
\begin{cspecific}
\begin{ompSyntax}
ompd_rc_t ompd_get_tool_data(
  void* \plc{handle}, 
  ompd_scope_t \plc{scope},
  ompd_word_t *\plc{value},
  ompd_address_t *\plc{ptr}
);
\end{ompSyntax}
\end{cspecific}

\descr
The \code{ompd_get_tool_data} function provides access to the OMPT tool data
stored for each scope. If the runtime library does not support OMPT then the 
function returns \code{ompd_rc_unsupported}.

\argdesc
The \plc{handle} argument provides an OpenMP scope handle. The \plc{scope} 
argument specifies the kind of scope provided in \plc{handle}. On return, 
the \plc{value} argument points to the \plc{value} field of the \code{ompt_data_t} 
union stored for the selected scope. On return, the \plc{ptr} argument points to 
the  \plc{ptr} field of the \code{ompt_data_t} union stored for the selected scope.

\crossreferences
\begin{itemize}
\item \code{ompt_data_t} type, see \specref{sec:ompt_data_t}.

\item \code{ompd_address_space_handle_t} type, 
see \specref{subsubsec:ompd_address_space_handle_t}.

\item \code{ompd_thread_handle_t} type, 
see \specref{subsubsec:ompd_thread_handle_t}.

\item \code{ompd_parallel_handle_t} type, 
see \specref{subsubsec:ompd_parallel_handle_t}.

\item \code{ompd_task_handle_t} type, see \specref{subsubsec:ompd_task_handle_t}.

\item \code{ompd_scope_t} type, see \specref{subsubsec:ompd_scope_t}.

\item \code{ompd_rc_t} type, see \specref{subsubsec:ompd_rc_t}.
\end{itemize}


%OMPD-TODO: OMPD-tr §7-11, §13 to go here

% %%%%%%%%%%%%%%%%%%%%%%%%%%%%%%%%%%%%%%%%%%%%%%%%%%%%%%%%%%%%%%%%%%%%%%%%%%%

%%%% \ompdsection{OMPD Tool Callback Interface}
%%%% \label{sec:ompd-callbacks}
%OMPD-TODO: OMPD-tr §15.10, §16 to go here

% %%%%%%%%%%%%%%%%%%%%%%%%%%%%%%%%%%%%%%%%%%%%%%%%%%%%%%%%%%%%%%%%%%%%%%%%%%%

\subsection{Runtime Entry Points for OMPD}
\label{subsec:runtime-entry-points-for-ompd}

The OpenMP implementation must define several entry point symbols 
through which execution must pass when particular events occur
\emph{and} data collection for OMPD is enabled. A tool can enable
notification of an event by setting a breakpoint at the address of 
the entry point symbol.

Entry point symbols have external \code{C} linkage and do not
require demangling or other transformations to look up their 
names to obtain the address in the OpenMP program. While each 
entry point symbol conceptually has a function type signature, 
it may not be a function. It may be a labeled location



\subsubsection{Beginning Parallel Regions}
\label{subsubsec:ompd_bp_parallel_begin}
\index{ompd_bp_parallel_begin@{\code{ompd_bp_parallel_begin}}}

\summary
Before starting the execution of an OpenMP parallel region, the 
implementation executes \code{ompd_bp_parallel_begin}.

\format
\begin{cspecific}
\begin{ompSyntax}
void ompd_bp_parallel_begin ( void );
\end{ompSyntax}
\end{cspecific}

\descr
The OpenMP implementation must execute \code{ompd_bp_parallel_begin} 
at every \plc{parallel-begin} event. At the point that the implementation 
reaches \code{ompd_bp_parallel_begin}, the binding for 
\code{ompd_get_curr_parallel_handle} is the parallel region that is 
beginning and the binding for \code{ompd_get_curr_task_handle}
is the task that encountered the parallel construct.

\crossreferences
\begin{itemize}
\item \code{parallel} construct, see \specref{sec:parallel Construct}.

\item \code{ompd_get_curr_parallel_handle}, 
see \specref{subsubsubsec:ompd_get_curr_parallel_handle}.

\item \code{ompd_get_curr_task_handle}, 
see \specref{subsubsubsec:ompd_get_curr_task_handle}.
\end{itemize}



\subsubsection{Ending Parallel Regions}
\label{subsubsec:ompd_bp_parallel_end}
\index{ompd_bp_parallel_end@{\code{ompd_bp_parallel_end}}}

\summary
After finishing the execution of an OpenMP parallel region, 
the implementation executes \code{ompd_bp_parallel_end}.

\format
\begin{cspecific}
\begin{ompSyntax}
void ompd_bp_parallel_end ( void );
\end{ompSyntax}
\end{cspecific}


\descr
The OpenMP implementation must execute \code{ompd_bp_parallel_end} at 
every \plc{parallel-end} event. At the point that the implementation reaches 
\code{ompd_bp_parallel_end}, the binding for \code{ompd_get_curr_parallel_handle} 
is the parallel region that is ending and the binding for 
\code{ompd_get_curr_task_handle} is the task that encountered the 
parallel construct. After executing \code{ompd_bp_parallel_end}, any 
\plc{parallel_handle} that was acquired for the parallel region is 
invalid and should be released.

\crossreferences
\begin{itemize}
\item \code{parallel} construct, see \specref{sec:parallel Construct}.

\item \code{ompd_get_curr_parallel_handle}, 
see \specref{subsubsubsec:ompd_get_curr_parallel_handle}.

\item \code{ompd_rel_parallel_handle}, 
see \specref{subsubsubsec:ompd_rel_parallel_handle}.

\item \code{ompd_get_curr_task_handle}, 
see \specref{subsubsubsec:ompd_get_curr_task_handle}.
\end{itemize}



\subsubsection{Beginning Task Regions}
\label{subsubsec:ompd_bp_task_begin}
\index{ompd_bp_task_begin@{\code{ompd_bp_task_begin}}}

\summary
Before starting the execution of an OpenMP task region, 
the implementation executes \code{ompd_bp_task_begin}.

\format
\begin{cspecific}
\begin{ompSyntax}
void ompd_bp_task_begin ( void );
\end{ompSyntax}
\end{cspecific}

\descr
The OpenMP implementation must execute \code{ompd_bp_task_begin} 
immediately before starting execution of a \plc{structured-block} 
associated with a non-merged task. At the point that the implementation 
reaches \code{ompd_bp_task_begin}, the binding for 
\code{ompd_get_curr_task_handle} is the task that is scheduled to execute.

\crossreferences
\begin{itemize}
\item \code{ompd_get_curr_task_handle}, 
see \specref{subsubsubsec:ompd_get_curr_task_handle}.
\end{itemize}



\subsubsection{Ending Task Regions}
\label{subsubsec:ompd_bp_task_end}
\index{ompd_bp_task_end@{\code{ompd_bp_task_end}}}

\summary
After finishing the execution of an OpenMP task region, 
the implementation executes \code{ompd_bp_task_end}.

\format
\begin{cspecific}
\begin{ompSyntax}
void ompd_bp_task_end ( void );
\end{ompSyntax}
\end{cspecific}

\descr
The OpenMP implementation must execute \code{ompd_bp_task_end} immediately
after completion of a \plc{structured-block} associated with a non-merged task.
At the point that the implementation reaches \code{ompd_bp_task_end}, the 
binding for \code{ompd_get_curr_task_handle} is the task that finished execution.
After executing \code{ompd_bp_task_end}, any \plc{task_handle} that was acquired 
for the task region is invalid and should be released.

\crossreferences
\begin{itemize}
\item \code{ompd_get_curr_task_handle}, 
see \specref{subsubsubsec:ompd_get_curr_task_handle}.

\item \code{ompd_rel_task_handle}, see \specref{subsubsubsec:ompd_rel_task_handle}.
\end{itemize}



\subsubsection{Beginning OpenMP Thread}
\label{subsubsec:ompd_bp_thread_begin}
\index{ompd_bp_thread_begin@{\code{ompd_bp_thread_begin}}}

\summary

When starting an OpenMP thread, the implementation executes
\code{ompd_bp_thread_begin}.

\format
\begin{cspecific}
\begin{ompSyntax}
void ompd_bp_thread_begin ( void );
\end{ompSyntax}
\end{cspecific}


\descr

The OpenMP implementation must execute 
\code{ompd_bp_thread_begin} at every \plc{native-thread-begin} and \plc{initial-thread-begin} event.
This must occur before the thread starts the execution of any
OpenMP region.

\crossreferences
\begin{itemize}
\item
  \code{parallel} construct, \specref{sec:parallel Construct}
\item
  Initial task, \specref{subsec:Initial Task}
\end{itemize}



\subsubsection{Ending OpenMP Thread}
\label{subsubsec:ompd_bp_thread_end}
\index{ompd_bp_thread_end@{\code{ompd_bp_thread_end}}}

\summary

When terminating an OpenMP thread, the implementation 
executes \code{ompd_bp_thread_end}.

\format
\begin{cspecific}
\begin{ompSyntax}
void ompd_bp_thread_end ( void );
\end{ompSyntax}
\end{cspecific}


\descr

The OpenMP implementation must execute 
\code{ompd_bp_thread_end} at every \plc{native-thread-end} and the \plc{initial-thread-end} event.
This must occur after the thread completes the execution of any OpenMP region.

After 
executing \code{ompd_bp_thread_end}, any \plc{thread_handle} acquired for this thread 
is invalid and should be released.


\crossreferences
\begin{itemize}
\item
  \code{parallel} construct, \specref{sec:parallel Construct}
\item
  Initial task, \specref{subsec:Initial Task}
\item
  \code{ompd_rel_thread_handle}, \specref{subsubsubsec:ompd_rel_thread_handle}
\end{itemize}






\subsubsection{Beginning OpenMP Device}
\label{subsubsec:ompd_bp_device_begin}
\index{ompd_bp_device_begin@{\code{ompd_bp_device_begin}}}

\summary
The OpenMP implementation must execute 
\code{ompd_bp_device_begin} at every \plc{device-initialize} event.


\format
\begin{cspecific}
\begin{ompSyntax}
void ompd_bp_device_begin ( void );
\end{ompSyntax}
\end{cspecific}


\descr

When initializing a device for executing a target region, the implementation must 
execute \code{ompd_bp_device_begin}.
This should occur before any OpenMP region's work executes on the device.

\crossreferences
\begin{itemize}
\item
  Device Initialization, \specref{subsec:Device Initialization}
\end{itemize}





\subsubsection{Ending OpenMP Device}
\label{subsubsec:ompd_bp_device_end}
\index{ompd_bp_device_end@{\code{ompd_bp_device_end}}}

\summary

When terminating an OpenMP thread, the implementation 
executes \code{ompd_bp_device_end}.

\format
\begin{cspecific}
\begin{ompSyntax}
void ompd_bp_device_end ( void );
\end{ompSyntax}
\end{cspecific}


\descr

The OpenMP implementation must execute 
\code{ompd_bp_device_end} at every \plc{device-finalize} event.
This should occur after the thread executes any OpenMP region.

After 
executing \code{ompd_bp_device_end}, any \plc{address_space_handle} acquired for this
device is invalid and should be released.

\crossreferences
\begin{itemize}
\item
  Device Initialization, \specref{subsec:Device Initialization}
\item
  \code{ompd_rel_address_space_handle},
\specref{subsubsubsec:ompd_rel_address_space_handle}
\end{itemize}




%\newpage %% HACK
\section{Tool Foundation}
\subsection{Data Types}
\subsection{\hcode{ompt_state_t}}
\label{sec:thread-states}
\label{sec:ompt_state_t}

\summary
If the OMPT interface is in the \plc{active} state then an OpenMP implementation
must maintain \plc{thread state} information for each thread. The thread 
state maintained is an approximation of the instantaneous state of a thread.

\format
\begin{ccppspecific}
A thread state must be one of the values of the enumeration type 
\code{ompt_state_t} or an implementation-defined state value of 512 or higher.

\begin{ompcEnum}
typedef enum ompt_state_t {
  ompt_state_work_serial                      = 0x000,
  ompt_state_work_parallel                    = 0x001,
  ompt_state_work_reduction                   = 0x002,

  ompt_state_wait_barrier                     = 0x010,
  ompt_state_wait_barrier_implicit_parallel   = 0x011,
  ompt_state_wait_barrier_implicit_workshare  = 0x012,
  ompt_state_wait_barrier_implicit            = 0x013,
  ompt_state_wait_barrier_explicit            = 0x014,

  ompt_state_wait_taskwait                    = 0x020,
  ompt_state_wait_taskgroup                   = 0x021,

  ompt_state_wait_mutex                       = 0x040,
  ompt_state_wait_lock                        = 0x041,
  ompt_state_wait_critical                    = 0x042,
  ompt_state_wait_atomic                      = 0x043,
  ompt_state_wait_ordered                     = 0x044,

  ompt_state_wait_target                      = 0x080,
  ompt_state_wait_target_map                  = 0x081,
  ompt_state_wait_target_update               = 0x082,

  ompt_state_idle                             = 0x100,
  ompt_state_overhead                         = 0x101,
  ompt_state_undefined                        = 0x102
} ompt_state_t;
\end{ompcEnum}
\end{ccppspecific}

\descr
A tool can query the OpenMP state of a thread at any time. If a 
tool queries the state of a thread that is not associated with OpenMP
then the implementation reports the state as \code{ompt_state_undefined}.

The value \code{ompt_state_work_serial} indicates that the thread 
is executing code outside all parallel regions.

The value \code{ompt_state_work_parallel} indicates that the thread 
is executing code within the scope of a parallel region construct.

The value \code{ompt_state_work_reduction} indicates that the thread 
is combining partial reduction results from threads in its team. An 
OpenMP implementation may never report a thread in this state; a 
thread that is combining partial reduction results may have its state 
reported as \code{ompt_state_work_parallel} or \code{ompt_state_overhead}.

The value \code{ompt_state_wait_barrier} indicates that the thread is 
waiting at either an implicit or explicit barrier. An implementation 
may never report a thread in this state; instead, a thread may have its 
state reported as \code{ompt_state_wait_barrier_implicit}  or 
\code{ompt_state_wait_barrier_explicit}, as appropriate.

The value \code{ompt_state_wait_barrier_implicit} indicates that the 
thread is waiting at an implicit barrier in a parallel region. An 
OpenMP implementation may report \code{ompt_state_wait_barrier} for 
implicit barriers.

The value \code{ompt_state_wait_barrier_implicit_parallel} indicates 
that the thread is waiting at an implicit barrier at the end of a parallel 
region. An OpenMP implementation may report \code{ompt_state_wait_barrier} 
or \code{ompt_state_wait_barrier_implicit} for these barriers.

The value \code{ompt_state_wait_barrier_implicit_workshare}  indicates 
that the thread is waiting at an implicit barrier at the end of a 
workshare-construct. An OpenMP implementation may report 
\code{ompt_state_wait_barrier} or \code{ompt_state_wait_barrier_implicit} 
for these barriers.

The value \code{ompt_state_wait_barrier_explicit} indicates that the 
thread is waiting in a \code{barrier} region. An OpenMP implementation
may report \code{ompt_state_wait_barrier} for these barriers.

The value \code{ompt_state_wait_taskwait} indicates that the thread is 
waiting at a taskwait construct. 

The value \code{ompt_state_wait_taskgroup} indicates that the thread is 
waiting at the end of a taskgroup construct. 

The value \code{ompt_state_wait_mutex} indicates that the thread is waiting 
for a mutex of an unspecified type. 

The value \code{ompt_state_wait_lock} indicates that the thread is waiting 
for a  lock or nest lock. 

The value \code{ompt_state_wait_critical} indicates that the thread is 
waiting to enter a critical region. 

The value \code{ompt_state_wait_atomic} indicates that the thread is 
waiting to enter an atomic region. 

The value \code{ompt_state_wait_ordered} indicates that the thread is 
waiting to enter an ordered region. 

The value \code{ompt_state_wait_target} indicates that the thread is 
waiting for a target region to complete.

The value \code{ompt_state_wait_target_map} indicates that the thread is 
waiting for a target data mapping operation to complete. An implementation may 
report \code{ompt_state_wait_target} for \code{target}~\code{data} constructs.

The value \code{ompt_state_wait_target_update} indicates that the thread is 
waiting for a \code{target}~\code{update} operation to complete. An implementation 
may report \code{ompt_state_wait_target} for \code{target}~\code{update} constructs.

The value \code{ompt_state_idle} indicates that the thread is idle, that  
is, it is not part of an OpenMP team.

The value \code{ompt_state_overhead} indicates that the thread is in the 
overhead state at any point while executing within the OpenMP runtime, 
except while waiting at a synchronization point.

The value \code{ompt_state_undefined} indicates that the native thread is 
not created by the OpenMP implementation.


\subsubsubsection{\hcode{ompt_frame_t}}
\index{frames}
\label{sec:ompt_frame_t}
\label{subsubsubsec:ompt_frame_t}

\summary
The \code{ompt_frame_t} type describes procedure frame information 
for an OpenMP task.

\syntax
\begin{ccppspecific}
\begin{ompSyntax}
typedef struct ompt_frame_t {
  ompt_data_t \plc{exit_frame};
  ompt_data_t \plc{enter_frame};
  int \plc{exit_frame_flags};
  int \plc{enter_frame_flags};
} ompt_frame_t;
\end{ompSyntax}
\end{ccppspecific}

\descr
Each \code{ompt_frame_t} object is associated with the task to which 
the procedure frames belong. Each non-merged initial, implicit, explicit, 
or target task with one or more frames on the stack of a native thread 
has an associated \code{ompt_frame_t} object.

The \plc{exit_frame} field of an \code{ompt_frame_t} object contains
information to identify the first procedure frame executing the task region.
The \plc{exit_frame} for the \code{ompt_frame_t} object associated with 
the \emph{initial task} that is not nested inside any OpenMP construct 
is \code{NULL}.

The \plc{enter_frame} field of an \code{ompt_frame_t} object contains
information to identify the latest still active procedure frame 
executing the task region before entering the OpenMP runtime 
implementation or before executing a different task. If a task with 
frames on the stack has not been suspended, the value of \plc{enter_frame} 
for the \code{ompt_frame_t} object associated with the task may 
contain \code{NULL}.

For \plc{exit_frame}, the \plc{exit_frame_flags} and, for \plc{enter_frame},
the \plc{enter_frame_flags} field indicates that the provided frame information 
points to a runtime or an application frame address. The same fields also 
specify the kind of information that is provided to identify the frame, These 
fields are a disjunction of values in the \code{ompt_frame_flag_t} enumeration type.

The lifetime of an \code{ompt_frame_t} object begins when a task is created
and ends when the task is destroyed. Tools should not assume that
a frame structure remains at a constant location in memory throughout the
lifetime of the task. A pointer to an \code{ompt_frame_t} object is passed 
to some callbacks; a pointer to the \code{ompt_frame_t} object of a task
can also be retrieved by a tool at any time, including in a signal
handler, by invoking the \code{ompt_get_task_info} runtime entry point 
(described in Section~\ref{sec:ompt_get_task_info}). A pointer to an 
\code{ompt_frame_t} object that a tool retrieved is valid as long as 
the tool does not pass back control to the OpenMP implementation.

\begin{note}
A monitoring tool that uses asynchronous sampling can observe values
of \plc{exit_frame} and \plc{enter_frame} at inconvenient times.
Tools must be prepared to handle \code{ompt_frame_t} objects observed 
just prior to when their field values will be set or cleared.
\end{note}



\subsubsubsection{\hcode{ompt_frame_flag_t}}
\label{subsubsec:ompt_frame_flag_t}

\summary
The \code{ompt_frame_flag_t} enumeration type defines valid frame 
information flags.

\syntax
\begin{ccppspecific}
\begin{ompSyntax}
typedef enum ompt_frame_flag_t {
  ompt_frame_runtime        = 0x00,
  ompt_frame_application    = 0x01,
  ompt_frame_cfa            = 0x10,
  ompt_frame_framepointer   = 0x20,
  ompt_frame_stackaddress   = 0x30
} ompt_frame_flag_t; 
\end{ompSyntax}
\end{ccppspecific}

\descr
The value \code{ompt_frame_runtime} of the \code{ompt_frame_flag_t} type
indicates that a frame address is a procedure frame in the OpenMP runtime 
implementation. The value \code{ompt_frame_application} of the 
\code{ompt_frame_flag_t} type indicates that an exit frame address is a 
procedure frame in the OpenMP application.

Higher number bits indicate the kind of provided information that is unique
for the particular frame pointer. The value \code{ompt_frame_cfa} indicates 
that a frame address specifies a \plc{canonical frame address}. The value 
\code{ompt_frame_framepointer} indicates that a frame address provides the 
value of the frame pointer register. The value \code{ompt_frame_stackaddress} 
indicates that a frame address specifies a pointer address that is
contained in the current stack frame.

\subsubsection{Wait Identifiers}
% omp_wait_id_t

% \subsubsubsection{\hcode{omp_wait_id_t}}
\label{sec:omp_wait_id_t}
\index{wait identifier}

Each thread instance maintains a \emph{wait identifier} of type \code{omp_wait_id_t}.
When a task executing on a thread is waiting for mutual exclusion, the thread's wait identifer indicates what the thread is awaiting.
A wait identifier may represent a critical section {\em name}, a lock, a program variable accessed in an atomic region, or a synchronization object internal to an OpenMP implementation.
% A thread's wait identifier can be retrieved on demand by invoking the \code{ompt_get_state} function (described in Section~\ref{sec:ompt_get_state}).


\begin{ccppspecific}
\begin{omptOther}
typedef uint64_t omp_wait_id_t;
\end{omptOther}
\end{ccppspecific}

\code{omp_wait_id_none} is defined as an instance of type \code{omp_wait_id_t} with the 
value 0.


When a thread is not in a wait state, the value of the thread's wait identifier is undefined.

% Tools should not assume that \code{omp_wait_id_t} values are small or densely allocated.

\subsection{Global Symbols}
%Many of the interfaces between tools and an OpenMP implementation are invisible to users. 
This section describes
%a few 
global symbols used by OMPT and OMPD tools to coordinate with an OpenMP implementation.
\subsubsection{\hcode{ompt_start_tool}}
\label{sec:ompt_start_tool}

\summary
If a tool wants to use the OMPT interface provided by an OpenMP implementation,
the tool must implement the function \code{ompt_start_tool} to announce its interest.

\format

\begin{cspecific}
\begin{omptOther}
ompt_start_tool_result_t *ompt_start_tool(
  unsigned int \plc{omp_version},
  const char *\plc{runtime_version}
);
\end{omptOther}
\end{cspecific}


\descr
For a tool to use the OMPT interface provided by an OpenMP implementation,
the tool must define a globally-visible implementation of the
function \code{ompt_start_tool}.

A tool may indicate its intent to use the OMPT interface provided
by an OpenMP implementation by having
\code{ompt_start_tool} return a non-null pointer to an
\code{ompt_start_tool_result_t} structure, which contains pointers to
tool initialization and finalization callbacks along with
a tool data word that an OpenMP implementation must pass by reference
to these callbacks.

A tool may use its argument \plc{omp_version} to determine
whether it is compatible with the OMPT interface provided by an OpenMP
implementation.

If a tool implements \code{ompt_start_tool} but has no interest in using
the OMPT interface in a particular execution,
\code{ompt_start_tool} should return \code{NULL}.

\argdesc

The argument \plc{omp_version}
is the value of the \code{_OPENMP} version macro
associated with the OpenMP API implementation. This value
identifies the OpenMP API version supported by an OpenMP implementation,
which specifies the version of the OMPT interface that it supports.

The argument \plc{runtime_version}
is a version string that unambiguously identifies the OpenMP implementation.

\constraints

The argument \plc{runtime_version} must be
an immutable string that is defined for the lifetime of a program
execution.

\effect
If a tool returns a non-null pointer to an
\code{ompt_start_tool_result_t} structure,
an OpenMP implementation will call the tool initializer specified by the
\plc{initialize} field in this structure before
beginning execution of any OpenMP construct
or completing execution of any environment routine invocation; the
OpenMP implementation will call the tool finalizer specified by the
\plc{finalize} field in this structure when the OpenMP
implementation shuts down.



\crossreferences
\begin{itemize}
    \item \code{ompt_start_tool_result_t}, see
     \specref{sec:ompt_start_tool_result_t}.
\end{itemize}


\subsubsection{\hcode{ompd_dll_locations}}
\label{subsubsec:ompd_dll_locations}
\index{ompd_dll_locations@{\code{ompd_dll_locations}}}

\summary
The global variable \code{ompd_dll_locations} indicates
where a tool should look for OMPD plugin(s) that are compatible
with the OpenMP implementation.

\begin{cspecific}
\begin{ompSyntax}
const char **ompd_dll_locations;
\end{ompSyntax}
\end{cspecific}


\descr
\code{ompd_dll_locations} is an \code{argv}-style vector of filename
strings that provide the names of any OMPD plugin implementations
that are compatible with the OpenMP runtime.
The vector is NULL-terminated.

The programming model or architecture of the third-party tool, and
hence that of the required OMPD plugin, might not match that of
the OpenMP program to be examined.
On platforms that support multiple programming models (\textit{e.g.},
32- v. 64-bit), or in heterogenous  environments where the architectures
of the OpenMP program and third-party tool may be be different,
OpenMP implementors are encourgaed to provide OMPD plugins for all models.
The vector, therefore, may name plugins that are not compatible
with the third-party tool.
This is legal, and it is up to the third-party tool to check that
a plugin is compatible.
(Typically, a tool might iterate over the vector until a compatible
plugin is found.)

\restrictions
\code{ompd_dll_locations} has external \code{C} linkage,
no demangling or other transformations are required by a third-party
tool before looking up its address in the OpenMP program.

The vector and its members must be fully initialized before
\code{ompd_dll_locations} is set to a non-NULL value.
That is, if \code{ompd_dll_locations} is not NULL, the vector
and its contents are valid.

\crossreferences
\begin{itemize}
\item
  \code{ompd_dll_locations_valid}, \specref{subsubsec:ompd_dll_locations_valid}
\item
  Finding the OMPD plugin, \specref{subsubsec:finding-the-ompd}
\end{itemize}

\subsubsection{\hcode{ompd_dll_locations_valid}}
\label{subsubsec:ompd_dll_locations_valid}
\index{ompd_dll_locations@{\code{ompd_dll_locations_valid}}}

\summary
The OpenMP runtime notifies third-party tools that \code{ompd_dll_locations}
is valid by allowing execution to pass through a location identified
by the symbol \code{ompd_dll_locations_valid}.


\begin{cspecific}
\begin{ompSyntax}
void ompd_dll_locations_valid(void);
\end{ompSyntax}
\end{cspecific}


\descr
Depending on how the OpenMP runtime is
implemented, \code{ompd_dll_locations} might not be a static
variable, and therefore needs to be initialized at runtime.  The
OpenMP runtime notifies third-party tools
that \code{ompd_dll_locations} is valid by having execution pass
through a location identified by the
symbol \code{ompd_dll_locations_valid}.
If \code{ompd_dll_locations} is NULL, a third-party tool, e.g., a
debugger can place a breakpoint at \code{ompd_dll_locations_valid}
to be notified when \code{ompd_dll_locations} has been initialized.
In practice, the symbol \code{ompd_dll_locations_valid} need not be
a function; instead, it may be a labeled machine instruction through
which execution passes once the vector is valid.



