\subsubsection{Frames}
\index{frames}
\subsubsubsection{\code{omp_frame_t}}
\label{sec:omp_frame_t}
\summary
The \code{omp_frame_t} type describes procedure frame information for an OpenMP task.

\syntax
\begin{ccppspecific}
\begin{ompSyntax}
typedef struct omp_frame_t {
  ompt_data_t \plc{exit_frame};
  ompt_data_t \plc{enter_frame};
  int \plc{frame_flags};
} omp_frame_t;
\end{ompSyntax}
\end{ccppspecific}

\descr

When executing an OpenMP program, at times, one or more procedure frames associated with
the OpenMP runtime may appear on a thread's stack between frames
associated with tasks. To help a tool determine whether a procedure
frame on the call stack belongs to a task or not,
for each task whose frames appear on the stack, the runtime
maintains an \code{omp_frame_t} object
that indicates a contiguous sequence of
procedure frames associated with the task.
Each \code{omp_frame_t} object is associated with the task to which the procedure frames belong.
Each non-merged initial, implicit, explicit, or target task with one or more frames on a thread's stack
will have an associated \code{omp_frame_t} object.



An \code{omp_frame_t} object associated with a task contains a pair
of pointers: \plc{exit_frame} and \plc{enter_frame}. The field names were
chosen, respectively, to reflect that they typically contain a pointer to a procedure frame on the stack when
\emph{exiting} the OpenMP runtime into code for a task or \emph{entering} the OpenMP runtime from a task.

The \plc{exit_frame} field of a task's \code{omp_frame_t} object
contains the canonical frame address for the procedure frame that
transfers control to the structured block for the task.
The value of \plc{exit_frame} is \code{NULL} until just prior to
beginning execution of the structured block for the task.
A task's \plc{exit_frame} may point to a procedure frame that belongs
to the OpenMP runtime or one that belongs to another task.
The \plc{exit_frame} for the \code{omp_frame_t} object associated
with an \emph{initial task} is \code{NULL}.

The \plc{enter_frame} field of a task's \code{omp_frame_t} object
contains the canonical frame address of a task procedure frame that invoked the
OpenMP runtime causing the current task to suspend and another task to
execute, or the first frame within the runtime as indicated by
\plc{frame_flags}.
If a task with frames on the stack has not suspended, the value of
\plc{enter_frame} for the \code{omp_frame_t} object
associated with the task may contain \code{NULL}.
The value of \plc{enter_frame} in a task's \code{omp_frame_t} is
reset to \code{NULL} just before a suspended task resumes execution.

An \code{omp_frame_t} objects's lifetime begins when a task is created
and ends when the task is destroyed. Tools should not assume that
a frame structure remains at a constant location in memory throughout
a task's lifetime. A pointer to a task's \code{omp_frame_t} object is passed to
some callbacks; a pointer to a task's \code{omp_frame_t} object
can also be retrieved by a tool at any time, including in a signal
handler, by invoking the
\code{ompt_get_task_info} runtime entry point (described in
Section~\ref{sec:ompt_get_task_info}).





\begin{table}
\begin{center}
\caption{States of an \code{omp_frame_t} Object\label{tab:frame}}
\begin{tabular}{|p{1in}||p{2in}|p{2in}|}
\hline
{\splc{exit_frame}} / {\splc{enter_frame}} 	state & {\splc{enter_frame}} is
{\scode{NULL}}
& {\splc{enter_frame}} is non-null \\
\hline
\hline
{\splc{exit_frame}} is {\scode{NULL}} &
case 1)  initial task during execution\newline
case 2) task that is created but not yet scheduled or already finished &
initial task suspended while another task executes
\\\hline
{\splc{exit_frame}} is non-null 	& non-initial task that has
been scheduled &
non-initial task
suspended while another task executes
\\\hline
\end{tabular}

\end{center}
\end{table}

Table~\ref{tab:frame} describes various states in which
an \code{omp_frame_t} object may be observed and their meaning.
In the presence of nested parallelism, a tool may
observe a sequence of \code{omp_frame_t} objects for a thread.
Appendix~\ref{chap:frames} illustrates
use of \code{omp_frame_t} objects with nested parallelism.

\begin{note}
A monitoring tool using asynchronous sampling can observe values
of \plc{exit_frame} and \plc{enter_frame} at inconvenient times.
Tools must be prepared to observe and handle \code{omp_frame_t}
objects observed just prior to when their field values will be set or
cleared.
\end{note}


\subsubsubsection{\code{omp_frame_t}}
\summary
The \code{omp_frame_flag_t} enumeration type defines valid frame information
flags.

\syntax
\begin{ccppspecific}
\begin{ompSyntax}
typedef enum omp_frame_flag_t {
  omp_frame_exit_runtime        = 0x00001,
  omp_frame_exit_application    = 0x00002,
  omp_frame_enter_runtime       = 0x10000,
  omp frame_enter_application   = 0x20000,
} omp_frame_flag_t; 
\end{ompSyntax}
\end{ccppspecific}

\descr
The value \code{omp_frame_exit_runtime} of the \code{omp_frame_flag_t} type
indicates that an exit frame address is a procedure frame in the OpenMP
runtime implementation.
The value \code{omp_frame_exit_application} of the \code{omp_frame_flag_t} type
indicates that an exit frame address is a procedure frame in the OpenMP
application.
The value \code{omp_frame_enter_runtime} of the \code{omp_frame_flag_t} type
indicates that an enter frame address is a procedure frame in the OpenMP
runtime implementation.
The value \code{omp_frame_enter_application} of the \code{omp_frame_flag_t} type
indicates that an enter frame address is a procedure frame in the OpenMP
application.
