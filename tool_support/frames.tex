\subsubsection{Frames}
\index{frames}
\subsubsubsection{\code{omp_frame_t}}
\label{sec:omp_frame_t}
\summary
The \code{omp_frame_t} type describes procedure frame information for an OpenMP task.

\syntax
\begin{ccppspecific}
\begin{ompSyntax}
typedef struct omp_frame_t {
  ompt_data_t \plc{exit_frame};
  ompt_data_t \plc{enter_frame};
  int \plc{exit_frame_flags};
  int \plc{enter_frame_flags};
} omp_frame_t;
\end{ompSyntax}
\end{ccppspecific}

\descr

Each \code{omp_frame_t} object is associated with the task to which the procedure frames belong.
Each non-merged initial, implicit, explicit, or target task with one or more frames on
the stack of a native thread has an associated \code{omp_frame_t} object.

The \plc{exit_frame} field of an \code{omp_frame_t} object contains
information to identify the first procedure frame executing the 
task region.
The \plc{exit_frame} for the \code{omp_frame_t} object associated with 
the \emph{initial task} that is not nested inside any OpenMP construct 
is \code{NULL}.

The \plc{enter_frame} field of an \code{omp_frame_t} object contains
information to identify the latest still active procedure frame 
executing the task region before entering the OpenMP runtime 
implementation or before executing a different task.
If a task with frames on the stack has not suspended, the value of
\plc{enter_frame} for the \code{omp_frame_t} object
associated with the task may contain \code{NULL}.

For \plc{exit_frame} the \plc{exit_frame_flags} and for \plc{enter_frame} 
the \plc{enter_frame_flags} field respectively indicates whether 
the provided frame information points to a runtime or an 
application frame address;
the same fields also specify the kind of information that is 
provided to identify the respective frame; 
these fields are a disjunction of values in the
\code{omp_frame_flag_t} enumeration type.

The lifetime of an \code{omp_frame_t} begins when a task is created
and ends when the task is destroyed. Tools should not assume that
a frame structure remains at a constant location in memory throughout
a task's lifetime. 
A pointer to an \code{omp_frame_t} object is passed to
some callbacks; a pointer to the \code{omp_frame_t} object of a task
can also be retrieved by a tool at any time, including in a signal
handler, by invoking the
\code{ompt_get_task_info} runtime entry point (described in
Section~\ref{sec:ompt_get_task_info}).
A pointer to an \code{omp_frame_t} object that a tool retrieved
is valid as long as the tool does not pass back control to the OpenMP
implementation.


%
% GOES TO EXAMPLES
%
%\begin{table}
%\begin{center}
%\caption{States of an \code{omp_frame_t} Object\label{tab:frame}}
%\begin{tabular}{|p{1in}||p{2in}|p{2in}|}
%\hline
%{\splc{exit_frame}} / {\splc{enter_frame}} 	state & {\splc{enter_frame}} is
%{\scode{NULL}}
%& {\splc{enter_frame}} is non-null \\
%\hline
%\hline
%{\splc{exit_frame}} is {\scode{NULL}} &
%case 1)  initial task during execution\newline
%case 2) task that is created but not yet scheduled or already finished &
%initial task suspended while another task executes
%\\\hline
%{\splc{exit_frame}} is non-null 	& non-initial task that has
%been scheduled &
%non-initial task
%suspended while another task executes
%\\\hline
%\end{tabular}
%
%\end{center}
%\end{table}
%
%Table~\ref{tab:frame} describes various states in which
%an \code{omp_frame_t} object may be observed and their meaning.
%In the presence of nested parallelism, a tool may
%observe a sequence of \code{omp_frame_t} objects for a thread.
%Appendix~\ref{chap:frames} illustrates
%use of \code{omp_frame_t} objects with nested parallelism.

\begin{note}
A monitoring tool using asynchronous sampling can observe values
of \plc{exit_frame} and \plc{enter_frame} at inconvenient times.
Tools must be prepared to observe and handle \code{omp_frame_t}
objects observed just prior to when their field values will be set or
cleared.
\end{note}


\subsubsubsection{\code{omp_frame_flag_t}}
\summary
The \code{omp_frame_flag_t} enumeration type defines valid frame information
flags.

\syntax
\begin{ccppspecific}
\begin{ompSyntax}
typedef enum omp_frame_flag_t {
  omp_frame_runtime        = 0x00,
  omp_frame_application    = 0x01,
  omp_frame_cfa            = 0x10,
  omp_frame_stackpointer   = 0x20
} omp_frame_flag_t; 
\end{ompSyntax}
\end{ccppspecific}

\descr
The value \code{omp_frame_runtime} of the \code{omp_frame_flag_t} type
indicates that a frame address is a procedure frame in the OpenMP
runtime implementation.
The value \code{omp_frame_application} of the \code{omp_frame_flag_t} type
indicates that an exit frame address is a procedure frame in the OpenMP
application.

Higher number bits indicate the kind of information provided which is unique
for the particular frame pointer.
The value \code{omp_frame_cfa} of the \code{omp_frame_flag_t} type
indicates that a frame address is specifying a canonical frame address.
The value \code{omp_frame_stackpointer} of the \code{omp_frame_flag_t} type
indicates that a frame address is specifying a stack pointer address.
