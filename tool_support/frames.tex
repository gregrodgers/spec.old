\subsubsection{\hcode{ompt_frame_t}}
\index{frames}
\label{sec:ompt_frame_t}
\label{subsubsubsec:ompt_frame_t}

\summary
The \code{ompt_frame_t} type describes procedure frame information 
for an OpenMP task.

\format
\begin{ccppspecific}
\begin{ompSyntax}
typedef struct ompt_frame_t {
  ompt_data_t \plc{exit_frame};
  ompt_data_t \plc{enter_frame};
  int \plc{exit_frame_flags};
  int \plc{enter_frame_flags};
} ompt_frame_t;
\end{ompSyntax}
\end{ccppspecific}

\descr
Each \code{ompt_frame_t} object is associated with the task to which 
the procedure frames belong. Each non-merged initial, implicit, explicit, 
or target task with one or more frames on the stack of a native thread 
has an associated \code{ompt_frame_t} object.

The \plc{exit_frame} field of an \code{ompt_frame_t} object contains
information to identify the first procedure frame executing the task region.
The \plc{exit_frame} for the \code{ompt_frame_t} object associated with 
the \emph{initial task} that is not nested inside any OpenMP construct 
is \code{NULL}.

The \plc{enter_frame} field of an \code{ompt_frame_t} object contains
information to identify the latest still active procedure frame 
executing the task region before entering the OpenMP runtime 
implementation or before executing a different task. If a task with 
frames on the stack has not been suspended, the value of \plc{enter_frame} 
for the \code{ompt_frame_t} object associated with the task may 
contain \code{NULL}.

For \plc{exit_frame}, the \plc{exit_frame_flags} and, for \plc{enter_frame},
the \plc{enter_frame_flags} field indicates that the provided frame information 
points to a runtime or an application frame address. The same fields also 
specify the kind of information that is provided to identify the frame, These 
fields are a disjunction of values in the \code{ompt_frame_flag_t} enumeration type.

The lifetime of an \code{ompt_frame_t} object begins when a task is created
and ends when the task is destroyed. Tools should not assume that
a frame structure remains at a constant location in memory throughout the
lifetime of the task. A pointer to an \code{ompt_frame_t} object is passed 
to some callbacks; a pointer to the \code{ompt_frame_t} object of a task
can also be retrieved by a tool at any time, including in a signal
handler, by invoking the \code{ompt_get_task_info} runtime entry point 
(described in Section~\ref{sec:ompt_get_task_info}). A pointer to an 
\code{ompt_frame_t} object that a tool retrieved is valid as long as 
the tool does not pass back control to the OpenMP implementation.

\begin{note}
A monitoring tool that uses asynchronous sampling can observe values
of \plc{exit_frame} and \plc{enter_frame} at inconvenient times.
Tools must be prepared to handle \code{ompt_frame_t} objects observed 
just prior to when their field values will be set or cleared.
\end{note}



\subsubsection{\hcode{ompt_frame_flag_t}}
\label{subsubsec:ompt_frame_flag_t}

\summary
The \code{ompt_frame_flag_t} enumeration type defines valid frame 
information flags.

\format
\begin{ccppspecific}
\begin{ompSyntax}
typedef enum ompt_frame_flag_t {
  ompt_frame_runtime        = 0x00,
  ompt_frame_application    = 0x01,
  ompt_frame_cfa            = 0x10,
  ompt_frame_framepointer   = 0x20,
  ompt_frame_stackaddress   = 0x30
} ompt_frame_flag_t; 
\end{ompSyntax}
\end{ccppspecific}

\descr
The value \code{ompt_frame_runtime} of the \code{ompt_frame_flag_t} type
indicates that a frame address is a procedure frame in the OpenMP runtime 
implementation. The value \code{ompt_frame_application} of the 
\code{ompt_frame_flag_t} type indicates that an exit frame address is a 
procedure frame in the OpenMP application.

Higher order bits indicate the kind of provided information that is unique
for the particular frame pointer. The value \code{ompt_frame_cfa} indicates 
that a frame address specifies a \plc{canonical frame address}. The value 
\code{ompt_frame_framepointer} indicates that a frame address provides the 
value of the frame pointer register. The value \code{ompt_frame_stackaddress} 
indicates that a frame address specifies a pointer address that is
contained in the current stack frame.
