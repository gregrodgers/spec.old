\subsection{Activating an OMPD Tool}
\label{subsec:activating}

The tool and the OpenMP program exist as separate processes. 
Thus, coordination is required between the OpenMP runtime
and the external tool for OMPD.

\subsubsection{Enabling the Runtime for OMPD}
\label{subsubsec:enabling-ompd}

In order to support third-party tools, the OpenMP runtime may need to collect
and to maintain information that it might not otherwise. The OpenMP runtime 
collects whatever information is necessary to support OMPD if the environment 
variable \code{OMP_DEBUG} is set to \plc{enabled}.

\crossreferences
\begin{itemize}
\item Activating an OMPT Tool, \specref{sec:ompt-initialization}

\item   \code{OMP_DEBUG}, \specref{sec:OMP_DEBUG}
\end{itemize}



\subsubsection{Finding the OMPD Library}
\label{subsubsec:finding-the-ompd}

An OpenMP runtime may have more than one OMPD libary. The tool must be able 
to locate the right library to use for the OpenMP program that it is examining.

As part of the OpenMP interface, OMPD requires the OpenMP runtime system to
provide a public variable \code{ompd_dll_locations}, which is an \code{argv}-style
vector of filename string pointers that provides the name(s) of any compatible 
OMPD library. This variable must have \code{C} linkage. The tool uses the name 
of the variable verbatim and, in particular, does not apply any name mangling 
before performing the look up.

\code{ompd_dll_locations} points to a NULL-terminated vector of zero or more 
NULL-terminated pathname strings that do not have any filename conventions. 
This vector must be fully initialized \emph{before} \code{ompd_dll_locations} 
is set to a non-null value, such that if a tool, such as a debugger, stops 
execution of the OpenMP program at any point at which \code{ompd_dll_locations} 
is non-null, then the vector of strings to which it points is valid and complete.

The programming model or architecture of the tool and, thus, that of OMPD does 
not have to match that of the OpenMP program that is being examined. The tool
must interpret the contents of \code{ompd_dll_locations} to find a suitable OMPD 
that matches its own architectural characteristics. On platforms that support 
different programming models (for example, 32-bit vs 64-bit), OpenMP 
implementations are encouraged to provide OMPD libraries for all models, and 
that can handle OpenMP programs of any model. Thus, for example, a 32-bit 
debugger that uses OMPD should be able to debug a 64-bit OpenMP program by 
loading a 32-bit OMPD implementation that can manage a 64-bit OpenMP runtime.

\crossreferences
\begin{itemize}
	\item Identifying the Matching OMPD, \specref{subsubsec:ompd_dll_locations}
\end{itemize}



\subsubsection{\hcode{ompd_dll_locations}}
\label{subsubsec:ompd_dll_locations}
\index{ompd_dll_locations@{\code{ompd_dll_locations}}}

\summary
The global variable \code{ompd_dll_locations} indicates
where a tool should look for OMPD plugin(s) that are compatible
with the OpenMP implementation.

\begin{cspecific}
\begin{ompSyntax}
const char **ompd_dll_locations;
\end{ompSyntax}
\end{cspecific}


\descr
\code{ompd_dll_locations} is an \code{argv}-style vector of filename
strings that provide the names of any OMPD plugin implementations
that are compatible with the OpenMP runtime.
The vector is NULL-terminated.

The programming model or architecture of the third-party tool, and
hence that of the required OMPD plugin, might not match that of
the OpenMP program to be examined.
On platforms that support multiple programming models (\textit{e.g.},
32- v. 64-bit), or in heterogenous  environments where the architectures
of the OpenMP program and third-party tool may be be different,
OpenMP implementors are encourgaed to provide OMPD plugins for all models.
The vector, therefore, may name plugins that are not compatible
with the third-party tool.
This is legal, and it is up to the third-party tool to check that
a plugin is compatible.
(Typically, a tool might iterate over the vector until a compatible
plugin is found.)

\restrictions
\code{ompd_dll_locations} has external \code{C} linkage,
no demangling or other transformations are required by a third-party
tool before looking up its address in the OpenMP program.

The vector and its members must be fully initialized before
\code{ompd_dll_locations} is set to a non-NULL value.
That is, if \code{ompd_dll_locations} is not NULL, the vector
and its contents are valid.

\crossreferences
\begin{itemize}
\item
  \code{ompd_dll_locations_valid}, \specref{subsubsec:ompd_dll_locations_valid}
\item
  Finding the OMPD plugin, \specref{subsubsec:finding-the-ompd}
\end{itemize}

\subsubsection{\hcode{ompd_dll_locations_valid}}
\label{subsubsec:ompd_dll_locations_valid}
\index{ompd_dll_locations@{\code{ompd_dll_locations_valid}}}

\summary
The OpenMP runtime notifies third-party tools that \code{ompd_dll_locations}
is valid by allowing execution to pass through a location identified
by the symbol \code{ompd_dll_locations_valid}.


\begin{cspecific}
\begin{ompSyntax}
void ompd_dll_locations_valid(void);
\end{ompSyntax}
\end{cspecific}


\descr
Depending on how the OpenMP runtime is
implemented, \code{ompd_dll_locations} might not be a static
variable, and therefore needs to be initialized at runtime.  The
OpenMP runtime notifies third-party tools
that \code{ompd_dll_locations} is valid by having execution pass
through a location identified by the
symbol \code{ompd_dll_locations_valid}.
If \code{ompd_dll_locations} is NULL, a third-party tool, e.g., a
debugger can place a breakpoint at \code{ompd_dll_locations_valid}
to be notified when \code{ompd_dll_locations} has been initialized.
In practice, the symbol \code{ompd_dll_locations_valid} need not be
a function; instead, it may be a labeled machine instruction through
which execution passes once the vector is valid.


