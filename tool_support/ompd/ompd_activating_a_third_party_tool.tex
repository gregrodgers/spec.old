\subsection{Activating an OMPD Tool}
\label{sec:ompd:activating}

The tool and the OpenMP program the tool controls
exist as separate processes.; thus coordination is required between the OpenMP runtime
%resident in the OpenMP program 
and the external tool for OMPD to be used successfully.

\subsubsection{Enabling the Runtime for OMPD}
\label{sec:ompd:enabling-ompd}

In order to support external tools, the OpenMP runtime may need to collect
and maintain information that it might otherwise not do, perhaps
for performance reasons, or because it is not otherwise needed.
The OpenMP runtime collects whatever information is necessary
to support OMPD if at least one of the following conditions is true:
\begin{enumerate}
\item
  the environment variable \code{OMP\_DEBUG} is set to \code{enabled}
\item
  the OpenMP program defines the function
  \code{int omp\_debug\_enable~(~void~)} and it returns logical true (i.e., non-zero) when
  the OpenMP runtime calls this function.
  Immediately before the OpenMP implementation initializes itself
  it will check to see whether the \code{ompd\_debug\_enable} function
  is defined in the OpenMP program or one of its dynamically-linked libraries.
  If a definition is found, the OpenMP will call the function it finds to determine what it returns.
%  The OpenMP runtime will enable data collection for the external tool
%  if \code{omp\_debug\_enable} returns logical true (i.e., non-zero).
\end{enumerate}
%These methods are equivalent, and either can be used.
In some cases it may not be possible to control an OpenMP program's
environment.
\code{omp\_debug\_enable} allows an OpenMP program itself to turn on
data collection for OMPD.
The function can be positioned in an otherwise empty DLL the
programmer can link with the OpenMP program.
This leaves the OpenMP program code unmodified.
A tool implementor may choose to distribute a DLL defining
\code{omp\_debug\_enable} as a convenience to programmers.

\crossreferences
\begin{itemize}
\item
  \code{OMP\_DEBUG}, \specref{sec:OMP_DEBUG}
\item
  \code{omp\_debug\_enable}, \specref{sec:ompd:omp_debug_enable}
\item
  Activating an OMPT Tool, \specref{sec:ompt-initialization}
\end{itemize}

\subsubsection{Finding the OMPD plugin}
\label{sec:ompd:finding-the-ompd}

An OpenMP runtime may have more than one matching OMPD plugins for 
tools to use.
The tool must be able to locate the right plugin to use
for the OpenMP program it is examining.

As part of the OpenMP interface, OMPD requires that the OpenMP
runtime system provides a public variable \code{ompd\_dll\_locations},
which is an \code{argv}-style vector of filename string pointers that
provides the pathnames(s) of any compatible OMPD plugin implementations.
\code{ompd\_dll\_locations} must have \code{C} linkage.
The tool uses the name of the variable verbatim,
and in particular, will not apply any name mangling before
performing the look up.
The pathnames may be relative or absolute.

\code{ompd\_dll\_locations} points to a NULL-terminated
vector of zero or more NULL-terminated pathname strings.
There are no filename conventions for pathname strings.
The last entry in the vector is NULL.
The vector of string pointers must be fully initialized \emph{before}
\code{ompd\_dll\_locations} is set to a non-NULL value,
such that if the tool, such as a debugger,
stops execution of the OpenMP program at any point where
\code{ompd\_dll\_locations} is non-NULL,
then the vector of strings it points to is valid and complete.

The programming model or architecture of the tool (and hence
that of the required OMPD) does not have to match that of the OpenMP program
being examined.
It is the responsibility of the tool to interpret the contents
of \code{ompd\_dll\_locations} to find a suitable OMPD that matches
its own architectural characteristics.
On platforms that support different programming models
(\textit{e.g.}, 32-bit vs 64-bit), OpenMP implementers are encouraged
to provide OMPD library for all models, and which can handle
OpenMP programs of any model.
Thus, for example, a 32-bit debugger using OMPD should be able
to debug a 64-bit OpenMP program
by loading a 32-bit OMPD that can manage a 64-bit OpenMP runtime.

\crossreferences
\begin{itemize}
	\item Identifying the Matching OMPD, \specref{sec:ompd:ompd_dll_locations}
	%\item \code{omp\_debug\_enable}, \specref{sec:ompd:omp_debug_enable}
	%\item Activating an OMPT Tool, \specref{sec:ompt-initialization}
\end{itemize}

\begin{comment}
Depending on how the OpenMP runtime is implemented, \code{ompd\_dll\_locations}
might not be a static variable, and therefore needs to be initialized
at runtime.
The OpenMP runtime notifies tools that \code{ompd\_dll\_locations}
is valid by allowing execution to pass through a location identified
by the symbol \code{ompd\_dll\_locations\_valid}.
This symbol has external, \code{C}, linkage.
Conceptually, \code{ompd\_dll\_locations\_valid} has the signature
\code{void ompd\_dll\_locations\_valid~(~void~);}.
If \code{ompd\_dll\_locations} is NULL, a tool like a debugger
can place a breakpoint at \code{ompd\_dll\_locations\_valid} to be notified
when the vector of name strings has been setup, and is valid.

\code{ompd\_dll\_locations\_valid} does not need not be a function,
and can instead be a labeled location in the code through which
execution passes once the vector is valid.

\crossreferences
\begin{itemize}
\item
  \code{ompd\_dll\_locations}, \specref{sec:ompd:ompd_dll_locations}
\item
  \code{ompd\_dll\_locations\_valid}, \specref{sec:ompd:ompd_dll_locations_valid}
\end{itemize}
\end{comment}


