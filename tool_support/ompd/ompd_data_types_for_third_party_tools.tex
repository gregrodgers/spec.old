
In this section, we define the types, structures, and functions for the OMPD API.

\ompdsubsection{Basic Types}
The following describes the basic OMPD API types.

\format
\vbox{
\ccppspecificstart
\begin{boxedcode}
/* unsigned integer large enough to hold a target address or a */
/* target segment value*/
typedef uint64_t ompd_taddr_t;
/* signed version of ompd_addr_t */
typedef int64_t  ompd_tword_t;   
/* identifies what a thread is  waiting for*/       
typedef uint64_t ompd_wait_id_t;
typedef struct \{
  /* target architecture specific segment value */
  ompd_taddr_t segment;
  /* target address in the segment */ 
  ompd_taddr_t address;
\} ompd_address_t;

#define OMPD_SEGMENT_UNSPECIFIED  ((ompd_taddr_t) 0)
#define OMPD_SEGMENT_TEXT         ((ompd_taddr_t) 1)
#define OMPD_SEGMENT_DATA         ((ompd_taddr_t) 2)
\end{boxedcode}
\ccppspecificend
}

The \code{ompd\_address\_t} is a structure that OMPD uses to specify target addresses, which 
may or may not be segmented.  The following rules apply for this structure:

\begin{itemize}
	%
	\item
	If the address space of the  OpenMP architecture is not segmented, the OMPD implementation 
	should use \code{OMPD\_SEGMENT\_UNSPECIFIED} for the segment value.
	%
	\item
	If the target architecture uses simple ``text'' and ``data'' segments,
	which is common on some systems, the OMPD implementation should use
	\code{OMPD\_SEGMENT\_TEXT} for the text segment value, and
	\code{OMPD\_SEGMENT\_DATA} for the data segment value.
	%
	%\item
	%The segment value for the NVIDIA\textsuperscript{\textregistered} GPU
	%target architecture should use a \code{ptxStorageKind} enumeration value as defined by the 
	%CUDA Debugger API.
	%This enumeration is defined by the \code{cudadebugger.h} header file contained within a 
	%CUDA SDK package.
	\item
	Otherwise, the segment value is target architecture specific.
\end{itemize}

\ompdsubsection{Operating System Thread Types}

An OpenMP runtime may use different thread implementations. OMPD uses the 
\code{ompd\_osthread\_kind\_t} type to describe the thread implementation upon which an 
OpenMP thread is overlaid.

\format
\vbox{
\ccppspecificstart
\begin{boxedcode}
typedef enum \{
  ompd_osthread_pthread,
  ompd_osthread_lwp,
  ompd_osthread_winthread
\} ompd_osthread_kind_t;
		
\end{boxedcode}
\ccppspecificend
}

The operating system-specific information can vary in size and format, and therefore is not 
explicitly represented in this API. Operating system-specific thread identifiers are passed
across the interface by reference, that is, by a pointer to where the information can be found.
In addition, the `kind' and size of the information are also passed.

When operating system-specific thread identifiers are passed as either `in' or `out' parameters, 
they are allocated and owned by the caller, which is responsible for their eventual disposal.

\ompdsubsection{OMPD Handle Types}

Each OMPD interface operation that applies to a particular address space, thread, parallel region, 
or task must  explicitly specify the target entity for the operation using a \emph{handle}.
OMPD employs handles for address spaces (for a host or target device), threads, parallel regions, 
and tasks. A handle for an entity is constant while the entity itself is alive. Handles are defined by 
an OMPD implementation, and are opaque to a third-party tool. The following define the OMPD 
handle types:

\format
\vbox{
\ccppspecificstart
\begin{boxedcode}
typedef struct _ompd_address_space_handle_s ompd_address_space_handle_t;
typedef struct _ompd_thread_handle_s ompd_thread_handle_t;
typedef struct _ompd_parallel_handle_s ompd_parallel_handle_t;
typedef struct _ompd_task_handle_s ompd_task_handle_t;
\end{boxedcode}
\ccppspecificend
}

Defining externally visible type names in this way introduces type safety to the interface, and helps 
to catch instances where incorrect handles are passed by the debugger to the OMPD 
implementation. The \code{struct}s do not need to be defined at all. The OMPD implementation 
would need to cast incoming (pointers to) handles to the appropriate internal, private types.

\ompdsubsection{Debugger Context Types}

The debugger contexts are opaque to the OMPD and are defined as follows:

\format
\vbox{
\ccppspecificstart
\begin{boxedcode}
typedef struct _ompd_address_space_context_s ompd_address_space_context_t;
typedef struct _ompd_thread_context_s ompd_thread_context_t;
\end{boxedcode}
\ccppspecificend
}

\ompdsubsection{Return Code Types}
Each OMPD operation has a return code. The return code types and their semantics are defined as 
follows:

\format
\vbox{
\ccppspecificstart
\begin{boxedcode}
typedef enum \{
  ompd_rc_ok = 0,  /* operation was successful */
  ompd_rc_unavailable = 1, /* info is not available (in this context) */
  ompd_rc_stale_handle = 2, /* handle is no longer valid */
  ompd_rc_bad_input = 3, /* bad input parameters (other than handle) */
  ompd_rc_error = 4, /* error */
  ompd_rc_unsupported = 5, /* operation is not supported */
  ompd_rc_needs_state_tracking = 6, /* needs runtime state tracking enabled */
  ompd_rc_incompatible = 7, /* target is not compatible with this OMPD */
  ompd_rc_target_read_error = 8, /* error reading from the target */
  ompd_rc_target_write_error = 9, /* error writing from the target */
  ompd_rc_nomem = 10, /* unable to allocate memory */
\} ompd_rc_t;	
\end{boxedcode}
\ccppspecificend
}

\ompdsubsection{OpenMP Scheduling Types}

The following enumeration defines \code{ompd\_sched\_t}, which is the OMPD API definition 
corresponding to the OpenMP enumeration type \code|omp\_sched\_t| (XXXX).
\code{ompd\_sched\_t} also defines \code{ompd\_sched\_vendor\_lo} and
\code{ompd\_sched\_vendor\_hi} to define the range of implementation-specific 
\code{omp\_sched\_t} values than can be handle by the OMPD API.

\begin{quote}
	\begin{lstlisting}

	\end{lstlisting}
\end{quote}

\format
\vbox{
\ccppspecificstart
\begin{boxedcode}
typedef enum \{
  ompd_sched_static = 1,
  ompd_sched_dynamic = 2,
  ompd_sched_guided = 3,
  ompd_sched_auto = 4,
  ompd_sched_vendor_lo = 5,
  ompd_sched_vendor_hi = 0x7fffffff
\} ompd_sched_t;
\end{boxedcode}
\ccppspecificend
}

\ompdsubsection{OpenMP Proc Binding Types}

The following enumeration defines \code{ompd\_proc\_bind\_t}, which is the OMPD
API definition corresponding to the OpenMP enumeration type
\code{omp\_proc\_bind\_t} (XXX).

\format
\vbox{
\ccppspecificstart
\begin{boxedcode}
typedef enum \{
  ompd_proc_bind_false = 0,
  ompd_proc_bind_true = 1,
  ompd_proc_bind_master = 2,
  ompd_proc_bind_close = 3,
  ompd_proc_bind_spread = 4
\} ompd_proc_bind_t;
\end{boxedcode}
\ccppspecificend
}

\ompdsubsection{Primitive Types}

The following structure contains members that the OMPD implementation can use
to interrogate the debugger about the ``sizeof'' of primitive types in the OpenMP architecture 
address space.

\format
\vbox{
\ccppspecificstart
\begin{boxedcode}
typedef struct \{
  int sizeof_char;
  int sizeof_short;
  int sizeof_int;
  int sizeof_long;
  int sizeof_long_long;
  int sizeof_pointer;
\} ompd_target_type_sizes_t;
\end{boxedcode}
\ccppspecificend
}

The following enumeration of primitive types are used by OMPD to express the primitive
type of data for target to host conversion.

\format
\vbox{
\ccppspecificstart
\begin{boxedcode}
typedef enum \{
  ompd_type_char = 0,
  ompd_type_short = 1,
  ompd_type_int = 2,
  ompd_type_long = 3,
  ompd_type_long_long = 4,
  ompd_type_pointer = 5
\} ompd_target_prim_types_t;
\end{boxedcode}
\ccppspecificend
}

\ompdsubsection{Runtime State Types}

The OMPD runtime states mirror those in OMPT (see XXX}). Note that there is no guarantee that 
the numeric values of the corresponding members of the enumerations are identical.

\format
\vbox{
\ccppspecificstart
\begin{boxedcode}
typedef enum \{
  /* work states (0..15) */
  ompd_state_work_serial = 0x00, /* working outside parallel */
  ompd_state_work_parallel = 0x01, /* working within parallel  */
  ompd_state_work_reduction = 0x02, /* performing a reduction */

  /* idle (16..31) */
  ompd_state_idle = 0x10, /* waiting for work */

  /* overhead states (32..63) */
  ompd_state_overhead = 0x20, /* non-wait overhead */

  /* barrier wait states (64..79) */
  ompd_state_wait_barrier = 0x40, /* generic barrier */
  ompd_state_wait_barrier_implicit = 0x41, /* implicit barrier */
  ompd_state_wait_barrier_explicit = 0x42, /* explicit barrier */

  /* task wait states (80..95) */
  ompd_state_wait_taskwait = 0x50, /* waiting at a taskwait */
  ompd_state_wait_taskgroup = 0x51, /* waiting at a taskgroup */

  /* mutex wait states (96..111) */
  ompd_state_wait_mutex = 0x60, /* waiting for any mutex kind */
  ompd_state_wait_lock = 0x61, /* waiting for lock */  
  ompd_state_wait_critical = 0x62, /* waiting for critical */
  ompd_state_wait_atomic = 0x63, /* waiting for atomic */
  ompd_state_wait_ordered = 0x64, /* waiting for ordered */

  /* misc (112..127) */
  ompd_state_undefined = 0x70, /* undefined thread state */
  ompd_state_first = 0x71, /* initial enumeration state */
\} ompd_state_t;
\end{boxedcode}
\ccppspecificend
}