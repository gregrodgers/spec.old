
In this section, we define the types, structures, and functions for the OMPD API.

\ompdsubsection{Basic Types}
The following describes the basic OMPD API types.

\format
\vbox{
\ccppspecificstart
\begin{boxedcode}
/* unsigned integer large enough to hold a target device */
/* address or a target device segment value */
typedef uint64_t ompd_addr_t;
/* signed version of ompd_addr_t */
typedef int64_t  ompd_word_t;   
/* identifies what a thread is waiting for */
typedef uint64_t ompd_wait_id_t;
typedef struct \{
  /* target device specific segment value */
  ompd_addr_t segment;
  /* target device address in the segment */ 
  ompd_addr_t address;
\} ompd_address_t;

#define OMPD_SEGMENT_UNSPECIFIED  ((ompd_addr_t) 0)
#define OMPD_SEGMENT_TEXT         ((ompd_addr_t) 1)
#define OMPD_SEGMENT_DATA         ((ompd_addr_t) 2)
\end{boxedcode}
\ccppspecificend
}

The \code{ompd\_address\_t} is a structure that OMPD uses to specify target device addresses, which 
may or may not be segmented.  The following rules apply for this structure:

\begin{itemize}
	%
	\item
	If the address space of the  OpenMP architecture is not segmented, the OMPD library 
	should use \code{OMPD\_SEGMENT\_UNSPECIFIED} for the segment value.
	%
	\item
	If the architecture of the target device uses simple ``text'' and ``data'' segments,
	which is common on some systems, the OMPD library should use
	\code{OMPD\_SEGMENT\_TEXT} for the text segment value, and
	\code{OMPD\_SEGMENT\_DATA} for the data segment value.
	%
	%\item
	%The segment value for the NVIDIA\textsuperscript{\textregistered} GPU
	%target architecture should use a \code{ptxStorageKind} enumeration value as defined by the 
	%CUDA Debugger API.
	%This enumeration is defined by the \code{cudadebugger.h} header file contained within a 
	%CUDA SDK package.
	\item
	Otherwise, the segment value is specific to the target device.
To avoid clashes, vendors are encouraged to register values they choose,
and what they represent,
in the technical report, "\emph{OMPD System-Specific Information}".
Where possible, OpenMP vendors should use existing values and
definitions to facilitate interoperability with existing tools.
\end{itemize}

\ompdsubsection{Operating System Thread Identifiers}

Different OpenMP runtimes may use different underlying (operating system)
thread implementations.
%% OMPD uses the 
%% \code{ompd\_osthread\_kind\_t} type to describe the thread implementation upon which an 
%% OpenMP thread is overlaid.
The type used to hold an operating system thread identifier can vary in size and format, and therefore is not 
explicitly represented in this API. Operating system thread identifiers are passed
across the interface using a thread-identifier `kind', a pointer to where
the thread identifer is stored, and the size of the thread identifier
in bytes.

The OMPD library and tool using it must agree on the format
of what is being passed.
%% Thus, operating system-specific thread identifiers are always accompanied
%% by a `kind' tag to indicate what the information represents.
This is determined by the \plc{kind} argument.
Each different kind of thread identifier uses a unique
unsigned 64-bit integer value.
The values less than \code{OMPD\_OSTHREAD\_USER\_LO} are reserved for
well-known thread identifier kinds.
The values \code{OMPD\_OSTHREAD\_USER\_LO} to \code{OMPD\_OSTHREAD\_USER\_HI}
inclusive are available for identifying implementation-specific thread kinds.
To avoid clashes, vendors are encouraged to register values they choose,
and what they represent (e.g., the layout and semantics of the information),
in the technical report, "\emph{OMPD System-Specific Information}".
Where possible, OpenMP vendors should use existing values and
definitions to facilitate interoperability with existing tools.

\format
\vbox{
\ccppspecificstart
\begin{boxedcode}
typedef uint64_t ompd_osthread_kind_t;

#define OMPD_OSTHREAD_PTHREAD 0
#define OMPD_OSTHREAD_USER_L0 16384
#define OMPD_OSTHREAD_USER_HI (~(ompd_osthread_kind_t) 0)
\end{boxedcode}
\ccppspecificend
}

OpenMP implementations commonly use Posix threads (aka \code{pthreads}) as the
underlying thread implementation.
The \code{OMPD\_OSTHREAD\_PTHREAD} kind indicates that the thread identifier
data is a \code{pthread\_t} in the format of the process being monitored
by the third-party tool, which may differ from the format used by the
OMPD library or the third-party tool.

When thread identifiers are passed as either `in' or `out' parameters, 
they are allocated and owned by the caller, which is responsible for their deallocation.

\ompdsubsection{OMPD Handle Types}

Each OMPD interface operation that applies to a particular address space, thread, parallel region, 
%or task must explicitly specify the target entity for the operation using a \emph{handle}.
or task must explicitly specify the device for the operation using a \emph{handle}.
OMPD employs handles for address spaces (for a host or target device), threads, parallel regions, 
and tasks. A handle for an entity is constant while the entity itself is alive. Handles are defined by 
the OMPD library, and are opaque to the using tool. The following define the OMPD 
handle types:

\format
\vbox{
\ccppspecificstart
\begin{boxedcode}
typedef struct _ompd_address_space_handle_s ompd_address_space_handle_t;
typedef struct _ompd_thread_handle_s ompd_thread_handle_t;
typedef struct _ompd_parallel_handle_s ompd_parallel_handle_t;
typedef struct _ompd_task_handle_s ompd_task_handle_t;
\end{boxedcode}
\ccppspecificend
}

Defining externally visible type names in this way introduces type safety to the interface, and helps 
to catch instances where incorrect handles are passed by the debugger to the OMPD 
implementation. The \code{struct}s do not need to be defined at all. The OMPD library 
would need to cast incoming (pointers to) handles to the appropriate internal, private types.

\ompdsubsection{Debugger Context Types}

The debugger contexts are opaque to the OMPD and are defined as follows:

\format
\vbox{
\ccppspecificstart
\begin{boxedcode}
typedef struct _ompd_address_space_context_s ompd_address_space_context_t;
typedef struct _ompd_thread_context_s ompd_thread_context_t;
\end{boxedcode}
\ccppspecificend
}

\ompdsubsection{Return Code Types}
Each OMPD operation has a return code. The return code types and their semantics are defined as 
follows:

\format
\vbox{
\ccppspecificstart
\begin{boxedcode}
typedef enum \{
  ompd_rc_ok = 0,  /* operation was successful */
  ompd_rc_unavailable = 1, /* info is not available (in this context) */
  ompd_rc_stale_handle = 2, /* handle is no longer valid */
  ompd_rc_bad_input = 3, /* bad input parameters (other than handle) */
  ompd_rc_error = 4, /* error */
  ompd_rc_unsupported = 5, /* operation is not supported */
  ompd_rc_needs_state_tracking = 6, /* needs runtime state tracking enabled */
  ompd_rc_incompatible = 7, /* device is not compatible with this OMPD */
  ompd_rc_device_read_error = 8, /* error reading from the device */
  ompd_rc_device_write_error = 9, /* error writing to the device */
  ompd_rc_nomem = 10, /* unable to allocate memory */
\} ompd_rc_t;	
\end{boxedcode}
\ccppspecificend
}

\ompdsubsection{OpenMP Scheduling Types}

The following enumeration defines \code{ompd\_sched\_t}, which is the OMPD API definition 
corresponding to the OpenMP enumeration type \code|omp\_sched\_t| (XXXX).
\code{ompd\_sched\_t} also defines \code{ompd\_sched\_vendor\_lo} and
\code{ompd\_sched\_vendor\_hi} to define the range of implementation-specific 
\code{omp\_sched\_t} values than can be handle by the OMPD API.

\begin{quote}
	\begin{lstlisting}

	\end{lstlisting}
\end{quote}

\format
\vbox{
\ccppspecificstart
\begin{boxedcode}
typedef enum \{
  ompd_sched_static = 1,
  ompd_sched_dynamic = 2,
  ompd_sched_guided = 3,
  ompd_sched_auto = 4,
  ompd_sched_vendor_lo = 5,
  ompd_sched_vendor_hi = 0x7fffffff
\} ompd_sched_t;
\end{boxedcode}
\ccppspecificend
}

\ompdsubsection{OpenMP Proc Binding Types}

The following enumeration defines \code{ompd\_proc\_bind\_t}, which is the OMPD
API definition corresponding to the OpenMP enumeration type
\code{omp\_proc\_bind\_t} (XXX).

\format
\vbox{
\ccppspecificstart
\begin{boxedcode}
typedef enum \{
  ompd_proc_bind_false = 0,
  ompd_proc_bind_true = 1,
  ompd_proc_bind_master = 2,
  ompd_proc_bind_close = 3,
  ompd_proc_bind_spread = 4
\} ompd_proc_bind_t;
\end{boxedcode}
\ccppspecificend
}

\ompdsubsection{Primitive Types}

The following structure contains members that the OMPD library can use
to interrogate the debugger about the ``sizeof'' of primitive types in the OpenMP architecture 
address space.

\format
\vbox{
\ccppspecificstart
\begin{boxedcode}
typedef struct \{
  int sizeof_char;
  int sizeof_short;
  int sizeof_int;
  int sizeof_long;
  int sizeof_long_long;
  int sizeof_pointer;
\} ompd_device_type_sizes_t;
\end{boxedcode}
\ccppspecificend
}

The following enumeration of primitive types are used by OMPD to express the primitive
%type of data for target to host conversion.
type of data for OpenMP device to OMPD host conversion.

\format
\vbox{
\ccppspecificstart
\begin{boxedcode}
typedef enum \{
  ompd_type_char = 0,
  ompd_type_short = 1,
  ompd_type_int = 2,
  ompd_type_long = 3,
  ompd_type_long_long = 4,
  ompd_type_pointer = 5
\} ompd_device_prim_types_t;
\end{boxedcode}
\ccppspecificend
}

\ompdsubsection{Runtime State Types}

The OMPD runtime states mirror those in OMPT (see XXX). Note that there is no guarantee that 
the numeric values of the corresponding members of the enumerations are identical.

\format
\vbox{
\ccppspecificstart
\begin{boxedcode}
typedef enum \{
  /* work states (0..15) */
  ompd_state_work_serial = 0x00, /* working outside parallel */
  ompd_state_work_parallel = 0x01, /* working within parallel  */
  ompd_state_work_reduction = 0x02, /* performing a reduction */

  /* idle (16..31) */
  ompd_state_idle = 0x10, /* waiting for work */

  /* overhead states (32..63) */
  ompd_state_overhead = 0x20, /* non-wait overhead */

  /* barrier wait states (64..79) */
  ompd_state_wait_barrier = 0x40, /* generic barrier */
  ompd_state_wait_barrier_implicit = 0x41, /* implicit barrier */
  ompd_state_wait_barrier_explicit = 0x42, /* explicit barrier */

  /* task wait states (80..95) */
  ompd_state_wait_taskwait = 0x50, /* waiting at a taskwait */
  ompd_state_wait_taskgroup = 0x51, /* waiting at a taskgroup */

  /* mutex wait states (96..111) */
  ompd_state_wait_mutex = 0x60, /* waiting for any mutex kind */
  ompd_state_wait_lock = 0x61, /* waiting for lock */  
  ompd_state_wait_critical = 0x62, /* waiting for critical */
  ompd_state_wait_atomic = 0x63, /* waiting for atomic */
  ompd_state_wait_ordered = 0x64, /* waiting for ordered */

  /* misc (112..127) */
  ompd_state_undefined = 0x70, /* undefined thread state */
  ompd_state_first = 0x71, /* initial enumeration state */
\} ompd_state_t;
\end{boxedcode}
\ccppspecificend
}
