\subsection{Runtime Entry Points for OMPD}
\label{subsec:runtime-entry-points-for-ompd}

The OpenMP implementation must define a number of entry point symbols
through which execution must pass when particular events occur
\emph{and} data collection for OMPD is enabled.
A tool can gain notification of the event by setting 
a breakpoint at the address of the entry point symbol.

\restrictions

Runtime entry points for OMPD have the following restrictions:
\begin{itemize}
\item Entry point symbols have external \code{C} linkage and no
demangling or other transformations are required by a tool
to look up its name to obtain the address in the OpenMP program.

\item Conceptually, an entry point symbol has a function type signature.
However, it does not need to be a function, but rather can be a labeled location
in the implementation code.
\end{itemize}



\subsubsection{Beginning Parallel Regions}
\label{subsubsec:ompd_bp_parallel_begin}
\index{ompd_bp_parallel_begin@{\code{ompd_bp_parallel_begin}}}

\summary

Before starting the execution of an OpenMP parallel region, the implementation executes
\code{ompd_bp_parallel_begin}.

\format
\begin{cspecific}
\begin{ompSyntax}
void ompd_bp_parallel_begin ( void );
\end{ompSyntax}
\end{cspecific}


\descr

The OpenMP implementation must execute 
\code{ompd_bp_parallel_begin} at every \plc{parallel-begin} event.

At the point the implementation reaches \code{ompd_bp_parallel_begin}
the binding for \code{ompd_get_curr_parallel_handle} is the parallel 
region that is beginning and 
the binding for \code{ompd_get_curr_task_handle}
is the task that encountered the parallel construct.


\crossreferences

\begin{itemize}
\item
  \code{parallel} construct, \specref{sec:parallel Construct}
\item
  \code{ompd_get_curr_parallel_handle},
\specref{subsubsubsec:ompd_get_curr_parallel_handle}
\item
  \code{ompd_get_curr_task_handle}, \specref{subsubsubsec:ompd_get_curr_task_handle}
\end{itemize}





\subsubsection{Ending Parallel Regions}
\label{subsubsec:ompd_bp_parallel_end}
\index{ompd_bp_parallel_end@{\code{ompd_bp_parallel_end}}}

\summary
After finishing the execution of an OpenMP parallel region, the implementation executes
\code{ompd_bp_parallel_end}.


\format
\begin{cspecific}
\begin{ompSyntax}
void ompd_bp_parallel_end ( void );
\end{ompSyntax}
\end{cspecific}


\descr

The OpenMP implementation must execute 
\code{ompd_bp_parallel_end} at every \plc{parallel-end} event.

At the point the implementation reaches \code{ompd_bp_parallel_end}
the binding for \code{ompd_get_curr_parallel_handle} is the parallel 
region that is ending and 
the binding for \code{ompd_get_curr_task_handle}
is the task that encountered the parallel construct.

After 
executing \code{ompd_bp_parallel_end}, any \plc{parallel_handle} acquired for this
parallel region is invalid and should be released.


\crossreferences
\begin{itemize}
\item
  \code{parallel} construct, \specref{sec:parallel Construct}
\item
  \code{ompd_get_curr_parallel_handle}, \specref{subsubsubsec:ompd_get_curr_parallel_handle}
\item
  \code{ompd_rel_parallel_handle}, \specref{subsubsubsec:ompd_rel_parallel_handle}
\item
  \code{ompd_get_curr_task_handle}, \specref{subsubsubsec:ompd_get_curr_task_handle}
\end{itemize}






\subsubsection{Beginning Task Regions}
\label{subsubsec:ompd_bp_task_begin}
\index{ompd_bp_task_begin@{\code{ompd_bp_task_begin}}}

\summary

Before starting the execution of an OpenMP task region, 
the implementation executes \code{ompd_bp_task_begin}.


\format
\begin{cspecific}
\begin{ompSyntax}
void ompd_bp_task_begin ( void );
\end{ompSyntax}
\end{cspecific}


\descr

The OpenMP implementation must execute \code{ompd_bp_task_begin} immediately
before starting execution of a \plc{structured-block} associated 
with a non-merged task. %(It might do so for merged)?

At the point the implementation reaches \code{ompd_bp_task_begin}
the binding for \code{ompd_get_curr_task_handle}
is the task that is scheduled to execute.

\crossreferences

\begin{itemize}
\item
  \code{ompd_get_curr_task_handle}, \specref{subsubsubsec:ompd_get_curr_task_handle}
\end{itemize}





\subsubsection{Ending Task Regions}
\label{subsubsec:ompd_bp_task_end}
\index{ompd_bp_task_end@{\code{ompd_bp_task_end}}}

\summary

After finishing the execution of an OpenMP task region, 
the implementation executes \code{ompd_bp_task_end}.

\format
\begin{cspecific}
\begin{ompSyntax}
void ompd_bp_task_end ( void );
\end{ompSyntax}
\end{cspecific}


\descr

The OpenMP implementation must execute \code{ompd_bp_task_end} immediately
after completion of a \plc{structured-block} associated with a non-merged task.
%(It might do so for merged)?

At the point the implementation reaches \code{ompd_bp_task_end}
the binding for \code{ompd_get_curr_task_handle}
is the task that finished execution.

After 
executing \code{ompd_bp_task_end}, any \plc{task_handle} acquired for this
task region is invalid and should be released.


\crossreferences
\begin{itemize}
\item
  \code{ompd_get_curr_task_handle}, \specref{subsubsubsec:ompd_get_curr_task_handle}
\item
  \code{ompd_rel_task_handle}, \specref{subsubsubsec:ompd_rel_task_handle}
\end{itemize}



\subsubsection{Beginning OpenMP Thread}
\label{subsubsec:ompd_bp_thread_begin}
\index{ompd_bp_thread_begin@{\code{ompd_bp_thread_begin}}}

\summary

When starting an OpenMP thread, the implementation executes
\code{ompd_bp_thread_begin}.

\format
\begin{cspecific}
\begin{ompSyntax}
void ompd_bp_thread_begin ( void );
\end{ompSyntax}
\end{cspecific}


\descr

The OpenMP implementation must execute 
\code{ompd_bp_thread_begin} at every \plc{native-thread-begin} and \plc{initial-thread-begin} event.
This must occur before the thread starts the execution of any
OpenMP region.

\crossreferences
\begin{itemize}
\item
  \code{parallel} construct, \specref{sec:parallel Construct}
\item
  Initial task, \specref{subsec:Initial Task}
\end{itemize}



\subsubsection{Ending OpenMP Thread}
\label{subsubsec:ompd_bp_thread_end}
\index{ompd_bp_thread_end@{\code{ompd_bp_thread_end}}}

\summary

When terminating an OpenMP thread, the implementation 
executes \code{ompd_bp_thread_end}.

\format
\begin{cspecific}
\begin{ompSyntax}
void ompd_bp_thread_end ( void );
\end{ompSyntax}
\end{cspecific}


\descr

The OpenMP implementation must execute 
\code{ompd_bp_thread_end} at every \plc{native-thread-end} and the \plc{initial-thread-end} event.
This must occur after the thread completes the execution of any OpenMP region.

After 
executing \code{ompd_bp_thread_end}, any \plc{thread_handle} acquired for this thread 
is invalid and should be released.


\crossreferences
\begin{itemize}
\item
  \code{parallel} construct, \specref{sec:parallel Construct}
\item
  Initial task, \specref{subsec:Initial Task}
\item
  \code{ompd_rel_thread_handle}, \specref{subsubsubsec:ompd_rel_thread_handle}
\end{itemize}






\subsubsection{Beginning OpenMP Device}
\label{subsubsec:ompd_bp_device_begin}
\index{ompd_bp_device_begin@{\code{ompd_bp_device_begin}}}

\summary
The OpenMP implementation must execute 
\code{ompd_bp_device_begin} at every \plc{device-initialize} event.


\format
\begin{cspecific}
\begin{ompSyntax}
void ompd_bp_device_begin ( void );
\end{ompSyntax}
\end{cspecific}


\descr

When initializing a device for executing a target region, the implementation must 
execute \code{ompd_bp_device_begin}.
This should occur before any OpenMP region's work executes on the device.

\crossreferences
\begin{itemize}
\item
  Device Initialization, \specref{subsec:Device Initialization}
\end{itemize}





\subsubsection{Ending OpenMP Device}
\label{subsubsec:ompd_bp_device_end}
\index{ompd_bp_device_end@{\code{ompd_bp_device_end}}}

\summary

When terminating an OpenMP thread, the implementation 
executes \code{ompd_bp_device_end}.

\format
\begin{cspecific}
\begin{ompSyntax}
void ompd_bp_device_end ( void );
\end{ompSyntax}
\end{cspecific}


\descr

The OpenMP implementation must execute 
\code{ompd_bp_device_end} at every \plc{device-finalize} event.
This should occur after the thread executes any OpenMP region.

After 
executing \code{ompd_bp_device_end}, any \plc{address_space_handle} acquired for this
device is invalid and should be released.

\crossreferences
\begin{itemize}
\item
  Device Initialization, \specref{subsec:Device Initialization}
\item
  \code{ompd_rel_address_space_handle},
\specref{subsubsubsec:ompd_rel_address_space_handle}
\end{itemize}

