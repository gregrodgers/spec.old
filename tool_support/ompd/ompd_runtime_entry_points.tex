\subsection{Runtime Entry Points for OMPD}
\label{subsec:runtime-entry-points-for-ompd}

Most of the tool's OpenMP-related activity on an OpenMP
program will be performed through the OMPD interface.
However, supporting OMPD introduced some requirements of the OpenMP
runtime.
These fall into three categories: entrypoints the user's code in
OpenMP program can call; locations in the OpenMP runtime through
which control must pass when specific events occur; and data that must
be accessible to the tool.

Neither a tool nor an OpenMP runtime system know what
application code a program will launch as parallel regions
or tasks until the program invokes the runtime system and
provides a code address as an argument.
To help a tool control the execution of an OpenMP program
launching parallel regions or tasks,
the OpenMP runtime must define a number of symbols
through which execution must pass when particular events occur
\emph{and} data collection for OMPD is enabled.
These locations may, but do not have to, be subroutines.
They may, for example, be labeled locations.
The symbols must all have external, \code{C}, linkage.

A tool can gain notification of the event by planting a breakpoint
at the corresponding named location.

\subsubsection{Beginning Parallel Regions}
\label{subsubsec:ompd_bp_parallel_begin}
\index{ompd_bp_parallel_begin@{\code{ompd_bp_parallel_begin}}}

\summary
The OpenMP runtime must execute through
\code{ompd_bp_parallel_begin} when a new parallel region is launched.


\begin{cspecific}
\begin{ompSyntax}
void ompd_bp_parallel_begin ( void );
\end{ompSyntax}
\end{cspecific}


\descr

When starting a new parallel region, the runtime must allow execution
to flow through \code{ompd_bp_parallel_begin}.
This should occur after a task encounters a parallel construct,
but before any implicit task starts to execute the parallel
region's work.

Control passes through \code{ompd_bp_parallel_begin}
once per region, and not once for each thread per region.

At the point where the runtime reaches \code{ompd_bp_parallel_begin},
a tool can map the encountering native thread to an OpenMP
thread handle using
\code{ompd_get_thread_handle}.
At this point the handle returned by \code{ompd_get_current_parallel_handle}
is that of the new parallel region.
The tool can find the entry point of the user code that
the new parallel region will execute by passing the parallel handle region
to \code{ompd_get_parallel_function}.

The number of threads participating in the parallel region is provided by
the internal variable \plc{ompd-team-size-var} from \code{ompd_get_icv_from_scope}.

The task handle returned by
\code{ompd_get_current_task_handle}
will be that of the task encountering the parallel construct.

The `reenter runtime' address in the information returned by
\code{ompd_get_task_frame}
will be that of the stack frame where the thread called the OpenMP
runtime to handle the parallel construct.
The `exit runtime' address will be for the stack frame where the thread
left the OpenMP runtime to execute the user code that encountered
the parallel construct.

\restrictions

\code{ompd_bp_parallel_begin} has external \code{C} linkage, and no
demangling or other transformations are required by a tool
to look up its address in the OpenMP program.

Conceptually \code{ompd_bp_parallel_begin} has the type signature
given above.
However, it does not need to be a function, but can be a labeled location
in the runtime code.

\crossreferences

\begin{itemize}
\item
  \code{ompd_get_thread_handle}, \specref{subsubsubsec:ompd_get_thread_handle}
\item
  \code{ompd_get_current_parallel_handle}, \specref{subsubsubsec:ompd_get_current_parallel_handle}
\item
  \code{ompd_get_task_function}, \specref{subsubsubsec:ompd_get_task_function}
\item \code{ompd_get_icv_from_scope} call, see 
  \specref{subsubsubsec:ompd_get_icv_from_scope}.
\item
  \code{ompd_get_current_task_handle}, \specref{subsubsubsec:ompd_get_current_task_handle}
\item
  \code{ompd_get_task_frame}, \specref{subsubsubsec:ompd_get_task_frame}
\end{itemize}

\subsubsection{Ending Parallel Regions}
\label{subsubsec:ompd_bp_parallel_end}
\index{ompd_bp_parallel_end@{\code{ompd_bp_parallel_end}}}

The OpenMP runtime must execute through \code{ompd_bp_parallel_end}
when a parallel region ends.


\begin{cspecific}
\begin{ompSyntax}
void ompd_bp_parallel_end ( void );
\end{ompSyntax}
\end{cspecific}


\descr

When a parallel region finishes, the OpenMP runtime must allow execution
to flow through \code{ompd_bp_parallel_end}.

Control passes through \code{ompd_bp_parallel_end}
once per region, and not once for each thread per region.

At the point the runtime reaches \code{ompd_bp_parallel_end}
the tool can map the encountering native thread
to an OpenMP thread handle using \code{ompd_get_thread_handle}.
\code{ompd_get_current_parallel_handle}
returns the handle of the terminating parallel region.

\code{ompd_get_current_task_handle}
returns the handle of the task that encountered the
parallel construct that initiated the parallel region just
terminating.
The `reenter runtime' address in the frame information returned by
\code{ompd_get_task_frame}
will be that for the stack frame in which the thread called the
OpenMP runtime to start the parallel construct just terminating.
The `exit runtime' address will refer to the stack frame where the
thread left the OpenMP runtime to execute the user code that
invoked the parallel construct just terminating.

\restrictions

\code{ompd_bp_parallel_end} has external \code{C} linkage, and no
demangling or other transformations are required by a tool
to look up its address in the OpenMP program.

Conceptually \code{ompd_bp_parallel_end} has the type signature
given above.
However, it does not need to be a function, but can be a labeled location
in the runtime code.

\crossreferences
\begin{itemize}
\item
  \code{ompd_get_thread_handle}, \specref{subsubsubsec:ompd_get_thread_handle}
\item
  \code{ompd_get_current_parallel_handle}, \specref{subsubsubsec:ompd_get_current_parallel_handle}
\item
  \code{ompd_get_current_task_handle}, \specref{subsubsubsec:ompd_get_current_task_handle}
\item
  \code{ompd_get_task_frame}, \specref{subsubsubsec:ompd_get_task_frame}
\end{itemize}


\subsubsection{Beginning Task Regions}
\label{subsubsec:ompd_bp_task_begin}
\index{ompd_bp_task_begin@{\code{ompd_bp_task_begin}}}

The OpenMP runtime must execute through \code{ompd_bp_task_begin}
when a new task is started.


\begin{cspecific}
\begin{ompSyntax}
void ompd_bp_task_begin ( void );
\end{ompSyntax}
\end{cspecific}


\descr

When starting a new task region, the OpenMP runtime system
must allow control to pass through \code{ompd_bp_task_begin}.

The OpenMP runtime system will execute through this location after the task
construct is encountered, but before the new explicit task starts.
At the point where the runtime reaches \code{ompd_bp_task_begin}
the tool can map the native thread to an OpenMP handle using
\code{ompd_get_thread_handle}.

\code{ompd_get_current_task_handle} returns the handle of the new task region.
The entry point of the user code to be executed by the new task
is returned from
\code{ompd_get_task_function}.

\restrictions

\code{ompd_bp_task_begin} has external \code{C} linkage, and no
demangling or other transformations are required by a tool
to look up its address in the OpenMP program.

Conceptually \code{ompd_bp_task_begin} has the type signature
given above.
However, it does not need to be a function, but can be a labeled location
in the runtime code.

\crossreferences
\begin{itemize}
\item
  \code{ompd_get_thread_handle}, \specref{subsubsubsec:ompd_get_thread_handle}
\item
  \code{ompd_get_current_task_handle}, \specref{subsubsubsec:ompd_get_current_task_handle}
\item
  \code{ompd_get_task_function}, \specref{subsubsubsec:ompd_get_task_function}
\end{itemize}

\subsubsection{Ending Task Regions}
\label{subsubsec:ompd_bp_task_end}
\index{ompd_bp_task_end@{\code{ompd_bp_task_end}}}

\summary
The OpenMP runtime must execute through \code{ompd_bp_task_end}
when a task region ends.


\begin{cspecific}
\begin{ompSyntax}
void ompd_bp_task_end ( void );
\end{ompSyntax}
\end{cspecific}


\descr

When a task region completes, the OpenMP runtime system
must allow execution to flow through the location \code{ompd_bp_task_end}.

At the point where the runtime reaches \code{ompd_bp_task_end}
the tool can use
\code{ompd_get_thread_handle}
to map the encountering native thread to the corresponding
OpenMP thread handle.
At this point \code{ompd_get_current_task_handle}
returns the handle for the terminating task.

\restrictions

\code{ompd_bp_task_end} has external \code{C} linkage, and no
demangling or other transformations are required by a tool
to look up its address in the OpenMP program.

Conceptually \code{ompd_bp_task_end} has the type signature
given above.
However, it does not need to be a function, but can be a labeled location
in the runtime code.

\crossreferences
\begin{itemize}
\item
  \code{ompd_get_thread_handle}, \specref{subsubsubsec:ompd_get_thread_handle}
\item
  \code{ompd_get_current_task_handle}, \specref{subsubsubsec:ompd_get_current_task_handle}
\end{itemize}

