\subsection{OMPD Tool Callback Interface}
\label{ompd:ompd-tool-callback-interface}

For the OMPD plugin to provide information about the internal state
of the OpenMP runtime system in an OpenMP process or core file,
it must have a means to extract information from
the OpenMP process that the tool is debugging.
The OpenMP process that the tool is operating on may be either a ``live'' process or a core file,
and a thread may be either a ``live'' thread in an OpenMP process,
or a thread in a core file.
To enable OMPD to extract state information from an OpenMP process or core file,
the tool must supply the OMPD with callback functions to inquire
about the size of primitive types in the device of the OpenMP process,
look up the addresses of symbols,
as well as read and write memory in the device.
OMPD then uses these callbacks to implement its interface operations.
Signatures for the tool callbacks used by OMPD are given below.

\subsubsection{Memory Management of OMPD Plugin}
\begin{comment}
The OMPD plugin \emph{must} obtain and release heap memory \emph{only}
using the callbacks provided to it by the tool.
It must \emph{not} call the heap manager directly using \code{malloc}.
For C++ implementations of OMPD this means that \code{new},
\code{new(throw)}, \code{new[]}, \code{delete}, \code{delete(throw)},
and \code{delete[]} in \emph{all} their variants \emph{must} be overloaded,
and implemented in terms of the callback functions
provided to it by the tool.
\end{comment}
The OMPD library must not access the heap manager directly. Instead, if it needs heap memory it 
must use the memory allocation and deallocation callback functions that are described in this 
section, \code{ompd\_callback\_memory\_alloc\_fn\_t} and \code{ompd\_callback\_memory\_free\_fn\_t},  which are 
provided by the tool to obtain and release heap memory. This will ensure that the library does not 
interfere with any custom memory management scheme that the tool may use.

If the OMPD library is implemented in \code{C++}, memory management operators like \code{new} 
and \code{delete} in all their variants, \emph{must all} be overloaded and implemented in terms of 
the callbacks provided by the tool. The OMPD library must be coded so that any of its 
definitions of \code{new} or \code{delete} do not interfere with any defined by the tool.

In some cases, the OMPD library will need to allocate memory to return results to the
tool. This memory will then be `owned' by the tool, which will be responsible for releasing it. It is 
therefore vital that the OMPD library and the tool use the same memory manager.

OMPD handles are created by the OMPD library. These are opaque to the 
tool, and depending on the specific implementation of OMPD may have complex 
internal structure. The tool cannot know whether the handle pointers returned by the API 
correspond to discrete heap allocations. Consequently, the tool must not simply deallocate a 
handle by passing an address it receives from the OMPD library to its own memory manager. 
Instead, the API includes functions that the tool must use when it no longer needs a handle.

Contexts are created by a tool and passed to the OMPD library. The OMPD 
library does not need to release contexts; instead this will be done by the tool after it releases any 
handles that may be referencing the contexts.

\subsubsubsection{ompd\_callback\_memory\_alloc\_fn\_t}
\label{ompd:ompd_callback_memory_alloc_fn_t}
\index{ompd\_callback\_memory\_alloc\_fn\_t@{\code{ompd\_callback\_memory\_alloc\_fn\_t}}}

\summary
The type signature of the callback routine provided by the tool
to be used by the OMPD plugin to allocate memory.

\vbox{
% the odd-looking spacing between type and argument name ensures
% they line up in the pdf
\cspecificstart
\begin{boxedcode}
typedef ompd\_rc\_t (*ompd\_callback\_memory\_alloc\_fn\_t) (
  ompd\_size\_t   \plc{nbytes},
  void        **\plc{ptr}
);
\end{boxedcode}
\cspecificend
}

\descr
The OMPD plugin may call the \code{ompd\_callback\_memory\_alloc\_fn\_t} callback function to allocate memory.

\argdesc
The argument \plc{nbytes} gives the size in bytes of the block of memory the
OMPD wants allocated.

The address of the newly allocated
block of memory is returned in \plc{*ptr}.
The newly allocated block is suitably aligned for any type of variable,
and is not guaranteed to be zeroed.

\crossreferences
\begin{itemize}
\item
  \code{ompd\_size\_t}, \specref{ompd:ompd_size_t}
\item
  \code{ompd\_rc\_t}, \specref{ompd:ompd_rc_t}
\end{itemize}

\subsubsubsection{ompd\_callback\_memory\_free\_fn\_t}
\label{ompd:ompd_callback_memory_free_fn_t}
\index{ompd\_callback\_memory\_free\_fn\_t@{\code{ompd\_callback\_memory\_free\_fn\_t}}}

\summary
The type signature of the callback routine provided by the 
tool to be used by the OMPD plugin to deallocate memory.

\vbox{
\cspecificstart
\begin{boxedcode}
typedef ompd\_rc\_t (*ompd\_callback\_memory\_free\_fn\_t) (
  void    *\plc{ptr}
);
\end{boxedcode}
\cspecificend
}
\descr
The OMPD plugin calls the \code{ompd\_callback\_memory\_free\_fn\_t} callback function to 
deallocate memory obtained from a prior call to the \code{ompd\_callback\_memory\_alloc\_fn\_t} 
callback function.

\argdesc
\plc{ptr} is the address of the block to be deallocated.

\crossreferences
\begin{itemize}
\item
  \code{ompd\_callback\_memory\_alloc\_fn\_t}, \specref{ompd:ompd_callback_memory_alloc_fn_t}
\item
  \code{ompd\_rc\_t}, \specref{ompd:ompd_rc_t}
\item
  \code{ompd\_callbacks\_t}, \specref{ompd:ompd_callbacks_t}
\end{itemize}

\subsubsection{Context Management and Navigation}

The tool provides the OMPD plugin with callbacks
to manage and navigate context relationships.

\subsubsubsection{ompd\_callback\_get\_thread\_context\_for\_thread\_id\_fn\_t}
\label{ompd:ompd_callback_get_thread_context_for_thread_id_fn_t}
% the use of \discretionary below overrides the character used for
% hyphenation when an optional linebreak is used.
% if \- were used, we'd get a "-" at the end of the line being
% broken.  We probably don't wantthat, but instead want just a plain
% linebreak without the "-".
\index{ompd\_callback\_get\_thread\_context\_for\_thread\_id\_fn\_t@{\code{ompd\_callback\_get\_thread\_context\_for\_\discretionary{}{}{}thread\_id\_fn\_t}}}

\summary
The type signature of the callback routine provided by the 
third party tool the OMPD library can use to map a
thread identifier to a tool \plc{thread context}.

\vbox{
% the odd-looking spacing between type and argument name ensures
% they line up in the pdf
\cspecificstart
\begin{boxedcode}
typedef ompd\_rc\_t
  (*ompd\_callback\_get\_thread\_context\_for\_thread\_id\_fn\_t) (
    \code{ompd\_address\_space\_context\_t}* \plc{address\_space\_context},
    \code{ompd\_thread\_id\_kind\_t} \plc{kind},
    \code{ompd\_size\_t} \plc{sizeof\_thread\_id},
    const void* \plc{thread\_id},
    \code{ompd\_thread\_context\_t}** \plc{thread\_context}
);
\end{boxedcode}
\cspecificend
}

\descr

This callback maps a thread identifier within the address
space identified by \plc{address\_space\_context} to a tool thread context. The OMPD plugin library 
can use the thread context, for example, to access
thread local storage (TLS).

\argdesc

The input argument \plc{address\_space\_context} is an opaque handle provided by the tool
to reference an address space.
The input arguments \plc{kind},  \plc{sizeof\_thread\_id}, and \plc{thread\_id} represent a thread identifier. 
On return the output argument \plc{thread\_context} provides an opaque handle to the OMPD 
plugin library that maps a thread identifier to a tool thread context.


%On success the tool \plc{thread context} is returned in \plc{*thread\_context}.
%The \plc{thread context} is created, and remains owned, by the tool.
%The OMPD plugin can assume that the \plc{thread context} is valid for
%as long as the tool is holding any references to \plc{thread handles}
%that may contain the \plc{thread context}.
%The OMPD plugin library can use the \plc{thread context}, for example, to access
%thread local storage (TLS).

%On failure, the callback returns an error code from \code{ompd\_rc\_t}.


\subsubsubsection{ompd\_callback\_get\_address\_space\_context\_for\_thread\_fn\_t}
\label{ompd:ompd_callback_get_address_space_context_for_thread_fn_t}
% the use of \discretionary below overrides the character used for
% hyphenation when an optional linebreak is used.
% if \- were used, we'd get a "-" at the end of the line being
% broken.  We probably don't wantthat, but instead want just a plain
% linebreak without the "-".
\index{ompd\_callback\_get\_address\_space\_context\_for\_thread\_fn\_t@{\code{ompd\_callback\_get\_address\_space\_context\_\discretionary{}{}{}for\_thread\_fn\_t}}}

\summary
The type signature of the callback routine provided by the 
tool the OMPD plugin can use to find the address space context
for a thread identified by a thread context.

\vbox{
% the odd-looking spacing between type and argument name ensures
% they line up in the pdf
\cspecificstart
\begin{boxedcode}
typedef ompd\_rc\_t
  (*ompd\_callback\_get\_address\_space\_context\_for\_thread\_fn\_t) (
    \code{ompd\_thread\_context\_t}          *\plc{thread\_context},
    \code{ompd\_address\_space\_context\_t}  **\plc{address\_space\_context}
);
\end{boxedcode}
\cspecificend
}
\descr
This callback maps a tool thread context to its address space context.

\argdesc
This callback finds the tool address space context for the thread
identified by the tool thread context \plc{thread\_context}.
The callback returns the address space context in \plc{*address\_space\_context}.
If \plc{thread\_context} refers to a host device thread,
this function returns the context for the host address space.
If \plc{thread\_context} refers to a device thread,
this function returns the device's address space.

\crossreferences
\begin{itemize}
\item
  \code{ompd\_address\_space\_context\_t}, \specref{ompd:ompd_address_space_context_t}
\item
  \code{ompd\_thread\_context\_t}, \specref{ompd:ompd_thread_context_t}
\item
  \code{ompd\_rc\_t}, \specref{ompd:ompd_rc_t}
\item
  \code{ompd\_callbacks\_t}, \specref{ompd:ompd_callbacks_t}
\end{itemize}

\subsubsubsection{ompd\_callback\_get\_process\_context\_for\_address\_space\_context\_fn\_t}
\label{ompd:ompd_callback_get_process_context_for_address_space_context}
% the use of \discretionary below overrides the character used for
% hyphenation when an optional linebreak is used.
% if \- were used, we'd get a "-" at the end of the line being
% broken.  We probably don't wantthat, but instead want just a plain
% linebreak without the "-".
\index{ompd\_callback\_get\_process\_context\_for\_context\_fn\_t@{\code{ompd\_callback\_get\_process\_context\_for\_\discretionary{}{}{}context\_fn\_t}}}

\summary
The type signature of the callback routine provided by the 
third party tool the OMPD library can use to map a tool address space context
to a tool address space context of the OpenMP process.

\vbox{
% the odd-looking spacing between type and argument name ensures
% they line up in the pdf
\cspecificstart
\begin{boxedcode}
typedef ompd\_rc\_t 
  (*ompd\_callback\_get\_process\_context\_for\_context\_fn\_t) (
    \code{ompd\_address\_space\_context\_t}*  \plc{target\_device\_address\_space\_context},
    \code{ompd\_address\_space\_context\_t}** \plc{host\_device\_address\_space\_context},
);
\end{boxedcode}
\cspecificend
}
\descr
This callback maps a tool address space context for a target device to
the address space context for the host device.

\argdesc
This callback maps the tool address space context 
for a target device provided in \plc{target\_device\_address\_space\_context} to
the address space context of its containing host device which it returns
in *\plc{host\_device\_address\_space\_context}.

\crossreferences
\begin{itemize}
\item
  \code{ompd\_address\_space\_context\_t}, \specref{ompd:ompd_address_space_context_t}
\item
  \code{ompd\_rc\_t}, \specref{ompd:ompd_rc_t}
\item
  \code{ompd\_callbacks\_t}, \specref{ompd:ompd_callbacks_t}
\end{itemize}
\subsubsection{Sizes of Primitive Types}

The tool provides the OMPD plugin with callbacks
to discover information about the sizes of the basic primitive types
in an address space.
This information is important to an OMPD plugin because the architecture
or programming model of the OpenMP runtime it is examining may be different
from the its own and that of the tool which loaded it.
On platforms with multiple programming models, an OMPD plugin may need
to be able to handle OpenMP runtimes using different sizes for the
basic types.

\crossreferences
\begin{itemize}
\item
  OMPD --- Overivew, \specref{ompd:ompd-overview}
\end{itemize}


\subsubsubsection{ompd\_callback\_sizeof\_fn\_t}
\label{ompd:ompd_callback_sizeof_fn_t}
\index{ompd\_callback\_sizeof\_fn\_t@{\code{ompd\_callback\_sizeof\_fn\_t}}}

\summary
The type signature of the callback routine provided by the 
tool the OMPD plugin can use to find the sizes of the primitive types
in an address space.

\vbox{
% the odd-looking spacing between type and argument name ensures
% they line up in the pdf
\cspecificstart
\begin{boxedcode}
typedef ompd\_rc\_t (*ompd\_callback\_sizeof\_fn\_t) (
    \code{ompd\_address\_space\_context\_t}  *\plc{address\_space\_context},
    \code{ompd\_device\_type\_sizes\_t}      *\plc{sizes}
);
\end{boxedcode}
\cspecificend
}
\descr
This callback may be called by the OMPD library to obtain the sizes of the basic
primitive types for a given address space.

\argdesc
The callback returns the sizes of
the basic primitive types used by the \plc{address\_space\_context}
in \plc{*sizes}.

\crossreferences
\begin{itemize}
\item
  \code{ompd\_address\_space\_context\_t}, \specref{ompd:ompd_address_space_context_t}
\item
  \code{ompd\_device\_type\_sizes\_t}, \specref{ompd:ompd_callback_device_type_sizes_t}
\item
  \code{ompd\_rc\_t}, \specref{ompd:ompd_rc_t}
\item
  \code{ompd\_callbacks\_t}, \specref{ompd:ompd_callbacks_t}
\end{itemize}

\subsubsection{ompd\_callback\_symbol\_addr\_fn\_t}
\label{ompd:ompd_callback_symbol_addr_fn_t}
\index{ompd\_callback\_symbol\_addr\_fn\_t@{\code{ompd\_callback\_symbol\_addr\_fn\_t}}}

\summary
The type signature of the callback provided by the tool the
OMPD plugin can use to look up the addresses of symbols in an OpenMP program.

\vbox{
% the odd-looking spacing between type and argument name ensures
% they line up in the pdf
\cspecificstart
\begin{boxedcode}
typedef ompd\_rc\_t (*ompd\_callback\_symbol\_addr\_fn\_t) (
    ompd\_address\_space\_context\_t*  \plc{address\_space\_context},
    ompd\_thread\_context\_t*         \plc{thread\_context},
    const char*                    \plc{symbol\_name},
    ompd\_address\_t*                \plc{symbol\_addr}
);
\end{boxedcode}
\cspecificend
}
\descr
This callback function may be called by the OMPD library to look up
addresses of symbols within an specified address space of the tool.

\argdesc
This callback looks up the symbol provided in \plc{symbol\_name}.
The \plc{thread\_context} is an optional input parameter which should
be NULL for global memory access.

If  \plc{thread\_context} is not NULL, 
it gives the thread specific
context for the symbol lookup, for the purpose of calculating thread
local storage~(TLS) addresses.

The \plc{symbol\_name} supplied by the OMPD plugin is used verbatim by
the tool, and in particular, no name mangling, demangling
or other transformations are performed prior to the lookup.

The callback returns the address
of the symbol in \plc{*symbol\_addr}.

\crossreferences
\begin{itemize}
\item
  \code{ompd\_address\_space\_context\_t}, \specref{ompd:ompd_address_space_context_t}
\item
  \code{ompd\_thread\_context\_t}, \specref{ompd:ompd_thread_context_t}
\item
  \code{ompd\_address\_t}, \specref{ompd:ompd_address_t}
\item
  \code{ompd\_rc\_t}, \specref{ompd:ompd_rc_t}
\item
  \code{ompd\_callbacks\_t}, \specref{ompd:ompd_callbacks_t}
\end{itemize}

\subsubsection{Accessing Memory in the OpenMP Program or Runtime}

The OMPD plugin may need to read from, or write to, the OpenMP program.
It cannot do this directly, but instead must use the callbacks provided
to it by the tool, which will perform the operation
on its behalf.

\subsubsubsection{ompd\_callback\_memory\_read\_fn\_t}
\label{ompd:ompd_callback_memory_read_fn_t}
\index{ompd\_callback\_memory\_read\_fn\_t@{\code{ompd\_callback\_memory\_read\_fn\_t}}}

\summary

The type signature of the callback provided by the tool the
OMPD plugin can use to read data out of an OpenMP program.


\vbox{
% the odd-looking spacing between type and argument name ensures
% they line up in the pdf
\cspecificstart
\begin{boxedcode}
typedef ompd\_rc\_t (*ompd\_callback\_memory\_read\_fn\_t) (
    ompd\_address\_space\_context\_t  *\plc{address\_space\_context},
    ompd\_thread\_context\_t         *\plc{thread\_context},
    const ompd\_address\_t          *\plc{addr},
    ompd\_size\_t                    \plc{nbytes},
    void                          *\plc{buffer}
);
\end{boxedcode}
\cspecificend
}
\descr
The OMPD plugin may call this function callback to have the tool read a block of data from a location within an
address space into a provided buffer.

\argdesc
The address from which the data are to be read out of the OpenMP program
specified by \plc{address\_space\_context} is given by \plc{addr}.
\plc{nbytes} gives the number of bytes to be transfered.
The \plc{thread\_context} argument is optional for global memory access,
and in this case should be NULL.
If it is not NULL, \plc{thread\_context} identifies the thread
specific context for the memory access for the purpose of accessing
thread local storage (TLS).

The data are returned through \plc{buffer}, which is allocated and
owned by the OMPD plugin.
The contents of the buffer are unstructured, raw bytes.
It is the responsibility of the OMPD plugin to arrange for
any transformations such as byte-swapping that may be necessary
(see~\specref{ompd:ompd_callback_device_host_fn_t}) to interpret the
data returned.

\crossreferences
\begin{itemize}
\item
  \code{ompd\_address\_space\_context\_t}, \specref{ompd:ompd_address_space_context_t}
\item
  \code{ompd\_thread\_context\_t}, \specref{ompd:ompd_thread_context_t}
\item
  \code{ompd\_address\_t}, \specref{ompd:ompd_address_t}
\item
  \code{ompd\_word\_t}, \specref{ompd:ompd_word_t}
\item
  \code{ompd\_rc\_t}, \specref{ompd:ompd_rc_t}
\item
  \code{ompd\_callback\_device\_host\_fn\_t}, \specref{ompd:ompd_callback_device_host_fn_t}
\item
  \code{ompd\_callbacks\_t}, \specref{ompd:ompd_callbacks_t}
\end{itemize}

\subsubsubsection{ompd\_callback\_memory\_write\_fn\_t}
\label{ompd:ompd_callback_memory_write_fn_t}
\index{ompd\_callback\_memory\_write\_fn\_t@{\code{ompd\_callback\_memory\_write\_fn\_t}}}

\summary

The type signature of the callback provided by the tool the
OMPD plugin can use to write data to an OpenMP program.

\vbox{
% the odd-looking spacing between type and argument name ensures
% they line up in the pdf
\cspecificstart
\begin{boxedcode}
typedef ompd\_rc\_t (*ompd\_callback\_memory\_write\_fn\_t) (
    ompd\_address\_space\_context\_t  *\plc{address\_space\_context},
    ompd\_thread\_context\_t         *\plc{thread\_context},
    const ompd\_address\_t          *\plc{addr},
    ompd\_size\_t                    \plc{nbytes},
    const void                    *\plc{buffer}
);
\end{boxedcode}
\cspecificend
}
\descr
The OMPD plugin may call this function callback to have the tool write a block of data
to a location within an address space from a provided buffer.

\argdesc
The address to which the data are to be written in the OpenMP program
specified by \plc{address\_space\_context} is given by \plc{addr}.
\plc{nbytes} gives the number of bytes to be transfered.
The \plc{thread\_context} argument is optional for global memory access,
and in this case should be NULL.
If it is not NULL, \plc{thread\_context} identifies the thread
specific context for the memory access for the purpose of accessing
thread local storage (TLS).

The data to be written are passed through \plc{buffer}, which is allocated and
owned by the OMPD plugin.
The contents of the buffer are unstructured, raw bytes.
It is the responsibility of the OMPD plugin to arrange for
any transformations such as byte-swapping that may be necessary
(see~\specref{ompd:ompd_callback_device_host_fn_t})
to render the data into a form compatible with the OpenMP runtime.

\crossreferences
\begin{itemize}
\item
  \code{ompd\_address\_space\_context\_t} \specref{ompd:ompd_address_space_context_t}
\item
  \code{ompd\_thread\_context\_t}, \specref{ompd:ompd_thread_context_t}
\item
  \code{ompd\_address\_t}, \specref{ompd:ompd_address_t}
\item
  \code{ompd\_word\_t}, \specref{ompd:ompd_word_t}
\item
  \code{ompd\_rc\_t}, \specref{ompd:ompd_rc_t}
\item
  \code{ompd\_callback\_device\_host\_fn\_t}, \specref{ompd:ompd_callback_device_host_fn_t}
\item
  \code{ompd\_callbacks\_t}, \specref{ompd:ompd_callbacks_t}
\end{itemize}

\subsubsection{Data Format Conversion}
\label{ompd:data-format-conversion}

The architecuture and/or programming-model of tool and
OMPD plugin may be different from that of the OpenMP program being
examined.
Consequently, the conventions for representing data will differ.
The callback interface includes operations for converting between
the conventions, such as byte order (`endianness'),
used by the tool and OMPD plugin on the
one hand, and the OpenMP program on the other.

\subsubsubsection{ompd\_callback\_device\_host\_fn\_t}
\label{ompd:ompd_callback_device_host_fn_t}
\index{ompd\_callback\_device\_host\_fn\_t@{\code{ompd\_callback\_device\_host\_fn\_t}}}

\summary

The type signature of the callback provided by the tool the
OMPD plugin can use to convert data between the formats used by the
tool and OMPD plugin, and the OpenMP program.

\vbox{
% the odd-looking spacing between type and argument name ensures
% they line up in the pdf
\cspecificstart
\begin{boxedcode}
typedef ompd\_rc\_t (*ompd\_callback\_device\_host\_fn\_t) (
    ompd\_address\_space\_context\_t  *\plc{address\_space\_context},
    const void                    *\plc{input},
    ompd\_size\_t                    \plc{unit\_size},
    ompd\_size\_t                    \plc{count},
    void                          *\plc{output}
);
\end{boxedcode}
\cspecificend
}
\descr
This callback function may be called by the OMPD plugin to convert
data between formats used by the tool and OMPD plugin, and the OpenMP program.
\argdesc
The OpenMP address space associated with the data is given by
\plc{address\_space\_context}.
The source and destination buffers are given by \plc{input}
and \plc{output}, respectively.
\plc{unit\_size} gives the size of each of the elements to be converted.
\plc{count} is the number of elements to be transformed.

The input and output buffers are allocated and owned by the OMPD plugin,
and it is its responsibility to ensure that the buffers are the correct
size, and eventually deallocated when they are no longer needed.

\crossreferences
\begin{itemize}
\item
  \code{ompd\_address\_space\_context\_t}, \specref{ompd:ompd_address_space_context_t}
\item
  \code{ompd\_rc\_t}, \specref{ompd:ompd_rc_t}
\item
  \code{ompd\_callbacks\_t}, \specref{ompd:ompd_callbacks_t}
\end{itemize}

\subsubsection{ompd\_callback\_print\_string\_fn\_t}
\label{ompd:ompd_callback_print_string_fn_t}
\index{ompd\_callback\_print\_string\_fn\_t@{\code{ompd\_callback\_print\_string\_fn\_t}}}

\summary

The type signature of the callback provided by the tool the
OMPD plugin can use to emit output.

\vbox{
% the odd-looking spacing between type and argument name ensures
% they line up in the pdf
\cspecificstart
\begin{boxedcode}
typedef ompd\_rc\_t (*ompd\_callback\_print\_string\_fn\_t) (
    const char        *\plc{string}
);
\end{boxedcode}
\cspecificend
}
\descr
The OMPD plugin emits output, such as logging or debug information,
using a callback supplied to it by the tool.
It should not emit output directly.
\argdesc
\plc{string} is the null-terminated string to be printed.
No conversion or formating is performed on the string.

\crossreferences
\begin{itemize}
\item
  \code{ompd\_rc\_t}, \specref{ompd:ompd_rc_t}
\item
  \code{ompd\_callbacks\_t}, \specref{ompd:ompd_callbacks_t}
\end{itemize}

\subsubsection{The Callback Interface}
\label{ompd:ompd_callbacks_t}
\index{ompd\_callbacks\_t@{\code{ompd\_callbacks\_t}}}

\summary

All the OMPD plugin's interactions with the OpenMP program
must be through a set of callbacks provided to it by the
tool which loaded it.
These callbacks must also be used for allocating or releasing resources,
such as memory, that the plugin needs.

\vbox{
% the odd-looking spacing between type and argument name ensures
% they line up in the pdf
\cspecificstart
\begin{boxedcode}
typedef struct \{
  ompd\_callback\_memory\_alloc\_fn\_t \plc{alloc\_memory};
  ompd\_callback\_memory\_free\_fn\_t  \plc{free\_memory};
  ompd\_callback\_print\_string\_fn\_t \plc{print\_string};
  ompd\_callback\_sizeof\_fn\_t       \plc{sizeof\_type};
  ompd\_callback\_symbol\_addr\_fn\_t  \plc{symbol\_addr\_lookup};
  ompd\_callback\_memory\_read\_fn\_t  \plc{read\_memory};
  ompd\_callback\_memory\_write\_fn\_t \plc{write\_memory};
  ompd\_callback\_device\_host\_fn\_t  \plc{device\_to\_host};
  ompd\_callback\_device\_host\_fn\_t  \plc{host\_to\_device};
  ompd\_callback\_get\_thread\_context\_for\_thread\_id\_fn\_t
                         \plc{get\_thread\_context\_for\_thread\_id};
  ompd\_callback\_get\_address\_space\_context\_for\_thread\_fn\_t
                         \plc{get\_address\_space\_context\_for\_thread};
  ompd\_callback\_get\_process\_context\_for\_address\_space\_context\_fn\_t
                         \plc{get\_containing\_process\_context};
\} ompd\_callbacks\_t;
\end{boxedcode}
\cspecificend
}

\descr
The set of callbacks the OMPD plugin should use is collected
in the \code{ompd\_callbacks\_t} record structure.
An instance of this type is passed to the OMPD plugin
as a parameter to \code{ompd\_initialize}.
Each field points to a function that the OMPD plugin should use
for interacting with the OpenMP program, or getting memory from
the tool.

The \plc{alloc\_memory} and \plc{free\_memory} fields are
pointers to functions the OMPD plugin uses to allocate and release
dynamic memory.

\plc{print\_string} points to a function that prints a string.

The architectures or programming models of the OMPD plugin and
party tool may be different from that of the OpenMP
program being examined.
\plc{sizeof\_type} points to function that allows
the OMPD plugin to determine the sizes of the basic integer
and pointer types used by the OpenMP program.
Because of the differences in architecture or programming model,
the conventions for representing data in the OMPD plugin and
the OpenMP program may be different.
The \plc{device\_to\_host} field points to a function which translates
data from the conventions used by the OpenMP program to that used
by the tool and OMPD plugin.
The reverse operation is performed by the function pointed
to by the \plc{host\_to\_device} field.

The OMPD plugin may need to access memory in the OpenMP program.
The \plc{symbol\_addr\_lookup} field points to a callback the
OMPD plugin can use to find the address of a global or thread
local storage (TLS) symbol.
The \plc{read\_memory} and \plc{write\_memory} fields are
pointers to functions for reading from, and writing to, global or TLS
memory in the OpenMP program, respectively.

\plc{get\_thread\_context\_for\_thread\_id} is a pointer to a function
the OMPD plugin can use to obtain a thread context that corresponds to
a thread identifier.
\plc{get\_address\_space\_context\_for\_thread} points to a callback
the OMPD plugin can use to get the tool context that `owns' the
thread represented by a tool thread context.

\crossreferences
\begin{itemize}
\item
  \code{ompd\_callback\_memory\_alloc\_fn\_t}, \specref{ompd:ompd_callback_memory_alloc_fn_t}
\item
  \code{ompd\_callback\_memory\_free\_fn\_t}, \specref{ompd:ompd_callback_memory_free_fn_t}
\item
  \code{ompd\_callback\_print\_string\_fn\_t}, \specref{ompd:ompd_callback_print_string_fn_t}
\item
  \code{ompd\_callback\_sizeof\_fn\_t}, \specref{ompd:ompd_callback_sizeof_fn_t}
\item
  \code{ompd\_callback\_symbol\_addr\_fn\_t}, \specref{ompd:ompd_callback_symbol_addr_fn_t}
\item
  \code{ompd\_callback\_memory\_read\_fn\_t}, \specref{ompd:ompd_callback_memory_read_fn_t}
\item
  \code{ompd\_callback\_memory\_write\_fn\_t}, \specref{ompd:ompd_callback_memory_write_fn_t}
\item
  \code{ompd\_callback\_device\_host\_fn\_t}, \specref{ompd:ompd_callback_device_host_fn_t}
\item
  \code{ompd\_callback\_get\_thread\_context\_for\_thread\_id\_fn\_t}, \specref{ompd:ompd_callback_get_thread_context_for_thread_id_fn_t}
\item
  \code{ompd\_callback\_get\_address\_space\_context\_for\_thread\_fn\_t}, \specref{ompd:ompd_callback_get_address_space_context_for_thread_fn_t}
\item
  \code{ompd\_callback\_get\_process\_context\_for\_address\_space\_context\_fn\_t}, \specref{ompd:ompd_callback_get_process_context_for_address_space_context}
\end{itemize}

