\section{OMPD Tool Callback Interface}
\label{subsec:ompd-tool-callback-interface}
\label{subsec:third-party-tool-callback-interface}

For the OMPD library to provide information about the internal state of the 
OpenMP runtime system in an OpenMP process or core file, it must have a means 
to extract information from the OpenMP process that the tool is debugging.
The OpenMP process on which the tool is operating may be either a ``live'' 
process or a core file, and a thread may be either a ``live'' thread in an 
OpenMP process, or a thread in a core file. To enable the OMPD library to 
extract state information from an OpenMP process or core file, the tool must 
supply the OMPD library with callback functions to inquire about the size of 
primitive types in the device of the OpenMP process, to look up the addresses 
of symbols, and to read and to write memory in the device. The OMPD library 
uses these callbacks to implement its interface operations. The OMPD library 
only invokes the callback functions in direct response to calls made by the 
tool to the OMPD library.



\subsection{Memory Management of OMPD Library}
The OMPD library must not access the heap manager directly. Instead, if it 
needs heap memory it must use the memory allocation and deallocation callback 
functions that are described in this section, 
\code{ompd_callback_memory_alloc_fn_t} 
(see \specref{subsubsubsec:ompd_callback_memory_alloc_fn_t}) and 
\code{ompd_callback_memory_free_fn_t} 
(see \specref{subsubsubsec:ompd_callback_memory_free_fn_t}), which are
provided by the tool to obtain and to release heap memory. This mechanism
ensures that the library does not interfere with any custom memory management 
scheme that the tool may use.

If the OMPD library is implemented in C++, memory management operators like 
\code{new} and \code{delete} in all their variants, \emph{must all} be overloaded 
and implemented in terms of the callbacks that the tool provides. The OMPD 
library must be coded so that any of its definitions of \code{new} or 
\code{delete} do not interfere with any that the tool defines.

In some cases, the OMPD library must allocate memory to return results to the 
tool. The tool then owns this memory and has the responsibility to release
it. Thus, the OMPD library and the tool must use the same memory manager.

The OMPD library creates OMPD handles, which are opaque to the tool and may have 
a complex internal structure. The tool cannot determine if the handle pointers 
that the API returns correspond to discrete heap allocations. Thus, the tool must 
not simply deallocate a handle by passing an address that it receives from the 
OMPD library to its own memory manager. Instead, the API includes functions that 
the tool must use when it no longer needs a handle.

A tool creates contexts and passes them to the OMPD library. The OMPD library 
does not release contexts; instead the tool release them after it releases any 
handles that may reference the contexts.



\subsubsection{\hcode{ompd_callback_memory_alloc_fn_t}}
\label{subsubsubsec:ompd_callback_memory_alloc_fn_t}
\index{ompd_callback_memory_alloc_fn_t@{\code{ompd_callback_memory}\discretionary{}{}{}\code{_alloc}\discretionary{}{}{}\code{_fn_t}}}

\summary
The \code {ompd_callback_memory_alloc_fn_t} type is the type signature of the 
callback routine that the tool provides to the OMPD library to allocate memory.

\format
\begin{cspecific}
\begin{ompSyntax}
typedef ompd_rc_t (*ompd_callback_memory_alloc_fn_t) (
  ompd_size_t \plc{nbytes},
  void **\plc{ptr}
);
\end{ompSyntax}
\end{cspecific}

\descr
The \code {ompd_callback_memory_alloc_fn_t} type is the type signature of the 
memory allocation callback routine that the tool provides. The OMPD library may 
call the \code{ompd_callback_memory_alloc_fn_t} callback function to allocate 
memory.

\argdesc
The \plc{nbytes} argument is the size in bytes of the block of memory to allocate.

The address of the newly allocated block of memory is returned in the location
to which the \plc{ptr} argument points. The newly allocated block is suitably 
aligned for any type of variable, and is not guaranteed to be zeroed.

\crossreferences
\begin{itemize}
\item \code{ompd_size_t}, see \specref{subsubsubsec:ompd_size_t}.

\item \code{ompd_rc_t}, see \specref{subsubsec:ompd_rc_t}.
\end{itemize}



\subsubsection{\hcode{ompd_callback_memory_free_fn_t}}
\label{subsubsubsec:ompd_callback_memory_free_fn_t}
\index{ompd_callback_memory_free_fn_t@{\code{ompd_callback_memory}\discretionary{}{}{}\code{_free}\discretionary{}{}{}\code{_fn_t}}}

\summary
The \code{ompd_callback_memory_free_fn_t} type is the type signature of the 
callback routine that the tool provides to the OMPD library to deallocate memory.

\format
\begin{cspecific}
\begin{ompSyntax}
typedef ompd_rc_t (*ompd_callback_memory_free_fn_t) (
  void *\plc{ptr}
);
\end{ompSyntax}
\end{cspecific}

\descr
The \code {ompd_callback_memory_free_fn_t} type is the type signature of the 
memory deallocation callback routine that the tool provides. The OMPD library may 
call the \code{ompd_callback_memory_free_fn_t} callback function to deallocate 
memory that was obtained from a prior call to the 
\code{ompd_callback_memory_alloc_fn_t} callback function.

\argdesc
The \plc{ptr} argument is the address of the block to be deallocated.

\crossreferences
\begin{itemize}
\item \code{ompd_rc_t}, see \specref{subsubsec:ompd_rc_t}.

\item \code{ompd_callback_memory_alloc_fn_t}, 
see \specref{subsubsubsec:ompd_callback_memory_alloc_fn_t}.

\item \code{ompd_callbacks_t}, see \specref{subsubsec:ompd_callbacks_t}.
\end{itemize}



\subsection{Context Management and Navigation}

\summary
The tool provides the OMPD library with callbacks
to manage and to navigate context relationships.



\subsubsection{\hcode{ompd_callback_get_thread_context_for}\discretionary{}{}{}\hcode{_thread}\discretionary{}{}{}\hcode{_id}\discretionary{}{}{}\hcode{_fn_t}}
\label{subsubsubsec:ompd_callback_get_thread_context_for_thread_id_fn_t}
% the use of \discretionary below overrides the character used for
% hyphenation when an optional linebreak is used.
% if \- were used, we'd get a "-" at the end of the line being
% broken.  We probably don't wantthat, but instead want just a plain
% linebreak without the "-".
\index{ompd_callback_get_thread_context_for_thread_id_fn_t@{{\code{ompd_callback_get_thread}\discretionary{}{}{}\code{_context}\discretionary{}{}{}\code{_for}\discretionary{}{}{}\code{_thread}\discretionary{}{}{}\code{_id}\discretionary{}{}{}\code{_fn_t}}}}

\summary
The \code{ompd_callback_get_thread_context_for_thread_id_fn_t} is the type 
signature of the callback routine that the tool provides to the OMPD library 
to map a thread identifier to a tool thread context.

\format
\begin{cspecific}
\begin{ompSyntax}
typedef ompd_rc_t
(*ompd_callback_get_thread_context_for_thread_id_fn_t) (
  ompd_address_space_context_t *\plc{address_space_context},
  ompd_thread_id_t \plc{kind},
  ompd_size_t \plc{sizeof_thread_id},
  const void *\plc{thread_id},
  ompd_thread_context_t **\plc{thread_context}
);
\end{ompSyntax}
\end{cspecific}

\descr
The \code{ompd_callback_get_thread_context_for_thread_id_fn_t} is the type 
signature of the context mapping callback routine that the tool provides. This 
callback maps a thread identifier to a tool thread context. The thread identifier 
is within the address space that \plc{address_space_context} identifies. The OMPD 
library can use the thread context, for example, to access thread local storage.

\argdesc
The \plc{address_space_context} argument is an opaque handle that the tool 
provides to reference an address space. The \plc{kind}, \plc{sizeof_thread_id}, 
and \plc{thread_id} arguments represent a native thread identifier. On return, 
the \plc{thread_context} argument provides an opaque handle that maps a native
thread identifier to a tool thread context.

\restrictions
Routines that use \code{ompd_callback_get_thread_context_for_thread_id_fn_t} 
have the following restriction:

\begin{itemize}
\item The provided \plc{thread_context} must be valid until the OMPD 
      library returns from the OMPD tool interface routine.
\end{itemize}

\crossreferences
\begin{itemize}
\item \code{ompd_size_t}, see \specref{subsubsubsec:ompd_size_t}.

\item \code{ompd_thread_id_t}, see \specref{ompd:ompd_thread_id_t}.

\item \code{ompd_address_space_context_t}, 
see \specref{subsubsec:ompd_address_space_context_t}.

\item \code{ompd_thread_context_t}, see \specref{subsubsec:ompd_thread_context_t}.

\item \code{ompd_rc_t}, see \specref{subsubsec:ompd_rc_t}.
\end{itemize}



\subsubsection{\hcode{ompd_callback_sizeof_fn_t}}
\label{subsubsubsec:ompd_callback_sizeof_fn_t}
\index{ompd_callback_sizeof_fn_t@{\code{ompd_callback}\discretionary{}{}{}\code{_sizeof}\discretionary{}{}{}\code{_fn_t}}}

\summary
The \code{ompd_callback_sizeof_fn_t} type is the type signature of the callback 
routine that the tool provides to the OMPD library to determine the sizes of the 
primitive types in an address space.

\format
\begin{cspecific}
\begin{ompSyntax}
typedef ompd_rc_t (*ompd_callback_sizeof_fn_t) (
  ompd_address_space_context_t *\plc{address_space_context},
  ompd_device_type_sizes_t *\plc{sizes}
);
\end{ompSyntax}
\end{cspecific}

\descr
The \code{ompd_callback_sizeof_fn_t} is the type signature of the type-size query 
callback routine that the tool provides. This callback provides the sizes of the 
basic primitive types for a given address space.

\argdesc
The callback returns the sizes of the basic primitive types used by the 
address space context that the \plc{address_space_context} argument specifies
in the location to which the \plc{sizes} argument points.

\crossreferences
\begin{itemize}
\item \code{ompd_address_space_context_t}, 
see \specref{subsubsec:ompd_address_space_context_t}.

\item \code{ompd_rc_t}, see \specref{subsubsec:ompd_rc_t}.

\item \code{ompd_device_type_sizes_t}, 
see \specref{subsubsec:ompd_device_type_sizes_t}.

\item \code{ompd_callbacks_t}, see \specref{subsubsec:ompd_callbacks_t}.
\end{itemize}



\subsection{Accessing Memory in the OpenMP Program or Runtime}

The OMPD library may need to read from or to write to the OpenMP program.
It cannot do this directly. Instead the OMPD library must use callbacks 
that the tool provides so that the tool performs the operation.

\subsubsection{\hcode{ompd_callback_symbol_addr_fn_t}}
\label{subsubsubsec:ompd_callback_symbol_addr_fn_t}
\index{ompd_callback_symbol_addr_fn_t@{\code{ompd_callback_symbol}\discretionary{}{}{}\code{_addr}\discretionary{}{}{}\code{_fn_t}}}

\summary
The \code{ompd_callback_symbol_addr_fn_t} type is the type signature of the 
callback that the tool provides to look up the addresses of symbols in an 
OpenMP program.

\format
\begin{cspecific}
\begin{ompSyntax}
typedef ompd_rc_t (*ompd_callback_symbol_addr_fn_t) (
  ompd_address_space_context_t *\plc{address_space_context},
  ompd_thread_context_t *\plc{thread_context},
  const char *\plc{symbol_name},
  ompd_address_t *\plc{symbol_addr},
  const char *\plc{file_name}
);
\end{ompSyntax}
\end{cspecific}

\descr
The \code{ompd_callback_symbol_addr_fn_t} is the type signature of the 
symbol-address query callback routine that the tool provides. This callback 
looks up addresses of symbols within a specified address space.

\argdesc
This callback looks up the symbol provided in the \plc{symbol_name} argument.

The \plc{address_space_context} argument is the tool's representation of the 
address space of the process, core file, or device.

The \plc{thread_context} argument is NULL for global memory access. If  
\plc{thread_context} is not NULL, \plc{thread_context} gives the thread specific 
context for the symbol lookup, for the purpose of calculating thread local 
storage addresses. If \plc{thread_context} is non-null then the thread 
to which \plc{thread_context} refers must be associated with either the process 
or the device that corresponds to the \plc{address_space_context} argument.

The tool uses the \plc{symbol_name} argument that the OMPD library supplies 
verbatim. In particular, no name mangling, demangling or other transformations 
are performed prior to the lookup. The \plc{symbol_name} parameter must correspond
to a statically allocated symbol within the specified address space. The symbol 
can correspond to any type of object, such as a variable, thread local storage 
variable, function, or untyped label. The symbol can have a local, global, or 
weak binding.

The \plc{file_name} argument is an optional input parameter that indicates the 
name of the shared library in which the symbol is defined, and is intended to 
help the third party tool disambiguate symbols that are defined multiple times 
across the executable or shared library files. The shared library name may not 
be an exact match for the name seen by the tool. If \plc{file_name} is NULL then
the tool first tries to find the symbol in the executable file, and, if the 
symbol is not found, the tool tries to find the symbol in the shared libraries 
in the order in which the shared libraries are loaded into the address space. 
If \plc{file_name} is non-null then the tool first tries to find the symbol in 
the libraries that match the name in the \plc{file_name} argument and, if the 
symbol is not found, the tool then uses the same procedure as when 
\plc{file_name} is NULL.

The callback does not support finding symbols that are dynamically allocated on 
the call stack, or statically allocated symbols that are defined within the scope 
of a function or subroutine.

The callback returns the symbol's address in the location to which
\plc{symbol_addr} points.

\restrictions
Routines that use the \code{ompd_callback_symbol_addr_fn_t} type have
the following restrictions:

\begin{itemize}
\item The \plc{address_space_context} argument must be non-null.
\item The symbol that the \plc{symbol_name} argument specifies must be defined. 
\end{itemize}

\crossreferences
\begin{itemize}
\item \code{ompd_address_t}, see \specref{subsubsubsec:ompd_address_t}.

\item \code{ompd_address_space_context_t}, 
see \specref{subsubsec:ompd_address_space_context_t}.

\item \code{ompd_thread_context_t}, see \specref{subsubsec:ompd_thread_context_t}.

\item \code{ompd_rc_t}, see \specref{subsubsec:ompd_rc_t}.

\item \code{ompd_callbacks_t}, see \specref{subsubsec:ompd_callbacks_t}.
\end{itemize}



\subsubsection{\hcode{ompd_callback_memory_read_fn_t}}
\label{subsubsubsec:ompd_callback_memory_read_fn_t}
\index{ompd_callback_memory_read_fn_t@{\code{ompd_callback_memory}\discretionary{}{}{}\code{_read}\discretionary{}{}{}\code{_fn_t}}}

\summary
The \code{ompd_callback_memory_read_fn_t} type is the type signature of the 
callback that the tool provides to read data from an OpenMP program.

\format
\begin{cspecific}
\begin{ompSyntax}
typedef ompd_rc_t (*ompd_callback_memory_read_fn_t) (
  ompd_address_space_context_t *\plc{address_space_context},
  ompd_thread_context_t *\plc{thread_context},
  const ompd_address_t *\plc{addr},
  ompd_size_t \plc{nbytes},
  void *\plc{buffer}
);
\end{ompSyntax}
\end{cspecific}

\descr
The \code{ompd_callback_memory_read_fn_t} is the type signature of the read 
callback routines that the tool provides. 

The \code{read_memory} callback copies a block of data from \plc{addr} within 
the address space to the tool \plc{buffer}. 

The \code{read_string} callback copies a string to which \plc{addr} points, 
including the terminating null byte (\code{'\0'}), to the tool \plc{buffer}. 
At most \plc{nbytes} bytes are copied. If a null byte is not among the first 
\plc{nbytes} bytes, the string placed in \plc{buffer} is not null-terminated.

\argdesc
The address from which the data are to be read from the OpenMP program
specified by \plc{address_space_context} is given by \plc{addr}. while
\plc{nbytes} gives the number of bytes to be transferred. The \plc{thread_context} 
argument is optional for global memory access, and in this case should be NULL.
If it is non-null, \plc{thread_context} identifies the thread specific context 
for the memory access for the purpose of accessing thread local storage.

The data are returned through \plc{buffer}, which is allocated and owned by 
the OMPD library. The contents of the buffer are unstructured, raw bytes. The
OMPD library must arrange for any transformations such as byte-swapping that 
may be necessary (see~\specref{subsubsubsec:ompd_callback_device_host_fn_t}) 
to interpret the data.

\crossreferences
\begin{itemize}
\item \code{ompd_size_t}, see \specref{subsubsubsec:ompd_size_t}.

\item \code{ompd_address_t}, see \specref{subsubsubsec:ompd_address_t}.

\item \code{ompd_address_space_context_t}, 
see \specref{subsubsec:ompd_address_space_context_t}.

\item \code{ompd_thread_context_t}, see \specref{subsubsec:ompd_thread_context_t}.

\item \code{ompd_rc_t}, see \specref{subsubsec:ompd_rc_t}.

\item \code{ompd_callback_device_host_fn_t}, 
see \specref{subsubsubsec:ompd_callback_device_host_fn_t}.

\item \code{ompd_callbacks_t}, see \specref{subsubsec:ompd_callbacks_t}.
\end{itemize}



\subsubsection{\hcode{ompd_callback_memory_write_fn_t}}
\label{subsubsubsec:ompd_callback_memory_write_fn_t}
\index{ompd_callback_memory_write_fn_t@{\code{ompd_callback_memory}\discretionary{}{}{}\code{_write}\discretionary{}{}{}\code{_fn_t}}}

\summary
The \code{ompd_callback_memory_write_fn_t} type is the type signature of the 
callback that the tool provides to write data to an OpenMP program.

\format
\begin{cspecific}
\begin{ompSyntax}
typedef ompd_rc_t (*ompd_callback_memory_write_fn_t) (
  ompd_address_space_context_t *\plc{address_space_context},
  ompd_thread_context_t *\plc{thread_context},
  const ompd_address_t *\plc{addr},
  ompd_size_t \plc{nbytes},
  const void *\plc{buffer}
);
\end{ompSyntax}
\end{cspecific}

\descr
The \code{ompd_callback_memory_write_fn_t} is the type signature of the write
callback routine that the tool provides. The OMPD library may call this callback 
to have the tool write a block of data to a location within an address space 
from a provided buffer.

\argdesc
The address to which the data are to be written in the OpenMP program
that \plc{address_space_context} specifies is given by \plc{addr}. The
\plc{nbytes} argument is the number of bytes to be transferred. The 
\plc{thread_context} argument is optional for global memory access,
and, in this case, should be NULL. If it is non-null then \plc{thread_context} 
identifies the thread-specific context for the memory access for the purpose 
of accessing thread local storage.

The data to be written are passed through \plc{buffer}, which is allocated and
owned by the OMPD library. The contents of the buffer are unstructured, raw bytes.
The OMPD library must arrange for any transformations such as byte-swapping that 
may be necessary (see~\specref{subsubsubsec:ompd_callback_device_host_fn_t})
to render the data into a form that is compatible with the OpenMP runtime.

\crossreferences
\begin{itemize}
\item \code{ompd_size_t}, see \specref{subsubsubsec:ompd_size_t}.

\item \code{ompd_address_t}, see \specref{subsubsubsec:ompd_address_t}.

\item \code{ompd_address_space_context_t},
see \specref{subsubsec:ompd_address_space_context_t}.

\item \code{ompd_thread_context_t}, see \specref{subsubsec:ompd_thread_context_t}.

\item \code{ompd_rc_t}, see \specref{subsubsec:ompd_rc_t}.

\item \code{ompd_callback_device_host_fn_t}, 
see \specref{subsubsubsec:ompd_callback_device_host_fn_t}.

\item \code{ompd_callbacks_t}, see \specref{subsubsec:ompd_callbacks_t}.
\end{itemize}



\subsection{Data Format Conversion: \hcode{ompd_callback_device_host_fn_t}}
\label{subsubsec:data-format-conversion}
\label{subsubsubsec:ompd_callback_device_host_fn_t}
\index{ompd_callback_device_host_fn_t@{\code{ompd_callback_device}\discretionary{}{}{}\code{_host}\discretionary{}{}{}\code{_fn_t}}}

\summary
The \code{ompd_callback_device_host_fn_t} type is the type signature of the 
callback that the tool provides to convert data between the formats that the
tool and the OMPD library use and that the OpenMP program uses.

\format
\begin{cspecific}
\begin{ompSyntax}
typedef ompd_rc_t (*ompd_callback_device_host_fn_t) (
  ompd_address_space_context_t *\plc{address_space_context},
  const void *\plc{input},
  ompd_size_t \plc{unit_size},
  ompd_size_t \plc{count},
  void *\plc{output}
);
\end{ompSyntax}
\end{cspecific}

\descr
The architecture and/or programming-model of the tool and the OMPD library 
may be different from that of the OpenMP program that is being examined. Thus,
the conventions for representing data may differ. The callback interface includes 
operations to convert between the conventions, such as the byte order (endianness),
that the tool and OMPD library use and the one that the OpenMP program uses.
The callback with the \code{ompd_callback_device_host_fn_t} type signature 
convert data between formats

\argdesc
The \plc{address_space_context} argument specifies the OpenMP address space 
that is associated with the data. The \plc{input} argument is the source buffer
and the \plc{output} argument is the destination buffer. The \plc{unit_size} 
argument is the size of each of the elements to be converted. The \plc{count} 
argument is the number of elements to be transformed.


The OMPD library allocates and owns the input and output buffers. It must
ensure that the buffers have the correct size, and are eventually deallocated 
when they are no longer needed.

\crossreferences
\begin{itemize}
\item \code{ompd_size_t}, see \specref{subsubsubsec:ompd_size_t}.

\item \code{ompd_address_space_context_t}, 
see \specref{subsubsec:ompd_address_space_context_t}.

\item \code{ompd_rc_t}, see \specref{subsubsec:ompd_rc_t}.

\item \code{ompd_callbacks_t}, see \specref{subsubsec:ompd_callbacks_t}.
\end{itemize}



\subsection{Output: \hcode{ompd_callback_print_string_fn_t}}
\label{subsubsec:output}
\label{subsubsubsec:ompd_callback_print_string_fn_t}
\index{ompd_callback_print_string_fn_t@{\code{ompd_callback_print}\discretionary{}{}{}\code{_string}\discretionary{}{}{}\code{_fn_t}}}

\summary
The \code{ompd_callback_print_string_fn_t} type is the type signature of the 
callback that tool provides so that the OMPD library can emit output.

\format
\begin{cspecific}
\begin{ompSyntax}
typedef ompd_rc_t (*ompd_callback_print_string_fn_t) (
  const char *\plc{string},
  int \plc{category}
);
\end{ompSyntax}
\end{cspecific}

\descr
The OMPD library may call the \code{ompd_callback_print_string_fn_t} callback 
function to emit output, such as logging or debug information. The tool may
set the \hcode{ompd_callback_print_string_fn_t} callback function to NULL to
prevent the OMPD library from emitting output; the OMPD may not write to file 
descriptors that it did not open.

\argdesc
The \plc{string} argument is the null-terminated string to be printed.
No conversion or formatting is performed on the string.

The \plc{category} argument is the implementation-defined category of 
the string to be printed. 

\crossreferences
\begin{itemize}
\item \code{ompd_rc_t}, see \specref{subsubsec:ompd_rc_t}.

\item \code{ompd_callbacks_t}, see \specref{subsubsec:ompd_callbacks_t}.
\end{itemize}



\subsection{The Callback Interface}
\label{subsubsec:ompd_callbacks_t}
\index{ompd_callbacks_t@{\code{ompd_callbacks_t}}}

\summary
All OMPD library interactions with the OpenMP program must be through a set of 
callbacks that the tool provides. These callbacks must also be used for allocating 
or releasing resources, such as memory, that the library needs.

\format
\begin{cspecific}
{\small
\begin{ompSyntax}
typedef struct {
  ompd_callback_memory_alloc_fn_t \plc{alloc_memory};
  ompd_callback_memory_free_fn_t \plc{free_memory};
  ompd_callback_print_string_fn_t \plc{print_string};
  ompd_callback_sizeof_fn_t \plc{sizeof_type};
  ompd_callback_symbol_addr_fn_t \plc{symbol_addr_lookup};
  ompd_callback_memory_read_fn_t \plc{read_memory};
  ompd_callback_memory_write_fn_t \plc{write_memory};
  ompd_callback_memory_read_fn_t \plc{read_string};
  ompd_callback_device_host_fn_t \plc{device_to_host};
  ompd_callback_device_host_fn_t \plc{host_to_device};
  ompd_callback_get_thread_context_for_thread_id_fn_t
    \plc{get_thread_context_for_thread_id};
} ompd_callbacks_t;
\end{ompSyntax}
}
\end{cspecific}

\descr
The set of callbacks that the OMPD library must use is collected in the 
\code{ompd_callbacks_t} record structure. An instance of this type is passed 
to the OMPD library as a parameter to \code{ompd_initialize} 
(see~\specref{subsubsubsec:ompd_initialize}). Each field points to a function 
that the OMPD library must use to interact with the OpenMP program or for 
memory operations.

The \plc{alloc_memory} and \plc{free_memory} fields are pointers to functions 
the OMPD library uses to allocate and to release dynamic memory.

\plc{print_string} points to a function that prints a string.

The architectures or programming models of the OMPD library and third party 
tool may be different from that of the OpenMP program that is being examined.
\plc{sizeof_type} points to function that allows the OMPD library to determine 
the sizes of the basic integer and pointer types that the OpenMP program uses.
Because of the differences in architecture or programming model, the conventions 
for representing data in the OMPD library and the OpenMP program may be different.
The \plc{device_to_host} field points to a function that translates data from the 
conventions that the OpenMP program uses to those that the tool and OMPD library
use. The reverse operation is performed by the function to which the 
\plc{host_to_device} field points.

The \plc{symbol_addr_lookup} field points to a callback that the OMPD library 
can use to find the address of a global or thread local storage symbol. The 
\plc{read_memory}, \plc{read_string}, and \plc{write_memory} fields are 
pointers to functions for reading from and writing to global memory or 
thread local storage in the OpenMP program.

The \plc{get_thread_context_for_thread_id} field is a pointer to a function
that the OMPD library can use to obtain a thread context that corresponds to
a native thread identifier.

\crossreferences
\begin{itemize}
\item \code{ompd_callback_memory_alloc_fn_t}, 
see \specref{subsubsubsec:ompd_callback_memory_alloc_fn_t}.

\item \code{ompd_callback_memory_free_fn_t}, 
see \specref{subsubsubsec:ompd_callback_memory_free_fn_t}.

\item \code{ompd_callback_get_thread_context_for_thread_id_fn_t}, 
see \specref{subsubsubsec:ompd_callback_get_thread_context_for_thread_id_fn_t}.

\item \code{ompd_callback_sizeof_fn_t}, 
see \specref{subsubsubsec:ompd_callback_sizeof_fn_t}.

\item \code{ompd_callback_symbol_addr_fn_t}, 
see \specref{subsubsubsec:ompd_callback_symbol_addr_fn_t}.

\item \code{ompd_callback_memory_read_fn_t}, 
see \specref{subsubsubsec:ompd_callback_memory_read_fn_t}.

\item \code{ompd_callback_memory_write_fn_t}, 
see \specref{subsubsubsec:ompd_callback_memory_write_fn_t}.

\item \code{ompd_callback_device_host_fn_t}, 
see \specref{subsubsubsec:ompd_callback_device_host_fn_t}.

\item \code{ompd_callback_print_string_fn_t}, 
see \specref{subsubsubsec:ompd_callback_print_string_fn_t}
\end{itemize}
