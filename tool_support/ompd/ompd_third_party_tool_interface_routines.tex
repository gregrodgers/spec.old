
\subsection{Third Party Tool Interface Routines}
\label{sec:ompd-api}

\ompdsubsection{Per OMPD Library Initialization}

The OMPD library must be initialized exactly once after it is loaded, and finalized exactly once
before it is unloaded. Per target process or core file initialization and finalization are also required. 
A third-party tool starts the initialization by calling \code{ompd\_initialize}. Once loaded, the 
debugger can determine the version of the OMPD API supported by the DLL by 
calling \code{ompd\_rc\_t ompd\_get\_version}:

\format
\vbox{
\ccppspecificstart
\begin{boxedcode}
ompd_rc_t ompd_get_version(int *\plc{version});
\end{boxedcode}
\ccppspecificend
}

On success this should return \code{ompd\_rc\_ok}; \code{ompd\_rc\_bad\_input} indicates that the 
argument is invalid. Other errors could be reported by \code{ompd\_rc\_error}. A descriptive string 
describing the OMPD implementation is returned by the function \code{ompd\_get\_version\_string}:

\format
\vbox{
\ccppspecificstart
\begin{boxedcode}
ompd_rc_t ompd_get_version_string(const char **\plc{string});
\end{boxedcode}
\ccppspecificend
}

The return values are the same as \code{ompd\_get\_version\_string}. The string returned by the 
OMPD library is `owned' by the library, and it must not be modified or released by the tool.
It is guaranteed to remain valid for as long as the library is loaded. 
\code{ompd\_get\_version\_string} 
may be called before \code{ompd\_initialize}. Accordingly, the OMPD library must not use heap or 
stack memory for the string it returns to the tool.

The signatures of \code{ompd\_get\_version} and \code{ompd\_get\_version\_string} are 
guaranteed not to change in future versions of the API. In contrast, the type definitions and 
prototypes in the rest of the API do not carry the same guarantee. Therefore a third-party tool 
should check the version of the API of a loaded OMPD library before calling any other function of 
the API.

The third-party tool must provide the OMPD library with a set of callback functions 
that enable OMPD to allocate and deallocate memory in the tool's address space, to lookup the 
sizes of basic primitive types in the target, to lookup symbols in the target, as well as to read and 
write memory in the target. These callback functions are provided to the OMPD library via a 
table---a list of function pointers---of type \code{ompd\_callbacks\_t}.

\format
\vbox{
\ccppspecificstart
\begin{boxedcode}
ompd_rc_t ompd_initialize(const ompd_callbacks_t *\plc{callbacks});
\end{boxedcode}
\ccppspecificend
}

%The type \refdef{\texttt{ompd\_callbacks\_t}}{callbacks-t:def} is defined in
%Section~\ref{sec:ompd_data_types}. The argument is guaranteed to be valid for the duration of 
%the call.
%The OMPD library cannot assume that \texttt{callbacks} will remain valid
%after the call returns back to the debugger.
%\begin{notes}
%ilaguna: We need to be more specific here. What does the previous sentence 
%mean?	
%\end{notes}

On success, \code{ompd\_initialize} returns \code{ompd\_rc\_ok}. If the \code{data} argument is 
invalid, \code{ompd\_rc\_bad\_input} should be returned. All other errors will be reported by 
\code{ompd\_rc\_error}.

The above initialization is performed for each OMPD library that is loaded by a third-party tool. 
There may more than one library present in a thid-party tool, such as a debugger, because the tool 
may be controlling a number of targets that may be using different runtime systems which require 
different OMPD libraries. This initialization must be performed exactly once before the tool
can begin operating on a target process or core file.

\ompdsubsection{Per Target Initialization}

A third-party tool initializes a session working on a target process or core
file by calling \code{ompd\_process\_initialize}:

\format
\vbox{
\ccppspecificstart
\begin{boxedcode}
ompd_rc_t ompd_process_initialize(
  ompd_address_space_context_t  *\plc{context},   /* IN */
  ompd_address_space_handle_t  **\plc{handle}   /* OUT */
);
\end{boxedcode}
	\ccppspecificend
}

The \code{context} argument is the pointer to the third-party tool's host
address space context object for the target process or core file. The OMPD implementation 
returns a pointer to the address space handle in \code{*handle}, which the tool is responsible for 
releasing when it is no longer needed. This function must be called before any OMPD operations 
are performed on the target.
\code{ompd\_process\_initialize} gives the OMPD library an opportunity to confirm that it is capable 
of handling the target process or core file identified by the context. Incompatibility is signaled by a 
return value of \code{ompd\_rc\_incompatible}. On return, the handle is owned by the tool, 
which must release it using \code{ompd\_release\_address\_space\_handle}.

\ompdsubsection{Per Target Finalization}

When the tool is finished working on the target address space for a process or core file, it calls
\code{ompd\_release\_address\_space\_handle}
to tell the OMPD implementation that it not longer needs the address space,
and to give the OMPD implementation an opportunity to release any
resources it may have related to the handle.

\ompdsubsection{Per OMPD Library Finalization}

When the tool is finished with the OMPD library it should call \code{ompd\_finalize} before 
unloading the library.

\format
\vbox{
\ccppspecificstart
\begin{boxedcode}
ompd_rc_t ompd_finalize(void);
\end{boxedcode}
\ccppspecificend
}

This must be the last call the tool makes to the library before
unloading it. The call to \code{ompd\_finalize} gives the OMPD library a chance to
free up any remaining resources it may be holding.

The OMPD library may implement a \emph{finalizer} section. This will execute as the library is 
unloaded, and therefore after the tool's call to \code{ompd\_finalize}. The OMPD library is allowed 
to use the callbacks (provided to it earlier by the third-party tool after the call to 
\code{ompd\_initialize}) during finalization.

\ompdsubsection{Memory Management}

The OMPD library must not access the heap manager directly. Instead, if it needs heap memory it 
must use the memory allocation and deallocation callback functions provided by the third-party tool 
to obtain and release heap memory. This will ensure that the library does not interfere with any 
custom memory management scheme that the tool may use.

If the OMPD library is implemented in \code{C++}, memory management operators like \code{new} 
and \code{delete} in all their variants, \emph{must all} be overloaded and implemented in terms of 
the callbacks provided by the third-party tool. The OMPD library must be coded so that any of its 
definitions of \code{new} or \code{delete} do not interfere with any defined by the tool.

In some cases, the OMPD library will need to allocate memory to return results to the third-party 
tool. This memory will then be `owned' by the tool, which will be responsible for releasing it. It is 
therefore vital that the OMPD library and the tool use the same memory manager.

OMPD handles are created by the OMPD implementation. These are opaque to the 
third-party tool, and depending on the specific implementation of OMPD may have complex 
internal structure. The tool cannot know whether the handle pointers returned by the API 
correspond to discrete heap allocations. Consequently, the tool must not simply deallocate a 
handle by passing an address it receives from the OMPD library to its own memory manager. 
Instead, the API includes functions that the tool must use when it no longer needs a handle.

Contexts are created by a third-party tool and passed to the OMPD implementation. The OMPD 
library does not need to release contexts; instead this will be done by the tool after it releases any 
handles that may be referencing the contexts.

\ompdsubsection{Thread and Signal Safety}

The OMPD implementation does not need to be reentrant. It is the responsibility 
of the third-party tool to ensure that only one thread enters the OMPD library at a time. The OMPD 
implementation must not install signal handlers or otherwise interfere with the tool's signal 
configuration.

\ompdsubsection{ Initialization}

\ompdcallbacksignature{ompd\_process\_initialize}
\summary
A tool obtains an address space handle when it initializes
a session on a live process or core file by calling \code{ompd\_process\_initialize}.

\format
\vbox{
\ccppspecificstart
\begin{boxedcode}
ompd_rc_t ompd_initialize(const ompd_callbacks_t *\plc{callbacks});
\end{boxedcode}
\ccppspecificend
}

\descr
On return from \code{ompd\_process\_initialize} the address
space handle is owned by the tool.

\argdesc
\crossreferences

\ompdcallbacksignature{ompd\_release\_address\_space\_handle}
\summary
A third-party tool calls \code{ompd\_release\_address\_space\_handle} to release an address space 
handle.

\format
\vbox{
\ccppspecificstart
\begin{boxedcode}
ompd_rc_t ompd_release_address_space_handle(
  ompd_address_space_handle_t *\plc{handle}   /* IN */
);
\end{boxedcode}
\ccppspecificend
}

\descr
When the debugger is finished with the target address space handle it
should call \code{ompd\_release\_address\_space\_handle} to release
the handle and give the OMPD implementation the opportunity to release
any resources it may have related to the target.

\argdesc

\crossreferences


\ompdsubsection{Thread Handles}

\ompdcallbacksignature{ompd\_get\_threads}
\summary
The \code{ompd\_get\_threads} enables a tool to
to obtain pointers to handles for all OpenMP threads associated
with an address space handle.

\format
\vbox{
\ccppspecificstart
\begin{boxedcode}
ompd_rc_t ompd_get_threads(
  ompd_address_space_handle_t *\plc{handle},   /* IN */
  ompd_thread_handle_t ***\plc{thread_handle_vector},   /* OUT */
  int *\plc{num_handles}   /* OUT */
);
\end{boxedcode}
\ccppspecificend
}

\descr
A successful invocation of  \code{|ompd\_get\_threads} returns
a pointer to a vector of pointers to handles in \code{*thread\_handle\_vector}
and returns the number of handle pointers in \code{*num\_handles}.
This call yields meaningful results only if all OpenMP threads in
the target process are stopped; otherwise, the OpenMP runtime may be creating and/or destroying
threads during or after the call, rendering useless the
vector of handles returned.

The \code{num\_handles} pointer argument must be valid.
The \code{thread\_handle\_vector} pointer argument may be NULL,
in which case the number of handles that would have been returned
had the argument not been NULL is returned in \code{*num\_handles}.
This allows the debugger to find out how many OpenMP threads are
running in the address space when it is not interested in the handles 
themselves.

The OMPD library gets the memory required for the vector of pointers to thread 
handles using the memory allocation routine in the callbacks it received during
the call to \code{ompd\_initialize}. If the OMPD implementation needs to allocate heap memory for 
the thread handles, it must use the callbacks to acquire this memory. On return, the vector and the 
thread handles are `owned' by the debugger, and the debugger is responsible for releasing them 
when they are no longer required.

The thread handles must be released by calling 
\code{ompd\_release\_thread\_handle}. The vector was allocated by the OMPD implementation 
using the allocation routine in the callbacks it received during initialization 
%(see \refdef{\code{ompt\_initialize}}{initialize:def}); 
the debugger must deallocate the vector in a 
compatible manner.

\argdesc

\crossreferences


\ompdcallbacksignature{ompd\_get\_thread\_in\_parallel}
\summary
The  \code{ompd\_get\_thread\_in\_parallel} operation enables a tools to obtain handles for all 
OpenMP threads associated with a parallel region.

\format
\vbox{
\ccppspecificstart
\begin{boxedcode}
ompd_rc_t ompd_get_thread_in_parallel(
  ompd_parallel_handle_t *\plc{parallel_handle},   /* IN */
  ompd_thread_handle_t ***\plc{thread_handle_vector},   /* OUT */
  int *\plc{num_handles}   /* OUT */
);
\end{boxedcode}
\ccppspecificend
}

\descr
A successful invocation of  \code{ompd\_get\_thread\_in\_parallel} returns a pointer to a vector of 
pointers to thread handles in \code{*thread\_handle\_vector}, and returns the number of handles in 
\code{*num\_handles}. This call yields meaningful results only if all OpenMP threads in the parallel 
region are stopped; otherwise, the OpenMP runtime may be creating and/or destroying threads 
during or after the call, rendering useless the vector of handles returned.

The \code{num\_handles} pointer argument must be valid. The \code{thread\_handle\_vector} 
pointer argument may be NULL, in which case the number of handles that would have been 
returned had the argument not been NULL is returned in \code{*num\_handles}.

The OMPD library must obtain the memory for the vector of pointers to thread handles
using the memory allocation callback function that was passed to it during \code{ompd\_initialize}.
If the OMPD implementation needs to allocate heap memory for the thread handles it must use the 
callbacks to acquire this memory. After the call the vector and the thread handles are `owned' by 
the third-party tool, which is responsible for releasing them. The thread handles must be released 
by calling \code{ompd\_thread\_handle}. The vector was allocated by the OMPD implementation 
using the allocation routine in the callbacks; the debugger must deallocate  the vector in a 
compatible manner.

\argdesc

\crossreferences


\ompdcallbacksignature{ ompd\_get\_master\_thread\_in\_parallel }
\summary
The  \code{ompd\_get\_master\_thread\_in\_parallel} operation
enables a third-party tool to obtain a handle for the OpenMP master thread in a parallel region. 

\format
\vbox{
\ccppspecificstart
\begin{boxedcode}
ompd_rc_t ompd_get_master_thread_in_parallel(
  ompd_parallel_handle_t *\plc{parallel_handle},   /* IN */
  ompd_thread_handle_t **\plc{thread_handle}   /* OUT */
); 
\end{boxedcode}
\ccppspecificend
}

\descr
A successful invocation of \code{ompd\_get\_master\_thread\_in\_parallel} returns a handle
for the thread that encountered the parallel construct. This call yields meaningful results only if an 
OpenMP thread in the parallel region is stopped; otherwise, the parallel region is not guaranteed to 
be alive. 

On success \code{ompd\_get\_master\_thread\_in\_parallel} returns \code{ompd\_rc\_ok}.
A pointer to the thread handle is returned in \code{*thread\_handle}. After the call the thread 
handle is owned by the tool, which must release it when it is no longer required by calling
\code{ompd\_release\_thread\_handle}.

\argdesc

\crossreferences


\ompdcallbacksignature{ ompd\_release\_thread\_handle  }
\summary
This operation releases a thread handle.

\format
\vbox{
\ccppspecificstart
\begin{boxedcode}
ompd_rc_t ompd_release_thread_handle(
  ompd_thread_handle_t *\plc{thread_handle}   /* IN */
);
\end{boxedcode}
\ccppspecificend
}

\descr
Thread handles are opaque to third-party tools, which therefore cannot release them directly.
Instead, when the tool is finished with a thread handle it must pass it to the OMPD 
\code{ompd\_release\_thread\_handle} routine for disposal.

\argdesc

\crossreferences


\ompdcallbacksignature{ ompd\_thread\_handle\_compare  }
\summary
The \code{ompd\_thread\_handle\_compare} operation allows third-party tools to compare 
two thread handles.

\format
\vbox{
\ccppspecificstart
\begin{boxedcode}
ompd_rc_t ompd_thread_handle_compare(
  ompd_thread_handle_t *\plc{thread_handle_1},   /* IN */
  ompd_thread_handle_t *\plc{thread_handle_2},   /* IN */
  int *\plc{cmp_value}   /* OUT */
);
\end{boxedcode}
\ccppspecificend
}

\descr
The internal structure of thread handles is opaque to a tool. While the tool can easily compare 
pointers to thread handles, it cannot determine whether handles of two different addresses
refer to the same underlying thread. This function can be used to compare thread handles.

On success, \code{ompd\_thread\_handle\_compare} returns in \code{*cmp\_value} a signed 
integer value that indicates how the underlying threads compare: a value less than, equal to, or 
greater than 0 indicates that the thread corresponding to \code{thread\_handle\_1} is, respectively, 
less than, equal to, or greater than that corresponding to \code{thread\_handle\_2}.

%\begin{notes}
%	ilaguna: do we need to give intuition about what we mean by thread1 < thread2 
%	(or vice versa)? Will the OMPD DLL maintain a total order or a partial order of 
%	thread handles? If thread1 < thread2, and thread2 < thread3, is thread1 < 
%	thread3 or can thread1 > thread3?
%\end{notes}

For OMPD implementations that always have a single, unique, underlying thread handle for a given 
thread, this operation reduces to a simple comparison of the pointers. However, other 
implementations may take a different approach, and therefore the only reliable way of determining 
whether two different pointers to thread handles refer the same or distinct threads is to use
\code{ompd\_thread\_handle\_compare}.

Allowing thread handles to be compared allows the tool to hold them in ordered collections.
The means by which thread handles are ordered is implementation-defined.

\argdesc

\crossreferences

\ompdcallbacksignature{  ompd\_get\_thread\_handle\_string\_id }
\summary
The \code{ompd\_get\_thread\_handle\_string\_id} function returns a string
that contains a unique printable value that identifies the thread.

\format
\vbox{
\ccppspecificstart
\begin{boxedcode}
ompd_rc_t ompd_get_thread_handle_string_id(
  ompd_thread_handle_t *\plc{thread_handle},   /* IN */
  char **\plc{string_id}   /* OUT */
);
\end{boxedcode}
\ccppspecificend
}

\descr
The string that is returned by the function should be a single sequence of alphanumeric or 
underscore characters, and NULL terminated.
%\begin{notes}
%	ilaguna: Why allowing only alphanumeric or underscore characters? As an 
%	implementer I may want to use colon or slash characters for more structured 
%	names.
%\end{notes}

The OMPD implementation allocates the string returned in \code{*string\_id} using the allocation 
routine in the callbacks passed to it during initialization. On return the string is owned by the 
debugger, which is responsible for deallocating it. The contents of the strings returned for thread 
handles which compare as equal with \code{ompd\_thread\_handle\_compare} must be the same.

\argdesc

\crossreferences

\ompdsubsection{Parallel Region Handles}

\ompdcallbacksignature{  ompd\_get\_top\_parallel\_region }
\summary
The  \code{ompd\_get\_top\_parallel\_region} operation enables the debugger to obtain a pointer to 
the parallel handle for the innermost, or topmost, parallel region associated with an OpenMP thread.

\format
\vbox{
\ccppspecificstart
\begin{boxedcode}
ompd_rc_t ompd_get_top_parallel_region(
  ompd_thread_handle_t *\plc{thread_handle},   /* IN */
  ompd_parallel_handle_t **\plc{parallel_handle}   /* OUT */
);
\end{boxedcode}
\ccppspecificend
}

\descr
This call is meaningful only if the thread whose handle is provided is stopped. The parallel handle 
must be released by calling \code{ompd\_release\_parallel\_handle}.

\argdesc

\crossreferences

\ompdcallbacksignature{ ompd\_get\_enclosing\_parallel\_handle  }
\summary
The  \code{ompd\_get\_enclosing\_parallel\_handle}  operation enables a third-party tool to obtain a 
pointer to the parallel handle for the parallel region enclosing the parallel region specified by 
\code{parallel\_handle}.

\format
\vbox{
\ccppspecificstart
\begin{boxedcode}
ompd_rc_t ompd_get_enclosing_parallel_handle(
  ompd_parallel_handle_t *\plc{parallel_handle},   /* IN */
  ompd_parallel_handle_t **\plc{enclosing_parallel_handle}   /* OUT */
);
\end{boxedcode}
\ccppspecificend
}

\descr
This call is meaningful only if at least one thread in the parallel region is stopped. On success 
\code{ompd\_get\_enclosing\_parallel\_handle} returns \code{ompd\_rc\_ok}.
A pointer to the parallel handle for the enclosing region is returned in 
\code{*enclosing\_parallel\_handle}. After the call the handle is owned by the tool, which must 
release it when it is no longer required by calling \code{ompd\_release\_parallel\_handle}.

\argdesc

\crossreferences

\ompdcallbacksignature{ ompd\_get\_task\_parallel\_handle  }
\summary
The  \code{ompd\_get\_task\_parallel\_handle}  operation enables a third-party tool to obtain a 
pointer to the parallel handle for the parallel region enclosing the task region specified by 
\code{task\_handle}.

\format
\vbox{
\ccppspecificstart
\begin{boxedcode}
ompd_rc_t ompd_get_task_parallel_handle(
  ompd_task_handle_t *\plc{task_handle},   /* IN */
  ompd_parallel_handle_t **\plc{task_parallel_handle}   /* OUT */
);
\end{boxedcode}
\ccppspecificend
}

\descr
This call is meaningful only if at least one thread in the parallel region is stopped. On success 
\code{ompd\_get\_task\_parallel\_handle} returns \code{ompd\_rc\_ok}. A pointer to the parallel 
regions handle is returned in \code{*task\_parallel\_handle}. The parallel handle is owned by the 
tool, which must release it by calling \code{ompd\_release\_parallel\_handle}.

\argdesc

\crossreferences

\ompdcallbacksignature{ ompd\_release\_parallel\_handle  }
\summary
This operation allows releasing a parallel region handle.

\format
\vbox{
\ccppspecificstart
\begin{boxedcode}
ompd_rc_t ompd_release_parallel_handle(
  ompd_parallel_handle_t *\plc{parallel_handle}   /* IN */
);
\end{boxedcode}
\ccppspecificend
}

\descr
Parallel region handles are opaque to the third-party tool, which therefore
cannot release them directly. Instead, when the tool is finished with a parallel region handle
it must must pass it to the OMPD \code{ompd\_release\_parallel\_handle}
routine for disposal.

\argdesc

\crossreferences

\ompdcallbacksignature{ ompd\_parallel\_handle\_compare  }
\summary
The \code{ompd\_parallel\_handle\_compare} operation allows a third-party tool to compare two 
parallel region handles.

\format
\vbox{
\ccppspecificstart
\begin{boxedcode}
ompd_rc_t ompd_parallel_handle_compare(
  ompd_parallel_handle_t *\plc{parallel_handle_1},   /* IN */
  ompd_parallel_handle_t *\plc{parallel_handle_2},   /* IN */
  int *\plc{cmp_value}   /* OUT */
);
\end{boxedcode}
\ccppspecificend
}

\descr
The internal structure of parallel region handles is opaque to the tool. While the tool can easily 
compare pointers to parallel region handles, it cannot determine whether handles at two different 
addresses refer to the same underlying parallel region.

On success, \code{ompd\_parallel\_handle\_compare} returns in \code{*cmp\_value} a signed 
integer value that indicates how the underlying parallel regions compare: a value less than, equal 
to, or greater than 0 indicates that the region corresponding to \code{parallel\_handle\_1} is,
respectively, less than, equal to, or greater than that corresponding to \code{parallel\_handle\_2}.

For OMPD implementations that always have a single, unique, underlying parallel region handle for 
a given parallel region, this operation reduces to a simple comparison of the pointers. However, 
other implementations may take a different approach, and therefore the only reliable way of 
determining whether two different pointers to parallel regions handles refer the same or distinct
parallel regions is to use \code{ompd\_parallel\_handle\_compare}.

Allowing parallel region handles to be compared allows the debugger to hold them in ordered 
collections. The means by which parallel region handles are ordered is implementation-defined.

\argdesc

\crossreferences

\ompdcallbacksignature{  ompd\_get\_parallel\_handle\_string\_id }
\summary
The \code{ompd\_get\_parallel\_handle\_string\_id} function returns a string that contains a unique 
printable value that identifies the parallel region.

\format
\vbox{
\ccppspecificstart
\begin{boxedcode}
ompd_rc_t ompd_get_parallel_handle_string_id(
  ompd_parallel_handle_t *\plc{parallel_handle},   /* IN */
  char **\plc{string_id}   /* OUT */
);
\end{boxedcode}
\ccppspecificend
}

\descr
The returned string should be a single sequence of alphanumeric or underscore characters, and 
NULL terminated. The OMPD implementation allocates the string returned in \code{*string\_id}
using the allocation routine in the callbacks passed to it during initialization. On return the string is 
owned by the tool, which is responsible for deallocating it.

The contents of the strings returned for parallel regions handles which compare as equal with
\code{ompd\_parallel\_handle\_compare} must be the same.

\argdesc

\crossreferences

\ompdsubsection{Task Handles}

\ompdcallbacksignature{  ompd\_get\_top\_task\_region }
\summary
A third-party tool uses the \code{ompd\_get\_top\_task\_region} operation to obtain a pointer to the 
task handle for the innermost, or topmost, task region associated with an OpenMP thread.

\format
\vbox{
\ccppspecificstart
\begin{boxedcode}
ompd_rc_t ompd_get_top_task_region(
  ompd_thread_handle_t *\plc{thread_handle},   /* IN */
  ompd_task_handle_t **\plc{task_handle}   /* OUT */
);
\end{boxedcode}
\ccppspecificend
}

\descr
This call is meaningful only if the thread whose handle is provided is stopped.
The task handle must be released by calling \code{ompd\_release\_task\_handle}.

\argdesc

\crossreferences

% ilaguna: I'm not sure where to place this text.
\littleheader{Retrieve the handle for an enclosing task.}
The OMPD API includes operations to obtain the handle of the parent of a task represented by a 
given task handle. There are two notions of parenthood. The \emph{scheduling} parent task is the 
task that was active when the child task was scheduled to run. The \emph{generating} is the task 
that encountered the OpenMP that caused the child task to be created.

The generating and scheduling parents need not be the same. This might happen if the thread 
executing a task encounters an OpenMP construct. When this happens, the thread will enter the 
runtime. The runtime will set up the tasks to implement the OpenMP program construct, and then 
call its schedular to choose a task to execute. If the scheduler chooses a task other than one of 
these newly created tasks to run, the scheduling parent of the selected task will not be the same 
as its generating parent. The former is the task that the thread was executing most recently,
and from which it entered the runtime. The later is the task which encountered the OpenMP 
construct it is executing.

\ompdcallbacksignature{  ompd\_get\_generating\_ancestor\_task\_region }
\summary
A third-party tool uses \code{ompd\_get\_generating\_ancestor\_task\_region} to obtain
a pointer to the task handle for the task that encountered the OpenMP construct which caused the 
task represented by \code{task\_handle} to be created.

\format
\vbox{
\ccppspecificstart
\begin{boxedcode}
ompd_rc_t ompd_get_generating_ancestor_task_region(
  ompd_task_handle_t *\plc{task_handle},   /* IN */
  ompd_task_handle_t **\plc{parent_task_handle}   /* OUT */
);
\end{boxedcode}
\ccppspecificend
}

\descr
This call is meaningful only if the thread executing the task specified by
\verb|task_handle|  is stopped.

\argdesc

\crossreferences


\ompdcallbacksignature{  ompd\_get\_scheduling\_ancestor\_task\_region }
\summary
The \code{ompd\_get\_scheduling\_ancestor\_task\_region} returns the
scheduling parent of the task represented \code{task\_handle}.

\format
\vbox{
\ccppspecificstart
\begin{boxedcode}
ompd_rc_t ompd_get_scheduling_ancestor_task_region(
  ompd_task_handle_t *\plc{task_handle},   /* IN */
  ompd_task_handle_t **\plc{parent_task_handle}   /* OUT */
);
\end{boxedcode}
\ccppspecificend
}

\descr
In this operation, the scheduling parent task is the OpenMP task that was active when
the child task was scheduled. This call is meaningful only if the thread executing the task specified 
by \code{task\_handle}  is stopped. The parent task handle must be released by calling 
\code{ompd\_release\_task\_handle}.

\argdesc

\crossreferences

\ompdcallbacksignature{ ompd\_get\_implicit\_task\_in\_parallel  }
\summary
The  code{ompd\_get\_implicit\_task\_in\_parallel}  operation enables a third-party tool to obtain a 
vector of pointers to task handles for all implicit tasks associated with a parallel region.

\format
\vbox{
\ccppspecificstart
\begin{boxedcode}
ompd_rc_t ompd_get_implicit_task_in_parallel(
  ompd_parallel_handle_t *\plc{parallel_handle},   /* IN */
  ompd_task_handle_t ***\plc{task_handle_vector},   /* OUT */
  int *\plc{num_handles}   /* OUT */
);
\end{boxedcode}
\ccppspecificend
}

\descr
This call is meaningful only if all threads associated with the parallel region are stopped.
The OMPD library must use the memory allocation callback to obtain the memory for the vector of 
pointers to task handles returned by the operation. If the OMPD implementation needs to allocate 
heap memory for the task handles it returns, it must use the callbacks to acquire this memory.
After the call the vector and the task handles are `owned' by the tool, which is responsible for 
deallocating them. The task handles must be released calling \code{ompd\_release\_task\_handle}.
The vector was allocated by the OMPD implementation using the
allocation routine passed to it during the call to \code{ompd\_initialize}.
The tool itself must deallocate the vector in a compatible manner.

\argdesc

\crossreferences

\ompdcallbacksignature{ ompd\_release\_task\_handle }
\summary
This operation releases a task handle.

\format
\vbox{
\ccppspecificstart
\begin{boxedcode}
ompd_rc_t ompd_release_task_handle(
  ompd_task_handle_t *\plc{task_handle}   /* IN */
);
\end{boxedcode}
\ccppspecificend
}

\descr
Task handles are opaque to the third-party tool, which therefore cannot release
them directly. Instead, when the debugger is finished with a task handle it must
pass it to the OMPD \code{ompd\_release\_task\_handle} routine
for disposal.

\argdesc

\crossreferences


\ompdcallbacksignature{  ompd\_task\_handle\_compare }
\summary
The \code{ompd\_task\_handle\_compare} operations allows a third-party tool to compare task 
handles.

\format
\vbox{
\ccppspecificstart
\begin{boxedcode}
ompd_rc_t ompd_task_handle_compare(
  ompd_task_handle_t *\plc{task_handle_1},   /* IN */
  ompd_task_handle_t *\plc{task_handle_2},   /* IN */
  int *\plc{cmp_value}   /* OUT */
);
\end{boxedcode}
\ccppspecificend
}

\descr
The internal structure of task handles is opaque to the tool. While the tool can easily compare 
pointers to task handles, it cannot determine whether handles at two different addresses refer
to the same underlying task.

On success, \code{ompd\_task\_handle\_compare} returns in \code{*cmp\_value} a signed integer 
value that indicates how the underlying tasks compare: a value less than, equal to, or greater than 
0 indicates that the task corresponding to \plc{task\_handle\_1} is, respectively, less than, equal 
to, or greater than that corresponding to \plc{task\_handle\_2}.

For OMPD implementations that always have a single, unique, underlying task handle for a given 
task, this operation reduces to a simple comparison of the pointers. However, other 
implementations may take a different approach, and therefore the only reliable way of determining 
whether two different pointers to task handles refer the same or distinct task is to use 
\code{ompd\_task\_handle\_compare}.

Allowing task handles to be compared allows the tool to hold them in ordered collections.
The means by which task handles are ordered is implementation-defined.

\argdesc

\crossreferences

\ompdcallbacksignature{  ompd\_get\_task\_handle\_string\_id }
\summary
The \code{ompd\_get\_task\_handle\_string\_id} function returns a string that contains a unique 
printable value that identifies the task.

\format
\vbox{
\ccppspecificstart
\begin{boxedcode}
ompd_rc_t ompd_get_task_handle_string_id (
  ompd_task_handle_t *\plc{task_handle},   /* IN */
  char **\plc{string_id}   /* OUT */
);
\end{boxedcode}
\ccppspecificend
}

\descr
The returned string should be a single sequence of alphanumeric or underscore characters, and 
NULL terminated. The OMPD implementation allocates the string returned in \code{*string\_id}
using the allocation routine in the callbacks passed to it during initialization. On return the string is 
owned by the tool, which is responsible for deallocating it.

The contents of the strings returned for task handles which compare as equal with
\code{ompd\_task\_handle\_compare} must be the same.

\argdesc

\crossreferences



\ompdcallbacksignature{ ompd\_get\_num\_threads }
\summary
Determine the number of threads associated with a parallel region.

\format
\ccppspecificstart
\begin{boxedcode}
ompd\_rc\_t ompd\_get\_num\_threads(
  ompd\_parallel\_handle\_t *\plc{parallel\_handle},                         /* IN */
  ompd\_tword\_t *\plc{val}                                     /* OUT */
);
\end{boxedcode}
\ccppspecificend

\descr

\argdesc

\crossreferences


\ompdcallbacksignature{ ompd\_get\_level }
\summary
Determine the nesting depth of a particular parallel region.

\format
\ccppspecificstart
\begin{boxedcode}
ompd\_rc\_t ompd\_get\_level (
  ompd\_task\_handle\_t *\plc{task\_handle},                                 /* IN */
  ompd\_tword\_t *\plc{val}                                         /* OUT */
); 
\end{boxedcode}
\ccppspecificend

\descr

\argdesc

\crossreferences


\ompdcallbacksignature{ ompd\_get\_active\_level }
\summary
Determine the number of enclosing  parallel regions.

\format
\ccppspecificstart
\begin{boxedcode}
ompd\_rc\_t ompd\_get\_active\_level(
  ompd\_task\_handle\_t *\plc{task\_handle},                                 /* IN */
  ompd\_tword\_t *\plc{val}                                         /* OUT */
); 
\end{boxedcode}
\ccppspecificend

\descr
\code{ompd\_get\_active\_level} returns the number of nested, active
parallel regions enclosing the parallel region specified by its handle.

\argdesc

\crossreferences


\ompdcallbacksignature{ ompd\_get\_parallel\_function  }
\summary
Parallel Function Entry Point
\format
\ccppspecificstart
\begin{boxedcode}
ompd\_rd\_t ompd\_get\_parallel\_function(
  ompd\_parallel\_handle\_t *\plc{parallel\_handle},                         /* IN */
  ompd\_address\_t *\plc{entry\_point}                             /* OUT */
);
\end{boxedcode}
\ccppspecificend

\descr
The \code{ompd\_get\_parallel\_function} returns the
entry point of the code that corresponds to the body of
the parallel construct.
\argdesc

\crossreferences


\ompdcallbacksignature{ ompd\_get\_task\_function }
\summary
Task Function Entry Point
\format
\ccppspecificstart
\begin{boxedcode}
ompd\_rc\_t ompd\_get\_task\_function (
  ompd\_task\_handle\_t *\plc{task\_handle},                                 /* IN */
  ompd\_address\_t *\plc{entry\_point}                                 /* OUT */
);
\end{boxedcode}
\ccppspecificend

\descr
The \code{ompd\_get\_task\_function} returns the entry point of the code
that corresponds to the body of code executed by the task:

\argdesc

\crossreferences


\ompdcallbacksignature{ ompd\_get\_max\_threads }
\summary

\format
\ccppspecificstart
\begin{boxedcode}
ompd\_rc\_t ompd\_get\_max\_threads (
  const ompd\_task\_handle\_t  *\plc{task\_handle},                           /* IN */
  ompd\_tword\_t              *\plc{val}                                   /* OUT */
); 
\end{boxedcode}
\ccppspecificend

\descr
The \code{ompd\_get\_max\_threads} function returns the value of the
target's \emph{nthreads-var} ICV ,
and corresponds to the \code{omp\_get\_max\_threads} function
in the OpenMP runtime API.
This returns an upper bound on the number threads that could be used
to form a new team if a \code{parallel} construct without a
\code{num\_threads} clause were encountered.

The \emph{nthreads-var} ICV is defined in OpenMP as a list
(\S2.3.2 of~\cite{OpenMP}).
Like \code{omp\_get\_max\_threads}, \code{ompd\_get\_max\_threads}
returns the first element of the list.

\argdesc

\crossreferences
(\S2.3.1)

\ompdcallbacksignature{ ompd\_get\_thread\_num }
\summary

\format
\ccppspecificstart
\begin{boxedcode}
ompd\_rc\_t ompd\_get\_thread\_num (
  const ompd\_thread\_handle\_t *\plc{thread\_handle},                       /* IN */
  ompd\_tword\_t *\plc{val}                                 /* OUT */
);
\end{boxedcode}
\ccppspecificend

\descr
\code{ompd\_get\_thread\_num} is the third-party variant of the 
\code{omp\_get\_thread\_num} runtime routine, and returns the logical
thread number of the thread in the team.


\argdesc

\crossreferences


\ompdcallbacksignature{ ompd\_in\_parallel }
\summary

\format
\ccppspecificstart
\begin{boxedcode}
ompd\_rc\_t ompd\_in\_parallel (
  const ompd\_task\_handle\_t  *\plc{task\_handle},                           /* IN */
  ompd\_tword\_t              *\plc{val}                                   /* OUT */
); 
\end{boxedcode}
\ccppspecificend

\descr
\code{ompd\_in\_parallel} returns logical true (\textit{i.e.}, \code{*val != 0})
if \emph{active-levels-var}
ICV (\S2.3.1 of~\cite{OpenMP}) is greater than~0, and false (0) otherwise.
The routine corresponds to \code{omp\_in\_parallel} in the OpenMP runtime.

\argdesc

\crossreferences


\ompdcallbacksignature{ ompd\_in\_final }
\summary

\format
\ccppspecificstart
\begin{boxedcode}
ompd\_rc\_t ompd\_in\_final (
  const ompd\_task\_handle\_t  *\plc{task\_handle},                           /* IN */
  ompd\_tword\_t              *\plc{val}                                   /* OUT */
); 
\end{boxedcode}
\ccppspecificend

\descr
\code{ompd\_in\_final} corresponds to \code{omp\_in\_final} and returns
logical true if the task is a final task.

\argdesc

\crossreferences


\ompdcallbacksignature{ ompd\_get\_dynamic }
\summary

\format
\ccppspecificstart
\begin{boxedcode}
ompd\_rc\_t ompd\_get\_dynamic (
  const ompd\_task\_handle\_t  *\plc{task\_handle},                           /* IN */
  ompd\_tword\_t              *\plc{val}                                   /* OUT */
);
\end{boxedcode}
\ccppspecificend

\descr
\code{ompd\_get\_dynamic} returns the value of the
\emph{dyn-var} ICV (\S2.3.1 of~\cite{OpenMP}),
and corresponds to the \code{omp\_get\_dynamic} member of the OpenMPI API.

\argdesc
\emph{dyn-var}~determines whether dynamic adjustment of the number
of threads is enabled or disabled.

\crossreferences


\ompdcallbacksignature{ ompd\_get\_nested }
\summary

\format
\ccppspecificstart
\begin{boxedcode}
ompd\_rc\_t ompd\_get\_nested (
  const ompd\_task\_handle\_t  *\plc{task\_handle},                           /* IN */
  ompd\_tword\_t              *\plc{val}                                   /* OUT */
);
\end{boxedcode}
\ccppspecificend

\descr
\code{ompd\_get\_nested} corresponds to \code{omp\_get\_nested},
and returns the value of the \emph{nest-var} ICV (\S2.3.1 of~\cite{OpenMP}).

\argdesc
\emph{nest-var} determines if nested parallelism is enabled;
a logical true value indicates that it is, false that it is not.

\crossreferences


\ompdcallbacksignature{ ompd\_get\_max\_active\_levels }
\summary

\format
\ccppspecificstart
\begin{boxedcode}
ompd\_rc\_t ompd\_get\_max\_active\_levels (
  const ompd\_thread\_handle\_t  *\plc{thread\_handle},                       /* IN */
  ompd\_tword\_t                *\plc{val}                                 /* OUT */
);
\end{boxedcode}
\ccppspecificend

\descr
The maximum number of nested levels parallelism is returned by
\code{get\_max\_active\_levels}.
This operation corresponds to the OpenMP routine
\code{omp\_get\_max\_active\_levels}
and the ICV \emph{max-active-levels-var}.%

\argdesc

\crossreferences
(\S2.3.1 of~\cite{OpenMP})


%\begin{notes}
%	Ariel: I think this may need a little attention.
%	What is the scope of this operation?  The OpenMP4 docment refers
%	to a device.
%	
%	John: The OpenMP spec leaves ``device'' kind of vague.
%	The glossary says: ``An implementation defined logical execution engine.
%	COMMENT: A device could have one or more processors.''
%	And to a certain extent, I'm not sure it matters to OMPD.
%	``3.2.16 \code{omp\_get\_max\_active\_levels}'' in the OpenMP spec
%	implies that a thread is required, which is all I think OMPD
%	needs to care about.
%	
%	Ariel:
%	I suppose that the thread has a device associated with it.
%\end{notes}

%%%%%%%%%%%%%%%%%%%%%%%%%%%%%%%%%%%%%%%%%%%%%%%%%%%%%%%%%%%%%%%%%%%%%%
%
\ompdcallbacksignature{ ompd\_get\_schedule }
\summary

\format
\ccppspecificstart
\begin{boxedcode}
ompd\_rc\_t ompd\_get\_schedule (
  ompd\_task\_handle\_t        *\plc{task\_handle},                           /* IN */
  ompd\_sched\_t              *\plc{kind},                                 /* OUT */
  ompd\_tword\_t              *\plc{modifier}                              /* OUT */
); 
\end{boxedcode}
\ccppspecificend

\descr

\code{ompd\_get\_schedule} returns information about the schedule that is
applied when \code{runtime} scheduling is used.
%
This information is represented in the target by the
\emph{run-sched-var} ICV (\S3.2.1 of~\cite{OpenMP}).
\argdesc
OpenMP defines a minimum set of values in the enumeration type
\code{omp\_sched\_t} (\S3.2.12 of~\cite{OpenMP}).
%
The OMPD API defines
\code{ompd\_sched\_t}, which contains the
corresponding OpenMP enumeration values and ``lo'' and ``hi'' values
for the range of implementation-specific scheduling values that can be
represented by the OMPD API.
%
The scheduling kind is returned in \code{*kind}.
%
The interpretation of \code{*modifier} depends on the value of
\code{*kind}.
%
See \S3.2.12 and \S3.2.13 of~\cite{OpenMP} for further details.

\crossreferences


%%%%%%%%%%%%%%%%%%%%%%%%%%%%%%%%%%%%%%%%%%%%%%%%%%%%%%%%%%%%%%%%%%%%%%
%
\ompdcallbacksignature{ ompd\_get\_proc\_bind }
\summary

\format
\ccppspecificstart
\begin{boxedcode}
ompd\_rc\_t ompd\_get\_proc\_bind (
  ompd\_task\_handle\_t        *\plc{task\_handle},                           /* IN */
  ompd\_proc\_bind\_t          *\plc{bind}                                  /* OUT */
); 
\end{boxedcode}
\ccppspecificend

\descr
\code{ompd\_get\_proc\_bind} returns the value of the task's
\emph{bind-var} ICV (\S2.3.1 of~\cite{OpenMP}), which ``controls the
binding of the OpenMP threads to places,'' or ``default thread
affinity policies.''
%
The OMPD API defines
\code{ompd\_proc\_bind\_t}, which
contains the corresponding OpenMP enumeration values.
%

\argdesc
The binding is returned in \code{*bind}.
%

\crossreferences
See \S3.2.22 of~\cite{OpenMP} for further details.

\ompdcallbacksignature{ ompd\_is\_implicit }
\summary

\format
\ccppspecificstart
\begin{boxedcode}
ompd\_rc\_t ompd\_is\_implicit (
  ompd\_task\_handle\_t  *\plc{task\_handle},                                 /* IN */
  ompd\_tword\_t        *\plc{val}                                         /* OUT */
); 
\end{boxedcode}
\ccppspecificend

\descr
\code{ompd\_is\_implicit} returns logical true (\textit{i.e.}, \code{*val != 0})
if a task is implicit, and false (0) otherwise.
The routine has no corresponding call in the OpenMP runtime.

\argdesc

\crossreferences



%%%%%%%%%%%%%%%%%%%%%%%%%%%%%%%%%%%%%%%%%%%%%%%%%%%%%%%%%%%%%%%%%%%%%%
%
\ompdcallbacksignature{ ompd\_get\_task\_frame }
\summary

\format
\ccppspecificstart
\begin{boxedcode}
ompd\_rc\_t ompd\_get\_task\_frame (
  ompd\_task\_handle\_t        *\plc{task\_handle},                           /* IN */
  ompd\_address\_t            *\plc{exit\_runtime\_addr},                    /* OUT */
  ompd\_address\_t            *\plc{reenter\_runtime\_addr}                  /* OUT */
);
\end{boxedcode}
\ccppspecificend

\descr
The function \code{ompd\_get\_task\_frame}
ia a third-party versions of \code{ompt\_get\_task\_frame}.
%
\code{ompd\_get\_task\_frame} is discussed under Stack Unwinding in
Section~\ref{stack-unwinding:sec}.
%

\argdesc

\crossreferences

%

%%%%%%%%%%%%%%%%%%%%%%%%%%%%%%%%%%%%%%%%%%%%%%%%%%%%%%%%%%%%%%%%%%%%%%%
%%
%\subsubsection{Stack Unwinding}
%\label{stack-unwinding:sec}
%%
%\begin{notes}
%	JVD: This section needs careful review by the OpenMP Tools Working
%	Group to ensure its correctness.  It depends on whether or not John
%	Mellor-Crummey's 07/16/15 email proposal to omp-tools@openmp.org to
%	change the semantics of the \code{reenter\_runtime\_addr} field is
%	adopted.  What we decide, OMPD and OMPT should be consistent.
%\end{notes}
%
%The \code{ompd\_get\_task\_frame} function returns stack frame
%information about the target thread associated with the task.
%%
%This routine corresponds to \code{ompt\_get\_task\_frame} in the OMPT
%API, and the approach for stack inspection is similar to that
%described in Appendix~B of~\cite{ompt-tr2}.
%%
%The \code{exit\_runtime\_addr} gives the address of the frame at which
%the thread \emph{left} the OpenMP runtime to execute the user code
%associated with the task.
%%
%The \code{reenter\_runtime\_addr} is the address of the frame that
%called the OpenMP runtime.
%\begin{notes}
%	JVD: Follows John Mellor-Crummey's
%	07/16/15 email proposal to omp-tools@openmp.org to change the
%	semantics of the \code{reenter\_runtime\_addr} field.)
%\end{notes}
%%
%The debugger can unwind a thread's stack by getting the
%thread's current task using
%\refdef{\code{ompd\_get\_top\_task\_region}}{get-top-task-region:def}.
%\begin{notes}
%	JVD: This assumes that the thread is ``bound'' to the task
%	handle. Is that correct?
%\end{notes}
%%
%Using the task handle, the debugger can find the thread's exit and
%reentry stack frame addresses using \code{ompd\_get\_task\_frame}.
%%
%It can then use \refdef{\code{ompd\_get\_scheduling\_ancestor\_task\_region}}{get-scheduling-ancestor-task-region:def} to find the next task in the
%thread's stack, and then call \code{ompd\_get\_task\_frame} for the parent task.
%%
%The frames between the scheduling parent task's reenter address and the top
%task's exit address are frames in which the thread is executing OpenMP
%runtime code.
%\begin{notes}
%	JVD: Is this still accurate given John M-C's
%	proposed new semantics? I think with the new semantics, the addresses
%	are always for user frames, not OpenMP runtime frames, so ``between''
%	means \emph{exclusive} of the frame addresses.
%\end{notes}
%%
%This process can be repeated to allow all frames in the thread's
%backtrace that correspond to execution in the OpenMP runtime to be
%identified.
%
%Using the generating ancestor operation \refdef{\code{ompd\_get\_generating\_ancestor\_task\_region}}{get-generating-ancestor-task-region:def},
%the same process can be used to determine the chain logical ancestors.
%Note that in this case the logical backtrace may traverse
%across the stacks of different threads.
%
%The position within the stack frame where the runtime addresses point
%is implementation defined.
%
%
%
%
%


\ompdcallbacksignature{ ompd\_get\_thread\_handle }
\summary
Mapping an operating system thread to an OMPD thread handle.
\format
\ccppspecificstart
\begin{boxedcode}
ompd\_rc\_t ompd\_get\_thread\_handle(
  ompd\_address\_space\_handle\_t   *\plc{handle},                            /* IN */
  ompd\_osthread\_kind\_t           \plc{kind},                              /* IN */
  ompd\_size\_t                    \plc{sizeof\_osthread},                   /* IN */
  const void                    *\plc{osthread},                          /* IN */
  ompd\_thread\_handle\_t         **\plc{thread\_handle}                     /* OUT */
);
\end{boxedcode}
\ccppspecificend

\descr
OMPD provides the function \code{ompd\_get\_thread\_handle}
to inquire whether an operating system thread is an OpenMP
thread or not.
If the function returns \code{ompd\_rc\_ok}, then the operating
system thread is an OpenMP thread and \code{*thread\_handle}
will be initialized to a pointer to the thread handle for
the OpenMP thread.

\argdesc

The operating system ID \plc{*osthread} is guaranteed
to be valid for the duration of the call.
If the OMPD implementation needs to retain the operating system-specific
thread identifier it must copy it.

The thread handle \plc{*thread\_handle} returned by the OMP implementation
is `owned' by the debugger, which must release it by calling
\code{ompd\_release\_thread\_handle}.
If \plc{os\_thread} does not refer to an OpenMP thread,
\code{ompd\_get\_thread\_handle} returns \code{ompd\_rc\_bad\_input}
and \plc{*thread\_handle} is also set to NULL.
\crossreferences


\ompdcallbacksignature{ ompd\_get\_osthread }
\summary
Mapping an OMPD thread handle to an operating system thread.
\format
\ccppspecificstart
\begin{boxedcode}
ompd\_rc\_t ompd\_get\_osthread(
  ompd\_thread\_handle\_t *\plc{thread\_handle},                             /* IN */
  ompd\_osthread\_kind\_t \plc{kind},                                      /* IN */
  ompd\_size\_t \plc{sizeof\_osthread},                           /* IN */
  void  *\plc{osthread}                                  /* OUT */
);
\end{boxedcode}
\ccppspecificend

\descr
\code{ompd\_get\_osthread} performs the mapping between an OMPD
thread handle and an operating system-specific thread identifier.

\argdesc
The caller indicates what \plc{kind} of operating system-specific thread
identifier it wants by setting the \plc{kind}
`in' parameter.
It also passes a pointer to the buffer into which the OMPD
implementation writes the operating system-specific thread identifier,
and the size of the buffer, to the OMPD implementation.
The buffer is owned by the debugger.

On success \code{ompd\_get\_osthread} returns \code{rc\_ok},
and returns the operating system-specific thread identifier in
\plc{*osthread}.
If the operation fails, the OMPD implementation returns
the appropriate value from \code{ompd\_rc\_t}.
Note that the operation should fail if the OMPD implementation is
unable to return an operating system-specific identifier of the
requested `kind' or size.

\crossreferences



\ompdcallbacksignature{ ompd\_get\_state }
\summary
Thread State Inquiry Analogue
\format
\ccppspecificstart
\begin{boxedcode}
ompd\_rc\_t ompd\_get\_state (
  ompd\_thread\_handle\_t *\plc{thread\_handle},                             /* IN */
  ompd\_state\_t *\plc{state},                                    /* OUT */
  ompd\_wait\_id\_t    *\plc{wait\_id}                                   /* OUT */
);
\end{boxedcode}
\ccppspecificend

\descr
The function \code{ompd\_get\_state} is a  third-party version of
\code{ompt\_get\_state}. 
The only difference between the OMPD and OMPT counterparts
is that the OMPD version must supply a thread handle to provide
a context for this inquiry.
% Calls to this function need not return meaningful results unless the thread 
%is stopped.

\argdesc
The states are represented by values of the enumeration type
\code{ompd\_state\_t}.
The symbolic names of the members of \code{ompd\_state\_t} should
match those of the OMPT enumeration type \code{omp\_state\_e}.
However, there is no guarantee that the numeric values of the corresponding
symbolic constants are identical.

\crossreferences
