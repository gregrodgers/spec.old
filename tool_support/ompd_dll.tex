\subsubsection{\hcode{ompd_dll_locations}}
\label{subsubsec:ompd_dll_locations}
\index{ompd_dll_locations@{\code{ompd_dll_locations}}}

\summary
The global variable \code{ompd_dll_locations} indicates
where a tool should look for OMPD libraries that are compatible
with the OpenMP implementation.

\begin{cspecific}
\begin{ompSyntax}
const char **ompd_dll_locations;
\end{ompSyntax}
\end{cspecific}


\descr
\code{ompd_dll_locations} is an \code{argv}-style vector of filename
strings that provide the names of any OMPD libraries
that are compatible with the OpenMP runtime.
The vector is NULL-terminated.

The programming model or architecture of the third-party tool, and
hence that of the required OMPD library, might not match that of
the OpenMP program to be examined.
On platforms that support multiple programming models (\textit{e.g.},
32- v. 64-bit), or in heterogenous  environments where the architectures
of the OpenMP program and third-party tool may be be different,
OpenMP implementors are encouraged to provide OMPD libraries for all models.
The vector, therefore, may name libraries that are not compatible
with the third-party tool.
This is legal, and it is up to the third-party tool to check that
a library is compatible.
(Typically, a tool might iterate over the vector until a compatible
library is found.)

\restrictions
\code{ompd_dll_locations} has external \code{C} linkage,
no demangling or other transformations are required by a third-party
tool before looking up its address in the OpenMP program.

The vector and its members must be fully initialized before
\code{ompd_dll_locations} is set to a non-NULL value.
That is, if \code{ompd_dll_locations} is not NULL, the vector
and its contents are valid.

\crossreferences
\begin{itemize}
\item
  \code{ompd_dll_locations_valid}, \specref{subsubsec:ompd_dll_locations_valid}
\item
  Finding the OMPD library, \specref{subsubsec:finding-the-ompd}
\end{itemize}

\subsubsection{\hcode{ompd_dll_locations_valid}}
\label{subsubsec:ompd_dll_locations_valid}
\index{ompd_dll_locations@{\code{ompd_dll_locations_valid}}}

\summary
The OpenMP runtime notifies third-party tools that \code{ompd_dll_locations}
is valid by allowing execution to pass through a location identified
by the symbol \code{ompd_dll_locations_valid}.


\begin{cspecific}
\begin{ompSyntax}
void ompd_dll_locations_valid(void);
\end{ompSyntax}
\end{cspecific}


\descr
Depending on how the OpenMP runtime is
implemented, \code{ompd_dll_locations} might not be a static
variable, and therefore needs to be initialized at runtime.  The
OpenMP runtime notifies third-party tools
that \code{ompd_dll_locations} is valid by having execution pass
through a location identified by the
symbol \code{ompd_dll_locations_valid}.
If \code{ompd_dll_locations} is NULL, a third-party tool, e.g., a
debugger can place a breakpoint at \code{ompd_dll_locations_valid}
to be notified when \code{ompd_dll_locations} has been initialized.
In practice, the symbol \code{ompd_dll_locations_valid} need not be
a function; instead, it may be a labeled machine instruction through
which execution passes once the vector is valid.

