\subsubsection{\code{ompd\_dll\_locations}}
\label{sec:ompd:ompd_dll_locations}
\index{ompd\_dll\_locations@{\code{ompd\_dll\_locations}}}

\summary
The global variable \code{ompd\_dll\_locations} indicates
where a tool should look for OMPD plugin(s) that are compatible 
with the OpenMP implementation.  
\vbox{
\cspecificstart
\begin{boxedcode}
const char **ompd\_dll\_locations;
\end{boxedcode}
\cspecificend
}

\descr
\code{ompd\_dll\_locations} is an \code{argv}-style vector of filename
strings that provide the names of any OMPD plugin implementations
that are compatible with the OpenMP runtime.
The vector is NULL-terminated.

The programming model or architecture of the third-party tool, and
hence that of the required OMPD plugin, might not match that of
the OpenMP program to be examined.
On platforms that support multiple programming models (\textit{e.g.},
32- v. 64-bit), or in heterogenous  environments where the architectures
of the OpenMP program and third-party tool may be be different,
OpenMP implementors are encourgaed to provide OMPD plugins for all models.
The vector, therefore, may name plugins that are not compatible
with the third-party tool.
This is legal, and it is up to the third-party tool to check that
a plugin is compatible.
(Typically, a tool might iterate over the vector until a compatible
plugin is found.)

\restrictions
\code{ompd\_dll\_locations} has external \code{C} linkage,
no demangling or other transformations are required by a third-party
tool before looking up its address in the OpenMP program.

The vector and its members must be fully initialized before
\code{ompd\_dll\_locations} is set to a non-NULL value.
That is, if \code{ompd\_dll\_locations} is not NULL, the vector
and its contents are valid.

\crossreferences
\begin{itemize}
\item
  \code{ompd\_dll\_locations\_valid}, \specref{sec:ompd:ompd_dll_locations_valid}
\item
  Finding the OMPD plugin, \specref{sec:ompd:finding-the-ompd}
\end{itemize}

\subsubsection{\code{ompd\_dll\_locations\_valid}}
\label{sec:ompd:ompd_dll_locations_valid}
\index{ompd\_dll\_locations@{\code{ompd\_dll\_locations\_valid}}}

\summary
The OpenMP runtime notifies third-party tools that \code{ompd\_dll\_locations}
is valid by allowing execution to pass through a location identified
by the symbol \code{ompd\_dll\_locations\_valid}.

\vbox{
\cspecificstart
\begin{boxedcode}
void ompd\_dll\_locations\_valid(void);
\end{boxedcode}
\cspecificend
}

\descr
Depending on how the OpenMP runtime is
implemented, \code{ompd\_dll\_locations} might not be a static
variable, and therefore needs to be initialized at runtime.  The
OpenMP runtime notifies third-party tools
that \code{ompd\_dll\_locations} is valid by having execution pass
through a location identified by the
symbol \code{ompd\_dll\_locations\_valid}.
If \code{ompd\_dll\_locations} is NULL, a third-party tool, e.g., a
debugger can place a breakpoint at \code{ompd\_dll\_locations\_valid}
to be notified when \code{ompd\_dll\_locations} has been initialized.
In practice, the symbol \code{ompd\_dll\_locations\_valid} need not be
a function; instead, it may be a labeled machine instruction through
which execution passes once the vector is valid.

