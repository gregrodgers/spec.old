\subsubsection{Thread States}
\label{sec:thread-states}

To enable a tool to understand the behavior of an executing program, 
an OpenMP implementation maintains a state for each thread. 
The state maintained for a thread is an
approximation of the thread's instantaneous state. 

\vbox{
\begin{ccppspecific}
A thread's state will be one of the values of the  
enumeration type \code{omp\_state\_t} or
an implementation-defined state value of 512 or higher. 
Thread states in the enumeration fall into several classes: 
work, barrier wait, task wait, mutex wait, target wait, 
and miscellaneous. 

\begin{boxedcode}
typedef enum omp_state_t \{
  omp_state_work_serial                      = 0x000,
  omp_state_work_parallel                    = 0x001,
  omp_state_work_reduction                   = 0x002,

  omp_state_wait_barrier                     = 0x010,
  omp_state_wait_barrier_implicit_parallel   = 0x011,
  omp_state_wait_barrier_implicit_workshare  = 0x012,
  omp_state_wait_barrier_implicit            = 0x013,
  omp_state_wait_barrier_explicit            = 0x014,

  omp_state_wait_taskwait                    = 0x020,
  omp_state_wait_taskgroup                   = 0x021,

  omp_state_wait_mutex                       = 0x040,
  omp_state_wait_lock                        = 0x041,
  omp_state_wait_critical                    = 0x042,
  omp_state_wait_atomic                      = 0x043,
  omp_state_wait_ordered                     = 0x044,

  omp_state_wait_target                      = 0x080,
  omp_state_wait_target_map                  = 0x081,
  omp_state_wait_target_update               = 0x082,

  omp_state_idle                             = 0x100, 
  omp_state_overhead                         = 0x101, 
  omp_state_undefined                        = 0x102
\} omp_state_t;
\end{boxedcode}
\end{ccppspecific}
}


A tool can query the OpenMP state of a thread at any time. 
If a tool queries the state of a thread that is not associated 
with OpenMP, the implementation reports the state as \code{omp\_state\_undefined}.



Some values of the enumeration type \code{omp\_state\_t} are used by all 
OpenMP implementations, 
e.g., \code{omp\_state\_work\_serial}, 
which indicates that a thread is executing in a serial region, and  
\code{omp\_state\_work\_parallel}, 
which indicates that a thread is executing in a parallel region.
Other values of the enumeration type describe a thread's state at 
different levels of specificity. 
For instance, an OpenMP implementation may use 
the state \code{omp\_state\_wait\_barrier}  to represent all 
waiting at barriers. It may differentiate between waiting at implicit or explicit barriers using
\code{omp\_state\_wait\_barrier\_implicit} and \code{omp\_state\_wait\_barrier\_explicit}. 
To provide full detail about the type of an implicit barrier, a runtime may report 
\code{omp\_state\_wait\_barrier\_implicit\_parallel} or 
\code{omp\_state\_wait\_barrier\_implicit\_workshare} as appropriate.

For states that represent waiting, an OpenMP implementation has the 
choice of transitioning a thread to such states early or late.
For instance, when an OpenMP thread is trying to acquire a lock,
there are several points at which an OpenMP implementation
transition the thread to the \code{omp\_state\_wait\_lock} state.
One implementation may transition the thread to the state 
early before the thread attempts to acquire a
lock. Another implementation may transition the thread to the state 
late, only if the thread begins to spin or
block to wait for an unavailable lock. A third implementation
may transition the thread to the state even later, e.g., only
after the thread waits for a significant amount of time. 

The following sections describe the classes of states and the states in each class.
\subsubsubsection{Work States}
An OpenMP implementation reports a thread in a work state 
when the thread is performing serial work, parallel work, or a reduction.

\begin{description}

\item \code{omp\_state\_work\_serial} 

  The thread is executing code outside all parallel regions. 

\item \code{omp\_state\_work\_parallel} 

  The thread is executing code within the scope of a parallel region construct.

\sloppy
\item \code{omp\_state\_work\_reduction} 
 
  The thread is combining partial reduction results from threads in its team. 
  An OpenMP implementation  
  might never report a thread in this state; a thread
  combining partial reduction results may have its state reported as
  \code{omp\_state\_work\_parallel} or \code{omp\_state\_overhead}.

\end{description}


\subsubsubsection{Barrier Wait States}

An OpenMP implementation reports that a thread is in a barrier wait state 
when the thread is awaiting completion of a barrier.


\begin{description}

  \item \code{omp\_state\_wait\_barrier} 
  
  \sloppy
  The thread is waiting at either an implicit or explicit barrier.
  A thread may enter this state
  early, when the thread encounters a barrier, or late, when the
  thread begins to wait at the barrier. An implementation may never report a thread in this state; instead, a thread may have its state reported
  as \code{omp\_state\_wait\_barrier\_implicit}  or \code{omp\_state\_wait\_barrier\_explicit}, as appropriate.
  
  \item \code{omp\_state\_wait\_barrier\_implicit} 
  
  \sloppy
  The thread is waiting at an implicit barrier in a parallel region. 
  A  thread may enter this state
  early, when the thread encounters a barrier, or late, when the
  thread begins to wait at the barrier.
  An OpenMP implementation may report \code{omp\_state\_wait\_barrier} 
  for implicit barriers.
  
  \item \code{omp\_state\_wait\_barrier\_implicit\_parallel} 

  The description of when a thread reports a state associated with an implicit barrier
  is described for state \code{omp\_state\_wait\_barrier\_implicit}.  
  An OpenMP implementation may report \code{omp\_state\_wait\_barrier\_implicit\_parallel} 
  for an implicit barrier that occurs at the end of a parallel region. 
  As explained in \specref{sec:ompt_callback_sync_region_t},
  reporting the state \code{omp\_state\_wait\_barrier\_implicit\_parallel} 
  permits a weaker contract between a runtime and a tool that 
  enables a simpler and faster implementation of parallel regions.

  \item \code{omp\_state\_wait\_barrier\_implicit\_workshare} 

  The description of when a thread reports a state associated with an implicit barrier
  is described for state \code{omp\_state\_wait\_barrier\_implicit}.  
  An OpenMP implementation may report \code{omp\_state\_wait\_barrier\_implicit\_parallel} 
  for an implicit barrier that occurs at the end of a worksharing construct.

  \item \code{omp\_state\_wait\_barrier\_explicit} 

  The thread is waiting at an explicit barrier  in a parallel region. 
  A thread may enter this state
  early, when the thread encounters a barrier, or late, when the
  thread begins to wait at the barrier.
  An implementation may report \code{omp\_state\_wait\_barrier} 
  for explicit barriers.

  
\end{description}
  
\subsubsubsection{Task Wait States}

\begin{description}

\item \code{omp\_state\_wait\_taskwait} 

  The thread is waiting at a taskwait construct. A 
  thread may enter this state early, when the
  thread encounters a taskwait construct, or late, when the thread
  begins to wait for an uncompleted task.

\item \code{omp\_state\_wait\_taskgroup} 

  The thread is waiting at the end of a taskgroup construct. A 
  thread may enter this state early, when the
  thread encounters the end of a taskgroup construct, or late, when the thread
  begins to wait for an uncompleted task.

\end{description}


\subsubsubsection{Mutex Wait States}

OpenMP provides several mechanisms that enforce mutual exclusion:
locks as well as critical, atomic, and ordered sections.  This
grouping contains all states used to indicate that a thread is
awaiting exclusive access to a lock, critical section, variable,
or ordered section.

An OpenMP implementation may report a thread waiting for any type
of mutual exclusion using either a state that precisely identifies
the type of mutual exclusion, or  a more generic state such as
\code{omp\_state\_wait\_mutex} or \code{omp\_state\_wait\_lock}.  This
flexibility may significantly simplify the maintenance of states
associated with mutual exclusion in the runtime when various
mechanisms for mutual exclusion rely on a common implementation,
e.g., locks.

% Section~\ref{sec:wait-identifier} describes how each thread maintains a wait identifier to identify what a thread is awaiting. Before a thread enters any state indicating that it is awaiting mutual exclusion, the OpenMP runtime will update the thread's wait identifier to indicate what the thread is awaiting. 

\begin{description}

\item \code{omp\_state\_wait\_mutex}

  The thread is waiting for a mutex of an unspecified type. A 
  thread may enter this state early, when a thread encounters a lock acquisition or a region that requires mutual exclusion, or late, when the thread begins to wait.

\item \code{omp\_state\_wait\_lock}

  The thread is waiting for a  lock  or nest lock. A 
  thread may enter this state early, when a thread
  encounters a lock \code{set} routine, or late, when the thread
  begins to wait for a lock.

\item \code{omp\_state\_wait\_critical} 

  The thread is waiting to enter a critical region. A 
  thread may enter this state early, when the
  thread encounters a critical construct, or late, when the thread
  begins to wait to enter the critical region. 


\item \code{omp\_state\_wait\_atomic} 

  The thread is waiting to enter an atomic region. A 
  thread may enter this state early, when the thread
  encounters an atomic construct, or late, when the thread begins
  to wait to enter the atomic region. 
  An implementation may opt not to report
  this state when using atomic hardware instructions that support non-blocking atomic implementations.
  

\item \code{omp\_state\_wait\_ordered} 

  The thread is waiting to enter an ordered region. A 
  thread may enter this state early, when the thread encounters
  an ordered construct, or late, when the thread begins
  to wait to enter the ordered region. 
  
\end{description}
  
\subsubsubsection{Target Wait States}

\begin{description}

\item \code{omp\_state\_wait\_target} 

  The thread is waiting for a target region to complete.
  
\item \code{omp\_state\_wait\_target\_map} 

  The thread is waiting for a target data mapping operation to complete. 
  An implementation may report \code{omp\_state\_wait\_target} 
  for target data constructs.

\item \code{omp\_state\_wait\_target\_update} 

  The thread is waiting for a target  update operation to complete. 
  An implementation may report \code{omp\_state\_wait\_target} 
  for target update constructs.

\end{description}


\subsubsubsection{Miscellaneous States}

\begin{description}
\item \code{omp\_state\_idle} 

  The thread is idle, waiting for work.

\item \code{omp\_state\_overhead} 

  A thread may be reported as being in the overhead state at any point while 
  executing within an OpenMP runtime, except while waiting indefinitely
  at a synchronization point.
%  e.g., while preparing to execute a parallel, task, or worksharing construct. 
  An OpenMP implementation report a thread's state as a work state for
  some or all of the time the thread spends in executing in the OpenMP runtime.

\item \code{omp\_state\_undefined} 

  This state is reserved for threads that are not user threads,
  initial threads, threads currently in an OpenMP team, or threads
  waiting to become part of an OpenMP team.

\end{description}
