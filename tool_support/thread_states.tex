\subsubsection{Thread States}
\label{sec:thread-states}
\label{sec:omp_state_t}
\summary
If the OMPT interface is in state \plc{active}, an OpenMP implementation
must maintain \plc{thread state} information for each thread.
The thread state maintained is an approximation of the instantaneous
state of a thread.

\format
\begin{ccppspecific}
A thread state must be one of the values of the
enumeration type \code{omp_state_t} or
an implementation-defined state value of 512 or higher.

\begin{ompcEnum}
typedef enum omp_state_t {
  omp_state_work_serial                      = 0x000,
  omp_state_work_parallel                    = 0x001,
  omp_state_work_reduction                   = 0x002,

  omp_state_wait_barrier                     = 0x010,
  omp_state_wait_barrier_implicit_parallel   = 0x011,
  omp_state_wait_barrier_implicit_workshare  = 0x012,
  omp_state_wait_barrier_implicit            = 0x013,
  omp_state_wait_barrier_explicit            = 0x014,

  omp_state_wait_taskwait                    = 0x020,
  omp_state_wait_taskgroup                   = 0x021,

  omp_state_wait_mutex                       = 0x040,
  omp_state_wait_lock                        = 0x041,
  omp_state_wait_critical                    = 0x042,
  omp_state_wait_atomic                      = 0x043,
  omp_state_wait_ordered                     = 0x044,

  omp_state_wait_target                      = 0x080,
  omp_state_wait_target_map                  = 0x081,
  omp_state_wait_target_update               = 0x082,

  omp_state_idle                             = 0x100,
  omp_state_overhead                         = 0x101,
  omp_state_undefined                        = 0x102
} omp_state_t;
\end{ompcEnum}
\end{ccppspecific}

\descr

A tool can query the OpenMP state of a thread at any time.
If a tool queries the state of a thread that is not associated
with OpenMP, the implementation reports the state as \code{omp_state_undefined}.


The value \code{omp_state_work_serial} indicates that the thread is executing code 
outside all parallel regions.

The value \code{omp_state_work_parallel} indicates that the thread is executing code 
within the scope of a parallel region construct.

The value \code{omp_state_work_reduction} indicates that the thread is combining partial 
reduction results from threads in its team.
An OpenMP implementation might never report a thread in this state; a thread combining 
partial reduction results may have its state reported as \code{omp_state_work_parallel} 
or \code{omp_state_overhead}.

The value \code{omp_state_wait_barrier} indicates that the thread is waiting at either an 
implicit or explicit barrier.
An implementation may never report a thread in this state; instead, a thread may have its 
state reported as \code{omp_state_wait_barrier_implicit}  or 
\code{omp_state_wait_barrier_explicit}, as appropriate.

The value \code{omp_state_wait_barrier_implicit} indicates that the thread is waiting at 
an implicit barrier in a parallel region. An OpenMP implementation may report 
\code{omp_state_wait_barrier} for implicit barriers.

The value \code{omp_state_wait_barrier_implicit_parallel}  indicates that the thread is 
waiting at an implicit barrier at the end of a parallel region. An OpenMP implementation 
may report \code{omp_state_wait_barrier} or \code{omp_state_wait_barrier_implicit} for 
these barriers.

The value \code{omp_state_wait_barrier_implicit_workshare}  indicates that the thread is 
waiting at an implicit barrier at the end of a workshare-construct. An OpenMP 
implementation may report \code{omp_state_wait_barrier} or 
\code{omp_state_wait_barrier_implicit} for these barriers.

The value \code{omp_state_wait_barrier_explicit} indicates that the thread is waiting in 
a \code{barrier} region. An OpenMP implementation may report 
\code{omp_state_wait_barrier} for these barriers.

The value \code{omp_state_wait_taskwait} indicates that the thread is waiting at a 
taskwait construct. 

The value \code{omp_state_wait_taskgroup} indicates that the thread is waiting at the end 
of a taskgroup construct. 

The value \code{omp_state_wait_mutex} indicates that the thread is waiting for a mutex of 
an unspecified type. 

The value \code{omp_state_wait_lock} indicates that the thread is waiting for a  lock  or 
nest lock. 

The value \code{omp_state_wait_critical} indicates that the thread is waiting to enter a 
critical region. 

The value \code{omp_state_wait_atomic} indicates that the thread is waiting to enter an 
atomic region. 

The value \code{omp_state_wait_ordered} indicates that the thread is waiting to enter an 
ordered region. 


The value \code{omp_state_wait_target} indicates that the thread is waiting for a target 
region to complete.

The value \code{omp_state_wait_target_map} indicates that the  thread is waiting for a 
target data mapping operation to complete.
An implementation may report \code{omp_state_wait_target} for target data constructs.

The value \code{omp_state_wait_target_update} indicates that the thread is waiting for a 
target  update operation to complete.
An implementation may report \code{omp_state_wait_target} for target update constructs.

The value \code{omp_state_idle} indicates that the thread is idle, that is not part of an 
OpenMP team.

The value \code{omp_state_overhead} indicates that the thread is in the overhead state at 
any point while executing within an OpenMP runtime, except while waiting indefinitely at 
a synchronization point.

The value \code{omp_state_undefined} indicates that the native thread is not created by 
the OpenMP implementation.

