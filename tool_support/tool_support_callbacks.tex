% This is an included file. See the master file for more information.
%
% When editing this file:
%
%    1. To change formatting, appearance, or style, please edit openmp.sty.
%
%    2. Custom commands and macros are defined in openmp.sty.
%
%    3. Be kind to other editors -- keep a consistent style by copying-and-pasting to
%       create new content.
%
%    4. We use semantic markup, e.g. (see openmp.sty for a full list):
%         \code{}     % for bold monospace keywords, code, operators, etc.
%         \plc{}      % for italic placeholder names, grammar, etc.
%
%    5. There are environments that provide special formatting, e.g. language bars.
%       Please use them whereever appropriate.  Examples are:
%
%         \begin{fortranspecific}
%         This is text that appears enclosed in blue language bars for Fortran.
%         \end{fortranspecific}
%
%         \begin{note}
%         This is a note.  The "Note -- " header appears automatically.
%         \end{note}
%
%    6. Other recommendations:
%         Use the convenience macros defined in openmp.sty for the minor headers
%         such as Comments, Syntax, etc.
%
%         To keep items together on the same page, prefer the use of
%         \begin{samepage}.... Avoid \parbox for text blocks as it interrupts line numbering.
%         When possible, avoid \filbreak, \pagebreak, \newpage, \clearpage unless that's
%         what you mean. Use \needspace{} cautiously for troublesome paragraphs.
%
%         Avoid absolute lengths and measures in this file; use relative units when possible.
%         Vertical space can be relative to \baselineskip or ex units. Horizontal space
%         can be relative to \linewidth or em units.
%
%         Prefer \emph{} to italicize terminology, e.g.:
%             This is a \emph{definition}, not a placeholder.
%             This is a \plc{var-name}.
%


\subsection{OMPT Tool Callback Signatures and Trace Records}
\label{sec:ompt-tool-callbacks}

The C/C++ header file (omp-tools.h) provides the definitions of 
the types that are specified throughout this subsection.

\restrictions
\begin{itemize}
\item Tool callbacks may not use OpenMP directives or call any runtime 
      library routines described in Section~\ref{chap:Runtime Library Routines}.
\end{itemize}

\subsubsection{Initialization and Finalization Callback Signature}

\subsubsubsection{\hcode{ompt_initialize_t}}
\label{sec:ompt_initialize_t}

\summary
A callback with type signature \code{ompt_initialize_t} initializes 
use of the OMPT interface.

\format

\begin{ccppspecific}
\begin{omptInquiry}
typedef int (*ompt_initialize_t) (
  ompt_function_lookup_t \plc{lookup},
  ompt_data_t *\plc{tool_data}
);
\end{omptInquiry}
\end{ccppspecific}


\descr
To use the OMPT interface, an implementation of \code{ompt_start_tool} must 
return a non-null pointer to an \code{ompt_start_tool_result_t} structure 
that contains a non-null pointer to a tool initializer with type signature 
\code{ompt_initialize_t}. An OpenMP implementation will call the initializer
after fully initializing itself but before beginning execution of any OpenMP 
construct or completing execution of any environment routine invocation.

The initializer returns a non-zero value if it succeeds.

\argdesc
The \plc{lookup} argument is a callback to an OpenMP runtime routine that 
must be used to obtain a pointer to each runtime entry point in the OMPT 
interface. The \plc{tool_data} argument is a pointer to the \plc{tool_data} 
field in the \code{ompt_start_tool_result_t} structure that \code{ompt_start_tool}
returned. The expected actions of an initializer are described in 
Section~\ref{sec:tool-initialize}.

\crossreferences
\begin{itemize}
\item \code{ompt_start_tool_result_t}, see
\specref{sec:ompt_start_tool_result_t}.

\item \code{ompt_data_t}, see \specref{sec:ompt_data_t}.

\item \code{ompt_function_lookup_t}, see
\specref{sec:ompt_function_lookup_t}.

\item \code{ompt_start_tool}, see \specref{sec:ompt_start_tool}.
\end{itemize}



\subsubsubsection{\hcode{ompt_finalize_t}}
\label{sec:ompt_finalize_t}

\summary
A tool implements a finalizer with the type signature \code{ompt_finalize_t} 
to finalize the tool's use of the OMPT interface.

\format

\begin{ccppspecific}
\begin{omptInquiry}
typedef void (*ompt_finalize_t) (
  ompt_data_t *\plc{tool_data}
);
\end{omptInquiry}
\end{ccppspecific}


\descr
To use the OMPT interface, an implementation of\code{ompt_start_tool} must return 
a non-null pointer to an \code{ompt_start_tool_result_t} structure that contains a
non-null pointer to a tool finalizer with type signature \code{ompt_finalize_t}.
An OpenMP implementation will call the tool finalizer after the last OMPT 
\plc{event} as the OpenMP implementation shuts down.

\argdesc
The \plc{tool_data} argument is a pointer to the \plc{tool_data} field in 
the \code{ompt_start_tool_result_t} structure returned by \code{ompt_start_tool}.

\crossreferences
\begin{itemize}
\item \code{ompt_start_tool_result_t}, see
\specref{sec:ompt_start_tool_result_t}.

\item \code{ompt_data_t}, see \specref{sec:ompt_data_t}.

\item \code{ompt_start_tool}, see \specref{sec:ompt_start_tool}.
\end{itemize}



\subsubsection{Event Callback Signatures and Trace Records}
\index{event callback signatures}
\label{sec:ToolsSupport_callback_signatures}

This section describes the signatures of tool callback functions that an OMPT
tool may register and that are called during runtime of an OpenMP program. An 
implementation may also provide a trace of events per device. Along with the 
callbacks, in the following standard trace records are defined. For the trace 
records, tool data arguments are replaced by an ID, which must be initialized 
by the OpenMP implementation. Each of \plc{parallel_id}, \plc{task_id}, and 
\plc{thread_id} must be unique per target region. Tool implementations of 
callbacks are not required to be \emph{async signal safe}.

\crossreferences
\begin{itemize}
\item \code{ompt_id_t}, see \specref{sec:ompt_id_t}.

\item \code{ompt_data_t}, see \specref{sec:ompt_data_t}.
\end{itemize}



\subsubsubsection{\hcode{ompt_callback_thread_begin_t}}
\index{ompt_callback_thread_begin_t@{\code{ompt_callback_thread_begin_t}}}
\label{sec:ompt_callback_thread_begin_t}

\summary
The \code{ompt_callback_thread_begin_t} type is used for callbacks
that are dispatched when native threads are created.

\format
\begin{ccppspecific}
\begin{omptCallback}
typedef void (*ompt_callback_thread_begin_t) (
  ompt_thread_t \plc{thread_type},
  ompt_data_t *\plc{thread_data}
);
\end{omptCallback}
\end{ccppspecific}

\record
\begin{ccppspecific}
\begin{omptRecord}
typedef struct ompt_record_thread_begin_t {
  ompt_thread_t \plc{thread_type};
} ompt_record_thread_begin_t;
\end{omptRecord}
\end{ccppspecific}

\argdesc
The \plc{thread_type} argument indicates the type of the new thread: initial, 
worker, or other. The binding of the \plc{thread_data} argument is the new thread.

\crossreferences
\begin{itemize}
\item \code{parallel} consruct, see \specref{sec:parallel Construct}.

\item \code{teams} consruct, see \specref{sec:teams Construct}.

\item Initial task, see \specref{subsec:Initial Task}.

\item \code{ompt_data_t} type, see \specref{sec:ompt_data_t}.

\item \code{ompt_thread_t} type, see \specref{sec:ompt_thread_t}.
\end{itemize}



\subsubsubsection{\hcode{ompt_callback_thread_end_t}}
\index{ompt_callback_thread_end_t@{\code{ompt_callback_thread_end_t}}}
\label{sec:ompt_callback_thread_end_t}

\summary
The \code{ompt_callback_thread_end_t} type is used for callbacks 
that are dispatched when native threads are destroyed.

\format
\begin{ccppspecific}
\begin{omptCallback}
typedef void (*ompt_callback_thread_end_t) (
  ompt_data_t *\plc{thread_data}
);
\end{omptCallback}
\end{ccppspecific}

\argdesc
The binding of the \plc{thread_data} argument is the thread that will be destroyed.

\crossreferences
\begin{itemize}
\item \code{parallel} consruct, see \specref{sec:parallel Construct}.

\item \code{teams} consruct, see \specref{sec:teams Construct}.

\item Initial task, see \specref{subsec:Initial Task}.

\item \code{ompt_data_t} type, see \specref{sec:ompt_data_t}.

\item \code{ompt_record_ompt_t} type, see \specref{sec:ompt_record_ompt_t}.
\end{itemize}



\subsubsubsection{\hcode{ompt_callback_parallel_begin_t}}
\index{ompt_callback_parallel_begin_t@{\code{ompt_callback_parallel_begin_t}}}
\label{sec:ompt_callback_parallel_begin_t}

\summary
The \code{ompt_callback_parallel_begin_t} type is used for callbacks 
that are dispatched when \code{parallel} and \code{teams} regions start.

\format
\begin{ccppspecific}
\begin{omptCallback}
typedef void (*ompt_callback_parallel_begin_t) (
  ompt_data_t *\plc{encountering_task_data},
  const omp_frame_t *\plc{encountering_task_frame},
  ompt_data_t *\plc{parallel_data},
  unsigned int \plc{requested_parallelism},
  int \plc{flags},
  const void *\plc{codeptr_ra}
);
\end{omptCallback}
\end{ccppspecific}

\record
\begin{ccppspecific}
\begin{omptRecord}
typedef struct ompt_record_parallel_begin_t {
  ompt_id_t \plc{encountering_task_id};
  ompt_id_t \plc{parallel_id};
  unsigned int \plc{requested_parallelism};
  int \plc{flags};
  const void *\plc{codeptr_ra};
} ompt_record_parallel_begin_t;
\end{omptRecord}
\end{ccppspecific}

\argdesc
The binding of the \plc{encountering_task_data} argument is the encountering task.

The \plc{encountering_task_frame} argument points to the frame object that is
associated with the encountering task.

The binding of the \plc{parallel_data} argument is the \code{parallel} or \code{teams}
region that is beginning.

The \plc{requested_parallelism} argument indicates the number of threads or 
teams that the user requested.

The \plc{flags} argument indicates whether the code for the region is inlined 
into the application or invoked by the runtime and also whether the region is 
a \code{parallel} or \code{teams} region. Valid values for \plc{flags} are a 
disjunction of elements in the enum \code{ompt_parallel_flag_t}.

The \plc{codeptr_ra} argument relates the implementation of an OpenMP region 
to its source code. If a runtime routine implements the region associated with 
a callback that has type signature \code{ompt_callback_parallel_begin_t} then 
\plc{codeptr_ra} contains the return address of the call to that runtime routine.  
If the implementation the region is inlined then \plc{codeptr_ra} contains the
return address of the invocation of the callback. If attribution to source code 
is impossible or inappropriate, \plc{codeptr_ra} may be \code{NULL}.

\crossreferences
\begin{itemize}
\item \code{parallel} consruct, see \specref{sec:parallel Construct}.

\item \code{teams} consruct, see \specref{sec:teams Construct}.

\item \code{ompt_data_t} type, see \specref{sec:ompt_data_t}.

\item \code{ompt_parallel_flag_t} type, see \specref{sec:ompt_parallel_flag_t}.

\item \code{omp_frame_t} type, see \specref{sec:omp_frame_t}.
\end{itemize}



\subsubsubsection{\hcode{ompt_callback_parallel_end_t}}
\index{ompt_callback_parallel_end_t@{\code{ompt_callback_parallel_end_t}}}
\label{sec:ompt_callback_parallel_end_t}

\summary
The \code{ompt_callback_parallel_end_t} type is used for callbacks 
that are dispatched when \code{parallel} and \code{teams} regions ends.

\format
\begin{ccppspecific}
\begin{omptCallback}
typedef void (*ompt_callback_parallel_end_t) (
  ompt_data_t *\plc{parallel_data},
  ompt_data_t *\plc{encountering_task_data},
  int \plc{flags},
  const void *\plc{codeptr_ra}
);
\end{omptCallback}
\end{ccppspecific}

\record
\begin{ccppspecific}
\begin{omptRecord}
typedef struct ompt_record_parallel_end_t {
  ompt_id_t \plc{parallel_id};
  ompt_id_t \plc{encountering_task_id};
  int \plc{flags};
  const void *\plc{codeptr_ra};
} ompt_record_parallel_end_t;
\end{omptRecord}
\end{ccppspecific}


\argdesc
The binding of the \plc{parallel_data} argument is the \code{parallel} or 
\code{teams} region that is ending.

The binding of the \plc{encountering_task_data} argument is the encountering task.

The \plc{flags} argument indicates whether the execution of the region is inlined 
into the application or invoked by the runtime and also whether it is a 
\code{parallel} or \code{teams} region. Values for \plc{flags} are a
disjunction of elements in the enum \code{ompt_parallel_flag_t}.

The \plc{codeptr_ra} argument relates the implementation of an OpenMP region
to its source code. If a runtime routine implements the region associated with
a callback that has type signature \code{ompt_callback_parallel_end_t} then
\plc{codeptr_ra} contains the return address of the call to that runtime routine.
If the implementation of the region is inlined then \plc{codeptr_ra} contains the
return address of the invocation of the callback. If attribution to source code
is impossible or inappropriate,\plc{codeptr_ra} may be \code{NULL}.

\crossreferences
\begin{itemize}
\item \code{parallel} consruct, see \specref{sec:parallel Construct}.

\item \code{teams} consruct, see \specref{sec:teams Construct}.

\item \code{ompt_data_t} type signature, see \specref{sec:ompt_data_t}.

\item \code{ompt_parallel_flag_t} type signature, 
see \specref{sec:ompt_parallel_flag_t}.
\end{itemize}



\subsubsubsection{\hcode{ompt_callback_master_t}}
\index{ompt_callback_master_t@{\code{ompt_callback_master_t}}}
\label{sec:ompt_callback_master_t}

\summary
The \code{ompt_callback_master_t} type is used for callbacks that are 
dispatched when \code{master} regions start and end.

\format
\begin{ccppspecific}
\begin{omptCallback}
typedef void (*ompt_callback_master_t) (
  ompt_scope_endpoint_t \plc{endpoint},
  ompt_data_t *\plc{parallel_data},
  ompt_data_t *\plc{task_data},
  const void *\plc{codeptr_ra}
);
\end{omptCallback}
\end{ccppspecific}

\record
\begin{ccppspecific}
\begin{omptRecord}
typedef struct ompt_record_master_t {
  ompt_scope_endpoint_t \plc{endpoint};
  ompt_id_t \plc{parallel_id};
  ompt_id_t \plc{task_id};
  const void *\plc{codeptr_ra};
} ompt_record_master_t;
\end{omptRecord}
\end{ccppspecific}

\argdesc
The \plc{endpoint} argument indicates that the callback signals
the beginning of a scope or the end of a scope.

The binding of the \plc{parallel_data} argument is the current parallel region.

The binding of the \plc{task_data} argument is the encountering task.

The \plc{codeptr_ra} argument relates the implementation of an OpenMP region
to its source code. If a runtime routine implements the region associated with
a callback that has type signature \code{ompt_callback_master_t} then
\plc{codeptr_ra} contains the return address of the call to that runtime routine.
If the implementation of the region is inlined then \plc{codeptr_ra} contains the
return address of the invocation of the callback. If attribution to source code
is impossible or inappropriate,\plc{codeptr_ra} may be \code{NULL}.

\crossreferences
\begin{itemize}
\item \code{master} construct, see \specref{sec:master}.

\item \code{ompt_data_t} type signature, see \specref{sec:ompt_data_t}.

\item \code{ompt_scope_endpoint_t} type, see \specref{sec:ompt_scope_endpoint_t}.
\end{itemize}



\subsubsubsection{\hcode{ompt_callback_task_create_t}}
\index{ompt_callback_task_create_t@{\code{ompt_callback_task_create_t}}}
\label{sec:ompt_callback_task_create_t}

\summary
The \code{ompt_callback_task_create_t} type is used for callbacks that are 
dispatched when \code{task} regions or initial tasks are generated.

\format
\begin{ccppspecific}
\begin{omptCallback}
typedef void (*ompt_callback_task_create_t) (
  ompt_data_t *\plc{encountering_task_data},
  const omp_frame_t *\plc{encountering_task_frame},
  ompt_data_t *\plc{new_task_data},
  int \plc{flags},
  int \plc{has_dependences},
  const void *\plc{codeptr_ra}
);
\end{omptCallback}
\end{ccppspecific}

\record
\begin{ccppspecific}
\begin{omptRecord}
typedef struct ompt_record_task_create_t {
  ompt_id_t \plc{encountering_task_id};
  ompt_id_t \plc{new_task_id};
  int \plc{flags};
  int \plc{has_dependences};
  const void *\plc{codeptr_ra};
} ompt_record_task_create_t;
\end{omptRecord}
\end{ccppspecific}

\argdesc
The binding of the \plc{encountering_task_data} argument is the encountering task.
This argument is \code{NULL} for an initial task.

The \plc{encountering_task_frame} argument points to the frame object
associated with the encountering task. This argument is \code{NULL} 
for an initial task.

The binding of the \plc{new_task_data} argument is the generated task.

The \plc{flags} argument indicates the kind of the task (initial, explicit, 
or target) that is generated. Values for \plc{flags} are a disjunction of 
elements in the \code{ompt_task_flag_t} enumeration type.

The \plc{has_dependences} argument is \plc{true} if the generated task 
has dependences and \plc{false} otherwise.

The \plc{codeptr_ra} argument relates the implementation of an OpenMP region
to its source code. If a runtime routine implements the region associated with
a callback that has type signature \code{ompt_callback_task_create_t} then
\plc{codeptr_ra} contains the return address of the call to that runtime routine.
If the implementation of the region is inlined then \plc{codeptr_ra} contains the
return address of the invocation of the callback. If attribution to source code
is impossible or inappropriate,\plc{codeptr_ra} may be \code{NULL}.

\crossreferences
\begin{itemize}
\item \code{task} construct, see \specref{subsec:task Construct}.

\item Initial task, see \specref{subsec:Initial Task}.

\item \code{ompt_data_t} type, see
\specref{sec:ompt_data_t}.

\item \code{ompt_task_flag_t} type, see
\specref{sec:ompt_task_flag_t}.

\item \code{omp_frame_t} type, see
\specref{sec:omp_frame_t}.
\end{itemize}



\subsubsubsection{\hcode{ompt_callback_dependences_t}}
\index{ompt_callback_dependences_t@{\code{ompt_callback_dependences_t}}}
\label{sec:ompt_callback_dependences_t}

\summary
The \code{ompt_callback_dependences_t} type is used for callbacks that are 
related to dependences and that are dispatched when new tasks are generated 
and when \code{ordered} constructs are encountered.

\format
\begin{ccppspecific}
\begin{omptCallback}
typedef void (*ompt_callback_dependences_t) (
  ompt_data_t *\plc{task_data},
  const ompt_dependence_t *\plc{deps},
  int \plc{ndeps}
);
\end{omptCallback}
\end{ccppspecific}

\record
\begin{ccppspecific}
\begin{omptRecord}
typedef struct ompt_record_dependences_t {
  ompt_id_t \plc{task_id};
  ompt_dependence_t \plc{dep};
  int \plc{ndeps};
} ompt_record_dependences_t;
\end{omptRecord}
\end{ccppspecific}

\argdesc
The binding of the \plc{task_data} argument is the generated task.

The \plc{deps} argument lists dependences of the new task or the 
dependence vector of the ordered construct.

The \plc{ndeps} argument specifies the length of the list passed
by the \plc{deps} argument. The memory for \plc{deps} is owned by 
the caller; the tool cannot rely on the data after the callback returns.

The performance monitor interface for tracing activity on target devices 
provides one record per dependence.

\crossreferences
\begin{itemize}
\item \code{ordered} construct, see \specref{subsec:ordered Construct}.

\item \code{depend} clause, see \specref{subsec:depend Clause}.

\item \code{ompt_data_t} type, see
\specref{sec:ompt_data_t}.

\item \code{ompt_dependence_t} type, see
\specref{sec:ompt_dependence_t}.
\end{itemize}



\subsubsubsection{\hcode{ompt_callback_task_dependence_t}}
\index{ompt_callback_task_dependence_t@{\code{ompt_callback_task_dependence_t}}}
\label{sec:ompt_callback_task_dependence_t}
\summary
The \code{ompt_callback_task_dependence_t} type is used for callbacks that are 
dispatched when unfulfilled task dependences are encountered.

\format
\begin{ccppspecific}
\begin{omptCallback}
typedef void (*ompt_callback_task_dependence_t) (
  ompt_data_t *\plc{src_task_data},
  ompt_data_t *\plc{sink_task_data}
);
\end{omptCallback}
\end{ccppspecific}

\record
\begin{ccppspecific}
\begin{omptRecord}
typedef struct ompt_record_task_dependence_t {
  ompt_id_t \plc{src_task_id};
  ompt_id_t \plc{sink_task_id};
} ompt_record_task_dependence_t;
\end{omptRecord}
\end{ccppspecific}

\argdesc
The binding of the \plc{src_task_data} argument is a running task
with an outgoing dependence.

The binding of the \plc{sink_task_data} argument is a task with an
unsatisfied incoming dependence.

\crossreferences
\begin{itemize}
\item \code{depend} clause, see \specref{subsec:depend Clause}.

\item \code{ompt_data_t} type signature, see
\specref{sec:ompt_data_t}.
\end{itemize}



\subsubsubsection{\hcode{ompt_callback_task_schedule_t}}
\index{ompt_callback_task_schedule_t@{\code{ompt_callback_task_schedule_t}}}
\label{sec:ompt_callback_task_schedule_t}

\summary
The \code{ompt_callback_task_schedule_t} type is used for callbacks that are 
dispatched when task scheduling decisions are made.

\format
\begin{ccppspecific}
\begin{omptCallback}
typedef void (*ompt_callback_task_schedule_t) (
  ompt_data_t *\plc{prior_task_data},
  ompt_task_status_t \plc{prior_task_status},
  ompt_data_t *\plc{next_task_data}
);
\end{omptCallback}
\end{ccppspecific}

\record
\begin{ccppspecific}
\begin{omptRecord}
typedef struct ompt_record_task_schedule_t {
  ompt_id_t \plc{prior_task_id};
  ompt_task_status_t \plc{prior_task_status};
  ompt_id_t \plc{next_task_id};
} ompt_record_task_schedule_t;
\end{omptRecord}
\end{ccppspecific}

\argdesc
The \plc{prior_task_status} argument indicates the status of
the task that arrived at a task scheduling point.

The binding of the \plc{prior_task_data} argument is the task that
arrived at the scheduling point.

The binding of the \plc{next_task_data} argument is the task that
is resumed at the scheduling point. This argument is \code{NULL} if 
the callback is dispatched for a \plc{task-fulfill} event.

\crossreferences
\begin{itemize}
\item Task scheduling, see \specref{subsec:Task Scheduling}.

\item \code{ompt_data_t} type, see \specref{sec:ompt_data_t}.

\item \code{ompt_task_status_t} type, see \specref{sec:ompt_task_status_t}.
\end{itemize}



\subsubsubsection{\hcode{ompt_callback_implicit_task_t}}
\index{ompt_callback_implicit_task_t@{\code{ompt_callback_implicit_task_t}}}
\label{sec:ompt_callback_implicit_task_t}

\summary
The \code{ompt_callback_implicit_task_t} type is used for callbacks that are 
dispatched when initial tasks and implicit tasks are generated and completed.

\format
\begin{ccppspecific}
\begin{omptCallback}
typedef void (*ompt_callback_implicit_task_t) (
  ompt_scope_endpoint_t \plc{endpoint},
  ompt_data_t *\plc{parallel_data},
  ompt_data_t *\plc{task_data},
  unsigned int \plc{actual_parallelism},
  unsigned int \plc{index},
  int \plc{flags}
);
\end{omptCallback}
\end{ccppspecific}

\record
\begin{ccppspecific}
\begin{omptRecord}
typedef struct ompt_record_implicit_t {
  ompt_scope_endpoint_t \plc{endpoint};
  ompt_id_t \plc{parallel_id};
  ompt_id_t \plc{task_id};
  unsigned int \plc{actual_parallelism};
  unsigned int \plc{index};
  int \plc{flags};
} ompt_record_implicit_t;
\end{omptRecord}
\end{ccppspecific}

\argdesc
The \plc{endpoint} argument indicates that the callback signals
the beginning of a scope or the end of a scope.

The binding of the \plc{parallel_data} argument is the current parallel 
region. For the \plc{implicit-task-end} event, this argument is \code{NULL}.

The binding of the \plc{task_data} argument is the implicit task that
executes the structured block of the parallel region.

The \plc{actual_parallelism} argument indicates the number of threads in 
the \code{parallel} region or the number of teams in the \code{teams} region.
For initial tasks, that are not closely nested in a \code{teams} construct, 
this argument is \code{1}. For the \plc{implicit-task-end} and the 
\plc{initial-task-end} events, this argument is \code{0}.

The \plc{index} argument indicates the thread number or team number of the 
calling thread, within the team or league that is executing the parallel or 
\code{teams} region to which the implicit task region binds. For initial tasks, 
that are not created by a \code{teams} construct, this argument is \code{1}.

The \plc{flags} argument indicates the kind of the task (initial or implicit).

\crossreferences
\begin{itemize}
\item \code{parallel} consruct, see \specref{sec:parallel Construct}.

\item \code{teams} consruct, see \specref{sec:teams Construct}.

\item \code{ompt_data_t} type, see \specref{sec:ompt_data_t}.

\item \code{ompt_scope_endpoint_t} enumeration type, see
\specref{sec:ompt_scope_endpoint_t}.
\end{itemize}



\subsubsubsection{\hcode{ompt_callback_sync_region_t}}
\index{ompt_callback_sync_region_t@{\code{ompt_callback_sync_region_t}}}
\label{sec:ompt_callback_sync_region_t}

\summary
The \code{ompt_callback_sync_region_t} type is used for callbacks that are 
dispatched when barrier regions, \code{taskwait} regions, and \code{taskgroup}
regions begin and end and when waiting begins and ends for them as well as 
for when reductions are performed.

\format
\begin{ccppspecific}
\begin{omptCallback}
typedef void (*ompt_callback_sync_region_t) (
  ompt_sync_region_t \plc{kind},
  ompt_scope_endpoint_t \plc{endpoint},
  ompt_data_t *\plc{parallel_data},
  ompt_data_t *\plc{task_data},
  const void *\plc{codeptr_ra}
);
\end{omptCallback}
\end{ccppspecific}

\record
\begin{ccppspecific}
\begin{omptRecord}
typedef struct ompt_record_sync_region_t {
  ompt_sync_region_t \plc{kind};
  ompt_scope_endpoint_t \plc{endpoint};
  ompt_id_t \plc{parallel_id};
  ompt_id_t \plc{task_id};
  const void *\plc{codeptr_ra};
} ompt_record_sync_region_t;
\end{omptRecord}
\end{ccppspecific}

\argdesc
The \plc{kind} argument indicates the kind of synchronization.

The \plc{endpoint} argument indicates that the callback signals
the beginning of a scope or the end of a scope.

The binding of the \plc{parallel_data} argument is the current parallel region.
For the \plc{barrier-end} event at the end of a parallel region this argument 
is \code{NULL}.

The binding of the \plc{task_data} argument is the current task.

The \plc{codeptr_ra} argument relates the implementation of an OpenMP region
to its source code. If a runtime routine implements the region associated with
a callback that has type signature \code{ompt_callback_sync_region_t} then
\plc{codeptr_ra} contains the return address of the call to that runtime routine.
If the implementation of the region is inlined then \plc{codeptr_ra} contains the
return address of the invocation of the callback. If attribution to source code
is impossible or inappropriate,\plc{codeptr_ra} may be \code{NULL}.

\crossreferences
\begin{itemize}
\item \code{barrier} construct, see \specref{subsec:barrier Construct}.

\item Implicit barriers, see \specref{subsec:implict-barrier}.

\item \code{taskwait} construct, see \specref{subsec:taskwait Construct}.

\item \code{taskgroup} construct, see \specref{subsec:taskgroup Construct}.

\item Properties common to all reduction clauses,
see \specref{subsubsec:Properties Common To All Reduction Clauses}.

\item \code{ompt_data_t} type, see \specref{sec:ompt_data_t}.

\item \code{ompt_scope_endpoint_t} type, see \specref{sec:ompt_scope_endpoint_t}.

\item \code{ompt_sync_region_t} type, see \specref{sec:ompt_sync_region_t}.
\end{itemize}



\subsubsubsection{\hcode{ompt_callback_mutex_acquire_t}}
\index{ompt_callback_mutex_acquire_t@{\code{ompt_callback_mutex_acquire_t}}}
\label{sec:ompt_callback_mutex_acquire_t}

\summary
The \code{ompt_callback_mutex_acquire_t} type is used for callbacks that are 
dispatched when locks are initialized, acquired and tested and when \code{critical} 
regions, \code{atomic} regions, and \code{ordered} regions are begun.

\format
\begin{ccppspecific}
\begin{omptCallback}
typedef void (*ompt_callback_mutex_acquire_t) (
  ompt_mutex_t \plc{kind},
  unsigned int \plc{hint},
  unsigned int \plc{impl},
  omp_wait_id_t \plc{wait_id},
  const void *\plc{codeptr_ra}
);
\end{omptCallback}
\end{ccppspecific}

\record
\begin{ccppspecific}
\begin{omptRecord}
typedef struct ompt_record_mutex_acquire_t {
  ompt_mutex_t \plc{kind};
  unsigned int \plc{hint};
  unsigned int \plc{impl};
  omp_wait_id_t \plc{wait_id};
  const void *\plc{codeptr_ra};
} ompt_record_mutex_acquire_t;
\end{omptRecord}
\end{ccppspecific}

\argdesc
The \plc{kind} argument indicates the kind of the lock involved.

The \plc{hint} argument indicates the hint that was provided when initializing
an implementation of mutual exclusion. If no hint is available when a thread 
initiates acquisition of mutual exclusion, the runtime may supply 
\code{omp_sync_hint_none} as the value for \plc{hint}.

The \plc{impl} argument indicates the mechanism chosen by the runtime to implement 
the mutual exclusion.

The \plc{wait_id} argument indicates the object being awaited.

The \plc{codeptr_ra} argument relates the implementation of an OpenMP region
to its source code. If a runtime routine implements the region associated with
a callback that has type signature \code{ompt_callback_mutex_acquire_t} then
\plc{codeptr_ra} contains the return address of the call to that runtime routine.
If the implementation of the region is inlined then \plc{codeptr_ra} contains the
return address of the invocation of the callback. If attribution to source code
is impossible or inappropriate,\plc{codeptr_ra} may be \code{NULL}.

\crossreferences
\begin{itemize}
\item \code{critical} construct, see \specref{subsec:critical Construct}.

\item \code{atomic} construct, see \specref{subsec:atomic Construct}.

\item \code{ordered} consruct, see \specref{subsec:ordered Construct}.

\item \code{omp_init_lock} and \code{omp_init_nest_lock} routines,
see \specref{subsec:omp_init_lock and omp_init_nest_lock}.

\item \code{ompt_mutex_t} type, see \specref{sec:ompt_mutex_t}.

\item \code{omp_wait_id_t} type, see \specref{sec:omp_wait_id_t}.
\end{itemize}



\subsubsubsection{\hcode{ompt_callback_mutex_t}}
\index{ompt_callback_mutex_t@{\code{ompt_callback_mutex_t}}}
\label{sec:ompt_callback_mutex_t}

\summary
The \code{ompt_callback_mutex_t} type is used for callbacks that indicate
important synchronization events.

\format
\begin{ccppspecific}
\begin{omptCallback}
typedef void (*ompt_callback_mutex_t) (
  ompt_mutex_t \plc{kind},
  omp_wait_id_t \plc{wait_id},
  const void *\plc{codeptr_ra}
);
\end{omptCallback}
\end{ccppspecific}

\record
\begin{ccppspecific}
\begin{omptRecord}
typedef struct ompt_record_mutex_t {
  ompt_mutex_t \plc{kind};
  omp_wait_id_t \plc{wait_id};
  const void *\plc{codeptr_ra};
} ompt_record_mutex_t;
\end{omptRecord}
\end{ccppspecific}

\argdesc
The \plc{kind} argument indicates the kind of mutual exclusion event.

The \plc{wait_id} argument indicates the object being awaited.

The \plc{codeptr_ra} argument relates the implementation of an OpenMP region
to its source code. If a runtime routine implements the region associated with
a callback that has type signature \code{ompt_callback_mutex_t} then
\plc{codeptr_ra} contains the return address of the call to that runtime routine.
If the implementation of the region is inlined then \plc{codeptr_ra} contains the
return address of the invocation of the callback. If attribution to source code
is impossible or inappropriate,\plc{codeptr_ra} may be \code{NULL}.

\crossreferences
\begin{itemize}
\item \code{critical} construct, see \specref{subsec:critical Construct}.

\item \code{atomic} construct, see \specref{subsec:atomic Construct}.

\item \code{ordered} consruct, see \specref{subsec:ordered Construct}.

\item \code{omp_destroy_lock} and \code{omp_destroy_nest_lock} routines,
see \specref{subsec:omp_destroy_lock and omp_destroy_nest_lock}.

\item \code{omp_set_lock} and \code{omp_set_nest_lock} routines,
see \specref{subsec:omp_set_lock and omp_set_nest_lock}.

\item \code{omp_unset_lock} and \code{omp_unset_nest_lock} routines,
see \specref{subsec:omp_unset_lock and omp_unset_nest_lock}.

\item \code{omp_test_lock} and \code{omp_test_nest_lock} routines,
see \specref{subsec:omp_test_lock and omp_test_nest_lock}.

\item \code{ompt_mutex_t} type signature, see \specref{sec:ompt_mutex_t}.

\item \code{omp_wait_id_t} type signature, see \specref{sec:omp_wait_id_t}.
\end{itemize}



\subsubsubsection{\hcode{ompt_callback_nest_lock_t}}
\index{ompt_callback_nest_lock_t@{\code{ompt_callback_nest_lock_t}}}
\label{sec:ompt_callback_nest_lock_t}
\summary
The \code{ompt_callback_nest_lock_t} type is used for callbacks that 
indicate that a thread that owns a nested lock has performed an action 
related to the lock but has not relinquished ownership of it.

\format
\begin{ccppspecific}
\begin{omptCallback}
typedef void (*ompt_callback_nest_lock_t) (
  ompt_scope_endpoint_t \plc{endpoint},
  omp_wait_id_t \plc{wait_id},
  const void *\plc{codeptr_ra}
);
\end{omptCallback}
\end{ccppspecific}

\record
\begin{ccppspecific}
\begin{omptRecord}
typedef struct ompt_record_nest_lock_t {
  ompt_scope_endpoint_t \plc{endpoint};
  omp_wait_id_t \plc{wait_id};
  const void *\plc{codeptr_ra};
} ompt_record_nest_lock_t;
\end{omptRecord}
\end{ccppspecific}

\argdesc
The \plc{endpoint} argument indicates that the callback signals
the beginning of a scope or the end of a scope.

The \plc{wait_id} argument indicates the object being awaited.

The \plc{codeptr_ra} argument relates the implementation of an OpenMP region
to its source code. If a runtime routine implements the region associated with
a callback that has type signature \code{ompt_callback_nest_lock_t} then
\plc{codeptr_ra} contains the return address of the call to that runtime routine.
If the implementation of the region is inlined then \plc{codeptr_ra} contains the
return address of the invocation of the callback. If attribution to source code
is impossible or inappropriate,\plc{codeptr_ra} may be \code{NULL}.

\crossreferences
\begin{itemize}
\item \code{omp_set_nest_lock} routine, 
see \specref{subsec:omp_set_lock and omp_set_nest_lock}.

\item \code{omp_unset_nest_lock} routine, 
see \specref{subsec:omp_unset_lock and omp_unset_nest_lock}.

\item \code{omp_test_nest_lock} routine, 
see \specref{subsec:omp_test_lock and omp_test_nest_lock}.

\item \code{ompt_scope_endpoint_t} type signature, see 
\specref{sec:ompt_scope_endpoint_t}.

\item \code{omp_wait_id_t} type signature, see \specref{sec:omp_wait_id_t}.
\end{itemize}



\subsubsubsection{\hcode{ompt_callback_work_t}}
\index{ompt_callback_work_t@{\code{ompt_callback_work_t}}}
\label{sec:ompt_callback_work_t}
\summary
The \code{ompt_callback_work_t} type is used for callbacks that
are dispatched when worksharing regions, loop-related regions, 
and \code{taskloop} regions begin and end.

\format
\begin{ccppspecific}
\begin{omptCallback}
typedef void (*ompt_callback_work_t) (
  ompt_work_t \plc{wstype},
  ompt_scope_endpoint_t \plc{endpoint},
  ompt_data_t *\plc{parallel_data},
  ompt_data_t *\plc{task_data},
  uint64_t \plc{count},
  const void *\plc{codeptr_ra}
);
\end{omptCallback}
\end{ccppspecific}

\record
\begin{ccppspecific}
\begin{omptRecord}
typedef struct ompt_record_work_t {
  ompt_work_t \plc{wstype};
  ompt_scope_endpoint_t \plc{endpoint};
  ompt_id_t \plc{parallel_id};
  ompt_id_t \plc{task_id};
  uint64_t \plc{count};
  const void *\plc{codeptr_ra};
} ompt_record_work_t;
\end{omptRecord}
\end{ccppspecific}

\argdesc
The \plc{wstype} argument indicates the kind of region.

The \plc{endpoint} argument indicates that the callback signals
the beginning of a scope or the end of a scope.

The binding of the \plc{parallel_data} argument is the current parallel region.

The binding of the \plc{task_data} argument is the current task.

The \plc{count} argument is a measure of the quantity of work involved in 
the construct. For a worksharing-loop construct, \plc{count} represents the 
number of iterations of the loop. For a \code{taskloop} construct, \plc{count} 
represents the number of iterations in the iteration space, which may be the 
result of collapsing several associated loops. For a \code{sections} construct, 
\plc{count} represents the number of sections. For a \code{workshare} construct,
\plc{count} represents the units of work, as defined by the \code{workshare} 
construct. For a \code{single} construct, \plc{count} is always 1. When the 
\plc{endpoint} argument signals the end of a scope, a \plc{count} value of 0 
indicates that the actual \plc{count} value is not available.

The \plc{codeptr_ra} argument relates the implementation of an OpenMP region
to its source code. If a runtime routine implements the region associated with
a callback that has type signature \code{ompt_callback_work_t} then
\plc{codeptr_ra} contains the return address of the call to that runtime routine.
If the implementation of the region is inlined then \plc{codeptr_ra} contains the
return address of the invocation of the callback. If attribution to source code
is impossible or inappropriate,\plc{codeptr_ra} may be \code{NULL}.

\crossreferences
\begin{itemize}
\item Worksharing constructs, see \specref{sec:Worksharing Constructs} and 
\specref{subsec:Worksharing-Loop Construct}.

\item Loop-related constructs, see \specref{sec:LoopRelatedConstructs}.

\item \code{taskloop} construct, see \specref{subsec:taskloop Construct}.

\item \code{ompt_data_t} type signature, see
\specref{sec:ompt_data_t}.

\item \code{ompt_scope_endpoint_t} type signature, see
\specref{sec:ompt_scope_endpoint_t}.

\item \code{ompt_work_t} type signature, see
\specref{sec:ompt_work_t}.
\end{itemize}



\subsubsubsection{\hcode{ompt_callback_flush_t}}
\index{ompt_callback_flush_t@{\code{ompt_callback_flush_t}}}
\label{sec:ompt_callback_flush_t}
\summary
The \code{ompt_callback_flush_t} type is used for callbacks that are
dispatched when \code{flush} constructs are encountered.

\format
\begin{ccppspecific}
\begin{omptCallback}
typedef void (*ompt_callback_flush_t) (
  ompt_data_t *\plc{thread_data},
  const void *\plc{codeptr_ra}
);
\end{omptCallback}
\end{ccppspecific}

\record
\begin{ccppspecific}
\begin{omptRecord}
typedef struct ompt_record_flush_t {
  const void *\plc{codeptr_ra};
} ompt_record_flush_t;
\end{omptRecord}
\end{ccppspecific}

\argdesc
The binding of the \plc{thread_data} argument is the executing thread.

The \plc{codeptr_ra} argument relates the implementation of an OpenMP region
to its source code. If a runtime routine implements the region associated with
a callback that has type signature \code{ompt_callback_flush_t} then
\plc{codeptr_ra} contains the return address of the call to that runtime routine.
If the implementation of the region is inlined then \plc{codeptr_ra} contains the
return address of the invocation of the callback. If attribution to source code
is impossible or inappropriate,\plc{codeptr_ra} may be \code{NULL}.

\crossreferences
\begin{itemize}
\item \code{flush} construct, see \specref{subsec:flush Construct}.

\item \code{ompt_data_t} type signature, see \specref{sec:ompt_data_t}.
\end{itemize}



\subsubsubsection{\hcode{ompt_callback_dispatch_t}}
\index{ompt_callback_dispatch_t@{\code{ompt_callback_dispatch_t}}}
\label{sec:ompt_callback_dispatch_t}

\summary
The \code{ompt_callback_dispatch_t} type is used for callbacks that are
dispatched when a thread begins to execute a section or loop iteration.

\format
\begin{ccppspecific}
\begin{omptCallback}
typedef void (*ompt_callback_dispatch_t) (
  ompt_data_t *\plc{parallel_data},
  ompt_data_t *\plc{task_data},
  ompt_dispatch_t \plc{kind},
  ompt_data_t \plc{instance} 
);
\end{omptCallback}
\end{ccppspecific}

\record
\begin{ccppspecific}
\begin{omptRecord}
typedef struct ompt_record_dispatch_t {
  ompt_id_t \plc{parallel_id};
  ompt_id_t \plc{task_id};
  ompt_dispatch_t \plc{kind};
  ompt_data_t \plc{instance}; 
} ompt_record_dispatch_t;
\end{omptRecord}
\end{ccppspecific}

\argdesc
The binding of the \plc{parallel_data} argument is the current parallel region.

The binding of the \plc{task_data} argument is the implicit task that executes
the structured block of the parallel region.

The \plc{kind} argument indicates whether a loop iteration or a section is being 
dispatched.

For a loop iteration, the \plc{instance.value} argument contains the iteration 
variable value. For a structured block in the \code{sections} construct, 
\plc{instance.ptr} contains a code address that identifies the structured block.  
In cases where a runtime routine implements the structured block associated with 
this callback, \plc{instance.ptr} contains the return address of the call to the 
runtime routine. In cases where the implementation of the structured block is 
inlined, \plc{instance.ptr} contains the return address of the invocation of 
this callback.

\crossreferences
\begin{itemize}
\item \code{sections} and \code{section} constructs, 
see \specref{subsec:sections Construct}.

\item Worksharing-loop construct, see \specref{subsec:Worksharing-Loop Construct}.

\item \code{taskloop} construct, see \specref{subsec:taskloop Construct}.

\item \code{ompt_data_t} type signature, see   \specref{sec:ompt_data_t}.

\item \code{ompt_dispatch_t} type, see \specref{sec:ompt_dispatch_t}.
\end{itemize}



\subsubsubsection{\hcode{ompt_callback_target_t}}
\index{ompt_callback_target_t@{\code{ompt_callback_target_t}}}
\label{sec:ompt_callback_target_t}

\summary
The \code{ompt_callback_target_t} type is used for callbacks that are
dispatched when a thread begins to execute a device construct.

\format
\begin{ccppspecific}
\begin{omptCallback}
typedef void (*ompt_callback_target_t) (
  ompt_target_t \plc{kind},
  ompt_scope_endpoint_t \plc{endpoint},
  uint64_t \plc{device_num},
  ompt_data_t *\plc{task_data},
  ompt_id_t \plc{target_id},
  const void *\plc{codeptr_ra}
);
\end{omptCallback}
\end{ccppspecific}

\record
\begin{ccppspecific}
\begin{omptRecord}
typedef struct ompt_record_target_t {
  ompt_target_t \plc{kind};
  ompt_scope_endpoint_t \plc{endpoint};
  uint64_t \plc{device_num};
  ompt_data_t *\plc{task_data};
  ompt_id_t \plc{target_id};
  const void *\plc{codeptr_ra};
} ompt_record_target_t;
\end{omptRecord}
\end{ccppspecific}


\argdesc
The \plc{kind} argument indicates the kind of target region.

The \plc{endpoint} argument indicates that the callback signals 
the beginning of a scope or the end of a scope.

The \plc{device_num} argument indicates the id of the device that
will execute the target region.

The binding of the \plc{task_data} argument is the generating task.

The binding of the \plc{target_id} argument is the target region.

The \plc{codeptr_ra} argument relates the implementation of an OpenMP region
to its source code. If a runtime routine implements the region associated with
a callback that has type signature \code{ompt_callback_target_t} then
\plc{codeptr_ra} contains the return address of the call to that runtime routine.
If the implementation of the region is inlined then \plc{codeptr_ra} contains the
return address of the invocation of the callback. If attribution to source code
is impossible or inappropriate,\plc{codeptr_ra} may be \code{NULL}.

\crossreferences
\begin{itemize}
\item \code{target}~\code{data} construct, see \specref{subsec:target data Construct}.

\item \code{target}~\code{enter}~\code{data} construct, 
see \specref{subsec:target enter data Construct}.

\item \code{target}~\code{exit}~\code{data} construct, 
see \specref{subsec:target exit data Construct}.

\item \code{target} construct, see \specref{subsec:target Construct}.

\item \code{target}~\code{update} construct, 
see \specref{subsec:target update Construct}.

\item \code{ompt_id_t} type, see \specref{sec:ompt_id_t}.

\item \code{ompt_data_t} type signature, see \specref{sec:ompt_data_t}.

\item \code{ompt_scope_endpoint_t} type signature, see
\specref{sec:ompt_scope_endpoint_t}.

\item \code{ompt_target_t} type signature, see \specref{sec:ompt_target_t}.
\end{itemize}



\subsubsubsection{\hcode{ompt_callback_device_load_t}}
\index{ompt_callback_target_code_t@{\code{ompt_callback_device_load_t}}}
\label{sec:ompt_callback_device_load_t}

\summary
The \code{ompt_callback_device_load_t} type is used for callbacks that 
the OpenMP runtime invokes to indicate that it has just loaded code onto 
the specified device.

\format
\begin{ccppspecific}
\begin{omptCallback}
typedef void (*ompt_callback_device_load_t) (
  uint64_t \plc{device_num},
  const char *\plc{filename},
  int64_t \plc{offset_in_file},
  void *\plc{vma_in_file},
  size_t \plc{bytes}
  void *\plc{host_addr},
  void *\plc{device_addr},
  uint64_t \plc{module_id}
);
\end{omptCallback}
\end{ccppspecific}

\argdesc
The \plc{device_num} argument specifies the device.

The \plc{filename} argument indicates the name of a file in which the device 
code can be found. A NULL \plc{filename} indicates that the code is not available 
in a file in the file system.

The \plc{offset_in_file} argument indicates an offset into \plc{filename} at 
which the code can be found. A value of -1 indicates that no offset is provided.

\code{ompt_addr_none} is defined as a pointer with the value \textasciitilde 0.

The \plc{vma_in_file} argument indicates an virtual address in \plc{filename} 
at which the code can be found. A value of \code{ompt_addr_none} indicates that 
a virtual address in the file is not available.

The argument \plc{bytes} indicates the size of the device code object in bytes.

The \plc{host_addr} argument indicates the address at which a copy of the device 
code is available in host memory. A value of \code{ompt_addr_none} indicates that 
a host code address is not available.

The \plc{device_addr} argument indicates the address at which the device code has 
been loaded in device memory. A value of \code{ompt_addr_none} indicates that a 
device code address is not available.

The \plc{module_id} argument is an identifier that is associated with the device 
code object.

\crossreferences
\begin{itemize}
\item Device directives, see \specref{sec:Device Directives}.
\end{itemize}



\subsubsubsection{\hcode{ompt_callback_device_unload_t}}
\index{ompt_callback_target_code_t@{\code{ompt_callback_device_unload_t}}}
\label{sec:ompt_callback_device_unload_t}

\summary
The \code{ompt_callback_device_unload_t} type is used for callbacks that 
the OpenMP runtime invokes to indicate that it is about to unload code from
the specified device.

\format
\begin{ccppspecific}
\begin{omptCallback}
typedef void (*ompt_callback_device_unload_t) (
  uint64_t \plc{device_num},
  uint64_t \plc{module_id}
);
\end{omptCallback}
\end{ccppspecific}

\argdesc
The \plc{device_num} argument specifies the device.

The \plc{module_id} argument is an identifier that is associated 
with the device code object.

\crossreferences
\begin{itemize}
\item Device directives, see \specref{sec:Device Directives}.
\end{itemize}



\subsubsubsection{\hcode{ompt_callback_target_data_op_t}}
\index{ompt_callback_target_data_op_t@{\code{ompt_callback_target_data_op_t}}}
\label{sec:ompt_callback_target_data_op_t}
\summary
The \code{ompt_callback_target_data_op_t} type is used for callbacks that are
dispatched when a thread maps data to a device.

\format
\begin{ccppspecific}
\begin{omptCallback}
typedef void (*ompt_callback_target_data_op_t) (
  ompt_id_t \plc{target_id},
  ompt_id_t \plc{host_op_id},
  ompt_target_data_op_t \plc{optype},
  void *\plc{src_addr},
  int \plc{src_device_num},
  void *\plc{dest_addr},
  int \plc{dest_device_num},
  size_t \plc{bytes},
  const void *\plc{codeptr_ra}
);
\end{omptCallback}
\end{ccppspecific}

\record
\begin{ccppspecific}
\begin{omptRecord}
typedef struct ompt_record_target_data_op_t {
  ompt_id_t \plc{host_op_id};
  ompt_target_data_op_t \plc{optype};
  void *\plc{src_addr};
  int \plc{src_device_num};
  void *\plc{dest_addr};
  int \plc{dest_device_num};
  size_t \plc{bytes};
  ompt_device_time_t \plc{end_time};
  const void *\plc{codeptr_ra};
} ompt_record_target_data_op_t;
\end{omptRecord}
\end{ccppspecific}

\descr
A registered \code{ompt_callback_target_data_op} callback is dispatched when device 
memory is allocated or freed, as well as when data is copied to or from a device.

\begin{note}
An OpenMP implementation may aggregate program variables and data operations 
upon them.  For instance, an OpenMP implementation may synthesize a composite 
to represent multiple scalars and then allocate, free, or copy this composite 
as a whole rather than performing data operations on each scalar individually.  
Thus, callbacks may not be dispatched as separate data operations on each variable.
\end{note}

\argdesc
The \plc{host_op_id} argument is a unique identifer for a data 
operations on a target device.

The \plc{optype} argument indicates the kind of data mapping.

The \plc{src_addr} argument indicates the data address before the operation, 
where applicable.

The \plc{src_device_num} argument indicates the source device number
for the data operation, where applicable.

The \plc{dest_addr} argument indicates the data address after the operation.

The \plc{dest_device_num} argument indicates the destination device
number for the data operation.

It is implementation defined whether in some operations \plc{src_addr}
or \plc{dest_addr} may point to an intermediate buffer.

The \plc{bytes} argument indicates the size of data.

The \plc{codeptr_ra} argument relates the implementation of an OpenMP region
to its source code. If a runtime routine implements the region associated with
a callback that has type signature \code{ompt_callback_target_data_op_t} then
\plc{codeptr_ra} contains the return address of the call to that runtime routine.
If the implementation of the region is inlined then \plc{codeptr_ra} contains the
return address of the invocation of the callback. If attribution to source code
is impossible or inappropriate,\plc{codeptr_ra} may be \code{NULL}.

\crossreferences
\begin{itemize}
\item \code{map} clause, see \specref{subsec:map Clause}.

\item \code{ompt_id_t} type, see \specref{sec:ompt_id_t}.

\item \code{ompt_target_data_op_t} type signature, see
\specref{sec:ompt_target_data_op_t}.
\end{itemize}



\subsubsubsection{\hcode{ompt_callback_target_map_t}}
\index{ompt_callback_target_map_t@{\code{ompt_callback_target_map_t}}}
\label{sec:ompt_callback_target_map_t}
\summary
The \code{ompt_callback_target_map_t} type is used for callbacks that are
dispatched to indicate data mapping relationships.

\format
\begin{ccppspecific}
\begin{omptCallback}
typedef void (*ompt_callback_target_map_t) (
  ompt_id_t \plc{target_id},
  unsigned int \plc{nitems},
  void **\plc{host_addr},
  void **\plc{device_addr},
  size_t *\plc{bytes},
  unsigned int *\plc{mapping_flags},
  const void *\plc{codeptr_ra}
);
\end{omptCallback}
\end{ccppspecific}

\record
\begin{ccppspecific}
\begin{omptRecord}
typedef struct ompt_record_target_map_t {
  ompt_id_t \plc{target_id};
  unsigned int \plc{nitems};
  void **\plc{host_addr};
  void **\plc{device_addr};
  size_t *\plc{bytes};
  unsigned int *\plc{mapping_flags};
  const void *\plc{codeptr_ra};
} ompt_record_target_map_t;
\end{omptRecord}
\end{ccppspecific}

\descr
An instance of a \code{target}, \code{target data}, \code{target enter data}, or 
\code{target exit data} construct may contain one or more \code{map} clauses.
An OpenMP implementation may report the set of mappings associated with \code{map} 
clauses for a construct with a single \code{ompt_callback_target_map} callback to 
report the effect of all mappings or multiple \code{ompt_callback_target_map} 
callbacks with each reporting a subset of the mappings. Furthermore, an OpenMP 
implementation may omit mappings that it determines are unnecessary. If an OpenMP 
implementation issues multiple \code{ompt_callback_target_map} callbacks, these 
callbacks may be interleaved with \code{ompt_callback_target_data_op} callbacks
used to report data operations associated with the mappings.

\argdesc
The binding of the \plc{target_id} argument is the target region.

The \plc{nitems} argument indicates the number of data mappings that 
this callback reports.

The \plc{host_addr} argument indicates an array of host data addresses.

The \plc{device_addr} argument indicates an array of device data addresses.

The \plc{bytes} argument indicates an array of size of data.

The \plc{mapping_flags} argument indicates the kind of data mapping. 
Flags for a mapping include one or more values specified by the 
\code{ompt_target_map_flag_t} type.

The \plc{codeptr_ra} argument relates the implementation of an OpenMP region
to its source code. If a runtime routine implements the region associated with
a callback that has type signature \code{ompt_callback_target_map_t} then
\plc{codeptr_ra} contains the return address of the call to that runtime routine.
If the implementation of the region is inlined then \plc{codeptr_ra} contains the
return address of the invocation of the callback. If attribution to source code
is impossible or inappropriate,\plc{codeptr_ra} may be \code{NULL}.

\crossreferences
\begin{itemize}
\item \code{target}~\code{data} construct, see \specref{subsec:target data Construct}.

\item \code{target}~\code{enter}~\code{data} construct,
see \specref{subsec:target enter data Construct}.

\item \code{target}~\code{exit}~\code{data} construct,
see \specref{subsec:target exit data Construct}.

\item \code{target} construct, see \specref{subsec:target Construct}.

\item \code{ompt_id_t} type, see \specref{sec:ompt_id_t}.

\item \code{ompt_target_map_flag_t} type, see \specref{sec:ompt_target_map_flag_t}.

\item \code{ompt_callback_target_data_op_t}, 
see \specref{sec:ompt_callback_target_data_op_t}.
\end{itemize}



\subsubsubsection{\hcode{ompt_callback_target_submit_t}}
\index{ompt_callback_target_submit_t@{\code{ompt_callback_target_submit_t}}}
\label{sec:ompt_callback_target_submit_t}
\summary
The \code{ompt_callback_target_submit_t} type is used for callbacks that are
dispatched when an initial task is created on a device.

\format
\begin{ccppspecific}
\begin{omptCallback}
typedef void (*ompt_callback_target_submit_t) (
  ompt_id_t \plc{target_id},
  ompt_id_t \plc{host_op_id},
  unsigned int \plc{requested_num_teams}
);
\end{omptCallback}
\end{ccppspecific}

\record
\begin{ccppspecific}
\begin{omptRecord}
typedef struct ompt_record_target_kernel_t {
  ompt_id_t \plc{host_op_id};
  unsigned int \plc{requested_num_teams};
  unsigned int \plc{granted_num_teams};
  ompt_device_time_t \plc{end_time};
} ompt_record_target_kernel_t;
\end{omptRecord}
\end{ccppspecific}

\descr
A thread dispatches a registered \code{ompt_callback_target_submit} callback 
on the host when a target task creates an initial task on a target device.

\argdesc
The \plc{target_id} argument is a unique identifier for the associated target region.

The \plc{host_op_id} argument is a unique identifer for the initial task 
on the target device.

The \plc{requested_num_teams} argument is the number of teams that the host 
requested to execute the kernel. The actual number of teams that execute the 
kernel may be smaller and generally will not be known until the kernel begins 
to execute on the device.

The \plc{target_id} argument indicates the instance of the \code{target} construct 
to which the computation belongs.

The \plc{host_op_id} argument provides a unique host-side identifier that 
represents the computation on the device.

If \code{ompt_set_trace_ompt} has configured the device to trace kernel execution
then the device will log a \code{ompt_record_target_kernel_t} record in a trace. 
The fields in the record are as follows:

\begin{itemize}
\item The \plc{host_op_id} field contains a unique identifier that can be used 
      to correlate a \code{ompt_record_target_kernel_t} record with its associated 
      \code{ompt_callback_target_submit} callback on the host;
\item The \plc{requested_num_teams} field contains the number of teams that the 
      host requested to execute the kernel;
\item The \plc{granted_num_teams} field contains the number of teams that the 
      device actually used to execute the kernel;
\item The time when the initial task began execution on the device is recorded 
      in the \plc{time} field of an enclosing \code{ompt_record_t} structure; and
\item The time when the initial task completed execution on the device is recorded 
      in the \plc{end_time} field.
\end{itemize}

\crossreferences
\begin{itemize}
\item \code{target} construct, see \specref{subsec:target Construct}.

\item \code{ompt_id_t} type, see \specref{sec:ompt_id_t}.
\end{itemize}



\subsubsubsection{\hcode{ompt_callback_buffer_request_t}}
\index{ompt_callback_buffer_request_t@{\code{ompt_callback_buffer_request_t}}}
\label{sec:ompt_callback_buffer_request_t}

\summary
The \code{ompt_callback_buffer_request_t} type is used for callbacks that are
dispatched when a buffer to store event records for a device is requested.

\format
\begin{ccppspecific}
\begin{omptCallback}
typedef void (*ompt_callback_buffer_request_t) (
  uint64_t \plc{device_num},
  ompt_buffer_t **\plc{buffer},
  size_t *\plc{bytes}
);
\end{omptCallback}
\end{ccppspecific}

\descr
A callback with type signature  \code{ompt_callback_buffer_request_t} requests 
a buffer to store trace records for the specified device. A buffer request 
callback may set \plc{*bytes} to 0 if it does not provide a buffer. If a 
callback sets \plc{*bytes} to 0, further recording of events for the device is
disabled until the next invocation of \code{ompt_start_trace}. This action
causes the device to drop future trace records until recording is restarted.

\argdesc
The \plc{device_num} argument specifies the device.

The \plc{*buffer} argument points to a buffer where device events may be 
recorded. The \plc{*bytes} argument indicates the length of that buffer.

\crossreferences
\begin{itemize}
\item \code{ompt_buffer_t} type, see \specref{sec:ompt_buffer_t}.
\end{itemize}



\subsubsubsection{\hcode{ompt_callback_buffer_complete_t}}
\index{ompt_callback_buffer_complete_t@{\code{ompt_callback_buffer_complete_t}}}
\label{sec:ompt_callback_buffer_complete_t}

\summary
The \code{ompt_callback_buffer_complete_t} type is used for callbacks that are
dispatched when devices will not record any more trace records in an event buffer 
and all records written to the buffer are valid.

\format
\begin{ccppspecific}
\begin{omptCallback}
typedef void (*ompt_callback_buffer_complete_t) (
  uint64_t \plc{device_num},
  ompt_buffer_t *\plc{buffer},
  size_t \plc{bytes},
  ompt_buffer_cursor_t \plc{begin},
  int \plc{buffer_owned}
);
\end{omptCallback}
\end{ccppspecific}

\descr
A callback with type signature \code{ompt_callback_buffer_complete_t} provides 
a buffer that contains trace records for the specified device. Typically, a tool 
will iterate through the records in the buffer and process them.

The OpenMP implementation makes these callbacks on a thread that is not an 
OpenMP master or worker thread.

The callee may delete the buffer if the \plc{buffer_owned} argument is 0.

The buffer completion callback is not required to be \emph{async signal safe}.

\argdesc
The \plc{device_num} argument indicates the device which the buffer contains events.

The \plc{buffer} argument is the address of a buffer that was previously
allocated by a \emph{buffer request} callback.

The \plc{bytes} argument indicates the full size of the buffer.

The \plc{begin} argument is an opaque cursor that indicates the position 
of the beginning of the first record in the buffer.

The \plc{buffer_owned} argument is 1 if the data to which the buffer points
can be deleted by the callback and 0 otherwise. If multiple devices accumulate 
trace events into a single buffer, this callback may be invoked with a pointer 
to one or more trace records in a shared buffer with \plc{buffer_owned} = 0. 
In this case, the callback may not delete the buffer.

\crossreferences
\begin{itemize}
\item \code{ompt_buffer_t} type, see \specref{sec:ompt_buffer_t}.

\item \code{ompt_buffer_cursor_t} type, see \specref{sec:ompt_buffer_cursor_t}.
\end{itemize}



\subsubsubsection{\hcode{ompt_callback_control_tool_t}}
\index{ompt_callback_control_tool_t@{\code{ompt_callback_control_tool_t}}}
\label{sec:ompt_callback_control_tool_t}

\summary
The \code{ompt_callback_control_tool_t} type is used for callbacks that 
dispatch \plc{tool-control} events.

\format

\begin{ccppspecific}
\begin{omptCallback}
typedef int (*ompt_callback_control_tool_t) (
  uint64_t \plc{command},
  uint64_t \plc{modifier},
  void *\plc{arg},
  const void *\plc{codeptr_ra}
);
\end{omptCallback}
\end{ccppspecific}

\descr
Callbacks with type signature \code{ompt_callback_control_tool_t} may return 
any non-negative value, which will be returned to the application as the return 
value of the \code{omp_control_tool} call that triggered the callback.

\argdesc
The \plc{command} argument passes a command from an application to a tool. 
Standard values for \plc{command} are defined by \code{omp_control_tool_t} 
in \specref{sec:control_tool}.

The \plc{modifier} argument passes a command modifier from an application to a tool.

The \plc{command} and \plc{modifier} arguments may have tool-specific values.
Tools must ignore \plc{command} values that they are not designed to handle.

The \plc{arg} argument is a void pointer that enables a tool and an application 
to exchange arbitrary state. The \plc{arg} argument may be \code{NULL}.

The \plc{codeptr_ra} argument relates the implementation of an OpenMP region
to its source code. If a runtime routine implements the region associated with
a callback that has type signature \code{ompt_callback_control_tool_t} then
\plc{codeptr_ra} contains the return address of the call to that runtime routine.
If the implementation of the region is inlined then \plc{codeptr_ra} contains the
return address of the invocation of the callback. If attribution to source code
is impossible or inappropriate,\plc{codeptr_ra} may be \code{NULL}.

\constraints
Tool-specific values for \plc{command} must be $\geq$ 64.

\crossreferences
\begin{itemize}
\item \code{omp_control_tool_t} enumeration type, see \specref{sec:control_tool}.
\end{itemize}



\subsubsubsection{\hcode{ompt_callback_cancel_t}}
\index{ompt_callback_cancel_t@{\code{ompt_callback_cancel_t}}}
\label{sec:ompt_callback_cancel_t}

\summary
The \code{ompt_callback_buffer_cancel_t} type is used for callbacks that are 
dispatched for \plc{cancellation}, \plc{cancel} and \plc{discarded-task} events.

\format
\begin{ccppspecific}
\begin{omptCallback}
typedef void (*ompt_callback_cancel_t) (
  ompt_data_t *\plc{task_data},
  int \plc{flags},
  const void *\plc{codeptr_ra}
);
\end{omptCallback}
\end{ccppspecific}

\argdesc
The binding of the \plc{task_data} argument is the task that encounters a 
\code{cancel} construct, a \code{cancellation point} construct, or a construct 
defined as having an implicit cancellation point.

The \plc{flags} argument, defined by the \code{ompt_cancel_flag_t} enumeration
type, indicates whether cancellation is activated by the current task, or detected 
as being activated by another task. The construct that is being canceled is also 
described in the \plc{flags} argument. When several constructs are detected as being
concurrently canceled, each corresponding bit in the argument will be set.

The \plc{codeptr_ra} argument relates the implementation of an OpenMP region
to its source code. If a runtime routine implements the region associated with
a callback that has type signature \code{ompt_callback_cancel_t} then
\plc{codeptr_ra} contains the return address of the call to that runtime routine.
If the implementation of the region is inlined then \plc{codeptr_ra} contains the
return address of the invocation of the callback. If attribution to source code
is impossible or inappropriate,\plc{codeptr_ra} may be \code{NULL}.

\crossreferences
\begin{itemize}
\item \code{omp_cancel_flag_t} enumeration type, see \specref{sec:ompt_cancel_flag_t}.
\end{itemize}



\subsubsubsection{\hcode{ompt_callback_device_initialize_t}}
\index{ompt_callback_flush_t@{\code{ompt_callback_device_initialize_t}}}
\label{sec:ompt_callback_device_initialize_t}

\summary 
The \code{ompt_callback_device_initialize_t} type is used for callbacks that 
initialize device tracing interfaces.

\format
\begin{ccppspecific}
\begin{omptCallback}
typedef void (*ompt_callback_device_initialize_t) (
  uint64_t \plc{device_num},
  const char *\plc{type},
  ompt_device_t *\plc{device},
  ompt_function_lookup_t \plc{lookup},
  const char *\plc{documentation}
);
\end{omptCallback}
\end{ccppspecific}

\descr
Regisration of a callback with type signature 
\code{ompt_callback_device_initialize_t} for the 
\code{ompt_callback_device_initialize} event enables asynchronous 
collection of a trace for a device. The OpenMP implementation invokes 
this callback after OpenMP is initialized for the device but before 
execution of any OpenMP construct on the device is begun.

\argdesc
The \plc{device_num} argument identifies the logical device that is being initialized.

The \plc{type} argument is a character string that indicates the type of 
the device. A device type string is a semicolon separated character string 
that includes at a minimum the vendor and model name of the device. These
names may be followed by a semicolon-separated sequence of properties that 
describe the hardware or software of the device.

The \plc{device} argument points to an opaque object that represents 
the target device instance. Functions in the device tracing interface 
use this pointer to identify the device that is being addressed.

The \plc{lookup} argument points to a runtime callback that a tool must 
use to obtain pointers to runtime entry points in the device's OMPT tracing 
interface. If a device does not support tracing then \plc{lookup} is \code{NULL}.

The \plc{documentation} argument is a string that describes how to use any 
device-specific runtime entry points that can be obtained through the 
\plc{lookup} rgument. This documentation string may be a pointer to external
documentation, or it may be inline descriptions that include names and type 
signatures for any device-specific interfaces that are available through the
\plc{lookup} argument along with descriptions of how to use these interface 
functions to control monitoring and analysis of device traces.

\constraints
The \plc{type} and \plc{documentation} arguments must be immutable strings 
that are defined for the lifetime of a program execution.

\effect
A device initializer must fulfill several duties. First, the \plc{type} 
argument should be used to determine if any special knowledge about the 
hardware and/or software of a device is employed. Second, the \plc{lookup} 
argument should be used to look up pointers to runtime entry points in the 
OMPT tracing interface for the device. Finally, these runtime entry points
should be used to set up tracing for the device.

Initialization of tracing for a target device is described in 
\specref{sec:tracing-device-activity}.

\crossreferences
\begin{itemize}
\item \code{ompt_function_lookup_t}, see \specref{sec:ompt_function_lookup_t}.
\end{itemize}



\subsubsubsection{\hcode{ompt_callback_device_finalize_t}}
\index{ompt_callback_flush_t@{\code{ompt_callback_device_finalize_t}}}
\label{sec:ompt_callback_device_finalize_t}

\summary 
The \code{ompt_callback_device_initialize_t} type is used for callbacks that 
finalize device tracing interfaces.

\format
\begin{ccppspecific}
\begin{omptCallback}
typedef void (*ompt_callback_device_finalize_t) (
  uint64_t \plc{device_num}
);
\end{omptCallback}
\end{ccppspecific}

\argdesc
The \plc{device_num} argument identifies the logical device that is being finalized.

\descr
An OpenMP implementation dispatches a finalization callback for a
device immediately prior to finalizing the device. Prior to dispatching
a finalization callback for a device on which tracing is active,
the OpenMP implementation will stop tracing on the device and
synchronously flush all trace records for the device 
that have not yet been reported to the tool. 
If any trace records for the device need to be flushed,
the OpenMP implementation will issue one or more
buffer completion callbacks with type signature 
\code{ompt_callback_buffer_complete_t}
as needed.

\crossreferences
\begin{itemize}
\item \code{ompt_callback_buffer_complete_t}, 
see \specref{sec:ompt_callback_buffer_complete_t}.
\end{itemize}
